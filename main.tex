% Options for packages loaded elsewhere
\PassOptionsToPackage{unicode}{hyperref}
\PassOptionsToPackage{hyphens}{url}
%
\documentclass[
  a5paper,
  smalldemyvopaper,10pt,twoside,onecolumn,openright,extrafontsizes,hidelinks]{memoir}

\usepackage{amsmath,amssymb}
\usepackage{iftex}
\ifPDFTeX
  \usepackage[T1]{fontenc}
  \usepackage[utf8]{inputenc}
  \usepackage{textcomp} % provide euro and other symbols
\else % if luatex or xetex
  \usepackage{unicode-math}
  \defaultfontfeatures{Scale=MatchLowercase}
  \defaultfontfeatures[\rmfamily]{Ligatures=TeX,Scale=1}
\fi
\usepackage{lmodern}
\ifPDFTeX\else  
    % xetex/luatex font selection
\fi
% Use upquote if available, for straight quotes in verbatim environments
\IfFileExists{upquote.sty}{\usepackage{upquote}}{}
\IfFileExists{microtype.sty}{% use microtype if available
  \usepackage[]{microtype}
  \UseMicrotypeSet[protrusion]{basicmath} % disable protrusion for tt fonts
}{}
\makeatletter
\@ifundefined{KOMAClassName}{% if non-KOMA class
  \IfFileExists{parskip.sty}{%
    \usepackage{parskip}
  }{% else
    \setlength{\parindent}{0pt}
    \setlength{\parskip}{6pt plus 2pt minus 1pt}}
}{% if KOMA class
  \KOMAoptions{parskip=half}}
\makeatother
\usepackage{xcolor}
\setlength{\emergencystretch}{3em} % prevent overfull lines
\setcounter{secnumdepth}{5}
% Make \paragraph and \subparagraph free-standing
\makeatletter
\ifx\paragraph\undefined\else
  \let\oldparagraph\paragraph
  \renewcommand{\paragraph}{
    \@ifstar
      \xxxParagraphStar
      \xxxParagraphNoStar
  }
  \newcommand{\xxxParagraphStar}[1]{\oldparagraph*{#1}\mbox{}}
  \newcommand{\xxxParagraphNoStar}[1]{\oldparagraph{#1}\mbox{}}
\fi
\ifx\subparagraph\undefined\else
  \let\oldsubparagraph\subparagraph
  \renewcommand{\subparagraph}{
    \@ifstar
      \xxxSubParagraphStar
      \xxxSubParagraphNoStar
  }
  \newcommand{\xxxSubParagraphStar}[1]{\oldsubparagraph*{#1}\mbox{}}
  \newcommand{\xxxSubParagraphNoStar}[1]{\oldsubparagraph{#1}\mbox{}}
\fi
\makeatother


\providecommand{\tightlist}{%
  \setlength{\itemsep}{0pt}\setlength{\parskip}{0pt}}\usepackage{longtable,booktabs,array}
\usepackage{calc} % for calculating minipage widths
% Correct order of tables after \paragraph or \subparagraph
\usepackage{etoolbox}
\makeatletter
\patchcmd\longtable{\par}{\if@noskipsec\mbox{}\fi\par}{}{}
\makeatother
% Allow footnotes in longtable head/foot
\IfFileExists{footnotehyper.sty}{\usepackage{footnotehyper}}{\usepackage{footnote}}
\makesavenoteenv{longtable}
\usepackage{graphicx}
\makeatletter
\def\maxwidth{\ifdim\Gin@nat@width>\linewidth\linewidth\else\Gin@nat@width\fi}
\def\maxheight{\ifdim\Gin@nat@height>\textheight\textheight\else\Gin@nat@height\fi}
\makeatother
% Scale images if necessary, so that they will not overflow the page
% margins by default, and it is still possible to overwrite the defaults
% using explicit options in \includegraphics[width, height, ...]{}
\setkeys{Gin}{width=\maxwidth,height=\maxheight,keepaspectratio}
% Set default figure placement to htbp
\makeatletter
\def\fps@figure{htbp}
\makeatother

% typographical packages
\usepackage{microtype}
\usepackage{setspace}
\tolerance=6000
\hyphenpenalty=1000

% typographical settings for the body text
\setlength{\parskip}{0em}
\setlength{\parindent}{1em}
\linespread{1}

% DEFINITIONS TITLE PAGE / COPYRIGHT
\newcommand{\titleoriginal}{The Sovereign Individual}
\newcommand{\subtitleoriginal}{Mastering the Transition to the Information Age}
\newcommand{\yearoriginal}{1999}
\newcommand{\subtitletranslation}{XXX}
\newcommand{\yeartranslation}{XXX}
\newcommand{\stringtranslation}{XXX}
\newcommand{\stringlicense}{XXX}
\newcommand{\stringpublisher}{XXX}
\newcommand{\ISBNHC}{XXX}
\newcommand{\ISBNSC}{XXX}
\newcommand{\ISBNEBOOK}{XXX}
\newcommand{\ISBNAUDIO}{XXX}
\newcommand{\press}{Konsensus Network}
\newcommand{\translatorone}{XXX}
\newcommand{\translators}{
\large\textit{\stringtranslation:}\\
\translatorone\\
}

% PHYSICAL DOCUMENT SETUP
\setstocksize{210mm}{148mm}
\settrimmedsize{210mm}{148mm}{*}
\setbinding{7mm}
\setlrmarginsandblock{15mm}{16mm}{*}
\setulmarginsandblock{16mm}{16mm}{*}
\setlength{\skip\footins}{18pt} % More space between the text and the footnote line

% FONTS
\usepackage{fontspec}
\setmainfont{stone-serif}[
    Path=./fonts/stone-serif-itc-pro/,
    Scale=0.83,
    Extension=.OTF,
    UprightFont=*-Regular,
    BoldFont=*-SemiBd,
    ItalicFont=*-MediumIt,
    BoldItalicFont=*-SemiBdIt
    ]

\setsansfont{stone-sans}[
    Path=./fonts/stone-sans/,
    Scale=0.85,
    Extension=.otf,
    UprightFont=*-Medium,
    BoldFont=*-Semibold,
    ItalicFont=*-MediumItalic,
    BoldItalicFont=*-SemiBoldItalic
    ]

\usepackage{lettrine}
\setcounter{DefaultLines}{3}
\renewcommand{\DefaultLoversize}{0.1}
\renewcommand{\DefaultLraise}{0}
\renewcommand{\LettrineTextFont}{}
\setlength{\DefaultFindent}{\fontdimen2\font}
\setlength{\DefaultNindent}{0em}

% custom second title page
\makeatletter
\newcommand*\halftitlepage{\begingroup % Misericords, T&H p 153
  \setlength\drop{0.1\textheight}
  %\begin{center}
  \vspace*{\drop}
  \rule{\textwidth}{0in}\par
  {\Large\sffamily\thetitle\par}
  \rule{\textwidth}{0in}\par
  \vfill
  %\end{center}
\endgroup}
\makeatother

% custom title page
\makeatletter
\newlength\drop
\newcommand*\titleM{\begingroup % Misericords, T&H p 153
  \setlength\drop{0.15\textheight}
  %\begin{center}
  \vspace*{\drop}
  {\HUGE\sffamily\thetitle\par}
  \vspace{2em}
  {\Large\sffamily\textit\subtitletranslation\par}
  \vspace{4em}
  \rule{5.5cm}{0.3mm}\par
  \vspace{4em}
  {\Large\sffamily\textit\theauthor\par}
  \vspace{6em}
  % {\footnotesize\sffamily\textit\translators\par}
  \vfill
  \includegraphics[width=3.5cm]{figures/knw.png}\par
  %\end{center}
\endgroup}
\makeatother

% copyright page
\makeatletter
\newcommand*\copyrightpage{\begingroup
  \setlength\drop{0.1\textheight}
  \vphantom{just for the drop}
  \vfill
  \begin{footnotesize}
  \noindent \copyright\space \yearoriginal: \theauthor
  \par\noindent \textit{\titleoriginal: \subtitleoriginal}
  \vspace{0.5\baselineskip}
  \par\noindent \copyright\space \yeartranslation\space \stringtranslation: \translatorone
  \par\noindent \textit{\thetitle: \subtitletranslation}
  \vspace{\baselineskip}
  \par\noindent \textit{\stringlicense}
  \vspace{0.5\baselineskip}
  \par\noindent \stringpublisher: \href{https://konsensus.network}{\textit{konsensus.network}}
  \vspace{0.5\baselineskip}
  \par\noindent v1.0.0
  \vspace{0.5\baselineskip}
  \setlength{\parindent}{2em}% default 20pt
  \par\noindent ISBN \ISBNHC \:Hardcover
  \par\hspace{0.28\parindent}\ISBNSC \:Paperback
  \par\hspace{0.28\parindent}\ISBNEBOOK \:E-book\par
  \setlength{\parindent}{0pt}
  \end{footnotesize}
  \vspace{3em}
  \par\noindent \href{https://konsensus.network}{\includegraphics[width=1cm]{figures/freestarfish.png}}
  \par\noindent \href{https://konsensus.network}{\includegraphics[width=3.5cm]{figures/knw.png}}
  \setcounter{footnote}{0}
  \clearpage
\endgroup}
\makeatother

% HEADER AND FOOTER MANIPULATION
% for normal pages
\nouppercaseheads
\headsep = 4mm
\makepagestyle{mystyle} 
\makeevenhead{mystyle}{\scriptsize\sffamily\mdseries\thepage}{}{}
\makeoddhead{mystyle}{\scriptsize\sffamily\mdseries\leftmark}{}{\scriptsize\sffamily\mdseries\thepage}
\makeevenfoot{mystyle}{}{}{}
\makeoddfoot{mystyle}{}{}{}
\makeatletter

% for pages where chapters begin
\makepagestyle{plain}
\makerunningwidth{plain}{\headwidth}
\makeevenfoot{plain}{}{}{}
\makeoddfoot{plain}{}{}{}
\pagestyle{mystyle}

\newif\ifmainmatter
\appto\mainmatter{\mainmattertrue}
\appto\backmatter{\mainmatterfalse}
\appto\appendix{\mainmatterfalse}

\renewcommand\chaptermark[1]{%
  \markboth{\MakeUppercase{%
    \ifmainmatter~\oldstylenums\thechapter.~\fi#1}}{}}%

% TOC
\usepackage[]{tocloft}
\renewcommand{\cftsectiondotsep}{\cftnodots}
\renewcommand{\cftpartfont}{\Large\sffamily\MakeUppercase}
\renewcommand{\cftchapterfont}{\small\sffamily}
\renewcommand{\cftsectionfont}{\Small\sffamily}
\renewcommand{\cftpartpagefont}{\Large\sffamily}
\renewcommand{\cftchapterpagefont}{\small}
\renewcommand{\cftchapterpresnum}{KAPITEL\space}
\renewcommand{\cftchapternumwidth}{7em}
\setlength{\cftchapterindent}{0em}
\setlength{\cftsectionindent}{5em}
\setlength{\cftbeforechapterskip}{0.8em}
\setsecnumdepth{chapter}
\setcounter{tocdepth}{0}


% Redefine footnote presentation
\makeatletter
\renewcommand\@makefntext[1]{%
  \noindent\hb@xt@2em{% <-- Box of fixed size for footnote number and space
    \@thefnmark\quad}% <-- Footnote number followed by a quad space
  \parbox[t]{\dimexpr\linewidth-2em}{#1}% <-- Parbox to control the width of footnote content
}
\makeatother

% layout check and fix
\checkandfixthelayout

% COUNTERS FOOTNOTES
\usepackage{chngcntr}
\counterwithout*{footnote}{chapter}

% TITLE FORMATTING
\usepackage{titlesec}

% Define chapter format with titlesec
\titleformat
    {\chapter}[display]
    {\huge\sffamily} % Main title font style
    {\Large\sffamily\chaptertitlename~\thechapter} % "Chapter N" format
    {0pt} % Space between the chapter number and title
    {\Huge} % Chapter title formatting
    [\vspace{10pt}\Large\textit{\chaptersubtitle}] % Subtitle formatting

% Command to set the subtitle (empty by default)
\newcommand{\chaptersubtitle}{}

% Automatically render the subtitle (if set) after the chapter title
\titleformat{\chapter}[display]
  {\huge\sffamily}
  {\Large\sffamily\chaptertitlename\ \thechapter}
  {0pt}
  {\Huge}
  [\ifx\chaptersubtitle\empty\else\vspace{10pt}\Large\textit{\chaptersubtitle}\fi]

% Command to set subtitle manually after chapter rendering
\newcommand{\setsubtitle}[1]{%
  \renewcommand{\chaptersubtitle}{#1}%
  \chaptermark{\chaptersubtitle} % Update subtitle for header/footer
}

\titleformat
  {\section}[block]
  {\sffamily\large\bfseries}
  {}
  {0pt}
  {}
  
\titlespacing*{\section}{0pt}{2em}{0.5em}

\titleformat{\subsection}{\sffamily\bfseries}{}{}{}
\titlespacing*{\subsection}{0pt}{2em}{0em}

% QUOTE FORMATTING
\renewenvironment{quote}%
               {\list{}{\rightmargin=.6cm\leftmargin=.6cm}%
                \itshape \item[]}% <- The effect of \samepage is local!!!
               {\endlist}

% LAYOUT CHECK AND FIX
\checkandfixthelayout

% CUSTOM TITLE PAGE
\makeatletter
\def\@maketitle{%
  % the half title page
  \pagestyle{empty}
  \halftitlepage
  \cleardoublepage

  % the title page
  \titleM
  \clearpage

  % the copyright page
  \copyrightpage
  \cleardoublepage
  \pagestyle{mystyle}
}
\makeatother
% END PREAMBLE
\makeatletter
\@ifpackageloaded{bookmark}{}{\usepackage{bookmark}}
\makeatother
\makeatletter
\@ifpackageloaded{caption}{}{\usepackage{caption}}
\AtBeginDocument{%
\ifdefined\contentsname
  \renewcommand*\contentsname{Table of contents}
\else
  \newcommand\contentsname{Table of contents}
\fi
\ifdefined\listfigurename
  \renewcommand*\listfigurename{List of Figures}
\else
  \newcommand\listfigurename{List of Figures}
\fi
\ifdefined\listtablename
  \renewcommand*\listtablename{List of Tables}
\else
  \newcommand\listtablename{List of Tables}
\fi
\ifdefined\figurename
  \renewcommand*\figurename{Figure}
\else
  \newcommand\figurename{Figure}
\fi
\ifdefined\tablename
  \renewcommand*\tablename{Table}
\else
  \newcommand\tablename{Table}
\fi
}
\@ifpackageloaded{float}{}{\usepackage{float}}
\floatstyle{ruled}
\@ifundefined{c@chapter}{\newfloat{codelisting}{h}{lop}}{\newfloat{codelisting}{h}{lop}[chapter]}
\floatname{codelisting}{Listing}
\newcommand*\listoflistings{\listof{codelisting}{List of Listings}}
\makeatother
\makeatletter
\makeatother
\makeatletter
\@ifpackageloaded{caption}{}{\usepackage{caption}}
\@ifpackageloaded{subcaption}{}{\usepackage{subcaption}}
\makeatother

\ifLuaTeX
\usepackage[bidi=basic]{babel}
\else
\usepackage[bidi=default]{babel}
\fi
\babelprovide[main,import]{english}
% get rid of language-specific shorthands (see #6817):
\let\LanguageShortHands\languageshorthands
\def\languageshorthands#1{}
\ifLuaTeX
  \usepackage{selnolig}  % disable illegal ligatures
\fi
\usepackage{bookmark}

\IfFileExists{xurl.sty}{\usepackage{xurl}}{} % add URL line breaks if available
\urlstyle{same} % disable monospaced font for URLs
\hypersetup{
  pdftitle={The Sovereign Individual},
  pdfauthor={James Dale Davidson \& Lord William Rees-Mogg},
  pdflang={en},
  hidelinks,
  pdfcreator={LaTeX via pandoc}}


\title{The Sovereign Individual}
\usepackage{etoolbox}
\makeatletter
\providecommand{\subtitle}[1]{% add subtitle to \maketitle
  \apptocmd{\@title}{\par {\large #1 \par}}{}{}
}
\makeatother
\subtitle{De overgang naar het informatietijdperk}
\author{James Dale Davidson \& Lord William Rees-Mogg}
\date{2024-09-10}

\begin{document}
\frontmatter
\maketitle

\renewcommand*\contentsname{Contents}
{
\setcounter{tocdepth}{0}
\tableofcontents
}

\mainmatter
\bookmarksetup{startatroot}

\chapter*{About this book}\label{about-this-book}

\markboth{About this book}{About this book}

\bookmarksetup{startatroot}

\chapter{De overgang in het jaar
2000}\label{de-overgang-in-het-jaar-2000}

\begin{quote}
`Het lijkt wel alsof er iets monumentaals op til is: grafieken tonen de
jaarlijkse groei van bevolkingsaantallen, de concentratie van
koolstofdioxide in de atmosfeer, het aantal webadressen en de megabytes
per dollar. Alle cijfers stijgen naar een asymptoot net na de
eeuwwisseling: de singulariteit. Het einde van alles wat we kennen. Het
begin van iets wat we wellicht nooit zullen doorgronden.' \footnote{Perry
  Barlow, `Thinking Locally, Acting Globally', \emph{Time}, 15 januari
  1996, p.~57.} - Danny Hillis
\end{quote}

\section{Voorgevoelens}\label{voorgevoelens}

Het jaar 2000 houdt de westerse verbeelding al eeuwenlang in zijn greep.
Nadat men destijds verwachtte dat de wereld bij de millenniumwisseling
van het eerste millennium na Christus zou instorten, keken theologen,
evangelisten, dichters en zieners reikhalzend uit naar een decennium vol
ingrijpende gebeurtenissen. Zelfs de beroemde Isaac Newton speculeerde
ooit dat de wereld in het jaar 2000 ten onder zou gaan. Michel de
Nostradamus, wiens profetieën al sinds 1568 door generaties zijn
bekeken, voorspelde de komst van de derde antichrist in juli
1999.\footnote{John Dos Passos, \emph{The Big Money} (New York:
  \emph{Harcourt, Brace \& Co.}, 1936).} De Zwitserse psycholoog Carl
Jung, expert op het gebied van het `collectieve onbewuste', zag al in
1997 de geboorte van een nieuw tijdperk opdoemen. Men maakt zulke
voorspellingen vaak belachelijk, net als de nuchtere prognoses van
economen zoals Dr.~Edward Yardeni van \emph{Deutsche Bank Securities},
die verwachte dat computerstoringen op middernacht van het millennium de
hele wereldeconomie compleet zouden ontwrichten.\footnote{Clarke, op.
  cit., p.~29.} Of je het Y2K-probleem nu beschouwt als ongegronde
hysterie, aangewakkerd door programmeurs en adviseurs op het gebied van
informatietechnologie, of als een mysterieus fenomeen waarin technologie
zich ontvouwt in samenhang met profetische ideeën, je kunt niet
ontkennen dat de drempel van het millennium voor veel meer opwinding
zorgt dan de gebruikelijke, morbide onzekerheid over de toekomst.

Het optimisme dat westerse samenlevingen de afgelopen 250 jaar
kenmerkte, wordt nu overschaduwd door een gevoel van naderend onheil.
Overal zie je het terug: in de gezichtsuitdrukkingen van mensen, in hun
gesprekken, in peilingen en zelfs op de stembus. Net zoals dat een
subtiele verandering in de ionen in de atmosfeer, nog voordat de wolken
donker worden en de bliksem slaat, al verraadt dat er een onweersbui aan
komt, zo hangt er in de schemering van het millennium een voorgevoel van
verandering in de lucht. Men voelt op ieders eigen manier dat hun
levenswijze ten dode opgeschreven is. Bij het aflopen van het decennium,
komen zowel een moorddadige eeuw als een glorieus millenium vol
menselijke verwezenlijkingen tot zijn eind. Alles vindt zijn einde in
het jaar 2000.

\begin{quote}
Want wat verborgen is zal ontdekt worden, en wat geheim is, zal bekend
worden\\
-- Mattheüs 10:26
\end{quote}

Wij zijn ervan overtuigd dat de moderne fase van de westerse beschaving
ten einde loopt -- en in dit boek leggen we uit waarom. Net als veel
andere werken proberen we via een donkere spiegel te kijken en de vage
contouren en verhoudingen van een naderende toekomst te schetsen. In
zekere zin is het de bedoeling van dit werk om `apocalyptisch' te zijn
in de letterlijke betekenis van het woord. `Apokalypsis' is Grieks voor
`onthulling'. Wij geloven dat een nieuw tijdperk -- het
informatietijdperk -- op het punt staat zich te `onthullen'.

\begin{quote}
We observeren de eerste tekenen van een nieuwe logische ruimte, een
ogenblikkelijke elektronische alomtegenwoordigheid waar iedereen toegang
toe heeft, in kan gaan en kan ervaren. Kortom, we bevinden ons in de
begindagen van een nieuwe gemeenschapsvorm. De virtuele gemeenschap
wordt het model voor een seculier koninkrijk der hemelen; zoals Jezus
zei dat er vele huizen waren in het koninkrijk van zijn Vader, zo zijn
er ook talloze virtuele gemeenschappen, elk afgestemd op hun eigen
wensen en behoeften.\\
-- Michael Grasso\footnote{Geciteerd in Kline en Burstein, op. cit.,
  p.~05.}
\end{quote}

\section{The fourth stage of human
society}\label{the-fourth-stage-of-human-society}

Het thema van dit boek richt zich op de opkomst van een nieuwe
machtsrevolutie die individuen bevrijdt en de traditionele natiestaat
van de twintigste eeuw verdringt. Innovaties die de logica van geweld
radicaal herschrijven, verschuiven de grenzen waarbinnen de toekomst
vorm krijgt. Als onze conclusies kloppen, sta je op de drempel van de
meest ingrijpende revolutie in de geschiedenis. Veel sneller dan de
meesten zich voorstellen, zal de microprocessor de natiestaat
ondermijnen en uiteindelijk vernietigen, waarna nieuwe vormen van
sociale organisatie zullen ontstaan. Deze transformatie wordt
allesbehalve eenvoudig.

De uitdaging die eraan verbonden is, wordt des te groter omdat deze zich
in een ongekend tempo voltrekt, veel sneller dan we in het verleden
hebben gezien. In de gehele menselijke geschiedenis -- van de oertijd
tot nu -- kennen we slechts drie fundamentele ontwikkelingsstadia: (1)
jager-verzamelaarsmaatschappijen, (2) agrarische samenlevingen en (3)
industriële samenlevingen. Aan de horizon doemt nu iets compleet nieuws
op: informatiemaatschappijen, het vierde stadium van sociale
organisatie.

Bij elk van de voorgaande samenlevingsstadia kende men een eigen,
duidelijk afgebakende fase in de evolutie en beheersing van geweld.
Zoals we uitvoerig toelichten, zullen informatiemaatschappijen de
voordelen van geweld drastisch verkleinen, mede omdat ze niet langer
gebonden zijn aan één vaste locatie. De virtuele wereld van cyberspace
-- die romanschrijver William Gibson omschreef als een `consensuele
hallucinatie' -- zal voor pestkoppen zo onbereikbaar blijven als de
verbeelding reikt. In dit nieuwe millennium zal de winst uit de
grootschalige beheersing van gewelds veel lager uitvallen dan op welk
moment sinds vóór de Franse Revolutie dan ook. Dit heeft verstrekkende
gevolgen. Eén daarvan is een toename van criminaliteit. Als de beloning
voor grootschalig geweld instort, zal de prikkel voor kleinschalig
geweld naar alle waarschijnlijkheid juist groeien. Geweld krijgt dan een
meer willekeurige, lokale aard, terwijl georganiseerde criminaliteit in
omvang toeneemt; wij lichten hierover verder toe.

Een andere logische consequentie van de afnemende voordelen van geweld
is het vervagen van de politieke rol. Er is veel bewijs dat het
vertrouwen in de burgerlijke mythen van de twintigste-eeuwse natiestaat
snel afneemt. De ondergang van het communisme is slechts één opvallend
voorbeeld. Zoals we in detail onderzoeken, bewijzen de afbrokkelende
moraliteit en de groeiende corruptie onder leiders van westerse
regeringen dat het potentieel van de natiestaat is uitgeput -- het gaat
hier niet om een willekeurige ontwikkeling. Zelfs vele leiders geloven
de holle frasen die zij verkondigen niet meer, en ook het publiek
schenkt die woorden geen vertrouwen meer.

\subsection{Geschiedenis herhaalt
zich}\label{geschiedenis-herhaalt-zich}

Deze situatie vertoont opvallende overeenkomsten met het verleden.
Telkens wanneer technologische veranderingen de oude structuren losmaken
van de nieuwe economische drijfkrachten, veranderen ook de morele normen
en gaan mensen degenen die de oude instituties leiden met groeiende
minachting behandelen. Deze wijdverspreide afkeer komt vaak al duidelijk
naar voren voordat er zelfs een nieuwe, samenhangende ideologie van
verandering ontstaat. Zo was dat bijvoorbeeld in de late vijftiende
eeuw, toen de middeleeuwse kerk nog het overheersende instituut van het
feodalisme vormde. Ondanks het brede geloof in `de heiligheid van het
priesterlijke ambt', behandelden mensen zowel de hoge als de lage
geestelijkheid met grote minachting. Dit is vergelijkbaar met de
populaire houding ten opzichte van politici en bureaucraten in deze
tijd.

We zijn ervan overtuigd dat we veel kunnen leren door het einde van de
vijftiende eeuw, toen het leven doordrenkt was met georganiseerde
religie, te vergelijken met de hedendaagse wereld, waarin politiek
overal de boventoon voert. Destijds waren de kosten om de
geïnstitutionaliseerde religie in stand te houden historisch extreem
hoog, net zoals tegenwoordig de kosten voor het ondersteunen van de
overheid absurd ver zijn opgelopen.

We hebben gezien hoe de georganiseerde religie veranderde na de
buskruitrevolutie. Technologische innovaties stimuleerden het verkleinen
van religieuze instellingen en dwongen tot drastisch kostenbesparingen.
Een vergelijkbare technologische revolutie dreigt aan het begin van dit
millennium de natiestaat radicaal te verkleinen.

\begin{quote}
Vandaag, na meer dan een eeuw aan elektronische technologie, hebben we
ons centrale zenuwstelsel zelf uitgebreid in een mondiale omarming,
waarbij we zowel ruimte als tijd, voor zover onze planeet betreft,
hebben afgeschaft.\footnote{Friedrich A. von Hayek, de denationalisatie
  van geld (London: \emph{Institute of Economic Affairs}, 1976), p.~47.}
\end{quote}

\subsection{De informatierevolutie}\label{de-informatierevolutie}

Naarmate grote systemen verder instorten, verliest systematische dwang
zijn bepalende rol in het vormgeven van het economisch leven en de
inkomensverdeling. Efficiëntie krijgt al snel voorrang boven de bevelen
van de machthebbers bij het organiseren van sociale instituties. Dit
houdt in dat provincies en zelfs steden die effectief eigendomsrechten
bewaken en de rechtspraak draaiende houden, terwijl ze weinig middelen
verbruiken, in het informatietijdperk als levensvatbare soevereiniteiten
kunnen functioneren -- iets wat de afgelopen vijf eeuwen zelden het
geval was.

Binnen cyberspace ontstaat een geheel nieuw terrein voor economische
activiteiten dat niet vatbaar is voor fysiek geweld. De grootste
voordelen zullen vooral de `cognitieve elite' treffen, die zich steeds
minder laat beperken door politieke grenzen. Zij voelen zich net zo
thuis in Frankfurt, Londen, New York, Buenos Aires, Los Angeles, Tokio
en Hong Kong. Binnen de rechtsgebieden zal de inkomensongelijkheid
toenemen, terwijl tussen deze gebieden juist meer gelijkheid ontstaat.

Het soevereine individu onderzoekt de maatschappelijke en financiële
gevolgen van deze revolutionaire ommekeer. Wij willen je helpen de
kansen van dit nieuwe tijdperk te grijpen en ervoor zorgen dat de impact
ervan je niet verwoest. Zelfs als slechts de helft van onze
verwachtingen uitkomt, ondervind je een verandering van ongekende
proporties, zonder werkelijk historisch precedent.

De transformatie in het jaar 2000 verandert radicaal het karakter van de
wereldeconomie en gebeurt bovendien veel sneller dan elke voorgaande
fasewisseling. In tegenstelling tot de agrarische revolutie kost het de
informatierevolutie geen millennia om haar volledige impact te bereiken,
en in tegenstelling tot de industriële revolutie spreidt haar effect
zich niet uit over een periode van eeuwen, maar ervaar je de volledige
reikwijdte ervan binnen één mensenleven.

Daarbij vindt deze transformatie vrijwel gelijktijdig over de hele
wereld plaats. Technologische en economische innovaties beperken zich
niet langer tot geïsoleerde uithoeken van de wereld. De ommekeer raakt
vrijwel alle hoeken van de aarde en markeert een breuk met het verleden
die zo ingrijpend is dat hij het bijna mythische rijk van de goden tot
leven wekt, zoals de oude Grieken ooit voorstelden.

Veel meer dan men zich nu durft voor te stellen, zal blijken dat het in
het nieuwe millennium buitengewoon lastig -- zo niet onmogelijk -- is om
veel van onze hedendaagse instellingen te behouden. Zodra
informatiesamenlevingen opkomen, onderscheiden ze zich van industriële
samenlevingen op dezelfde manier als het Griekenland van Aeschylus
verschilde van de wereld van de grotbewoners.

\section{\texorpdfstring{\emph{`Prometheus unbound'}: De opkomst van het
soevereine
individu}{`Prometheus unbound': De opkomst van het soevereine individu}}\label{prometheus-unbound-de-opkomst-van-het-soevereine-individu}

\begin{quote}
Ik ken geen bemoedigender feit dan het onmiskenbare vermogen van de mens
om zijn leven te verrijken door doelgerichte inzet -- Henry David
Thoreau
\end{quote}

De aanstaande transformatie brengt zowel goed als slecht nieuws met zich
mee. Het positieve is dat de informatierevolutie mensen als nooit
tevoren bevrijdt. Voor het eerst geniet degene die zichzelf kan
bijscholen de volledige vrijheid om eigen ideeën te ontwikkelen en
maximaal te profiteren van zijn productiviteit. Genialiteit komt tot
bloei wanneer overheidsinmenging stopt en raciale en etnische
vooroordelen hun invloed verliezen. Als je echt bekwaam bent, laat je je
in de informatiemaatschappij niet tegenhouden door bekrompen
opvattingen. Het maakt niet uit wat de meeste mensen van je ras,
uiterlijk, leeftijd, seksuele voorkeur of kapsel vinden. In de
cybereconomie blijft je identiteit anoniem. Al ben je misschien minder
fraai, mollig, ouder of heb je een beperking; dankzij de volledige
kleurenblinde anonimiteit op de nieuwe frontlinies van de cyberspace,
concurreert iedereen op gelijke voet met de jonge en knappe onder ons,.

\subsection{Ideeën worden rijkdom}\label{ideeuxebn-worden-rijkdom}

Zodra competenties zich openbaren, worden ze beloond als nooit tevoren.
In een wereld waarin ideeën de grootste bron van rijkdom vormen in
plaats van louter fysiek kapitaal, heeft iedereen met een heldere geest
de potentie om rijk te worden. Het informatietijdperk belooft een
tijdperk van sociale mobiliteit te worden en opent talloze gelijke
kansen voor de miljarden mensen in regio's die nooit ten volle hebben
geprofiteerd van de welvaart van de industriële samenleving. De
slimsten, succesvolste en meest ambitieuze treden op als ware
zelfstandigen.

In eerste instantie bereikt slechts een enkeling volledige financiële
soevereiniteit, maar dat doet niets af aan de voordelen van financiële
onafhankelijkheid. Dat niet iedereen een gigantisch fortuin vergaart,
betekent immers niet dat het streven naar rijkdom zinloos is. Er zijn
25.000 miljonairs voor elke miljardair. Als je wel miljonair bent maar
geen miljardair, ben je zeker niet arm. In de toekomst zul je je
financiële succes niet alleen koppelen aan het aantal nullen van je
vermogen, maar ook aan de mate waarin je erin slaagt om volledige
persoonlijke autonomie te bereiken. Hoe slimmer je bent, hoe minder
energie je nodig hebt om de overgang naar financiële onafhankelijkheid
te maken. Zelfs mensen met bescheiden middelen zullen vooruitkomen zodra
de politieke druk op de wereldeconomie afneemt. Onovertroffen financiële
onafhankelijkheid wordt dan een haalbaar doel voor jou en je kinderen.

Op het hoogste productiviteitsniveau concurreren en communiceren
onafhankelijke individuen op een wijze die doet denken aan de relaties
tussen de Griekse goden. De ongrijpbare berg Olympus van het komende
millennium bevindt zich in cyberspace -- een wereld zonder fysieke vorm
die desalniettemin tegen het tweede decennium van dit nieuwe millennium
uit zal groeien tot 's werelds grootste economie. Tegen 2025 telt de
cybereconomie miljoenen deelnemers. Sommigen zullen een vermogen
ontwikkelen vergelijkbaar met dat van Bill Gates -- elk met een waarde
van meer dan 10 miljard dollar -- terwijl de cyberarmen bestaan uit
mensen die minder dan 200.000 dollar per jaar verdienen. Er komt geen
cyberwelzijn, geen cybertaksen én geen cyberregering. De cybereconomie
kan, in plaats van China, wel eens het grootste economische fenomeen van
de komende dertig jaar worden.

Het goede nieuws is dat politici in dit nieuwe rijk de handel niet
zullen beheersen, onderdrukken of reguleren -- net zoals de wetgevers
van de oude Griekse stadstaten zeker niet in staat waren om een stukje
van Zeus' baard te knippen. Dat komt de rijken ten goede en is nog beter
nieuws voor de minder vermogenden. De door de politiek opgelegde
obstakels en lasten belemmeren immers meer het rijk worden dan het rijk
blijven. De voordelen van afnemende opbrengsten uit geweld en het
decentraliseren van rechtsgebieden zullen voor ieder energiek en
ambitieus persoon ruimte creëren om te profiteren van de dood van de
politiek. Ook de afnemers van overheidsdiensten zullen hier voordeel uit
halen, omdat ondernemers de vruchten van de concurrentie verder
uitbreiden. Veel van de vindingrijkheid werd vroeger juist gekanaliseerd
naar militaire inspanningen omdat er nog weinig ruimte was voor
economische concurrentie tussen jurisdicties. Maar de opkomst van de
cybereconomie zal zorgen voor nieuwe vormen van concurrentie bij het
leveren van diensten die traditioneel door de staat werden uitgevoerd.
Met meer rechtsgebieden ontstaat er meer ruimte om te experimenteren met
nieuwe methoden voor het afdwingen van contracten en het beschermen van
mensen en hun eigendommen. Het vrijmaken van een groot deel van de
wereldeconomie van politieke controle dwingt de resterende overheden om
marktgerichtere methoden toe te passen. Op den duur hebben ze geen
andere optie meer dan de inwoners van hun jurisdicties als klanten te
behandelen, in plaats van hen te onderwerpen aan afpersing, net zoals de
georganiseerde misdaad.

\subsection{Voorbij de politiek}\label{voorbij-de-politiek}

Wat de mythologie ooit als het domein van de goden afschilderde, wordt
voor het individu een haalbare keuze -- een leven buiten de invloed van
koningen en raden. Eerst zullen tientallen mensen, daarna honderden en
uiteindelijk miljoenen zich bevrijden uit de greep van de politiek.
Terwijl zij zich hiervan ontdoen, hervormen zij de werking van
overheden, waardoor de sfeer van dwang afneemt en particuliere controle
over middelen toeneemt.

De opkomst van het autonome individu zal opnieuw de bijzondere
voorspellende kracht van mythen bevestigen. Vroege agrarische volkeren,
die nauwelijks begrip hadden van de natuurwetten, verbeeldden zich dat
`bovennatuurlijke krachten' overal aanwezig waren. Deze krachten werden
soms aangewend door mensen of door `goden in menselijke gedaante', die
op mensen leken en onder hen leefden, zoals Sir James George Frazer in
\emph{The Golden Bough} omschreef als `een universele democratie'.

Toen de oude Grieken zich voorstelden dat de kinderen van Zeus onder hen
leefden, lieten zij zich meevoeren door een diep geloof in magie. Zoals
andere primitieve agrarische volkeren bewonderden zij de natuur en waren
zij ervan overtuigd dat de kracht van de individuele wil -- oftewel
magie -- de natuurlijke orde in beweging zette. Hun visie op de natuur
en de goden straalde geen zelfbewuste, profetische inslag uit. Op het
gebied van microtechnologie waren zij totaal onvoorbereid. Ze konden
zich onmogelijk voorstellen welke invloed deze technologie duizenden
jaren later zou hebben op de marginale productiviteit van mensen ---
laat staan hoe het het evenwicht tussen macht en efficiëntie zou
veranderen, en een revolutie zou ontketenen in de manier waarop bezit
wordt gecreëerd en beschermd. Toch vertoont wat zij zich voorstelden in
hun mythen een merkwaardige overeenkomst met de wereld die u
waarschijnlijk zult meemaken.

\subsection{Alt.abracadabra}\label{alt.abracadabra}

De `abracadabra' die wordt uitgesproken bij het gebruik van magie
vertoont bijvoorbeeld een merkwaardige gelijkenis met het invoeren van
het wachtwoord dat nodig is om toegang te krijgen tot een computer. In
zekere zin kunnen we met de snelle verwerking van computers de magie van
de lampgeest al nabootsen. De eerste generaties van `digitale dienaren'
gehoorzamen nu al de bevelen van degenen die de computers beheersen
waarin ze zijn opgesloten --- net zoals dat geesten ooit opgesloten
zaten in magische lampen. De virtuele realiteit van
informatietechnologie breidt het rijk van menselijke verlangens uit,
zodat bijna alles wat men zich maar kan voorstellen werkelijkheid lijkt
te worden. `Telepresentie' stelt mensen in staat om afstanden met een
bijna bovennatuurlijke snelheid te overbruggen en op afstand
gebeurtenissen te volgen, precies zoals de Grieken beweerden dat Hermes
en Apollo dat konden. Na verloop van tijd genieten de soevereine
individuen van het informatietijdperk, net als de goden uit oude mythen,
van een soort `diplomatieke immuniteit' tegen de politieke beproevingen
die sterfelijke mensen door de eeuwen heen hebben gekend.

Het nieuwe, soevereine individu opereert in dezelfde fysieke omgeving
als de gewone burger, maar politiek gezien leeft hij in een eigen sfeer.
Met een veel groter arsenaal aan middelen tot zijn beschikking en zonder
de beperkingen van talrijke vormen van dwang, gaat dit individu
overheden hervormen en economieën herinrichten in het nieuwe millennium.
De volledige impact van deze verschuiving is bijna onvoorstelbaar.

\subsection{Genialiteit en nemesis}\label{genialiteit-en-nemesis}

Voor wie houdt van menselijke ambitie en succes, opent het
informatietijdperk een schat aan kansen. Dit is zonder twijfel het beste
nieuws in generaties, maar het gaat ook gepaard met minder positieve
ontwikkelingen. De nieuwe maatschappelijke ordening, die voortvloeit uit
de overwinning van de individuele autonomie en het realiseren van echte
gelijke kansen op basis van verdienste, zal leiden tot ruime individuele
vrijheid en grote beloningen voor wie uitblinkt. Dit betekent echter wel
dat mensen veel meer verantwoordelijkheid voor hun eigen leven moeten
nemen dan ze in het industriële tijdperk gewend waren. Tegelijk zal de
onterecht verhoogde levensstandaard, waar de inwoners van geavanceerde
industriële samenlevingen in de twintigste eeuw van profiteerden, sterk
afnemen. Op dit moment verdient de top 15 procent van de wereldbevolking
gemiddeld \$21.000 per persoon per jaar, terwijl de overige 85 procent
gemiddeld slechts \$1.000 tot hun beschikking heeft. Dit enorme, in de
loop der tijd opgepotte voordeel zal onvermijdelijk verdwijnen in de
nieuwe realiteit van het informatietijdperk.

Naarmate dit gebeurt, implodeert het vermogen van natiestaten om op
grote schaal inkomen te herverdelen. Informatietechnologie zorgt voor
een scherpe toename van de concurrentie tussen rechtsgebieden. Wanneer
technologie mobiel is en transacties plaatsvinden in de cyberspace,
zoals steeds vaker het geval zal zijn, zullen overheden niet langer in
staat zijn om meer te vragen voor hun diensten dan wat die diensten
daadwerkelijk waard zijn voor de mensen die ervoor betalen. Iedereen met
een laptop en een satellietverbinding kan elk informatiebedrijf vrijwel
overal runnen, wat nagenoeg het totaal van de wereldwijde financiële
handel van biljoenen dollars omvat.

Dit betekent dat om een hoog inkomen te realiseren, men zich niet langer
gedwongen hoeft te voelen om in een rechtsgebied met hoge belastingen te
wonen. In de toekomst, wanneer het merendeel van de welvaart wereldwijd
verdiend en uitgegeven kan worden, zullen overheden die te hoge tarieven
rekenen voor domicilie simpelweg hun beste klanten doen vertrekken. Als
onze redenering klopt, en daar zijn wij van overtuigd, zal de natiestaat
in de huidige vorm niet overleven.

\section{Het einde van naties}\label{het-einde-van-naties}

Veranderingen die de macht van gevestigde instellingen ondermijnen, zijn
zowel verontrustend als riskant. Net zoals dat monarchen, edellieden,
pausen en andere machthebbers in de vroege moderne tijd meedogenloos
vochten om hun verworven privileges te behouden, zo zullen ook de
huidige overheden geweld inzetten --- vaak heimelijk en willekeurig ---
om de veranderende tijd tegen te houden. Verzwakt door de technologische
uitdagingen gaat de staat steeds autonomer wordende individuen, voorheen
haar gehoorzame burgers, benaderen met dezelfde meedogenloosheid en
diplomatie als waarmee zij tot nu toe met andere regeringen omging.

De komst van deze nieuwe fase in de geschiedenis ging op 20 augustus
1998 met een knal van start, toen de Verenigde Staten voor ongeveer 200
miljoen dollar aan Tomahawk BGM-109 kruisraketten afvuurden op doelen
die naar verluidt in verband stonden met de verbannen Saoedische
miljonair Osama bin Laden. Bin Laden werd de eerste persoon in de
geschiedenis op wiens satelliettelefoon raketaanvallen werden gericht.
Tegelijkertijd verwoestten de Verenigde Staten een farmaceutische
fabriek in Khartoem, Soedan, die met Bin Laden werd geassocieerd. Bin
Ladens opkomst als de belangrijkste vijand van de Verenigde Staten
illustreert een fundamentele verandering in de aard van oorlogsvoering.
Één enkel individu, al bezit hij honderden miljoenen dollars, kan nu als
een geloofwaardige bedreiging worden gezien voor de machtigste militaire
macht van het industriële tijdperk. In uitspraken die doen denken aan de
Koude Oorlogpropaganda over de Sovjet- Unie portretteerden de president
van de Verenigde Staten en zijn nationale veiligheidsspecialisten Bin
Laden -- een privépersoon -- als een transnationale terrorist en de
voornaamste vijand van de Verenigde Staten.

Dezelfde militaire logica die ervoor zorgde dat Osama bin Laden als
oppervijand van de Verenigde Staten werd gezien, zal zich ook aftekenen
in de verhouding tussen overheden en hun burgers. Steeds zwaardere
repressie zal een logisch gevolg zijn van het ontstaan van een nieuwe
machtsverhouding tussen overheden en individuen. Door technologische
vooruitgang worden individuen meer dan ooit soeverein, waardoor ze ook
zo behandeld zullen worden -- soms met geweld als vijanden, soms als
gelijkwaardige onderhandelingspartners, soms als bondgenoten. Hoe
meedogenloos overheden zich ook zullen opstellen tijdens deze
overgangsperiode, een fusie van de belastingdienst met de
inlichtingendienst zal hen weinig opleveren. Overheden zullen in
toenemende mate moeten onderhandelen met autonome individuen, omdat de
controle over hun middelen steeds verder zal verslappen.

De veranderingen die de Informatie-Revolutie teweegbrengt, veroorzaken
niet alleen fiscale crises bij overheden, maar zorgen er ook voor dat
alle grote structuren uiteenvallen. In de twintigste eeuw zijn al
veertien rijken verdwenen. Het verval van rijken maakt deel uit van een
proces dat uiteindelijk ook de natiestaat zal doen instorten. Overheden
moeten zich aanpassen aan de groeiende zelfstandigheid van het individu.
De capaciteit om belastingen te innen daalt met 50 tot 70 procent, wat
kleinere rechtsgebieden bevoordeelt. Het vaststellen van concurrerende
voorwaarden om bekwame individuen en hun kapitaal aan te trekken, slaagt
in enclaves veel eenvoudiger dan op continentale schaal.

Wij zijn ervan overtuigd dat hedendaagse barbaren steeds vaker op de
achtergrond de macht zullen grijpen, zodra de moderne natiestaat
uiteenvalt. Groepen zoals de Russische \emph{mafiya}, die de
overblijfselen van de voormalige Sovjet-Unie uitbuitte, andere etnische
misdaadbendes, \emph{nomenklaturen}{[}\^{}27{]}, drugsbaronnen en
afvallige geheime diensten zullen hun eigen regels maken, en dat doen ze
al.~Veel meer dan men doorgaans denkt, hebben deze moderne barbaren de
vorm van de natiestaat al binnengedrongen zonder haar uiterlijk
wezenlijk te hebben veranderd. Ze fungeren als microparasieten die zich
tegoed doen aan een systeem in verval. Net zo gewelddadig en
meedogenloos als een staat in oorlog, passen deze groepen de technieken
van de staat op een kleinere schaal toe. Hun toenemende invloed en macht
draagt bij aan het krimpende belang van de politiek. Microprocessoren
verkleinen de benodigde middelen die nodig zijn om effectief geweld in
te zetten en te beheersen. Naarmate de technologische revolutie vordert,
organiseert geweld zich steeds vaker buiten centrale controle en richt
de aanpak om het in te dammen zich meer op efficiëntie dan op de omvang
van macht.

\subsection{Geschiedenis in omgekeerde
richting}\label{geschiedenis-in-omgekeerde-richting}

De nieuwe logica van het informatietijdperk keert het proces waardoor de
natiestaat zich in de afgelopen vijf eeuwen heeft ontwikkeld om. Lokale
machtscentra zullen opnieuw aan invloed winnen, terwijl de staat
uiteenvalt in gefragmenteerde, overlappende soevereine eenheden.6 De
groeiende invloed van de georganiseerde misdaad is slechts één
afspiegeling van deze trend. Multinationale ondernemingen moeten
inmiddels al het niet-essentiële werk uitbesteden, en sommige
conglomeraten, zoals AT\&T, Unisys en ITT, splitsten zich op in meerdere
ondernemingen om winstgevender te kunnen opereren. De natiestaat zal
uiteenvallen zoals een onhandelbaar conglomeraat, maar dat gebeurt
waarschijnlijk pas nadat financiële crises haar hiertoe hebben
gedwongen.

Niet alleen verandert de machtsbalans in de wereld, maar ook de aard van
arbeid verandert radicaal. Dit betekent dat de manier waarop bedrijven
opereren onvermijdelijk zal veranderen. De `virtuele onderneming'
illustreert een fundamentele transformatie in de bedrijfsvoering,
mogelijk gemaakt door de dalende kosten voor informatie en transacties.~
We onderzoeken de gevolgen van de informatierevolutie en haar invloed op
het uiteenvallen van bedrijven en het verdwijnen van de `goede baan'. In
het informatietijdperk betekent een `baan' simpelweg een taak die je
uitvoert, in plaats van een vaste positie. Microprocessors creëren
geheel nieuwe mogelijkheden voor economische activiteiten die alle
territoriale grenzen overschrijden. Deze grensoverschrijdende
ontwikkeling is wellicht de meest revolutionaire sinds Adam en Eva
werden verbannen uit het paradijs onder het vonnis van hun Schepper: `In
het zweet van uw gezicht zult gij uw brood verdienen.'~ Naarmate
technologie onze hulpmiddelen radicaal vernieuwt, veroudert ons
rechtssysteem, verandert onze moraal en verschuift onze perceptie. Dit
boek legt uit hoe.

Dankzij microprocessoren en de razendsnelle vooruitgang in communicatie
kan iedereen nu zelf bepalen waar hij of zij werkt. Transacties via
internet worden steeds beter versleuteld en zullen voor
belastingambtenaren binnenkort vrijwel onmogelijk te onderscheppen zijn.
Belastingvrij kapitaal groeit `offshore' aanzienlijk sneller dan
binnenlands vermogen, dat nog steeds belast wordt volgens de hoge
heffingen van de twintigste-eeuwse natiestaat. Na de millenniumwisseling
zal een groot deel van de wereldhandel verhuizen naar het nieuwe domein
van de cyberspace, een gebied waar regeringen even weinig invloed hebben
als over de zeebodem of de verre planeten. In de cyberspace verdwijnen
de dreigingen van fysiek geweld, de alfa en omega van de oude politiek.
Hier ontmoeten de minderbedeelden en de machthebbers elkaar op gelijke
voet. Cyberspace vormt de ultieme offshore jurisdictie: een economie
zonder belastingen, een soort Bermuda in de lucht, vol diamanten.

Wanneer dit ultieme belastingparadijs volledig toegankelijk is voor het
bedrijfsleven, zal vrijwel al het vermogen als offshorekapitaal
fungeren, volledig onder beheer van hun eigenaar. Dit zal een
kettingreactie in gang zetten. De staat is er inmiddels aan gewend
geraakt om haar belastingbetalers te behandelen zoals een boer omgaat
met zijn koeien, die hij op een weiland plaatst om gemolken te worden.
Voor je het weet, zullen de koeien vleugels krijgen.

\subsection{De wraak van de
natiestaat}\label{de-wraak-van-de-natiestaat}

Zoals een boze boer zal de staat in eerste instantie wanhopige pogingen
ondernemen om haar ontsnapte kudde in toom te houden. Ze zal heimelijke,
en zelfs gewelddadige middelen inzetten om de toegang tot bevrijdende
technologieën te beperken. Dergelijke noodmaatregelen werken slechts
tijdelijk, als ze überhaupt effect hebben. De natiestaat van de
twintigste eeuw, met al haar arrogantie, zal uitgeput raken dankzij haar
dalende belastinginkomsten.

Wanneer de staat haar uitgaven niet langer kan dekken door de
belastingen te verhogen, zal ze nog wanhopigere maatregelen aanwenden.
Daaronder valt ook het drukken van geld. Overheden zijn er inmiddels aan
gewend dat ze een monopolie op de valuta hebben en deze naar eigen
inzicht kunnen devalueren. Deze ogenschijnlijk willekeurige inflatie
kenmerkt het monetaire beleid van vrijwel alle staten in de twintigste
eeuw. Zelfs de sterkste munteenheid van de naoorlogse periode, de Duitse
mark, verloor tussen 1 januari 1949 en het einde van juni 1995 maar
liefst 71 procent van haar koopkracht, terwijl in dezelfde periode de
koopkracht van de Amerikaanse dollar 84 procent daalde.7 Die inflatie
werkt als een belasting op iedereen die de valuta aanhoudt. Zoals we
later zullen zien, zal inflatie als inkomstenbron voor de overheid
grotendeels verdwijnen door de opkomst van cybergeld. Men zal de
nationale monopolies op de uitgifte en regulering van geld, waar de
overheid gedurende de moderne tijd van heeft geprofiteerd, door nieuwe
technologie kunnen omzeilen. De kredietcrisissen die Azië, Rusland en
andere opkomende economieën in 1997 en 1998 teisterden, tonen immers aan
dat nationale valuta en kredietbeoordelingen achterhaald zijn en de
soepele werking van de wereldeconomie verstoren. Juist het feit dat de
overheid eist dat alle transacties binnen haar jurisdictie in de
nationale valuta worden uitgevoerd, maakt de economie kwetsbaar voor
fouten van centrale bankiers en aanvallen van speculanten, die de ene
deflatoire crisis na de andere veroorzaakten. In het informatietijdperk
zullen individuen over cybervaluta kunnen beschikken en daarmee hun
monetaire onafhankelijkheid terugwinnen. Wanneer zij hun eigen monetaire
beleid via het internet kunnen voeren, zal het minder, of helemaal niet
meer relevant zijn dat de staat controle blijft houden op de geldprinter
van het industriële tijdperk. Hun invloed op de wereldwijde welvaart
wordt dan ingehaald door wiskundige algoritmen die geen fysieke vorm
kennen. In het nieuwe millennium zal cybergeld, beheerd door
particuliere markten, het fiatgeld, uitgegeven door overheden,
vervangen. Alleen de armen zullen het slachtoffer worden van inflatie en
de daaropvolgende deflatoire spiralen --- gevolgen van de kunstmatige
hefboomwerking die fiatgeld in de economie brengt.

Zonder de gebruikelijke mogelijkheden om belastingen te heffen en geld
in omloop te brengen, zullen regeringen -- zelfs in doorgaans beschaafde
landen -- zich van hun nare kant laten zien. Naarmate de
inkomstenbelasting steeds minder effectief blijkt te worden, zullen oude
en arbitraire vormen van dwang weer onder het stof vandaan gehaald
worden. De ultieme vorm van bronbelasting -- feitelijke of zelfs
openlijke gijzelneming -- zal ingezet worden door overheden die wanhopig
proberen te voorkomen dat rijkdom aan hun greep ontsnapt. Ongelukkige
burgers zullen dan als gijzelaars voor losgeld worden vastgehouden;
bijna zoals we zagen in de middeleeuwen. Ondernemingen die diensten
verlenen ter bevordering van de individuele vrijheid raken verstrikt in
infiltratie, sabotage en ontwrichting. Willekeurige onteigening van
eigendom -- alledaags in de Verenigde Staten, waar dit vijfduizend keer
per week gebeurt -- zal alleen maar toenemen. Regeringen zullen
mensenrechten schenden, de vrije informatiestroom censureren, nuttige
technologieën saboteren en soms nog veel erger handelen. Net zoals de
inmiddels verdwenen Sovjet-Unie tevergeefs probeerde de toegang tot
personal computers en Xerox-machines te beperken, zullen westerse
regeringen met totalitaire middelen proberen de cybereconomie te
onderdrukken.

\section{Terugkeer van de luddieten}\label{terugkeer-van-de-luddieten}

Dergelijke methoden kunnen echter bij bepaalde bevolkingsgroepen in de
smaak vallen. De positieve ontwikkelingen van de individuele bevrijding
en autonomie zullen voor velen die niet tot de cognitieve elite horen,
juist als slecht nieuws overkomen. Het meeste verzet zal vermoedelijk
komen van mensen met middelmatig talent in de huidige rijke landen. Zij
zullen ervaren dat informatietechnologie een bedreiging vormt voor hun
levensstijl. Degenen die profiteren van de georganiseerde dwang --
waaronder miljoenen die een herverdeeld inkomen ontvangen van de
overheid -- kunnen de nieuwe vrijheid, zoals gerealiseerd door
soevereine individuen, verafschuwen.\footnote{Zie Carleton S. Coon,
  \emph{The Hunting Peoples} (New York: Nick Lyons Books, 1971), p.~275.}
Hun verontwaardiging onderstreept de waarheid dat `waar je staat, wordt
bepaald door waar je zit.'

\begin{quote}
\emph{Soms vroeg ik me af hoe ik zo'n diepe meelevende pijn kon voelen
voor het lot van een paar mannen die ik niet kende, terwijl zij in een
honkbalstadion honderden mijlen verderop een wedstrijd speelden tegen
een groep vreemden. Het antwoord is eenvoudig. Ik hield van mijn teams.
Hoewel er risico aan verbonden was, was betrokkenheid de moeite waard.
Sport deed mijn bloed sneller stromen, maakte me opgewonden en liet mijn
hart bonzen. Ik vond het fijn om iets op het spel te hebben. Het leven
kwam tot leven tijdens een wedstrijd.}

\emph{-- Craig Lambert}
\end{quote}

Het zou echter misleidend zijn om alle negatieve gevoelens die tijdens
de aankomende transitiecrisis opkomen, louter toe te schrijven aan een
kille hunkering om ten koste van anderen te leven. Er spelen echter nog
meer factoren mee. De aard van de menselijke samenleving wijst erop dat
er onvermijdelijk een misplaatste morele dimensie meespeelt in de
naderende luddistische reactie. Die kille hunkering verschuilt zich in
feite achter een morele façade. Wij belichten de morele en moralistische
aspecten van de transitiecrisis. Niets zet mensen zo sterk in beweging
als de overtuiging dat ze moreel in hun recht staan -- meer nog dan puur
eigenbelang. Hoewel de band met de burgerlijke mythes uit de twintigste
eeuw snel verzwakt, bestaan er nog altijd echte gelovigen. Iedereen die
in die eeuw volwassen werd, raakte doordrenkt met de plichten en
verantwoordelijkheden van het burgerschap uit die tijd. De overgebleven
morele principes uit de industriële samenleving zullen ten minste enkele
neo-luddistische aanvallen op informatietechnologieën in gang zetten.

In dit opzicht blijkt het te verwachten geweld deels een uiting te zijn
van wat wij `moreel anachronisme' noemen: het toepassen van morele
richtlijnen uit de ene economische periode op de omstandigheden van een
andere. Iedere fase van de samenleving vraagt om eigen morele normen die
mensen helpen de specifieke valkuilen te overwinnen die bij die manier
van leven horen. Een agrarische samenleving functioneerde niet volgens
de morele regels van een rondtrekkende Eskimostam, en zo kan ook de
informatiesamenleving niet voldoen aan de principes die ooit het succes
van een militante industriële staat in de twintigste eeuw mogelijk
maakten. Dat lichten we graag toe.

De komende jaren zal moreel anachronisme zich ook in de kernlanden van
het Westen duidelijk manifesteren, net zoals het de afgelopen vijf
eeuwen in de landen daar ver vandaan te zien was. Westerse kolonisten en
militaire expedities veroorzaakten dit soort crises zodra zij in
aanraking kwamen met inheemse jagers- en verzamelaarsstammen, en
volkeren wiens samenlevingen nog volgens agrarische modellen
functioneerden. Het invoeren van nieuwe technologieën in
anachronistische omgevingen leidde tot verwarring en morele crises. Het
succes van christelijke missionarissen bij de bekering van miljoenen
inheemse mensen is voor een groot deel te danken aan de lokale crises
die ontstonden door de plotselinge invoering van nieuwe machtsstructuren
van buitenaf. Dergelijke botsingen herhaalden zich keer op keer, van de
zestiende eeuw tot de vroege decennia van de twintigste eeuw. Wij
verwachten vergelijkbare conflicten in het begin van het nieuwe
millennium, naarmate de informatiesamenleving het begint over te nemen
van de gevestigde industriële systemen.

\subsection{De nostalgie naar dwang}\label{de-nostalgie-naar-dwang}

De opkomst van het soevereine individu zal niet door iedereen als een
veelbelovende, nieuwe fase in de geschiedenis worden omarmd, zelfs niet
door degenen die er het meeste voordeel uit halen. Iedereen zal gemengde
gevoelens hebben. Veel mensen zullen innovaties afkeuren die de
territoriale natiestaat ondermijnen. Het is de menselijke natuur om
iedere ingrijpende verandering vrijwel altijd als een dramatische
achteruitgang te beschouwen. Vijfhonderd jaar geleden beweerden de
hoflieden rondom de hertog van Bourgondië dat de opkomende innovaties,
die het feodalisme ondermijnden, slecht waren. Zij meenden dat de wereld
snel in verval zou raken, toevallig precies op het moment tijdens de
Renaissance waarin historici later een explosie van menselijk potentieel
constateerden. Zo zou deze periode in het volgende millennium als een
nieuwe renaissance kunnen worden gezien, terwijl in de vermoeide ogen
van de twintigste eeuw vooral angst is af te lezen.

Er is een grote kans dat degenen die aanstoot nemen aan deze nieuwe
ontwikkelingen, en vele anderen die erdoor benadeeld raken, onaangenaam
zullen reageren. Hun nostalgie naar dwang zal waarschijnlijk in geweld
uitmonden. Ontmoetingen met deze nieuwe `Luddieten' zorgen er in ieder
geval voor dat de overstap naar radicaal nieuwe vormen van sociale
organisatie voor iedereen voor wat ellende zal zorgen. Zet je schrap en
bereid je voor op de duik. Nu de snelheid van verandering het morele en
economische aanpassingsvermogen van velen in onze generatie te boven
gaat, kun je rekenen op een felle, verontwaardigde weerstand tegen de
Informatierevolutie, ondanks de grote belofte die zij brengt om de
toekomst te bevrijden.

Je moet dit soort vervelende ontwikkelingen begrijpen en je er op
voorbereiden. Er wacht een transitiecrisis op ons. Deflatoire crises,
zoals de Aziatische financiële crisis van 1997 en 1998, zullen blijven
opduiken zolang de verouderde instellingen uit het industriële tijdperk
niet opgewassen zijn tegen de uitdagingen van de nieuwe,
grensoverschrijdende economie. De nieuwe informatie- en
communicatietechnologieën ondermijnen de moderne staat sterker dan welke
andere politieke bedreiging dan ook sinds de tijd van Columbus. Dat is
van belang, want machthebbers hebben zelden vreedzaam gereageerd op
ontwikkelingen die hun gezag ondermijnen, en dat zal nu waarschijnlijk
niet anders zijn.

De botsing tussen het nieuwe en het oude zal de eerste jaren van het
nieuwe millennium bepalen. Wij voorzien een periode van groot gevaar en
grote beloningen, waarin op sommige terreinen de beschaving sterk
achteruitgaat en op andere terreinen ongekende kansen ontstaan. Steeds
autonomere individuen en failliete, wanhopige regeringen zullen
tegenover elkaar komen te staan aan beide zijden van een nieuwe kloof.
Voor het einde van deze transitie voorzien we een ingrijpende
verandering in het concept van soevereiniteit en de vrijwel volledige
ondergang van de politiek. Waar de staat nu nog de middelen beheerst,
zullen straks vrijwel alle overheidsdiensten geprivatiseerd zijn. Om
onontkoombare redenen -- die we in dit boek uitvoerig bespreken -- zal
informatietechnologie de capaciteit van de staat zodanig ondermijnen dat
zij niet meer in staat is om voor haar diensten meer te vragen dan deze
daadwerkelijk waard zijn voor de burgers.

\begin{quote}
Regeringen zullen zich moeten bezighouden met wat soevereiniteit
betekent.

-- Robert Martin
\end{quote}

\subsection{Soevereiniteit door
markten}\label{soevereiniteit-door-markten}

In een mate die we tien jaar geleden nauwelijks konden voorstellen,
verkrijgen individuen via marktmechanismen steeds meer vrijheid ten
opzichte van territoriale natiestaten. Deze staten zien hun gezag snel
verdwijnen en lopen het risico om failliet te gaan. Hoe machtig ze ook
lijken, behouden zij enkel de mogelijkheid om te vernietigen, niet om
bevelen uit te delen. Hun intercontinentale raketten en vliegdekschepen
functioneren inmiddels als relikwieën, even imposant als zinloos,
vergelijkbaar met het laatste oorlogspaard uit het feodale tijdperk.

Informatietechnologie breidt de markten drastisch uit doordat het de
wijze verandert waarop activa worden gecreëerd en beschermd. Dit is
werkelijk revolutionair. Het belooft zelfs een nog ingrijpendere impact
te hebben op de industriële samenleving dan de introductie van buskruit
ooit had op het feodale landbouwsysteem. De transformatie rond het jaar
2000 betekent de commercialisering van soevereiniteit en het einde van
de traditionele politiek -- precies zoals vuurwapens ooit de ondergang
van het eedgebonden feodalisme inluidden. Burgerschap zal verdwijnen,
net zoals de ridderlijkheid ooit verdween.

Wij zijn ervan overtuigd dat het tijdperk van de individuele economische
soevereiniteit aanbreekt. Net zoals staalfabrieken,
telefoonmaatschappijen, mijnen en spoorwegen die ooit genationaliseerd
waren, wereldwijd in rap tempo werden geprivatiseerd, zul je spoedig de
ultieme vorm van privatisering zien: de ingrijpende denationalisering
van het individu. Het soevereine individu van het nieuwe millennium
behoort niet langer tot de staatsbezittingen, een impliciet activum op
de balans van de schatkist. Na de transitie van het jaar 2000 worden
voormalige burgers geen burgers meer, maar klanten.

\subsection{Bandbreedte overwint
grenzen}\label{bandbreedte-overwint-grenzen}

Het commercialiseren van soevereiniteit zal het traditionele burgerschap
binnen de natiestaat gedateerd maken, vergelijkbaar met de ridderlijke
eeden na de ineenstorting van het feodale systeem. In tegenstelling tot
de burger, verplicht tot belastingbetalingen en gebonden aan een
machtige staat, worden de soevereine individuen van de eenentwintigste
eeuw klanten van overheden die opereren in een `nieuwe logische ruimte'.
Zij zullen onderhandelen over de~beperkte diensten die zij van het
overheidsapparaat wensen en zullen daarvoor betalen via contractuele
overeenkomsten. De overheden van het informatietijdperk zullen zich
volgens geheel andere principes moeten organiseren dan we de afgelopen
eeuwen gewend waren. Sommige rechtsgebieden en diensten met betrekking
tot soevereiniteit zullen ontstaan via `assortive matching', een systeem
waarbij overeenkomsten in voorkeuren, zoals commerciële belangen, de
basis vormen waarop virtuele jurisdicties een klantenbestand opbouwen.

In zeldzame gevallen kunnen de nieuwe soevereiniteiten restanten zijn
van middeleeuwse organisaties, zoals de 900 jaar oude Soevereine
Militaire Hospitaalorde van St.~Johannes van Jeruzalem, van Rhodos en
van Malta. Deze orde, beter bekend als de Ridders van Malta, vormt een
belangengroep van rijke katholieken met 10.000 leden en een jaarlijkse
inkomstenstroom van enkele miljarden. Zij geven hun eigen paspoorten,
postzegels en geld uit en onderhouden volledige diplomatieke
betrekkingen met zeventig landen. Momenteel onderhandelt de orde met de
Republiek Malta over de terugname van Fort St.~Angelo. Door het fort in
bezit te nemen, verkrijgen de Ridders het ontbrekende element van
territorialiteit, zodat zij als soevereine entiteit erkend kunnen
worden. De Ridders van Malta zouden zo opnieuw een soevereine microstaat
kunnen vormen, gesteund door hun lange geschiedenis. Vanuit Fort
St.~Angelo staken zij in 1565 tijdens de Grote Belegering de Turken de
rug toe en regeerden zij Malta vele jaren, totdat Napoleon hen in 1798
verdreef. Als de Ridders van Malta in de komende jaren blijken terug te
keren, zou dat het duidelijkste bewijs zijn dat het systeem van moderne
natiestaten --- dat na de Franse Revolutie zijn intrede deed --- slechts
een intermezzo was in de langere lijn van de geschiedenis, waarin het
normaal was dat verschillende vormen van soevereiniteit naast elkaar
bestonden.

Een ander, totaal verschillend model voor postmoderne soevereiniteit op
basis van assortive matching zie je terug in het
Iridium-satelliettelefoonnetwerk. In eerste instantie lijkt het vreemd
om een mobiele telefoniedienst als een vorm van soevereiniteit te
beschouwen, maar internationale autoriteiten hebben Iridium al erkend
als een virtueel land. Zoals je wellicht weet, biedt Iridium een
wereldwijde mobiele telefoniedienst waardoor abonnees via één nummer
oproepen kunnen ontvangen, waar ze zich ook bevinden -- of je nu in
Featherston, Nieuw-Zeeland bent of in de Boliviaanse Chaco. Om te
garanderen dat de oproepen overal de juiste abonnees bereiken, stemden
internationale telecomautoriteiten ermee in Iridium als een virtueel
land te erkennen, compleet met een eigen landcode: 8816. Het vereist dan
ook geen ingewikkelde logica om te beargumenteren dat als een virtueel
land van satelliettelefoonabonnees soeverein kan zijn, meer
samenhangende virtuele gemeenschappen op het grensoverschrijdende
internet dit ook zullen kunnen zijn.

Bandbreedte -- oftewel de draagcapaciteit van een communicatiemedium --
is sinds de uitvinding van de transistor sneller gegroeid dan de
rekenkracht. Als deze trend zich voortzet, zijn wij ervan overtuigd dat
de bandbreedte binnen enkele jaren -- kort na de millenniumwisseling --
zodanig toeneemt dat men de `metaverse' kan realiseren, de alternatieve
cyberspacewereld zoals bedacht door sciencefictionauteur Neal
Stephenson. Stephenson's `metaverse' staat voor een hechte virtuele
gemeenschap met haar eigen regels. Wij geloven dat, naarmate de
cybereconomie groeit, de deelnemers uiteindelijk vrijgesteld worden van
de verouderde wetten van traditionele natiestaten. De nieuwe
cybergemeenschappen zullen minstens even welvarend zijn en hun belangen
net zo effectief behartigen als de Soevereine Militaire Hospitaalorde
van St.~Johannes van Jeruzalem, van Rhodos en van Malta. Dankzij
geavanceerde communicatietechnologieën en de mogelijkheden binnen
informatieoorlogvoering zullen zij zich beter weten stand te houden. We
onderzoeken daarnaast ook andere modellen van gefragmenteerde
soevereiniteit, waarin kleine groepen feitelijk de soevereiniteit van
zwakkere natiestaten kunnen leasen en hun eigen economische
toevluchtsoorden kunnen exploiteren -- vergelijkbaar met hoe vrije
havens en vrijhandelszones dat nu doen.

We hebben een nieuwe morele woordenschat nodig om de relaties tussen
soevereine individuen en wat er overblijft van de overheid te
beschrijven. Naarmate de spelregels van deze nieuwe verhoudingen
duidelijker worden, zullen ze waarschijnlijk stuiten op weerstand van
mensen die zijn grootgebracht met het idee ``burger'' te zijn van een
twintigste-eeuwse natiestaat. Het verdwijnen van naties en de
`denationalisering van het individu' doen enkele gekoesterde opvattingen
vervagen, zoals het principe van `gelijke bescherming onder de wet', dat
uitgaat van machtsverhoudingen die spoedig tot het verleden behoren.
Naarmate virtuele gemeenschappen sterkere samenhang ontwikkelen, zullen
zij eisen dat hun leden volgens hun eigen regels ter verantwoording
worden geroepen in plaats van volgens de wetten van de voormalige
natiestaten waarin zij wonen. Binnen hetzelfde geografische gebied
zullen opnieuw meerdere rechtsstelsels naast elkaar functioneren, zoals
we in de oudheid en middeleeuwen zagen.

Net zoals dat de pogingen om de riddermacht te behouden, faalden door de
opkomst van vuurwapens, zijn de moderne noties van nationalisme en
burgerschap gedoemd om tot irrelevantie te vervallen door de
ontwikkeling van microtechnologie. Ze zullen uiteindelijk belachelijk
lijken, net zoals de heilige principes van het vijftiende-eeuwse
feodalisme, waarmee al in de zestiende eeuw de spot werd gedreven. De
gekoesterde burgerlijke waarden van de twintigste eeuw zullen
lachwekkende anachronismen blijken te zijn voor komende generaties na de
transformatie van het jaar 2000. De Don Quichot van de eenentwintigste
eeuw wordt geen dwaalridder die de glorie van het feodalisme doet
herleven, maar zal een bureaucraat in een bruin pak zijn-- een
belastingambtenaar die erop uit is om de burgers te controleren.

\section{Herleving van de wetten van de
mark}\label{herleving-van-de-wetten-van-de-mark}

We denken zelden aan overheden als concurrerende partijen, behalve in
heel algemene zin. Daardoor is ons besef van wat soevereiniteit kan
inhouden, verzwakt. Vroeger, toen de machtsverhoudingen het voor
verschillende groepen bemoeilijkten om een stabiel monopolie op dwang te
vestigen, was de macht vaak gefragmenteerd, overlapten rechtsgebieden
elkaar gedeeltelijk en oefenden diverse entiteiten één of meer kenmerken
van soevereiniteit uit. Vaak bleek de zogenaamde heerser in de praktijk
nauwelijks enige macht te bezitten. Tegenwoordig concurreren overheden
die minder sterk zijn dan natiestaten actief om op lokaal niveau een
monopolie op dwang uit te oefenen. Deze concurrentie heeft geleid tot
veranderingen in de wijze waarop geweld wordt beheerst en loyaliteit
wordt gewonnen -- veranderingen die spoedig weer zichtbaar zullen
worden.

Toen de macht van heren en vorsten nog beperkt was en de aanspraken van
één of meer groepen elkaar overlapten bij grensgebieden, kon geen van
beide partijen de overhand krijgen. In de middeleeuwen kende men
talrijke grensgebieden -- de zogenaamde `marken' -- waar
soevereiniteiten in elkaar overliepen. Deze conflictrijke gebieden
hielden decennialang, zo niet eeuwenlang, stand in de grenzen van
Europa. Men vond marken tussen gebieden onder Keltische en Engelse
invloed in Ierland; tussen Wales en Engeland; tussen Schotland en
Engeland; tussen Italië en Frankrijk; tussen Frankrijk en Spanje; tussen
Duitsland en de Slavische grensstreken van Centraal-Europa; en tussen de
christelijke koninkrijken van Spanje en het islamitische koninkrijk
Granada. Dergelijke grensgebieden ontwikkelden unieke institutionele en
juridische structuren, en we verwachten dat we dit fenomeen in het
komende millennium opnieuw zullen zien terugkeren. Doordat twee
autoriteiten met elkaar concurreerden, betaalden de inwoners van deze
streken zelden belasting. Bovendien konden zij doorgaans zelf bepalen
wiens wetten zij zouden volgen, door middel van juridische instrumenten
als de `erkenning' en de `beslaglegging' -- concepten die tegenwoordig
vrijwel geheel zijn verdwenen. Wij voorzien dat zulke principes een
prominente rol zullen spelen in het recht van informatiesamenlevingen.

\subsection{Het overstijgen van
nationaliteit}\label{het-overstijgen-van-nationaliteit}

oor de opkomst van de natiestaat was het lastig om precies vast te
stellen hoeveel soevereine entiteiten de wereld telde, omdat zij op
ingewikkelde wijze in elkaar overliepen en allerlei vormen van
organisatie hanteerden. Dat zal in de toekomst opnieuw zo zijn. Binnen
de natiestaat werden de grenzen tussen territoria scherp afgebakend,
maar in het informatietijdperk zullen deze lijnen weer vervagen. In het
nieuwe millennium raakt soevereiniteit opnieuw gefragmenteerd en duiken
er nieuwe entiteiten op die slechts enkele, maar niet alle kenmerken
bezitten die we met overheden associëren.

Sommige van deze nieuwe entiteiten -- vergelijkbaar met de Tempeliers en
andere religieuze militaire orden uit de middeleeuwen -- kunnen met
aanzienlijke rijkdom en militaire kracht opereren, ook al beschikken zij
niet over een vast grondgebied. Ze organiseren zich op basis van
principes die niets met nationaliteit te maken hebben. In de
middeleeuwen ontleenden leden en leiders van religieuze organisaties,
die in delen van Europa soevereine macht uitoefenden, hun gezag niet aan
een nationale identiteit. Zij vertegenwoordigden diverse etnische
achtergronden en verklaarden hun trouw aan God, niet aan een bepaalde
band die een nationaliteit zou moeten scheppen.

\subsection{Koopliedenrepublieken van
cyberspace}\label{koopliedenrepublieken-van-cyberspace}

Je zult tevens zien dat samenwerkingsverbanden tussen kooplieden en
vermogende individuen met semisouvereine bevoegdheden weer opbloeien,
zoals de Hanze, een middeleeuwse confederatie van kooplieden. De Hanze,
die actief was op de Franse en Vlaamse markten, groeide uit tot een
organisatie die kooplieden uit zestig steden verenigde.9 De `Hanseatic
League' -- zoals men haar in het Engels op een overbodige wijze aanduidt
(de letterlijke vertaling is immers: `Liga liga') -- vormde een verbond
van Germaanse koopmansgilden dat haar leden bescherming bood en
handelsverdragen onderhandelde. Ze kreeg in diverse steden in
Noord-Europa en de Baltische regio semisouvereine bevoegdheden.
Vergelijkbare entiteiten zullen in het nieuwe millennium opkomen als
vervanging van de stervende natiestaat, en zullen bescherming bieden en
bijdragen aan het handhaven van contracten in een onveilige wereld.

Kortom, de toekomst zal waarschijnlijk niet voldoen aan de verwachtingen
van degenen die de burgerlijke mythes van de industriële samenleving van
de twintigste eeuw hebben geïnternaliseerd, waaronder ook de illusies
van de sociale democratie, dat ooit de meest getalenteerde geesten
opzweepte en motiveerde. Deze mensen gaan ervan uit dat samenlevingen
zich ontwikkelen op de manier die voor de overheid het meest wenselijk
is -- het liefst als reactie op opiniepeilingen en nauwkeurig getelde
stemmen. Dit bleek echter nooit zo evident als vijftig jaar geleden werd
gedacht. Vandaag de dag is het een anachronisme, net zozeer een
overblijfsel van het industrialisme als een roestige schoorsteen. Deze
burgerlijke mythes laten niet alleen een denkwijze zien die
maatschappelijke problemen als oplosbaar via technische ingrepen
beschouwt, maar onthullen ook een vals vertrouwen dat hulpbronnen en
individuen in de toekomst even kwetsbaar blijven voor politieke dwang
als in de twintigste eeuw. Wij betwijfelen dat. Het zijn marktkrachten,
niet politieke meerderheden, die samenlevingen zullen dwingen zich
opnieuw in te richten op manieren die de publieke opinie noch zal
begrijpen, noch zal verwelkomen. Als dat gebeurt, blijkt de naïeve
opvatting dat geschiedenis is wat mensen willen dat zij is, buitengewoon
misleidend.

Het zal daarom essentieel zijn om de wereld vanuit een nieuw perspectief
te benaderen. Dat betekent dat u de zaken van buitenaf zult moeten
observeren en veel vanzelfsprekend geachte aannames in vraag zult moeten
stellen, zodat nieuwe inzichten kunnen ontstaan. Als u er niet in slaagt
het conventionele denken te overstijgen in een tijd waarin dit denken de
aansluiting met de werkelijkheid verliest, loopt u het risico ten prooi
te vallen aan een epidemie van desoriëntatie. Die desoriëntatie leidt
tot fouten die uw bedrijf, uw investeringen en uw levensstijl in gevaar
kunnen brengen.

\begin{quote}
\emph{Het universum beloont ons als we het doorgronden en straft ons als
we dat niet doen. Als wij het universum doorgronden, slagen onze plannen
en voelen wij ons op ons best. Maar als wij proberen te vliegen door van
een klif te springen en simpelweg met onze armen te fladderen, dan zal
het universum ons ten val brengen.}\footnote{Gregg, op. cit., p.~23.}

-- \emph{JACK COHEN EN IAN STEWART}
\end{quote}

\subsection{De wereld vanuit een nieuw
perspectief}\label{de-wereld-vanuit-een-nieuw-perspectief}

Om u voor te bereiden op de toekomst, moet u begrijpen waarom de wereld
die gaat komen anders zal zijn dan de voorspellingen van de meeste
experts. Dit betekent dat u grondig moet kijken naar de verborgen
oorzaken van verandering. Wij hebben geprobeerd dit inzichtelijk te
maken met een onconventionele analyse, die wij de studie van
`megapolitiek' noemen. In de twee eerdere delen, \emph{Blood in the
Streets} en \emph{The Great Reckoning}, betoogden we dat de voornaamste
oorzaken van verandering niet te vinden zijn in politieke manifesten of
in de uitspraken van overleden economen, maar in de verborgen factoren
die de grenzen waarover macht wordt uitgeoefend, doen veranderen. Vaak
zorgen subtiele wijzigingen in klimaat, topografie, micro-organismen en
technologie ervoor dat de logica van geweld verschuift. Deze
veranderingen transformeren ook de manier waarop mensen hun
levensonderhoud organiseren en zich verdedigen.

We zijn ons ervan bewust dat onze benadering voor het doorgronden van
wereldveranderingen sterk afwijkt van die van de meeste voorspellers.
Wij beweren niet dat wij meer weten over bepaalde `onderwerpen' dan
degenen die hun hele carrière hebben gewijd aan diepgaande
specialisatie. Integendeel, wij bekijken de zaken juist van buitenaf en
beschikken over kennis van de onderwerpen waarop wij onze voorspellingen
baseren. Voor ons draait het er vooral om te herkennen waar de grenzen
van de noodzaak liggen. Als die grenzen veranderen, verandert de
samenleving onvermijdelijk, ongeacht wat mensen erover wensen.

Vanuit ons perspectief schuilt de sleutel tot het begrijpen van de
ontwikkeling van samenlevingen in het doorgronden van de factoren die de
kosten en baten van geweld bepalen. Elke menselijke samenleving, van
jagerstammen tot grote rijken, wordt gevormd door de wisselwerking van
megapolitieke krachten die de dominante versie van de `wetten van de
natuur' vastleggen. Het leven is overal complex. Lammeren en leeuwen
bewaken een fragiele balans, waarbij zij subtiel op elkaar inwerken. Als
leeuwen ineens sneller zouden worden, zouden zij prooien kunnen vangen
die voorheen ontsnapten. En stel dat lammeren plotseling vleugels zouden
krijgen, dan zouden de leeuwen verhongeren. Het vermogen om geweld in te
zetten en zich ertegen te verdedigen is de cruciale factor die het leven
aan de marge beïnvloedt.

Wij hebben een goede reden om geweld centraal in onze theorie over
megapolitiek te plaatsen. Het beheersen van geweld vormt het grootste
dilemma voor iedere samenleving. Zoals we schreven in \emph{The Great
Reckoning}:

\begin{quote}
\emph{De reden dat mensen tot geweld overgaan, is simpelweg dat het vaak
loont. In zekere zin is het meest voor de hand liggende wat een mens kan
doen als hij geld wil, het gewoon afpakken ervan. Dit geldt evenzeer
voor een leger dat een olieveld verovert als voor een enkele crimineel
die zomaar een portemonnee pakt. Macht zoekt, zoals William Playfair
schreef, `altijd de gemakkelijkste weg naar rijkdom door degenen aan te
vallen die erover beschikken.'~}

\emph{De uitdaging voor voorspoed ligt juist in het feit dat roofzuchtig
geweld onder bepaalde omstandigheden zeer winstgevend is. Oorlog
herschrijft de spelregels, verandert de verdeling van eigendommen en
inkomen, en bepaalt zelfs wie er leeft en wie sterft. Juist het feit dat
geweld loont, maakt het zo moeilijk te beheersen.}\footnote{Boyden, op.
  cit., p.~69.}
\end{quote}

Het denken in deze termen hielp ons ontwikkelingen te voorspellen
waarover zelfs de meest doorgewinterde experts ervan overtuigd waren dat
ze nooit zouden plaatsvinden. Zo was bijvoorbeeld \emph{Blood in the
Streets}, dat begin 1987 verscheen, onze poging om de eerste signalen in
kaart te brengen van de grootschalige megapolitieke revolutie die zich
nu ontvouwt. Destijds stelden wij dat technologische vernieuwing de
mondiale machtsverhoudingen radicaal zou verstoren. Onze belangrijkste
stellingen waren:

\begin{itemize}
\item
  Wij voorspelden dat het Amerikaanse overwicht zou afnemen, wat zou
  leiden tot economische onevenwichtigheden en tegenspoed, waaronder een
  beurscrash in de stijl van 1929. Hoewel de experts vrijwel unaniem
  weigerden te geloven dat zoiets kon gebeuren, werden de markten in
  oktober 1987, amper zes maanden na onze voorspelling, geschokt door de
  heftigste verkoopgolf van de eeuw.
\item
  Wij waarschuwden de lezers voor de ineenstorting van het communisme.
  Wederom lachten de experts ons uit, maar in 1989 deden zich
  gebeurtenissen voor die ``niemand had kunnen voorzien.'' De Berlijnse
  Muur viel en revoluties deden de communistische regimes, van de
  Baltische staten tot zelfs Boekarest, verdwijnen.
\item
  Wij legden uit waarom het multi-etnische rijk, dat door de
  bolsjewistische elite werd overgenomen van de tsaren, onvermijdelijk
  uiteen zou vallen. Eind december 1991 hing de hamer- en sikkelvlag
  voor de laatste keer boven het Kremlin, waarna de Sovjetunie voorgoed
  ophield te bestaan.
\item
  Midden in de intensieve wapenwedloop onder Reagan voorspelden wij dat
  de wereld op de drempel stond van een ingrijpende ontwapening. Ook dit
  werd als onwaarschijnlijk -- zo niet belachelijk -- afgedaan, maar in
  de daaropvolgende zeven jaar vond wel de meest ingrijpende ontwapening
  plaats sinds het einde van de Eerste Wereldoorlog.
\item
  Terwijl experts in Noord-Amerika en Europa naar Japan wezen als bewijs
  dat overheden markten succesvol konden manipuleren, stelden wij het
  tegendeel vast. Wij voorspelden dat de bloeiperiode van de Japanse
  financiële bubbel zou eindigen in een ineenstorting. Niet lang na de
  val van de Berlijnse Muur stortte de Japanse aandelenmarkt in en
  verloor bijna de helft van haar waarde. Wij blijven ervan overtuigd
  dat het dieptepunt van deze neerwaartse beweging het verlies van 89
  procent, zoals Wall Street na 1929 heeft geleden, kan evenaren of
  zelfs overtreffen.
\item
  Terwijl bijna iedereen, van het middenklassegezin tot de grootste
  vastgoedinvesteerders ter wereld, ervan overtuigd leek dat
  vastgoedmarkten enkel zouden stijgen, waarschuwden wij voor een
  naderende vastgoedcrisis. Binnen vier jaar verloren
  vastgoedinvesteerders wereldwijd meer dan \$1 biljoen (\$1.000
  miljard)~ toen de vastgoedmarkt kelderde.
\item
  Al ruim voordat experts de achteruitgang in de inkomens van arbeiders
  onderkenden, voorspelden wij in \emph{Blood in the Streets} dat deze
  daling structureel zou blijven bestaan. Nu, bijna een decennium later,
  begint een slapende wereld eindelijk in te zien dat dit klopt. Het
  gemiddelde uurloon in de Verenigde Staten is gedaald tot een niveau
  lager dan tijdens de tweede Eisenhower-regering. In 1993 bedroeg het
  gemiddelde jaarlijkse uurloon, gecorrigeerd voor inflatie, \$18.808.
  In 1957, toen Eisenhower werd ingezworen voor zijn tweede
  ambtstermijn, was het gemiddelde jaarlijkse uurloon in de VS \$18.903.
\end{itemize}

Hoewel de centrale thema's uit \emph{Blood in the Streets} achteraf
gezien verrassend nauwkeurig bleken, noemden de hoeders van het
conventionele denken deze inzichten nog enkele jaren geleden pure onzin.
In 1987 bestempelde een recensent in \emph{Newsweek} onze analyse als
`een ondoordachte aanval op de rede', waarmee hij de bekrompenheid van
de laat-industriële mentaliteit treffend illustreerde.

Je zou denken dat \emph{Newsweek} en vergelijkbare publicaties inmiddels
inzagen dat onze analysemethode waardevolle inzichten bood in hoe de
wereld veranderde. Maar niets is minder waar. De eerste uitgave van
\emph{The Great Reckoning} werd met dezelfde gniffelende vijandigheid
ontvangen als \emph{Blood in the Streets}. Zelfs het \emph{Wall Street
Journal} wees onze analyse resoluut af als het geklets van `je suffe
tante'.~

Ondanks al het gegiechel, bleken de thema's van \emph{The Great
Reckoning} minder belachelijk dan de hoeders van de orthodoxie deden
vermoeden.~

Ook hebben we onze voorspelling over de ondergang van de Sovjet-Unie
verder uitgewerkt door te onderzoeken waarom Rusland en de andere
voormalige Sovjetrepublieken een toekomst tegemoet gingen vol toenemende
burgerlijke onrust, hyperinflatie en dalende levensstandaarden.

\begin{itemize}
\item
  We legden uit waarom de jaren 1990 een decennium van krimp zouden
  worden, met voor het eerst een wereldwijde inkrimping van zowel
  overheden als bedrijven.
\item
  We voorspelden tevens dat een ingrijpende herdefiniëring van de
  voorwaarden voor inkomensherverdeling stond te gebeuren, wat zou
  leiden tot flinke bezuinigingen op uitkeringen. Overal,van Canada tot
  Zweden, verschenen de eerste signalen van een fiscale crisis, en
  Amerikaanse politici begonnen te spreken over `het beëindigen van de
  welvaartsstaat zoals wij die kennen.'
\item
  Wij voorzagen en legden uit dat de `nieuwe wereldorde' uiteindelijk
  als een `nieuwe wereldwanorde' zou uitpakken. Lange tijd vóórdat de
  gruweldaden in Bosnië de krantenkoppen overspoelden, waarschuwden wij
  dat Joegoslavië in een burgeroorlog zou instorten.
\item
  Nog vóórdat Somalië in volledige anarchie verviel, legden wij uit hoe
  de dreigende ineenstorting van regeringen in Afrika ervoor zou zorgen
  dat sommige landen daar feitelijk onder curatele zouden worden
  gesteld.
\item
  Wij voorspelden en verduidelijkten waarom de militante islam het
  marxisme als leidende ideologie in de confrontatie met het Westen zou
  verdringen. Jaren voordat de bomaanslag in Oklahoma plaatsvond en men
  probeerde het World Trade Center op te blazen, lieten wij zien waarom
  de Verenigde Staten te maken zouden krijgen met een toename aan
  terrorisme.
\item
  Nog vóórdat de krantenkoppen over rellen in Los Angeles, Toronto en
  andere steden verschenen, legden wij uit hoe de opkomst van criminele
  subculturen onder stedelijke minderheden de weg vrijmaakte voor
  wijdverspreid crimineel geweld.
\item
  We voorspelden ook `de laatste depressie van de twintigste eeuw', die
  in 1989 in Azië begon en zich vanaf de periferie naar het centrum van
  het mondiale systeem verspreidde. Wij stelden dat de Japanse
  aandelenmarkt het voorbeeld van Wall Street na 1929 zou volgen, wat
  uiteindelijk zou uitmonden in een kredietcrisis en depressie.
  Overheden in Japan en elders grepen fors in, waardoor de financiële
  markten tijdelijk niet volledig weerspiegelden hoe slecht de
  kredietsituatie daadwerkelijk was. Het verplaatste slechts de
  economische problemen en verergerde ze. Hierdoor kwamen overheden
  wereldwijd onder druk te staan om competitieve monetaire devaluaties
  door te voeren, en dreigde zelfs een systeemwijde kredietcrisis, zoals
  de crisis in de jaren 1930, die wereldwijd economieën deed instorten.
\end{itemize}

\begin{quote}
\emph{The Great Reckoning} belichtte ook een reeks controversiële theses
die nog niet bevestigd zijn, of het door ons voorspelde
ontwikkelingsniveau nog niet hebben bereikt:
\end{quote}

\begin{itemize}
\item
  We voorspelden dat de Japanse beurs dezelfde weg zou inslaan als Wall
  Street na 1929, wat zou uitmonden in een kredietcrisis en economische
  depressie. Ondanks dat de werkloosheid in landen als Spanje en Finland
  zelfs hoger uitviel dan in de jaren '30, en sommige landen -- zoals
  Japan -- met lokale depressies kampten, is er nog geen wereldwijde
  kredietcrash geweest zoals die destijds hele economieën onderuit
  haalde.
\item
  We stelden dat het uiteenvallen van het centraal geleide systeem in de
  voormalige Sovjet-Unie zou leiden tot de verspreiding van kernwapens
  naar ministaten, terroristische groeperingen en criminele bendes. Tot
  grote opluchting van de wereld is dat niet gebeurd, althans niet in de
  mate die wij vreesden. Persberichten melden dat Iran diverse tactische
  kernwapens op de zwarte markt heeft aangeschaft en dat de Duitse
  autoriteiten meerdere pogingen tot de verkoop van nucleaire materialen
  hebben verijdeld. Er is overigens geen melding gemaakt van
  daadwerkelijke inzet of gebruik van kernwapens uit het arsenaal van de
  voormalige Sovjet-Unie.
\item
  Ten slotte legden we uit waarom de zogenaamde `War on Drugs' een
  recept was voor het ondermijnen van de politie- en rechtssystemen in
  landen met wijdverspreid drugsgebruik, vooral in de Verenigde Staten.
  Dankzij de tientallen miljarden dollars aan verborgen monopoliewinsten
  die drugshandelaren jaarlijks binnen harken, beschikken ze over zowel
  de middelen als de prikkel om zelfs schijnbaar stabiele landen te
  corrumperen. De internationale media hebben zo nu en dan gesuggereerd
  dat drugsgeld tot in de hoogste regionen van de Amerikaanse politiek
  is doorgedrongen, maar het volledige verhaal is nog niet verteld.
\end{itemize}

\subsection{Kijken waar anderen niet
kijken}\label{kijken-waar-anderen-niet-kijken}

Hoewel sommige van onze voorspellingen achteraf fout bleken of met de
huidige kennis als onjuist worden bestempeld, blijft het totaalplaatje
standhouden tegen de kritiek. Veel van de ontwikkelingen die
waarschijnlijk een rol zouden gaan spelen in de economische~
geschiedenis van de jaren 1990 werden al eerder voorspeld en toegelicht
in \emph{The Great Reckoning}. Wij voorspelden niet simpelweg een
voortzetting van de bestaande trends, maar wezen op ingrijpende breuken
met wat sinds de Tweede Wereldoorlog als normaal werd beschouwd. We
waarschuwden dat de jaren negentig drastisch anders zouden zijn dan de
voorgaande vijf decennia. Als je het nieuws van 1991 tot en met 1995
volgt, zie je dat de thema's uit \emph{The Great Reckoning} bijna
dagelijks werkelijkheid werden.

Wij interpreteren deze ontwikkelingen niet als losse tegenslagen, maar
als schokgolven langs één doorlopende breuklijn. De oude orde wordt op
zijn kop gezet door een~ megapolitieke aardbeving die een revolutie in
de politieke instellingen zal veroorzaken, en de manier waarop kritische
geesten de wereld beschouwen radicaal zal veranderen.

Ondanks de cruciale rol die geweld speelt in het functioneren van onze
wereld, krijgt het opvallend weinig serieuze aandacht. De meeste
politieke analisten en economen doen alsof geweld slechts een kleine
ergernis is, vergelijkbaar met een vlieg die rond een taart zoemt, en
niet als de bakker die de taart heeft gemaakt.

\subsection{Een andere grondlegger van de
megapolitiek}\label{een-andere-grondlegger-van-de-megapolitiek}

Sterker nog, er is zo weinig helder nagedacht over de rol van geweld in
de geschiedenis dat een bibliografie met alle megapolitieke analyses op
één vel papier zou passen. In \emph{The Great Reckoning} baseerden we
onze redenering onder meer op een bijna volledig vergeten klassieker
over megapolitieke analyse, namelijk \emph{An Enquiry into the Permanent
Causes of the Decline and Fall of Powerful and Wealthy Nations} van
William Playfair, gepubliceerd in 1805. Eén van onze uitgangspunten is
tevens het werk van Frederic C. Lane. Lane, een historicus die zich
onder meer richtte op de middeleeuwen, publiceerde in de jaren '40 en
'50 verschillende scherpe essays over de rol van geweld in de
geschiedenis. Wellicht was zijn essay \emph{Economic Consequences of
Organized Violence}, dat in 1958 in het \emph{Journal of Economic
History} verscheen, het meest omvattende van deze werken. Buiten
professionele economische en geschiedkundige kringen heeft slechts een
enkeling het gelezen en lijken de meesten de ware betekenis ervan niet
te hebben doorgrond. Net als Playfair richtte Lane zich tot een publiek
dat op dat moment nog niet bestond.

\subsection{Inzichten voor het
informatietijdperk}\label{inzichten-voor-het-informatietijdperk}

Lane publiceerde zijn werk over geweld en de economische betekenis van
oorlog ruim voor de intrede van het informatietijdperk. Hij schreef
zeker niet met de introductie van~ microprocessing of andere
technologische revoluties die zich tegenwoordig ontvouwen in het
vooruitzicht. Toch bieden zijn inzichten omtrent geweld een kader om te
begrijpen hoe de informatierevolutie de samenleving zal
herstructureren.~Het toekomstbeeld dat Lane schetste, bleek in
werkelijkheid een blik terug in de tijd.

Hij was een middeleeuwse historicus, met name gespecialiseerd in de
handelsstad Venetië, waar het fortuin op en neer schommelde in een
wereld vol geweld. Toen hij nadacht over de opkomst en ondergang van
Venetië, viel hem op dat de manier waarop geweld wordt georganiseerd en
gecontroleerd een cruciale rol speelt in hoe schaarse middelen worden
ingezet.\footnote{Shaw and Wong, op. cit., p.~69}

Wij zijn ervan overtuigd dat Lanes analyses over de concurrerende
toepassingen van geweld ons veel inzicht kunnen geven in hoe het leven
in het informatietijdperk waarschijnlijk zal verlopen.~Maar verwacht
niet dat de meeste mensen zo'n extreem abstract betoog zullen opmerken,
laat staan het daadwerkelijk volgen.~Terwijl de wereld haar aandacht
richt op partijdige debatten en excentrieke persoonlijkheden, glipt de
dynamiek van de megapolitiek bijna onopgemerkt voorbij.~De gemiddelde
Noord-Amerikaan besteedt waarschijnlijk honderd keer meer aandacht aan
O.J. Simpson dan aan de nieuwste microtechnologieën, die zijn baan
overbodig lijken te maken en het politieke systeem, waar hij op
vertrouwt voor zijn werkloosheidsuitkeringen, ondermijnen.

\section{De ijdelheid van wensen}\label{de-ijdelheid-van-wensen}

De neiging om over het hoofd te zien wat van fundamenteel belang is, is
niet alleen voorbehouden voor de bankzitter die tv kijkt. Volgens
klassieke denkers verandert de wereld omdat de overtuigingen van mensen
veranderen -- een van de hardnekkige illusies die de natiestaat in stand
houdt. Zelfs ogenschijnlijk scherpe analisten komen vaak met
verklaringen en voorspellingen waarbij grote historische gebeurtenissen
worden voorgesteld alsof ze voort zijn gekomen uit wensdenken. Een
opvallend voorbeeld van dit type redenering verscheen op de opiniepagina
van de \emph{New York Times} in een artikel van Nicholas Colchester,
precies op het moment dat wij \emph{Goodbye, Nation State, Hello\ldots{}
What?} schreven. Niet alleen behandelde hij het onderwerp, de ondergang
van de natiestaat, precies het thema dat wij bespreken, maar plaatste
hij zichzelf ook als een treffend voorbeeld van hoe ver ons denken van
de norm afwijkt. Colchester is geen eenvoudige denker. Hij was
redactioneel directeur van de \emph{Economist Intelligence Unit}. Als
iemand een realistisch wereldbeeld belichaamt, dan is dat ongetwijfeld
hij. Toch betoogt hij in zijn artikel op meerdere punten dat `de komst
van een internationale overheid' nu onvermijdelijk is.

Waarom? Omdat de natiestaat wankelt en niet langer in staat is de
economische krachten in toom te houden.

Wat ons betreft is deze veronderstelling zo goed als absurd. Het is een
misvatting om te denken dat een nieuwe bestuursvorm automatisch opkomt
zodra een bestaande faalt. Volgens die logica hadden Haïti en Zaïre al
lang beter bestuur moeten kennen, simpelweg omdat wat zij hadden zo
overduidelijk ontoereikend was.

Het standpunt van Colchester, breed gedeeld door de weinigen in
Noord-Amerika en Europa die over zulke zaken nadenken, houdt totaal geen
rekening met de grotere megapolitieke krachten die bepalen welke
politieke systemen werkelijk levensvatbaar zijn. Dat vormt de kern van
dit boek. Wanneer we rekening houden met de technologieën die het nieuwe
millennium vormgeven, zullen we waarschijnlijk niet één wereldregering
krijgen, maar eerder microregeringen of zelfs anarchistische toestanden.

Voor elke serieuze analyse van de rol van geweld in het bepalen van de
regels waar iedereen zich aan moet houden, zijn er tientallen boeken
geschreven over de miniscule details van graansubsidies en honderden
over obscure aspecten van monetair beleid. Het gebrek aan een doordachte
benadering van de cruciale kwesties die echt het verloop van de
geschiedenis bepalen, weerspiegelt waarschijnlijk de relatieve
stabiliteit van de machtsstructuur van de afgelopen eeuwen. De vogel die
op de rug van een nijlpaard in slaap valt, maakt zich geen zorgen over
zijn zitplek totdat het dier in beweging komt. Dromen, mythen en
fantasieën spelen een veel grotere rol in het vormgeven van wat men de
`sociale wetenschappen' noemt dan men doorgaans vermoedt.

Nergens is dit duidelijker dan in de talloze teksten over economische
rechtvaardigheid. Terwijl men eindeloos praat en schrijft over wat
eerlijk is of niet, wordt zelden diepgaand onderzocht hoe geweld de
samenleving vormgeeft, en dus de grenzen bepaalt waarbinnen de economie
opereert. Moderne ideeën over economische rechtvaardigheid
veronderstellen echter vaak dat de samenleving dient aangestuurd te
worden door een uiterst krachtig dwangapparaat --- één dat in staat is
om welvaart af te nemen en te herverdelen. Deze vorm van macht is
slechts een paar generaties geleden ontstaan, en nu is ze aan het
afbrokkelen.

\subsection{Big Brother over de sociale
zekerheid}\label{big-brother-over-de-sociale-zekerheid}

Door industriële technologie beschikten overheden in de twintigste eeuw
over meer controle-instrumenten dan ooit tevoren. Voor een tijd leek het
onvermijdelijk dat overheden het geweld zo effectief zouden
monopoliseerden dat er nauwelijks nog ruimte zou blijven voor
individuele autonomie. Halverwege de eeuw durfde niemand te verlangen
naar een triomf van het soevereine individu.

Enkele van de scherpste waarnemers uit het midden van de twintigste eeuw
waren ervan overtuigd dat de drang van natiestaten om de macht te
centraliseren uiteindelijk zou leiden tot een totalitaire overheersing
van alle levensdomeinen. In George Orwells \emph{1984,} uitgegeven in
1949, hield `Big Brother' elk individu nauwlettend in de gaten, terwijl
hij vergeefs probeerde ook maar een spoor van autonomie en eigenwaarde
te behouden. Het leek een verloren strijd. Friedrich von Hayeks
\emph{The Road to Serfdom,} uit 1944, benaderde het onderwerp op een
wetenschappelijke manier en betoogde dat vrijheid ten onder zou gaan aan
een nieuwe vorm van economische controle, waardoor de staat uiteindelijk
de absolute macht zou krijgen. Deze werken schreef men vóór de opkomst
van microprocessoren, die later een scala aan technologieën zouden
introduceren waarmee zelfs kleine groepen en individuen onafhankelijk
van de centrale autoriteit kunnen opereren.

Hoe scherp Hayek en Orwell ook waren, ze bleken uiteindelijk te
pessimistisch. De geschiedenis heeft enkele verrassingen voortgebracht.
Het totalitaire communisme hield nauwelijks stand tot 1984. Een nieuwe
variant van lijfeigenschap zou in het volgende millennium nog kunnen
ontstaan als overheden erin slagen de bevrijdende kracht van
microtechnologie te onderdrukken. Het is echter veel waarschijnlijker
dat we ongekende kansen en meer individuele autonomie zullen ervaren.
Waar onze ouders zich zorgen over maakten, blijkt mogelijk helemaal geen
probleem te vormen. Wat zij als vaste en blijvende kenmerken van het
sociale leven beschouwden, lijkt nu gedoemd te verdwijnen. Zodra de nood
het eist, passen we ons aan en organiseren we ons leven naar de nieuwe
werkelijkheid.

\subsection{De risico's van
voorspellen}\label{de-risicos-van-voorspellen}

We wagen ongetwijfeld een stukje van onze waardigheid als we proberen om
ingrijpende veranderingen in de organisatie van ons leven en in de
cultuur die ons verbindt te voorspellen en te verklaren. De meeste
voorspellingen blijken uiteindelijk belachelijk, en hoe dramatischer de
voorspelde verandering, hoe beschamender zij doorgaans uitpakken. De
wereld gaat niet ten onder, de ozonlaag verdwijnt niet en de voorspelde
ijstijd maakt plaats voor wereldwijde opwarming. Ondanks alle signalen
die wijzen op het tegenovergestelde, zit er nog steeds olie in de tank.
Meneer Antrobus, de doorsnee-man uit \emph{The Skin of Our Teeth}, weet
te voorkomen dat hij bevriest, overleeft oorlogen en dreigende
economische rampen, en wordt oud, terwijl hij de weldoordachte
waarschuwingen van experts links laat liggen.

De meeste pogingen om de toekomst te `onthullen' komen al snel komisch
over. Zelfs als eigenbelang onze blik op scherp zet, blijft onze visie
op de toekomst vaak kortzichtig. In 1903 stelde Mercedes: `Er zullen
nooit 1 miljoen auto's wereldwijd rondrijden, omdat het onwaarschijnlijk
is dat 1 miljoen ambachtslieden tot chauffeurs zullen kunnen worden
opgeleid.'\footnote{Voor meer details over de Kafirs, zie Schuyler
  Jones, \emph{Men of Influence in Nuristan} (Londen: Seminar Press,
  1974).}

Dit besef zou ons tot zwijgen moeten brengen, maar dat gebeurt niet. We
schuwen het niet om onze welverdiende portie spot te incasseren. Als we
er ver naast zitten, mogen toekomstige generaties naar hartenlust
lachen, als ze zich überhaupt herinneren wat we hebben gezegd. Het
durven uiten van een gedachte houdt altijd het risico in dat je het mis
hebt. We zijn heus niet zo star of voorzichtig dat we geen fouten durven
maken. Liever delen we ideeën die u mogelijk iets opleveren, dan dat we
ze inslikken uit angst dat ze achteraf overdreven of pijnlijk blijken.

Arthur C. Clarke merkte scherp op dat er twee essentiële redenen zijn
waarom voorspellingen over de toekomst doorgaans mislukken: `gebrek aan
moed en gebrek aan verbeeldingskracht.' Hij schreef dat `een gebrek aan
moed het vaakst voorkomt. Het treedt op wanneer, ondanks dat alle
relevante feiten bekend zijn, de voorspeller niet inziet dat ze
onvermijdelijk tot één conclusie leiden. Sommige van deze voorspellingen
zijn zo belachelijk dat ze bijna ongelooflijk
lijken.'\footnote{Zie Samuel L. Popkin, \emph{The Rational Peasant}
  (Berkeley: University of California Press 1979), p.~13.}\footnote{Zie
  Bois, op. cit.}

Als onze benadering van de informatierevolutie tekortschiet -- en dat
zal onvermijdelijk gebeuren -- komt dat eerder door een gebrek aan
verbeeldingskracht dan door een gebrek aan moed. Het voorspellen van de
toekomst is immers altijd een gedurfde onderneming die terecht scepsis
oproept. Wellicht zal de tijd aantonen dat onze conclusies de bal
compleet mis slaan. In tegenstelling tot Nostradamus gedragen wij ons
niet als profeten. We doen geen vage voorspellingen door met een
toverstok in een kom water te roeren of door horoscopen te raadplegen,
en we gebruiken ook geen cryptisch taalgebruik. We willen u een scherpe
en onbevooroordeelde kijk geven op zaken die wel eens van groot
persoonlijk belang zouden kunnen blijken.

Wij vinden het onze plicht om onze ideeën uiteen te zetten, ook al
zullen sommigen ze ervaren als heiligschennis, omdat ze anders wellicht
onopgemerkt blijven. In de gesloten denkwijze van de late
geïndustrialiseerde samenleving gaan ideeën niet zo vrij rond als via de
gevestigde media zou moeten.~

Dit boek is tot stand gekomen vanuit een constructieve instelling. Het
is het derde boek dat we gezamenlijk schrijven, waarin we de
verschillende fasen van de ingrijpende verandering die momenteel gaande
is analyseren. Net als \emph{Blood in the Streets} en \emph{The Great
Reckoning} vormt dit een gedachte-experiment. Het onderzoekt hoe de
industriële samenleving ten onder zal gaan en vervolgens opnieuw vorm
zal krijgen. Wij voorzien dat de komende jaren verbluffende paradoxen
aan het licht zullen komen. Enerzijds zul je de opkomst van een nieuwe
vrijheid ervaren, waarin het soevereine individu centraal staat, en mag
je verwachten dat de productiviteit vrijwel volledig bevrijd zal worden.
Tegelijkertijd zien wij het verval van de moderne natiestaat voor ons.
Veel garanties omtrent gelijkheid, die in de twintigste eeuw als
vanzelfsprekend werden beschouwd, zullen met die staat verdwijnen. Wij
zijn ervan overtuigd dat de representatieve democratie, zoals we die nu
kennen, zal vervagen en plaats zal maken voor een keuzedemocratie in de
digitale marktplaats. Indien onze inzichten kloppen, wordt de politiek
van de volgende eeuw veel diverser en minder belangrijk dan we nu gewend
zijn.

We zijn ervan overtuigd dat onze redenering goed te volgen is, ook al
brengt ze u langs intellectueel ruig terrein en minder vertrouwde
uithoeken. Mocht onze intentie op bepaalde punten niet glashelder
overkomen, dan is dat niet omdat we geheimzinnig willen doen of de
klassieke dubbelzinnigheid toepassen, zoals in de vage cryptische
uitspraken van sommige anderen. Wij gebruiken geen dubbelzinnigheid.
Indien onze argumenten onduidelijk lijken, komt dat vooral doordat we er
niet in geslaagd zijn onze overtuigende ideeën simpel en toegankelijk te
verwoorden. In tegenstelling tot veel voorspellers willen wij dat je
onze gedachtegang doorgrondt en zelfs eigen maakt. Onze visie steunt
niet op spirituele zweverigheid of kosmische bewegingen, maar op
ouderwetse, ongepolijste logica. Om zeer logische redenen zijn we ervan
overtuigd dat microprocessing de natiestaat onvermijdelijk zal
ondermijnen en vernietigen, terwijl het tegelijkertijd nieuwe vormen van
sociale organisatie zal voortbrengen. Het is niet alleen mogelijk, maar
ook noodzakelijk dat u zich een beeld vormt van bepaalde aspecten van de
nieuwe manier van leven --- die misschien eerder werkelijkheid wordt dan
u verwacht.

\subsection{Ironieën van een voorspelde
toekomst}\label{ironieuxebn-van-een-voorspelde-toekomst}

Al eeuwenlang wordt het einde van dit millennium beschouwd als een
beladen moment in de geschiedenis. Meer dan 850 jaar geleden stelde
St.~Malachy vast dat het jaar 2000 de datum van het Laatste Oordeel zou
zijn. In 1934 voorspelde de Amerikaanse helderziende Edgar Cayce dat de
aarde in 2000 op haar as zou draaien, waardoor Californië in tweeën zou
splijten en New York City en Japan overstromingen zouden ondervinden. In
1980 verklaarde de Japanse raketwetenschapper Hideo Itokawa dat de
uitlijning van de planeten in een `Grand Cross' op 18 augustus 1999
grootschalige milieucatastrofes zou veroorzaken, met als uiteindelijke
gevolg het einde van het menselijk leven op aarde.\footnote{Zie Frances
  en Joseph Gies, \emph{Cathedral, Forge, and Waterwheel: Technology and
  Invention in the Middle Ages} (New York: HarperCollins, 1994), p.40.}

Zulke apocalyptische visioenen worden al snel het onderwerp van spot.
Hoewel het jaar 2000 een symbolisch en opvallend rond getal is, is het
in feite niet meer dan een toevallig product van de westerse,
christelijke kalender. In andere kalenders en jaartellingen beginnen de
eeuwen en millennia op totaal andere tijdstippen in de geschiedenis.
Volgens de islamitische kalender komt 2000 n.Chr. overeen met het jaar
1378, wat op het eerste gezicht een gewoon jaartal lijkt. In de Chinese
kalender, die zich elke zestig jaar herhaalt, valt 2000 in een
drakenjaar, onderdeel van een voortdurende cyclus die al millennia
doorgaat. Er schuilt echter meer achter het jaar 2000 dan louter
religieuze betekenis. Het belang ervan wordt niet alleen gedragen door
de christelijke traditie, maar ook door de beperkingen van de
informatietechnologie in de twintigste eeuw.

Het zogenaamd Y2K-probleem, een potentieel verwoestende logische fout in
miljarden regels computercode, dreigt op middernacht van het millennium
de vitale onderdelen van de industriële samenleving plat te leggen. Veel
computers en microprocessors maken nog steeds gebruik van software uit
de begintijd van de computer, toen geheugenruimte --- à \$600.000 per
megabyte --- kostbaarder was dan goud. Om dure opslagruimte te besparen
gebruikten de eerste programmeurs slechts de laatste twee cijfers van
het jaartal. Deze gewoonte om tweecijferige datumvelden te gebruiken
werd overgenomen in de meeste software voor mainframecomputers, en vond
later ook brede toepassing in pc's en zogenoemde embedded chips ---
microprocessors die bijna alles aansturen: van videorecorders tot het
ontstekingssysteem van auto's, van beveiligingssystemen tot telefoons,
van schakeltechniek in het telefoonnetwerk tot proces- en
controlesystemen in fabrieken, energiecentrales, olieraffinaderijen,
chemische installaties, pijpleidingen en meer. In zo'n verkort veld
wordt het jaar 1999 bijvoorbeeld weergegeven als `99'. De vraag is
echter: wat gebeurt er in het jaar 2000? Een groot deel van de computers
zouden de `00' mogelijk interpreteren als het jaar 1900. Hierdoor zouden
tal van niet-geüpdatete computers en andere digitale apparaten het jaar
2000 onjuist kunnen registreren.

Dit zou tot ernstige datacorruptie kunnen leiden, wat onbedoeld ook
nieuwe mogelijkheden voor informatieoorlogvoering introduceert. In de
informatiesamenleving zullen potentiële vijanden complete systemen
kunnen ontregelen met `logische bommen': stukjes code die cruciale data
saboteren en zo alles ontwrichten.

Tijdens een militaire operatie hoef je bijvoorbeeld niet per se een
vliegtuig neer te schieten als je de gegevens, die essentieel zijn voor
de veilige werking ervan, kunt verstoren. Datacorruptie kan bijna net zo
verwoestend zijn als fysieke wapens wanneer het gaat om het ontregelen
van een moderne samenleving. De mogelijke verstrekkende gevolgen hiervan
zouden bij nader inzien voor de hand moeten liggen. Zo meldde de
\emph{Mail of London} op 14 december 1997 dat luchtvaartmaatschappijen
wereldwijd van plan waren om op 1 januari 2000 honderden vluchten te
annuleren, uit angst dat de luchtverkeersleidingssystemen zouden falen.
Niet alleen deze systemen, maar ook de datumgevoelige functies in de
vliegtuigen zelf zouden risico lopen. Volgens Boeing hebben veel
vliegtuigen Y2K-aanpassingen nodig. Een groot deel van de apparatuur zou
storingen kunnen vertonen wanneer ze een gebeurtenis op een ongeldige
datum registreren. De computergestuurde fly-by-wire-systemen die
vliegtuigen aansturen, zouden defect kunnen raken als hun programmering
zou concluderen dat cruciaal onderhoud voor het laatst in 1900 heeft
plaatsgevonden. Ze kunnen zelfs in een foutlus vastlopen en daardoor
uitschakelen.

De potentieel dodelijke kettingreacties van een logische tijdbom die
cruciale besturingssystemen platlegt, zouden de eeuwwisseling op een
onvergetelijk onaangename wijze kunnen markeren. Bedenk dat allerlei
apparaten in een foutlus kunnen raken en uitschakelen -- en dat kan
gebeuren, ook als je het geluk hebt niet midden in de lucht te zitten
wanneer het nieuwe millennium aanbreekt.

De potentieel dodelijke kettingreacties van een logische tijdbom die
cruciale besturingssystemen platlegt, zouden de eeuwwisseling op een
onvergetelijk onaangename wijze kunnen markeren. Zelfs als je op
oudejaarsnacht niet in een vliegtuig zit, kunnen talloze apparaten om je
heen crashen door foutlussen en je dagelijkse leven flink verstoren.
Zorg ervoor dat je ongelukken voorkomt, of ze nu ontstaan door
pacemakers die niet Y2K-conform zijn of door dronken millenniumvierders.
Als zulke pacemakers kunnen falen, kan ook het telefoonsysteem het
begeven, waardoor je mogelijk niet eens de hulpdiensten kunt bereiken.

Tenzij je je in een regio met slecht functionerende infrastructuur
bevindt, ben je eraan gewend dat je gewoon je telefoon oppakt en direct
kunt bellen. Gelukkig hoeven we ons zelden druk te maken over de
technische details van telefoonnetwerken. De schakelapparatuur en
routers van de telecominfrastructuur blijken echter erg afhankelijk van
datumvelden. Alle verbindingen worden opgeslagen met datum en tijd, wat
essentieel is voor het berekenen van de gespreksduur en de daarop
gebaseerde facturering. Als je op 31 december 1999 om 23:59:30 een
gesprek van precies één minuut voert en het systeem om 00:00:30 dit
gesprek registreert met een negatieve duur van meer dan 99 jaar, kunnen
foutlussen ontstaan en schakelt het systeem zich mogelijk uit. Hoewel
langeafstands firma\textquotesingle s, en lokale dienstverleners
waarschijnlijk ook, enorme bedragen investeren in het updaten van hun
schakelsystemen zodat deze Y2K-conform worden, kan het hele netwerk in
de problemen komen als zelfs een paar kleinere bedrijven niet voldoen en
uitvallen. In dat scenario mag je van geluk spreken als je op 1 januari
2000 überhaupt nog een verbinding krijgt.

Zoals Y2K-expert Peter de Jager het verwoordde: ``Als we het vermogen
verliezen om te bellen, verliezen we alles. We verliezen elektronische
betalingen, we verliezen de handel, we verliezen het bankkantoor.'' En
de gevolgen van Y2K-fouten zouden zelfs nog verder kunnen reiken.

Niemand weet precies in hoeverre cruciale systemen door het
millenniumprobleem ten onder zullen gaan. Embedded systemen -- die niet
herprogrammeerd kunnen worden en dus vervangen moeten worden zodra zij
datumgevoelige fouten vertonen -- vind je onder andere in auto's,
vrachtwagens en bussen die na 1976 gebouwd zijn. De kans om betrokken te
raken bij een ongeluk met voertuigen, bestuurd door mensen met
niet-conforme pacemakers, zal uiteindelijk dus misschien niet heel groot
zijn omdat die voertuigen mogelijk niet eens zullen starten. Ook in
energiecentrales, water- en rioleringssystemen, medische apparatuur,
militaire uitrusting, vliegtuigen, offshore olieplatforms, olietankers,
alarmsystemen en liften komen embedded systemen veel voor. Hoewel veel
installaties met microprocessors zelf geen datumgevoelige functies
uitvoeren, zijn ze voor hun interne werking vaak wel afhankelijk van een
klok, die mogelijk Y2K-gevoelig is.

\section{Mainframes en de
Y2K-tijdbom}\label{mainframes-en-de-y2k-tijdbom}

De grootschalige bestuurs- en controlesystemen van de overheid en grote
bedrijven, waarin enorme transacties via mainframecomputers worden
verwerkt, stonden aanvankelijk centraal in de Y2K-bezorgdheid. Deze
systemen draaien op grote machines met software die vaak decennia oud en
grotendeels niet Y2K-conform is. Daarom richtten de eerste
Y2K-waarschuwingen, voor het eerst geuit door Peter de Jager begin jaren
1990, zich voornamelijk op de noodzaak om de besturingssystemen van
grote, multiprocessor-mainframes te moderniseren. De heer de Jager uitte
de zorg dat er mogelijk niet genoeg programmeurs beschikbaar zullen zijn
die kennis hebben van COBOL -- de oude computertaal waarin de
noodzakelijke patches en reparaties aan datumgevoelige code uitgevoerd
moeten worden -- zelfs als ieder bedrijf en overheidsinstantie met een
kwetsbaar systeem enkele jaren daarvoor een noodprogramma zou hebben
opgestart. Aangezien dit niet is gebeurd en veel beheerders van
datumgevoelige informatiesystemen pas recent begonnen zijn met het
onderzoeken van hun kwetsbaarheden, mag je er met grote zekerheid van
uitgaan dat veel mainframesystemen niet goed voorbereid zullen zijn.

Dit vormt uiteraard een groot probleem, omdat er in de huidige economie
echt geen alternatief bestaat voor computerverwerking. De meeste
bedrijven die groot genoeg zijn om een mainframe te gebruiken voor het
verwerken van hun transacties, zijn namelijk afhankelijk van een
dusdanig groot transactievolume dat ze niet meer met de ouderwetse
papieren systemen uit de negentiende eeuw kunnen functioneren. Als die
ondernemingen noodgedwongen weer op papieren documenten zouden moeten
vertrouwen, zouden zij slechts een fractie van hun gebruikelijke
transacties kunnen afhandelen. Het plotselinge inkomensverlies, als
gevolg van zo'n dramatische daling in bedrijfsactiviteit, zou het
voortbestaan van bijna alle bedrijven, op enkele van de best
gekapitaliseerde na, ernstig in gevaar brengen.

Bijna alle financiële processen -- zoals facturering, inkoop- en
loonadministratie, voorraadbeheer en naleving van regelgeving -- zullen
compleet worden verstoord. Enorme hoeveelheden data gaan verloren als
computers crashen of foutieve informatie uitspuwen door het
Y2K-probleem. In sommige gevallen blijkt het zelfs een zegen als
systemen direct crashen, in plaats van dat hun data geleidelijk wordt
gecorrumpeerd tot een massale storing het probleem aan de kaak stelt.
Wat zal er gebeuren wanneer een backup-programma bestanden die
oorspronkelijk op 07/04/99 zijn aangemaakt, kopieert naar een update op
01/04/00? Wie zal het zeggen? Zal de computer een betaling voor een
verzekeringspolis die op 4 januari `1900' uitgevoerd is, interpreteren
als een signaal dat de polis al een eeuw in gebreke is, met als gevolg
dat de polis wordt geannuleerd en uit de administratie verdwijnt? Zullen
computers van banken en financiële instellingen proberen om honderd jaar
rente in rekening te brengen voor leningen die de overgang naar het
nieuwe millennium overspannen? Zullen banken en effectenmaatschappijen
nauwkeurig rekeningstanden bijhouden en tijdig toegang geven tot
tegoeden? Dit zijn slechts enkele van de intrigerende vraagstukken
waarmee je te maken krijgt als gevolg van het Y2K-probleem.

\begin{quote}
`Dit is mogelijk het meest destructieve onderdeel van het jaar
2000-probleem. Dit betreft niet het ongemak wanneer uw salaris enkele
dagen te laat komt. Dit gaat over echte chaos op straat.'\\
- Dr.~Leon Kappelman, medevoorzitter van de Year 2000-werkgroep van de
Society For Information Management.
\end{quote}

Ook moet je je afvragen wat er gebeurt als de elektriciteit uitvalt door
Y2K-gerelateerde storingen. Zonder stroom functioneren de meeste
systemen niet, ook degene die niet direct door Y2K-problemen getroffen
zijn, zoals je koelkast, vriezer en mogelijk zelfs je verwarming.
Y2K-problemen kunnen de veiligheidsgerelateerde toegangs- en
beheerssystemen in kerncentrales verstoren. Zo dragen medewerkers in
kerninstallaties dosimetrie-apparaten die de hoeveelheid
stralingsblootstelling bijhouden die zij in de faciliteit ontvangen.
Deze apparaten worden regelmatig gecontroleerd en de gegevens over de
blootstellingsniveaus worden opgeslagen in een computersysteem dat de
toegang van medewerkers tot de faciliteit beheert. Uiteraard, als deze
bestuurlijke computers uitvallen, raken al die uitgebreide controles,
die bedoeld zijn om een veilige werking en goed onderhoud te garanderen,
volledig in de war. Belangrijker nog: een memo van de Nuclear Regulatory
Commission merkt op dat veel `niet-veiligheidsgerelateerde, maar
belangrijke computergebaseerde systemen, hoofdzakelijk databases en de
dataverzameling die onmisbaar is voor de werking van de centrale,'
datumgevoelig zijn.

Ook reguliere elektriciteitscentrales lopen net zo goed risico op
Y2K-verstoringen. Kolencentrales zijn immers gevoelig voor
onderbrekingen in het transportsysteem dat de kolen naar de ketels
vervoert. In het winterseizoen van 1997/1998 moesten exploitanten van
kolencentrales in sommige gevallen hun productie verminderen door
vertragingen in de leveringsketen van Western Coal, veroorzaakt door de
fusie van de spoorwegsystemen van Southern Pacific en Union Pacific. Het
probleem ontstond door incompatibiliteiten tussen de computerbestuurde
controle- en dispatchsystemen die beide spoorwegmaatschappijen
hanteerden. Een woordvoerder van Union Pacific noemde de integratie van
de twee systemen een `nachtmerrie', ondanks dat Union Pacific
Technologies als koploper wordt gezien in de ontwikkeling van
geautomatiseerde transportsystemen. Door de programmeerproblemen kon de
spoorwegmaatschappij de bewegingen van haar goederenwagons niet
nauwkeurig volgen. Het onvermogen van Union Pacific om de systemen van
Southern Pacific succesvol te integreren, is een slecht voorteken voor
wat er kan gebeuren wanneer de logische tijdbommen van Y2K het
transport, de elektriciteitsopwekking en andere economische sectoren
verstoren.

De grootste zorg rond het elektriciteitsnet komt voort uit het feit dat
computers voortdurend toezicht houden op het hele systeem en het actief
aansturen,~ waardoor elektriciteit efficiënt wordt verplaatst van
overschotgebieden naar regio's met een tekort. Computers monitoren dit
proces nauwgezet om spanningspieken en storingen te voorkomen. Elke
overdracht wordt geregistreerd met vermelding van tijd en datum,
vergelijkbaar met de registratie bij een telefoongesprek. Robuuste
mechanische relais maken de verbindingen weliswaar mogelijk, maar
computersystemen sturen ze aan. Die computercontroles, die essentieel
zijn voor de vermogensbalancering, kunnen echter om dezelfde redenen
falen als telefonienetwerken. In Noord-Amerika zijn de systemen die de
stroomverdeling regelen sterk afhankelijk van telefoonverbindingen zoals
T-1-lijnen en draadloze netwerken. Als het telefonienetwerk uitvalt, kun
je er dus ook op rekenen dat de elektriciteitsvoorziening instort. En de
ervaring in Canada in januari 1998 toont aan hoe lastig het is om het
systeem weer aan de praat te krijgen zodra de elektriciteit over een
groot gebied wegvalt. Stroomuitvallen zouden oncomfortabel lang kunnen
aanhouden.

\section{Y2K en het nucleaire
arsenaal}\label{y2k-en-het-nucleaire-arsenaal}

Als de stroom in het hart van de winter zou uitvallen, betekent dat in
de moderne economie een enorme verstoring en zelfs een potentiële
bedreiging voor de volksgezondheid, vooral voor mensen die afhankelijk
zijn van elektrische verwarming en medische apparatuur. Het ergste
scenario is echter nog grimmiger. John Koskinen, hoofd van het Y2K
Conversion Council van president Clinton, waarschuwt dat de Amerikaanse
militaire arsenalen mogelijk niet meer functioneren zodra de klok op 31
december 1999 middernacht slaat. Hoewel hij geen onnodige paniek wil
zaaien, voegt Koskinen eraan toe: `Er moet zich zorgen over worden
gemaakt.' Een van de zorgen over de nucleaire raketten was dat `als de
data niet naar behoren functioneert, ze vanzelf zouden kunnen lanceren.'
Natuurlijk geldt dezelfde zorg -- zo niet meer -- voor de Russische
nucleaire raketten. Ruslands faillissement heeft de noodzakelijke
upgrades voor Y2K-compatibiliteit nog problematischer gemaakt dan in de
Verenigde Staten, en er is bewijs dat Rusland de Y2K-conversie nog niet
serieus neemt. Hoewel iedereen hoopt dat er geen toevallige lanceringen
plaatsvinden, bestaat er weinig twijfel dat de jaarwisseling naar 2000
de wereldwijde veiligheid in gevaar zal kunnen brengen, simpelweg
doordat militaire communicatiesystemen in veel landen mogelijk niet naar
behoren zullen functioneren. Zoals Koskinen het verwoordt: `Als je in
een land zit en plotseling niet precies kunt achterhalen wat er gaande
is, omdat je communicatiesystemen niet optimaal werken, raak je nog
nerveuzer.' Neem dat dus zeker mee in je lijst met Y2K-zorgen. Deze
logische tijdbom kan de lancering van echte explosieve wapens
bewerkstelligen, wat het gevaar aantoont dat voortkomt uit
informatieoorlogvoering tegen gecentraliseerde beheers- en
controlesystemen.

Terroristen die een gecentraliseerd systeem willen treffen, zouden juist
31 december 1999 kunnen kiezen als moment om toe te slaan, omdat op dat
moment veel systemen op hun meest kwetsbaar zijn. Niet alleen raken de
communicatiesystemen zwaar overbelast, met als gevolg dat de stroom
uitvalt, voertuigen niet starten en de 112-diensten van de politie,
brandweer en ambulance niet bereikbaar zijn, maar ook tal van andere
voorzieningen, zoals de luchtverkeersleiding, kunnen het begeven. Zonder
stroom is er geen kraanwater, vallen rioolinstallaties uit en zullen
verkeerslichten ermee ophouden. Binnen enkele uren nadat het
transportsysteem instort, zal het voedsel in supermarkten uitverkocht
zijn, of zelfs geplunderd. Recente ervaringen in Amerikaanse steden
laten zien dat een gebrek aan stroom, water, verwarming, licht en
communicatie met hulpdiensten, waaronder politie en brandweer, samen het
verval van de beschaving kunnen inluiden. Hoewel niemand met zekerheid
kan vaststellen wat de impact van het Y2K-probleem zal zijn, zou dit
kunnen uitmonden in plunderingen en rellen op straat, zeker als het
duidelijk wordt dat er problemen zullen ontstaan bij het uitgeven van
loon-, bijstands- en pensioenbetalingen.

\begin{quote}
'Wij zullen niet zijn wat wij geweest zijn, maar we zullen beginnen
anders te zijn.'17\\
- Joachim de Fiore
\end{quote}

Doemvoorspellingen rondom het nieuwe millennium berusten niet per se op
de christelijke theologie, maar sluiten naadloos aan bij de
millennialtraditie van Joachim de Fiore. Zijn meditaties deden hem
immers geloven dat Christus slechts `het tweede scharnier in de
geschiedenis' vormde en dat er onvermijdelijk een volgend scharnier zou
volgen.18 Zo stelt filosoof Michael Grosso dat de informatierevolutie de
menselijke geschiedenis stuurt naar de verwezenlijking van de
profetische visie van de Westerse wereld. Hij noemt dit `technocalypse.'
Of technologische ontwikkelingen nu wel of niet door millenniale visies
worden beïnvloed, het Y2K-fenomeen blijft een product van de
overheersende westerse blik op de tijd. Op een merkwaardige manier sluit
het aan bij dromen, visioenen of hun numerieke interpretaties, zoals
Newtons duiding van de profetieën van Daniël. Deze intuïtieve gedachten
ontstaan vanuit een perspectief dat de geboorte van Christus als
middelpunt van de geschiedenis beschouwt. Bovendien versterken grote,
ronde getallen dit perspectief, want elke handelaar herkent de
bijzondere aantrekkingskracht die zij uitoefenen.

Het tweeduizendste jaar van onze jaartelling kan niet anders dan een
brandpunt worden voor de verbeelding van intuïtieve denkers.

Critici zouden deze voorspellingen zonder moeite belachelijk kunnen
maken, zelfs zonder ~in te gaan op de dubbelzinnige en omstreden
theologische begrippen van de Apocalyps en het Laatste Oordeel, die
juist zoveel kracht aan deze vooruitzichten geven. Opmerkelijk genoeg
weegt de Y2K-bug zwaarder dan de historische onnauwkeurigheden in onze
tijdrekening, die het symbolische gewicht van het jaar 2000 binnen de
Christelijke jaartelling juist hadden kunnen afzwakken. Het jaar 2000
kan een keerpunt in de geschiedenis betekenen, simpelweg omdat het de
overgang naar een nieuw millennium markeert. In principe begint het
volgende millennium pas in 2001, aangezien het jaar 2000 slechts het
tweeduizendste jaar sinds de geboorte van Christus aanduidt. Tenminste,
dat zou zo zijn als Christus daadwerkelijk in het eerste jaar van ons
huidige tijdperk was geboren, maar dat is niet zo. In 533, toen men
bepaalde om de geboorte van Christus als uitgangspunt te nemen voor de
westerse jaartelling in plaats van de oprichtingsdatum van Rome, maakten
de monniken die deze nieuwe conventie invoerden een rekenfout.
Tegenwoordig gaan we ervan uit dat hij in 4 v.Chr. werd geboren. Volgens
deze berekening eindigt de volledige periode van tweeduizend jaar sinds
zijn geboorte al rond 1997. Dit is ook de reden waarom Carl Jung dit
ogenschijnlijk vreemde jaartal als startdatum voor een Nieuwe Tijd
voorstelde.

Lach erom als je wilt, maar wij wijzen intuïtieve inzichten in de
geschiedenis beslist niet af. Hoewel ons betoog op logica rust en niet
op vooringenomen veronderstellingen, raken we diep onder de indruk van
de profetische kracht van het menselijk bewustzijn. Keer op keer
bevestigt het de visioenen van krankzinnigen, helderzienden en heiligen.
Hetzelfde geldt voor de transformatie die met het jaar 2000 op handen
lijkt te komen. De datum, die al lang verankerd is in de westerse
verbeelding, lijkt het keerpunt te markeren dat bewijst dat de
geschiedenis een bestemming heeft. We kunnen niet precies verklaren
waarom dat zo is, maar we hebben er desondanks alle vertrouwen in.

Onze intuïtie vertelt ons dat de geschiedenis een bestemming kent en dat
vrije wil en determinisme twee zijden van dezelfde medaille vormen. De
menselijke interacties die de geschiedenis vormgeven, lijken erop te
wijzen dat een bepaald lot hen stuurt. Net zoals een elektronenplasma --
een dicht samengepakt gas van elektronen -- zich als een complex systeem
gedraagt, handelen mensen op een vergelijkbare manier. De individuele
bewegingsvrijheid van elektronen blijkt immers verenigbaar te zijn met
een sterk geordend collectief gedrag. Zoals David Ohm over een
elektronenplasma opmerkte, is de menselijke geschiedenis `een hoogst
georganiseerd systeem dat zich als een geheel gedraagt.'

Het doorgronden van de werking van de wereld betekent dat je leert
aanvoelen hoe de menselijke samenleving handelt volgens de wiskundige
wetten van de natuur. De werkelijkheid verloopt niet lineair, terwijl de
meeste mensen juist een lineair patroon verwachten. Om de dynamiek van
verandering echt te begrijpen, moet je inzien dat onze samenleving --
net als andere complexe systemen in de natuur -- cyclische patronen en
abrupte breuken vertoont. Dit houdt in dat bepaalde aspecten van de
geschiedenis de neiging hebben zich te herhalen en dat ingrijpende
veranderingen vaak plotseling plaatsvinden in plaats van geleidelijk.

Te midden van de vele ritmes in het menselijk bestaan duikt er een
raadselachtige vijfhonderdjarige cyclus op, die telkens samenvalt met
grote omslagen in de westerse geschiedenis. Nu het jaar 2000 nadert,
valt het op dat het laatste decennium van elke eeuw die deelbaar is door
vijf, altijd gepaard gaat met een diepgaande transitie binnen de
Westerse wereld. Dit patroon markeert nieuwe fasen van sociale
organisatie, vergelijkbaar met de manier waarop geboorte en dood de
cyclus van menselijke generaties bepalen. Dit is al zo sinds ten minste
500 v.Chr., toen de Griekse democratie opkwam door de constitutionele
hervormingen van Kleisthenes in 508 v.Chr. De volgende vijfhonderd jaar
kenden een periode van groei en versterking van de Oude economie, die
culmineerde in de geboorte van Christus in 4 v.Chr. Deze periode bleek
voor dat economische systeem tevens de tijd met de grootste welvaart te
zijn, waarin de rentevoeten hun laagste niveau bereikten. Een niveau dat
niet meer bereikt is tot in de moderne tijd.

In de daaropvolgende vijfhonderd jaar nam de welvaart geleidelijk af,
wat uiteindelijk leidde tot de ineenstorting van het Romeinse Rijk tegen
het einde van de vijfde eeuw na Christus. Het is de moeite waard om de
samenvatting van William Playfair hier te herhalen: ``Het hoogtepunt van
de Romeinse macht viel samen met de geboorte van Christus, in de tijd
van keizer Augustus. Vanaf dat moment ging het rijk langzaam achteruit,
tot in 490 n.~Chr. haar laatste legioenen uiteenvielen en de westerse
wereld afgleed in wat we nu kennen als de Donkere middeleeuwen.''

De daaropvolgende vijfhonderd jaar werden gekenmerkt door een krimpende
economie; de langafstandshandel kwam tot stilstand, steden raakten
verlaten, geld kwam in onbruik en kunst en geletterdheid verdwenen bijna
volledig. De ineenstorting van het West-Romeinse Rijk leidde tot het
verval van een effectief rechtsstelsel, waardoor er primitieve
regelingen ontstonden voor het oplossen van geschillen. Tegen het einde
van de vijfde eeuw werd bloedwraak steeds meer de norm, en in het jaar
500 vond het eerste geregistreerde geval van een godsoordeel plaats.

Nogmaals, duizend jaar geleden markeerde het laatste decennium van de
tiende eeuw een andere enorme omwenteling in sociale en economische
systemen. Wellicht is de minst bekende van deze transities de feodale
revolutie, die begon in een periode van totale economische en politieke
onrust. In \emph{Transformation of the Year One Thousand} stelt Guy
Bois, professor middeleeuwse geschiedenis aan de Universiteit van
Parijs, dat deze breuk aan het einde van de tiende eeuw zorgde voor de
volledige ineenstorting van de overblijfselen van oude instituties en de
opkomst van iets nieuws uit de anarchie van het feodalisme.20 Zoals
Raoul Glaber het verwoordde: 'Er werd gezegd dat de hele wereld in één
klap de lompen van de oudheid van zich afschudde.'21 Het nieuwe systeem
dat plotseling opkwam, stimuleerde de geleidelijke heropleving van de
economische groei en zorgde ervoor dat de daaropvolgende vijfhonderd
jaar -- die we nu de middeleeuwen noemen -- gepaard gingen met een
heropleving van het gebruik van geld en internationale handel, naast de
herontdekking van rekenkunde, geletterdheid en tijdsbesef.

In het laatste decennium van de vijftiende eeuw vond een nieuw keerpunt
plaats. Europa herstelde zich uit het door de Zwarte Dood veroorzaakte
demografische tekort en begon vrijwel onmiddellijk haar macht over de
wereld uit te breiden. De `buskruitrevolutie', de `renaissance' en de
`reformatie' benoemden elk een aspect van deze overgangsperiode die de
moderne tijd inluidde. Het begon spectaculair toen Karel VIII Italië
binnenviel met nieuwe bronzen kanonnen. Dit opende de weg naar nieuwe
werelddelen, zoals bleek uit Columbus' reis naar Amerika in 1492. Die
ontdekking stimuleerde de meest dramatische economische groei die de
mensheid ooit heeft gekend en ging gepaard met een revolutie in de
natuurkunde en astronomie, wat de moderne wetenschap deed ontstaan.
Bovendien verspreidde de boekdruktechnologie de ideeën op grote schaal.

Wij bevinden ons nu op de drempel van een nieuwe millenniale
transformatie. De uitgebreide systemen voor beheer en controle die we
uit het industriële tijdperk hebben geërfd, kunnen bij de overgang naar
het nieuwe millennium uit elkaar vallen zoals de paardenkar. Of de
Y2K-logicabom nu wel of niet zal leiden tot een onmiddellijke
ineenstorting van de industriële samenleving maakt niet uit, het einde
ervan is hoe dan ook nabij. Wij verwachten dat de opkomst van de
informatiesamenleving de wereld ingrijpend zal veranderen, op manieren
die in dit boek worden toegelicht. Het is begrijpelijk dat je twijfels
hebt omdat een cyclus die zich slechts tweemaal per millennium herhaalt
niet genoeg herhalingen kent voor statistische significantie. Economen
benaderen zelfs kortere cycli met scepsis, omdat ze meer overtuigend
statistisch bewijs verlangen. Professor Dennis Robertson schreef ooit
dat we enkele eeuwen moeten afwachten alvorens met zekerheid te kunnen
spreken over het bestaan van de vier en acht tot tien jaar durende
handelscycli.22 Volgens dat criterium zou hij ongeveer dertigduizend
jaar moeten wachten om zeker te weten dat de vijfhonderdjarige cyclus
geen toevalligheid betreft. Wij zijn minder dogmatisch en erkennen
liever dat de patronen van de werkelijkheid complexer zijn dan de
statische en lineaire evenwichtsmodellen die de meeste economen
hanteren.

Wij menen dat de komst van het jaar 2000 meer betekent dan slechts een
praktische onderverdeling van een eindeloos tijdscontinuüm. Wij geloven
dat het een keerpunt vormt tussen de Oude Wereld en een opkomende Nieuwe
Wereld. Het industriële tijdperk verdwijnt in hoog tempo, en ironisch
genoeg kan de ondergang ervan worden versneld doordat men in vroege
computers, door het dure geheugen, tweecijferige datumvelden ging
gebruiken. Toen Hollerithponskaarten slechts tachtig tekens konden
bevatten, leek het verstandig om om datums af te korten. In
tegenstelling tot de verwachtingen van de vroege programmeurs hield deze
afkorting echter vier decennia stand tot het einde van het millennium,
met een onbedoelde logicabom, die een groot deel van de industriële
samenleving zou kunnen verwoesten, tot gevolg. Het Office of Management
and Budget van de Amerikaanse overheid omschreef het computerprobleem in
\emph{Getting Federal Computers Ready for 2000}, een rapport van 7
februari 1997: `Tenzij ze gerepareerd of vervangen worden, zullen ze bij
de overgang van de eeuw op één van de volgende drie manieren falen:
ofwel wijzen ze legitieme invoer af, ofwel berekenen ze foutieve
resultaten, ofwel functioneren ze simpelweg niet.' Deze drie gevolgen
zouden samen de industriële samenleving compleet kunnen lamleggen. In
elk geval staat massaproductietechnologie op het punt te worden
overschaduwd door een nieuwe miniaturisatietechnologie, en een crisis op
korte termijn zal dat proces alleen maar versnellen. Dankzij de nieuwe
informatietechnologie is een wetenschap op het gebied van niet-lineaire
dynamica ontstaan, waarvan de verrassende conclusies nog slechts
afzonderlijke elementen zijn die samen moeten worden geweven tot een
coherent wereldbeeld. We leven in het tijdperk van de computer, maar
onze dromen worden nog steeds op het weefgetouw gesponnen. We houden
vast aan de metaforen en denkbeelden van het industrialisme. Denkers die
stierven lang voordat bijna iedereen die nu leeft geboren werd, zoals
Adam Smith en Karl Marx, hebben aangetoond dat onze politieke arena nog
steeds langs de industriële scheidslijn tussen rechts en links
balanceert. Wij stellen echter dat het `gezonde verstand' van het
industriële tijdperk op vele terreinen niet meer van toepassing is in de
radicaal veranderende wereld.

Meer dan vijfentachtig jaar na de dag in 1911 waarop Oswald Spengler
werd getroffen door een voorgevoel van een naderende wereldoorlog en `de
ondergang van het Westen', zien wij eveneens '\emph{een historische
fasewisseling} plaatsvinden \ldots{} precies op het moment dat al
honderden jaren voorbestemd lijkt te zijn.'23 Net als Spengler voorzien
wij de naderende ondergang van de westerse beschaving en daarmee de
ineenstorting van de wereldorde die de afgelopen vijf eeuwen heeft
gedomineerd, sinds Columbus naar het westen zeilde om contact te leggen
met de Nieuwe Wereld. Maar in tegenstelling tot Spengler zien wij in het
komende millennium juist de geboorte van een nieuwe fase binnen de
westerse beschaving.

{[}\^{}27{]} De nomenklatura vormt de gevestigde elite die in de
voormalige Sovjet-Unie en andere door de staat beheerde economieën de
leiding voerde.

\bookmarksetup{startatroot}

\chapter{}\label{section}

\textbf{HOOFDSTUK 2}

\textbf{MEGAPOLITIEKE TRANSFORMATIES IN HISTORISCH PERSPECTIEF}

``\emph{In de geschiedenis, net als in de natuur, zijn geboorte en dood
voortdurend in balans''} - JOHAN HUIZINGA30

\textbf{Het verval van de moderne wereld}

Naar onze mening ben je getuige van niets minder dan het verval van de
Moderne Tijd. Het is een ontwikkeling gedreven door een meedogenloze
maar verborgen logica. Meer dan we doorgaans beseffen, en meer dan CNN
en de kranten ons vertellen, zal het volgende millennium niet langer
"modern" zijn. We zeggen dit niet om te suggereren dat je een wilde of
primitieve toekomst te wachten staat, hoewel dat mogelijk is, maar om te
benadrukken dat het tijdperk in de geschiedenis dat nu aanbreekt
kwalitatief zal verschillen van degene waarin je geboren bent.

Er komt iets nieuws aan. Net zoals dat agrarische samenlevingen
verschilden van jagers-verzamelaarsbenden, en industriële samenlevingen
radicaal verschilden van feodale of yeomanboerensystemen, zo zal de
komende Nieuwe Wereld een radicale breuk betekenen met al het
voorgaande.

In het nieuwe millennium zullen het economische en politieke leven niet
langer georganiseerd worden op gigantische schaal onder de heerschappij
van de natiestaat, zoals dat tijdens de moderne eeuwen het geval was. De
beschaving die ons wereldoorlogen, de lopende band, sociale zekerheid,
inkomstenbelasting, deodorant en de broodrooster bracht, is stervende.
Deodorant en de broodrooster zullen misschien overleven. De andere niet.
Als een oude, ooit machtige man heeft de natiestaat een toekomst die in
jaren en dagen te tellen is, en niet langer in eeuwen en decennia.

Overheden hebben al veel van hun regulerende en dwingende macht
verloren. De ineenstorting van het communisme markeerde het einde van
een lange cyclus van vijf eeuwen, waarin de omvang van macht de
efficiëntie van de overheid overschaduwde. Het was een tijd waarin
geweld in toenemende mate loonde, maar die tijd is voorbij. De wereld is
al begonnen aan een historische omwenteling van formaat. Misschien zegt
een toekomstige Gibbon, die de ondergang van het Moderne Tijdperk
beschreef, wel dat dat tijdperk al voorbij was toen jij dit boek in
handen kreeg. Terugkijkend kan hij zeggen, net als wij, dat het eindigde
met de val van de Berlijnse Muur in 1989, of met het uiteenvallen van de
Sovjet-Unie in 1991. Beide data kunnen een bepalende gebeurtenis blijken
in de evolutie van de beschaving, als het einde van wat we nu kennen als
de Moderne Tijd.

Het vierde stadium van de menselijke ontwikkeling komt eraan, en
misschien is het minst voorspelbare kenmerk ervan de nieuwe naam
waaronder het bekend zal worden. Noem het "Post-Modern." Noem het de
"Cybersamenleving" of het "Informatietijdperk." Of verzin zelf een naam.
Niemand weet welke bijnaam aan de volgende fase van de geschiedenis zal
blijven plakken.

We weten niet eens of de vijfhonderd jaar durende periode van de
geschiedenis die net eindigt, zal blijven worden beschouwd als "modern."
Als toekomstige historici kennis van etymologie hebben, zal dat niet zo
zijn. Een passendere titel zou kunnen zijn "Het Tijdperk van de Staat"
of "Het Tijdperk van Geweld." Zo\textquotesingle n naam zou echter
buiten de tijdsgeest vallen die momenteel de tijdperken van de
geschiedenis benoemt. "Modern," volgens het Oxford English Dictionary,
betekent "behorend tot de tegenwoordige en recente tijden, onderscheiden
van het verre verleden... In historisch gebruik gewoonlijk toegepast (in
tegenstelling tot antiek en middeleeuws) op de tijd volgend op de
MIDDELEEUWEN."

Westerse mensen dachten bewust aan zichzelf als "modern" pas toen ze
gingen begrijpen dat de middeleeuwse periode voorbij was. Voor 1500 had
niemand ooit aan de feodale eeuwen gedacht als een "midden"-periode in
de Westerse beschaving. De reden is duidelijk bij reflectie: voordat een
tijdperk redelijkerwijs kan worden gezien als ingeklemd in het "midden"
van twee andere historische tijdperken, moet het al tot een einde zijn
gekomen. Degenen die tijdens de feodale eeuwen leefden, hadden zich niet
kunnen voorstellen dat ze in een tussenhuis woonden tussen de oudheid en
de moderne beschaving, totdat het hen daagde niet alleen dat de
middeleeuwse periode voorbij was, maar ook dat de middeleeuwse
beschaving dramatisch verschilde van die van de Oudheid.

Menselijke culturen hebben blinde vlekken. We hebben geen vocabulaire om
paradigmaveranderingen in de grootste kaders van het leven te
beschrijven, vooral als die om ons heen gebeuren. Ondanks de vele
dramatische veranderingen die zich hebben ontvouwd sinds de tijd van
Mozes, hebben slechts enkele ketters de moeite genomen om na te denken
over hoe de overgangen van de ene fase van beschaving naar de andere
werkelijk plaatsvinden.

Hoe worden ze veroorzaakt? Wat hebben ze met elkaar gemeen? Aan welke
patronen kunnen we zien wanneer ze beginnen en hoe weten we wanneer ze
voorbij zijn? Wanneer zullen Groot-Brittannië of de Verenigde Staten ten
einde komen? Voor deze vragen zal het lastig zijn om conventionele
antwoorden te vinden.

\textbf{Het taboe op vooruitziendheid}

"Buiten" een bestaand systeem zien is als een toneelknecht die een
dialoog probeert te forceren met een personage in een toneelstuk. Het
schendt een conventie die helpt het systeem te laten functioneren. In
elke maatschappij geldt een stilzwijgend verbod: denk vooral niet na
over hoe het systeem ooit kan ophouden te bestaan, of wat ervoor in de
plaats zou kunnen komen. Impliciet is ieder systeem het laatste of het
enige systeem dat ooit zal bestaan, ook al wordt dit niet openlijk
uitgesproken. Van de mensen die ooit een geschiedenisboek hebben
gelezen, zal slechts een minderheid zo\textquotesingle n aanname
realistisch vinden als ze ermee geconfronteerd zouden worden. Desondanks
is dat de conventie die de wereld regeert. Ieder sociaal systeem doet
alsof haar regels nooit zullen worden vervangen, hoe sterk of zwak zij
ook aan de macht probeert vast te houden. Ze zijn het laatste woord. Of
misschien het enige woord. Primitievelingen geloven dat hun ideeën de
enige mogelijke manier beschrijven om het leven te organiseren.
Economisch meer gecompliceerde systemen die een gevoel van geschiedenis
incorporeren spannen hiervan gewoonlijk de kroon. Of het nu gaat om
Chinese mandarijnen aan het hof van de keizer, de marxistische
nomenklatura in Stalins Kremlin, of leden van het Amerikaanse Huis van
Afgevaardigden: de machthebbers verbeelden zich óf dat er geen
geschiedenis bestaat, óf dat zij het hoogtepunt ervan vormen ---
verheven boven iedereen die hen voorging, en als voorhoede van alles wat
nog komen zal.

Om bijna onvermijdelijke redenen is dit de waarheid. Hoe duidelijker het
is dat een systeem zijn einde nadert, hoe minder bereid mensen zullen
zijn om zich aan haar wetten te houden. Elke sociale organisatie zal
daarom de neiging hebben om analyses die zijn ondergang voorzien te
ontmoedigen of te bagatelliseren. Alleen al door dit feit worden grote
keerpunten in de geschiedenis zelden opgemerkt terwijl ze gebeuren. Eén
ding kun je in elk geval zeker weten over de toekomst: conventionele
denkers zullen grote veranderingen niet verwelkomen of aankondigen.

Conventionele informatiebronnen zullen je geen objectieve en tijdige
waarschuwing geven over hoe de wereld verandert en waarom. Als je de
grote overgang die nu gaande is wilt begrijpen, heb je geen andere keuze
dan het zelf uit te zoeken.

\textbf{Voorbij het voor de hand liggende}

Dit betekent verder kijken dan onze neus lang is. De geschiedenis toont
dat overgangen mogelijk decennia of zelfs eeuwen nadat ze gebeuren niet
worden erkend, zelfs wanneer die achteraf gezien onmiskenbaar reëel
bleken te zijn. Neem de val van Rome. Het was waarschijnlijk de
belangrijkste historische ontwikkeling in het eerste millennium van het
christelijke tijdperk. Zelfs lang nadat Rome gevallen was, hield men de
illusie van haar voortbestaan in stand, vergelijkbaar met het gebalsemde
lichaam van Lenin. Wie alleen naar het officiële nieuws luisterde, had
pas door dat Rome was gevallen toen dat allang geen verschil meer
maakte.

Dit kwam niet alleen door de beperkte communicatiemogelijkheden van de
antieke wereld. De uitkomst zou grotendeels hetzelfde zijn geweest als
CNN al zou hebben bestaan en haar videoband draaide in september 476.
Toen werd de laatste Romeinse keizer in het Westen, Romulus Augustulus,
gevangengenomen in Ravenna, en werd gedwongen om in een villa in
Campanië te pensioneren. Zelfs als Wolfe Blitzer daar zou zijn geweest
om met minicamera\textquotesingle s het nieuws in het jaar 476 op te
nemen, is het onwaarschijnlijk dat hij of iemand anders zou hebben
gedurfd die gebeurtenissen te karakteriseren als het einde van het
Romeinse Rijk. Dat is natuurlijk precies wat latere historici
uiteindelijk wel concludeerden.

De redacteuren van CNN zouden waarschijnlijk geen goedkeuring hebben
gegeven voor een voorpagina-artikel met de titel "Rome is gevallen." De
machthebbers ontkenden dit. "Nieuws"-kanalen omarmen zelden controverse
omdat dit hun inkomsten zou kunnen ondermijnen. Ze zijn vaak partijdig,
soms zelfs in grote mate. Maar ze rapporteren zelden conclusies die
abonnees zouden kunnen overtuigen om hun abonnementen op te zeggen en
ervan door te gaan. Daarom zou slechts een enkeling de val van Rome
hebben gerapporteerd, zelfs als het technologisch mogelijk zou zijn
geweest. Experts zouden meteen geroepen hebben dat het onzin was om te
beweren dat Rome gevallen was. Zoiets zeggen was niet goed voor de
handel --- en mogelijk zelfs gevaarlijk voor je gezondheid. De macht lag
toen bij barbaren, en die hielden vol dat Rome zoals men dat kende, nog
altijd bestond.

Het probleem was echter niet alleen dat de autoriteiten zeiden:
"Rapporteer dit niet of we zullen je doden." Wat nog meer meespeelde,
was dat Rome tegen het einde van de vijfde eeuw al zo ver was
afgetakeld, dat de ``val'' voor de meeste mensen die het meemaakten
nauwelijks als zodanig werd opgemerkt. Pas een generatie later
suggereerde Graaf Marcellinus in feite voor het eerst dat "Het
West-Romeinse Rijk ten onder ging met Augustulus." Het duurde nog
decennia, misschien zelfs eeuwen, voordat het algemeen erkend werd dat
het Romeinse Rijk in het Westen niet langer bestond. Karel de Grote
geloofde in het jaar 800 zonder twijfel dat hij een legitieme Romeinse
keizer was.

Het punt is niet dat Karel de Grote en allen die in conventionele termen
dachten over het Romeinse Rijk na 476 idioten waren, integendeel.
Maatschappelijke veranderingen worden vaak vaag of dubbelzinnig
voorgesteld. Als invloedrijke instellingen die vaagheid gebruiken om een
voor hen gunstige conclusie te versterken, zelfs als die grotendeels op
schijn berust, zal alleen iemand met een sterk karakter en uitgesproken
overtuigingen het aandurven die tegen te spreken. Wie zich inleeft in
het leven van een Romein in de vijfde eeuw begrijpt hoe makkelijk het
was om te geloven dat alles nog steeds hetzelfde was gebleven. Die
zekerheid was de optimistische conclusie. Iets anders denken had
beangstigend kunnen zijn, en waarom tot een beangstigende conclusie
komen, als er ook een geruststellende voorhanden was?

Er waren uiteindelijk ook goede redenen om te denken dat het leven
gewoon door zou gaan zoals altijd. Dat deed het in het verleden ook. De
Romeinse legers, vooral aan de grenzen van het rijk, werden al
eeuwenlang geïnfiltreerd door barbaren. Tegen de derde eeuw was het voor
het leger bijna een gewoonte geworden om regelmatig een nieuwe keizer
uit te roepen. Tegen de vierde eeuw waren zelfs officieren
gegermaniseerd en vaak analfabeet. Voor de val van Romulus Augustulus
werden al talrijke keizers op brute wijze van de troon gestoten. Zijn
vertrek zou voor zijn tijdgenoten niet anders hebben geleken dan de vele
andere omwentelingen in deze chaotische periode. Hij werd zelfs
weggestuurd met een pensioen, al was het slechts een korte periode
voordat hij werd vermoord. Dit bewees dat het systeem nog overeind
stond. De optimist vond juist dat Odoacer, die Romulus Augustulus
afzette, het rijk herenigde, en niet vernietigde. Odoacer, een zoon van
Attila\textquotesingle s rechterhand Edecon, was een intelligente man.
Hij riep zichzelf niet uit tot keizerd. In plaats daarvan riep hij de
Senaat bijeen en haalde zijn al te beïnvloedbare leden over dat zij het
keizerschap en dus de soevereiniteit over het hele rijk aan Zeno, de
keizer in het verre Byzantium, zouden aanbieden. Odoacer was slechts
Zeno's patricius, aangesteld om Italië te leiden.

Zoals Will Durant schreef in \emph{The Story of Civilization}, leken
deze veranderingen niet op de "val van Rome" maar slechts
"verwaarloosbare verschuivingen op het nationale toneel." Toen Rome
viel, zei Odoacer dat Rome bleef bestaan. Net als bijna iedereen deed
hij maar al te graag alsof alles hetzelfde bleef. Ze wisten dat "de oude
glorie van Rome" veel beter was dan de barbarij die haar verving. Zelfs
de barbaren dachten dat. Zoals C. W. Previte-Orton schreef in \emph{The
Shorter Cambridge Medieval History}, was het eind van de vijfde eeuw,
toen ``de keizers waren vervangen door barbaarse Germaanse koningen,''
een ``aanhoudende schijnvertoning.''

\textbf{"Hardnekkige schijnvertoning"}

Deze ``schijnvertoning'' hield de façade van het oude systeem overeind,
ook al was de kern ervan al ``aangetast door barbarij.'' De oude
regeringsvormen bleven hetzelfde toen de laatste keizer werd vervangen
door een barbaarse "luitenant". De Senaat kwam nog steeds bijeen. "De
pretoriaanse prefectuur en andere hoge ambten gingen door, en werden
gehouden door vooraanstaande Romeinen." Consuls werden nog steeds voor
een jaar aangesteld. "De Romeinse burgerlijke administratie bleef
onaangetast." Sterker nog, op sommige vlakken bleef het intact tot de
opkomst van het feodalisme aan het einde van de tiende eeuw. Bij
openbare gelegenheden werd nog steeds gebruik gemaakt van de oude
keizerlijke insignes. Het christendom was nog steeds de
staatsgodsdienst. De barbaren deden nog steeds alsof ze trouw
verschuldigd waren aan de Oostelijke keizer in Constantinopel, en aan de
tradities van het Romeinse recht. Maar, in Durant\textquotesingle s
woorden, "in het Westen was het grote Rijk er niet meer."

\textbf{Nou en?}

De val van Rome lijkt misschien iets van lang geleden, maar is
verrassend relevant als je kijkt naar hoe de wereld er nu voorstaat. De
meeste boeken over de toekomst zijn eigenlijk boeken over het heden. We
hebben geprobeerd dat gebrek te verhelpen door van dit boek over de
toekomst allereerst een boek over het verleden te maken. Wij denken dat
je een beter perspectief zult krijgen op wat de toekomst brengt als we
belangrijke megapolitieke punten over de logica van geweld illustreren
met echte voorbeelden uit het verleden. De geschiedenis is een geweldige
leermeester. De verhalen die het te vertellen heeft zijn interessanter
dan wat wij kunnen verzinnen, en veel van de meest interessante verhalen
gaan over de val van Rome. Ze bevatten belangrijke lessen die relevant
kunnen zijn voor jouw toekomst in het informatietijdperk.

Een van de best beschreven voorbeelden van een grote overgang in de
geschiedenis, waarbij de overheid implodeerde, is de val van Rome. De
overgang rond het jaar 1000 ging ook gepaard met de ineenstorting van
centrale autoriteit, en leidde tot een toename in de complexiteit en
omvang van economische activiteiten. De Buskruitrevolutie aan het eind
van de vijftiende eeuw bracht grote institutionele veranderingen met
zich mee, die juist de schaal van bestuur vergrootten in plaats van
verkleinden. Vandaag de dag, voor het eerst in duizend jaar, ondermijnen
en vernietigen de megapolitieke omstandigheden in het Westen overheden
en vele andere grootschalige instituties.

Uiteraard verschilden de oorzaken van de bestuurlijke implosie aan het
einde van het Romeinse Rijk aanzienlijk van de factoren die een rol
spelen bij de opkomst van het Informatietijdperk. Een van de redenen is
dat het Romeinse imperium simpelweg te grootschalig was. Het werd
onmogelijk om de economie van geweld te handhaven. De kosten om de ver
uitgespreide grenzen van het rijk te verdedigen, overtroffen de
economische voordelen die een oude agrarische economie kon opbrengen. De
last van belasting en regulering die nodig waren om de militaire
inspanning te financieren, steeg tot boven de draagkracht van de
economie. Corruptie werd endemisch. Militaire commandanten, zoals
historicus Ramsay MacMullen heeft gedocumenteerd, spendeerden een groot
deel van hun tijd aan het misbruiken van hun positie, voor "illegale
winsten". Dit deden ze door de bevolking af te persen, wat de
vierde-eeuwse waarnemer Synesius beschreef als "de vredesoorlog, bijna
erger dan de barbarenoorlog, en voortkomend uit het gebrek aan
discipline van het leger en de hebzucht van de officieren."

Een andere belangrijke factor die bijdroeg aan de val van Rome was een
demografisch tekort veroorzaakt door de Antonijnse pest. De sterke krimp
van de Romeinse bevolking droeg op veel vlakken duidelijk bij aan
economische en militaire zwakte. Vandaag de dag is daar nog geen sprake
van, althans nog niet. Op langere termijn, misschien, zal de gesel van
nieuwe "plagen" de uitdagingen van technologische devolutie in het
nieuwe millennium verergeren. De ongekende toename van de menselijke
bevolking in de twintigste eeuw creëert een verleidelijk doelwit voor
snel muterende microparasieten. Angsten over het Ebola-virus, of iets
dergelijks, dat metropolitane bevolkingen binnenvalt, kunnen gegrond
zijn. Maar dit is niet de plaats om de co-evolutie van mensen en ziekten
te overwegen. Hoe interessant dat onderwerp ook is, ons argument op dit
punt gaat niet over waarom Rome viel, of over de vraag of de wereld op
dit moment kwetsbaar is voor sommige van dezelfde invloeden die
bijdroegen aan de Romeinse achteruitgang. Het gaat over iets anders -
namelijk de manier waarop de grote transformaties van de geschiedenis
worden waargenomen, of liever, verkeerd waargenomen terwijl ze gebeuren.

Mensen zijn altijd en overal tot op zekere hoogte conservatief, met een
kleine "c." Men is terughoudend wanneer het gaat over het loslaten van
traditionele sociale conventies, het ondermijnen van erkende instituties
en het ter discussie stellen van de wetten en waarden waarop zij
gefundeerd waren. Maar weinig mensen kunnen zich voorstellen dat
schijnbaar kleine veranderingen in klimaat, technologie of een andere
variabele op de een of andere manier verantwoordelijk zouden kunnen zijn
voor het doorbreken van verbindingen met de wereld waarin hun ouders
zijn opgegroeid. De Romeinen waren terughoudend om de veranderingen die
zich om hen heen ontvouwden te erkennen. En dat zijn wij ook.

Toch, of je het nu erkent of niet, we gaan door een verandering van
historisch seizoen, een transformatie in de manier waarop mensen hun
levensonderhoud organiseren en zichzelf verdedigen, die zo diepgaand is
dat het onvermijdelijk de hele samenleving zal transformeren. Het zal
zelfs zo diepgaand zijn dat het, om het goed te begrijpen, noodzakelijk
is om bijna niets voor vanzelfsprekend te beschouwen. Steeds opnieuw zul
je de neiging hebben om te geloven dat de komende Informatie
Samenlevingen grote gelijkenissen zullen hebben met de industriële
samenlevingen waar je in opgroeide. Wij betwijfelen dat. Microprocessing
zal de mortel in de bakstenen oplossen. Het zal de logica van geweld zo
diepgaand veranderen dat het dramatisch de manier zal veranderen waarop
mensen hun levensonderhoud organiseren en zichzelf verdedigen. Toch zal
men geneigd zijn om de onvermijdelijkheid van deze veranderingen te
bagatelliseren, of om te discussiëren over hun wenselijkheid alsof
industriële instellingen per decreet zouden kunnen bepalen hoe de
geschiedenis evolueert.

\textbf{De grote illusie}

Auteurs die op vele manieren beter geïnformeerd zijn dan wij, zullen je
desondanks op het verkeerde been zetten als het gaat over de toekomst,
omdat zij de werking van samenlevingen slechts oppervlakkig analyseren.
David Kline en Daniel Burstein hebben bijvoorbeeld een goed onderzocht
boek geschreven, getiteld \emph{Road Warriors: Dreams and Nightmares
Along the Information Highway}. Het zit vol bewonderenswaardige details,
maar veel van deze details worden aangedragen om een illusie te
beargumenteren, namelijk het idee `dat burgers gezamenlijk en bewust
kunnen ingrijpen om de spontane economische en natuurlijke processen om
hen heen vorm te geven.' Hoewel het misschien niet voor de hand liggend
is, is dit met de stelling dat het feodalisme zou hebben overleefd als
iedereen zich opnieuw had toegewijd aan de ridderlijkheid. Niemand in
een hof van de late vijftiende eeuw zou bezwaar hebben gemaakt tegen
zo\textquotesingle n sentiment. Sterker nog, het zou ketterij zijn
geweest om dat te doen. Maar het zou ook volledig misleidend zijn
geweest, een voorbeeld van de slang die de toekomst in zijn oude huid
probeert te wringen.

De fundamentele oorzaken van verandering zijn juist niet onderworpen aan
bewuste controle. Het zijn de factoren die de omstandigheden veranderen
waaronder geweld loont. Ze staan zelfs zo ver af van elke vorm van
bewuste manipulatie, dat ze in een wereld die doordrenkt is van
politiek, geen onderwerp van politiek gemanoeuvreer vormen. Niemand
heeft ooit in een demonstratie geroepen: "Verhoog schaalvoordelen in het
productieproces." Geen spandoek heeft ooit geëist: "Vind een
wapensysteem uit dat het belang van de infanterie verhoogt." Geen
kandidaat heeft ooit beloofd om "de balans tussen efficiëntie en
schaalgrootte van de bescherming tegen geweld te veranderen." Zulke
slogans zouden belachelijk zijn, juist omdat deze doelen buiten ieders
vermogen liggen om ze bewust te beïnvloeden. Toch, zoals we zullen zien,
bepalen deze variabelen in veel grotere mate hoe de wereld werkt dan
welk politiek platform dan ook.

Als je er goed over nadenkt, wordt al snel duidelijk dat belangrijke
historische keerpunten zelden primair worden gedreven door menselijke
verlangens. Ze gebeuren niet omdat mensen het zat worden van één manier
van leven en plotseling een andere prefereren. Als je er even bij
stilstaat, zul je begrijpen waarom. Als wat mensen denken en verlangen
als enige zouden bepalen wat er gebeurt, dan zouden alle abrupte
veranderingen in de geschiedenis worden verklaard door wilde
stemmingswisselingen, los van enige verandering in de feitelijke
levensomstandigheden. In feite gebeurt dit nooit. Alleen in gevallen van
medische problemen, die een paar mensen treffen, zien we willekeurige
fluctuaties in stemming die volledig los lijken te staan van een
objectieve oorzaak.

Over het algemeen besluiten grote aantallen mensen niet plotseling en
allemaal tegelijk hun manier van leven op te geven, simpelweg omdat ze
dat grappig vinden. Geen enkele jager-verzamelaar heeft ooit gezegd: "Ik
ben het zat om in prehistorische tijden te leven; ik leef liever als een
boer in een boerendorp." Elke beslissende omslag in gedragspatronen en
waarden is zonder uitzondering een reactie op een werkelijke verandering
in de omstandigheden van het leven. In die zin, althans, zijn mensen
altijd realistisch. Als hun opvattingen abrupt veranderen, wijst dat
waarschijnlijk erop dat ze zijn geconfronteerd met een afwijking van de
vertrouwde omstandigheden: een invasie, een plaag, een plotselinge
klimatologische verschuiving, of een technologische revolutie die hun
levensonderhoud of hun vermogen om zichzelf te verdedigen verandert.

Grote veranderingen in de geschiedenis zijn meestal niet wat mensen
willen. Ze verstoren juist de rust en stabiliteit waar de meeste mensen
naar verlangen. Wanneer verandering optreedt, veroorzaakt het typisch
wijdverspreide desoriëntatie, vooral onder degenen die hun inkomen of
sociale status verliezen. Je zult tevergeefs kijken naar opiniepeilingen
of stemmingsbarometers om een begrip te krijgen van hoe de komende
megapolitieke overgang zich waarschijnlijk zal ontvouwen.

\textbf{Leven zonder vooruitziende blik}

Als we er niet in slagen om de grote overgang, die om ons heen gaande
is, waar te nemen, komt dat deels doordat we die niet willen zien. Onze
jagende voorouders waren misschien net zo koppig, maar zij hadden een
beter excuus. Niemand had tienduizend jaar geleden de gevolgen van de
Agrarische Revolutie kunnen voorzien. Sterker nog, men voorzag toen
überhaupt niet veel meer dan waar ze de volgende maaltijd zouden kunnen
vinden. Toen men aan landbouw begon, bestond er geen verslaglegging van
gebeurtenissen uit het verleden, om daaruit lessen te trekken voor de
toekomst. De Westerse tijdsindeling, zoals seconden, minuten, uren,
dagen, enzovoort, om de jaren af te meten, bestond nog niet eens.
Jager-verzamelaars leefden in het "eeuwige heden," zonder kalenders, en
inderdaad, zonder enige geschreven verslagen. Ze kenden geen wetenschap,
en beschikten over geen enkel ander intellectueel instrument om oorzaak
en gevolg te begrijpen dan hun eigen intuïtie. Wat betreft
vooruitkijken, waren onze primitieve voorouders blind. Om de bijbelse
metafoor te citeren, ze hadden nog niet van de vrucht der kennis
gegeten.

\textbf{Leren van het verleden}

Gelukkig staan wij er beter voor. De afgelopen vijfhonderd generaties
hebben ons een analytisch vermogen geschonken dat onze voorouders
misten. Wetenschap en wiskunde hebben geholpen vele geheimen van de
natuur te openbaren, waardoor we een begrip van oorzaak en gevolg hebben
dat grenst aan het magische vergeleken met dat van de vroege
jager-verzamelaars. Computationele algoritmen, ontstaan dankzij de komst
van snelle computers, hebben nieuwe inzichten opgeleverd in de werking
van complexe, dynamische systemen zoals de menselijke economie. De
zorgvuldige ontwikkeling van politieke economie zelf, hoewel het verre
van perfect is, heeft ons begrip van de factoren die menselijk handelen
beïnvloeden, vergroot. Een belangrijk inzicht daaruit is dat mensen
altijd en overal geneigd zijn op prikkels te reageren, niet altijd zo
mechanisch als economen zich voorstellen, maar ze reageren wel. Kosten
en baten doen ertoe. Veranderingen in externe omstandigheden die de
baten verhogen van bepaald gedrag of de kosten ervan verlagen, zullen
leiden tot meer van dat gedrag, als de rest onveranderd blijft.

\textbf{Prikkels doen ertoe}

Het feit dat mensen de neiging hebben om te reageren op kosten en baten
is een essentieel element bij het doen van voorspellingen. Je kunt met
grote zekerheid zeggen dat als je een biljet van honderd dollar op
straat laat vallen, iemand het snel zal oprapen, of je nu in New York,
Mexico-Stad of Moskou bent. Dit is niet zo triviaal als het lijkt. Het
toont aan waarom de slimme mensen die zeggen dat voorspellen onmogelijk
is, ongelijk hebben. Elke voorspelling die accuraat de impact van
prikkels op gedrag anticipeert zal het waarschijnlijk grotendeels bij
het juiste eind hebben. Hoe groter de geanticipeerde verandering in
kosten en baten, hoe minder voor de hand liggend de voorspelling
waarschijnlijk zal zijn.

De meest verstrekkende voorspellingen van allemaal zullen waarschijnlijk
voortkomen uit het herkennen van de implicaties van verschuivende
megapolitieke variabelen. Geweld is de ultieme grens die gedrag bepaalt;
dus, als je kunt begrijpen hoe de logica van geweld zal veranderen, kun
je met grote nauwkeurigheid voorspellen waar mensen het equivalent van
honderd-dollarbiljetten in de toekomst zullen laten vallen of oprapen.

We bedoelen hiermee niet dat je het onkenbare kunt weten. Wij kunnen je
niet vertellen hoe je winnende loterijnummers kunt voorspellen, of welke
willekeurige gebeurtenis dan ook. We hebben geen manier om te weten of
of wanneer een terrorist een atoomexplosie in Manhattan zal doen
ontploffen, of als een asteroïde Saoedi-Arabië zal treffen. We kunnen de
komst van een nieuwe ijstijd, een plotselinge vulkaanuitbarsting, of de
opkomst van een nieuwe ziekte niet voorspellen. Het aantal onkenbare
gebeurtenissen die de koers van de geschiedenis zouden kunnen veranderen
is groot. Maar het onkenbare weten is heel anders dan de gevolgen
overzien van wat al bekend is. Als u een bliksemflits ver weg ziet, kunt
u met een grote zekerheid voorspellen dat een donderslag zal volgen. Het
voorspellen van de gevolgen van megapolitieke transities omvat veel
langere tijdsbestekken, en minder zekere verbanden, maar het idee is
hetzelfde.

Megapolitieke katalysatoren voor verandering verschijnen gewoonlijk ruim
voordat hun gevolgen zich manifesteren. Het duurde vijfduizend jaar
voordat de volledige implicaties van de Agrarische Revolutie aan het
licht kwamen. De overgang van een agrarische samenleving naar een
industriële samenleving gebaseerd op fabricage en chemische kracht
ontvouwde zich sneller. Het duurde eeuwen. De overgang naar de
Informatiemaatschappij zal nog sneller gebeuren, waarschijnlijk binnen
een leven. Toch, zelfs rekening houdend met de versnelling van de
geschiedenis, kunt je verwachten dat decennia zullen verstrijken voordat
de volledige megapolitieke impact van de huidige informatietechnologie
wordt gerealiseerd.

\textbf{Grote en kleine megapolitieke overgangen}

Dit hoofdstuk analyseert enkele gemeenschappelijke kenmerken van
megapolitieke overgangen. In volgende hoofdstukken kijken we meer
nauwkeurig naar de Agrarische Revolutie, en de overgang van boerderij
naar fabriek, de tweede van de grote faseveranderingen. Binnen het
agrarische stadium van de beschaving waren er vele kleine megapolitieke
overgangen zoals de val van Rome en de feodale revolutie van het jaar
1000. Deze markeerden de verschuivingen in de machtsbalans: overheden
kwamen en gingen, en de opbrengsten van de landbouw gingen van de ene
groep over naar de andere. De eigenaren van uitgestrekte landgoederen
ten tijden van het Romeinse Rijk, vrije boeren in de Europese vroege
middeleeuwen, en de heren en lijfeigenen van de feodale periode aten
allemaal graan van dezelfde akkers. Ze leefden onder zeer verschillende
overheden dankzij de cumulatieve impact van verschillende technologieën,
fluctuaties in het klimaat, en de ontwrichtende invloed van ziektes.

We beweren niet al deze veranderingen grondig te verklaren, hoewel we
een beeld hebben geschetst van de manier waarop veranderende
megapolitieke variabelen de machtsuitoefening in het verleden hebben
beïnvloed. Overheden zijn gegroeid en gekrompen naarmate megapolitieke
schommelingen de kosten van machtsprojectie verlaagden en verhoogden.

Hier zijn enkele samenvattende punten die je in gedachten zou moeten
houden wanneer je probeert de Informatierevolutie te begrijpen:

\begin{quote}
1. ~ Een verschuiving in de megapolitieke machtsfundamenten voltrekt
zich doorgaans ver ver vóór de daadwerkelijke revoluties in de
machtsuitoefening.

2.~ Inkomens dalen meestal wanneer een grote overgang begint, vaak omdat
een samenleving door bevolkingsdruk kwetsbaar is geworden en hulpbronnen
heeft gemarginaliseerd.

3.~ "Buiten" een systeem zien is gewoonlijk taboe. Mensen zijn vaak
blind voor de logica van geweld in de bestaande samenleving en daardoor
vrijwel altijd ook voor veranderingen in die logica, of die nu verborgen
of openlijk zijn. Megapolitieke transities worden zelden herkend voordat
ze zich voordoen.

4.~ Grote overgangen gaan altijd gepaard met een culturele revolutie, en
leiden gewoonlijk tot botsingen tussen aanhangers van de oude en de
nieuwe waarden.

5. ~ Megapolitieke overgangen zijn nooit populair, omdat ze moeizaam
verworven intellectueel kapitaal achterhaal maken en gevestigde morele
voorschriften ondermijnen. Ze worden niet op algemeen verzoek
uitgevoerd, maar als reactie op veranderende externe omstandigheden die
de logica van geweld in de lokale context veranderen.

6.~ Transities naar nieuwe manieren om het levensonderhoud te
organiseren of naar nieuwe typen overheden, beperken zich aanvankelijk
tot de gebieden waar de megapolitieke katalysatoren werkzaam zijn.

7. ~ Met de mogelijke uitzondering van de vroege stadia van landbouw,
hebben eerdere overgangen altijd perioden van sociale chaos en verhoogd
geweld betrokken vanwege desoriëntatie en de ineenstorting van het oude
systeem.

8.~ Corruptie, moreel verval, en inefficiëntie lijken signaalkenmeren te
zijn van de laatste stadia van een systeem.

9.~ Het groeiende belang van technologie bij het vormgeven van de logica
van geweld heeft geleid tot een versnelling van de geschiedenis,
waardoor er voor elke opeenvolgende transitie minder tijd voor
aanpassing overblijft dan ooit tevoren.
\end{quote}

\textbf{De geschiedenis versnelt}

Nu gebeurtenissen zich vele malen sneller ontvouwen dan tijdens
voorgaande transformaties, zou een vroegtijdig inzicht in hoe de wereld
zal veranderen veel meer voor je kunnen betekenen dan voor je voorouders
op een vergelijkbaar keerpunt in het verleden. Zelfs als de eerste
boeren op miraculeuze wijze de volledige megapolitieke implicaties van
het bewerken van de aarde hadden begrepen, zou deze informatie praktisch
nutteloos zijn geweest omdat duizenden jaren zouden verstrijken voordat
de overgang naar de nieuwe fase van de samenleving voltooid was.

Vandaag de dag is dat anders. De geschiedenis is in een
stroomversnelling geraakt. Voorspellingen die de megapolitieke gevolgen
van nieuwe technologie correct inschatten, zijn vandaag de dag
waarschijnlijk veel nuttiger. Het kunnen doorgronden van de implicaties
van de huidige overgang naar de informatiemaatschappij is vele malen
waardevoller dan dat het volledige begrip van de gevolgen van de
transitie van landbouw naar industrie toen was. Simpel gezegd, de
actie-horizon voor megapolitieke voorspellingen is ingekort tot de
nuttigste tijdsduur: de duur van een mensenleven.

\emph{"Terugkijkend over de eeuwen, of zelfs alleen kijkend naar het
heden, kunnen we duidelijk waarnemen dat velen hun inkomen hebben
gekregen, vaak een zeer goed inkomen, door middel van hun speciale
vaardigheid in het toepassen van wapens van geweld, en dat hun
handelingen in grote mate bepaalden hoe schaarse middelen werden
ingezet."} - FREDERIC C. LANE

Onze studie van de megapolitiek is een poging om precies dat te doen:
het in kaart brengen van de gevolgen van de factoren die de grenzen
verschuiven waarbinnen geweld wordt uitgeoefend.

Deze megapolitieke factoren bepalen grotendeels wanneer en waar geweld
loont. Ze beïnvloeden ook de verdeling van het inkomen. Zoals economisch
historicus Frederic Lane zo duidelijk verwoordde, speelt de manier
waarop geweld wordt georganiseerd en beheerst een grote rol bij het
bepalen van "\emph{hoe schaarse middelen worden ingezet.}"

\textbf{Een spoedcursus megapolitiek}

Het concept megapolitiek is een krachtig idee. Het helpt enkele van de
grote mysteries van de geschiedenis te verklaren, zoals hoe regeringen
opkomen en vallen, wat voor soorten instellingen ze worden, de timing en
uitkomst van oorlogen, en patronen van economische welvaart en
achteruitgang. Door de kosten en baten van het projecteren van macht te
verhogen of te verlagen, regeert megapolitiek het vermogen van mensen om
hun wil aan anderen op te leggen. Sinds de vroegste menselijke
samenlevingen is dit het geval geweest, en dat is het nog steeds. In
\emph{Blood in the Streets} en \emph{The Great Reckoning} onderzochten
we vele van de belangrijke verborgen megapolitieke factoren die de
evolutie van de geschiedenis bepalen. Om de gevolgen van megapolitieke
verandering te begrijpen, moet je inzicht krijgen in de factoren die
revoluties in het gebruik van geweld realiseren. Deze variabelen kunnen
enigszins willekeurig worden gegroepeerd in vier categorieën:
topografie, klimaat, microben, en technologie.

\begin{quote}
1. ~ \textbf{Topografie} is een cruciale factor, zoals bewezen door het
feit dat, in tegenstelling tot op land, de controle van geweld op open
zee nooit is gemonopoliseerd. Geen enkele overheid is er ooit in
geslaagd om er enkel haar wetten te laten gelden. Om te begrijpen hoe de
organisatie van geweld en bescherming zal evolueren naarmate de economie
migreert naar cyberspace, is dit van belang.
\end{quote}

Topografie, in combinatie met klimaat, speelde een grote rol in de
vroege geschiedenis. De eerste staten kwamen op in overstromingsvlaktes,
omringd door woestijn, zoals in Mesopotamië en Egypte, waar water voor
irrigatie overvloedig was maar omliggende regio\textquotesingle s te
droog waren om kleinschalige landbouw mogelijk te maken. In deze context
betaalden individuele landbouwers een hoge prijs als zij weigerden samen
te werken aan het behoud van het politieke systeem. Zonder irrigatie,
die alleen op grote schaal kon worden verschaft, zouden gewassen niet
groeien. Geen gewassen betekende verhongering. Wie het water in de
woestijn beheerste, bezat de macht --- en dat leidde tot een despotische
en steenrijke overheid.

Zoals we analyseerden in The Great Reckoning, speelden topografische
omstandigheden ook een grote rol in de welvaart van vrije boeren in het
oude Griekenland, waardoor het zich kon ontwikkelen tot de wieg van de
Westerse democratie. Gegeven de primitieve transportmogelijkheden die
drieduizend jaar geleden in de Middellandse Zee-regio gebruikelijk
waren, was het vrijwel onmogelijk voor personen, die meer dan een paar
kilometer van de zee woonden, om te concurreren in de productie van
waardevolle gewassen: olijven en druiven. Wanneer olie en wijn over land
moesten vervoerd worden, waren de transportkosten zo hoog dat verkoop
met winst onmogelijk was. Door de grillige kustlijn van Griekenland lag
het grootste deel van het land op niet meer dan twintig mijl van de zee.
Dit gaf Griekse boeren een beslissend voordeel ten opzichte van hun
potentiële concurrenten in landinwaarts gelegen gebieden.

Door dit handelsvoordeel verdienden Griekse boeren hoge inkomens met de
controle van slechts kleine stukken grond. Deze hoge inkomens stelden
hen in staat om kostbare wapenuitrusting te kopen. De beroemde hoplieten
van het oude Griekenland waren boeren of landheren die zichzelf op eigen
kosten bewapenden. De Griekse hoplieten waren zowel goed bewapend als
sterk gemotiveerd, en vormden een militaire macht die men niet kon
negeren. De Griekse democratie ontstond uit de topografische
omstandigheden, net zoals andere soort topografieën de despotische
systemen van Egypte en elders mogelijk maakten.

\begin{quote}
2.~ \textbf{Klimaat} bepaalt ook mede de grenzen waarbinnen brute kracht
kan worden uitgeoefend. Een verandering van klimaat was de katalysator
voor de eerste grote overgang van foerageren naar landbouw.
\end{quote}

Het einde van de laatste ijstijd, ongeveer dertienduizend jaar geleden,
leidde tot een radicale verandering van de vegetatie. In het Nabije
Oosten, waar de ijstijd als eerste tot een einde kwam, begonnen bossen
zich door een geleidelijke stijging in temperatuur en regenval te
verspreiden naar gebieden die voorheen bestonden uit grasland. De snelle
verspreiding van beukenbossen in het bijzonder beperkte het menselijke
dieet ernstig. Zoals Susan Alling Gregg het verwoordde in \emph{Foragers
and Farmers}:

\emph{"De ontwikkeling van beukenbossen moet ernstige gevolgen hebben
gehad voor lokale mensen-, planten- en dierenpopulaties. De kroonlaag
van een eikenbos is relatief open en laat grote hoeveelheden zonlicht de
bosbodem bereiken. Daardoor ontstaat een wilde onderbegroeiing van
gemengde struiken, kruiden en grassen, en deze diversiteit aan planten
ondersteunt op zijn beurt een verscheidenheid aan wild. Daarentegen is
de kroonlaag van een beukenbos gesloten en is de bosbodem zwaar
beschaduwd. Buiten een korte bloeiperiode van lentebloemen voorafgaand
aan het uitlopen van de bladeren, groeien er alleen soorten die goed
tegen schaduw kunnen, zoals zeggen, varens en enkele grassoorten."}

Na verloop van tijd drongen dichte bossen de open vlaktes binnen en
verspreidden zich door Europa richting de oostelijke steppen. De bossen
verdrongen de graslanden die grote dieren onderhielden, waardoor het
voor de menselijke jagers steeds moeilijker werd om zichzelf te
onderhouden.

De populatie van jager-verzamelaars was te sterk gegroeid tijdens de
welvarende periode gedurende de ijstijd, waardoor ze zichzelf niet meer
kon ondersteunen met de krimpende kuddes grote zoogdieren, waarvan vele
soorten tot uitsterven werden gejaagd. De overgang naar landbouw was
niet hun voorkeur, maar eerder een gedwongen improvisatie om tekorten in
het dieet aan te vullen. Jagen bleef dominant in de meer noordelijke
gebieden, waar de verwarmingstrend de habitats van grote zoogdieren niet
nadelig had beïnvloed, en in tropische regenwouden, waar de wereldwijde
opwarming niet leidde tot het verminderen van de voedselvoorzieningen.
Sinds de opkomst van landbouw zijn veranderingen veel vaker veroorzaakt
door afkoeling dan door opwarming van het klimaat.

Als het klimaat zal blijven fluctueren in de toekomst, zal het nuttig
zijn om een bescheiden begrip te hebben van de rol die
klimaatverandering speelde in vroegere maatschappijen. Het is
bijvoorbeeld zo dat een daling van één graad Celsius gemiddeld het
groeiseizoen met drie tot vier weken vermindert en honderdvijftig meter
afhaalt van de maximale hoogte waarop gewassen kunnen worden verbouwd.
Dit zegt iets over de grenzen waarbinnen mensen in de toekomst zullen
moeten handelen. Je kunt deze kennis gebruiken om veranderingen in
alles, van graanprijzen tot de prijzen van landbouwgrond, te
voorspellen. Je zou zelfs de waarschijnlijke impact van vallende
temperaturen op reële inkomens en politieke stabiliteit kunnen
inschatten. In het verleden zijn overheden omvergeworpen toen, door
meerdere opeenvolgende mislukte oogsten, de voedselprijzen verhoogden en
de koopkracht opdroogde.

Het is bijvoorbeeld geen toeval dat de zeventiende eeuw, de koudste in
de moderne periode, ook een periode van revolutie wereldwijd was. Een
verborgen megapolitieke oorzaak van dit ongenoegen was scherp kouder
weer. Het was zelfs zo koud dat wijn op de tafel van de "Zonnekoning" in
Versailles bevroor. Verkorte groeiseizoenen veroorzaakten mislukte
oogsten en ondermijnden reële inkomens. Door het koudere weer begon de
welvaart af te nemen en gleed de wereld rond 1620 een lange wereldwijde
depressie in. Die bleek buitengewoon ontwrichtend. De economische crisis
van de zeventiende eeuw leidde tot een golf van opstanden wereldwijd,
met een piek in het jaar 1648 --- precies tweehonderd jaar vóór een
andere, bekendere cyclus van revoluties. Tussen 1640 en 1650 waren er
opstanden in Ierland, Schotland, Engeland, Portugal, Catalonië,
Frankrijk, Moskou, Napels, Sicilië, Brazilië, Bohemen, Oekraïne,
Oostenrijk, Polen, Zweden, Nederland, en Turkije. Zelfs China en Japan
werden overspoeld met onrust.

Het is wellicht ook geen toeval dat het mercantilisme overheerste in de
zeventiende eeuw, tijdens een periode van krimpende handel. Economisch
isolationisme was misschien het meest uitgesproken aan het einde van de
eeuw, "toen een verschrikkelijke hongersnood plaatsvond." In de
achttiende eeuw, vooral na 1750, hadden warmere temperaturen en betere
oogsten de inkomens in West-Europa voldoende verhoogd, waardoor de vraag
naar productiegoederen op gang kwam. Naarmate het vrijemarktbeleid
breder werd ingevoerd, ontstond een zichzelf versterkende impuls van
economische groei. De industrie breidde zich uit tot een ongekende
schaal --- het begin van wat we de Industriële Revolutie zijn gaan
noemen. Technologie en productie werden steeds belangrijker en
verminderden de invloed van het weer op de economische cycli.

Je moet de impact van een plotse verlaging van de temperatuur echter
niet onderschatten. Zelfs vandaag de dag kan het de koopkracht flink
onder druk zetten, ook in welvarende gebieden zoals Noord-Amerika.
Maatschappijen hebben een sterke tendens om zichzelf crisis-gevoelig te
maken wanneer de bestaande instellingen hun beste tijd hebben gehad. In
het verleden manifesteerde deze tendens zich vaak in een bevolkingsgroei
die de grenzen van wat het land kon dragen, opzocht. Dit gebeurde zowel
voor de overgang van het jaar 1000 als aan het einde van de vijftiende
eeuw. De flinke daling van de koopkracht, veroorzaakt door mislukte
oogsten en lagere opbrengsten, speelde in beide gevallen een
significante rol in het vernietigen van de overheersende instellingen.
Op dit moment zijn die problemen vooral zichtbaar in de
consumentenkredietmarkten. Toen plotselinge temperatuurdalingen de
opbrengsten van oogsten verminderden en het beschikbaar inkomen afnam,
leidde dat tot wanbetalingen van schulden en opstanden tegen de
belastingheffing. Het verleden leert ons dat deze omstandigheden ook in
de toekomst zouden kunnen resulteren in zowel economische afsluiting als
politieke onrust.

\begin{quote}
3.~ \textbf{Microben} hebben door hun invloed op ziekten en immuniteit
vaak bepaald hoe macht kon worden ingezet. Dit was zeker het geval in de
Europese verovering van de Nieuwe Wereld, zoals we onderzochten in
\emph{The Great Reckoning}. Europese kolonisten, doordrenkt met ziektes
uit ontwikkelde agrarische samenlevingen, hadden al relatieve immuniteit
tegen kinderinfecties zoals de mazelen. De Indianen die ze ontmoetten
leefden grotendeels in dun bevolkte jager-verzamelaarsstammen, en hadden
nog geen immuniteit tegen Westerse ziekten ontwikkeld. Ze werden
daardoor gedecimeerd. Vaak stierven de meesten zelfs al voordat ze met
blanke mensen in aanraking kwamen, doordat geïnfecteerde Indianen die
aan de kust voor het eerst Europeanen ontmoetten naar het binnenland
reisden, en zo de ziektes verspreidden.
\end{quote}

Er zijn ook microbiologische grenzen voor de uitoefening van macht. In
\emph{Blood in the Streets} bespraken we hoe malaria het voor blanken
eeuwenlang onmogelijk maakte om een invasie in tropisch Afrika te
realiseren. Vóór de ontdekking van kinine in het midden van de
negentiende eeuw konden blanke legers niet overleven in
regio\textquotesingle s met malaria, hoe superieur hun wapens ook waren.

De interactie tussen mensen en microben heeft ook belangrijke
demografische effecten voortgebracht die de kosten en baten van geweld
hebben veranderd. Wanneer het aantal sterfgevallen sterk fluctueert
vanwege epidemische ziekten, hongersnood, of andere oorzaken, neemt de
kans om te sterven door oorlogsvoering af. Vanaf de zestiende eeuw
werden uitschieters van het sterftecijfer steeds minder frequent, wat
helpt verklaren waarom gezinnen kleiner werden en waarom er vandaag de
dag, vergeleken met vroeger, veel minder tolerantie is voor plotselinge
sterfgevallen in oorlog. Hierdoor is ook de tolerantie voor imperialisme
afgenomen en zijn de kosten om aan machtsprojectie te doen verhoogd in
gemeenschappen met een laag geboortecijfer.

Hedendaagse samenlevingen, bestaande uit kleine families, vinden zelfs
een klein aantal oorlogsdoden vaak onacceptabel. Vroege moderne
samenlevingen waren daarentegen veel toleranter wat betreft de
menselijke kosten die horen bij het imperialisme. Vóór de aanvang van
deze eeuw kregen de meeste ouders vele kinderen, waarvan al werd
verwacht dat enkelen plotseling zouden sterven door ziekte. In deze
periode, waarin jong overlijden doodnormaal was, waren aspirant-soldaten
en hun families veel meer bereid om de gevaren van het slagveld te
trotseren.

\emph{"Machines zijn agressief. De wever wordt een web, de machinist een
machine. Als je gereedschappen niet gebruikt, gebruiken zij jou."} -
Emerson

\begin{quote}
4.~ \textbf{Technologie} heeft tijdens de moderne eeuwen verreweg de
grootste rol gespeeld in het bepalen van de kosten en baten van het
projecteren van macht. Dit boek beargumenteert dat het dat zal blijven
doen. Technologie heeft verschillende cruciale dimensies:
\end{quote}

\textbf{A. Balans tussen offensief en defensief.} De balans tussen
aanval en verdediging, bepaald door de toegang tot wapentechnologie,
bepaalt de schaal van de politieke organisatie. Wanneer offensieve
capaciteiten stijgen, overheerst het vermogen om macht op afstand te
projecteren, jurisdicties hebben dan de neiging om te consolideren, en
overheden vormen zich op grotere schaal. In tijden zoals nu, nemen
defensieve mogelijkheden toe. Hierdoor wordt het duurder om macht buiten
de kerngebieden te projecteren. Rechtsgebieden hebben de neiging te
versnipperen en grote overheden vallen uiteen in kleinere.

\textbf{B. Gelijkheid en de overheersing van de infanterie.} Een
belangrijk kenmerk dat de mate van gelijkheid onder burgers bepaalt, is
de aard van wapentechnologie. Wapens die relatief goedkoop zijn, zorgen
dat de macht gelijker verdeeld wordt, omdat ze kunnen worden ingezet
door niet-professionals. Dit versterkt het militaire belang van
infanterie. Toen Thomas Jefferson zei dat "alle mensen als gelijken zijn
geschapen", klonk dat veel geloofwaardiger dan het eeuwen daarvoor zou
hebben gedaan. Een boer met zijn jachtgeweer was niet alleen even goed
bewapend als de typische Britse soldaat met zijn Brown Bess, hij was
zelfs beter bewapend. De boer met het geweer kon op de soldaat schieten
vanaf een grotere afstand, en met grotere nauwkeurigheid. Dit was een
duidelijk verschil met de Middeleeuwen, toen een boer met een hooivork -
meer kon hij zich niet veroorloven - nauwelijks had kunnen hopen stand
te houden tegen een zwaar bewapende ridder te paard. Niemand schreef in
1276 dat "alle mensen als gelijken zijn geschapen." Op dat moment, in de
meest fundamentele zin, waren mensen niet gelijk. Één enkele ridder kon
veel meer brute kracht uitoefenen dan tientallen boeren bij elkaar.

\textbf{C. Voor- en nadelen van schaal op vlak van geweld.} Een andere
variabele die helpt bepalen of er een paar grote overheden of vele
kleine zullen zijn, is de organisatorische schaal die vereist is om de
heersende wapens in te zetten. Als geweld in toenemende mate loont,
hebben overheden voordeel wanneer ze op grote schaal opereren, en hebben
ze de neiging groter te worden. Wanneer een kleine groep echter erin
slaagt hun middelen effectief genoeg in te zetten om de grotere macht te
kunnen weerstaan, heeft soevereiniteit de neiging om te fragmenteren,
wat het geval was tijdens de Middeleeuwen. Kleine, onafhankelijke
autoriteiten voerden veel van de functies van de overheid uit. Zoals we
in een later hoofdstuk zullen bespreken, geloven we dat het
Informatietijdperk de opkomst van cybersoldaten zal brengen, die de
voorbodes zullen zijn van de ontbinding van centrale macht.
Cybersoldaten zouden niet alleen door natiestaten kunnen worden ingezet,
maar ook door hele kleine organisaties, en zelfs door individuen.
Oorlogen van het volgende millennium zullen enkele ``veldslagen'' kennen
die volledig zonder bloedvergieten zullen verlopen, met computers.

\textbf{D. Schaalvoordelen van productie.} Een andere cruciale factor
bij het bepalen of de uiteindelijke machtsuitoefening lokaal of op
afstand plaatsvindt, is de schaal van de bedrijven waarin mensen hun
geld verdienen. Als belangrijke bedrijven alleen goed kunnen
functioneren op grote schaal, binnen een groot handelsgebied, dan kunnen
overheden die dat mogelijk maken voldoende bijkomende inkomsten afromen
om een grote overheid te onderhouden. Onder zulke omstandigheden
functioneert de hele wereldeconomie gewoonlijk effectiever als één
sterke wereldmacht alle anderen domineert, zoals het Britse Rijk dat
deed in de negentiende eeuw. Soms leiden de megapolitieke variabelen er
echter toe dat de schaal van economische handel in elkaar stort. Als het
onderhouden van een groot handelsgebied steeds minder economisch
voordeel oplevert, kunnen grotere overheden, die voorheen profiteerden
van de voordelen van grotere handelsgebieden, beginnen uiteen te vallen
-~ zelfs in situaties waarin het strategische evenwicht tussen
offensieve en defensieve middelen grotendeels constant blijft.

\textbf{E. Verspreiding van technologie.} Nog een factor die invloed
heeft op macht, is hoe wijdverspreid belangrijke technologieën zijn.
Wanneer het mensen lukt om wapens of productiemiddelen te accumuleren of
te monopoliseren, hebben ze de neiging om macht te centraliseren. Zelfs
technologieën met een overwegend defensief karakter, zoals het
machinegeweer, bleken tijdens de periode waarin ze niet wijd verspreid
waren, krachtige offensieve wapens te zijn die bijdroegen aan een
groeiende schaal van de overheid. Toen de Europese machten laat in de
negentiende eeuw een monopolie op machinegeweren hadden, waren ze in
staat om die wapens tegen volkeren aan de grenzen van hun territoria in
te zetten, en zo hun koloniale rijken enorm uit te breiden. Later, in de
twintigste eeuw, toen machinegeweren wijd beschikbaar werden, vooral in
de nasleep van Wereldoorlog II, werden ze juist ingezet om rijken te
vernietigen. Als alle andere omstandigheden gelijk blijven, geldt: hoe
breder sleuteltechnologieën verspreid zijn, des te breder de macht
verspreid zal zijn en des te kleiner de optimale schaal van de overheid.

\textbf{De snelheid van megapolitieke verandering}

Hoewel technologie vandaag verreweg de belangrijkste factor is, en
schijnbaar in toenemende mate, hebben de vier grote megapolitieke
factoren in het verleden elk een rol gespeeld in het bepalen van de
schaal waarop macht kon worden uitgeoefend.

Samen bepalen deze factoren of het loont om geweld op grotere schaal toe
te passen. Dit bepaalt hoe belangrijk vuurkracht is in verhouding tot
efficiënt gebruik van middelen. Het heeft ook een sterke invloed op de
verdeling van inkomen in de markt. De vraag is: welke rol zullen deze
factoren in de toekomst spelen? Om hier een antwoord op te geven is het
cruciaal om te weten dat deze megapolitieke variabelen zich met enorm
uiteenlopende snelheden ontwikkelen.

Topografie is vrijwel onveranderlijk geweest doorheen de hele
geschiedenis. Buiten kleine lokale effecten, zoals het dichtslibben van
havens, het opspuiten van land of erosie, is de topografie van de aarde
bijna hetzelfde vandaag als toen Adam en Eva uit Eden stapten. En het
zal waarschijnlijk zo blijven totdat een nieuwe ijstijd de landschappen
van de continenten opnieuw vormgeeft, of totdat een andere ingrijpende
gebeurtenis het aardoppervlak verstoort. Op een fundamenteler niveau
lijken geologische tijdperken te veranderen over periodes van 10 tot 40
miljoen jaar, wellicht als gevolg van grote meteorietinslagen. Ooit
kunnen er opnieuw geologische omwentelingen zijn die de topografie van
onze planeet aanzienlijk zullen veranderen. Als dat gebeurt, mag je er
wel van uit gaan dat zowel het honkbal- als het cricketseizoen zullen
worden geannuleerd.

Het klimaat verandert veel frequenter dan topografie. In het laatste
miljoen jaar heeft klimaatverandering voor het merendeel van de bekende
veranderingen in het aardoppervlak gezorgd. Tijdens ijstijden groeven
gletsjers nieuwe valleien, veranderden de loop van rivieren, scheidden
eilanden van continenten of voegden ze samen door het zeeniveau te
verlagen.

Fluctuaties in het klimaat hebben een significante rol gespeeld in de
geschiedenis. Eerst in het bewerkstelligen van de Agrarische Revolutie
na het einde van de laatste ijstijd, en later in het destabiliseren van
regimes tijdens perioden van koudere temperaturen en droogte.
Recentelijk zijn er zorgen geweest over de mogelijke impact van
"wereldwijde opwarming." Deze zorgen kunnen niet zonder meer worden
weggewuifd. Toch, vanuit een langer perspectief genomen, lijkt juist een
verschuiving naar een kouder, en niet een warmer klimaat, een groter
risico te zijn. Studies van temperatuurfluctuaties, gebaseerd op de
analyse van zuurstofisotopen in kernmonsters van de oceaanbodem, tonen
dat de huidige periode de tweede warmste is in meer dan 2 miljoen jaar.
Als het kouder zou worden, zoals dat in de zeventiende eeuw gebeurde,
zou dat een destabiliserende werking kunnen hebben op de megapolitiek.
In die zin is het huidige alarmisme over de opwarming van de aarde
wellicht geruststellend. Als deze waarschuwingen kloppen, betekent dat
dat de temperaturen binnen het abnormaal warme en relatief milde bereik
dat we de afgelopen drie eeuwen hebben meegemaakt, blijven fluctueren.

Het tempo van verandering door microben is echter een puzzel. Microben
kunnen zeer snel muteren. Dit is vooral waar voor virussen. De gewone
verkoudheid, bijvoorbeeld, muteert op een bijna kaleidoscopische manier.
Hoewel deze mutaties in hoog tempo plaatsvinden, is hun effect op het
verleggen van de grenzen waarbinnen macht wordt uitgeoefend aanzienlijk
minder abrupt dan dat van technologische veranderingen. Waarom? Een deel
van de reden is dat microben er meer baat bij hebben om hun gastheer
enkel te infecteren dan om hem te doden. Virulente infecties die hun
gastheren te gemakkelijk doden, hebben de neiging om zichzelf in het
proces uit te roeien. Microparasieten kunnen alleen overleven als ze hun
gastheren niet te snel of allemaal tegelijk doden.

Dat wil natuurlijk niet zeggen dat er geen uitbarstingen van dodelijke
ziektes kunnen voorkomen die de machtsbalans veranderen. Zulke episodes
kwamen prominent in de geschiedenis voor. De Zwarte Dood decimeerde
aanzienlijke delen van de Euraziatische bevolking en bracht ernstige
schade toe aan de internationale economie van de veertiende eeuw.

\textbf{Wat had kunnen zijn}

Je kunt geschiedenis bekijken als wat er daadwerkelijk is gebeurd, maar
ook als wat er had kunnen gebeuren. Het was niet onmogelijk voor
microparasieten om in de moderne tijd nog grotere schade te hebben
aangericht. Stel je voor dat er, net als malaria maar dan erger, een
ziekte was geweest die de macht van het Westen aan haar grenzen had
kunnen tegenhouden. De eerste Portugese zeevaarders die richting Afrika
gingen, hadden zomaar een dodelijk retrovirus kunnen oplopen -- een nog
besmettelijkere vorm van aids -- waardoor die hele nieuwe handelsroute
naar Azië er nooit was gekomen. Ook Columbus en de eerste kolonisten in
de Nieuwe Wereld hadden ziekten kunnen tegenkomen die hen op dezelfde
manier zouden teisteren als de inheemse lokale bevolkingen, die door
mazelen en andere Westerse kinderziekten werden getroffen. Toch gebeurde
dit niet, waardoor je bijna het gevoel krijgt alsof de geschiedenis een
soort lotsbestemming kent.

Microben droegen in de moderne periode minder bij aan het afremmen van
de consolidatie van macht, maar faciliteerden het eerder. Westerse
troepen en kolonisten ervaarden dat de technologische voordelen, die hen
toelieten om macht uit te oefenen, werden versterkt door de
microbiologische. Westerlingen waren gewapend met ongeziene biologische
wapens, namelijk hun relatieve immuniteit voor kinderziekten die
inheemse volkeren vaak juist verwoestten. Dit gaf Westerse reizigers een
duidelijk voordeel op hun minder dicht bevolkte tegenstanders. De
overdracht van ziekten ontwikkelde zich bijna geheel in één richting,
van Europa naar buiten. Er was geen gelijkaardige overdracht in de
andere richting, van de periferie naar de kern.

Als een mogelijk tegenvoorbeeld hebben sommigen beweerd dat Westerse
ontdekkingsreizigers syfilis van de Nieuwe Wereld naar Europa
importeerden. Dit is discutabel. Als het echter waar is, bleek het geen
significante barrière voor de uitoefening van macht te zijn. Syfilis had
vooral als gevolg dat de seksuele normen en gewoonten in het Westen
veranderden. Vanaf het einde van de vijftiende eeuw tot het laatste
kwart van de twintigste eeuw was de impact van microben op industriële
samenleving steeds minder desastreus. Ondanks de persoonlijke tragedies
en het ongeluk, veroorzaakt door uitbraken van tuberculose, polio, en
griep, kwamen er in de moderne periode geen nieuwe ziekten op die de
megapolitieke impact van de Antonijnse plagen of de Zwarte Dood zelfs
benaderden. Tijdens de moderne periode verbeterde de volksgezondheid, en
werden vaccinaties en tegengif wijdverbreid. In grote lijnen verminderde
dit de invloed van infectieuze microben op de machtsprojectie, waardoor
het relatieve belang van technologie sterk toenam.

De recente opkomst van AIDS en waarschuwingen over de potentiële
verspreiding van exotische virussen zijn hints dat er geen garantie is
dat de rol van microben in de toekomst even goedaardig zal zijn als dat
het over de afgelopen vijfhonderd jaar is geweest. Maar het is
onmogelijk om te weten of, of wanneer, een nieuwe plaag de wereld zal
infecteren. De kans dat een uitbarsting van microparasieten, zoals een
virale pandemie, de megapolitieke dominantie van technologie zou kunnen
verstoren, is vele malen groter dan een verandering van klimaat of
topografie.

We hebben geen manier om grote veranderingen in het leven op aarde zoals
we dat kennen te monitoren of te voorspellen. Dus houden we onze vingers
gekruist en hopen we dat de belangrijkste megapolitieke krachten in het
komende millennium technologisch zullen zijn, en niet biologisch van
aard. Als geluk aan de zijde van de mensheid blijft, zal technologie als
de leidende megapolitieke variabele aan belang blijven toenemen.

Dit was echter niet altijd zo, zoals een overzicht van de eerste grote
megapoliteike transformatie, de Agrarische Revolutie, duidelijkt toont.

\bookmarksetup{startatroot}

\chapter{\texorpdfstring{\textbf{TEN OOSTEN VAN
EDEN}}{TEN OOSTEN VAN EDEN}}\label{ten-oosten-van-eden}

\emph{De landbouwrevolutie en de verfijning van geweld}

\begin{quote}
`Toen vroeg de HEER: 'Waar is je broer Abel?' Kaïn antwoordde: `Dat weet
ik niet. Moet ik soms waken over mijn broer?' `Wat heb je gedaan?' zei
de HEER. `Hoor toch hoe het bloed van je broer uit de aarde tot mij
roept!' - GENESIS 4:9-10
\end{quote}

Vijfhonderd generaties geleden begon de eerste faseverandering in de
organisatie van de menselijke samenleving.{[}\^{}51{]} Onze voorouders
in verscheidene regio's pakten met tegenzin primitieve werktuigen op,
zoals scherp gemaakte stokken en provisorische schoffels, en gingen aan
het werk. Toen ze de eerste gewassen zaaiden, legden ze ook een nieuwe
basis voor de macht in de wereld. De landbouwrevolutie was de eerste
grote economische en sociale revolutie. Deze revolutie begon met de
verdrijving uit Eden en ontwikkelde zich zo langzaam dat het het jagen
en verzamelen, toen de twintigste eeuw aanbrak, nog niet volledig had
verdrongen in alle geschikte gebieden op aarde. Deskundigen geloven dat
de landbouw, zelfs in het Nabije Oosten waar het voor het eerst
ontstond, werd ingevoerd via `een lang, stapsgewijs proces' dat
`wellicht vijfduizend jaar of meer in beslag nam'.{[}\^{}52{]}

Het lijkt wellicht overdreven om een proces dat zich over millennia
uitstrekte als een `revolutie' te omschrijven. Toch was de komst van de
landbouw precies dat: een vertraagde revolutie die het menselijk leven,
door de logica van geweld te veranderen, transformeerde. Waar landbouw
vaste grond kreeg, werd geweld een steeds zichtbaarder aspect van het
dagelijkse leven. Zij die in staat waren om geweld te organiseren of
beheersen, gingen de samenleving domineren.

Wie de agrarische revolutie begrijpt, heeft een eerste stap gezet in het
doorgronden van de informatierevolutie. De introductie van het ploegen
en oogsten is een schoolvoorbeeld van hoe de maatschappelijke
organisatie ingrijpend kan veranderen door een ogenschijnlijk simpele
verandering in de aard van arbeid. Als je deze revolutie uit het
verleden in het juiste perspectief kunt plaatsen, zul je beter kunnen
inschatten hoe de geschiedenis zich zal ontvouwen als reactie op de
nieuwe logica van geweld, die met microprocessoren haar intrede deed.

Om het revolutionaire karakter van de landbouw op waarde te schatten,
moet je eerst een beeld hebben van hoe de oermaatschappij functioneerde.
We hebben dit thema uitgebreid verkend in \emph{The Great Reckoning},
maar zullen het hieronder kort schetsen. Gedurende een lange
prehistorische periode van relatieve stilstand, waarin het menselijk
bestaan nauwelijks evolueerde van generatie op generatie, waren
jager-verzamelaarsamenlevingen de enige sociale organisatiestructuren.
Antropologen stellen dat de mens gedurende 99 procent van zijn bestaan
op aarde een jager en verzamelaar is geweest. Cruciaal voor het
langdurige succes, maar ook het uiteindelijke falen van
jager-verzamelaarsgroepen, is het feit dat ze op zeer kleine schaal in
een uitgestrekt gebied moesten opereren.

Ze konden alleen overleven waar de bevolkingsdichtheid gering was. Denk,
om te begrijpen waarom, aan de problemen die grotere groepen met zich
mee zouden hebben gebracht. Ten eerste zou een groep van duizend jagers,
die tegelijk door het landschap trok, zoveel tumult hebben veroorzaakt
dat het wild op de vlucht zou zijn gejaagd. Erger nog, zelfs als een
klein leger jagers er nu en dan in zou zijn geslaagd om een enorme kudde
wild in het nauw te drijven, zou het voedsel dat ze verzamelden --
inclusief vruchten en eetbare planten uit de natuur -- niet lang in
overvloed zijn gebleven. Een grote groep foerageerders zou het landschap
hebben uitgeput door roofbouw, vergelijkbaar met een uitgehongerd leger
tijdens de Dertigjarige Oorlog. Om deze uitputting te beperken, moesten
de groepen dus klein blijven. Zoals Stephen Boyden schrijft in
\emph{Western Civilization in Biological Perspective}:
`Jager-verzamelaarsgroepen tellen doorgaans tussen de vijfentwintig en
vijftig individuen'.\footnote{Perry Barlow, `Thinking Locally, Acting
  Globally', \emph{Time}, 15 januari 1996, p.~57.}

Vandaag de dag is leven op tienduizend hectare in een gematigd klimaat
een luxe die slechts voor de allerrijksten is weggelegd. Een familie van
jager-verzamelaars had aan minder nauwelijks genoeg gehad. Ze hadden
doorgaans duizenden hectare per persoon nodig, zelfs in de meest
vruchtbare gebieden. Dit verklaart waarom periodes van snelle
bevolkingsgroei, juist in periodes met goede omstandigheden voor
landbouw, konden uitmonden in crisissen. Jager-verzamelaars hadden
zoveel ruimte nodig dat hun gemeenschappen nauwelijks dichter bevolkt
waren dan het leefgebied van beren, lang voor de eerste akkers
ontstonden.

Op kleine verschillen na leek het menselijke dieet op dat van een beer.
Deze foeragerende samenlevingen waren voor hun voedsel afhankelijk van
wat het open landschap en nabijgelegen wateren te bieden hadden. Hoewel
sommigen visten, waren de meeste jagers voor een derde tot een vijfde
van hun voedselinname afhankelijk van de eiwitten van grote zoogdieren.
Naast enkele eenvoudige werktuigen en voorwerpen die ze met zich
meedroegen, hadden jager-verzamelaars vrijwel geen technologie tot hun
beschikking. Ze misten doorgaans effectieve methoden om vlees of ander
voedsel langdurig te bewaren. Het meeste voedsel moest kort na het
verzamelen worden geconsumeerd, anders bedierf het. Dit betekent
natuurlijk niet dat sommige jager-verzamelaars geen bedorven voedsel
aten. Eskimo's, zo meldt Boyden, `zouden een bijzondere voorliefde
hebben voor voedsel in staat van ontbinding'.\footnote{John Dos Passos,
  \emph{The Big Money} (New York: \emph{Harcourt, Brace \& Co.}, 1936).}
Hij citeert de waarnemingen van deskundigen die stellen dat Eskimo's
`viskoppen begraven en ze laten wegrotten tot de graten dezelfde
consistentie hebben als het vlees. Vervolgens kneden ze de stinkende
massa tot een pasta en eten die op'; ze zijn ook dol op de `vette,
madenrijke larven van de kariboevlieg, rauw geserveerd\ldots{}
hertenkeutels, die ze als bessen eten\ldots{} en merg van meer dan een
jaar oud, wemelend van de maden'.\footnote{Clarke, op. cit., p.~29.}

Op enkele delicatessen na legden jager-verzamelaars nauwelijks
voedseloverschotten aan. Zoals antropoloog Gregg opmerkt, `slaan
rondtrekkende bevolkingsgroepen over het algemeen geen voedselvoorraden
aan als buffer tegen seizoensgebonden of onverwachte schaarste.' Er viel
bij de jager-verzamelaars bijgevolg weinig te stelen. Het was onhoudbaar
in omstandigheden waarin geen voedseloverschot kon worden bewaard, om
een arbeidsdeling op te bouwen waarin mensen gespecialiseerd waren in
het toepassen van geweld. De logica van de jacht bracht ook met zich mee
dat geweld tussen jager-verzamelaarsgroepen nooit grootschalig kon zijn,
omdat de groepen zelf klein moesten blijven.

De kleine omvang van de jager-verzamelaarsgroepen was ook op een andere
manier voordelig. De leden van zulke kleine groepen kenden elkaar door
en door, waardoor ze effectiever konden samenwerken. Besluitvorming
wordt lastiger naarmate een groep groter wordt, omdat er steeds meer
tegenstrijdige belangen ontstaan. Denk maar eens aan hoe moeilijk het is
om met twaalf man een etentje te organiseren. Stel je voor hoe hopeloos
de taak zou zijn geweest om honderden of duizenden mensen te organiseren
tijdens een voortdurend verplaatsend feestmaal. Omdat zij geen
blijvende, afzonderlijke politieke organisatie of bureaucratie hadden,
zoals vereist voor oorlogsspecialisatie, moesten
jager-verzamelaarsgroepen vertrouwen op overtuiging en consensus --
principes die het best werken bij kleine groepen met een relatief
gemoedelijke stemming.

Of het er daadwerkelijk gemoedelijk aan toe ging in groepen
jager-verzamelaars staat echter ter discussie. Sir Henry Maine verwijst
naar `de universele oorlogszuchtigheid van de primitieve mens'. In zijn
woorden: \textquotesingle Het is niet vrede dat natuurlijk en primitief
is, maar oorlog.'\footnote{Geciteerd in Kline en Burstein, op. cit.,
  p.~05.} Zijn visie wordt ondersteund door het werk van evolutionair
biologen. R. Paul Shaw en Yuwa Wong merken op: `Er zijn sterke
aanwijzingen dat veel van de zichtbare verwondingen op overblijfselen
van de \emph{Australopithecus}, \emph{Homo erectus} en \emph{Homo
sapiens} uit de vierde Europese ijstijd en de daaraan voorafgaande
periodes het gevolg waren van gevechten.'\footnote{Friedrich A. von
  Hayek, de denationalisatie van geld (London: \emph{Institute of
  Economic Affairs}, 1976), p.~47.} Anderen trekken dit echter in
twijfel. Deskundigen als Stephen Boyden beweren dat primitieve groepen
doorgaans niet oorlogszuchtig of gewelddadig waren. Er ontstonden
sociale conventies om interne spanningen te verminderen en het
makkelijker te maken om de jachtbuit te verdelen. Vooral in gebieden
waar men op groter wild joeg, dat voor een enkele jager moeilijk te
vellen was, ontstonden religieuze en sociale regels die de herverdeling
van al het gevangen wild onder de hele groep faciliteerden. Wanneer een
calorierijke buit moest worden verdeeld, kregen andere jagers de eerste
prioriteit. Noodzaak, en niet sentiment, was hierbij de drijfveer. De
eersten die aanspraak maakten op de buit waren de economisch meest
bekwame en militair sterkste leden, niet de zieken en de zwakken. Een
belangrijke reden voor deze voorrangsregel was ongetwijfeld het feit dat
jagers in de fleur van hun leven ook de militair sterkste leden van de
kleine groep waren. Door hen het eerste recht op de buit te geven,
werden mogelijk dodelijke interne conflicten in de groep tot een minimum
beperkt.

Zolang de bevolkingsdichtheid van de jager-verzamelaars laag bleef,
waren hun goden niet militant, maar ze belichaamden natuurkrachten of de
dieren waarop ze jaagden. De beperkte hoeveelheid bezittingen en de
uitgestrekte, open gebieden maakten oorlogsvoering in de meeste gevallen
overbodig. Buiten de eigen kleine familie of clan waren er weinig buren
die een bedreiging konden vormen. Doordat jager-verzamelaars rondtrokken
op zoek naar voedsel, werden bezittingen die het strikte minimum
overschreden een last. Wie weinig bezat, had vanzelfsprekend ook weinig
last van schendingen van het eigendomsrecht. Omdat men weinig
investeerde in een specifieke plek, koos men er bij conflicten vaak voor
om de wegen te scheiden. Vluchten was een eenvoudige oplossing voor
persoonlijke vetes of andere buitensporige eisen. Dit betekent niet dat
de vroege mens vreedzaam was. Mogelijk waren ze gewelddadig en
onaangenaam in een mate die we ons nauwelijks kunnen voorstellen. Maar
als ze geweld gebruikten, was dat meestal om persoonlijke redenen of,
wat misschien nog erger is, voor de sport.

Jagers-verzamelaars leefden in kleine groepen waarin, afgezien van het
onderscheid tussen de geslachten, nauwelijks sprake was van een
arbeidsverdeling. Ze hadden geen georganiseerde overheid, doorgaans geen
permanente nederzettingen en geen mogelijkheid om rijkdom te vergaren.
Zelfs elementaire bouwstenen van de beschaving, zoals een geschreven
taal, waren onbekend in de oereconomie. Zonder geschreven taal konden er
geen officiële archieven worden bijgehouden en dus geen geschiedenis.

\subsection{\texorpdfstring{\textbf{Overbejaging}}{Overbejaging}}\label{overbejaging}

Het bestaan als jager-verzamelaar kende heel andere prikkels dan die
waaraan we sinds de opkomst van de landbouw gewend zijn geraakt. Voor
het leven als jager-verzamelaar was maar weinig kapitaal nodig. Een paar
primitieve gereedschappen en wapens volstonden. Er waren geen
mogelijkheden om te investeren en er was zelfs geen particulier
grondbezit, met uitzondering van steengroeven waar vuursteen of
speksteen werd gewonnen.\footnote{Zie Carleton S. Coon, \emph{The
  Hunting Peoples} (New York: Nick Lyons Books, 1971), p.~275.} Zoals
antropologe Susan Alling Gregg in \emph{Foragers and Farmers} schreef,
`was het bezit van en de toegang tot hulpbronnen gemeenschappelijk
eigendom van de groep.'\footnote{Gregg, op. cit., p.~23.} Op zeldzame
uitzonderingen na, zoals vissers die aan de oevers van meren leefden,
hadden jagers-verzamelaars doorgaans geen vaste verblijfplaats. Omdat ze
geen permanent onderkomen hadden, was er voor hen weinig reden om hard
te werken voor bezittingen of het onderhoud ervan. Ze hoefden geen
hypotheek of belastingen te betalen en geen meubels te kopen. Het kleine
aantal consumptiegoederen dat ze bezaten, bestonden uit dierenhuiden en
persoonlijke versieringen die de groepsleden zelf maakten. Er waren
amper prikkels om iets te verzamelen dat als geld had kunnen dienen,
want er was nauwelijks iets te koop. In zulke omstandigheden stelde het
begrip \textquotesingle sparen\textquotesingle{} voor
jagers-verzamelaars weinig voor.

Omdat er geen reden was om te verdienen en er nauwelijks sprake was van
een arbeidsverdeling, moet het idee van hard werken als deugd vreemd
voor ze zijn geweest. Behalve in tijden van uitzonderlijke tegenspoed,
wanneer een langdurige inspanning vereist was om iets eetbaars te
vinden, verrichtte men weinig arbeid omdat er simpelweg weinig nodig
was. Er viel letterlijk niets te winnen door meer te presteren dan het
strikte minimum dat nodig was om te overleven. Voor de leden van een
doorsnee jagers-verzamelaarsgroep betekende dat een werkweek van slechts
acht tot vijftien uur.\footnote{Boyden, op. cit., p.~69.} Omdat de
arbeid van een jager de voedselvoorraad niet deed toenemen maar enkel
kon uitputten, droeg iemand die zich heldhaftig uitsloofde om meer
dieren te doden of meer vruchten te plukken dan men kon opeten voor het
bedierf, niets bij aan de welvaart. Integendeel, overbejaging
verminderde de kans om in de toekomst voedsel te vinden en had dus een
schadelijke invloed op het welzijn van de groep. Daarom bestraften of
verstootten sommige jagers-verzamelaars, zoals de Eskimo's, groepsleden
die zich aan overbejaging schuldig maakten.

Het voorbeeld van de Eskimo's die overbejaging bestraffen, is bijzonder
veelzeggend, omdat ze, veel meer dan anderen, vlees hadden kunnen
bewaren door het te bevriezen. Bovendien was het voor hen mogelijk om
ten minste een deel van de oliën, die uit grote zeedieren werden
gewonnen, op te slaan. Dat jagers-verzamelaars er over het algemeen voor
kozen om dit niet te doen, weerspiegelt hun veel passievere omgang met
de natuur. Het zou ook wat kunnen zeggen over de mate waarin cognitie en
mentale processen door cultuur worden beïnvloed. Het leerproces en
gedrag worden begrensd door de complexe omgeving, waardoor de toepassing
van sommige strategieën veel moeilijker kan zijn dan op het eerste
gezicht lijkt. Zoals R. Paul Shaw en Yuwa Wong schreven: `Omdat niches
in veel opzichten verschillen, verschillen ook de
leerpatronen.'\footnote{Shaw and Wong, op. cit., p.~69}

Vanuit dit perspectief bracht de opkomst van de landbouw meer met zich
mee dan enkel een verandering in het voedingspatroon; het ontketende ook
een grote revolutie in de organisatie van het economische leven en de
cultuur, en een transformatie van de logica van geweld. Landbouw bracht
grootschalige kapitaalgoederen in de vorm van grond en soms
irrigatiesystemen voort. De gewassen en gedomesticeerde dieren die de
boeren verbouwden en verzorgden, waren waardevolle bezittingen. Ze
konden worden opgeslagen, gehamsterd en gestolen. Omdat de gewassen
gedurende het hele groeiseizoen verzorgd moesten worden, van het zaaien
tot de oogst, werd wegtrekken bij dreigingen minder aantrekkelijk,
vooral in droge streken waar de mogelijkheden om gewassen te verbouwen
beperkt waren tot de kleine gebieden met een betrouwbare watertoevoer.
Naarmate ontsnappen moeilijker werd, namen de kansen op georganiseerde
afpersing en plunderingen toe. Boeren waren tijdens de oogsttijd vaak
het slachtoffer van overvallen. Geleidelijk nam de schaal van
oorlogvoering dus toe.

Hierdoor neigde de omvang van samenlevingen te vergroten, omdat
gewelddadige conflicten meestal werden gewonnen door de grotere groep.
Naarmate de concurrentie om grond en de controle over de opbrengst ervan
toenam, werden maatschappijen steeds minder mobiel. Er ontstond een
duidelijke arbeidsverdeling. Voor het eerst deden loonarbeid en
slavernij hun intrede. Boeren en veehouders specialiseerden zich in
voedselproductie. Pottenbakkers maakten opslagpotten voor voedsel.
Priesters baden tot hun god(en) voor regen en rijke oogsten.
Specialisten in geweld -- de voorvaderen van de overheid -- richtten
zich steeds meer op plundering en bescherming tegen plundering. Samen
met de priesters werden zij de eerste rijken in de geschiedenis.

In de vroege stadia van agrarische samenlevingen eisten deze krijgers
een deel van de jaarlijkse oogst op in ruil voor bescherming. Op plekken
waar de dreiging minimaal was, wisten zelfstandige boeren soms een
relatief grote mate van autonomie te behouden. Naarmate de
bevolkingsdichtheid echter toenam en de strijd om voedsel verhevigde,
vooral in woestijnachtige gebieden waar productieve grond schaars was,
kon de krijgersgroep een groot deel van de totale opbrengst voor zich
opeisen. Met de inkomsten uit deze afgedwongen heffingen, die opliepen
tot wel 25 procent van de graanoogst en de helft van de groei van de
veestapel, stichtten deze krijgers de eerste staten. Door landbouw nam
het belang van dwang aanzienlijk toe. De sterke toename van middelen die
buitgemaakt konden worden, leidde tot een forse stijging van het aantal
plunderingen.

Het duurde millennia voordat de logica van de agrarische revolutie zich
volledig manifesteerde. Lange tijd leefden dunbevolkte gemeenschappen
van boeren in gematigde streken mogelijk grotendeels op dezelfde wijze
als hun jagende en verzamelende voorouders. Waar grond en regenval
overvloedig waren, verbouwden boeren hun gewassen op kleine schaal
zonder veel gewelddadige inmenging. Maar toen de bevolking over een
periode van enkele duizenden jaren groeide, kregen zelfs boeren in
dunbevolkte gebieden te maken met onvoorspelbare plunderingen, waardoor
ze soms niet genoeg zaaigoed overhielden voor de oogst van het volgende
jaar. Wederzijdse plunderingen, oftewel anarchie, was een van de
uitersten, net als onbeschermde gemeenschappen die leefden zonder een
gespecialiseerde organisatie die het monopolie op geweld had.

Na verloop van tijd verspreidde de aan landbouw inherente logica van
geweld zich over een steeds groter gebied. De regio's waar landbouw en
veeteelt konden voortbestaan zonder de strooptochten van overheden,
werden teruggedrongen tot enkele zeer afgelegen gebieden. De
Kafir-regio's van Afghanistan, om een extreem voorbeeld te noemen, boden
weerstand tegen de invoering van een centrale overheid tot in het
laatste decennium van de negentiende eeuw. In dat proces waren ze echter
al eeuwen eerder veranderd in een vrij militante samenleving,
georganiseerd op basis van verwantschapslijnen. Dergelijke structuren
waren niet in staat om op grote schaal militaire macht te mobiliseren.
Tot de Britten moderne wapens naar de regio brachten, bleven de Kafirs
onafhankelijk in hun afgelegen Bashgal- en Waigal-valleien, omdat hun
bolwerken werden beschermd door de topografie, met hoge bergen en
woestijnen die hen scheidden van veroveraars van buitenaf.\footnote{Voor
  meer details over de Kafirs, zie Schuyler Jones, \emph{Men of
  Influence in Nuristan} (Londen: Seminar Press, 1974).}

Na verloop van tijd drukte de fundamentele logica van de agrarische
revolutie haar stempel op de samenlevingen waar landbouw werd toegepast.
Door de landbouw konden menselijke gemeenschappen op een veel grotere
schaal functioneren. Ongeveer tienduizend jaar geleden begonnen steden
op te komen. Hoewel klein naar de huidige maatstaven, waren ze de centra
van de eerste `beschavingen', een woord afgeleid van \emph{civitas}, wat
in het Latijn `burgerschap' of `inwoner van een stad' betekent. Omdat de
landbouw bezittingen voortbracht die geplunderd en beschermd moesten
worden, ontstond ook de noodzaak voor een voorraadadministratie.
Belasting heffen is onmogelijk als men geen administratie kan voeren en
ontvangstbewijzen kan uitschrijven. De symbolen die in de boekhouding
werden gebruikt, vormden de beginselen van het schrift, een vernieuwing
die bij jagers en verzamelaars nooit had bestaan.

De landbouw verlegde ook de horizon waarbinnen de mens problemen moest
oplossen. Jagersstammen leefden met een zeer beperkte tijdshorizon. Ze
ondernamen zelden projecten die langer dan enkele dagen duurden. Het
planten en oogsten van een gewas nam echter maanden in beslag. Projecten
met een langere doorlooptijd dreven boeren ertoe hun aandacht op de
sterren te richten. Gedetailleerde astronomische waarnemingen waren een
voorwaarde voor het opstellen van almanakken en kalenders die als
leidraad dienden voor de beste zaai- en oogsttijden. Met de komst van de
landbouw verruimde de horizon van de jagers zich.

\section{\texorpdfstring{\textbf{Eigendom}}{Eigendom}}\label{eigendom}

De overgang naar een sedentaire agrarische samenleving leidde tot het
ontstaan van particulier eigendom. Het is evident dat niemand bereid was
een heel groeiseizoen op de akkers te zwoegen, om daarna toe te kijken
hoe een ander het resultaat van zijn arbeid opeiste. Het idee van
eigendom ontstond als een onvermijdelijk gevolg van de landbouw. De
helderheid van het concept van particulier eigendom werd echter
ondermijnd door de logica van geweld die met de introductie van de
landbouw gepaard ging. De opkomst van eigendom werd vertroebeld door het
feit dat de megapolitieke macht van individuen niet langer zo
gelijkwaardig was als in jager-verzamelaarssamenlevingen, waar elke
gezonde volwassen man een jager was en even zwaar bewapend was als ieder
ander. De landbouw leidde tot een specialisatie in geweld. Juist omdat
er iets te stelen viel, werden investeringen in betere wapens lonend.
Het gevolg was diefstal, vaak op een sterk georganiseerde schaal.

De machtigen konden nu een nieuwe vorm van roof organiseren: een
geweldsmonopolie, oftewel een `overheid'. Dit zorgde voor een scherpe
tweedeling in samenlevingen, waardoor zeer verschillende omstandigheden
ontstonden voor degenen die profiteerden van de plunderingen en de massa
armen die de velden bewerkten. De enkelingen die de militaire macht in
handen hadden, konden nu rijk worden, samen met anderen die hun gunst
genoten. De god-koningen en hun bondgenoten, de diverse lagere, lokale
machthebbers die over de eerste staten in het Nabije Oosten heersten,
genoten van eigendomsvormen die veel meer op de moderne leken dan die
van de grote massa die onder hen zwoegde.

Natuurlijk is het anachronistisch om in de vroege agrarische
samenlevingen te spreken van een onderscheid tussen particulier en
publiek bezit. De heersende god-koning beschikte over alle middelen van
de staat op een manier die nauwelijks te onderscheiden was van het bezit
van een uitgestrekt landgoed. Net als in de feodale periode van de
Europese geschiedenis was al het bezit onderworpen aan de heerschappij
van hogere machthebbers. Wie lager op de hiërarchische ladder stond,
merkte dat zijn eigendom te allen tijde onderhevig was aan de grillen
van de heerser.

Het feit dat de heerser niet door de wet werd beperkt, betekende echter
niet dat hij het zich kon veroorloven om zomaar alles in beslag te nemen
wat hij wilde. De vrijheid van de farao werd ingeperkt door de heersende
kosten en baten, net zoals die van de premier van Canada vandaag de dag.
Sterker nog, de farao werd veel meer belemmerd dan hedendaagse leiders,
vanwege de beperkte transport- en communicatiemogelijkheden. Alleen al
het verplaatsen van een buit, vooral als het voornamelijk
landbouwproducten waren, zorgde voor grote verliezen door bederf en
diefstal. Door meer ambtenaren aan te stellen die elkaar in de gaten
hielden, werd diefstal beperkt, maar stegen de kosten voor de farao
aanzienlijk. Decentrale autoriteit verbeterde soms de productie, maar
vormde ook krachtige lokale leiders die soms de dynastieke controle
uitdaagden. Zelfs despoten in het Oosten waren niet helemaal vrij om te
doen wat ze wilden; ze moesten de bestaande machtsverhoudingen
respecteren.

Hoewel iedereen, inclusief de rijken, onderhevig was aan willekeurige
onteigening, konden sommigen toch eigen bezit vergaren. Net als nu
besteedde de staat een groot deel van haar inkomsten aan openbare
werken. Projecten als irrigatiesystemen, religieuze monumenten en
koninklijke crypten boden architecten en ambachtslieden de kans om een
inkomen te verdienen. Sommige individuen in een bevoorrechte positie
slaagden erin om een flinke hoeveelheid privébezit op te bouwen. Veel
van de overgebleven spijkerschrifttabletten uit Sumer, een oude
Mesopotamische beschaving, bevatten handelsdocumenten, vooral over het
overdragen van eigendomsrechten.

In de vroege landbouwsamenlevingen bestond privé-eigendom wel, maar
zelden aan de onderkant van de sociale piramide. De overgrote
meerderheid van de bevolking bestond uit boeren die te arm waren om veel
rijkdom te vergaren. Sterker nog, op enkele uitzonderingen na, waren de
meeste boeren tot in de moderne tijd zo arm dat ze bij elke droogte,
overstroming of plaag, voortdurend het risico liepen om van de honger om
te komen. Daarom moesten boeren hun zaken zo organiseren dat de risico's
in slechte jaren tot een minimum beperkt bleven. In de brede, verarmde
lagen van de samenleving bestond een primitievere vorm van eigendom. Die
vergrootte de overlevingskans, maar ging ten koste van de mogelijkheid
om kapitaal te vergaren en hogerop te komen in het economische systeem.

\subsection{\texorpdfstring{\textbf{Boerenverzekering}}{Boerenverzekering}}\label{boerenverzekering}

De oplossing lag in wat antropologen en sociale historici omschrijven
als het `gesloten dorp'. Bijna elke boerensamenleving in de premoderne
tijden kende het `gesloten dorp' als belangrijkste vorm van economische
organisatie. Anders dan in modernere economische stelsels, waar
individuen doorgaans met vele kopers en verkopers op een open markt
handelen, functioneerden de huishoudens in het gesloten dorp samen als
een soort collectief of een grote familie. Dit was geen open
marktplaats, maar een gesloten systeem waarin alle economische
transacties van het dorp doorgaans werden afgesloten met één monopolist:
de lokale landheer, of zijn vertegenwoordigers. Het dorp maakte
gezamenlijk afspraken met de heer, waarbij men doorgaans een hoog
percentage van de oogst leverde in plaats van een vaste huurprijs. Deze
proportionele pacht betekende dat de landheer een deel van het risico
van een slechte oogst voor zijn rekening nam. Uiteraard streek de
landheer ook het grootste deel van de potentiële winst op. Meestal
zorgden de landheren ook voor het zaaigoed.

Deze regeling beperkte ook het gevaar van hongersnood. Het vereiste dat
de landheer, in plaats van de boer, een onevenredig deel van zijn
oogstaandeel bewaarde. Omdat de landbouwopbrengsten in het verleden in
veel gebieden schrikbarend laag waren, moesten er soms wel twee zaden
worden geplant voor elke drie die men oogstte. Onder zulke
omstandigheden zou een slechte oogst massale hongersnood betekenen. De
boeren hadden natuurlijk de rationele voorkeur voor een regeling waarbij
de landheer in hun voortbestaan investeerde. Hoewel ze moesten kopen
tegen gemonopoliseerde prijzen, goedkoop moesten verkopen en de landheer
van arbeid in natura moesten voorzien, vergrootten de boeren hun
overlevingskansen. Om soortgelijke redenen gaf de gewone boer in een
gesloten dorp de zekerheid van privébezit op. Door zich afhankelijk te
maken van de dorpsleider, maakte het gezin meer kans om te profiteren
van de regelmatige herverdeling van akkers. Het kwam geregeld voor dat
de dorpsleider de beste stukken grond voor zichzelf en zijn kring hield.
Dat was echter een risico dat de boeren moesten tolereren om te kunnen
genieten van de overlevingsverzekering die het diffuse dorpsbezit van de
akkers bood. In tijden waarin de oogstopbrengsten erbarmelijk laag
waren, kon een verschil in de groeiomstandigheden tussen akkers die een
steenworp van elkaar verwijderd lagen, het verschil maken tussen
hongersnood en overleven. Boeren kozen vaak voor de regeling die het
risico op verlies verkleinde, zelfs als dat betekende dat ze elke hoop
op grotere welvaart moesten opgeven.

Over het algemeen kwam risicomijdend gedrag veel voor bij alle groepen
die op de rand van het bestaan balanceerden. De enorme uitdaging om te
overleven in premoderne samenlevingen heeft altijd het gedrag van de
armen beperkt. Een interessant kenmerk van deze risicoaversie, zoals
besproken in \emph{The Great Reckoning}, is dat het de bandbreedte van
vreedzaam economisch gedrag dat individuen maatschappelijk gezien
mochten vertonen, verkleinde. Taboes en sociale beperkingen remden
experimenten en innovatief gedrag af, zelfs ten koste van potentieel
voordelige verbeteringen in bestaande werkwijzen.\footnote{Zie Samuel L.
  Popkin, \emph{The Rational Peasant} (Berkeley: University of
  California Press 1979), p.~13.} Dit was niet onlogisch, aangezien
experimenteren tot grotere verschillen in uitkomsten leidt. Een grotere
variatie in de resultaten kan niet alleen hogere opbrengsten betekenen,
maar er is ook het risico op rampzalige verliezen voor mensen die
nauwelijks kunnen overleven. Een groot deel van de culturele energie van
arme landbouwsamenlevingen is altijd gericht geweest op het onderdrukken
van experimenten. Deze onderdrukking was in feite hun alternatief voor
een verzekering. Zouden ze een verzekering hebben gehad, of voldoende
spaargeld om hun experimenten zelf te dekken, dan zouden zulke sterke
sociale taboes niet nodig zijn geweest om hun overleven te waarborgen.

Culturen zijn functionele aanpassingsmechanismen, geen voorkeuren. Wat
in de ene omgeving werkt, kan elders nutteloos of zelfs schadelijk zijn.
Mensen leven in een grote verscheidenheid aan leefomgevingen. Het brede
scala aan potentiële niches waarin wij leven, vereist gedragsvarianten
die te complex zijn om door instinct alleen te worden gestuurd. Daarom
wordt gedrag cultureel geprogrammeerd. Voor de overgrote meerderheid in
veel agrarische samenlevingen was de cultuur voornamelijk gericht op
overleven, en nauwelijks meer dan dat, terwijl de luxe om deel te nemen
aan open markten voor een selecte groep was weggelegd.

Persoonlijke bekwaamheid en individuele keuze, of zoals de moderne
Amerikanen zeggen ``the pursuit of happiness'', werden onderdrukt door
taboes en maatschappelijke beperkingen die juist onder de armen het
sterkst waren. Zulke beperkingen werden in maatschappijen met een
beperkte productiviteit slechts met grote moeite overwonnen. Waar en
wanneer de landbouwproductiviteit hoger lag, zoals in het oude
Griekenland, vonden kleine megapolitieke revoluties plaats. Eigendom nam
modernere vormen aan. `Allodium', of vrij eigendom, deed zijn intrede.
Grond werd doorgaans verpacht tegen een vaste vergoeding, en de pachter
nam het economische risico voor zijn rekening, maar genoot ook van een
groter deel van de winst als de oogst goed was. Met meer spaarmiddelen
wordt het mogelijk om het risico van riskant gedrag zelf te dragen.
Onder dergelijke omstandigheden konden zelfstandige boeren zich boven de
boerenstand verheffen en soms zelfs een eigen vermogen opbouwen.

Het is kenmerkend voor sociale organisatiestructuren dat
eigendomsrechten en verhoudingen meer marktgericht worden, vooral aan de
economische top of soms zelfs in de volledige economie, wanneer een
samenleving zich uit de armoede werkt. Het is hierbij belangrijk om op
te merken dat de meest voorkomende organisatievorm van de agrarische
samenleving historisch gezien in wezen feodaal was, met marktwerking aan
de top en een gesloten dorpssysteem aan de basis. De grote massa boeren
was in bijna alle premoderne agrarische samenlevingen aan het land
gebonden. Zolang de landbouwproductiviteit laag bleef, of hogere
productiviteit afhankelijk was van centrale irrigatiesystemen, bleven de
vrijheid en eigendomsrechten van individuele boeren onderaan minimaal.
In zulke omstandigheden overheersten feodale eigendomsvormen. Grondbezit
was gebaseerd op pacht, niet op een volledige eigendomstitel, en het
recht op verkoop, schenking en vererving was doorgaans beperkt.

Het feodalisme in zijn diverse vormen was niet enkel een reactie op de
voortdurende dreiging van geweld. Het was tevens een antwoord op de
ontzettend lage productiviteit. In agrarische samenlevingen gingen die
twee vaak hand in hand, en ze versterkten elkaar met regelmaat. Wanneer
het openbare gezag instortte, namen eigendomsrechten en welvaart over
het algemeen evenredig af. Een ineenstortende productiviteit ondermijnde
op haar beurt eveneens het gezag. Hoewel niet elke droogte of ongunstige
klimaatverandering leidde tot het verval van het openbare gezag, was dit
in veel gevallen wel zo.

\section{\texorpdfstring{\textbf{De feodale revolutie van het jaar
1000}}{De feodale revolutie van het jaar 1000}}\label{de-feodale-revolutie-van-het-jaar-1000}

Dit was het geval bij de transformatie van het jaar 1000, die de feodale
revolutie in gang zette.\footnote{Zie Bois, op. cit.} De politieke en
economische realiteit rond die periode week op cruciale punten af van
het beeld dat we nu van de Middeleeuwen hebben. In de eerste eeuwen na
de val van Rome kwijnde de economie van West-Europa weg. De Germaanse
koninkrijken die in de gebieden van het voormalige Romeinse Rijk
opkwamen, hadden tal van functies van de Romeinse staat overgenomen, zij
het op een veel minder ambitieus niveau. De infrastructuur werd min of
meer aan haar lot overgelaten. Naarmate de eeuwen verstreken, raakten
bruggen en aquaducten in verval en werden ze onbruikbaar. Hoewel
Romeinse munten nog in gebruik waren, raakten ze nagenoeg uit
circulatie. De bloeiende grondhandel uit de Romeinse tijd viel stil.
Steden, ooit administratieve centra, verdwenen vrijwel geheel samen met
het vermogen van de staat om belasting te heffen. En daarmee verdwenen
ook bijna alle andere kenmerken van beschaving.

De `donkere middeleeuwen' heten niet voor niets zo. Geletterdheid werd
zo zeldzaam dat wie kon lezen en schrijven vrijstelling van
rechtsvervolging kon verwachten voor vrijwel elke misdaad, moord
inbegrepen. Artistieke, wetenschappelijke en technische vaardigheden die
in de Romeinse tijd sterk ontwikkeld waren, raakten in de vergetelheid.
Van de aanleg van wegen tot het enten van wijnstokken en fruitbomen,
West-Europa paste veel technieken die ooit algemeen bekend waren en op
hoog niveau werden toegepast, niet langer toe. Zelfs een oeroud werktuig
als de pottenbakkersschijf verdween op veel plaatsen. De
mijnbouwactiviteit nam af, de metaalbewerking liep terug, en
irrigatiewerken in het Middellandse Zeegebied raakten door verwaarlozing
in verval.' \footnote{Zie Frances en Joseph Gies, \emph{Cathedral,
  Forge, and Waterwheel: Technology and Invention in the Middle Ages}
  (New York: HarperCollins, 1994), p.40.} Zoals historicus Georges Duby
opmerkte: `Aan het einde van de zesde eeuw was Europa een uiterst
onbeschaafde plek.' \footnote{Geciteerd in ibid., p.42.} Hoewel er rond
het jaar 800 onder het bewind van Karel de Grote een korte heropleving
van centraal gezag was, viel al snel na zijn dood alles weer uiteen.

Een verrassend gevolg van dit sombere tafereel was dat de ineenstorting
van de Romeinse staat de levensstandaard van kleine boeren
waarschijnlijk voor enkele eeuwen juist verhoogde. De Germaanse
koninkrijken die West-Europa tijdens de donkere middeleeuwen
domineerden, namen enkele van de wat vrijere sociale gewoonten over die
hun voorouderlijke stammen kenden, zoals de juridische gelijkheid van
vrije boeren. Als gevolg daarvan waren kleine boeren in de donkere
middeleeuwen veel vrijer dan ze in de feodale eeuwen zouden geweest
zijn. Hieruit kunnen we tevens afleiden dat ze welvarender waren. Zoals
we eerder zagen bij de analyse van eigendomsvormen onder verschillende
productiviteitsniveau's, is vrij eigendom historisch vaak gepaard gegaan
met de relatieve welvaart van kleine boeren. De gesloten dorpen en
feodale eigendomsvormen ontstonden vooral in gebieden waar de
mogelijkheid van kleine boeren om in hun levensonderhoud te voorzien
meer onder druk stond.

Toegegeven, door de vrijwel volledige ineenstorting van de economie
tijdens de donkere middeleeuwen liepen kleine boeren de voordelen van
handel en grotere afzetmarkten mis. Door het wegvallen van de steden
stortte de geldeconomie in, maar tegelijkertijd hoefde het platteland
niet meer op te draaien voor de kosten van een verstikkend
overheidsapparaat. Guy Bois schreef hierover dat de Romeinse stad een
parasitaire gemeenschap was en geen productiecentrum: `In de Romeinse
tijd was de voornaamste functie van een stad politiek van aard. Ze
leefde voornamelijk van de inkomsten die via de grondbelasting uit de
omliggende gebieden werden aangevoerd\ldots{} De stad produceerde in
feite weinig tot niets voor het omliggende platteland'.\footnote{Bois,
  op. cit., p.78.} De ineenstorting van het Romeinse gezag bevrijdde de
boeren op het platteland grotendeels van belastingen, die `tussen een
kwart en een derde van de bruto-opbrengst van het land' opslokte,
`zonder de diverse afpersingspraktijken mee te rekenen waar kleine en
middelgrote grondbezitters onder te lijden hadden'.\footnote{Ibid.,
  p.118.} Door de extreme belastingdruk, die soms met de dood werd
afgedwongen, vertrokken veel mensen en lieten hun grond achter. De
barbaren schaften deze belastingen gelukkig af.

\subsection{\texorpdfstring{\textbf{\emph{Agri
deserti}}}{Agri deserti}}\label{agri-deserti}

De onderdrukking door de overheid nam dusdanig af door de barbaarse
veroveringen dat er voor de armen een mogelijkheid ontstond om eigen
bezit te verwerven en te behouden. Een deel van de \emph{agri deserti},
de boerderijen die door hun eigenaren, vanwege de roofbelastingen in de
nadagen van het Romeinse Rijk waren verlaten, werd opnieuw in productie
genomen. Ondanks de barre omstandigheden van die tijd en het feit dat de
oogsten naar moderne maatstaven belachelijk laag waren, waren de donkere
middeleeuwen een periode van relatieve welvaart voor de kleine boeren in
Europa. Ze genoten toen van een sterke machtspositie die ze pas in de
moderne tijd opnieuw zouden bereiken. Enerzijds waren er minder
arbeidskrachten beschikbaar om het vruchtbare land, waarvan grote
stukken braak waren komen te liggen, te bewerken. Epidemieën, oorlogen
en landverlating door eigenaren die het instortende Romeinse Rijk
ontvluchtten, hadden eerder gecultiveerde gebieden aanzienlijk ontvolkt.
Een ander voordeel voor kleine boeren in de Donkere Middeleeuwen kwam
voort uit de invoering van nieuwe landbouwtechnologie in de zesde eeuw:
de zware ploeg, vaak op wielen gemonteerd. In combinatie met een
verbeterd harnas waarmee meerdere ossen konden worden ingezet, maakte
deze technologie het veel eenvoudiger om beboste gronden in Noord-Europa
te ontginnen.\footnote{Gies, op. cit., p.45.}

Onder dergelijke omstandigheden verdween de markt voor grond vrijwel
volledig. Nieuwe landbouwgrond kon simpelweg worden verkregen door de
bossen te verwijderen en een deel van elk nieuw perceel af te staan aan
de lokale autoriteiten. Door dit proces, in het Engels bekend als
\emph{assarting}, kon de bevolkingsgroei na de val van Rome eeuwenlang
worden opgevangen. \emph{Assarting} werd met name aantrekkelijk in
dunbevolkte noordelijke regio's, nadat hogere temperaturen in de achtste
eeuw de landbouw productiever maakten.

De leiders van de Germaanse stammen die voormalige Romeinse gebieden
veroverden, vestigden zich als grootgrondbezitters. Het grootste deel
van de rest van de bevolking bewerkte kleine percelen, maar onder hele
omstandigheden dan die van het latere feodalisme. De rijkere
landeigenaren, oftewel de meesters, vormden zo'n 7 tot 10 procent van de
bevolking. Het lijkt erop dat vóór het jaar 1000 tweederde van de
dorpelingen in een doorsnee Franse streek vrij landeigenaar
was.\footnote{Bois, op. cit., p.~116.} Ze bezaten ongeveer de helft van
alle gecultiveerde grond.\footnote{Ibid., p.26.} Er waren weinig
lijfeigenen. \emph{Coloni}, ofwel pachtboeren, maakten niet meer dan 5
procent van de bevolking uit. Slavernij bleef bestaan, maar op een veel
kleinere schaal dan in de Romeinse tijd.

De germaanse koninkrijken die Rome opvolgden, werden militair verdedigd
door alle vrije mannen die op oproep van de graaf, de plaatselijke
vertegenwoordiger van de koning, de wapens opnamen. Zelfs "kleine en
middelgrote grondbezitters" moesten zich groeperen en mensen uit hun
midden afvaardigen om in de infanterie te dienen. \footnote{Ibid., p.64.}
In het edict van Pîtres beval Karel de Kale alle mannen die daartoe in
staat waren om te paard ten strijde te trekken. Paus Gregorius II had
een eeuw voordien getracht deze militaire verplichting kracht bij te
zetten door in 732 de consumptie van paardenvlees door de mens te
verbieden.\footnote{Gies, op. cit., p.47.} Toch was er nog weinig
juridisch of statusonderscheid tussen de infanterie van vrije
grondbezitters en de cavalerie. Alle vrije mannen namen deel aan
plaatselijke rechtbanken en konden bij de graaf een verzoek indienen om
geschillen te beslechten, een ambt dat al sinds de late romeinse tijd
bestond. Van adel was er nog geen sprake.

\begin{quote}
`Een sociaal fenomeen dat nieuw was als massaal verschijnsel dook in de
jaren 980 plots op: sociale achteruitgang. De eerste slachtoffers waren
de kleine allodiumbezitters.'\footnote{Bois, op. cit., p.52.} - Guy Bois
\end{quote}

In de loop van de Donkere Middeleeuwen raakten de structuren die de
zelfstandigheid van kleine boeren en vrije grondeigenaars in de
Germaanse koninkrijken hadden gewaarborgd, steeds meer uit balans door
een reeks ontwikkelingen:

\begin{enumerate}
\def\labelenumi{\arabic{enumi}.}
\item
  De bevolkingsaantallen herstelden zich geleidelijk, waardoor de druk
  \textgreater{} op het gebruik van de grond toenam. In de loop van
  enkele eeuwen \textgreater{} werd een groot deel van de meest
  vruchtbare, nog onontgonnen grond \textgreater{} gecultiveerd, een
  groei die zich met name in Noord-Europa \textgreater{} voordeed. De
  toename van het aantal boeren in verhouding tot de \textgreater{}
  beschikbare hoeveelheid grond deed de waarde van de arbeid van
  \textgreater{} elke boer dalen. De meeste vrije stukken land raakten
  door \textgreater{} erfopvolging versnipperd in steeds kleinere
  percelen, omdat het in \textgreater{} de vroege middeleeuwen
  gebruikelijk was om de nalatenschap van \textgreater{} ouders gelijk
  te verdelen over de kinderen. De versnippering van \textgreater{} het
  grondbezit in een tijd van bevolkingsgroei dreef de waarde van
  \textgreater{} grond opnieuw op en leidde tegen het midden van de
  tiende eeuw tot \textgreater{} een heropleving van de actieve handel
  in grond.
\item
  In de laatste decennia van de tiende eeuw daalden de temperaturen
  \textgreater{} plotseling, wat een verwoestend effect had op de
  \textgreater{} landbouwopbrengst. Drie opeenvolgende misoogsten
  leidden tussen \textgreater{} 982 en 984 tot een ernstige hongersnood.
  Na nog een mislukte oogst \textgreater{} in 994 sloeg de hongersnood
  opnieuw toe.\footnote{Ibid., p.150} In 997 werd het \textgreater{}
  probleem van de dalende opbrengsten vervolgens verergerd door een
  \textgreater{} plaag, die kleine familiebedrijven bijzonder hard trof
  omdat de \textgreater{} kleine boeren niet over de middelen beschikten
  om de arbeidskracht \textgreater{} van weggevallen familieleden te
  vervangen. Door deze \textgreater{} opeenstapeling van misoogsten en
  rampen raakten de zelfstandige \textgreater{} boeren aanvankelijk diep
  in de schulden. Toen een herstel van de \textgreater{} opbrengsten
  uitbleef, konden ze hun schulden niet meer aflossen.
\item
  De machtsverhoudingen raakten steeds verder uit balans door het
  \textgreater{} toenemende belang van de zware cavalerie. Frances Gies,
  een \textgreater{} historicus gespecialiseerd in de middeleeuwen,
  beschrijft hoe de \textgreater{} gepantserde cavalerist zich
  ontwikkelde tot de middeleeuwse \textgreater{} ridder:
\end{enumerate}

\begin{quote}
Hoewel een ridder oorspronkelijk een figuur van bescheiden komaf was,
die zich door zijn dure paard en harnas boven de boerenstand verhief,
wist hij zijn maatschappelijke positie geleidelijk te verbeteren tot hij
tot de adelstand behoorde. Ridders bleven weliswaar de laagste rang
binnen de aristocratie, maar het ridderschap kreeg een unieke status,
waardoor de ridderslag een eer werd die zelfs door de hoge adel en
koningshuizen werd begeerd. Die status was voornamelijk het resultaat
van het streven van de kerk om het ridderschap een christelijk karakter
te geven. Dit deed ze door de ridderceremonie te heiligen en een
gedragscode, de ridderlijkheid, te stimuleren. Het was een code die
misschien vaker werd geschonden dan nageleefd, maar die een onmiskenbare
invloed had op het denken en handelen van latere generaties. \footnote{Gies,
  op. cit., p.2.}
\end{quote}

Zoals we in \emph{The Great Reckoning} vertelden, kreeg de bewapende
ridder te paard door de uitvinding van de stijgbeugel een enorm
aanvalsvermogen. Hij kon nu in volle vaart aanvallen zonder uit het
zadel te worden geworpen wanneer hij een doelwit raakte met zijn lans.
De militaire waarde van de zware cavalerie nam verder toe door een
Aziatische uitvinding die in de tiende eeuw in West-Europa haar intrede
deed: het vastgenagelde ijzeren hoefijzer. Dit verbeterde het
uithoudingsvermogen van het paard aanzienlijk.\footnote{Ibid., p.46.}
Ook het gevormde zadel, de spoor en het stangbit droegen bij aan de
toegenomen slagkracht van de ridder. Ze boden meer controle over het
paard en maakten het eenvoudiger om met één hand zware wapens te
gebruiken.\footnote{Ibidem, blz. 56-57.} Deze op het eerste gezicht
kleine technologische verbeteringen maakten kleine boeren op militair
vlak vrijwel irrelevant, omdat ze de middelen niet hadden om een
strijdros te onderhouden of zichzelf goed te bewapenen. De grote paarden
die speciaal voor de strijd werden gefokt, maar goedkoper waren en
bekend stonden als `destriers', waren evenveel waard als vier ossen of
veertig schapen. De duurdere oorlogspaarden kostten tien ossen of
honderd schapen. Ook een harnas kostte een bedrag dat geen enkele kleine
landeigenaar zich kon permitteren, namelijk het equivalent van zestig
schapen.\footnote{Ibidem, blz. 58.}

\begin{enumerate}
\def\labelenumi{\arabic{enumi}.}
\setcounter{enumi}{3}
\tightlist
\item
  Ook het koudere weer, de mislukte oogsten, de hongersnoden en de
  \textgreater{} plagen in de aanloop naar het jaar 1000 beïnvloedden
  het gedrag \textgreater{} van de mensen. Velen waren ervan overtuigd
  dat het einde van de \textgreater{} wereld of de wederkomst van
  Christus voor de deur stond. Vrome of \textgreater{} angstige
  landeigenaren, groot en klein, schonken hun grond aan de
  \textgreater{} kerk in voorbereiding op de apocalyps.
\end{enumerate}

\subsection{\texorpdfstring{\textbf{`Alleen een arme man verkoopt
grond'}}{`Alleen een arme man verkoopt grond'}}\label{alleen-een-arme-man-verkoopt-grond}

De onrustige omstandigheden aan het einde van de tiende eeuw legden de
basis voor de feodale revolutie. Door opeenvolgende misoogsten en rampen
raakten de zelfstandige boeren diep in de schulden. Toen de oogsten zich
niet herstelden, kwamen de vrije boeren in een uitzichtloze situatie
terecht. Markten leggen altijd de grootste druk op de zwakste partijen.
Dat is zelfs een van hun deugden; ze bevorderen efficiëntie door
eigendom van zwakkere eigenaren af te nemen. Maar in het Europa van de
late tiende eeuw was zelfvoorzieningslandbouw vrijwel het enige beroep.
Families die hun grond verloren, raakten hun enige middel om te
overleven kwijt. Geconfronteerd met dit onaantrekkelijke vooruitzicht,
besloten veel, zo niet de meeste, vrije boeren tijdens de feodale
revolutie hun akkers weg te geven. In de woorden van Guy Bois: `De enige
zekere manier voor een boer om het land dat hij bewerkte te behouden,
was om het eigendom ervan af te staan aan de kerk, zodat hij het
vruchtgebruik kon behouden.'\footnote{Bois, op. cit., blz. 87.} Anderen
stonden al hun grond, of een deel ervan, af aan rijkere boeren in wie ze
vertrouwen hadden, zoals vriendelijke buren of familieleden.

Deze eigendomsoverdrachten vonden plaats onder de voorwaarde dat de
boer, zijn familie en zijn afstammelingen op de akkers mochten blijven
werken. De arme boeren konden tevens rekenen op de wederzijdse steun van
de meer vermogende landeigenaren, nu de `edelen', die zich een paard en
een harnas konden veroorloven en zo de uitgebreide landgoederen
bescherming boden. Vanuit het oogpunt van de nieuwe lijfeigene kan een
dergelijke overeenkomst worden gezien als een tussenstation tussen het
behoud van economisch eigendom en een gedwongen verkoop. Vaker wel dan
niet was het een aanbod dat hij niet kon weigeren.

De dalende productiviteit bracht de arme boeren niet alleen in een
wanhopig economisch dilemma, maar ontketende ook een golf van
roofzuchtig geweld die de eigendomszekerheid ondermijnde. Wie niet over
de middelen beschikte om een deel van de beschikbare, maar ontoereikende
voorraad paarden en voer te bemachtigen, merkte plotseling dat hij en
zijn bezit niet langer veilig waren. Om hun dilemma in hedendaagse
termen te vatten: het was alsof je vandaag de dag gedwongen werd jezelf
te bewapenen met een nieuw type wapen, maar de kosten daarvan \$100.000
bedroegen. Als je die prijs niet kon betalen, kon je enkel hopen op
genade van degenen die dat wel konden.

Binnen een paar jaar stortte het vermogen van de koning en de
rechtbanken om de orde te handhaven volledig in.\footnote{Ibidem. Hoewel
  de precieze opeenvolging van gebeurtenissen tijdens de feodale
  revolutie door een gebrek aan bronnen moeilijk te reconstrueren is,
  lijkt de these van Guy Bois ons in grote lijnen juist. Ze is niet
  alleen op zichzelf plausibel, maar biedt ook een verklaring voor
  anderszins afwijkende feiten én sluit bovendien aan bij onze
  theorieën.} Iedereen met een harnas en een paard kon nu zijn eigen wet
stellen. Wat volgde leek op een middeleeuwse \emph{Blade Runner}, een
roerige tijd vol geweld en rooftochten, terwijl de officiële
machthebbers machteloos toekeken. Plunderingen en aanvallen door
gewapende ridders ontwrichtten het platteland. Het is echter helemaal
niet vanzelfsprekend dat alle slachtoffers van deze plunderingen arm
waren. Integendeel, de oudere, fysiek zwakkere of slecht voorbereide
grootgrondbezitters vormden juist aantrekkelijke doelwitten. Bij hen
viel meer te stelen.

Het was geen toeval dat dit gebeurde op het moment dat kouder weer,
hongersnood en de pest de middelen schaars maakten. De megapolitieke
omstandigheden die de ineenstorting van het gezag in de hand werkten,
bestonden al een tijdje. Pas toen er een crisis ontstond, kwam hun
vermogen om de machtsverhoudingen te veranderen tot uiting. Misoogsten
en hongersnoden lijken precies die rol te hebben vervuld. Hoewel de
exacte opeenvolging van gebeurtenissen moeilijk te reconstrueren is,
lijkt het erop dat de plunderingen, althans gedeeltelijk, werden
veroorzaakt door de wanhopige omstandigheden. Zodra het geweld was
losgebarsten, bleek dat niemand de macht kon mobiliseren om het te
stoppen. De overgrote meerderheid van slecht bewapende boeren kon zeker
weinig uitrichten. Een enkele geharnaste ridder te paard was al genoeg
om tientallen boeren te overmeesteren. Vrije boeren konden, net als de
gevestigde autoriteiten, de koningen en hun graven, niets beginnen tegen
de landroof door gewapende strijders.

\subsection{\texorpdfstring{\textbf{`De
godsvrede'}}{`De godsvrede'}}\label{de-godsvrede}

In deze wanhopige omstandigheden droeg de Kerk bij aan het ontstaan van
het feodalisme door een wapenstilstand te proberen te bewerkstelligen op
het gewelddadige platteland. Historicus Guy Bois beschreef de situatie
als volgt: `De onmacht van de politieke autoriteiten was zodanig dat de
Kerk hun rol overnam in een poging om de orde te herstellen, in de
beweging die bekendstaat als 'de godsvrede'. `Vredesconcilies'
vaardigden een reeks verboden uit, op straffe van banvloeken, en in
grote `vredesvergaderingen' legden krijgers hun eden af. De beweging
ontstond in Zuid-Frankrijk (Concilie van Charroux in 989, Concilie van
Narbonne in 990) en verspreidde zich geleidelijk\ldots``\footnote{Ibidem,
  blz. 136.}

De overeenkomst die de Kerk sloot, hield de erkenning in van de
heerschappij van gewapende ridders in lokale gemeenschappen, in ruil
voor het staken of matigen van het geweld en de plunderingen. Na de
toename van het geweld aan het einde van de tiende eeuw verschenen
titels als `nobilis' en `miles' in eigendomsakten, als teken van gezag.
Daarmee werd de adel als aparte sociale klasse geboren. In oudere akten,
opgesteld voor exact dezelfde personen, ontbraken zulke titels nog
volledig.\footnote{Ibid., p.~57 en passim.}

Door de dalende productiviteit en de economische onzekerheid van kleine
grondbezitters leidde de megapolitieke macht van gewapende ridders
onvermijdelijk tot grondbezit op basis van feodale leenverhoudingen.
Tegen het einde van het eerste kwart van de elfde eeuw waren de vrije
boeren grotendeels verdwenen. Hun allodiale bezittingen waren geslonken
tot een fractie van hun vroegere omvang en werden nog maar een deel van
de tijd bewerkt. De kleine boeren of hun afstammelingen werden horigen
die de meeste tijd al werkend doorbrachten op de landgoederen van
feodale heren, zowel wereldlijke als kerkelijke.

De teloorgang van de openbare orde die met de feodale revolutie gepaard
ging, leidde tot gedragsveranderingen die het feodalisme versterkten.
Daartoe behoorde een explosieve toename van de kastelenbouw. Kastelen
doken voor het eerst op in Noordwest-Europa als primitieve houten
bouwwerken in de nasleep van de invallen van de Vikingen in de negende
eeuw. Oorspronkelijk waren het commandoposten voor Karolingische
ambtenaren, maar na de feodale revolutie werden ze erfelijk bezit. Deze
vroege bolwerken waren veel primitiever dan hun latere versies, maar
desondanks moeilijk aan te vallen. Eenmaal gebouwd, konden kastelen
slechts met de grootste moeite met de grond gelijkgemaakt worden.
Naarmate kastelen het platteland begonnen te domineren, verloor het
centrale gezag zijn greep: de koning en zijn graven waren nauwelijks nog
in staat de macht van lokale heren te doorbreken.

\subsection{\texorpdfstring{\textbf{De bijdrage van de kerk aan de
productiviteit}}{De bijdrage van de kerk aan de productiviteit}}\label{de-bijdrage-van-de-kerk-aan-de-productiviteit}

Feodalisme ontstond als antwoord op het wegvallen van orde tijdens een
tijd van beperkte productiviteit in de landbouwsamenleving. In het begin
van dit systeem speelde de Kerk een belangrijke en economisch
productieve rol. Enkele bijdragen van de Kerk:

\begin{enumerate}
\def\labelenumi{\arabic{enumi}.}
\tightlist
\item
  In een omgeving waar militaire macht gedecentraliseerd was, bevond
  \textgreater{} de kerk zich in een unieke positie om de vrede te
  bewaren en \textgreater{} orderegels te ontwikkelen die versnipperde,
  lokale \textgreater{} soevereiniteiten overstegen. Dit was een taak
  waar geen enkele \textgreater{} seculiere macht tegen opgewassen was.
  De observaties van de grote \textgreater{} religieuze autoriteit A. R.
  Radcliffe-Brown zijn hier rechtstreeks \textgreater{} van toepassing.
  Hij wees erop dat ``de sociale functie van een \textgreater{} religie
  onafhankelijk is van haar waarheid of onwaarheid". Zelfs
  \textgreater{} religies die''absurd en weerzinwekkend zijn, zoals die
  van sommige \textgreater{} primitieve stammen, kunnen belangrijke en
  effectieve onderdelen \textgreater{} van het maatschappelijke
  raderwerk zijn''.\footnote{A. R. Radcliffe-Brown, `Religie en
    samenleving', in \emph{Structuur en functie in de primitieve
    maatschappij} (Londen: Cohen \& West, 1952), p.~153-177.} Dit was
  zeker het \textgreater{} geval voor de kerk in de vroege stadia van
  het feodalisme. Ze \textgreater{} hielp, zoals alleen een religie dat
  kon, bij het opstellen van \textgreater{} regels die mensen in staat
  stelden om perverse prikkels en \textgreater{} gedragsdilemma's te
  overwinnen. Sommige hiervan waren morele \textgreater{} dilemma's die
  eigen zijn aan het menselijk bestaan. Maar andere \textgreater{} waren
  lokale dilemma's, die uniek waren voor de heersende \textgreater{}
  megapolitieke omstandigheden. In de laatste jaren van de tiende
  \textgreater{} eeuw speelde de middeleeuwse kerk een bijzondere rol
  bij het \textgreater{} herstellen van de orde op het platteland. Door
  religieuze en \textgreater{} ceremoniële steun te verlenen aan lokale
  autoriteiten, verlaagde \textgreater{} de kerk de kosten om op zijn
  minst zwakke, lokale \textgreater{} geweldsmonopolies te vestigen.
  Door op deze manier bij te dragen \textgreater{} aan de orde, schiep
  de kerk de voorwaarden die uiteindelijk \textgreater{} leidden tot
  stabielere machtsverhoudingen.
\end{enumerate}

De kerk bleef nog lange tijd een rol spelen in het beteugelen van
privéoorlogen en buitensporig geweld dat de wereldlijke autoriteiten
anders niet konden indammen. Het relatief grote belang van de Kerk ten
opzichte van de wereldlijke autoriteiten blijkt uit het feit dat tegen
de elfde eeuw de parochie de voornaamste bestuurlijke eenheid werd in
grote delen van West-Europa, en niet langer de civiele indelingen zoals
de \emph{ager} en \emph{pagus} (de omgeving rond een stadje), die sinds
de Romeinse tijd tot in de Donkere Middeleeuwen hadden
standgehouden..\footnote{Bois, op. cit., p.~36.}

\begin{enumerate}
\def\labelenumi{\arabic{enumi}.}
\setcounter{enumi}{1}
\item
  Het was vooral de kerk die zorg droeg voor het behoud en de
  \textgreater{} verspreiding van technische kennis en informatie. De
  kerk \textgreater{} financierde universiteiten en zorgde voor het
  weinige onderwijs \textgreater{} dat de middeleeuwse maatschappij
  genoot. De kerk bood ook een \textgreater{} mechanisme voor het
  vermenigvuldigen van boeken en manuscripten, \textgreater{} waaronder
  vrijwel alle toenmalige informatie over landbouw en \textgreater{}
  veeteelt. De scriptoria van de benedictijnenkloosters kunnen
  \textgreater{} worden gezien als een alternatieve technologie voor de
  drukpers, \textgreater{} die toen nog niet bestond. Hoe kostbaar en
  inefficiënt de \textgreater{} scriptoria ook waren, ze vormden in de
  feodale periode praktisch \textgreater{} het enige mechanisme om
  geschreven kennis te vermenigvuldigen en \textgreater{} te bewaren.
\item
  Mede doordat haar boerderijbeheerders geletterd waren, droeg de Kerk
  \textgreater{} sterk bij aan het verhogen van de
  landbouwproductiviteit in \textgreater{} Europa, vooral in de vroege
  fase van het feodalisme. Vóór de \textgreater{} dertiende eeuw waren
  de beheerders van wereldlijke heren vrijwel \textgreater{} allemaal
  analfabeten, die hun administratie bijhielden met een \textgreater{}
  ingenieus stelsel van merktekens. Al waren deze boeren nog zo
  \textgreater{} slim, ze hadden niet de mogelijkheid om te profiteren
  van \textgreater{} verbeteringen in productiemethoden die ze niet zelf
  bedachten of \textgreater{} met eigen ogen zagen. De kerk was daarom
  onmisbaar voor het \textgreater{} verbeteren van de kwaliteit van
  graan, fruit en fokvee. Omdat haar \textgreater{} bezittingen over
  heel Europa verspreid lagen, kon de Kerk \textgreater{} hoogwaardig
  zaaigoed en fokmateriaal inzetten waar de productie \textgreater{}
  achterbleef. De vraag naar wijn voor de misdiensten in \textgreater{}
  Noord-Europa spoorde monniken aan om te experimenteren met
  \textgreater{} druivensoorten die beter bestand waren tegen koudere
  klimaten. De \textgreater{} kerk hielp ook op andere manieren de
  productiviteit van de \textgreater{} middeleeuwse landbouw te
  verhogen. Kleine, slecht bruikbare lapjes \textgreater{} grond die de
  Kerk in de feodale periode verwierf, werden vaak \textgreater{}
  opnieuw ingedeeld zodat ze makkelijker te bewerken werden. De kerk
  \textgreater{} voorzag ook in de aanvullende diensten die kleine
  \textgreater{} boerengemeenschappen nodig hadden. In veel gebieden
  maalden \textgreater{} kerkelijke molens graan tot meel.
\item
  De kerk nam veel taken op zich die tegenwoordig onder de overheid
  \textgreater{} vallen, waaronder het voorzien in openbare
  infrastructuur. Dit is \textgreater{} een voorbeeld van hoe de kerk,
  in een tijdperk van versnipperd \textgreater{} gezag, hielp bij het
  oplossen van wat economen `dilemma's van \textgreater{} publieke
  goederen' noemen. Specifieke kloosterordes uit de vroege
  \textgreater{} middeleeuwen legden zich toe op civieltechnische taken,
  zoals het \textgreater{} aanleggen van wegen, het herbouwen van
  ingestorte bruggen en het \textgreater{} herstellen van vervallen
  Romeinse aquaducten. Ze ontgonnen ook \textgreater{} land, bouwden
  dammen en legden moerassen droog. Een nieuwe \textgreater{}
  kloosterorde, de kartuizers, groef de eerste `artesische' put in
  \textgreater{} Artois, Frankrijk. Met behulp van klopboren groeven ze
  een klein \textgreater{} gat dat diep genoeg was om een put te creëren
  waarvoor geen pomp \textgreater{} nodig was.\footnote{Gies, op. cit.,
    p.~112.} De cisterciënzerorde nam de bouw en het onderhoud
  \textgreater{} van kwetsbare zeeweringen en dijken in de Lage Landen
  op zich. \textgreater{} Boeren droegen land over aan
  cisterciënzerkloosters en pachtten \textgreater{} het daarna terug,
  terwijl de monniken de volledige \textgreater{} verantwoordelijkheid
  voor het onderhoud en de herstelwerkzaamheden \textgreater{} op zich
  namen. Cisterciënzers namen ook het voortouw in de \textgreater{}
  ontwikkeling van wateraangedreven machines, die voor uiteenlopende
  \textgreater{} doeleinden werden gebruikt, zoals `stampen, heffen,
  malen en \textgreater{} persen'.\footnote{Ibid., p.~114.} Het klooster
  van Clairvaux groef een kanaal van twee \textgreater{} mijl lang vanaf
  de rivier de Aube.\footnote{Ibid., p.117.} De kerk faciliteerde ook
  \textgreater{} het aanleggen van nieuwe wegen en bruggen op plaatsen
  waar \textgreater{} bevolkingscentra zich buiten het bereik van de
  oude Romeinse \textgreater{} garnizoenswegen hadden verplaatst.
  Bisschoppen verleenden aflaten \textgreater{} aan lokale heren die
  rivierovergangen bouwden of herstelden en \textgreater{} herbergen
  voor reizigers onderhielden. Een monnikenorde, opgericht
  \textgreater{} door de heilige Bénézet, de Frères Pontifes of
  `Broeders van de \textgreater{} Brug', bouwde een aantal van de
  langste bruggen die destijds \textgreater{} bestonden, waaronder de
  Pont d'Avignon, een gigantische \textgreater{} constructie met twintig
  bogen over de Rhône, met aan één uiteinde \textgreater{} een kapel die
  ook als tolhuis diende. Zelfs de London Bridge, die \textgreater{} tot
  de negentiende eeuw standhield, werd gebouwd door een kapelaan
  \textgreater{} en werd deels gefinancierd met een bijdrage van duizend
  mark van \textgreater{} de pauselijke legaat.\footnote{De details over
    bruggen en infrastructuur zijn voornamelijk afkomstig uit ibid.,
    pp.148-54.}
\item
  De kerk hielp ook een complexere markt tot ontwikkeling te brengen.
  \textgreater{} De bouw van kathedralen, bijvoorbeeld, is van een
  andere orde dan \textgreater{} openbare infrastructuur zoals bruggen
  en aquaducten. In principe \textgreater{} werden kerkgebouwen alleen
  voor religieuze diensten gebruikt en \textgreater{} niet als
  verkeersaders voor de handel. Toch mag niet worden \textgreater{}
  vergeten dat de bouw van kerken en kathedralen hielp om markten
  \textgreater{} voor tal van ambachtelijke en technische vaardigheden
  te creëren \textgreater{} en te versterken. Net zoals de militaire
  uitgaven van de \textgreater{} natiestaat tijdens de Koude Oorlog
  onbedoeld de voedingsbodem voor \textgreater{} het internet vormden,
  zo leidde de bouw van middeleeuwse \textgreater{} kathedralen tot
  andere neveneffecten, zoals de bevordering van de \textgreater{}
  handel. De kerk was de voornaamste klant voor de bouwsector en
  \textgreater{} ambachtslieden. Kerkelijke aankopen van zilver voor de
  \textgreater{} communiediensten, kandelaars en decoratieve
  kunstwerken, creëerden \textgreater{} een markt voor luxegoederen die
  anders niet zou hebben bestaan.
\end{enumerate}

Op vele manieren hielp de kerk de hevigheid van het geweld te temperen
dat tijdens en na de `feodale revolutie' door gewapende ridders werd
ontketend. Vooral in de eerste eeuwen van het feodalisme leverde de kerk
een aanzienlijke bijdrage aan het verbeteren van de productiviteit van
de landbouweconomie. Het was een onmisbare instelling, die goed aansloot
bij de behoeften van de agrarische samenleving aan het einde van de
donkere middeleeuwen.

\subsection{\texorpdfstring{\textbf{Kwetsbaarheid voor
geweld}}{Kwetsbaarheid voor geweld}}\label{kwetsbaarheid-voor-geweld}

Gedurende ``dertig à veertig jaar van gewelddadige onrust'' voltrok zich
rond het jaar 1000 de feodale revolutie\footnote{Bois, op. cit., p.136.}.
Dit was net als de val van Rome vijf eeuwen eerder een uniek gebeuren,
voortgekomen uit een complex samenspel van invloeden. Toch weerspiegelt
de triomf van mali homines (slechte lieden) en de onderdrukking die ze
teweegbrachten in één opzicht perfect de fundamentele kwetsbaarheid voor
geweld van de landbouwsamenleving. Waar de jager-verzamelaarsfase van
het menselijke bestaan relatief vreedzaam was, betekende de overgang
naar landbouw een radicale sprong in gestructureerd geweld en
overheersing.

Vanaf het allereerste begin weerspiegelde zich dit in de militant
ingestelde culturen van de landbouwvolkeren. De goden van de vroege
landbouwsamenlevingen waren goden van regen en overstroming; hun
functies weerspiegelden de grote zorg van die samenlevingen voor de
factoren die de oogstopbrengsten bepaalden. De brenger van regen of
water was vaak ook de oorlogsgod, en werd aangeroepen door de eerste
koningen, die vooral als krijgsheren optraden.\footnote{Zie Norman Cohn,
  \emph{Cosmos, Chaos, and the World to Come: The Ancient Roots of the
  Apocalyptic Faith} (New Haven: Yale University Press, 1993), hfst.
  I-3, met name p.60.}

De nauwe band tussen landbouw en oorlogsvoering vond zijn weerslag in de
religieuze verbeelding van de mensen wier levens werden getransformeerd
door de innovaties van de landbouwrevolutie. De verdrijving uit het hof
van Eden kan worden gezien als een beeldend verslag van de overgang van
een jager-verzamelaarssamenleving naar een landbouwsamenleving; een
overgang van een vrij leven, waarin voedsel met weinig moeite uit de
overvloed van de natuur werd geplukt, naar een leven van hard labeur.

\section{\texorpdfstring{\textbf{Het verloren
paradijs}}{Het verloren paradijs}}\label{het-verloren-paradijs}

Landbouw stuurde de mensheid een volkomen nieuwe richting uit. De eerste
boeren zaaiden letterlijk de kiem van de beschaving. Uit hun gezwoeg
kwamen steden voort, legers, rekenkunde, astronomie, kerkers, wijn en
whisky, het geschreven woord, koningen, slavernij en oorlog. Maar
ondanks al het drama dat de landbouw het leven zou brengen, lijkt de
stap weg van de oereconomie vanaf het begin enorm impopulair te zijn
geweest. Daarvan getuigt het verhaal uit het boek Genesis, dat vertelt
over de verdrijving uit het paradijs. De bijbelse parabel van het hof
van Eden is een nostalgische herinnering aan het onbezorgde leven van de
jager-verzamelaar in de wildernis. Geleerden menen dat het woord `Eden'
afkomstig is van een Soemerisch woord voor `wildernis'.\footnote{Bruce
  Ni. Metzger en Michael D. Coogan, red., \emph{The Oxford Companion to
  the Bible} (Oxford: Oxford University Press, 1993), p.178.}

De overgang van een vrij en dunbevolkt leven in de wildernis naar een
sedentair bestaan in een boerendorp werd diep betreurd, een spijt die
niet alleen in de Bijbel tot uiting komt, maar ook in de aanhoudende
weerzin van de mens om `s ochtends op te staan en aan het werk te gaan.
Zoals Stephen Boyden schreef in \emph{Western Civilization in Biological
Perspective}, was de nieuwe levenswijze die met de landbouw gepaard ging
'evodeviant'.\footnote{Boyden, op. cit., p.~118.} Vóór de komst van de
landbouw leefden duizenden generaties mensen zoals Adam in het hof van
Eden, op uitnodiging van zijn Schepper: `Van alle bomen in het hof moogt
gij vrij eten'. Jagers en verzamelaars hadden geen gewassen te
verbouwen, geen kudde om te hoeden en geen belastingen te betalen. Net
als zwervers trokken jager-verzamelaars waar ze maar heen wilden,
werkten ze weinig en legden ze aan niemand verantwoording af.

Met de komst van de landbouw begon een nieuwe levenswijze, en wel op
veel dwingendere voorwaarden. `Ook zal ze u doornen en distels
voortbrengen, en gij zult het gewas des velds eten. In het zweet uws
aanschijns zult gij brood eten'. Landbouw was hard werken. De
herinnering aan het leven vóór de landbouw was die van een verloren
paradijs.

Meer dan ze zich ooit hadden kunnen voorstellen, creëerden de boeren
nieuwe omstandigheden die de logica van geweld radicaal veranderden. Het
is geen toeval dat het boek Genesis van Kaïn, de eerste moordenaar, `een
landbouwer' maakt. Het maakt inderdaad deel uit van de wonderbaarlijke
profetische kracht van de Bijbel dat het verhaal werd toevertrouwd aan
herders, die maar al te goed begrepen hoe de landbouw geweld in de hand
werkte. In enkele verzen omvat het bijbelse verslag een logica die
duizenden jaren nodig had om zich te ontvouwen. De landbouw was een
broedplaats voor geschillen. De landbouw creëerde op grote schaal
plaatsgebonden kapitaal, wat de opbrengst van geweld verhoogde en de
bescherming van bezittingen erg bemoeilijkte. Voor het eerst maakte de
landbouw zowel misdaad als bestuur rendabel.

\bookmarksetup{startatroot}

\chapter{\texorpdfstring{\textbf{De laatste dagen van de
politiek}}{De laatste dagen van de politiek}}\label{de-laatste-dagen-van-de-politiek}

\emph{Parallellen tussen het seniele verval van de heilige moederkerk en
de bemoeizuchtige staat}

\begin{quote}
`Ik geloof, en hoop, dat politiek en economie in de toekomst minder
centraal zullen staan dan vroeger. Er komt een tijd waarop de meeste van
onze huidige debatten over deze onderwerpen volstrekt banaal of
betekenisloos zullen lijken, net als de theologische discussies waaraan
de scherpste geesten van de middeleeuwen hun energie verspilden.'{[}1{]}
- ARTHUR C. CLARKE
\end{quote}

Spreken over het naderende einde van de politiek lijkt belachelijk of
optimistisch, afhankelijk van je instelling. Toch is dat waarschijnlijk
wat de Informatierevolutie met zich meebrengt. Voor lezers die zijn
opgegroeid in een eeuw doordrenkt van politiek, lijkt het idee dat het
leven zonder politiek kan verlopen misschien een fantasie. Het is net
alsof je beweert dat iemand kan leven door simpelweg voedingsstoffen uit
de lucht op te nemen. Toch is politiek in de moderne zin, als bezigheid
gericht op het beheersen en rationaliseren van staatsmacht, grotendeels
een moderne uitvinding. Wij geloven dat het zal eindigen met de moderne
wereld, net zoals het kluwen van feodale plichten dat mensen in de
middeleeuwen bezighield, eindigde met de middeleeuwen . Tijdens de
feodale periode, zoals historicus Martin van Creveld opmerkt, ``bestond
politiek niet (het concept moest nog worden uitgevonden en dateert pas
uit de zestiende eeuw).''{[}2{]}

Het idee dat politiek, zoals wij dat nu kennen, vóór de moderne tijd
niet bestond, lijkt misschien verrassend, zeker omdat Aristoteles in de
tijd van Alexander de Grote een essay met die titel schreef. Maar kijk
goed. Woorden in oude teksten zijn niet per se hedendaagse begrippen.
Aristoteles schreef ook een essay getiteld Sofistische Weerleggingen --
een term die tegenwoordig net zo betekenisloos is als "politiek" dat was
in de middeleeuwen . Het woord werd simpelweg niet gebruikt. De eerste
bekende verschijning in het Engels dateert uit 1529. Zelfs toen leek
"politics" een negatieve bijklank te hebben, afgeleid van het Oudfranse
woord \emph{politique}, dat werd gebruikt om ``opportunisten en mensen
zonder ruggengraat'' te beschrijven.

Het duurde bijna tweeduizend jaar voordat Aristoteles' sluimerende
concept de betekenis kreeg die we nu kennen. Waarom? Voordat de moderne
wereld het woord van Aristoteles zinvol kon gebruiken, waren er
megapolitieke omstandigheden nodig die de opbrengsten van geweld
drastisch verhoogden. De Buskruitrevolutie, die we analyseerden in
\emph{The Great Reckoning}, zorgde precies daarvoor. De opbrengsten van
geweld bereikten ongekende hoogten. Daardoor werd de vraag wie de staat
controleerde belangrijker dan ooit tevoren. Het was dan ook logisch en
onvermijdelijk dat politiek ontstond uit de strijd om deze vergrote
machtsmiddelen.

Politiek begon vijf eeuwen geleden met de vroege stadia van het
industrialisme. Nu sterft het. Een afkeer van politiek en politici
verspreidt zich over de wereld. Je ziet het in het nieuws en de
speculaties rond de verborgen details van Whitewater en de nauwelijks
verhulde moord op Vincent Foster. Je ziet het in talloze andere
schandalen rond president Bill Clinton. Je ziet het in berichten over de
verduistering van geld door vooraanstaande congresleden via het
postkantoor van het Huis van Afgevaardigden. Je ziet het in schandalen
die tot ontslagen leidden in de kring rond John Major, en soortgelijke
schandalen in Frankrijk, waarbij twee recente premiers betrokken waren:
Édouard Balladur en Alain Juppé. Nog grotere schandalen kwamen aan het
licht in Italië, waar de zevenvoudige premier Giulio Andreotti
terechtstond op beschuldigingen van onder meer banden met de maffia en
het bevel tot de moord op onderzoeksjournalist Mino Pecorelli. Nog
andere schandalen bezoedelden de reputatie van de Spaanse premier Felipe
González. In Japan verloren vier premiers hun functie in de eerste vijf
jaar van de jaren 1990 door corruptiebeschuldigingen. Het Canadese
ministerie van Justitie stelde in een brief aan de Zwitserse
autoriteiten dat voormalig premier Brian Mulroney smeergeld had
ontvangen bij de verkoop van Airbus-vliegtuigen aan Air Canada ter
waarde van 1,8 miljard Canadese dollar. {[}3{]} Willy Claes,
secretaris-generaal van de NAVO, moest aftreden onder verdenking van
corruptie. Zelfs in Zweden moest Mona Sahlin, vicepremier en beoogd
premier, aftreden wegens beschuldigingen dat ze overheidskredietkaarten
gebruikte voor de aankoop van luiers en andere huishoudelijke artikelen.
Vrijwel overal waar je kijkt in landen met volwassen verzorgingsstaten
die ooit bekend stonden voor hun goede bestuur, verafschuwen mensen hun
politieke leiders .

\subsection{\texorpdfstring{\textbf{Minachting als leidende
indicator}}{Minachting als leidende indicator}}\label{minachting-als-leidende-indicator}

Morele verontwaardiging over corrupte leiders is geen op zichzelf staand
historisch verschijnsel, maar een veelvoorkomende voorbode van
verandering. Het gebeurt steeds weer wanneer een tijdperk overgaat in
een ander. Zodra technologische vooruitgang de kloof vergroot tussen
oude instituties en nieuwe economische krachten, verschuiven morele
maatstaven, en beginnen mensen de leiders met toenemende minachting te
behandelen. Deze wijdverbreide afkeer wordt zichtbaar ruim voordat er
een samenhangende nieuwe ideologie van verandering ontstaat. Terwijl wij
dit schrijven, is er nog weinig sprake van een duidelijke afwijzing van
politiek. Dat komt later. De meeste van je tijdgenoten hebben zich nog
niet gerealiseerd dat een leven zonder politiek mogelijk is. Wat we in
de laatste jaren van de twintigste eeuw zien, is een niet-gearticuleerde
minachting.

Aan het einde van de vijftiende eeuw speelde zich iets soortgelijks af,
maar toen lag de focus niet op de politiek, maar op de religie. Ondanks
de overtuiging over `de heiligheid van het priesterlijke ambt'
behandelden mensen zowel de hoge als de lage geestelijkheid met grote
minachting, vergelijkbaar met hoe we vandaag de dag tegen politici en
bureaucraten aankijken. Er was een algemene overtuiging dat de hoge
geestelijkheid corrupt, werelds en omkoopbaar was, en dat was niet
zonder reden. Verschillende pausen uit die periode brachten openlijk
buitenechtelijke kinderen voort. De lage geestelijkheid stond nog lager
aangeschreven, omdat zij in overvloed aanwezig was in zowel het
platteland als in de steden. Ze bedelden voor aalmoezen en boden vaak
Gods genade en vergeving van zonden te koop aan voor contant geld.

Onder de `korst van oppervlakkige vroomheid'{[}4{]} schuilde een corrupt
en steeds disfunctioneler wordend systeem. Veel mensen verloren al lang
hun respect voor degenen die het bestuurden, nog voordat iemand de moed
had om op te merken dat het systeem niet meer functioneerde. Een leven
dat volledig doordrenkt was van religie, waarin men geen onderscheid
maakte tussen het geestelijke en het wereldse, had al zijn mogelijkheden
opgebruikt. Het einde daarvan was dan ook onvermijdelijk, ruim lang
voordat Luther zijn 95 stellingen op de kerkdeur van Wittenberg
spijkerde.

\section{\texorpdfstring{\textbf{Een seculiere
reformatie}}{Een seculiere reformatie}}\label{een-seculiere-reformatie}

Wij zijn ervan overtuigd dat de reactie op de verzadigingspolitiek een
vergelijkbaar traject volgt.

De val van de Sovjet-Unie en de afwijzing van het socialisme maken deel
uit van een alomvattend patroon van depolitisering dat de wereld
teistert. Dit blijkt vooral uit de groeiende minachting voor de
regeringsleiders wereldwijd. Dit gevoel wordt slechts deels gedreven
door het besef dat ze corrupt zijn en geneigd om ``aflaten'' te verkopen
voor politieke problemen, in ruil voor campagnegiften of andere
voordelen om hun persoonlijke financiën te spekken.

\subsection{De reactie tegen politici wordt ook gedreven door het
groeiende besef dat veel van wat zij tegen hoge kosten doen, zinloos is,
net zoals het organiseren van nog een pelgrimstocht van boetelingen die
blootsvoets door de sneeuw lopen, of het stichten van weer een nieuwe
orde bedelmonniken aan het einde van de vijftiende eeuw. Het droeg
weinig bij aan het verbeteren van de productiviteit of het verlichten
van de druk op de
levensstandaard.}\label{de-reactie-tegen-politici-wordt-ook-gedreven-door-het-groeiende-besef-dat-veel-van-wat-zij-tegen-hoge-kosten-doen-zinloos-is-net-zoals-het-organiseren-van-nog-een-pelgrimstocht-van-boetelingen-die-blootsvoets-door-de-sneeuw-lopen-of-het-stichten-van-weer-een-nieuwe-orde-bedelmonniken-aan-het-einde-van-de-vijftiende-eeuw.-het-droeg-weinig-bij-aan-het-verbeteren-van-de-productiviteit-of-het-verlichten-van-de-druk-op-de-levensstandaard.}

\subsection{\texorpdfstring{\textbf{De laatste dagen van de heilige
moederkerk}}{De laatste dagen van de heilige moederkerk}}\label{de-laatste-dagen-van-de-heilige-moederkerk}

Aan het einde van de middeleeuwen was de monolithische kerk als
instituut verouderd en contraproductief geworden, een duidelijke
verandering ten opzichte van haar positieve economische bijdrage vijf
eeuwen eerder. Zoals we in het vorige hoofdstuk bespraken, speelde de
kerk aan het einde van de tiende eeuw een leidende rol bij het
herstellen van orde en het bevorderen van economisch herstel na de
anarchie aan het einde van de donkere middeleeuwen. Destijds was de kerk
onmisbaar voor het voortbestaan van grote aantallen kleine pachters en
horigen, die het merendeel van de West-Europese bevolking vormden. Aan
het einde van de vijftiende eeuw was de kerk een zware last voor de
productiviteit geworden. De lasten die ze de bevolking oplegde, zetten
de levensstandaard onder druk.

Vrijwel hetzelfde kan vandaag de dag gezegd worden over de natiestaat.
Het ontstaan ervan was een noodzakelijke adaptatie aan de nieuwe
megapolitieke omstandigheden die vijf eeuwen geleden ontstonden door de
buskruitrevolutie. De natiestaat vergrootte de reikwijdte van markten en
verving gefragmenteerde lokale autoriteiten op een moment dat grotere
handelsgebieden hoge opbrengsten voortbrachten. Het feit dat kooplieden
bijna overal in Europa spontaan de zijde van de monarch kozen toen deze
zijn macht probeerde te consolideren, is op zich al veelzeggend bewijs
dat de natiestaat in zijn vroege vorm gunstig was voor de handel. Het
hielp de lasten op de handel verlichten die werden opgelegd door feodale
heren en lokale machthebbers.

In een wereld waarin de opbrengsten van geweld hoog en stijgend waren,
was de natiestaat een nuttig instituut. Maar vijf eeuwen later, nu dit
millennium ten einde loopt, zijn de megapolitieke omstandigheden
veranderd. De opbrengsten van geweld dalen, en de natiestaat is, net als
de kerk aan het einde van de middeleeuwen, een anachronisme geworden dat
groei en productiviteit belemmert.

Net zoals de middeleeuwse kerk destijds, heeft de hedendaagse natiestaat
haar mogelijkheden uitgeput. Ze verkeert in faillissement en is
uitgegroeid tot een seniel systeem. Vijf eeuwen lang domineerde ze als
de overheersende vorm van sociale organisatie, maar nu de omstandigheden
die tot haar ontstaan leidden niet langer bestaan, is ze rijp voor de
val, en die is onvermijdelijk. Technologie ontketent een revolutie in de
uitoefening van macht die de natiestaat zonder twijfel zal vernietigen,
net zoals buskruitwapens en de drukpers ooit het middeleeuwse
kerkmonopolie deden instorten.

Als onze redenering klopt, zal de natiestaat worden vervangen door
nieuwe vormen van soevereiniteit, sommige uniek in de geschiedenis,
andere zullen doen denken aan de stadstaten en middeleeuwse
handelsrepublieken van de premoderne wereld. Wat oud was, zal na het
jaar 2000 weer nieuw zijn. En wat ooit onvoorstelbaar was, zal alledaags
worden. Naarmate technologie in schaal afneemt, zullen overheden merken
dat zij net als bedrijven moeten concurreren om inkomsten, waarbij zij
geen hogere prijs meer kunnen vragen voor hun diensten dan wat deze
waard zijn voor degenen die ervoor betalen. De volledige gevolgen van
deze verandering zijn vrijwel niet te voorzien.

\section{\texorpdfstring{\textbf{Toen en
nu}}{Toen en nu}}\label{toen-en-nu}

Iets soortgelijks had vijfhonderd jaar geleden gezegd kunnen worden,
rond het begin van de vijftiende eeuw. Net als nu stond de westerse
beschaving toen aan de vooravond van een ingrijpende transformatie.
Hoewel bijna niemand het wist, was de middeleeuwse samenleving aan het
sterven. Haar ondergang werd noch breed voorzien, noch begrepen. Toch
was de heersende stemming er een van diepe somberheid. Dat is
gebruikelijk aan het einde van een tijdperk, wanneer conventionele
denkers aanvoelen dat alles uit elkaar valt, dat `de valk de valkenier
niet meer hoort.' Maar hun mentale inertie is vaak te groot om de
gevolgen van de opkomende machtsstructuren te begrijpen. De middeleeuwse
historicus Johan Huizinga schreef over de laatste dagen van de
middeleeuwen : `De kroniekschrijvers van de vijftiende eeuw waren
vrijwel allemaal het slachtoffer van een volkomen miskenning van hun
tijd, waarvan de werkelijke drijvende krachten aan hun aandacht
ontsnapten.'{[}5{]}

\subsection{\texorpdfstring{\textbf{Mythen
verraden}}{Mythen verraden}}\label{mythen-verraden}

Grote veranderingen in de onderliggende machtsdynamiek brengen
conventionele denkers vaak in verwarring, omdat ze de mythen ontmaskeren
die het oude systeem rechtvaardigen maar geen echte verklaringskracht
hebben. Aan het einde van de middeleeuwen , net als nu, was er een
bijzonder grote kloof tussen de gangbare mythen en de werkelijkheid.
Zoals Huizinga zei over de Europeanen in de late vijftiende eeuw:
``Alles in hun denkwijze draaide om de fictie dat de wereld werd
bestuurd volgens de idealen van de ridderlijkheid.'' Dat lijkt sterk op
de hedendaagse veronderstelling dat de wereld wordt geregeerd door
stemmen en populariteitswedstrijden. Geen van de twee overtuigingen
blijkt steek te houden als je ze zorgvuldig bekijkt. De gedachte dat het
verloop van de geschiedenis wordt bepaald door democratische stemrondes
is net zo absurd als het middeleeuwse idee dat dit gebeurt op basis van
een verfijnde gedragscode die ridderlijkheid heet.

Dat zo'n uitspraak bijna als ketterij wordt bestempeld, toont hoe ver
het conventionele denken verwijderd is van een realistisch begrip van de
machtdynamiek in de laat-industriële samenleving. Het is een onderwerp
dat we in dit boek grondig onderzoeken. Naar onze mening was stemmen een
gevolg, geen oorzaak, van de megapolitieke omstandigheden die de moderne
natiestaat voortbrachten. Massademocratie en het burgerschapsbegrip
bloeiden op met de groei van de natiestaat. Ze zullen verzwakken
naarmate de natiestaat verzwakt, en dat zal in Washington net zoveel
ontzetting veroorzaken als het verval van de ridderlijkheid vijfhonderd
jaar geleden aan het hof van de hertog van Bourgondië.

\section{\texorpdfstring{\textbf{Parallellen tussen ridderlijkheid en
burgerschap}}{Parallellen tussen ridderlijkheid en burgerschap}}\label{parallellen-tussen-ridderlijkheid-en-burgerschap}

Als je begrijpt hoe en waarom het belang van ridderlijke eden verdween
bij de overgang naar een industriële samenleving, zul je beter kunnen
inzien hoe burgerschap zoals we dat nu kennen zou kunnen verdwijnen in
het Informatietijdperk. Beide dienden een vergelijkbare functie: ze
maakten de uitoefening van macht mogelijk onder twee totaal
verschillende megapolitieke omstandigheden.

Feodale eden overheersten in een tijd waarin defensieve technologie de
overhand had, soevereiniteit gefragmenteerd was, en zowel individuen als
corporaties zelfstandig militaire macht uitoefenden. Vóór de
Buskruitrevolutie werden oorlogen meestal uitgevochten door kleine
groepen gewapende mannen. Zelfs de machtigste vorsten beschikten niet
over \emph{militum perpetuum}, ofwel staande legers. Zij baseerden hun
militaire macht op hun vazallen, de hoge edelen, die op hun beurt
steunden op hun eigen vazallen, de lage edelen, die op hun beurt weer
steunden op hun vazallen, de ridders. Deze hele keten van trouw strekte
zich uit over de hele hiërarchie, tot aan de laagste sociale klasse die
nog als waardig werd beschouwd om wapens te dragen.

\subsection{\texorpdfstring{\textbf{Uniformen of
afwijkingen?}}{Uniformen of afwijkingen?}}\label{uniformen-of-afwijkingen}

In tegenstelling tot een modern leger trok een middeleeuws leger vóór de
opkomst van het burgerschap niet het slagveld op in uniformen.
Integendeel, iedere vazal of leenman, elke ridder, baron of heer droeg
een uniek uniform dat zijn rang en positie in het feodale systeem
symboliseerde. In plaats van uniformiteit waren er juist verschillen die
de verticale structuur van de samenleving benadrukten, waarin elke
positie uniek was. Zoals Huizinga zei, werden middeleeuwse krijgers
gekenmerkt door ``uiterlijke tekenen van ... verschillen: livreien,
kleuren, emblemen, strijdkreten.''{[}7{]}

Oorlogen werden ook niet uitsluitend gevoerd door regeringen of naties.
Zoals Martin van Creveld opmerkt, geven moderne opvattingen over oorlog,
gestileerd door strategen zoals Carl von Clausewitz, een verkeerd beeld
van premoderne conflicten. Van Creveld schrijft:

\begin{quote}
\emph{Duizend jaar na de val van Rome werd een gewapend conflict gevoerd
door verschillende soorten sociale entiteiten. Onder hen bevonden zich
barbaarse stammen, de kerk, feodale baronnen van alle rangen, vrije
steden en zelfs particuliere individuen. En de ``legers'' van die
periode waren totaal niet te vergelijken met de legers die we
tegenwoordig kennen; het is immers moeilijk een woord te vinden dat
recht doet aan hun aard. Oorlog werd gevoerd door zwermen dienaren die
in militaire kledij verschenen en hun heer volgden.{[}8{]}}
\end{quote}

Onder zulke omstandigheden was het voor de heer van groot belang dat
zijn vazallen daadwerkelijk ``hun militaire kleding aantrokken en
volgden.'' Vandaar de grote nadruk op de ridderlijke eed.

De eer van de middeleeuwse ridder en de plicht van de dienstplichtige
soldaat vervulden vergelijkbare functies. De middeleeuwer was door eden
gebonden aan individuen en de Kerk, net zoals moderne mensen door
burgerschap aan de natiestaat gebonden zijn. Het breken van een eed was
in de middeleeuwen gelijk aan hoogverraad. Mensen in de late
middeleeuwen gingen erg ver om het breken van hun eden te vermijden, net
zoals dat miljoenen moderne burgers in de Wereldoorlogen onder vuur van
machinegeweren vijandelijke posities bestormden om hun plicht als
burgers te vervullen.

Zowel ridderlijkheid als burgerschap voegden een extra dimensie toe aan
de eenvoudige afweging die niet-geïndoctrineerde mensen anders ervan zou
weerhouden het slagveld op te gaan en daar te blijven wanneer het zwaar
werd. Zowel ridderlijkheid als burgerschap brachten mensen ertoe te
doden en hun leven te riskeren. Alleen veeleisende en overdreven
waarden, krachtig versterkt door invloedrijke instituties, kunnen die
functie vervullen.

\subsection{\texorpdfstring{\textbf{Het omzeilen van de
kosten-batenanalyse}}{Het omzeilen van de kosten-batenanalyse}}\label{het-omzeilen-van-de-kosten-batenanalyse}

Het succes en voortbestaan van elk systeem hangt af van het vermogen om
militaire inspanning te mobiliseren in tijden van conflict en crisis. De
beslissing van een middeleeuwse ridder of een soldaat in de loopgraven
van de Eerste Wereldoorlog om zijn leven in de strijd te riskeren, was
duidelijk niet gebaseerd op een nuchtere kosten-batenanalyse. Zelden
worden oorlogen zo gemakkelijk uitgevochten, of wegen de beloningen voor
degenen die het zware werk doen zo sterk op tegen de mogelijke kosten,
dat een leger van economische opportunisten kan worden gerekruteerd om
het slagveld te betreden. Bij vrijwel elke oorlog en de meeste gevechten
zijn er momenten waarop het tij in een oogwenk kan keren. Militaire
historici weten dat het verschil tussen nederlaag en overwinning vaak
bepaald wordt door de moed, dapperheid en felheid waarmee individuele
soldaten hun taak uitvoeren. Als de strijders niet bereid zijn te
sterven voor een stuk grond dat na de strijd niets waard is, zullen ze
waarschijnlijk niet zegevieren tegen een in andere opzichten
gelijkwaardige vijand.

Dit heeft grote consequenties. Een soeverein die desertie goed kan
tegenhouden en zijn troepen inzetbaar houdt, vergroot aanzienlijk de
kans op het winnen van oorlogen. In een oorlog zorgen de meest
effectieve waardensystemen ervoor dat mensen zich gedragen op manieren
die een rationele berekening uitsluit. Geen enkele organisatie kan
militaire macht effectief mobiliseren als de individuen die ze naar het
slagveld stuurt vrij zijn om hun eigen voordeel te berekenen en
vervolgens mee te vechten of weg te lopen. In dat geval zouden ze
vrijwel nooit vechten. Alleen onder de meest gunstige of meest wanhopige
omstandigheden zou een rationeel persoon zich op basis van een
kortetermijnkosten-batenanalyse in een potentieel dodelijk gevecht
wagen. Misschien vecht Homo economicus op een zonnige dag, als de eigen
troepen overweldigend zijn, de vijand zwak, en de potentiële beloningen
van de strijd verleidelijk. Misschien. Hij zou ook kunnen vechten als
hij in een hoek wordt gedreven door rondzwervende kannibalen.

Maar dat zijn extreme omstandigheden. Hoe zit het met de meer
gebruikelijke oorlogsomstandigheden die noch aantrekkelijk genoeg zijn
voor een rationele afweging, noch zo uitzichtloos dat men geen andere
keuze heeft? Hier spelen concepten als ridderlijkheid en burgerschap een
cruciale rol bij het succesvol inzetten van militaire macht. Ver voordat
een gevecht begint, moeten dominante organisaties individuen overtuigen
dat het naleven van bepaalde plichten aan de heer of de natiestaat
belangrijker is dan het eigen leven. De mythen en rechtvaardigingen die
samenlevingen gebruiken om risico's op het slagveld te stimuleren, zijn
een essentieel onderdeel van hun militaire kracht.

Om effectief te kunnen zijn, moeten deze mythen aansluiten bij de
heersende geopolitieke omstandigheden. Het idee dat ridderlijkheid de
wereld regeert, heeft tegenwoordig weinig betekenis, zeker niet in een
stad als New York. Maar in feodale tijden was het juist de gekoesterde
mythe van het feodalisme. Het rechtvaardigde en verklaarde de plichten
die mensen met elkaar verbonden onder de overheersing van de kerk en een
oorlogszuchtige adel. In een periode waarin oorlogen, voortkomend uit
hebzucht, de norm waren, hing het uitoefenen van macht en het
voortbestaan van individuen af van de bereidheid van anderen om hun
beloften tot militaire dienst, vaak onder dwang, na te komen. Het was
natuurlijk cruciaal dat die beloften betrouwbaar waren.

\subsection{\texorpdfstring{\textbf{Vóór
nationaliteit}}{Vóór nationaliteit}}\label{vuxf3uxf3r-nationaliteit}

In de middeleeuwen was nationaliteit geen bepalende factor voor
soevereiniteit. Vorsten, bisschoppen en edelen beheerden hun gebieden
als privébezit. Op een manier die geen modern equivalent kent, konden
deze heren territoria verkopen of wegschenken, nieuwe verwerven via
overdracht of huwelijk, of door oorlogvoering. Tegenwoordig kun je je
nauwelijks voorstellen dat de Verenigde Staten onder de soevereiniteit
zouden vallen van een niet-Engelssprekende Portugese president, alleen
omdat hij toevallig trouwde met de dochter van de voormalige Amerikaanse
president. Toch was iets dergelijks in het middeleeuwse Europa heel
gewoon. De macht ging over door erfelijke opvolging. Steden en landen
wisselden van machthebber zoals antiek van eigenaar wisselt. In veel
gevallen kwamen machthebbers niet uit de gebieden waar hun eigendommen
zich bevonden. Soms spraken ze de lokale taal niet, of slecht met een
zwaar accent. Maar dat maakte weinig verschil voor de persoonlijke
plichten. Het deed er niet toe of een Spanjaard koning van Athene was,
of een Oostenrijker koning van Spanje.

\subsection{\texorpdfstring{\textbf{Corporatieve
soevereiniteit}}{Corporatieve soevereiniteit}}\label{corporatieve-soevereiniteit}

Soevereiniteit werd ook uitgeoefend door religieuze corporaties zoals de
Tempeliers, de Orde van Sint-Jan en de Duitse Orde. Deze hybride
instellingen kennen geen moderne tegenhangers. Ze combineerden
religieuze, sociale, gerechtelijke en financiële functies met
soevereiniteit over bepaalde gebieden. Hoewel zij territoriale
jurisdictie uitoefenden, waren ze bijna het tegenovergestelde van
hedendaagse overheden, omdat nationaliteit geen rol speelde in de
mobilisatie van hun steun of in hun bestuursstructuur. De leden en
officieren van deze religieuze orden kwamen uit alle delen van het
christelijke Europa, ofwel het toenmalige ``Christenheid''.

Men vond het niet noodzakelijk dat de heersers uit de lokale bevolking
afkomstig waren. In het gefragmenteerde bestuursmodel van de
middeleeuwen hing het verkrijgen van steun niet af van een nationale
identiteit of een plicht aan de staat, zoals tegenwoordig vaak het geval
is, maar van persoonlijke loyaliteit en traditionele banden die als
kwestie van persoonlijke eer hoog in het vaandel stonden. Iedereen kon
deze eden afleggen, ongeacht zijn afkomst, mits men op basis van zijn
sociale positie als waardig werd geacht.

\subsection{\texorpdfstring{\textbf{De eed}}{De eed}}\label{de-eed}

Ridderlijke eden verbonden mensen met elkaar en werden afgelegd op grond
van persoonlijke eer. Zoals Huizinga schreef: `Door een eed af te
leggen, legden mensen zichzelf een zekere ontzegging op als aansporing
tot het verrichten van de handelingen waartoe ze zich hadden
verplicht.'{[}10{]} Men hechtte zoveel belang aan het nakomen van eden
dat mensen vaak hun leven riskeerden of zware consequenties ondervonden
om te voorkomen dat zij de eed verbraken. Vaak verplichtte de eed de
betrokkenen tot specifieke handelingen uit eer, handelingen die jij en
de meeste lezers van dit boek waarschijnlijk als belachelijk zullen
ervaren.

Zo zwoeren de Ridders van de Ster bijvoorbeeld een eed nooit meer dan
vier hectare van het slagveld terug te trekken, een regel die er spoedig
toe leidde dat meer dan negentig van hen het leven verloren.109 Dit
verbod op zelfs een tactische terugtocht is irrationeel vanuit militair
oogpunt, maar was een veelvoorkomend gebod in ridderlijke eden. Vóór de
Slag bij Azincourt gaf de koning van Engeland het bevel dat
patrouillerende ridders hun harnas moesten afleggen, omdat het
onverenigbaar met hun eer zou zijn zich terug te trekken terwijl ze hun
wapenrusting droegen. De koning zelf raakte verdwaald en passeerde het
dorp waar de voorhoede van zijn leger de nacht doorbracht. Aangezien hij
zijn harnas droeg, verbood zijn ridderlijke eer hem om eenvoudigweg om
te keren toen hij zijn vergissing ontdekte. Hij bracht de nacht door op
een onbeschutte plek.

Hoe absurd dit voorbeeld ook lijkt, koning Hendrik had waarschijnlijk
terecht ingeschat dat hij meer risico zou lopen als hij zich terug zou
trekken en zijn eer zou schenden, dan als hij achter vijandelijke linies
zou overnachten. Het zou een demoraliserend signaal naar zijn hele leger
afgeven.

De middeleeuwse geschiedenis staat vol met voorbeelden van prominente
figuren die geloften nakwamen die voor ons absurd zouden lijken. Vaak
hadden deze daden geen enkel objectief voordeel, behalve het krachtig
tonen van de waarde die men aan de eed hechtte. Veelvoorkomende geloften
waren: één oog gesloten houden, alleen staand eten en drinken, of
zichzelf verminken door zich vrijwillig met ketens te boeien. Het dragen
van pijnlijke voetkettingen was wijdverbreid. In onze tijd zou iemand
die op straat rondloopt met een zware ketting aan zijn been vooral
gestoord gevonden worden, en geen bewondering voor zijn morele karakter
oproepen. In de ridderlijke context gold het echter als een ereteken. Er
waren talloze soortgelijke gebruiken. Zoals Huizinga beschrijft: velen
beloofden ``niet in een bed te slapen op zaterdag, geen vlees te eten op
vrijdag, enz. De zelfkastijding stapelde zich op: een edelman belooft
geen harnas te dragen, één dag per week geen wijn te drinken, niet in
een bed te slapen, niet zittend te eten, een boetekleed te dragen.''

Vasten is een gematigde voortzetting van deze zelfopgelegde ontberingen.
Fanatiekelingen richtten vaak orden op die hun leden zware ontberingen
oplegden als eerproef. Zo kleedden leden van de Orde van Clalois en
Galoises zich `s zomers in bont en bontgevoerde kappen en stookten ze
vuur in de haard, terwijl ze 's winters slechts een eenvoudige jas
zonder bont mochten dragen, geen mantels, hoeden of handschoenen, en
enkel lichte bedlakens hadden. Zoals Huizinga opmerkt: 'Het is niet
verwonderlijk dat veel leden aan kou overleden.'

\begin{quote}
`Middeleeuwse zelfkastijding was een gruwelijke marteling die mensen
zichzelf aandeden in de hoop dat een oordelende en straffende God zijn
roede weg zou leggen, hun zonden zou vergeven en hen zou sparen voor de
grotere kastijdingen die hen anders in deze wereld en de volgende zouden
treffen.'{[}11{]} - NORMAN COHN
\end{quote}

\subsection{\texorpdfstring{\textbf{Zelfkastijding, toen en
nu}}{Zelfkastijding, toen en nu}}\label{zelfkastijding-toen-en-nu}

Van geloften die gevaar en ontbering oplegden, was het slechts een
kleine stap naar beproevingen, pelgrimstochten, zelfkastijding, ongemak
en zelfs opzettelijke zelfverwonding. In de middeleeuwen werden deze
gezien als zeer waardevol en prijzenswaardig. Zulke gebaren toonden de
diepe toewijding aan een gelofte, een manier van denken die nog steeds
terug te vinden is in ontgroeningsrituelen van fraterniteiten of
studentenclubs.

Stikken van de hitte in de zomer, bevriezen in de winter, of blootsvoets
op pelgrimstocht door de sneeuw was relatief mild vergeleken met ``de
gruwelijke marteling'' van zelfkastijding. Deze typisch middeleeuwse
vorm van boetedoening ontstond vrijwel gelijktijdig met het begin van
het feodalisme. Het werd voor het eerst toegepast door kluizenaars in de
kloostergemeenschappen van Camaldoli en Fonte Avellana, aan het begin
van de elfde eeuw.{[}12{]}

Flagellanten liepen niet alleen maar blootsvoets door de kou, maar
organiseerden processies waarbij ze dag en nacht van de ene stad naar de
andere trokken. ``En telkens wanneer ze een stad binnenkwamen, stelden
ze zich in groepen op voor de kerk en geselden zichzelf
urenlang.''{[}13{]}

Wanneer mensen later terug kijken op het tijdperk van de natiestaat,
verwachten wij dat ze sommige handelingen in naam van burgerschap uit de
twintigste eeuw net zo absurd zullen vinden als dat wij zelfkastijding
nu vinden. Vanuit het perspectief van de informatiesamenleving zal het
schouwspel van soldaten die in de moderne tijd de halve wereld over
reizen om de dood te trotseren uit loyaliteit aan de natiestaat, als
grotesk en dwaas worden beschouwd. Het zal niet veel verschillen van
sommige buitengewone en overdreven ridderlijke rituelen, zoals rondlopen
met beenijzers, waar anderszins verstandige mensen in de feodale tijd
trots op waren.

\subsection{\texorpdfstring{\textbf{Ridderlijkheid maakt plaats voor
burgerschap}}{Ridderlijkheid maakt plaats voor burgerschap}}\label{ridderlijkheid-maakt-plaats-voor-burgerschap}

Toen de megapolitieke omstandigheden veranderden en de militaire functie
van de eed van trouw aan een heer achterhaald raakte, verdween
Ridderlijkheid en maakte het ruimte voor burgerschap. Het tijdperk van
buskruitwapens en industriële legers bracht totaal andere verhoudingen
tussen de strijders en hun bevelhebbers met zich mee. Burgerschap kwam
voort uit een periode waarin geweld steeds winstgevender werd en de
staat over veel meer middelen beschikte dan middeleeuwse oorlogvoerende
maatschappelijke entiteiten. Dankzij haar overweldigende macht en
rijkdom kon de natiestaat rechtstreeks onderhandelen met de massa
soldaten in haar leger.

Deze overeenkomsten waren voor de staat veel goedkoper en minder
problematisch dan de onderhandelingen met machtige heren en lokale
notabelen, die eisen die tegen hun belangen ingingen konden weigeren,
iets wat individuele burgers in de natiestaat niet konden.

Om redenen die we later uitgebreider behandelen, was burgerschap totaal
afhankelijk van het feit dat geen enkel individu of kleine groep,
megapolitiek gesproken, zelfstandig militaire macht kon uitoefenen.
Naarmate informatietechnologie de logica van oorlogsvoering verandert,
zullen de mythen rond burgerschap net zo onherroepelijk verouderen als
buskruit ooit de middeleeuwse ridderlijkheid overbodig maakte.

\subsection{\texorpdfstring{\textbf{Hell's Angels te
paard}}{Hell's Angels te paard}}\label{hells-angels-te-paard}

De ruiteraristocratie die West-Europa eeuwenlang domineerde, verschilde
sterk van het soort heren dat hun nazaten later werden. Ze waren ruw en
gewelddadig. Een soort van middeleeuwse tegenhanger van motorbendes. De
etiquette en schijn van ridderlijkheid dienden eerder om hun
buitensporigheden te temperen dan om hun werkelijke gedrag te
beschrijven. Zelfs een uitgebreid overzicht van ridderlijke regels en
verplichtingen zou weinig duidelijkheid geven over de werkelijke bron
van de adellijke macht.

\subsection{\texorpdfstring{\textbf{Perfectie als synoniem voor
uitputting}}{Perfectie als synoniem voor uitputting}}\label{perfectie-als-synoniem-voor-uitputting}

De opkomst van effectieve buskruitwapens aan het einde van de vijftiende
eeuw blies de aristocratie van ridders omver, net toen zij hun
krijgskunst tot in de perfectie hadden ontwikkeld. Dankzij zorgvuldig
fokken was er toen een strijdros van zestien handen hoog, groot genoeg
om een volledig gepantserde ridder comfortabel te dragen. Maar zoals C.
Northcote Parkinson scherp opmerkte: "perfectie wordt alleen bereikt
door instellingen die op instorten staan." Net toen het nieuwe strijdros
perfect was, werden buskruitwapens ingezet die paard en ridder van het
slagveld bliezen. Deze nieuwe wapens konden door gewone mensen worden
gebruikt. Ze vereisten weinig vaardigheid maar waren duur om in grote
aantallen aan te schaffen. Hun verspreiding verhoogde geleidelijk het
belang van handel ten opzichte van landbouw, watde basis was van de
feodale economie.

\subsection{\texorpdfstring{\textbf{Oorlog op grotere
schaal}}{Oorlog op grotere schaal}}\label{oorlog-op-grotere-schaal}

Hoe veroorzaakten vuurwapens zo'n transformatie? Ten eerste vergrootten
ze de schaal van gevechten, waardoor oorlog voeren al snel veel duurder
werd dan in de middeleeuwen. Vóór de Buskruitrevolutie vochten legers
meestal met zulke kleine groepen dat ze uit een klein en arm gebied
konden worden gerekruteerd. Buskruit gaf een voordeel bij grotere
oorlogen. Alleen leiders met rijke onderdanen konden onder de nieuwe
omstandigheden effectieve strijdkrachten op de been brengen. Leiders die
het beste gebruik maakten van de toenemende handel, meestal vorsten die
een verbond sloten met stedelijke kooplieden, hadden een
concurrentievoordeel op het slagveld. In woorden van van Creveld:
``Deels dankzij hun superieure financiële middelen konden zij meer
kanonnen kopen dan wie dan ook en de tegenstander aan flarden
schieten.''{[}15{]}

Pas eeuwen later zouden buskruitwapens hun volledige effect hebben in de
burgerlegers van de Franse Revolutie, maar al in de Renaissance toonde
de adoptie van militaire uniformen een vroege verandering in de
oorlogsvoering. De uniformen symboliseren treffend de nieuwe relatie
tussen krijger en natiestaat, die samenhing met de overgang van
ridderschap naar burgerschap. In feite sloot de nieuwe natiestaat een
``uniforme'' overeenkomst met haar burgers, in tegenstelling tot de
speciale, uiteenlopende verdragen die de vorst of paus sloot met een
lange keten vazallen onder het feodalisme. In het oude systeem had
iedereen een eigen plek in een hiërarchische opbouw. Iedereen had een
overeenkomst zo uniek als zijn familiewapen en de kleurrijke vaandels
die hij voerde.

\subsection{\texorpdfstring{\textbf{Het verlagen van de
opportuniteitskosten van
rijkdom}}{Het verlagen van de opportuniteitskosten van rijkdom}}\label{het-verlagen-van-de-opportuniteitskosten-van-rijkdom}

Vuurwapens veranderden de samenleving fundamenteel op nog een manier. Ze
scheidden macht van fysieke kracht, waardoor de opportuniteitskosten van
handel daalden. Rijke kooplieden hoefden voor hun verdediging niet
langer op hun eigen vaardigheid en kracht in een gevecht te vertrouwen,
of op onbetrouwbare huurlingen. Ze konden rekenen op bescherming door de
nieuwe, grotere legers van de grote vorsten. Zoals William Playfair over
de middeleeuwen zei: ``Toen er bij vijandigheden werd gedreigd met
menselijke kracht, was het onmogelijk om lang tegelijk rijk en machtig
te zijn." Toen buskruit kwam, werd het onmogelijk om machtig te zijn
zonder ook rijk te zijn.

\subsection{\texorpdfstring{\textbf{Status en statisch
begrip}}{Status en statisch begrip}}\label{status-en-statisch-begrip}

Net zoals dat de meeste mensen tegenwoordig niet voorbereid zijn op de
veranderende dynamiek van de informatiemaatschappij, bleven de
vooraanstaande denkers in de middeleeuwen achter bij het voorspellen en
doorgronden van de opkomst van de handel, die een cruciale rol speelde
in de vorming van de moderne tijd. Vijf eeuwen geleden zagen de mensen
hun snel veranderende samenleving als iets statisch. Zoals Huizinga
opmerkte: ``Zeer weinig eigendom was liquide, in de moderne zin, terwijl
macht nog niet hoofdzakelijk met geld werd geassocieerd. Het was eerder
inherent aan de persoon en berustte op een soort religieus ontzag dat
hij inboezemde. Het drukte zich uit in pracht en praal of een grote
groep trouwe volgers. Grootsheid in het feodale of hiërarchische denken
uitte zich via zichtbare tekenen\ldots.''{[}17{]}

Omdat men in de late middeleeuwen vooral aan status dacht, zagen zij
niet in dat kooplieden een belangrijke bijdrage konden leveren aan het
functioneren van het rijk. Kooplieden behoorden vrijwel altijd tot de
laagste van de drie standen, onder de adel en de geestelijkheid.

Zelfs de meest vooruitziende denkers van die tijd erkenden niet dat
handel en ander ondernemerschap buiten de landbouw een wezenlijke bron
van rijkdom konden zijn. Voor hen was armoede een deugd. Ze maakten
letterlijk geen onderscheid tussen een vermogende bankier en een
bedelaar. Zoals Huizinga verwoordde: ``Er werd in de derde stand in
principe geen onderscheid gemaakt tussen rijke en arme burgers, noch
tussen stadsbewoners en plattelandsmensen.''{[}18{]} In hun opvatting
deden beroep en rijkdom er niet toe; alleen de ridderlijke status telde.

Deze blindheid voor de economische dimensie van het leven werd ook
versterkt door de geestelijken, de ideologische hoeders van de
middeleeuwse samenleving. Zij schatten het belang van handel zo laag in
dat ze in de vijftiende eeuw een breed geprezen hervormingsprogramma
voorstelden dat alle niet-adellijke personen ertoe bonden zich
uitsluitend op ambachtelijk werk of landbouw te richten. Handel kreeg
werkelijk geen enkele ruimte.{[}19{]}

\begin{quote}
\emph{`Het jaar 1492, traditioneel gebruikt om de middeleeuwen van de
moderne geschiedenis te scheiden, is net zo geschikt als elk ander
scheidingspunt, want vanuit wereldhistorisch perspectief symboliseert
Columbus' reis het begin van een nieuwe relatie tussen West-Europa en de
rest van de wereld.'{[}20{]} - FREDERIC C. LANE}
\end{quote}

\section{\texorpdfstring{\textbf{De geboorte van het industriële
tijdperk}}{De geboorte van het industriële tijdperk}}\label{de-geboorte-van-het-industriuxeble-tijdperk}

Veel van de scherpzinnigste geesten van de vijftiende eeuw misten
volledig een van de belangrijkste ontwikkelingen uit de geschiedenis,
terwijl die zich recht onder hun ogen voltrok. De neergang van het
feodalisme markeerde het begin van de grote moderne fase van de Westerse
dominantie. Het was een periode waarin geweld steeds meer loonde en de
schaal van ondernemingen toenam. De moderne economie heeft in de landen
die er gebruik van maakten, gedurende de laatste twee en een halve eeuw,
een ongeëvenaarde stijging van de levensstandaard opgeleverd. De
aanjagers van deze veranderingen waren nieuwe technologieën, van
vuurwapens tot de drukpers, die de grenzen van het leven veranderden op
manieren die maar weinigen konden bevatten.

Tegen het laatste decennium van de vijftiende eeuw begonnen
ontdekkingsreizigers zoals Columbus net de toegang te openen tot
uitgestrekte, onbekende continenten. Voor het eerst in de eeuwenoude
menselijke geschiedenis werd de hele wereld in kaart gebracht.
Galjoenen, nieuwe varianten op de mediterraanse galei maar met hoge
masten, voeren rond de wereld en brachten routes in kaart die
handelswegen en paden voor ziekteverspreiding en verovering zouden
worden. Conquistadores, gewapend met hun nieuwe bronzen kanonnen,
bliezen op zee en land nieuwe horizonten open. Ze vonden rijkdommen in
goud en specerijen, plantten de zaden van nieuwe handelsgewassen zoals
tabak en aardappelen, en eigenden zich nieuw graasland toe voor hun vee.

\subsection{\texorpdfstring{\textbf{De eerste industriële
technologie}}{De eerste industriële technologie}}\label{de-eerste-industriuxeble-technologie}

Net zoals het kanon nieuwe economische mogelijkheden met zich meebracht,
opende de boekdrukkunst de deur naar een geheel nieuw intellectueel
tijdperk. De drukpers functioneerde als de eerste massaproductiemachine
en betekende daarmee het begin van het industrialisme. Hiermee
onderschrijven we het standpunt dat Adam Smith in \emph{`The Wealth of
Nations'} naar voren bracht, namelijk dat de industriële revolutie al
gaande was lang voordat hij schreef. Hoewel het systeem nog niet
volgroeid was, lagen de fundamenten van massaproductie en het
fabriekssysteem al stevig verankerd. Zijn beroemde voorbeeld van de
speldenfabrikanten illustreert dit treffend: Smith legt uit dat men
achttien afzonderlijke handelingen toepaste bij de productie van
spelden. Dankzij de gespecialiseerde technologie en arbeidsverdeling
maakte elke werknemer in één dag wel 4.800 keer zoveel spelden als
wanneer hij het op eigen kracht had moeten doen.{[}21{]}

Smiths voorbeeld maakt duidelijk dat de industriële revolutie al eeuwen
eerder begon dan historici doorgaans aannemen. De meeste leerboeken
situeren het begin ervan in het midden van de achttiende eeuw, wat geen
onredelijke datum is om het begin van de stijgende levensstandaard aan
te duiden. In werkelijkheid begon de megapolitieke transitie van het
feodalisme naar het industrialisme echter al aan het einde van de
vijftiende eeuw. De impact ervan had vrijwel onmiddellijk gevolgen voor
de heersende instituties, vooral merkbaar in de snel afnemende invloed
van de middeleeuwse kerk.

Historici die de industriële revolutie later plaatsen, meten in feite
iets anders: de stijging van de levensstandaard door massaproductie
aangedreven door machines. Dit verhoogde de waarde van ongeschoolde
arbeid en leidde tot dalende prijzen voor allerlei consumptiegoederen.
Dat de levensstandaard op verschillende tijdstippen in verschillende
landen sterk begon te stijgen, wijst erop dat er iets anders wordt
gemeten dan de megapolitieke overgang. De \emph{Cambridge Economic
History of Europe} spreekt expliciet van ``Industriële Revoluties'' in
meervoud, en koppelt deze aan de aanhoudende groei van nationale
inkomens. In Japan en Rusland werd deze inkomensgroei pas eind
negentiende eeuw ingezet. In andere delen van Azië en sommige delen van
Afrika was dit een fenomeen van de twintigste eeuw. In delen van Afrika
blijft aanhoudende groei tot op heden een droom. Maar dat betekent niet
dat deze regio's niet in het moderne tijdperk leven.

\subsection{\texorpdfstring{\textbf{Inkomensdaling in een
transitieperiode}}{Inkomensdaling in een transitieperiode}}\label{inkomensdaling-in-een-transitieperiode}

Inkomensgroei is niet synoniem met de komst van het industrialisme. De
overgang naar een industriële samenleving was een megapolitieke
gebeurtenis, niet rechtstreeks meetbaar in inkomensstatistieken. Voor
het grootste deel van de Europeanen daalden de reële inkomens zelfs
gedurende de eerste twee eeuwen van het Industriële Tijdperk. Pas na het
begin van de achttiende eeuw begonnen ze te stijgen, en bereikten ze pas
rond 1750 weer het niveau van 1250. Wij plaatsen de start van het
Industriële Tijdperk aan het eind van de vijftiende eeuw. Het waren de
industriële kenmerken van vroegmoderne technologie, zoals chemisch
aangedreven wapens en drukpersen, die de ineenstorting van het
feodalisme veroorzaakten.

\subsection{\texorpdfstring{\textbf{Verlaging van de kosten van
kennis}}{Verlaging van de kosten van kennis}}\label{verlaging-van-de-kosten-van-kennis}

De capaciteit om boeken massaal te produceren was enorm ondermijnend
voor middeleeuwse instellingen, net zoals microtechnologie ondermijnend
zal blijken voor de moderne natiestaat. De boekdrukkunst gooide een bom
onder het monopolie dat de kerk had op het woord van God, en creëerde
tegelijk een nieuwe markt voor ketterij. Ideeën die onverenigbaar waren
met de gesloten feodale samenleving verspreidden zich snel, via de 10
miljoen boeken die tegen het einde van de vijftiende eeuw waren
gepubliceerd. Omdat de kerk probeerde de drukpers te onderdrukken,
werden de meeste nieuwe werken uitgegeven in delen van Europa waar het
kerkelijke gezag het zwakst was. Dit lijkt sterk op de pogingen van de
Amerikaanse overheid om versleutelingstechnologie te onderdrukken. De
kerk ontdekte dat censuur de verspreiding van zulke technologie niet
stopte. Het zorgde er alleen voor dat deze technologie op haar meest
ondermijnende manier werd gebruikt.

\subsection{\texorpdfstring{\textbf{Ontwaarding van de
kloosters}}{Ontwaarding van de kloosters}}\label{ontwaarding-van-de-kloosters}

Veel ogenschijnlijk onschuldige toepassingen van de drukpers waren
ondermijnend vanwege hun inhoud. De wetenschap dat avonturiers en
kooplieden fortuinen konden verdienen, was op zich al genoeg om de
feodale structuur af te breken. De verleiding van nieuwe markten, samen
met de noodzaak en mogelijkheid om op grote schaal legers en vloten te
financieren, gaf geld een waarde die het in de feodale eeuwen niet had.
Door deze nieuwe investeringsmogelijkheden, versterkt door krachtige
wapens die geweld winstgevender maakten, werd het voor de landheer of
koopman steeds kostbaarder om zijn kapitaal aan de kerk te schenken. Zo
ondermijnde het ontstaan van investeringsmogelijkheden buiten grondbezit
de instituties van het feodalisme en tastte het de bijbehorende
ideologie aan.

De drukpers had nog een sterk subversief gevolg. Het verlaagde de kosten
voor het reproduceren van informatie. Een cruciale reden waarom
geletterdheid en economische vooruitgang tijdens de middeleeuwen zo
beperkt waren, was de hoge kostprijs van het handmatig kopiëren van
manuscripten. Zoals eerder besproken, nam de kerk na de val van Rome een
belangrijke productieve functie op zich: het reproduceren van boeken en
manuscripten in benedictijnse kloosters. Dit was een uiterst kostbare
bezigheid. Een van de meest ingrijpende gevolgen van de drukkunst was de
ontwaarding van de scriptoria, waar monniken dag in dag uit, maand na
maand werkten aan manuscripten die met drukpersen in enkele uren
gekopieerd konden worden. De nieuwe technologie maakte het benedictijnse
scriptorium tot een verouderd en duur middel voor het verspreiden van
kennis. Daardoor verloren de religieuze orden en de kerk die de
kopiisten ondersteunden aan economisch belang.

De massaproductie van boeken maakte een einde aan het monopolie van de
kerk op de Schrift, evenals op andere vormen van informatie. De bredere
beschikbaarheid van boeken verlaagde de kosten van geletterdheid en
vergrootte daarmee het aantal denkers dat in staat was om eigen
opvattingen te uiten over belangrijke, vooral theologische, onderwerpen.
Zoals theologisch historicus Euan Cameron stelde, legde ``een reeks
publicatiemijlpalen'' in de eerste twee decennia van de zestiende eeuw
de basis voor de toepassing van ``moderne tekstkritiek op de Schrift''.
Dit ``bedreigde het monopolie'' van de kerk ``door corrupte
interpretaties van teksten die traditionele dogma's ondersteunden ter
discussie te stellen.'' Deze nieuwe kennis stimuleerde de opkomst van
concurrerende protestantse sektes die hun eigen interpretaties van de
Bijbel wilden formuleren. Massaproductie van boeken verlaagde de kosten
van ketterij en gaf ketters een groot lezerspubliek.

De uitgeverij droeg ook bij aan de ondergang van het middeleeuwse
wereldbeeld. De grotere beschikbaarheid en lagere kosten van informatie
leidden tot een verschuiving van een wereldbeeld dat symboliek hoog in
het vaandel droeg naar een die causale verbanden als basis stelde. ``Het
symbolische wereldbeeld wordt gekenmerkt door een onberispelijke orde,
architectonische structuur en hiërarchische ondergeschiktheid. Elke
symbolische verbinding impliceert een rangorde of niveau van
heiligheid\ldots{} De walnoot symboliseert Christus; de zoete kern is
Zijn goddelijke natuur, de groene pulp aan de buitenkant is Zijn
menselijkheid, de houten schaal ertussen is het kruis. Zo verwijzen alle
dingen naar het eeuwige\ldots{}''

Een symbolische denkwijze paste niet alleen bij een hiërarchisch
gestructureerde samenleving, maar ook bij ongeletterdheid. Ideeën
overgebracht via symbolen in houtsneden waren toegankelijk voor een
ongeletterde bevolking. Daarentegen leidde de komst van de boekdrukkunst
in de moderne tijd tot de ontwikkeling van causale verbanden en het
gebruik van de wetenschappelijke methode, bedoeld voor een geletterd
publiek.

\section{\texorpdfstring{\textbf{Een parallel voor
vandaag}}{Een parallel voor vandaag}}\label{een-parallel-voor-vandaag}

De middeleeuwse samenleving, die halverwege de vijftiende eeuw nog zo
stabiel en zeker leek in haar overtuigingen, werd razendsnel
getransformeerd. Het monopolie van haar dominante instituut, de kerk,
werd uitgedaagd en omver geblazen. Gezag dat eeuwenlang onaantastbaar
was, kwam plotseling ter discussie te staan. Overtuigingen en
loyaliteiten die heiliger waren dan die van burgers tegenover een
moderne natiestaat, werden in enkele jaren heroverwogen en verworpen,
allemaal door een technologische revolutie die in het laatste decennium
van de vijftiende eeuw volledig tot uiting kwam.

Wij geloven dat er opnieuw een even ingrijpende verandering zal
plaatsvinden. De informatierevolutie zal het machtsmonopolie van de
natiestaat vernietigen, net zoals de buskruitrevolutie het
machtsmonopolie van de kerk vernietigde. De situatie aan het einde van
de vijftiende eeuw, waarin het leven volledig doordrenkt was van
georganiseerde religie, vertoont een opvallende gelijkenis met die van
vandaag, waarin de wereld verzadigd is met politiek. De kerk toen en de
natiestaat nu zijn beide voorbeelden van instellingen die tot een seniel
uiterste zijn uitgegroeid. Net als de laatmiddeleeuwse kerk is de
natiestaat aan het eind van de twintigste eeuw een diep in de schulden
gestoken instituut dat haar eigen kosten niet meer kan dragen. Haar
functioneren is steeds irrelevanter en zelfs contraproductief voor het
welzijn van degenen die tot voor kort nog tot haar trouwste aanhangers
behoorden.

\subsection{\texorpdfstring{\textbf{``Verarmd, gretig en
extravagant''}}{``Verarmd, gretig en extravagant''}}\label{verarmd-gretig-en-extravagant}

Vergelijkbaar met de overheid die vandaag weinig waar voor haar geld
biedt, deed de kerk dat aan het eind van de vijftiende eeuw ook niet.
Zoals kerkhistoricus Euan Cameron het verwoordde: ``Een verarmd lokaal
priesterschap leek weinig dienst te leveren voor het geld dat het eiste;
veel van wat geheven werd, `verdween' feitelijk in besloten kloosters of
in de duistere regionen van het hoger onderwijs of de kerkelijke
administratie. Ondanks de ruime giften aan bepaalde sectoren van de kerk
wist de instelling als geheel toch tegelijkertijd een indruk van
armoede, hebzucht en overdaad te wekken.'' De parallel met de overheid
aan het einde van de twintigste eeuw is moeilijk te ontkennen.

Religieuze gebruiken namen in de late vijftiende eeuw toe, net zoals
overheidsprogramma's dat doen in moderne verzorgingsstaten. Niet alleen
namen de bijzondere zegeningen eindeloos toe, samen met het aantal
heiligen en heiligenrelieken, maar elk jaar kwamen er meer kerken, meer
kloosters, meer abdijen, meer bedelordes, meer huiskapelaans, meer
predikambten, meer kathedraalkapittels, meer gesubsidieerde missen, meer
reliekencultussen, meer religieuze broederschappen, meer religieuze
feesten en nieuwe heilige dagen. De diensten werden langer. Gebeden en
hymnen werden ingewikkelder. De ene na de andere nieuwe bedelorde
verscheen om aalmoezen te vragen. Het resultaat was een institutionele
overbelasting vergelijkbaar met die in sterk gepolitiseerde
samenlevingen vandaag.

Nieuwe religieuze vieringen en feestdagen ontstonden overal aan de
lopende band. Kerkdiensten werden talrijker, met speciale vieringen ter
ere van de zeven smarten van Maria, haar zusters en alle heiligen uit de
stamboom van Jezus. Voor gelovigen werd het steeds kostbaarder en
belastender om aan hun religieuze plichten te voldoen, net zoals vandaag
de kosten om binnen de wet te blijven zijn geëxplodeerd.

\subsection{\texorpdfstring{\textbf{De onschuldigen
betalen}}{De onschuldigen betalen}}\label{de-onschuldigen-betalen}

Toen, net als nu, droegen de productieven steeds meer de last van
inkomensherverdeling. Deze kosten stegen door een verschuiving in het
gebruik van kapitaal sneller dan gezaghebbers beseften. Het relatieve
voordeel van grondbezit ten opzichte van geldkapitaal nam af. Toch bleef
het middeleeuwse denken vasthouden aan een statusgebonden samenleving,
waarin je positie in de sociale hiërarchie werd bepaald door afkomst in
plaats van de vaardigheid om kapitaal effectief in te zetten. Er werd
nauwelijks rekening gehouden met de stijgende opportuniteitskosten van
overdreven religieuze vieringen. Die lasten drukten vooral op ambitieuze
en hardwerkende boeren, burgers en kleine grondeigenaren, die meer dan
de aristocratie afhankelijk waren van zinvol kapitaalgebruik. Ze droegen
buitensporige kosten voor de overvloedige maaltijden bij talloze feesten
en feestdagen, en moesten ook de dure kerkelijke bureaucratie
bekostigen.

\subsection{\texorpdfstring{\textbf{Contraproductieve
regelgeving}}{Contraproductieve regelgeving}}\label{contraproductieve-regelgeving}

Aan het einde van de vijftiende eeuw oefende de kerk vrijwel alle
regulerende bevoegdheden uit, bevoegdheden die later door
overheidsinstanties werden overgenomen. De kerk domineerde essentiële
rechtsgebieden: het vastleggen van akten, het registreren van
huwelijken, het afhandelen van testamenten, het verlenen van
handelslicenties, het toekennen van grondeigendomstitels en het bepalen
van de voorwaarden voor het handelsverkeer. Het dagelijks leven werd
bijna even grondig geregeld door het kerkelijk recht als het nu wordt
door bureaucratie, en met een vergelijkbaar doel. Net zoals hedendaagse
regelgeving vol zit met verwarring en tegenstrijdigheden, gold dat ook
voor het kerkelijk recht vijfhonderd jaar geleden. Deze regelgeving
belemmerde het handelsverkeer zo sterk dat al snel duidelijk werd dat de
belangen van de regelgevers ver afstonden van het bevorderen van de
productiviteit.

Zo was het bijvoorbeeld verboden om een heel jaar lang zaken te doen op
de weekdag waarop de meest recente 28 december viel. Als dat een dinsdag
was, mocht er op geen enkele dinsdag legale handel plaatsvinden, als
verplichte uiting van vroomheid ter ere van de Kindermoord in Bethlehem.
In jaren waarin 28 december op een andere dag dan zondag viel,
belemmerde dit veel vormen van handel, wat de kosten opdreef door
transacties te vertragen of volledig onmogelijk te maken.

\subsection{\texorpdfstring{\textbf{Monopolieprijzen}}{Monopolieprijzen}}\label{monopolieprijzen}

Het kerkelijk recht werd ook gebruikt om monopolieprijzen te kunnen
handhaven. De kerk verdiende aanzienlijk aan de verkoop van aluin uit
haar mijnen in Tolfa, Italië. Toen sommige klanten in de
textielindustrie de voorkeur gaven aan goedkopere aluin uit Turkije,
probeerde het Vaticaan zijn monopolie te behouden via kerkelijk recht
door het gebruik van de goedkopere aluin zondig te verklaren. Handelaren
die toch de Turkse variant kochten, werden geëxcommuniceerd. Het bekende
verbod op vlees op vrijdag kwam voort uit dezelfde logica. De kerk was
niet alleen de grootste feodale grondeigenaar, maar bezat ook grote
visgronden. De Kerkvaders bedachten een religieuze noodzaak voor het
eten van vis, wat toevallig ook de vraag naar vis garandeerde ondanks de
destijds slechte transportmogelijkheden en hygiënische omstandigheden.

Net als de natiestaat vandaag, reguleerde de kerk in de late
middeleeuwen niet alleen specifieke industrieën om haar eigen belangen
direct te ondersteunen, maar ze gebruikte haar regelgevende macht ook om
op andere manieren inkomsten te genereren. Geestelijken deden hun best
om regels en edicten op te stellen die moeilijk na te leven waren.
Incest werd bijvoorbeeld zeer ruim gedefinieerd, zodat zelfs verre neven
en nichten of personen die alleen door huwelijk verwant waren, een
speciale goedkeuring van de kerk nodig hadden om te mogen trouwen. In
veel kleine Europese dorpen betekende dit dat bijna iedereen
zo\textquotesingle n vrijstelling moest kopen; een lucratieve bron van
inkomsten. Zelfs seks binnen het huwelijk werd streng gereguleerd. Seks
tussen echtgenoten was verboden op zondagen, woensdagen en vrijdagen, en
ook gedurende de veertig dagen voor Pasen en Kerstmis. Bovendien moesten
koppels drie dagen onthouding naleven vóór het ontvangen van de
communie. Dit betekende dat seks binnen het huwelijk gedurende minstens
55\% van het jaar verboden was zonder kerkelijke toestemming. Volgens
historicus E.J. Burford, in \emph{The Bishop's Brothels}, stimuleerden
deze ``idiote'' huwelijksregels de groei van de middeleeuwse
prostitutie, waaruit de kerk aanzienlijke winst haalde. De bisschop van
Winchester was volgens Burford eeuwenlang de beheerder van de Londense
bordeelwijk Bankside in Southwark. Kerkelijke winsten uit prostitutie
waren geen uitsluitend Engels fenomeen.

\begin{quote}
\emph{Paus Sixtus IV (ca. 1471), die naar verluidt syfilis opliep van
een van zijn vele minnaressen, was de eerste paus die prostituees
licenties verleende en een belasting op hun inkomsten instelde, waardoor
de pauselijke inkomsten flink toenamen. De Romeinse Curie financierde
deels de bouw van de Sint-Pietersbasiliek met deze belasting en de
verkoop van vergunningen. Zijn opvolger, paus Leo X, zou ongeveer 22.000
gouden dukaten hebben verdiend met de verkoop van
prostitutievergunningen, vier keer meer dan hij binnen haalde met de
verkoop van aflaten in Duitsland.{[}29{]}}
\end{quote}

Zelfs de beroemde celibaatsregel voor priesters was een lucratieve
inkomstenbron voor de middeleeuwse kerk. Volgens Burford hief de kerk
``een heffing genaamd cullagium op priesters met minnaressen". Dit was
zo winstgevend dat bisschoppen in Frankrijk en Duitsland het celibaat
zonder uitzondering oplegden, ondanks dat het Lateraans Concilie van
1215 deze''schandelijke handel waarbij prelaten toestemming tot zonde
verkopen'' veroordeelde. Het was slechts één van vele lucratieve markten
voor de verkoop van vergunningen om kerkelijke wetten te overtreden,
gedreven door dezelfde logica als die van hebzuchtige politici die
willekeurige reguleringsmacht over handel nastreven.

\subsection{\texorpdfstring{\textbf{Aflaatbrieven}}{Aflaatbrieven}}\label{aflaatbrieven}

De mogelijkheid om regelgeving naar eigen inzicht in te voeren hield ook
in dat vrijstellingen verkocht mochten worden om de schade van die
regels te compenseren. De kerk verkocht vergunningen, oftewel
`aflaatbrieven', die uiteenlopende privileges verleenden, van het
kwijtschelden van kleine handelsheffingen tot toestemming voor het eten
van zuivel tijdens de vastentijd. Deze aflaten werden niet alleen voor
hoge prijzen aan adel en welgestelden verkocht, maar ook als
loterijprijzen aangeboden, vergelijkbaar met moderne staatsloterijen, om
zo ook geld van de armen te innen. De handel in aflaatbrieven nam sterk
toe zodra de uitgaven van de kerk haar inkomsten overtroffen. Velen
concludeerden dat de institutionele kerk haar macht vooral inzette om
inkomsten te genereren. Zoals een hedendaagse criticus opmerkte:
`kerkelijk recht werd uitsluitend ingesteld om geld te verdienen; wie
christen wil zijn, moet zich vrijkopen.'{[}32{]}

\subsection{\texorpdfstring{\textbf{Bureaucratische
overbelasting}}{Bureaucratische overbelasting}}\label{bureaucratische-overbelasting}

De kosten voor het onderhouden van de geïnstitutionaliseerde religie
hadden aan het einde van de vijftiende eeuw een historisch extreem
bereikt, net zoals de kosten voor het onderhouden van de overheid
tegenwoordig een seniele piek hebben bereikt. Hoe meer het leven
doordrenkt werd met religie, hoe duurder en bureaucratischer de kerk
werd. In de woorden van Cameron: "Het was veel gemakkelijker om mensen
te vinden die de sterk toegenomen hoeveelheid kerkelijke functies aan
het einde van de middeleeuwen wilden vervullen dan om geld te vinden om
ze te betalen." Net zoals failliete overheden tegenwoordig op een
contraproductieve manier inkomsten zoeken, deed de kerk dat vijfhonderd
jaar geleden ook. Inderdaad, de geestelijken gebruikten sommige van
dezelfde roofzuchtige trucs die politici vandaag de dag inzetten.

Net als de natiestaat vandaag, consumeerde de middeleeuwse kerk
vijfhonderd jaar geleden meer middelen van de samenleving dan ooit
tevoren, of ooit weer zou doen. De kerk leek, net als de staat vandaag,
niet in staat te zijn om te functioneren en zichzelf te onderhouden,
zelfs niet met recordbedragen aan inkomsten. Vergelijkbaar met hoe de
staat laat-industriële economieën beheerst en in sommige West-Europese
landen meer dan de helft van de inkomsten besteedt, domineerde de kerk
de laat-feodale economie door middelen af te romen en groei te remmen.

\subsection{\texorpdfstring{\textbf{Begrotingstekorten in de vijftiende
eeuw}}{Begrotingstekorten in de vijftiende eeuw}}\label{begrotingstekorten-in-de-vijftiende-eeuw}

De kerk maakte gebruik van alle denkbare middelen om meer geld uit haar
onderdanen te persen om haar wildgroei aan bureaucratie te voeden.
Gebieden die direct onder het gezag van de kerk vielen, moesten steeds
hogere belastingen betalen. In provincies en koninkrijken waar de kerk
geen directe belastingmacht had, legde het Vaticaan ``annates'' op, een
betaling die door de lokale heerser moest worden gedaan in plaats van
directe kerkelijke belastingen.

De kerk, net als de staat tegenwoordig, plunderde ook haar eigen kas,
waarbij financiële middelen die bestemd waren voor specifieke doeleinden
werden weggesluisd om algemene overheadkosten te betalen. Beneficiën en
kerkelijke ambten werden openlijk verhandeld, evenals de inkomsten uit
tienden. In feite werden de rechten op tienden het kerkelijke equivalent
van staatsobligaties die moderne overheden uitgeven om hun chronische
tekorten te financieren.

Hoewel de kerk ideologisch de verdediger was van het feodalisme en
criticus van handel en kapitalisme, gebruikte zij, net als de moderne
natiestaat, elke beschikbare marketingtechniek om haar eigen inkomsten
te optimaliseren. De kerk dreef een bloeiende handel in sacramentalia,
waaronder gewijde kaarsen, palmtakken gezegend op Palmzondag, ``kruiden
gezegend op het feest van de Hemelvaart, en vooral verschillende soorten
heilig water.''136

Net als hedendaagse politici die burgers dreigen met minder
vuilnisophaaldiensten en andere ongemakken als zij weigeren hogere
belastingen te betalen, waren religieuze autoriteiten in de vijftiende
eeuw ook geneigd religieuze diensten stop te zetten om gemeenten te
chanteren tot het betalen van willekeurige boetes. Vaak werden boetes
opgelegd voor kleine overtredingen door enkelen die niet eens lid
hoefden te zijn van de betreffende gemeente. In 1436 liet bisschop
Jacques Du Chatelier, ``een zeer opzichtig en hebzuchtig man,'' de Kerk
van de Onschuldigen in Parijs tweeëntwintig dagen sluiten, totdat twee
bedelaars een onmogelijk hoge boete hadden voldaan. De mannen hadden
ruzie gemaakt in de kerk waarbij een paar druppels bloed waren vergoten.
De bisschop beweerde dat ze de kerk hadden ontheiligd. Hij stond niemand
toe de kerk te gebruiken voor bruiloften, begrafenissen of de normale
jaarlijkse sacramenten totdat de boete was betaald.137

\begin{quote}
\emph{De Italiaanse bordelen (om de paus te vermaken)\\
betalen jaarlijks twintigduizend dukaten.\\
Ze geven een priester, voor wat extra eer,\\
de winst van een hoer, of twee, of meer.\\
Het moet wel een heilige zijn, geen simpele vent,\\
die zo met de bordelen is bekend.138}

\emph{-- vijftiende-eeuwse Engelse ballade}
\end{quote}

\subsection{\texorpdfstring{\textbf{Haat jegens kerkelijke
leiders}}{Haat jegens kerkelijke leiders}}\label{haat-jegens-kerkelijke-leiders}

Geen wonder dat de publieke opinie aan het eind van de vijftiende eeuw
de hogere en lagere clerus verachtte, net zoals mensen in sterk
gepolitiseerde samenlevingen vandaag de dag een afkeer hebben van
bureaucraten en politici. Zoals Johan Huizinga schreef: ``Haat is het
juiste woord in deze context, want het was haat, latent, maar algemeen
en aanhoudend. Het volk raakte nooit uitgeput van het aanhoren van de
aanklachten tegen de ondeugden van de geestelijkheid.''139 Een deel van
de reden dat mensen overtuigd waren dat de kerk ``hebzuchtig en
verkwistend'' was, is dat dit ook daadwerkelijk klopte. ``De wereldsheid
van de hogere geestelijkheid en het verval van de lagere rangen''140 was
overduidelijk.

Van de pastoor tot de paus zelf leek de geestelijkheid zo corrupt als
alleen het personeel van een overheersende instelling kan zijn.
Vijfhonderd jaar geleden maakte paus Alexander VI zelfs figuren als
Giulio Andreotti en Bill Clinton tot voorbeelden van integriteit.
Alexander VI stond bekend om zijn losbandige feesten. Als kardinaal in
Siena organiseerde hij een berucht orgie waarbij alleen ``de mooiste
jonge vrouwen van Siena waren uitgenodigd, terwijl hun `echtgenoten,
vaders en broers' waren uitgesloten.''141 Die orgie was berucht, maar
bleek later tam vergeleken met wat volgde nadat Alexander paus werd. De
meest beruchte was de zogeheten Kastanje-ballet, waarbij ``vijftig van
de mooiste hoeren van Rome'' deelnamen aan een sekswedstrijd met
kerkvorsten en andere invloedrijke Romeinen. Zoals William Manchester
beschrijft: ``Dienaren hielden het aantal orgasmes van elke man bij,
want de paus bewonderde viriliteit\ldots{} Nadat iedereen uitgeput was,
deelde Zijne Heiligheid prijzen uit: mantels, laarzen, hoeden en fijne
zijden tunieken. De winnaars'', zo schreef de kroniekschrijver, ``waren
zij die het vaakst de hoeren hadden bemind.''142

Alexander verwekte minstens zeven en mogelijk acht buitenechtelijke
kinderen. Een van zijn vermeende zonen, Giovanni, was de zogeheten
\emph{Infans Romanus}, geboren uit Alexanders buitenechtelijke dochter,
Lucrezia Borgia, toen zij achttien was. In een geheime pauselijke bul
erkende Alexander het vaderschap van Giovanni. Als hij niet de vader
was, was hij zeker de grootvader aan beide kanten. De paus was betrokken
bij een driehoekige incestueuze verhouding met Lucrezia, die ook de
minnares was van Juan, hertog van Gandia, Alexanders oudste
buitenechtelijke zoon, én van een andere buitenechtelijke zoon,
kardinaal Cesare Borgia. Cesare was de kerkvorst die Niccolò Machiavelli
inspireerde tot \emph{Il Principe}. Cesare was een moordenaar, net als
de paus, die bekend stond als samenzweerder achter verschillende
moorden. Een van beiden werd vermoedelijk jaloers op Juan, wiens
levenloze lichaam op 15 juni 1497 uit de Tiber werd gevist.143

Het leiderschap van de laatmiddeleeuwse kerk was net zo corrupt als dat
van de moderne natiestaat.

\begin{quote}
\emph{`Vandaag ben ik twee keer vader geworden, Gods zegen
daarbij.'{[}42{]}}

\emph{- Rodolph Acricola, toen hij hoorde dat zijn minnares op de dag
van zijn verkiezing tot abt een zoon had gebaard.}
\end{quote}

\section{\texorpdfstring{\textbf{Hypocrisie}}{Hypocrisie}}\label{hypocrisie}

Onder een ``oppervlakkige laag van vroomheid'' was de laatmiddeleeuwse
samenleving opmerkelijk godslasterlijk, oneerbiedig en losbandig. Kerken
waren de favoriete ontmoetingsplaatsen voor jonge mannen en vrouwen, en
ook vaste plekken voor prostituees en verkopers van obscene prenten.
Historici melden dat ``de oneerbiedigheid in het dagelijks religieus
leven vrijwel geen grenzen kende.''145 Koorzangers die waren ingehuurd
om voor de zielen van de doden te zingen, voegden vaak onbeleefde
woorden toe aan heilige teksten tijdens de mis. Wakes en processies, die
een veel grotere rol speelden in het middeleeuwse religieuze leven dan
tegenwoordig, werden desondanks ``ontheiligd door platvloersheid, spot
en drankgebruik.'' Zo verklaarde Denis de Kartuizer, de toonaangevende
theologische autoriteit van het Europa van de late middeleeuwen.

Hoewel zo'n verslag afgedaan zou kunnen worden als het geklaag van een
stijve moraalridder, is het slechts een van de vele bronnen die
hetzelfde beeld schetsen. Er is alle reden om aan te nemen dat het
schunnige en het heilige vaak nauw verweven waren in het middeleeuwse
leven. Bedevaarten ontspoorden zo vaak in oproer en losbandigheid dat
idealistische hervormers tevergeefs pleitten voor hun afschaffing.
Lokale religieuze processies boden regelmatig aanleiding voor menigtes
om te vandaliseren, te plunderen en zich over te geven aan ongeremd
dronken wangedrag. Zelfs wanneer men stil zat om de mis te vieren, was
dat vaak geen sobere aangelegenheid. Grote hoeveelheden wijn werden in
de kerk geconsumeerd, vooral tijdens nachtfestiviteiten. Uit documenten
van het Concilie van Straatsburg blijkt dat er tijdens ``nachtelijke
waken'' op Sint Adolphus 1.100 liter wijn werd genuttigd, aangeboden
door het concilie zelf.

Jean Gerson, een invloedrijke theoloog uit de vijftiende eeuw, meldt dat
``de meest heilige feesten, zelfs kerstavond,'' werden doorgebracht ``in
losbandigheid, kaartspelen, vloeken en godslastering.'' Wanneer men op
dit gedrag werd aangesproken, beriepen gewone mensen zich op ``het
voorbeeld van de adel en de geestelijkheid, die zich op dezelfde wijze
gedragen zonder gevolgen.''147

\subsection{\texorpdfstring{\textbf{Vroomheid en
mededogen}}{Vroomheid en mededogen}}\label{vroomheid-en-mededogen}

De vroomheid die de alomtegenwoordigheid van de georganiseerde religie
in de late Middeleeuwen moest rechtvaardigen, diende hetzelfde doel als
het ``mededogen'' dat vandaag wordt gebruikt om politieke overheersing
van het leven te legitimeren. De verkoop van aflaten om een verlangen
naar vroomheid zonder moraal te bevredigen, is vergelijkbaar met royale
uitgaven aan sociale voorzieningen om de schijn van mededogen te wekken,
zonder daadwerkelijke naastenliefde. Of het effect van de gangbare
religieuze praktijken werkelijk moreel karakter verbeterde of zielen
redde, was grotendeels irrelevant, net zoals het nauwelijks van belang
lijkt of een sociaal programma daadwerkelijk het leven verbetert van wie
het zou moeten helpen. ``Vroomheid'', zoals ``mededogen'', was bijna een
bijgelovige bezwering.

In een tijd waarin oorzaak en gevolg nauwelijks werden begrepen,
doordrongen rituelen en sacramenten van de kerk elk aspect van het
leven. ``... een reis, een taak, een bezoek, werden allemaal begeleid
door duizend formaliteiten: zegeningen, ceremonies, formules.''148
Gebeden op perkament werden als kettingen omgehangen bij mensen met
koorts. Ondervoede meisjes hingen haarlokken voor het beeld van
Sint-Urbanus om verdere haaruitval te voorkomen. Boeren in Navarra
liepen al biddend voor regen in processie achter een beeld van
Sint-Pieter tijdens droogte.149 In afwezigheid van werkzame middelen
grepen mensen maar al te graag naar ``ineffectieve rituelen om hun angst
te verzachten.''150

\subsection{\texorpdfstring{\textbf{Twee zonden voor een
zegen}}{Twee zonden voor een zegen}}\label{twee-zonden-voor-een-zegen}

Mensen waren zo overtuigd van de wonderbaarlijke kracht van
heiligenrelieken dat de dood van een vroom persoon vaak leidde tot een
ware stormloop om het lichaam te verdelen. Nadat Thomas van Aquino
stierf in het klooster Fossanuova, onthoofdden en kookten de monniken
zijn lichaam om zijn botten in handen te krijgen. Toen Sint Elisabeth
van Hongarije werd opgebaard, rukte een menigte aanbidders linnenstrips
van haar gezicht en knipten haar haar, nagels en tepels af.

\subsection{\texorpdfstring{\textbf{Vroomheid zonder
deugd}}{Vroomheid zonder deugd}}\label{vroomheid-zonder-deugd}

De middeleeuwse mens zag de heiligen en hun relieken als wapens van het
geloof, in een wereld vol koude winters, donkere nachten, en
uitzichtloze ziektes. Ontberingen die voor moderne lezers vrijwel
onbekend zijn. Meer dan in de moderne tijd geloofden mensen in de
Middeleeuwen dat demonen werkelijk bestonden, dat God actief ingreep in
de wereld, en dat gebeden, boetedoening en bedevaarten goddelijke
gunsten zouden opleveren.

Zeggen dat mensen in God geloofden, schiet tekort om de intensiteit van
hun overtuiging weer te geven, net als de ogenschijnlijke
vanzelfsprekendheid waarmee middeleeuwse vroomheid samenging met zonde.
Het geloof in de werkzaamheid van rituelen, sacramenten en kerkelijke
handelingen was zo wijdverspreid dat het onvermijdelijk de urgentie van
deugdzaam gedrag ondermijnde. Voor elke zonde of geestelijk gebrek
bestond een remedie, een boetedoening die het verleden uitwiste, een
systeem dat uitmondde in een ``wiskunde van de verlossing.''152 Religie
werd zo alomtegenwoordig dat haar oprechtheid onvermijdelijk begon te
slijten. Zoals Huizinga het verwoordde: ``Wanneer religie doordringt tot
alle aspecten van het leven, betekent dat een voortdurende vermenging
van het heilige en de ondeugdelijke gedachte. Heilige dingen worden te
alledaags om nog diep gevoeld te worden.''153 En dat was ook de
realiteit.

\section{\texorpdfstring{\textbf{De verkleining van de
kerk}}{De verkleining van de kerk}}\label{de-verkleining-van-de-kerk}

Aan het einde van de vijftiende eeuw kwam de kerk niet alleen even
corrupt over als de hedendaagse natiestaat, maar werkte ze ook als een
zware rem op de economische groei. De kerk vergaarde op onproductieve
wijze enorme hoeveelheden kapitaal en legde lasten op die zowel de
productie als de handel binnen de samenleving beperkten. Deze lasten,
vergelijkbaar met wat tegenwoordig door de natiestaat wordt opgelegd,
waren talrijk. We weten wat er met de georganiseerde religie gebeurde na
de buskruitrevolutie: die ontwikkelingen leidden tot krachtige prikkels
om religieuze instellingen te verkleinen en hun kosten te drukken. Toen
de traditionele kerk weigerde hiertoe over te gaan, grepen protestantse
sekten de kans om te concurreren. Daarbij gebruikten ze bijna elk
denkbaar middel om de kosten van een vroom bestaan te verlagen:

\begin{itemize}
\item
  Ze bouwden sobere nieuwe kerken en verwijderden soms de altaren uit
  \textgreater{} oudere kerken, zodat kapitaal voor andere doeleinden
  vrijkwam.
\item
  Ze herformuleerden de christelijke leer op een wijze die de kosten
  \textgreater{} deed dalen, doordat zij het geloof als sleutel tot
  verlossing vóór \textgreater{} goede werken plaatsten.
\item
  Ze ontwikkelden een nieuwe, beknopte liturgie, schrapten of
  \textgreater{} beperkten feestdagen en schafden een boel sacramenten
  af.
\item
  Ze sloten kloosters en stopten met het verstrekken van aalmoezen aan
  \textgreater{} bedelaarsordes. Armoede veranderde daarmee van een
  apostolische \textgreater{} deugd tot een ongewenst en vaak
  verwijtbaar sociaal \textgreater{} probleem.{[}51{]}
\end{itemize}

Om te begrijpen hoe het inkrimpen van de kerk de productiviteit
bevrijdde, moet je eerst kijken naar de vele manieren waarop de kerk
vóór het einde van haar monopolie de groei belemmerde. Net zoals de
natiestaat dat vandaag doet, legde de kerk aan het eind van de
vijftiende eeuw een enorme berg aan overbodige kosten op.

1. Directe kosten zoals tienden, belastingen en heffingen voedden de
uitgedijde kerkelijke bureaucratie. Tienden kwamen ook voor in
protestantse kerken die de middeleeuwse ``Heilige Moederkerk''
vervingen, maar waren in stedelijke gebieden doorgaans moeilijk te
innen. Het einde van het kerkelijke monopolie leidde feitelijk tot
dalende marginale belastingtarieven in regio's met de meest ontwikkelde
handel.

2. Religieuze doctrines maakten sparen moeilijk. De aartsvijand van de
middeleeuwse kerk was de ``vrek,'' iemand die zijn goud opspaarde ten
koste van zijn ziel. De eis dat gelovigen ``goede doelen'' moesten
bekostigen, hield in dat men dure bijdragen moest leveren aan de kerk.
De leer van de ``satisfacties'' verplichtte wie zich zorgen maakte over
zijn redding om missen of kapellen te bekostigen om het vagevuur te
vermijden. Luther viel dit rechtstreeks aan in de achtste en dertiende
van zijn vijfennegentig stellingen. Hij schreef dat ``de stervende al
zijn schulden betaalt met zijn dood.''155 Met andere woorden: het
kapitaal van de protestantse gelovige bleef beschikbaar voor zijn
erfgenamen. Volgens de protestantse leer hoefden er geen kapellen meer
gefinancierd te worden, voorheen vaak dertig jaar lang, en bij zeer
rijken zelfs tot in de eeuwigheid.

3. De ideologie van de middeleeuwse kerk dreef men er toe om kapitaal in
te zetten voor het verzamelen van relieken. Grote bedragen gingen naar
reliekenculten voor de aanschaf van tastbare objecten die met Christus
of andere heiligen werden geassocieerd. De hele rijken stelden zelfs
persoonlijke reliekencollecties samen. Zo verzamelde keurvorst Frederik
van Saksen maar liefst negentienduizend relieken, waarvan sommige
afkomstig waren van een pelgrimstocht naar Jeruzalem in 1493. Zijn
collectie bevatte onder meer wat hij geloofde dat het lichaam van een
heilig kind was, melk van Maria en stro uit de stal van de Geboorte.156
Vermoedelijk was het rendement op het daarin geïnvesteerde kapitaal
laag. Door de nieuwe focus op het persoonlijke geloof en het idee van de
uitverkorenen verminderde het nut van het verzamelen van christelijke
attributen als geluksbrengers, en werd het voor de monarch voordeliger
om geld voor productievere doeleinden in te zetten.

4. De opkomst van protestantse bewegingen brak de economische monopolies
van de middeleeuwse kerk en leidde tot een aanzienlijke verzwakking van
de regelgeving. Zoals we hebben gezien, werd het kerkelijk recht vaak
aangepast om kerkelijke monopolies en commerciële belangen te steunen.
Doordat deze nieuwe stromingen geen gevestigde economische structuren
hoefden te beschermen, resulteerden hun doctrines in een vrijer systeem
met minder handelsbelemmeringen.

5. De protestantse revolutie schafte veel van de tijdrovende rituelen en
sacramenten van de middeleeuwse kerk af. Riten, sacramenten en heilige
dagen hadden tegen het einde van de vijftiende eeuw vrijwel de hele
kalender in beslag genomen. Deze overdaad aan rituelen was een logisch
gevolg van de kerkelijke stelling ``...dat men gebeden of erediensten
eindeloos kon herhalen en daar steeds opnieuw voordeel uit kon
halen.''157 En herhalen deden ze. De productiviteit lijdde onder steeds
langere en uitgebreidere diensten, verplichtingen tot het herhalen van
gebeden als boetedoening, en de wildgroei aan feestdagen waarop niet
gewerkt mocht worden. Tal van regels en rituelen onderbraken de dag en
de seizoenen, waardoor de tijd voor productieve arbeid sterk werd
beperkt. Dit had wellicht weinig invloed op het ritme van de
middeleeuwse landbouw, waarin meer dan 90 procent van de bevolking
werkzaam was. Gedurende het jaar waren er verschillende perioden waarin
veldarbeid niet dagelijks nodig was. De oogstopbrengst in de
middeleeuwen hing waarschijnlijk sterker af van het weer en
oncontroleerbare plagen dan van de hoeveelheid arbeid die men kon
uitvoeren boven op het minimum dat de kerkelijke kalender toestond.

Het grotere productiviteitsverlies zat niet zozeer in de landbouw, maar
in andere sectoren. De kerkelijke eisen waren veel minder verenigbaar
met ambachtelijk werk, productie, transport, handel of andere
activiteiten waarbij productiviteit en winstgevendheid sterk afhingen
van de hoeveelheid tijd dat men eraan besteedde.

Het is waarschijnlijk geen toeval dat de grote omwenteling aan het eind
van de vijftiende eeuw plaatsvond op een moment dat pachtprijzen stegen
en de reële lonen voor de boerenbevolking daalden. De toenemende
bevolkingsaantallen zette de opbrengst van de gemeenschappelijke
gronden, die zich vaak rond rivieren en beken bevonden, essentieel voor
begrazing, visvangst en brandhout, onder druk. De dalende
levensstandaard dwong boeren in toenemende mate om naar alternatieve
inkomstenbronnen te zoeken. Daardoor ging een groeiend deel van de
plattelandsbevolking zich richten op kleinschalige productie voor de
markt, vooral in textiel, in het proces dat bekendstaat als
``huisnijverheid'' of ``proto-industrialisatie.''158 De tijdrovende
ceremoniële lasten die de kerk oplegde, stonden pogingen van ambitieuze
boeren in de weg om hun landbouwinkomen aan te vullen via ambachtelijk
werk, en hinderden in het algemeen elke verschuiving van inspanning naar
nieuwe economische kansen.

Een van de meest tastbare protestantse bijdragen aan de productiviteit
was het afschaffen van veertig feestdagen. Dit bespaarde niet alleen de
aanzienlijke kosten van de festiviteiten, inclusief het voorzien van
eten en drinken op dorpsfeesten, maar leverde ook veel waardevolle tijd
op. Iedereen die ophield met het vieren van de afgeschafte feestdagen
kon impliciet meer dan driehonderd manuren per jaar toevoegen aan zijn
productiviteit. Kortom, het schrappen van de ceremoniële kerkelijke
overbelasting maakte de weg vrij voor een duidelijke toename van de
productiviteit, simpelweg door tijd vrij te maken die anders aan
kerkelijke rituelen verloren was gegaan.

6. De breuk in het kerkelijke monopolie maakte enorme hoeveelheden
bezittingen vrij die onder kerkelijk beheer lage rendementen opleverden,
een situatie met duidelijke parallellen met het staatsbezit in de late
twintigste eeuw. De kerk was verreweg de grootste feodale
grootgrondbezitter. Haar greep op het land was vergelijkbaar met die van
de staat in sterk gepolitiseerde samenlevingen vandaag de dag. In
sommige Europese landen zoals Bohemen bezit de staat meer dan 50 procent
van het totale landbezit. Volgens het kerkelijk recht mocht eenmaal
verworven kerkgrond niet worden afgestoten. Daardoor namen de
landbezittingen van de kerk gestaag toe, gevoed door testamentaire
giften van gelovigen ter financiering van sociale voorzieningen,
kapellen en andere kerkelijke activiteiten.

Hoewel het moeilijk is om de relatieve productiviteit van kerkelijke
bezittingen exact te meten, moet die tegen het einde van de middeleeuwen
aanzienlijk lager zijn geweest dan aan het begin van die periode. Tegen
de veertiende eeuw leidde de verschuiving van zelfvoorzienende landbouw
naar marktgerichte productie ertoe dat de meeste lekenheren analfabete
dorpshoofden inruilden voor professionele beheerders om hun opbrengsten
te maximaliseren. Hun prikkels zullen er waarschijnlijk toe geleid
hebben dat zij de opbrengsten van kerkelijk bezit snel overtroffen,
aangezien kerkgronden in theorie geen privéwinst opleverden. Sommige
wereldlijke prins-bisschoppen beheerden hun landerijen ongetwijfeld op
een manier die nauwelijks te onderscheiden was van die van lekenheren.
Toch moet de productiviteit van andere kerkelijke eigendommen sterk
geleden hebben onder het onverschillige beheer door een logge,
wijdvertakte organisatie, met gebreken vergelijkbaar met die van staats-
en collectief eigendom vandaag de dag. Het is bovendien duidelijk dat de
inbeslagname van kloosters middelen herverdeelde die na de uitvinding
van de boekdrukkunst niet langer nodig waren voor het handmatig
reproduceren van boeken en manuscripten.

7. Zoals we in \emph{The Great Reckoning} uitvoerig hebben beschreven,
reageerden sommige protestantse stromingen op de buskruitrevolutie door
handel te stimuleren via hun doctrines, bijvoorbeeld door het verbod op
woeker, oftewel rente op leningen, op te heffen. De ideologische
tegenstand van de middeleeuwse kerk tegen het kapitalisme remde de
groei. De kern van de kerkelijke leer was gericht op het versterken van
het feodale systeem, waar de kerk zelf, als grootste feodale
grootgrondbezitter, een enorm belang in had.

Bewust of onbewust maakte de kerk religieuze deugden van haar eigen
economische belangen, terwijl ze zich verzette tegen de opkomst van
nijverheid en onafhankelijke commerciële rijkdom, ontwikkelingen die het
feodale systeem dreigden te ondermijnen. Verboden op ``winstbejag''
golden voornamelijk voor commerciële transacties en zelden of nooit voor
feodale heffingen, en al helemaal niet voor de verkoop van aflaten.
Pogingen van de kerk om een ``rechtvaardige prijs'' vast te stellen voor
handelswaar onderdrukten systematisch het economische rendement van
goederen en diensten die de kerk zelf niet produceerde.

Het verbod op ``woeker'' was een duidelijk voorbeeld van de kerkelijke
weerstand tegen commerciële innovatie. Banken en krediet waren cruciaal
voor de ontwikkeling van grootschalige handelsondernemingen. Door
kredietverlening te beperken, remde de kerk de economische groei.

8. De meer subtiele impact van de nieuwe protestantse stromingen lag in
hun nadruk op de bijbel als tekst. Dit ondermijnde niet alleen de
ideologie, maar ook het denkpatroon van de middeleeuwse kerk, die beide
een remmende werking hadden op groei. De culturele programmering van de
late middeleeuwen leerde mensen de wereld te begrijpen via symbolische
gelijkenissen in plaats van via oorzaak en gevolg. Het zette
rationaliteit op zijn kop en stond haaks op een handelsgerichte manier
van leven. Symbolisch denken laat zich moeilijk vertalen naar
marktdenken.

Zoals de historicus Johan Huizinga beschreef: ``De drie standen
vertegenwoordigen de eigenschappen van de Maagd Maria: de zeven
keurvorsten van het rijk staan voor de deugden; de vijf steden van
Artesië en Henegouwen, die in 1477 trouw bleven aan het huis van
Bourgondië, zijn de vijf wijze maagden; ... Schoeisel staat voor
zorgvuldigheid, kousen voor volharding, de jarretel voor
vastberadenheid, enzovoorts.''

De denkwereld werd beheerst door dogma's, starre symbolen en allegorieën
die elk aspect van het leven koppelden aan hiërarchische religieuze
concepten. Elke handeling, elk object, elk getal en elke kleur was
onderdeel van een allesomvattend religieus systeem. Hierdoor werden de
gewone dingen des levens niet verklaard vanuit hun oorzakelijke
verbanden, maar vanuit statische symboliek. Alles stond voor iets
anders, en dat weer voor iets daarboven. Vaak leidde dit tot verwarring
in plaats van inzicht. Numerieke systemen, vooral met het getal zeven,
versterkten dit: zeven deugden, zeven hoofdzonden, zeven beden van het
Onze Vader, zeven gaven van de Heilige Geest, de zeven momenten van de
passie van Christus, zeven zaligsprekingen en zeven sacramenten,
``vertegenwoordigd door zeven dieren, gevolgd door zeven ziekten.''

\subsection{\texorpdfstring{\textbf{Vijftiende-eeuwse
journalistiek}}{Vijftiende-eeuwse journalistiek}}\label{vijftiende-eeuwse-journalistiek}

Een nieuwsbericht uit de vijftiende eeuw zou, als het al zou zijn
geschreven, geen van de klassieke journalistieke vragen rechtstreeks
hebben beantwoord (wie, wat, waar, wanneer, waarom), maar zou deze
slechts indirect hebben aangestipt via allegorische personificaties.
Lees bijvoorbeeld dit verslag uit een privé-dagboek over de
Bourgondische moorden in het vijftiende-eeuwse Parijs:

\begin{quote}
Dan rees de godin van de Twist, die woonde in de toren van Kwade Raad,
en zij wekte Toorn, de waanzinnige vrouw, en Hebzucht en Woede en Wraak,
en zij grepen allerlei wapens en wierpen Redelijkheid, Gerechtigheid,
Herinnering aan God en Gematigdheid op schandelijke wijze buiten. Toen
maakte Waanzin hen razend, en Moord en Slachting doodden, hakten neer,
brachten ter dood, en vermoordden allen die zij in de gevangenissen
vonden. \ldots{} en Hebzucht stopte haar rok in haar gordel samen met
Roof, haar dochter, en Diefstal, haar zoon. \ldots{} Daarna trokken de
voornoemde mensen onder leiding van hun godinnen, dat wil zeggen Toorn,
Hebzucht en Wraak, die hen door alle publieke gevangenissen van Parijs
leidden, enz.161
\end{quote}

De verschuiving weg van het middeleeuwse denken stimuleerde een moderne
denkwijze, waarin oorzaak en gevolg centraal stonden in plaats van
symbolische verbanden en allegorische personificatie.

Zonder de oprechtheid van de laatmiddeleeuwse kerk ter discussie te
stellen, is het duidelijk dat haar denken paste bij het agrarische
feodale systeem en weinig ruimte liet voor handel of industriële
ontwikkeling. De kerk functioneerde als een overheersende instelling die
morele, culturele en juridische beperkingen oplegde op manieren die
perfect aansloten bij de eisen van het feodalisme. Juist daarom waren ze
slecht afgestemd op de behoeften van een industriële samenleving, net
zoals de morele, culturele en juridische beperkingen van de moderne
natiestaat slecht geschikt zijn om handel in het informatietijdperk te
bevorderen. Wij geloven dat de staat, net als de kerk destijds, zal
worden hervormd om de nieuwe mogelijkheden te kunnen realiseren.

Het protestantse idee dat de hemel kon worden bereikt door geloof
alleen, zonder dure misvieringen voor overledenen, werd gepresenteerd
als een theologische kwestie. Het was echter theologie afgestemd op de
economische realiteit van een nieuw tijdperk. Het bood een duidelijk
goedkopere weg naar verlossing, precies op het moment waarop de
opportuniteitskosten om het opgeblazen kerkelijke apparaat in stand te
houden scherp toenamen. Mensen hadden er vroeger minder moeite mee om
hun geld aan de kerk te geven, omdat er geen alternatief was. Maar zodra
ze inzagen dat ze hun kapitaal konden verhonderdvoudigen door een
specerijenexpeditie naar het Oosten te financieren, of wel 40 procent
per jaar konden verdienen met de financiering van een bataljon voor de
koning, zochten ze begrijpelijkerwijs Gods genade daar waar hun eigen
belang lag.

Veel kooplieden en andere burgers werden al snel veel rijker dan hun
voorouders ooit waren geweest onder het feodalisme. De snelle stijging
van de levensstandaard onder kooplieden en kleine producenten in de
vroegmoderne periode werd breed gehaat door degenen wier inkomen en
levensstandaard samen met het feodale systeem instortten. De verzwakking
van het kerkelijke monopolie en de toenemende megapolitieke macht van de
rijken leidde tot een scherpe afname van de herverdeling van inkomens.
De boeren en stedelijke armen die geen directe voordelen van het nieuwe
systeem ondervonden, waren bitter jaloers op degenen die dat wel deden.
Huizinga beschreef de heersende houding, in wat een belangrijke parallel
met de informatierevolutie zou kunnen worden: ``Haat jegens rijke
mensen, vooral de nieuwe rijken, die toen zeer talrijk waren, was
gemeengoed.''162

Een even opvallende parallel ontstond door een enorme stijging van de
criminaliteit. Het instorten van de oude orde leidt bijna altijd tot een
toename van misdaad, zo niet tot de regelrechte anarchie van de feodale
revolutie die we in het vorige hoofdstuk behandelden. Aan het eind van
de Middeleeuwen nam de misdaad fors toe toen de oude sociale
controlesystemen verdwenen. In Huizinga's woorden: ``Misdaad werd steeds
meer gezien als een bedreiging voor de orde en samenleving.''163 In de
toekomst kan dat net zo'n bedreiging vormen.

De moderne wereld werd geboren uit de chaos die nieuwe technologieën,
nieuwe ideeën en de stank van buskruit met zich meebrachten.
Buskruitwapens en verbeterde scheepvaart destabiliseerden het militaire
fundament van het feodalisme, terwijl nieuwe communicatietechnologie
haar ideologie ondermijnde. Een van de zaken die de nieuwe druktechniek
aan het licht bracht, was de corruptie binnen de kerk, zowel binnen de
hiërarchie als onder de gewone geestelijken, die al in laag aanzien
stonden in een samenleving die religie paradoxaal genoeg als het
middelpunt van alles zag. Die paradox is duidelijk herkenbaar in de
hedendaagse desillusie over politici en bureaucraten, in een samenleving
die politiek eveneens centraal stelt.

Het einde van de vijftiende eeuw was een tijd van desillusie,
verwarring, pessimisme en wanhoop. Een tijd, net als nu.

{[}1{]} Clarke, op. cit., p.~9.

{[}2{]} Martin van Creveld, \emph{The Transformation of War} (New York:
\emph{The Free Press}, 1991), p.~52.

{[}3{]} John Urquhart, `Voormalig premier klaagt Canada aan wegens
laster in onderzoek naar vermeende Airbus-stekpenningen,' \emph{Wall
Street Journal}, 21 november 1995, p.~A11.

{[}4{]} Ibid., p.~150.

{[}5{]} Ibid., p.~56.

{[}6{]} Ibid., p.~65.

{[}7{]} Ibid., p.~22.

{[}8{]} van Creveld, op. cit., p.~52.

{[}9{]} Ibid., p.~83.

{[}10{]} Ibid., p.~88--89.

{[}11{]} Norman Cohn, \emph{De zoektocht naar het millennium:
revolutionaire millennialisten en mystieke anarchisten van de
middeleeuwen}, herziene en uitgebreide editie (Oxford: \emph{Oxford
University Press}, 1970), p.~127.

{[}12{]} Ibid.

{[}13{]} Ibid., p.~128.

{[}14{]} C. Northcote Parkinson, \emph{Parkinson's law and other studies
in administration} (Boston: Houghton Mifflin, 1957), p.~60, geciteerd in
Tilly, p.~4.

{[}15{]} van Creveld, op. cit., p.~50.

{[}16{]} Playfair, op. cit., p.~72.

{[}17{]} Huizinga, op. cit., p.~26.

{[}18{]} Ibid., p.~57.

{[}19{]} Ibid.

{[}20{]} Frederic C. Lane, \emph{Venetië: een maritieme republiek}
(Baltimore: \emph{Johns Hopkins University Press}, 1973), p.~275.

{[}21{]} Adam Smith, \emph{Een onderzoek naar de aard en oorzaken van de
rijkdom van de naties} (Chicago: \emph{University of Chicago Press},
1976), pp.~8--9.

{[}22{]} Zie H. J. Habakkuk en M. Postan, red., \emph{De Cambridge
economische geschiedenis van Europa}, vol.~6, \emph{De industriële
revolutie en daarna: inkomens, bevolking en technologische verandering}
(Cambridge: \emph{Cambridge University Press}, 1966).

{[}23{]} Euan Cameron, \emph{De Europese reformatie} (Oxford: \emph{The
Clarendon Press}, 1992), p.~68.

{[}24{]} Ibid.

{[}25{]} Huizinga, op. cit., p.~198.

{[}26{]} Cameron, op. cit., pp.~26--27.

{[}27{]} Huizinga, op. cit., p.~149.

{[}28{]} Cameron, op. cit., pp.~26--27.

{[}29{]} Ibid., p.~102.

{[}30{]} Ibid., p.~103.

{[}31{]} Huizinga, op. cit., p.~151.

{[}32{]} Cameron, op. cit., p.~31.

{[}33{]} Ibid., p.~24.

{[}34{]} Ibid., p.~15.

{[}35{]} Huizinga, op. cit., p.~27.

{[}36{]} Burford, op. cit., p.~103.

{[}37{]} Huizinga, op. cit., p.~173.

{[}38{]} Ibid.

{[}39{]} William Manchester, \emph{A world lit only by fire: the
medieval mind and the renaissance} (Boston: Little, Brown, 1992),
pp.~75--76.

{[}40{]} Ibid., p.~79.

{[}41{]} Ibid., pp.~82--84.

{[}42{]} Huizinga, op. cit., p.~154.

{[}43{]} Ibid., p.~155.

{[}44{]} Ibid.

{[}45{]} Ibid., p.~9.

{[}46{]} Deze voorbeelden van religieuze rituelen komen uit Cameron (op.
cit., pp.~10--11).

{[}47{]} Keith Thomas, \emph{Religion and the Decline of Magic} (London:
Penguin, 1971), p.~800, geciteerd in Cameron (op. cit., p.~10).

{[}48{]} Huizinga (op. cit., p.~161).

{[}49{]} Cameron (op. cit., p.~19).

{[}50{]} Huizinga, op. cit., p.~148.

{[}51{]} Voor meer details over de scherpe verschillen tussen
vijftiende-eeuwse en zestiende-eeuwse perspectieven op armoede, zie
Robert Jutte, \emph{Poverty and Deviance in Early Modern Europe}
(Cambridge: \emph{Cambridge University Press}, 1994), pp.~15--17.

{[}52{]} Cameron, op. cit., p.~127.

{[}53{]} Ibid., p.~148.

{[}54{]} Ibid., p.~11.

{[}55{]} Huizinga, op. cit.

{[}56{]} Ibid., p.~199.

{[}57{]} Ibid., p.~203.

{[}58{]} Ibid., p.~27.

{[}59{]} Ibid., p.~22.

\bookmarksetup{startatroot}

\chapter{\texorpdfstring{\textbf{Het leven en de gezondheid van de
natiestaat}}{Het leven en de gezondheid van de natiestaat}}\label{het-leven-en-de-gezondheid-van-de-natiestaat}

\emph{Democratie en nationalisme als strategische middelen in het
tijdperk van geweld}

\begin{quote}
\emph{`Het allerbelangrijkste voor oorlogsvoering is om voldoende
financiële middelen te hebben om alles te leveren wat nodig is.'
{[}\^{}163{]} -- Robert de Balsac, 1502}
\end{quote}

\section{\texorpdfstring{\textbf{Het puin van de
geschiedenis}}{Het puin van de geschiedenis}}\label{het-puin-van-de-geschiedenis}

Op 9 en 10 november 1989 zond de televisie wereldwijd beelden uit van
uitgelaten Oost-Berlijners die de Berlijnse Muur met voorhamers
afbraken. Beginnende ondernemers in de menigte raapten stukken van de
muur op, die later aan kapitalisten overal ter wereld werden verkocht
als souvenir. Jarenlang werd er een levendige handel in deze relikwieën
gedreven. Zelfs op het moment dat we dit schrijven, kom je nog af en toe
advertenties tegen in kleine tijdschriften waarin stukken oud Oost-Duits
beton worden aangeboden voor prijzen die normaal gesproken alleen voor
hoogwaardig zilvererts worden betaald. Degenen die de Berlijnse
Muur-souvenirs hebben gekocht, hoeven geen haast te hebben om ze te
verkopen. Zij bezitten aandenken aan iets groters dan de val van het
communisme. Wij geloven dat de Berlijnse Muur de belangrijkste berg
historisch puin is geworden sinds de muren van San Giovanni met de grond
gelijk werden gemaakt bijna vijf eeuwen eerder, in februari 1495.¹⁶⁵

Het afbreken van San Giovanni door de Franse koning Karel VIII was het
eerste salvo van de buskruitrevolutie. Het markeerde het einde van de
feodale fase van de geschiedenis en het begin van het industrialisme,
zoals we eerder hebben uiteengezet. De vernietiging van de Berlijnse
Muur markeert een andere historische scheidslijn: de overgang van het
industriële tijdperk naar het nieuwe informatietijdperk. Nooit eerder
was er zo'n grote symbolische triomf van efficiëntie over macht. Toen de
muren van San Giovanni vielen, was dat een onmiskenbare demonstratie dat
de economische opbrengsten van geweld sterk waren gestegen. De val van
de Berlijnse Muur zegt echter iets anders, namelijk dat de opbrengsten
van geweld nu dalen. Dit is iets wat nog maar weinigen zijn gaan
beseffen, maar het zal dramatische gevolgen hebben.

Om redenen die we in dit hoofdstuk behandelen, kan de Berlijnse Muur
veel symbolischer zijn voor het hele tijdperk van de industriële
natiestaat dan de menigte die nacht in Berlijn, of de miljoenen die van
een afstand keken, beseften. De Berlijnse Muur werd gebouwd met een heel
ander doel dan de muren van San Giovanni: om te voorkomen dat mensen aan
de binnenkant konden ontsnappen, in plaats van te voorkomen dat
roofdieren van buitenaf binnenkwamen. Dat feit alleen al is een
veelzeggend teken van de toename van de macht van de staat van de
vijftiende tot de twintigste eeuw. En dat op meer dan één manier.

Eeuwenlang maakte de natiestaat alle muren die naar buiten gericht waren
overbodig en onnodig. De gebieden waarin de staat als eerste een
monopolie op dwang realiseerde, werden zowel intern vreedzamer als
militair krachtiger dan welke soevereine macht de wereld ook had gekend.
De staat gebruikte de middelen die zij van een grotendeels ontwapende
bevolking afnam om kleinschalige roofdieren de kop in te drukken. De
natiestaat werd het succesvolste instrument in de geschiedenis om
middelen af te pakken. Dat succes was gebaseerd op het superieure
vermogen om de rijkdom van zijn burgers voor haar eigen doeleinden in te
zetten.

\begin{quote}
\emph{MTV is meer dan alleen een leverancier van muziekvideo's en een
promotietool van de platenindustrie. Het is het eerste echte mondiale
netwerk, dat vrijwel in elk land ter wereld een uniforme programmering
biedt. Daardoor creëert MTV een gedeeld gevoel van wereldwijde realiteit
voor zijn kijkers, kinderen en jongvolwassenen. Recent onderzoek toont
aan dat jongeren wereldwijd steeds vaker niet alleen gemeenschappelijke
popiconen en smaken delen, maar ook gelijke verwachtingen over hun
carrière, vergelijkbare waarden over wat zinvol is in het leven en wat
angst veroorzaakt, en een gemeenschappelijk besef dat politiek minder
belangrijk is bij het vormgeven van hun toekomst dan hun eigen
vaardigheden. {[}\^{}165{]} -- Jim Taylor and Watts Wacker, The 500-Year
Delta: What Happens After What Comes Next}
\end{quote}

\subsection{\texorpdfstring{\textbf{``Houd ervan of vertrek'' (tenzij je
rijk
bent)}}{``Houd ervan of vertrek'' (tenzij je rijk bent)}}\label{houd-ervan-of-vertrek-tenzij-je-rijk-bent}

Voordat de overgang van de natiestaat naar de nieuwe soevereiniteiten
van het informatietijdperk voltooid is, zullen veel inwoners van de
grootste en machtigste westerse natiestaten, net als hun tegenhangers in
Oost-Berlijn in 1989, plannen maken om hun eigen weg te vinden. Voor de
generaties die volwassen werden vóór de Tweede Wereldoorlog, of vroeg in
de Koude Oorlog, is het oversteken van grenzen traumatisch. Maar voor
nieuwe generaties, die hun oriëntatie baseren op een meer globale blik,
is het verlaten van het land van geboorte niet zo ondenkbaar als voor de
ouderen, die dieper geworteld zijn in de ideologie van de natiestaat.
Jim Taylor en Watts Wacker rapporteren de intrigerende resultaten van
een grootschalige enquête onder 20.000 middenklassers op vijf
continenten. In een steekproef tijdens het schooljaar 1995-96,
uitgevoerd door Brainwaves Group, een consumentenonderzoeksbureau uit
New York, was negen van de tien leerlingen het eens met de uitspraak:
``het is aan mij om uit het leven te halen wat ik wil.'' Opvallender
nog: ``bijna de helft van de tieners zei dat ze verwachten het land van
hun geboorte te verlaten om hun doelen na te streven.''167. Misschien
omdat hij, als eerste presidentskandidaat die campagne voerde op MTV, de
mentaliteit van de MTV-generatie goed aanvoelde, heeft Bill Clinton
geprobeerd het voor Amerikanen moeilijker te maken om ``het land van hun
geboorte te verlaten om hun doelen na te streven.'' In 1995, ongeveer op
hetzelfde moment dat de middelbare scholieren hun intenties uitspraken,
stelde de president van de Verenigde Staten de invoering van een
exitbelasting voor, een ``Berlijnse Muur voor kapitaal,'' die rijke
Amerikanen zou verplichten een aanzienlijk bedrag aan losgeld te betalen
om zelfs een deel van hun geld mee te nemen.

Clintons losgeld doet niet alleen denken aan het late Oost-Duitse beleid
om haar burgers als activa te behandelen, maar het herinnert ook aan de
steeds strengere maatregelen die genomen werden om de financiële positie
van het vervallende Romeinse Rijk te versterken. Dit fragment uit
\emph{The Cambridge Ancient History} vertelt het verhaal.

\begin{quote}
\emph{Zo begon de felle poging van de staat om de bevolking tot de
laatste druppel uit te knijpen. Omdat de economische middelen
tekortschoten, vochten de sterken om het grootste deel voor zichzelf
veilig te stellen, met een geweld en gewetenloosheid die goed pasten bij
hun oorsprong en bij een soldaat die gewend was aan plundering. De wet
werd onverbiddelijk hard toegepast op de bevolking. Soldaten traden op
als deurwaarders of zwierven als geheime politie door het land. Degenen
die het meest leden, waren natuurlijk de bezittende klasse. Het was
relatief gemakkelijk hun eigendommen in beslag te nemen, en in een
noodsituatie waren zij de klasse waar het meest en het snelst iets van
af te persen viel.{[}2{]}}
\end{quote}

Wanneer falende systemen daartoe in staat zijn, leggen ze vaak zware
sancties op degenen die proberen te ontsnappen. We citeren opnieuw
\emph{The Cambridge Ancient History}: ``Als de vermogende klasse haar
geld verstopte, of twee derde van haar bezit opofferde om een
magistratuur te ontvluchten, of zelfs haar gehele eigendom opgaf om
verlost te worden van de erfpacht, en de niet-vermogende klasse
vluchtte, reageerde de staat door de druk te verhogen.''

Houd dit in het achterhoofd wanneer je plannen maakt voor de toekomst.
De ondergang van staatsystemen verliep in het verleden zelden op
beleefde of ordelijke wijze. We noemden in hoofdstuk 2 de nare gewoonten
van Romeinse belastinginners. Het grote aantal verlaten boerderijen,
agri deserti, in West-Europa na de val van het Romeinse Rijk
weerspiegelde slechts een klein deel van een breder probleem. In feite
waren de afpersingen relatief mild in Gallië en in de grensgebieden die
het huidige Luxemburg en Duitsland omvatten. In Rome\textquotesingle s
meest vruchtbare regio, Egypte, waar de landbouw productiever was door
irrigatie, was het vertrek van eigenaren een nog groter probleem. De
vraag of men moest proberen te ontsnappen, het ultimum refugium zoals
het in het Latijn werd genoemd, werd de overheersende twijfel van bijna
iedereen met bezit. Archieven tonen aan dat ``onder de gewone vragen die
men in Egypte aan een orakel stelde, drie standaardtypen waren: `Word ik
bedelaar?', `Zal ik vluchten?' en `Zal mijn vlucht worden
tegengehouden?'\,''169

Clintons voorstel bevestigt die vragen. Het betreft immers een vroege
vorm van een ontsnappingsdrempel die naar verwachting zwaarder wordt
naarmate de fiscale middelen van de natiestaat krimpen. In tegenstelling
tot Honeckers betonnen afscheidingen en prikkeldraad is de Amerikaanse
vertrekbarrière uiteraard zachter en is het vooral gericht op
miljardairs met belastbare bezittingen boven \$600.000. Desalniettemin
zijn de argumenten ervoor erg vergelijkbaar met die van Honecker, die
ooit het beroemdste openbare werk van de inmiddels verdwenen Duitse
Democratische Republiek verdedigde. Honecker betoogde dat de Oost-Duitse
staat fors had geïnvesteerd in haar potentiële vluchtelingen, en dat het
economisch nadeel voor de staat zou veroorzaken als ze zouden
vertrekken. De staat rekende immers op hun inzet in Oost-Duitsland.

Als je uitgaat van het idee dat mensen eigendom zijn of zouden moeten
zijn van de staat, dan is Honeckers muur logisch. Berlijn zonder muur
was voor de communisten een maas in de wet, net zoals ontsnapping aan de
Amerikaanse belastingjurisdictie een maas in de wet was in de ogen van
Clintons belastingdienst. Naast het feit dat Clintons opmerkingen over
miljardairs die het land verlaten de typische politieke slordigheid met
cijfers toonden, waren ze vergelijkbaar met Honeckers, maar minder
steekhoudend, aangezien de VS nauwelijks economisch geïnvesteerd heeft
in rijke burgers die willen vluchten. Het gaat hier niet om mensen die
op staatskosten zijn opgeleid en weg willen glippen om elders als
advocaat te gaan werken. De overgrote meerderheid van degenen die onder
de exitbelasting zouden vallen, hebben hun vermogen door eigen
inspanning vergaard, ondanks, en niet dankzij, de Amerikaanse overheid.

Nu de hoogste 1 procent van de belastingbetalers 28,7 procent van de
totale inkomstenbelasting in de VS betaalt, is er geen sprake van dat de
rijken nalaten een reële investering van de staat in hun opleiding of
economische voorspoed terug te betalen. Integendeel. Degenen die het
grootste deel van de rekening betalen, betalen veel meer dan de waarde
van de voordelen die zij ontvangen. Met een gemiddelde jaarlijkse
belastingafdracht van meer dan \$125.000 kost de belasting de hoogste 1
procent van de Amerikaanse belastingbetalers veel meer dan zij beseffen.
Als we ervan uitgaan dat men over een periode van veertig jaar jaarlijks
10 procent rendement zou kunnen behalen op elke \$5.000 aan te veel
betaalde belasting, dan betekent dat aan het einde van die periode een
vermindering van het potentiële vermogen met \$2,2 miljoen. Bij een
rendement van 20 procent vermindert elke \$5.000 aan overbelasting het
nettovermogen met maar liefst \$44 miljoen.

Naarmate het millennium nadert, zullen de nieuwe megapolitieke
omstandigheden van het informatietijdperk het steeds duidelijker maken
dat de natiestaat, geërfd uit het industriële tijdperk, een roofzuchtige
instelling is. Met ieder jaar dat voorbijgaat, zal zij minder als een
zegen voor de welvaart blijken en meer als een obstakel, iets waaraan
het individu wil ontsnappen. Het is die ontsnapping die wanhopige
regeringen niet graag zullen toestaan. De stabiliteit en zelfs het
voortbestaan van de westerse verzorgingsstaten hangt af van hun vermogen
om een enorm deel van de totale wereldproductie te blijven onttrekken
voor herverdeling aan een deel van de kiezers in de OESO-landen. Dit
vereist dat de belastingen die worden opgelegd aan de meest productieve
burgers van de huidige rijke landen worden vastgesteld op
supermonopolietarieven, honderden of zelfs duizenden keren hoger dan de
werkelijke kosten van de diensten die overheden daarvoor leveren.

\section{\texorpdfstring{\textbf{Het leven en sterven van de
natiestaat}}{Het leven en sterven van de natiestaat}}\label{het-leven-en-sterven-van-de-natiestaat}

De val van de Berlijnse Muur was meer dan slechts een zichtbaar symbool
van het einde van het communisme. Het betekende een nederlaag voor het
mondiale systeem van natiestaten en een overwinning voor efficiëntie en
marktwerking. De machtsbasis die eens de geschiedenis bepaalde, is
ingrijpend verschoven. Wij zijn ervan overtuigd dat de val van de
Berlijnse Muur in 1989 het hoogtepunt betekende van het tijdperk van de
natiestaat -- een merkwaardige tweehonderdjarige periode die begon met
de Franse Revolutie. Staten bestaan al meer dan zesduizend jaar, maar
vóór de negentiende eeuw vertegenwoordigden zij slechts een klein deel
van de wereldwijde soevereine entiteiten. Hun opkomst begon en eindigde
met revoluties. De grote gebeurtenissen van 1789 dreven Europa tot het
vormen van werkelijk nationale regeringen. De grote gebeurtenissen van
1989 markeerden de dood van het communisme en de overwinning van
marktwerking op geconcentreerde macht. Die twee revoluties, precies
tweehonderd jaar uit elkaar, markeren het tijdperk waarin de natiestaat
overheerste in het systeem van de grote mogendheden. De grote
mogendheden beheersten op hun beurt de wereld en verspreidden of legden
staatsstelsels op, zelfs aan de meest afgelegen tribale enclaves.

Dat de staat uitgroeide tot het voornaamste middel om geweld te
organiseren, kwam niet door ideologie maar door de harde logica van
geweld. Het was, zoals wij het noemen, een megapolitieke gebeurtenis,
niet zozeer gedreven door de wensen van theoretici en staatslieden, of
zelfs door het gemanoeuvreer van generaals, maar door de verborgen
hefboom van geweld, die de loop van de geschiedenis beïnvloedde zoals
Archimedes de wereld had willen verplaatsen.

Staten zijn gedurende de afgelopen tweehonderd jaar van de moderne
periode de norm geweest, maar in het langere verloop van de geschiedenis
waren ze zeldzaam. Ze waren voor hun levensvatbaarheid altijd
afhankelijk van uitzonderlijke megapolitieke omstandigheden. Vóór de
moderne periode waren de meeste staten ``oosterse despotieën'',
agrarische samenlevingen in woestijnen die afhankelijk waren van de
controle over irrigatiesystemen. Zelfs het Romeinse Rijk was, via haar
controle over Egypte en Noord-Afrika, indirect een hydraulische
samenleving, maar niet genoeg om te overleven. Rome ontbeerde, net als
de meeste premoderne staten, uiteindelijk de capaciteit om het
geweldsmonopolie te handhaven, dat berust op de macht om mensen te
kunnen uithongeren. De Romeinse staat kon het water voor
landbouwgewassen buiten Afrika niet afsluiten door ongehoorzame mensen
de toegang tot het irrigatiesysteem te ontzeggen. Geen enkele
megapolitieke factor in de economie van de oudheid versterkte de macht
van geweld zo sterk als de aanwezigheid van zulke hydraulische systemen.
Wie in deze samenlevingen controle had over het water, kon middelen
buitmaken op een bijna vergelijkbaar niveau als het percentage van de
totale productie dat moderne natiestaten onttrekken.170

\subsection{\texorpdfstring{\textbf{Grootte boven
efficiëntie}}{Grootte boven efficiëntie}}\label{grootte-boven-efficiuxebntie}

Buskruit maakte het voor staten eenvoudiger om zich uit te breiden
voorbij de grenzen van rijstvelden en droge rivierdalen. De aard van
buskruitwapens en de opbouw van de industriële economie brachten flinke
schaalvoordelen in de oorlogsvoering met zich mee. Dit leidde tot hoge
én steeds verder toenemende opbrengsten van geweld. Zoals historicus
Charles Tilly het verwoordde: `{[}S{]}taten met de meeste dwangmiddelen
wonnen de oorlogen; efficiëntie (de verhouding tussen output en input)
kwam op de tweede plaats na effectiviteit (totale output).'{[}ˆ170{]}
Aangezien regeringen zich vrijwel altijd op grote schaal organiseerden,
hadden zelfs de weinige kleine soevereiniteiten, zoals Monaco of
Andorra, de erkenning van de grotere staten nodig om hun
onafhankelijkheid te waarborgen. Alleen grote overheden met een steeds
grotere controle over middelen konden op het slagveld concurreren.

\subsection{\texorpdfstring{\textbf{De grote onbeantwoorde
vraag}}{De grote onbeantwoorde vraag}}\label{de-grote-onbeantwoorde-vraag}

Dit brengt ons bij een van de grote onopgeloste raadsels van de moderne
geschiedenis: waarom de Koude Oorlog, die volgde op het systeem van
grote mogendheden, uiteindelijk communistische dictaturen tegenover
democratische verzorgingsstaten plaatste. Dit vraagstuk is zo weinig
onderzocht dat het voor velen zelfs aannemelijk leek toen een analist
van het Amerikaanse ministerie van Buitenlandse Zaken, Francis Fukuyama,
na de val van de Berlijnse Muur ``het einde van de geschiedenis''
uitriep. Het enthousiaste publiek dat zijn werk toejuichte, nam te veel
als vanzelfsprekend aan. Blijkbaar hadden noch de auteur, noch vele
anderen de fundamentele vraag gesteld: welke gemeenschappelijke
kenmerken van staatssocialisme en democratische verzorgingsstaten
maakten dat zij de laatste kandidaten voor wereldheerschappij werden?

Dit is een belangrijk vraagstuk. In de afgelopen vijf eeuwen zijn immers
tientallen concurrerende soevereiniteitsstelsels gekomen en gegaan,
waaronder absolute monarchieën, tribale enclaves, prins-bisdommen,
direct pauselijk gezag, sultansrijken, stadstaten en wederdoperkolonies.
Het is nu nauwelijks voor te stellen dat een ziekenhuisorganisatie,
uitgerust met eigen strijdkrachten, eeuwenlang de macht over een land
kon uitoefenen. Toch is er iets dergelijks gebeurd. Vanaf 1228 regeerde
de Duitse Orde van het Sint-Mariaziekenhuis te Jeruzalem, later verenigd
met de Zwaardbroeders van Lijfland, driehonderd jaar lang over
Oost-Pruisen en diverse gebieden in Oost-Europa, waaronder delen van
Litouwen en Polen. Toen kwam de buskruitrevolutie. Binnen enkele
decennia werd de Duitse Orde uit al hun gebieden verdreven en hun
grootmeester had militair niet meer belang dan een schaakkampioen.
Waarom? Waarom raakten zoveel andere soevereiniteitsvormen in verval,
terwijl aan het einde van het industriële tijdperk de grote machtsstrijd
om de wereldheerschappij werd uitgevochten tussen massademocratieën en
staatssocialistische systemen?

\subsection{\texorpdfstring{\textbf{Onbelemmerde
controle}}{Onbelemmerde controle}}\label{onbelemmerde-controle}

Als onze theorie van de megapolitiek klopt, is het antwoord eenvoudig.
Het is vergelijkbaar met de vraag waarom sumoworstelaars meestal dik
zijn. Het antwoord is dat een magere sumoworstelaar, hoe indrukwekkend
zijn verhouding tussen kracht en gewicht ook is, niet kan concurreren
met een andere worstelaar die gigantisch is. Zoals Tilly stelt, ging het
om ``effectiviteit (totale output)'', niet om ``efficiëntie (de
verhouding tussen output en input)''. In een wereld die steeds
gewelddadiger werd, domineerden de systemen die de beste toegang gaven
tot middelen om op grote schaal oorlog te voeren.

Hoe werkte dat?

In het geval van het communisme is het antwoord duidelijk. Onder het
communisme hadden degenen die de staat beheersten controle over vrijwel
alles. Tijdens de Koude Oorlog kon de KGB de tandenborstel van burgers
van de Sovjet-Unie afpakken als ze dat nuttig achtten. Ze hadden zelfs
je tanden kunnen afpakken. Volgens geloofwaardige schattingen, die nog
aannemelijker zijn geworden sinds de opening van voormalige
Sovjetarchieven in 1992, namen de geheime politie en andere agenten van
de late Sovjetstaat in vierenzeventig jaar tijd het leven van 50 miljoen
mensen. Het staatssocialistische systeem kon vrijwel alles binnen haar
grenzen mobiliseren voor haar leger, met weinig kans dat iemand daar
bezwaar tegen zou maken.

In het geval van westerse democratieën ligt het minder voor de hand,
deels omdat we geneigd zijn democratie scherp te contrasteren met het
communisme. In termen van het industriële tijdperk waren de twee
systemen inderdaad grote tegenpolen. Maar vanuit het perspectief van het
informatietijdperk hadden ze meer gemeen dan je zou denken. Beide
maakten ongehinderde overheidscontrole mogelijk. Het verschil was dat de
democratische verzorgingsstaat zelfs meer middelen in handen van de
staat bracht dan de staatssocialistische systemen.

Dit is een duidelijk voorbeeld van het zeldzame verschijnsel waarbij
minder juist meer is. Het staatssocialistische systeem ging uit van het
idee dat de staat alles bezat. De democratische verzorgingsstaat
daarentegen maakte meer bescheiden aanspraken. Dit creëerde betere
prikkels waardoor een grotere productie kon gerealiseerd worden. In
plaats van vanaf het begin alles op te eisen, lieten westerse regeringen
individuen eigendom bezitten en rijkdom vergaren. Vervolgens, nadat die
rijkdom was opgebouwd, begonnen de westerse natiestaten er een groot
deel van te belasten. Onroerendgoedbelasting, inkomstenbelasting en hoge
successierechten leverden de democratische verzorgingsstaat enorme
hoeveelheden middelen op, vergeleken met wat de staatssocialistische
systemen konden genereren.

\subsection{\texorpdfstring{\textbf{Inefficiëntie, waar het ertoe
deed}}{Inefficiëntie, waar het ertoe deed}}\label{inefficiuxebntie-waar-het-ertoe-deed}

Vergeleken met het communisme werkte de verzorgingsstaat inderdaad veel
efficiënter. Echter, als je deze vergelijkt met andere systemen voor het
vergaren van rijkdom, zoals een echte laissez‑faire‑enclave als
\emph{Hongkong}, bleek de verzorgingsstaat op dat gebied minder
efficiënt te opereren. Opnieuw werd duidelijk: minder kan meer
betekenen. Juist deze ogenschijnlijke inefficiëntie stelde de
verzorgingsstaat tijdens de megapolitieke periode van het industriële
tijdperk in staat zich te onderscheiden.

Wie doorgrondt waarom, ziet veel duidelijker wat de val van de Berlijnse
Muur en het verdwijnen van het communisme echt inhouden. Het suggereerde
niet dat de democratische verzorgingsstaat de toekomst had, maar eerder
dat haar evenbeeld vredig aan ouderdom was gestorven. Dezelfde
megapolitieke revolutie die het communisme heeft gedood, zal
waarschijnlijk ook de democratische verzorgingsstaten, zoals we die in
de twintigste eeuw hebben gekend, ondermijnen en vernietigen.

\section{\texorpdfstring{\textbf{Wie bestuurt de
overheid?}}{Wie bestuurt de overheid?}}\label{wie-bestuurt-de-overheid}

Herkennen waar de controle over de democratische regering ligt, is de
sleutel tot deze onconventionele conclusie. Dit is een vraagstuk dat
minder eenvoudig is dan het lijkt. In de moderne tijd werd de vraag wie
controle heeft over de overheid bijna altijd gesteld als een politieke
vraag. Daarop zijn veel antwoorden gegeven, maar vrijwel altijd ging het
om het identificeren van de politieke partij, groep of factie die op een
bepaald moment de controle had over een bepaalde staat. Je hebt vast wel
gehoord van overheden die werden bestuurd door kapitalisten, door
arbeiders, door katholieken of door islamitische fundamentalisten, door
tribale en raciale groepen, door Hutu's en door blanken. Je zult ook
gehoord hebben van overheden bestuurd door beroepsgroepen, zoals
advocaten of bankiers, of door buitenstedelijke belangen, door
stedelijke machtsblokken en door mensen die in de buitenwijken wonen. En
zeker zul je hebben gehoord van overheden bestuurd door politieke
partijen, zoals de democraten, conservatieven, christendemocraten,
liberalen, radicalen, republikeinen en socialisten.

Je hebt echter waarschijnlijk niet veel gehoord over een overheid die
werd bestuurd door haar klanten. De economische historicus Frederic Lane
legde in enkele van zijn heldere essays over de economische gevolgen van
geweld die eerder aan bod kwamen de basis voor een nieuwe manier om te
doorgronden waar de controle over een overheid ligt. Lane kwam op het
idee om de overheid als een economische eenheid te beschouwen die
bescherming verkoopt. Door de overheid in economische termen te
analyseren in plaats van politieke, bedacht hij dat er drie fundamentele
alternatieven zijn voor wie controle heeft over de overheid, elk met
verschillende prikkels: eigenaren, werknemers en klanten.

\subsection{\texorpdfstring{\textbf{Eigenaren}}{Eigenaren}}\label{eigenaren}

In zeldzame gevallen, zelfs in de huidige tijd, leidt één eigenaar een
overheid, meestal een heerser met erfopvolging, die feitelijk het land
bezit. Zo beschouwt de sultan van Brunei de overheid van Brunei deels
als privébezit. In de middeleeuwen kwam dit vaker voor, toen heren hun
leengoed als privé-eigendom inzetten om hun inkomsten te maximaliseren.

Lane verwoordde de prikkels voor `de eigenaren van de
productieproducerende onderneming' als volgt:

\begin{quote}
\emph{Om zijn winst te vergroten, zou hij proberen de kosten te drukken
zonder de prijzen te verlagen. Hij zou, net als Hendrik VII van Engeland
of Lodewijk XI van Frankrijk, goedkope trucs, of liever nog, zo goedkoop
mogelijke middelen inzetten om zijn legitimiteit te bevestigen, de
binnenlandse orde te handhaven en de aandacht van naburige vorsten af te
leiden, zodat zijn eigen militaire uitgaven laag blijven. Hij behaalde
winst door kostenverlagingen, door hogere heffingen die dankzij de
kracht van zijn monopolie mogelijk waren, of door een combinatie van
beide. {[}4{]}}
\end{quote}

Overheden die door eigenaren worden bestuurd, hebben sterke prikkels om
de kosten voor het bieden van bescherming of het monopoliseren van
geweld in een bepaald gebied te drukken. Maar zolang hun heerschappij
onbetwist blijft, vinden zij weinig reden om de prijs (belasting) die
zij aan hun klanten vragen te verlagen tot onder het niveau dat de
opbrengsten maximaliseert. Hoe hoger de prijs die een monopolist kan
vragen en hoe lager zijn daadwerkelijke kosten, hoe groter de winst zal
zijn. Het ideale fiscale beleid voor een door eigenaren beheerde
regering leidt dan ook tot een forse winst. Wanneer overheden erin
slagen hun inkomsten hoog te houden en tegelijkertijd hun uitgaven te
beperken, heeft dat een grote invloed op het gebruik van middelen.
Arbeid en andere waardevolle productiefactoren, die anders verspild
zouden worden aan het leveren van onnodig dure bescherming, komen dan
beschikbaar voor investeringen en andere doeleinden. Hoe meer winst de
vorst boekt door kosten te drukken, hoe meer middelen er vrijkomen.
Worden die middelen voor investeringen ingezet, stimuleren ze de
economische groei. Zelfs wanneer ze enkel voor luxe-uitgaven worden
ingezet, stimuleren ze toch de vorming van nieuwe markten die niet
zouden ontstaan als de middelen waren verkwist aan inefficiënte
``bescherming.''

\textbf{Werknemers}

Het is eenvoudig om de prikkels te typeren die overheersen bij overheden
die door hun werknemers worden bestuurd. Het zijn vergelijkbare prikkels
als in andere door werknemers geleide organisaties. Allereerst hebben
door werknemers bestuurde organisaties de neiging elk beleid te steunen
dat de werkgelegenheid vergroot en maatregelen af te wijzen die banen
verminderen. Zoals Lane het formuleerde: ``Wanneer werknemers de macht
hadden, lag hun prioriteit niet bij het drukken van beschermingskosten
en al helemaal niet het bij beperken van hun eigen loonlast. Het
maximaliseren van de omvang van de organisatie had meer hun
voorkeur.''173 Een door werknemers bestuurde overheid zou zelden
prikkels hebben om de kosten van de overheid of de prijs die aan hun
klanten wordt berekend te verlagen. Wanneer men op sterke weerstand
tegen de hoge prijzen stuit, zijn dit soort overheden echter geneigd om
hun inkomsten te laten dalen tot onder het niveau van hun uitgaven in
plaats van simpelweg minder uit te geven. Met andere woorden, deze
prikkels zorgen voor een neiging tot chronische tekorten, in
tegenstelling tot overheden die onder controle van eigenaren staan.

\subsection{\texorpdfstring{\textbf{Klanten}}{Klanten}}\label{klanten}

Zijn er voorbeelden van regeringen die door hun klanten werden
gecontroleerd? Ja. Lane vond zijn inspiratie in steden als Venetië, waar
kooplieden de macht hadden, en gebruikte dat om staatscontrole in
economische termen te duiden. Daar had een groep groothandelaars, die
bescherming nodig hadden, gedurende eeuwen effectief de controle over de
overheid. Zij waren werkelijk klanten van de beschermingsdienst die de
overheid leverde, geen eigenaars. Ze betaalden voor de dienst en
probeerden niet te profiteren van hun controle over het geweldsmonopolie
van de staat. Als sommigen dat toch deden, weerhielden andere klanten
hen voor lange tijd ervan om dat nogmaals te doen. Andere voorbeelden
van door klanten bestuurde overheden zijn democratieën en republieken
met een beperkt kiesrecht, zoals de democratieën van de oudheid of de
Amerikaanse republiek in haar beginperiode. In die tijd mochten alleen
degenen die voor de overheid betaalden, ongeveer 10 procent van de
bevolking, stemmen.

Overheden die door hun klanten worden bestuurd, hebben net als die van
eigenaars de prikkel om hun operationele kosten zoveel mogelijk te
beperken. Maar in tegenstelling tot overheden die door eigenaars of
werknemers worden gecontroleerd, hebben regeringen die daadwerkelijk
door hun klanten worden bestuurd ook de prikkel om de prijzen die zij
rekenen laag te houden. Waar klanten de macht hebben, zijn overheden
klein en doorgaans onopvallend, met lage bedrijfskosten, een minimaal
aantal werknemers en lage belastingen. Een overheid onder controle van
haar klanten stelt de belastingdruk niet in om de overheidsinkomsten te
maximaliseren, maar om het resultaat van haar klanten te optimaliseren.

Zoals typische ondernemingen in concurrerende markten zou zelfs een
monopolie dat door haar klanten wordt bestuurd, gedwongen worden om
efficiënt te functioneren. Het zou geen belastingen kunnen heffen die de
kosten met meer dan een minimale marge overschreden.

\section{\texorpdfstring{\textbf{De rol van democratie: kiezers als
werknemers en
klanten}}{De rol van democratie: kiezers als werknemers en klanten}}\label{de-rol-van-democratie-kiezers-als-werknemers-en-klanten}

Lane bekijkt de democratie op de traditionele manier en gaat ervan uit
dat zij ervoor zorgt dat ondernemingen die geweld inzetten en produceren
steeds meer onder de controle van hun klanten komen.{[}6{]} Natuurlijk
lijkt dat de politiek correcte conclusie, maar is dat wel zo? Wij
twijfelen daar sterk aan. Kijk eens goed hoe moderne democratieën nu
werkelijk functioneren.

In de eerste plaats missen ze de typische eigenschappen van competitieve
markten waarin klanten de voorwaarden dicteren. Democratische regeringen
besteden namelijk doorgaans slechts een fractie van hun totale uitgaven
aan hun kerntaak: bescherming. In de Verenigde Staten besteden de
deelstaten en lokale overheden bijvoorbeeld slechts 3,5 procent van hun
totale uitgaven aan politie, rechtbanken en gevangenissen. Tel de
militaire uitgaven daarbij op, en nog steeds gaat slechts ongeveer 10
procent van de inkomsten naar bescherming.

Een ander veelzeggend teken dat een massademocratie niet door haar
klanten wordt bestuurd, is het feit dat het in de hedendaagse politieke
cultuur, geërfd uit het industriële tijdperk, ondenkbaar zou zijn als
beleid inzake cruciale kwesties werkelijk zou worden bepaald door de
belangen van degenen die de rekeningen betalen. Stel je de
verontwaardiging voor als een Amerikaanse president of een Britse
premier zou voorstellen om de groep burgers die het merendeel van de
belastingen betaalt, te laten beslissen welke overheidsprogramma's
moeten doorgaan en welke groepen ambtenaren ontslagen zouden moeten
worden. Zo'n handelwijze zou indruisen tegen het fundamentele beeld van
wat een overheid behoort te doen, meer dan wanneer ambtenaren de macht
hadden om te bepalen wie extra belasting moest betalen.

Toch, als je erover nadenkt: wanneer klanten werkelijk aan het roer
staan, zou het juist schandalig worden gevonden als zij niet zouden
krijgen wat ze willen. Als je een winkel binnen zou lopen om meubels te
kopen, en de verkopers zouden je geld aannemen maar vervolgens negeren
ze je verzoeken en zouden anderen raadplegen over hoe jouw geld moest
worden besteed, zou je terecht verontwaardigd zijn. Je zou het ook niet
normaal of gerechtvaardigd vinden als de werknemers van de winkel zouden
betogen dat jij de meubels eigenlijk niet verdient, en dat deze in
plaats daarvan moeten worden geleverd aan iemand die zij meer geschikt
achten. Het feit dat iets zeer vergelijkbaars gebeurt in de omgang met
de overheid, laat zien hoe weinig controle haar ``klanten'' in
werkelijkheid hebben.

Hoe je het ook bekijkt, de kosten van democratische regeringen lopen
volledig uit de hand, in schril contrast met de omstandigheden in
competitieve markten, waar klanten de aanbieders dwingen efficiënt te
werken. De meeste democratieën kampen met structurele
begrotingstekorten. Dit is kenmerkend voor een fiscaal beleid dat
voortkomt uit de controle van de werknemers. Overheden blijken bijzonder
terughoudend als het gaat om het snijden in hun operationele kosten.
Wereldwijd horen we vrijwel altijd de klacht dat het erg moeilijk is om
eenmaal ingevoerde politieke programma's te stoppen. Ambtenaren ontslaan
blijkt vrijwel onmogelijk. Sterker nog, een van de belangrijkste
voordelen van de privatisering van functies, die voorheen door de staat
werden uitgevoerd, is dat particuliere controle het veel eenvoudiger
maakt om overbodige banen te schrappen. Van Groot-Brittannië tot
Argentinië laten nieuwe particuliere managers vaak 50 tot 95 procent van
de voorheen door de staat tewerkgestelde werknemers gaan.

Bedenk ook hoe de fiscale voorwaarden voor de beschermingsdienst van de
overheid worden bepaald. Je zult niet snel invloeden van competitie
vinden wanneer je de belastingtarieven analyseert. Zelfs de sporadische
discussies over belastingverlaging die de afgelopen jaren het politieke
debat doorbraken, tonen aan hoe ver een democratische overheid doorgaans
van controle door haar klanten is verwijderd. Voorstanders van lagere
belastingen betoogden soms dat de overheidsinkomsten juist zouden
stijgen, omdat de tarieven voorheen zo hoog waren dat ze economische
activiteit ontmoedigden.

De afweging die zij doorgaans wilden benadrukken, ging niet over
concurrentie tussen jurisdicties, maar over iets veel
verbazingwekkenders. Ze stelden niet dat de tarieven in de Verenigde
Staten of Duitsland niet hoger mochten zijn dan 15 procent, omdat de
belastingtarieven in Hongkong ook slechts 15 procent waren. Integendeel,
debatten over de belastingdruk gingen er meestal van uit dat de afweging
voor de belastingbetaler niet lag tussen zaken doen in de ene
jurisdictie of in een andere, maar tussen zaken doen tegen draconische
tarieven of een vakantie nemen. Er werd gezegd dat productieve
individuen die aan roofzuchtige belastingen onderworpen waren, hun werk
zouden neerleggen en gaan golfen als hun belastingdruk niet werd
verlicht.

Het feit dat zo'n argument zelfs kon ontstaan, laat zien hoe ver de
tarieven voor bescherming, die door democratische verzorgingsstaten
worden opgelegd, verwijderd zijn van een competitief uitgangspunt. De
voorwaarden van progressieve inkomstenbelasting, die in de loop van de
twintigste eeuw in elke democratische verzorgingsstaat ontstonden,
wijken dramatisch af van prijsbepalingen die door klanten zouden worden
geprefereerd. Het verschil valt op wanneer men belastingen voor een
beschermingsmonopolie afzet tegen de tarieven van telefoondiensten, die
tot voor kort overal monopolistisch waren. Klanten zouden moord en brand
schreeuwen als een telefoonbedrijf probeerde te factureren op dezelfde
basis als waarop inkomstenbelastingen worden geheven. Stel dat het
telefoonbedrijf een rekening van €50.000 stuurt voor een gesprek naar
Londen, alleen omdat je toevallig tijdens dat gesprek een deal van
€125.000 had gesloten. Noch jij, noch een andere klant met gezond
verstand zou dat betalen. Maar dat is precies de basis waarop
inkomstenbelastingen in elke democratische verzorgingsstaat worden
geheven.

Wanneer je goed kijkt naar de manier waarop industriële democratieën
hebben geopereerd, is het logischer om ze te beschouwen als een vorm van
overheid die door haar werknemers wordt bestuurd. Als je massademocratie
ziet als een door ambtenaren bestuurde overheid, wordt duidelijk waarom
beleidsveranderingen zo lastig zijn. De overheid lijkt in veel opzichten
bestuurd ten voordele van de werknemers. Zo lijken openbare ­scholen in
de meeste democratische landen chronisch slecht te functioneren, en
zonder zicht op verbetering. Als de klanten werkelijk aan het roer
stonden, zouden ze het gemakkelijker vinden om nieuwe beleidsrichtingen
uit te zetten. Degenen die voor de democratische overheid betalen,
bepalen zelden de voorwaarden van de overheidsuitgaven. In feite is de
overheid een coöperatief systeem dat geen privé-eigendom heeft en zich
gedraagt als een natuurlijk monopolie. Prijzen houden weinig verband met
kosten. De kwaliteit van de dienstverlening is over het algemeen laag in
vergelijking met die in de private sector. Klachten van klanten zijn
moeilijk te verhelpen. Kortom, massademocratie leidt tot controle over
de overheid door haar ``werknemers.''

Maar wacht. Je zou kunnen zeggen dat er in de meeste jurisdicties veel
meer kiezers zijn dan mensen op de loonlijst van de overheid. Hoe zou
het onder zulke omstandigheden mogelijk zijn dat werknemers domineren?
Dit probleem werd opgelost door het ontstaan van de verzorgingsstaat. Om
het aantal werknemers aan te vullen tot een werkbare meerderheid, werden
steeds meer kiezers praktisch op de loonlijst geplaatst door ze allerlei
vormen van staatssteun te geven. In feite werden de ontvangers van
uitkeringen en subsidies pseudo-werknemers van de overheid, die zich
konden ontdoen van het gedoe om dagelijks op het werk te verschijnen.
Het was een resultaat dat werd bepaald door de megapolitieke logica van
het industriële tijdperk.

Wanneer de omvang van het dwingende instituut belangrijker is dan de
efficiënte inzet van middelen, zoals het geval was vóór 1989, is het
vrijwel onmogelijk dat de meeste overheden door hun klanten worden
bestuurd. Zoals het voorbeeld van de late Sovjet Unie zo duidelijk
aantoonde, was het tot een paar jaar geleden mogelijk voor staten om
grote macht in de wereld uit te oefenen, zelfs terwijl ze op enorme
schaal middelen verspilden. Wanneer de opbrengsten van geweld hoog en
stijgend zijn, telt omvang meer dan efficiëntie. Grotere entiteiten
hebben dan de neiging om te winnen van kleinere. De overheden die
effectiever zijn in het mobiliseren van militaire middelen, zelfs als
dat verspilling betekent, hebben dan de neiging te zegevieren boven
degenen die hun middelen efficiënter gebruiken.

Denk eens na over wat dit betekent. Het impliceert onontkoombaar dat
wanneer omvang belangrijker is dan efficiëntie, overheden die door hun
klanten worden bestuurd, niet kunnen zegevieren en vaak ook niet kunnen
overleven. Onder zulke omstandigheden zullen de entiteiten die militair
het meest effectief zijn degenen zijn die de meeste middelen voor oorlog
kunnen opeisen. Echter, regeringen die echt worden bestuurd door hun
betalende klanten, zouden waarschijnlijk niet onbeperkt middelen van
iedereen kunnen afromen

Klanten wensen normaal gesproken dat de prijzen die zij voor een product
of dienst betalen, inclusief voor bescherming, worden verlaagd en onder
controle wroden gehouden. Als de westerse democratieën tijdens de Koude
Oorlog onder controle van de klant hadden gestaan, zou dat alleen al hun
militaire slagkracht hebben verminderd, doordat het vrijwel zeker de
toevoer van middelen naar de overheid zou hebben ingeperkt. Vergeet niet
dat in situaties waarin klanten heersen, zowel de prijzen als de kosten
nauwlettend beheerst dienen te worden. Maar dit is nauwelijks wat er
gebeurde. De verzorgingsstaten waren absolute winnaars van de
uitgavenwedstrijd tijdens de Koude Oorlog. Analisten van diverse
achtergronden stelden dat hun overwinning mede te danken was aan hun
capaciteit om zoveel geld uit te geven dat de Sovjet-Unie in financiële
problemen kwam.

Precies dit feit benadrukt hoe de inefficiënties van de democratie haar
op megapolitiek niveau dominant maakten tijdens een periode geweld
steeds meer opleverde. Massale militaire uitgaven vormen, met al hun
verspilling, duidelijk een suboptimale inzet van kapitaal voor het
opbouwen particuliere welvaart. We suggereerden eerder dat, hoewel
verzorgingsstaten economisch efficiënter waren in vergelijking tot
socialistische overheidssystemen, ze veel minder efficiënt zijn voor het
creëren van rijkdom dan laissez-faire enclaves, zoals Hongkong. Ironisch
genoeg was het juist deze inefficiëntie, vergeleken met een minder
belemmerd vrijemarktsysteem, die de democratische verzorgingsstaat
succesvol maakte onder de megapolitieke omstandigheden van het
industrialisme.

Hoe werd democratische inefficiëntie een succesfactor tijdens het
Tijdperk van Geweld? De sleutel tot het ontrafelen van deze schijnbare
paradox ligt in het erkennen van twee punten:

1. Succes voor een soevereiniteit in de moderne periode lag niet in het
creëren van rijkdom, maar in het creëren van een militaire macht die in
staat was overweldigend geweld tegen elke andere staat in te zetten.
Geld was nodig om dat te doen, maar geld zelf kon een veldslag niet
winnen. De uitdaging was niet om een systeem met de efficiëntste
economie of de hoogste groeisnelheid te creëren, maar een systeem dat
meer middelen kon onttrekken om het vervolgens in het leger te
kanaliseren. Militaire uitgaven zijn van nature een kostenpost waarvan
de financiële opbrengsten op zichzelf laag of afwezig zijn.

2. De gemakkelijkste manier om toestemming te krijgen voor het
investeren van geld in activiteiten met weinig of geen direct financieel
rendement, zoals belastingbetalingen, is om toestemming te vragen aan
iemand anders dan de persoon die het zal moeten opbrengen. Een van de
manieren waarop de Nederlanders Manhattan konden kopen voor
drieëntwintig dollar aan kralen, was dat de specifieke indianen aan wie
ze het aanbod deden, het eigenlijk niet bezaten. ``Getting to yes,''
zoals de marketingmensen zeggen, is onder die voorwaarden veel
gemakkelijker. Stel bijvoorbeeld dat wij als auteurs van dit boek niet
zouden willen dat u de kaftprijs, maar 40 procent van uw jaarlijkse
inkomen zou betalen voor een exemplaar. We zouden veel eerder
toestemming krijgen als we het aan iemand anders zouden vragen, en u
niet zelf zouden hoeven te benaderen. Sterker nog, we zouden veel
overtuigender kunnen zijn als we zouden kunnen vertrouwen op de
instemming van meerdere mensen die u zelfs niet kent. We zouden een
ad-hocverkiezing kunnen houden, wat H. L. Mencken, met minder
overdrijving dan hij misschien dacht, beschreef als ``een geavanceerde
veiling van gestolen goederen.'' En om het voorbeeld realistischer te
maken, zouden we ermee kunnen instemmen om een deel van het door u
betaalde geld te delen met deze anonieme omstanders in ruil voor hun
steun.

Dat is de rol die de moderne democratische verzorgingsstaat is gaan
vervullen. Het was een onovertroffen systeem in het industriële tijdperk
omdat het zowel efficiënt als inefficiënt was waar het ertoe deed. Het
combineerde de efficiëntie van particulier bezit en prikkels voor het
creëren van rijkdom met een mechanisme dat in wezen onbelemmerde toegang
tot die rijkdom faciliteerde. Democratie hield de portemonnees van
welvaartsproducenten open. Het kende militair succes in de periode
waarin het rendement van geweld wereldwijd op zijn hoogtepunt was, juist
omdat het het voor klanten moeilijk maakte om de belastingen die de
overheid inde, of andere manieren om middelen voor het leger te
financieren, zoals inflatie, te beperken.

\subsection{\texorpdfstring{\textbf{Waarom klanten niet konden
domineren}}{Waarom klanten niet konden domineren}}\label{waarom-klanten-niet-konden-domineren}

In de moderne periode waren degenen die voor ``bescherming'' betaalden
niet in staat hun middelen te onthouden van de soeverein, zelfs niet in
gezamenlijk verzet, omdat dit enkel zou hebben geleid tot overheersing
door andere, mogelijk nog vijandigere staten. Dit was een duidelijke
overweging tijdens de Koude Oorlog. De klanten, of belastingbetalers,
die een onevenredig groot deel van de kosten van de overheid in de
leidende westerse industriële staten droegen, hadden geen mogelijkheid
om hoge belastingen te weigeren. Het resultaat zou zijn geweest dat zij
zichzelf zouden blootstellen aan totale confiscatie door de Sovjet-Unie
of een andere agressieve groep met de macht om geweld te organiseren.

\subsection{\texorpdfstring{\textbf{Industrialisme en
democratie}}{Industrialisme en democratie}}\label{industrialisme-en-democratie}

Op langere termijn zou de massademocratie wel eens een anachronisme
kunnen blijken te zijn dat het einde van het Industriële Tijdperk niet
lang zal overleven. Zeker is dat massademocratie en de natiestaat samen
opkwamen met de Franse Revolutie aan het einde van de achttiende eeuw,
waarschijnlijk als reactie op een stijging van het reële inkomen. Rond
1750 begonnen de inkomens in West-Europa in snel tempo te stijgen, deels
als gevolg van warmer weer. Dit viel samen met een periode van
technologische innovatie die de geschoolde arbeid van ambachtslieden
verdrong door machines die bediend konden worden door ongeschoolde
arbeiders, zelfs vrouwen en kinderen. Deze nieuwe industriële apparatuur
verhoogde de lonen van ongeschoolde arbeiders, waardoor de
inkomensverdeling gelijker werd.

Het cruciale keerpunt van de revolutie was wellicht niet, zoals vaak
wordt gedacht, het merkwaardige idee dat mensen geneigd zijn in opstand
te komen wanneer de omstandigheden verbeteren. Belangrijker was mogelijk
het feit dat toen de inkomens een bepaald niveau hadden bereikt, het
voor de vroegmoderne staat eindelijk praktisch werd om de particuliere
tussenpersonen en machtige magnaten, met wie eerder over middelen werd
onderhandeld, te omzeilen en over te stappen op een systeem van
``directe heerschappij'', waarin een nationale overheid rechtstreeks
zaken deed met individuele burgers, hen tegen steeds hogere tarieven
belastte en slecht gecompenseerde militaire dienst eiste in ruil voor de
verstrekking van diverse voordelen.175

Omdat de opkomende middenklasse in relatief korte tijd genoeg geld had
verzameld om te belasten, was het voor de heersers niet langer
essentieel, zoals voorheen wel het geval was, om te onderhandelen met
machtige grondbezitters of kooplieden die, zoals de historicus Charles
Tilly schreef, ``in staat waren om de vorming van een machtige staat te
verhinderen'' die ``hun bezittingen in beslag zou nemen en hun
transacties zou inperken.''176 Je kunt duidelijk zien waarom regeringen
succesvoller middelen konden onttrekken van de maatschappij toen zij te
maken kregen met miljoenen individuele burgers, in plaats van met een
handjevol heren, hertogen, graven, bisschoppen, huurlingen, vrije steden
en andere semi-soevereine entiteiten met wie de heersers van Europese
staten zich vóór het midden van de achttiende eeuw genoodzaakt zagen te
onderhandelen.

Stijgende reële inkomens stelden regeringen in staat een strategie te
hanteren die meer middelen onder hun controle bracht. Kleine bedragen
die via belastingen van miljoenen werden geïnd, konden meer inkomsten
opleveren dan grotere sommen die door enkele machtigen werden betaald.
Bovendien was het veel makkelijker om met de velen om te gaan dan met de
enkelen, die over het algemeen onwillig waren hun geld af te staan en
veel beter in staat waren zich te verzetten.

De doorsnee boer, kleine koopman of arbeider beschikte immers over
verwaarloosbaar weinig middelen in vergelijking met de staat zelf. Het
was aan de vooravond van de Franse Revolutie voor een doorsnee
particulier in West-Europa volstrekt onmogelijk om succesvol met de
staat te onderhandelen over het verlagen van zijn belastingtarief, of om
een doeltreffend verzet te organiseren tegen overheidsplannen en -beleid
die zijn belangen bedreigden. Maar dit is precies wat machtige
particuliere magnaten eeuwenlang hadden gedaan en zouden blijven doen.
Zij boden daadwerkelijk weerstand en onderhandelden met heersers,
waardoor hun vermogen om middelen op te eisen werd ingeperkt.

\begin{quote}
\emph{``Oorlogsvoering versnelde de overgang van indirect naar direct
bestuur. Vrijwel elke staat die oorlog voert ontdekt dat hij de
inspanning niet kan bekostigen uit zijn opgebouwde reserves en lopende
inkomsten. Bijna alle oorlogvoerende staten lenen grootschalig, verhogen
belastingen en eisen de middelen voor de strijd, inclusief mannen, op
van onwillige burgers die hun middelen liever anders aanwenden.''177 --
CHARLES TILLY}
\end{quote}

Het voorbeeld van Polen in het midden van de achttiende eeuw illustreert
dit perfect. In 1760 telde het Poolse nationale leger achttienduizend
soldaten. Dit was een magere strijdmacht vergeleken met de legers van de
naburige machthebbers in Oostenrijk, Pruisen en Rusland, van wie het
kleinste staande leger al zo'n 100.000 man telde. In feite was het
Poolse nationale leger in 1760 zelfs klein in vergelijking met andere
eenheden binnen Polen. De gezamenlijke strijdkrachten van de Poolse adel
telden dertigduizend man.178

Als de Poolse koning de mogelijkheid had gehad om direct belasting te
heffen bij miljoenen individuele Polen, in plaats van afhankelijk te
zijn van de bijdragen van de invloedrijke Poolse magnaten, zou de
centrale overheid ongetwijfeld aanzienlijk hogere inkomsten hebben
kunnen verwerven en dus een omvangrijker leger hebben kunnen
financieren.

Ten opzichte van individuele burgers, die niet de mogelijkheid hadden om
collectief met miljoenen tegelijk op te treden, bleken centrale
autoriteiten overal onoverwinnelijk machtig. De koning van Polen had in
1760 echter niet de mogelijkheid zijn burgers rechtstreeks te belasten.
Hij moest zijn toevlucht nemen tot de heren, rijke kooplieden en andere
notabelen, die een kleine, hechte groep vormden. Zij hadden wél de
mogelijkheid om gezamenlijk op te treden om de koning ervan te
weerhouden hun middelen zonder hun toestemming op te eisen. Aangezien de
Poolse adel over veel meer troepen beschikte dan hijzelf, stond de
koning niet sterk genoeg om zijn zin door te duwen.

Uiteindelijk bleek dit militaire nadeel, het onvermogen om de rijken en
machtigen te omzeilen bij het vergaren van middelen, doorslaggevend in
het Tijdperk van Geweld. Binnen enkele jaren hield Polen op te bestaan
als onafhankelijke staat. Het werd veroverd door Oostenrijk, Pruisen en
Rusland, drie landen met legers die elk vele malen groter waren dan de
kleine Poolse strijdmacht. In elk van die landen hadden de heersers
manieren gevonden om de mogelijkheden van rijke kooplieden en de adel om
hun middelen te beschermen te omzeilen.

\subsection{\texorpdfstring{\textbf{Na de Franse
Revolutie}}{Na de Franse Revolutie}}\label{na-de-franse-revolutie}

De Franse Revolutie leidde tot een nog grotere toename van de omvang van
legers, een feit dat de kracht van de democratische strategie aantoonde
in een periode waarin het rendement van geweld toenam. De afspraak die
regeringen vanaf de Franse Revolutie maakten, was om een ongekende mate
van betrokkenheid in het leven van gewone mensen te bieden, in ruil voor
hun deelname aan oorlogen, in plaats van huurlingen, en het belasten van
een steeds groter deel van hun stijgende inkomens. Zoals Tilly zei:

\begin{quote}
\emph{``De reikwijdte van de staat breidde zich ver buiten het militaire
kerngebied uit, en haar burgers begonnen aanspraken te maken op een zeer
breed scala aan bescherming, rechtspraak, productie en distributie.
Naarmate nationale wetgevers hun bereik ver voorbij het goedkeuren van
belastingen uitbreidden, werden ze geconfronteerd met eisen van alle
goed georganiseerde belangengroepen die door de staat werden geraakt of
geraakt konden worden. Direct bestuur en massale nationale politiek
groeiden samen en versterkten elkaar.''¹⁷⁹}
\end{quote}

Dezelfde logica die in de achttiende eeuw gold, bleef gelden tot 1989,
toen de Berlijnse Muur viel. Naarmate het industriële tijdperk vorderde,
bleven de inkomens voor ongeschoold werk stijgen, waardoor
massademocratie een nog effectievere methode werd om de onttrekking van
middelen te optimaliseren. Als gevolg hiervan groeide de overheid
voortdurend, met jaarlijks gemiddeld ongeveer een half procent extra
belasting op het inkomen in een gemiddeld industrieel land gedurende de
twintigste eeuw.

Tijdens het industriële tijdperk voorafgaand aan 1989 kwam democratie
naar voren als de meest effectieve regeringsvorm op militair vlak, juist
omdat democratie het moeilijk of onmogelijk maakte om effectieve grenzen
te stellen aan de confiscatie van middelen door de staat. De ruime
invoering van sociale voorzieningen aan iedereen nodigde een meerderheid
van de kiezers uit om feitelijk ambtenaar van de overheid te worden. Dit
werd het overheersende politieke kenmerk van alle leidende industriële
landen, omdat kiezers een onvoldoende sterke positie hadden om hun rol
als klant voor bescherming te vervullen. Niet alleen stonden ze
tegenover de agressieve dreiging van communistische systemen, die grote
middelen voor militaire doeleinden konden genereren omdat de staat de
gehele economie beheerde, maar echte controle over de overheid was als
belastingbetaler ook om een andere reden onpraktisch.

Miljoenen gewone burgers kunnen niet effectief samenwerken om hun
belangen te beschermen. Omdat de obstakels voor hun samenwerking groot
zijn en het verdedigen van de gemeenschappelijke belangen van de groep
op individueel niveau weinig oplevert, zullen miljoenen gewone burgers
minder succesvol het hoofd kunnen bieden tegen de overheid als dat
kleinere groepen met gunstigere prikkels dat kunnen.

Andere dingen gelijk, zou je dus verwachten dat een groter deel van de
totale middelen door de overheid wordt opgeëist in een massademocratie
dan in een oligarchie of in een systeem van gefragmenteerde
soevereiniteit waar magnaten militaire macht hadden en hun eigen legers
op de been brachten, zoals dat overal in het vroegmoderne Europa vóór de
achttiende eeuw het geval was.

Een cruciale, zij het zelden onderzochte reden voor de groei van
democratische overheden in de westerse wereld is dus het relatieve
belang van onderhandelingskosten tijdens een periode waarin geweld
steeds meer loonde. Het kostte gewoon meer om middelen bij enkelen te
halen dan bij velen.

Een relatief kleine, elitegroep van rijken vormt een coherenter en
effectiever geheel dan een grote massa burgers. De kleine groep heeft
sterkere prikkels om samen te werken. Zij zal bijna onvermijdelijk
succesvoller zijn in het beschermen van haar belangen dan een grote
massa.¹⁸⁰ En zelfs als de meeste leden van de groep ervoor kiezen niet
samen te werken aan een gezamenlijke actie, kunnen een paar rijken
genoeg middelen inzetten om het werk gedaan te krijgen.

Met democratische besluitvorming kon de natiestaat veel vollediger macht
uitoefenen over miljoenen personen, die niet gemakkelijk konden
samenwerken om collectief hun belangen te verdedigen, terwijl een
kleinere groep de organisatorische hindernissen gemakkelijker kon
overwinnen en zo hun geconcentreerde belangen kon beschermen. Democratie
bood het extra voordeel van een gelegitimeerd besluitvormingsmechanisme,
waarmee de staat toegang kreeg tot de middelen van de rijken zonder dat
hun directe instemming nodig was. Het vulde de natiestaat aan omdat het
de concentratie van militaire macht in de handen van de machthebbers
vergemakkelijkte, in een tijd waarin de omvang van de ingezette kracht
belangrijker was dan de efficiëntie van de mobilisatie.

De Franse Revolutie bewees dit duidelijk. Doordat het de omvang van de
militaire macht op het slagveld vergrootte, hadden andere concurrerende
natiestaten weinig andere keuze dan een vergelijkbare organisatie na te
streven, waarvan de legitimiteit uiteindelijk verbonden werd aan
democratische besluitvorming. Samengevat: de democratische natiestaat
slaagde in de afgelopen twee eeuwen om deze verborgen redenen:

\begin{enumerate}
\def\labelenumi{\arabic{enumi}.}
\item
  Geweld loonde steeds meer, waardoor de schaal van de macht
  \textgreater{} belangrijker werd als bestuursprincipe dan efficiëntie.
\item
  Inkomens stegen voldoende boven het bestaansminimum, waardoor de
  \textgreater{} staat de mogelijkheid kreeg om grote hoeveelheden aan
  middelen te \textgreater{} innen zonder te hoeven onderhandelen met
  machtige magnaten die \textgreater{} weerstand konden bieden.
\item
  Democratie bleek voldoende verenigbaar met de werking van vrije
  \textgreater{} markten om het toenemen van de welvaart vol te kunnen
  houden.
\item
  Democratie faciliteerde een dominante overheid, bestuurd door haar
  \textgreater{} ``werknemers'', waardoor het moeilijk werd de uitgaven,
  inclusief \textgreater{} militaire uitgaven, te beperken.
\item
  Democratische besluiten bleken een effectief middel tegen het
  \textgreater{} vermogen van de rijken om gezamenlijk op te treden
  tegen het \textgreater{} vermogen van de natiestaat om belastingen te
  heffen of op andere \textgreater{} manieren inbreuk te maken op hun
  eigendom.
\end{enumerate}

Democratie werd de winnende militaire strategie omdat het het
vergemakkelijkte om meer middelen in handen van de staat te brengen. In
vergelijking met andere vormen van soevereiniteit die hun legitimiteit
ontleenden aan andere principes, zoals de feodale heffing, het goddelijk
recht van koningen, religieuze plicht of de vrijwillige bijdragen van de
rijken, werd massademocratie militair de krachtigste, omdat het de meest
zekere manier was om middelen te verzamelen in een industriële economie.

\begin{quote}
\emph{``De natie, als een cultureel gedefinieerde gemeenschap, is de
hoogste symbolische waarde van de moderniteit; ze is gezegend met een
quasi-heilig karakter dat alleen door religie wordt geëvenaard. In feite
is dit quasi-heilige karakter afgeleid van religie. In de praktijk is de
natie ofwel de moderne, seculiere vervanger van religie geworden, of
haar machtigste bondgenoot. In de moderne tijd worden de
gemeenschapsgevoelens die door de natie worden opgewekt hoog gewaardeerd
en gezocht als basis voor groepsloyaliteit\ldots{} Dat de moderne staat
hier vaak de begunstigde van is, hoeft nauwelijks te verbazen, gezien
haar opperste macht.''¹⁸¹\\
-- JOSEPH R. LLOBERA}
\end{quote}

\subsection{\texorpdfstring{\textbf{Nationalisme}}{Nationalisme}}\label{nationalisme}

Veel van hetzelfde kan gezegd worden van nationalisme, dat een
uitvloeisel werd van massademocratie. Staten die nationalisme konden
inzetten, ontdekten dat ze grotere legers tegen lagere kosten konden
mobiliseren. Nationalisme was een uitvinding die het voor een staat
mogelijk maakte om haar militaire effectiviteit op te schalen. Net als
de politiek zelf is nationalisme grotendeels een moderne uitvinding.
Zoals socioloog Joseph Llobera in zijn rijk gedocumenteerde boek over de
opkomst van nationalisme heeft aangetoond, is de natie een ingebeelde
gemeenschap die in grote mate tot stand kwam als een manier om
staatsmacht te mobiliseren tijdens de Franse Revolutie. Zoals hij stelt:
``In de moderne betekenis van de term bestaat nationaal bewustzijn pas
sinds de Franse Revolutie, sinds het moment waarop in 1789 de
Grondwetgevende Vergadering het volk van Frankrijk gelijkstelde met de
Franse natie.''¹⁸²

Nationalisme maakte het eenvoudiger om macht te mobiliseren en controle
uit te oefenen over grote aantallen mensen. Natiestaten werden gevormd
door het benadrukken van kenmerken die mensen gemeen hadden, met name de
gesproken taal. Dit vergemakkelijkte heerschappij zonder tussenkomst van
tussenpersonen. Het vereenvoudigde de taken van de bureaucratie.
Besluiten die slechts in één taal hoefden te worden uitgevaardigd,
konden sneller en met minder verwarring worden verspreid dan degenen die
vertaald moesten worden in alle talen van de toren van Babel.
Nationalisme maakte het dus goedkoper om grotere gebieden te beheersen.
Voor het nationalisme had de vroegmoderne staat de hulp nodig van heren,
hertogen, graven, bisschoppen, vrije steden en andere corporatieve en
etnische tussenpersonen, van belastingpachters tot militaire
leveranciers en huurlingen, om inkomsten te innen, troepen te werven en
andere overheidstaken uit te voeren.

Nationalisme was ook doorslaggevend voor het verlagen van de kosten voor
het mobiliseren van militair personeel door van groepsidentificatie met
de belangen van de staat aan te moedigen. Er was zo'n aanzienlijk
voordeel te behalen door groepsgevoel te koppelen aan de belangen van de
staat dat de meeste staten, zelfs de zogenaamd internationalistische
Sovjet-Unie, op nationalisme uitkwamen als aanvullende ideologie.

In een langere historische context is nationalisme net zo goed een
anomalie als de staat zelf. Zoals historicus William McNeill heeft
gedocumenteerd, waren poly-etnische soevereiniteiten vroeger de norm.¹⁸³

In McNeill's woorden: ``Het idee dat een overheid alleen rechtmatig over
burgers van een enkele ethnos zou moeten heersen, begon zich tegen het
einde van de Middeleeuwen in West-Europa te ontwikkelen.''¹⁸⁴ Een vroege
nationalistische entiteit was de Pruisische Liga (Preußischer Bund), die
in 1440 werd gevormd als verzet tegen het bewind van de Duitse Orde.
Sommige kenmerken van de orde zijn eerder genoemd als een uiterst
voorbeeld van een soevereiniteit die niet leek op de natiestaat. De
Duitse Orde was een soort gecharterde onderneming waarvan bijna geen
enkel lid afkomstig was uit Pruisen. Het hoofdkwartier verschoof in de
loop der tijd van Bremen en Lübeck naar Jeruzalem, Akko, Venetië en
uiteindelijk Marienberg aan de Wisła. Op een bepaald moment bestuurde
zij zelfs het district Burzenland in Transsylvanië. Het is dan ook niet
verrassend dat een soevereiniteit die zo weinig leek op een staat het
voorwerp werd van een van de eerste pogingen om nationaal gevoel te
mobiliseren als factor in de organisatie van macht. Het volgende laat
echter zien hoe anders het vroege nationalisme was ten opzichte van
latere varianten: de Duitstalige edelen van de Pruisische Liga
verzochten de koning van Polen om Pruisen onder Pools gezag te plaatsen,
grotendeels omdat de Poolse koning toen een relatief zwakke monarch was
waarvan niet verwacht werd dat hij met dezelfde strengheid zou regeren
als de Duitse Orde.

Nationalisme, in zijn vroege verschijningsvormen, kwam net voor de
Buskruitrevolutie op. Het bleef groeien naarmate de vroegmoderne staat
zich verder ontwikkelde, en werd erg belangrijk ten tijde van de Franse
Revolutie. Volgens ons heeft het nationalisme als krachtig idee zijn
terugtocht al ingezet.Het bereikte waarschijnlijk zijn hoogtepunt met
Woodrow Wilsons poging om aan elk etnisch volk in Europa een eigen staat
toe te wijzen na afloop van de Eerste Wereldoorlog. Tegenwoordig is het
een reactionaire kracht, aangewakkerd in regio's met dalende inkomens en
teruglopende vooruitzichten, zoals Servië.

Zoals we later zullen bespreken, verwachten wij dat nationalisme een
belangrijk onderwerp zal worden waar mensen met weinig vaardigheden,
nostalgisch naar dwang, zich achter zullen scharen wanneer de
welvaartsstaat instort in de westerse democratieën. Je hebt nog niets
gezien. Voor de meeste mensen in het Westen leek de ondergang van het
communisme relatief onschuldig. Je hebt een daling van militaire
uitgaven gezien, een instorting van de aluminiumprijzen, en een nieuwe
bron van ijshockeyspelers voor de NHL. Dat is het goede nieuws. Nieuws
dat de meeste mensen die in de twintigste eeuw volwassen werden, konden
toejuichen, zeker als ze fan waren van ijshockey. Het grootste deel van
het nieuws dat minder populair zal blijken, moet nog komen.

Nu het Industriële Tijdperk tot zijn einde komt, houden de megapolitieke
omstandigheden die democratie mogelijk maakten snel op te bestaan.
Daarom is het twijfelachtig dat massademocratie en de verzorgingsstaat
lang zullen overleven in de nieuwe megapolitieke omstandigheden van het
informatietijdperk.

\emph{``Het Congres was geen tempel van de democratie, maar een markt om
wetten te verhandelen.''\\
-- ALBERTO FUJIMORI, president van Peru}

Toekomstige historici zouden zelfs kunnen melden dat we de eerste
postmoderne staatsgreep al hebben gezien, de opmerkelijke verzegeling
van het congres in Peru in 1993. Dit was nauwelijks een gebeurtenis die
veel aandacht trok in de leidende industriële democratieën. Maar het zou
achteraf belangrijker kunnen blijken dan conventionele analisten
vermoeden. De weinigen die erover nadachten, zagen het vooral als nog
een machtsgreep van het soort dat Latijns-Amerika doorheen de
geschiedenis treurig genoeg al vaak heeft gekend. Wij zien het echter
mogelijk als de eerste stap richting de delegitimering van een
bestuursvorm waarvan de onmiddellijke megapolitieke bestaansreden met de
overgang naar het informatietijdperk is beginnen te verdwijnen.
Fujimori's sluiting van het congres is een symptoom van de uiteindelijke
waardevermindering van politieke beloften. Een soortgelijk lot kan
andere parlementen te wachten staan zodra hun krediet is uitgeput.

De technologische verschuiving die het industrialisme ondermijnt, heeft
veel landen opgescheept met regeringen die niet langer functioneren, of
slecht functioneren. Vooral parlementen lijken steeds disfunctioneler te
worden. Ze produceren wetten die vijftig jaar geleden misschien slechts
dom waren, maar die vandaag de dag gevaarlijk zijn. Dit was spectaculair
duidelijk in Peru, waar de interne soevereiniteit van de staat tegen
1993 vrijwel ingestort was.

\begin{quote}
\emph{``Aanvallen, ontvoeringen, verkrachtingen en moorden gingen hand
in hand met steeds agressiever rijgedrag en onveilige straten. De
politie heeft geleidelijk de controle over de situatie verloren en
sommige van haar leden zijn betrokken geraakt bij schandalen en
doorgewinterde criminelen geworden\ldots{} Mensen zijn geleidelijk
gewend geraakt om buiten de wet te leven. Diefstal, illegale
inbeslagnames en fabrieksbezettingen zijn alledaagse verschijnselen
geworden.''¹⁸⁵\\
-- HERNANDO DE SOTO}
\end{quote}

\subsection{\texorpdfstring{\textbf{Peru in
ruïnes}}{Peru in ruïnes}}\label{peru-in-ruuxefnes}

In zekere zin was Peru in 1993 geen moderne natiestaat meer. Het had nog
een vlag en een leger, maar de meeste instituties lagen in puin. Zelfs
de gevangenissen waren overgenomen door de gedetineerden. Deze
desintegratie kent meerdere oorzaken, maar de meeste verklaringen van
deskundigen missen de kern. Peru viel namelijk al vroeg ten prooi aan de
technologische veranderingen die gesloten economieën disfunctioneel
maken en de centrale autoriteit ondermijnen. Bovendien verergeren deze
megapolitieke spanningen wanneer besluitvormingsorganen, zoals het
Peruaanse congres, door perverse prikkels worden gedwongen de problemen
eigenlijk te verergeren in plaats van op te lossen.

De representatieve democratie in Peru was te vergelijken met een paar
valse dobbelstenen. Democratie was ongeëvenaard als middel om
staatsmacht uit te breiden, maar toen veranderende omstandigheden een
verandering van de macht eisten, werden de eigenschappen die haar in de
oude megapolitieke context zo succesvol hadden gemaakt, een bron van
toenemend disfunctioneren. De wetten die het congres aannam, verzwakten
razendsnel elk fundament voor waarde en respect voor de wet. Zoals de
Soto in \emph{The Other Path} verwoordt: `Kleine belangengroepen ruziën
met elkaar, veroorzaken faillissementen en beïnvloeden ambtenaren,
overheden kennen privileges toe, en de wet wordt ingezet om veel meer te
geven en af te nemen dan de moraal toelaat.' {[}18{]} Een congres dat,
zoals in Peru, volledig beheerst wordt door belangengroepen, heeft
dezelfde morele status als een bende die gestolen goederen veilt. Dit
maakte de vrije markt ontoegankelijk, waardoor de wet haar
geloofwaardigheid verloor. Zoals de Soto schrijft over de periode vóór
Fujimori:

\begin{quote}
\emph{Een complete omkering van doelen en middelen heeft het
maatschappelijke leven in Peru zo verstoord dat bepaalde daden, hoewel
wettelijk strafbaar, niet langer door het collectieve bewustzijn worden
veroordeeld. Smokkelen is daar een treffend voorbeeld van. Iedereen, van
aristocratische dames tot de meest nederige mannen, verwerft gesmokkelde
goederen. Niemand heeft er gewetensbezwaren tegen; integendeel, men
beschouwt het als een test van persoonlijke vindingrijkheid of als een
vorm van wraak op de staat. Deze infiltratie van geweld en criminaliteit
in het dagelijks leven gaat gepaard met toenemende armoede en ontbering.
In het algemeen is het reële gemiddelde inkomen van de Peruanen de
afgelopen tien jaar gestaag gedaald en bevindt het zich nu op het niveau
van twintig jaar geleden. Overal hopen bergen afval zich op. Dag en
nacht omsingelen legioenen bedelaars, autowassers en aaseters de
voorbijgangers op zoek naar geld. Geestelijk zieken zwermen naakt door
de straten en stinken naar urine. Kinderen, alleenstaande moeders en
gehandicapten bedelen op elke straathoek om aalmoezen. Het traditionele
centralisme van onze samenleving heeft duidelijk niet voldaan aan de
vele behoeften van een land in transitie.{[}19{]}}
\end{quote}

De Soto omschreef het loslaten van de groteske juridische economie ten
gunste van de zwarte markt, een ontwikkeling die al in gang was gezet
voordat Fujimori het congres opsloot, als `een onzichtbare revolutie.'

Wij staan positief tegenover de voordelen van de vrije markt, maar zijn
minder enthousiast over een samenleving waarin de wet net zo weinig
waard is als het geld. De wereld die de Soto in Peru schetste vóór 1993,
was een `Clockwork Orange'-wereld waarin overmatig gecentraliseerde en
dysfunctionele overheidsinstellingen de maatschappij letterlijk
vernietigden.

Dit was immers wat Fujimori wilde veranderen. Hij verlaagde de inflatie
drastisch door de geldpersen stil te leggen. Ook slaagde hij erin
vijftigduizend ambtenaren te ontslaan en diverse subsidies in te perken.
Hij begon met het in evenwicht brengen van de begroting. Zijn
hervormingsprogramma omvatte uitgebreide plannen om vrije markten tot
stand te brengen en de industrie te privatiseren. Maar, net als in de
voormalige Sovjet-Unie, had hij in 1993 de meest cruciale elementen van
zijn hervorming, waaronder de eerste ronde grootschalige privatiseringen
van staatsbanken, mijnbouwbedrijven en nutsbedrijven, nog niet
gerealiseerd. In plaats van deze broodnodige maatregelen te realiseren,
probeerde het Peruaanse congres, net als het Russische congres dat
Jeltsins hervormingen in Moskou blokkeerde, een stap achteruit te
zetten. Hun plan luidde: subsidies herstellen vanuit een lege schatkist,
de lonen spekken en alle gevestigde belangen beschermen, met name de
bureaucratie; precies wat je zou verwachten van een overheid die door
haar werknemers wordt beheerst.

Fujimori beweerde dat het congres van Peru aarzelend en corrupt was, een
uitspraak waar vrijwel iedereen het mee eens was. Hij voegde eraan toe
dat de aarzelende en corrupte werkwijze in het congres iedere poging om
de ineenstortende economie van Peru te hervormen of de gewelddadige
aanvallen van narco-terroristen en nihilistische guerrilla's van
\emph{het Sendero Luminoso (Shining Path)} tegen te gaan, onmogelijk
maakte.

\subsection{\texorpdfstring{\textbf{De
70-procentoplossing}}{De 70-procentoplossing}}\label{de-70-procentoplossing}

Dus sloot Fujimori het Congres, een daad die erop kon wijzen dat hij
even autoritair was als veel voorgaande Latijns-Amerikaanse leiders.
Maar wij dachten, en zeiden dat destijds ook, dat Fujimori terecht een
fundamentele belemmering voor hervorming had geïdentificeerd. De
buitensporige officiële lofzangen op het Peruaanse congres door
Amerikaanse redactieschrijvers en functionarissen van het State
Department werden niet gedeeld door het volk van Peru. Terwijl
Noord-Amerikanen deden alsof het Peruaanse congres de belichaming van
vrijheid en beschaving was, juichte het Peruaanse volk. De populariteit
van president Fujimori steeg boven de 70 procent toen hij het congres
naar huis stuurde. En later werd hij met een verpletterende meerderheid
herkozen voor een tweede termijn. De meeste burgers zagen hun wetgevende
macht blijkbaar meer als een hindernis voor hun welzijn dan als een
belichaming van hun rechten. In 1994 bereikte de reële economische groei
in Peru 12,9 procent, de hoogste ter wereld.

\subsection{\texorpdfstring{\textbf{Deflatie van politieke
beloften}}{Deflatie van politieke beloften}}\label{deflatie-van-politieke-beloften}

Wij zagen de onrust in Peru niet als een terugkeer naar de dictaturen
van het verleden maar eerder als een vroeg voorbeeld van een bredere
overgangscrisis. Het is te verwachten dat in tal van landen crises van
wanbestuur ontstaan wanneer politieke beloften niet worden nagekomen en
overheden hun krediet uitgeput hebben. Uiteindelijk zullen nieuwe
institutionele vormen moeten ontstaan die in staat zijn de vrijheid te
bewaren onder de nieuwe technologische omstandigheden, terwijl zij
tegelijkertijd uitdrukking en inhoud geven aan de gemeenschappelijke
belangen die alle burgers delen.

Bijna niemand denkt nog na over de onverenigbaarheid tussen sommige
instituties van de industriële overheid en de megapolitiek van de
postindustriële samenleving. Of deze tegenstrijdigheden nu expliciet
worden erkend of niet, hun gevolgen zullen steeds duidelijker worden
naarmate voorbeelden van politiek falen zich wereldwijd opstapelen.
Overheidsinstellingen die in de moderne tijd zijn ontstaan,
weerspiegelen de megapolitieke omstandigheden van een of meerdere eeuwen
geleden. Het informatietijdperk zal nieuwe
vertegenwoordigingsmechanismen vereisen om chronische disfunctionaliteit
en zelfs sociale ineenstorting te vermijden.

Toen de Berlijnse Muur in 1989 viel, betekende dat niet alleen het einde
van de Koude Oorlog; het was ook het signaal aan de buitenwereld dat er
een stille aardbeving in de fundamenten van de macht in de wereld gaande
was. Het betekende het einde van de lange periode van stijgende
rendementen op geweld. De val van het communisme, die wij in 1987
voorspelden in \emph{Blood in the Streets} en nog eerder in onze
maandelijkse nieuwsbrief \emph{Strategic Investment}, was niet louter
een verwerping van een ideologie. Het diende als zichtbaar symbool voor
de meest betekenisvolle verandering in de geschiedenis van geweld van de
laatste vijfhonderd jaar. Indien onze analyse juist is, zal de
maatschappelijke structuur zich aanpassen aan de groeiende
inefficiënties die gepaard gaan met het grootschalig inzetten van
geweld. De grenzen waarbinnen de toekomst zich moet afspelen, zijn
hertekend.

{[}1{]} Ibid., p.~39.

{[}2{]} The Cambridge Ancient History, op. cit., pp.~263--264.

{[}3{]} Cook et al., op. cit., p.~268.

{[}4{]} Lane, `Gevolgen van georganiseerd geweld', op. cit., p.~406.

{[}5{]} Ibid.

{[}6{]} 173

{[}7{]} Tilly, op. cit., pp.~96--126.

{[}8{]} Ibid., p.~130.

{[}9{]} Ibid., p.~110.

{[}10{]} Dit voorbeeld is te vinden in Ibid., p.~139.

{[}11{]} Ibid., p.~115.

{[}12{]} Zie Mancur Olson, \emph{The Logic of Collective Action}
(Cambridge: Harvard University Press, 1965).

{[}13{]} Josep R. Llobera, \emph{The God of Modernity: The Development
of Nationalism in Western Europe} (Oxford: Berg Publishers, 1994), pp.
ix--x.

{[}14{]} Ibid., p.~xiii.

{[}15{]} Zie William McNeill, \emph{Polyethnicity and National Unity in
World History} (Toronto: University of Toronto Press, 1986).

{[}16{]} Ibid., p.~7.

{[}17{]} Hernando de Soto, \emph{The Other Path} (New York: Harper \&
Row, 1989).

{[}18{]} Ibid.

{[}19{]} Ibid., p.~6.

\bookmarksetup{startatroot}

\chapter{\texorpdfstring{\textbf{De megapolitiek van het
informatietijdperk}}{De megapolitiek van het informatietijdperk}}\label{de-megapolitiek-van-het-informatietijdperk}

\emph{De triomf van efficiëntie boven macht}

\begin{quote}
\emph{``\ldots het is gecomputeriseerde informatie, niet mankracht of
massaproductie, die de Amerikaanse economie steeds meer aandrijft en die
oorlogen zal winnen in een wereld die is bekabeld voor 500 TV-kanalen.
De gecomputeriseerde informatie bestaat in de cyberspace, de nieuwe
dimensie die is ontstaan door de eindeloze productie van
computernetwerken, satellieten, modems, databanken en het publieke
internet.''¹⁸⁸ -- NEIL MUNRO}
\end{quote}

Op 30 december 1936 bezetten medewerkers die hogere salarissen eisten
met geweld twee hoofdvestigingen van General Motors in Flint, Michigan.
Ze legden machines stil, zetten de lopende banden uit en deden alsof ze
thuis waren. Werknemers die waren aangenomen om de fabrieken te
bedienen, gingen letterlijk zitten in een industriële confrontatie die
vele weken zou duren. Het was een drama, onderbroken door gewelddadige
rellen en de wisselende loyaliteit van de politie, de militie van
Michigan en politieke figuren op alle bestuursniveaus. Toen hun eisen
weinig vooruitgang boekten, sloeg de vakbond op 1 februari 1937 opnieuw
toe.

Vakbondsactivisten namen met geweld GM's Chevrolet-fabriek in Flint
over. Door de belangrijkste installaties van General Motors te bezetten
en te sluiten, slaagden de arbeiders erin om de productieve capaciteit
van het bedrijf bijna volledig te verlammen. In de tien dagen na de
inname van de derde fabriek produceerde GM in de Verenigde Staten
slechts 153 auto's.

We halen dit nieuwsbericht van zestig jaar geleden aan om de revolutie
in megapolitieke omstandigheden die nu gaande is, scherper te belichten.
De sit-downstaking bij GM gebeurde nog binnen het leven van sommige
lezers van dit boek. Toch geloven wij dat sit-downstakingen in het
informatietijdperk net zo anachronistisch zullen blijken te zijn als
slaven die door de woestijn sjouwen met enorme stenen om grafpiramides
voor de farao's te bouwen. Vakbonden en hun intimidatietactieken werden
in het industriële tijdperk zo vertrouwd dat ze een onbetwist deel van
het sociale landschap vormden. Ze berustten echter op speciale
megapolitieke omstandigheden die nu snel verdwijnen. Er zullen geen
Chevrolets en geen UAW (de vakbond in kwestie) zijn om te staken op de
Informatie-Supersnelweg.

Overheden zullen net als hun tegenhangers, de vakbonden, in verval
raken. Geïnstitutionaliseerde dwang van het soort dat een cruciale rol
speelde in de twintigste-eeuwse samenleving zal niet langer mogelijk
zijn. Technologie veroorzaakt een diepgaande verandering in de logica
van afpersing en bescherming.

\begin{quote}
\emph{``\ldots er zal geen eigendom zijn, geen heerschappij, geen
onderscheid tussen het mijne en het uwe; maar alleen datgene wat ieder
kan verkrijgen, en zolang hij het kan behouden.''¹⁸⁹ -- THOMAS HOBBES}
\end{quote}

\subsection{\texorpdfstring{\textbf{Afpersing en
bescherming}}{Afpersing en bescherming}}\label{afpersing-en-bescherming}

Door de geschiedenis heen is geweld een dolk geweest die op het hart van
de economie was gericht. Zoals Thomas Schelling scherp opmerkte: ``De
macht om pijn te doen, om dingen te vernietigen die iemand dierbaar
zijn, om pijn en verdriet toe te brengen, is een vorm van
onderhandelingsmacht, die niet makkelijk is te gebruiken, maar vaak
wordt toegepast. In de onderwereld vormt het de basis voor chantage,
afpersing en ontvoering; in de commerciële wereld voor boycots,
stakingen en uitsluitingen\ldots{} Het is vaak de basis voor discipline,
civiel en militair; en goden gebruiken het om discipline af te
dwingen.''¹⁹⁰

Het vermogen van de staat om belastingen te heffen, hangt af van
dezelfde kwetsbaarheden die ook ten grondslag liggen van particuliere
afpersing en chantage. Hoewel we het doorgaans niet in deze termen zien,
geeft het aandeel van de middelen dat dwingend wordt beheerd en besteed,
via misdaad en overheid, een ruwe maatstaf van het megapolitieke
evenwicht tussen afpersing en bescherming.

Als technologie de bescherming van bezittingen moeilijk maakt, zal
misdaad waarschijnlijk wijdverspreid zijn, net als vakbondsactiviteit.
Onder zulke omstandigheden zal de overheid dus een hoge premie kunnen
afdwingen voor het leveren van bescherming. Belastingen zullen dus hoog
zijn. Wanneer belastingen laag zijn en lonen vooral door marktwerking
worden bepaald in plaats van door politieke inmenging of dwang, zorgt
technologie ervoor dat de verhoudingen verschuiven in het voordeel van
bescherming, en niet van afpersing.

De technologische disbalans tussen afpersing en bescherming bereikte een
extreem aan het einde van het derde kwart van de twintigste eeuw. In
sommige geavanceerde westerse samenlevingen werd meer dan de helft van
de middelen door regeringen opgeëist. De inkomens van een groot deel van
de bevolking werden ofwel bij decreet vastgesteld, of beïnvloed door
dwang, bijvoorbeeld door stakingen en dreiging met andere vormen van
geweld. De verzorgingsstaat en de vakbond waren beide producten van
technologie, die samen profiteerden van de overwinning van macht boven
efficiëntie in de twintigste eeuw. Ze hadden niet kunnen bestaan zonder
de militaire en civiele technologieën die het rendement op geweld in het
industriële tijdperk verhoogden.

Het vermogen om bezittingen te creëren heeft altijd enige kwetsbaarheid
voor afpersing met zich meegebracht. Hoe waardevoller de bezittingen die
werden gecreëerd of beheerd, des te hoger de prijs die op de een of
andere manier betaald moest worden. Of je betaalde iedereen die machtig
genoeg was om je met geweld af te persen, of je betaalde voor militaire
macht die in staat was om elke poging tot afpersing met brute kracht te
verslaan.

\begin{quote}
\emph{``Er zal geen geweld meer worden gehoord in uw land, geen verderf
of vernieling binnen uw grenzen\ldots{}'' --} \emph{JESAJA 60:18}
\end{quote}

\subsection{\texorpdfstring{\textbf{De wiskunde van
bescherming}}{De wiskunde van bescherming}}\label{de-wiskunde-van-bescherming}

Nu kan de dolk van geweld binnenkort mogelijk worden getemd.
Informatie­technologie belooft het evenwicht tussen bescherming en
afpersing ingrijpend te veranderen, waardoor de bescherming van
bezittingen in veel gevallen veel eenvoudiger wordt en afpersing
moeilijker. De technologie van het informatietijdperk maakt het mogelijk
bezittingen te creëren die buiten het bereik van veel vormen van dwang
liggen.

Deze nieuwe asymmetrie tussen bescherming en afpersing berust op een
fundamentele wiskundige waarheid: het is gemakkelijker om te
vermenigvuldigen dan om te delen. Hoewel deze waarheid eenvoudig lijkt,
bleven de diepgaande gevolgen ervan onzichtbaar tot de komst van
microprocessors. Computers hebben in het afgelopen decennium miljarden
keren meer berekeningen uitgevoerd dan in de hele voorgaande
geschiedenis van de wereld.

Deze sprong in rekenkracht heeft ons voor het eerst in staat gesteld
enkele universele kenmerken van complexiteit te doorgronden. Wat de
computers laten zien, is dat complexe systemen alleen van onderaf kunnen
worden opgebouwd en begrepen. Het vermenigvuldigen van priemgetallen is
eenvoudig. Maar het uit elkaar halen van complexiteit door het product
van grote priemgetallen te ontbinden is vrijwel onmogelijk. Kevin Kelly,
redacteur van \emph{Wired}, verwoordt het zo:\\
``Het vermenigvuldigen van meerdere priemgetallen tot een groter product
is eenvoudig; elk kind op de basisschool kan dat, maar alle
supercomputers in de wereld stikken terwijl ze een product terug
proberen te ontleden in zijn eenvoudige priemgetallen.''¹⁹¹

\subsection{\texorpdfstring{\textbf{De logica van complexe
systemen}}{De logica van complexe systemen}}\label{de-logica-van-complexe-systemen}

De cybereconomie zal onvermijdelijk worden gevormd door deze
fundamantele wiskundige waarheid. Het manifesteert zich al in sterke
encryptiemethoden. Zoals we later in dit hoofdstuk zullen bespreken,
zullen deze algoritmen het mogelijk maken om een nieuw, beschermd domein
van cyberhandel te creëren waarin het geweldsmonopolie sterk wordt
teruggedrongen. Het evenwicht tussen afpersing en bescherming zal
drastisch doorslaan in de richting van bescherming. Dit zal de opkomst
bevorderen van een economie die meer steunt op spontane, adaptieve
mechanismen en minder op bewuste besluitvorming en toewijzing van
middelen via bureaucratie. Het nieuwe systeem, waarin bescherming
centraal zal staan, zal sterk verschillen van datgene wat voortkwam uit
de overheersing van dwang in het industriële tijdperk.

\subsection{\texorpdfstring{\textbf{Command-and-control systemen zijn
primitief}}{Command-and-control systemen zijn primitief}}\label{command-and-control-systemen-zijn-primitief}

We schreven in \emph{The Great Reckoning} dat de computer ons in staat
stelt om de complexiteit in een hele reeks systemen te ``zien'' die
voorheen onzichtbaar was.¹⁹² Geavanceerde rekenkracht maakt het niet
alleen mogelijk de dynamiek van complexe systemen beter te begrijpen,
maar ook om deze complexiteiten op productieve manieren te benutten. In
zekere zin is dit niet eens een keuze, maar het is onvermijdelijk als de
economie verder wil komen dan het starre centraal gestuurde
ontwikkelingsstadium. Een dergelijk systeem, dat afhankelijk is van
lineaire relaties, is in wezen primitief. De inbeslagname van middelen
door de overheid trekt onvermijdelijk middelen weg van complexe
toepassingen met hoge waarde naar primitieve toepassingen met lage
waarde. Het is een proces dat wordt beperkt door dezelfde wiskundige
asymmetrie die voorkomt dat het product van grote priemgetallen kan
worden ontbonden. Het verdelen van de buit kan nooit meer zijn dan
primitief.

\subsection{\texorpdfstring{\textbf{Alles wordt
complexer}}{Alles wordt complexer}}\label{alles-wordt-complexer}

Overal in het universum zien we systemen die naarmate ze evolueren
grotere complexiteit bereiken. Dit geldt in de astrofysica. Het geldt in
een plas water. Laat regenwater in een lage plek liggen en het zal
complexer worden. Geavanceerde systemen van elke soort zijn complexe
adaptieve systemen zonder een centrale autoriteit. Elk complex systeem
in de natuur, waarvan de markteconomie de meest evidente sociale
manifestatie is, berust op een verscheidenheid aan competenties.
Systemen die in de meest uiteenlopende omstandigheden het best
functioneren, danken hun robuustheid aan een spontane orde die ruimte
laat voor het onverwachte. Het leven zelf is zo'n complex systeem.
Miljarden potentiële genetische combinaties produceren één enkel
menselijk individu. Het ordenen daarvan zou elke bureaucratie doen
vastlopen.

Vijfentwintig jaar geleden kon dat slechts een intuïtie zijn. Vandaag is
het aantoonbaar. Naarmate computers ons meer inzicht geven in de
wiskunde achter kunstmatig leven, groeit ons begrip van de wiskunde van
het werkelijke leven, dat berust op biologische complexiteit. Dankzij
informatietechnologie kunnen deze geheimen van complexiteit worden
toegepast om economieën om te vormen tot steeds complexere systemen. Het
internet en het World Wide Web hebben al kenmerken aangenomen van een
organisch systeem, zoals Kevin Kelly suggereert in \emph{Out of Control:
The New Biology of Machines, Social Systems, and the Economic World}.¹⁹³
In zijn woorden is de natuur ``een ideeënfabriek. Vitale,
postindustriële paradigma's liggen verborgen in elke mierenhoop in de
jungle\ldots{} De grootschalige overdracht van het biologische naar
machines zou ons diep moeten verwonderen. Wanneer de vereniging van het
geboren en het gemaakte voltooid is, zullen onze creaties leren, zich
aanpassen, zichzelf genezen en evolueren. Dit is een kracht waar we
nauwelijks van hebben durven dromen.''¹⁹⁴

De gevolgen van deze ``grootschalige overdracht van het biologische naar
machines'' zullen inderdaad verstrekkend zijn. Sociale systemen hebben
altijd een sterke neiging gehad om de kenmerken van de heersende
technologie te imiteren. Marx had hierover gelijk. Gigantische fabrieken
vielen samen met het tijdperk van de grote overheid. Microprocessing is
nu bezig met het miniaturiseren van instituties. Als onze analyse klopt,
zal de technologie van het Informatietijdperk uiteindelijk een economie
scheppen die beter geschikt is om de voordelen van complexiteit te
benutten.

Toch zijn de megapolitieke dimensies van zo'n verandering zo weinig
begrepen dat zelfs de meesten die het wiskundige belang ervan hebben
erkend, dit op een anachronistische manier hebben gedaan. Het is
eenvoudigweg moeilijk om volledig te bevatten en te internaliseren dat
technologische verandering in de komende jaren de meeste politieke
vormen en concepten van de moderne wereld zal verouderen. Zo schreef de
overleden natuurkundige Heinz Pagels in zijn vooruitziende boek
\emph{The Dreams of Reason}: ``Ik ben ervan overtuigd dat de naties en
volkeren die de nieuwe wetenschap van complexiteit beheersen, de
economische, culturele en politieke supermachten van de volgende eeuw
zullen worden.''¹⁹⁵ Het is een indrukwekkende voorspelling. Wij geloven
echter dat het onjuist zal blijken, niet omdat het een verkeerde
observatie is, maar juist omdat het méér bewaarheid zal worden dan
Pagels durfde uit te spreken. Samenlevingen die zich herconfigureren tot
complexere adaptieve systemen zullen inderdaad opbloeien. Maar wanneer
dat gebeurt, zullen het waarschijnlijk geen naties zijn, laat staan
``politieke supermachten''. Het is waarschijnlijker dat de directe
begunstigden van de toegenomen complexiteit van de sociale systemen de
Soevereine Individuen van het nieuwe millennium zullen zijn.

Pagels' voorspelling komt neer op wat een sjamaan van een jagersgroep
vijfhonderd generaties terug bij het kampvuur zou hebben gezegd: ``Ik
ben ervan overtuigd dat de eerste jagersgroep die de nieuwe wetenschap
van geïrrigeerde landbouw beheerst, meer vrije tijd voor het vertellen
van verhalen zal hebben dan zelfs die kerels die de grote vissen vangen
daar bij het meer.'' Hoezeer hij ook gelijk had over het belang van
complexiteit, Pagels miste het meest fundamentele feit van allemaal:
wanneer de logica van geweld verandert, verandert de samenleving.

\section{\texorpdfstring{\textbf{De logica van
geweld}}{De logica van geweld}}\label{de-logica-van-geweld}

Om te begrijpen hoe en waarom, moeten we ons richten op diverse aspecten
van de megapolitiek die je zelden tegenkomt. Deze vraagstukken
onderzocht historicus Frederic C. Lane, wiens werk over geweld en de
economische betekenis van oorlog elders in dit boek aan bod komt. Toen
Lane in het midden van deze eeuw zijn werk schreef, was een
informatiesamenleving nog ondenkbaar. Onder die omstandigheden kon hij
er rotsvast van overtuigd zijn dat de strijd om geweld wereldwijd in
zijn definitieve fase beland was met de opkomst van de natiestaat. In
zijn werken geeft hij niet aan dat hij microprocessing had voorzien of
dat hij geloofde dat het technologisch haalbaar was om activa in de
cyberspace, een rijk zonder fysiek bestaan, te creëren. Lane ging niet
in op de implicaties van de mogelijkheid dat grote handelsvolumes
vrijwel immuun zouden kunnen worden gemaakt voor de invloed van geweld.

Hoewel Lane de huidige technologische revoluties niet had voorzien,
blijken zijn inzichten in de verschillende stadia van de monopolisatie
van geweld uit het verleden zo scherp dat ze duidelijk toepasbaar zijn
op de opkomende informatierevolutie. In zijn studie van de gewelddadige
middeleeuwse wereld richtte hij de aandacht op vraagstukken die
conventionele economen en historici vaak over het hoofd zien. Hij
realiseerde zich dat de manier waarop geweld wordt georganiseerd en
gecontroleerd een cruciale rol speelt bij de allocatie van schaarse
middelen. Lane erkende bovendien dat, hoewel de productie van geweld
doorgaans niet als onderdeel van de economische output wordt gezien, de
beheersing ervan essentieel is voor de economie. De voornaamste taak van
de overheid bestaat immers uit het bieden van bescherming tegen geweld.
Zoals hij het verwoordde:

\begin{quote}
\emph{`Elke economische onderneming heeft bescherming nodig en betaalt
daarvoor; bescherming tegen de vernietiging of de gewapende inbeslagname
van haar kapitaal en tegen de gewelddadige verstoring van haar arbeid.
In sterk georganiseerde samenlevingen behoort het leveren van deze
bescherming tot de taken van een speciale vereniging of onderneming die
men 'overheid' noemt. Inderdaad, een van de meest onderscheidende
kenmerken van overheden is hun poging om wet en orde te handhaven door
zelf macht in te zetten en op diverse wijzen het gebruik van geweld door
anderen te beheersen.'{[}9{]}}
\end{quote}

Dit punt is zo fundamenteel dat het niet in leerboeken of burgerlijke
discussies, die zogenaamd de politieke koers bepalen, wordt besproken.
Maar men kan het niet zomaar negeren als je de informatierevolutie, die
zich nu ontvouwt, wil doorgronden. De bescherming van leven en eigendom
is immers een essentiële behoefte waar elke samenleving in de
geschiedenis mee te maken heeft gehad. Het afweren van gewelddadige
agressie vormt het centrale dilemma in de geschiedenis en kent geen
eenvoudige oplossing, al bestaan er meerdere manieren om bescherming te
bieden.

\subsection{\texorpdfstring{\textbf{Het einde van een
tijdperk}}{Het einde van een tijdperk}}\label{het-einde-van-een-tijdperk}

Terwijl we dit schrijven, beginnen de megapolitieke gevolgen van het
informatietijdperk net voelbaar te worden. De economische verschuivingen
van de afgelopen decennia waren van industrie naar informatie en
rekenkracht, van machinale kracht naar microprocessing, van fabrieken
naar bureau's, van massaproductie naar kleine teams, of zelfs naar
zelfstandigen. Naarmate ondernemingen kleiner worden, neemt de kans op
sabotage en chantage op de werkvloer af. Vakbonden vinden het veel
lastiger om kleinere bedrijven te organiseren.

Dankzij microtechnologie kunnen ondernemingen kleiner en mobieler
opereren. Velen leveren diensten of producten die nauwelijks natuurlijke
hulpbronnen vereisen. In principe kunnen deze ondernemingen vrijwel
overal op aarde werken, omdat ze niet gebonden zijn aan een specifieke
locatie, zoals een mijn of een haven. Daardoor lopen ze op den duur veel
minder kans om door vakbonden of politici onder druk te worden gezet.
Een oud Chinees gezegde luidt: `Van alle zesendertig manieren om uit de
problemen te komen, is vertrekken de beste.'{[}10{]}

In het informatietijdperk zal die oosterse wijsheid haar vruchten
afwerpen. Wanneer het bedrijfsleven onaangenaam wordt door buitensporige
eisen op één locatie, wordt verhuizen een stuk eenvoudiger. Inderdaad,
zoals we straks bespreken, kunnen ondernemers in het informatietijdperk
virtuele ondernemingen opzetten, waarbij hun vestigingsplaats volledig
afhankelijk is van de spotmarkt. Een plotselinge toename van de pogingen
tot afpersing, of het nu door overheden of anderen komt, kan ertoe
leiden dat de activiteiten en activa van een virtuele onderneming met de
snelheid van het licht uit de betreffende jurisdictie verdwijnen.

De toenemende integratie van microtechnologie in industriële processen
betekent dat zelfs bedrijven die nog steeds producten met grote
schaalvoordelen produceren, niet langer even kwetsbaar zijn voor de
dreiging van geweld als vroeger. Een voorbeeld hiervan is het falen van
de langdurige staking van de United Auto Workers tegen Caterpillar, die
eind 1995, na bijna twee jaar, werd beëindigd. In tegenstelling tot de
assemblagelijnen van de jaren dertig maakt de huidige
Caterpillar-fabriek veel meer gebruik van vakbekwame werknemers. Onder
druk van buitenlandse concurrentie besteedde Caterpillar veel
laaggeschoold werk uit, sloot inefficiënte fabrieken en investeerde
bijna 2 miljard dollar in de computerisering van bestaande machines en
de installatie van assemblagerobots. De staking stimuleerde zelfs het
doorvoeren van deze arbeidsbesparende efficiëntieslag. Het bedrijf stelt
nu tweeduizend werknemers minder nodig te hebben dan toen het werk werd
neergelegd.¹⁹⁹

De megapolitieke omstandigheden van het productieproces zijn
ingrijpender veranderd dan de meeste mensen beseffen. Deze verandering
is nog niet duidelijk zichtbaar, deels omdat er altijd een vertraging
zit tussen een revolutie in megapolitieke omstandigheden en de
institutionele veranderingen die zij onvermijdelijk teweegbrengt.
Bovendien betekent de snelle evolutie van microprocessingtechnologie dat
er nu producten verschijnen waarvan de megapolitieke gevolgen al
voorzien kunnen worden, nog voordat ze bestaan. Zij zullen zorgen voor
een fundamenteel andere wereld.

\section{\texorpdfstring{\textbf{Uitbuiting van de kapitalisten door de
arbeiders}}{Uitbuiting van de kapitalisten door de arbeiders}}\label{uitbuiting-van-de-kapitalisten-door-de-arbeiders}

De aard van de technologie in het merendeel van de twintigste eeuw
zorgde ervoor dat eigenaars en managers weinig middelen hadden om een
gedwongen bezetting van een fabriek of een sit-downstaking tegen te
gaan. Zoals historicus Robert S. McElvaine opmerkte, maakte een
sit-downstaking het ``moeilijk voor werkgevers om de staking af te
breken zonder hetzelfde met hun eigen uitrusting te doen.''²⁰⁰ In feite
hielden de arbeiders het kapitaal van de eigenaars fysiek gegijzeld.

Om redenen die hieronder worden toegelicht, bleken grotere industriële
ondernemingen makkelijkere doelwitten voor vakbonden dan kleinere
bedrijven. In 1937 was General Motors misschien wel hét toonaangevende
industriële concern ter wereld. Haar fabrieken behoorden tot de grootste
en duurste verzameling machines ooit gebouwd en boden werk aan
tienduizenden arbeiders. Elk uur van elke dag dat de GM-fabrieken stil
lagen, kostte het bedrijf een klein fortuin. Een staking die wekenlang
onopgelost bleef, zoals die in de winter van 1936-37, betekende snel
oplopende verliezen.

\subsection{\texorpdfstring{\textbf{In verzet tegen vraag en
aanbod}}{In verzet tegen vraag en aanbod}}\label{in-verzet-tegen-vraag-en-aanbod}

General Motors capituleerde snel nadat de derde fabriek bezet werd en de
productie van auto's volledig stilviel. Dat was geen economische
beslissing gebaseerd op de vraag en het aanbod van arbeid. Integendeel,
toen General Motors toegaf aan de eisen van de vakbond waren er negen
miljoen werklozen in de VS, wel 14\% van de beroepsbevolking. Velen
waren bereid en bekwaam genoeg om de assemblagebanen bij GM in te
vullen, al zul je dat waarschijnlijk niet van de rapportage destijds
hebben gehoord. Een verfijnde etiquette verhulde een eerlijke analyse
van de arbeidsverhoudingen tijdens de industriële periode. Het idee dat
fabriekswerk, en dan met name in het midden van de twintigste eeuw,
vakmanschap vereiste, was grotendeels fictie. De meeste fabrieksbanen
konden door bijna iedereen uitgevoerd worden die op tijd kon opdagen. Ze
vereisten amper training en zelfs geen geletterdheid. Nog zo recent als
in de jaren '80 waren veel arbeiders van GM analfabeet of konden niet
rekenen, of beide, en tot in de jaren '90 kreeg een nieuwe werknemer
gemiddeld slechts één dag oriëntatie. Een baan die je in een dag kunt
leren, kun je geen vakmanschap noemen.

Terwijl zowel geschoolde als ongeschoolde arbeiders in de rij stonden om
smeekbedes te doen voor werk, slaagden de fabrieksarbeiders van GM er in
1937 toch in om hun werkgevers tot een loonsverhoging te dwingen. Hun
succes had veel meer te maken met de dynamiek van geweld dan met de
vraag en het aanbod van arbeid. In maart 1937, de maand na de schikking,
waren er 17 extra sit-downstakingen in de Verenigde Staten. De meeste
waren succesvol. Vergelijkbare gebeurtenissen vonden plaats in elk
geïndustrialiseerd land. De arbeiders namen simpelweg de fabrieken in
beslag en verkochten ze terug aan de eigenaars. Het was een uiterst
eenvoudige tactiek, en een die in de meeste gevallen winstgevend en leuk
was voor de deelnemers. Eén sit-downstaker schreef: ``Ik heb de tijd van
mijn leven, iets nieuws, iets anders, volop eten en muziek.''

De GM-sit-downstaking van 1936-37 en de andere gewelddadige
fabrieksovernames uit die tijd waren voorbeelden van een fenomeen dat
wij in \emph{Blood in the Streets} beschreven als ``de uitbuiting van de
kapitalisten door de arbeiders.'' Dit was niet de visie die Pete Seeger
in zijn droevige liederen verwerkte. Maar tenzij je een carrière als
volkszanger in een arbeidersbuurt overweegt, is het belangrijk niet te
focussen op de populaire interpretatie, maar op de onderliggende
werkelijkheid. Waar je ook in de geschiedenis kijkt, er is bijna altijd
een laag van rationalisatie en schijn die de ware megapolitieke
fundamenten van systematische afpersing verhult. Als je die
rationalisaties voor lief neemt, is de kans klein dat je werkelijk
begrijpt wat er gaande is.

\section{\texorpdfstring{\textbf{Het ontcijferen van de logica van
afpersing}}{Het ontcijferen van de logica van afpersing}}\label{het-ontcijferen-van-de-logica-van-afpersing}

Om de megapolitieke implicaties van de huidige overgang naar het
informatietijdperk te begrijpen, moet je de retoriek wegstrepen en je
concentreren op de werkelijke logica van geweld in de samenleving. Dit
is vergelijkbaar met het pellen van een te rijpe ui. Het kan tranen in
je ogen brengen, maar kijk niet weg. We beginnen met de logica van
afpersing op de werkplek en breiden de analyse vervolgens uit naar
bredere kwesties, zoals de creatie en bescherming van bezittingen en de
aard van de moderne overheid. In grotere mate dan de meeste mensen zich
realiseren, was de welvaart van de overheid, net als die van vakbonden,
direct gecorreleerd aan de mate waarin afpersing effectief ingezet kan
worden. In de negentiende eeuw was de effectiviteit ervan lager dan in
de twintigste. In het volgende millennium zal zij bijna tot nul afnemen.

De hele logica van overheid en de aard van macht zijn getransformeerd
door microprocessing. Dit lijkt op het eerste gezicht misschien
overdreven, maar kijk aandachtig mee. De welvaart van regeringen is in
de twintigste eeuw hand in hand gegaan met de welvaart van vakbonden.
Vóór de twintigste eeuw onttrokken de meeste overheden aanzienlijk
minder middelen dan de militante verzorgingsstaten die we nu kennen.
Evenzo waren vakbonden vóór deze eeuw kleine of onbelangrijke factoren
in het economische leven. Het vermogen van arbeiders om hun werkgevers
te dwingen tot het betalen van hogere lonen dan de marktprijs, vond haar
oorsprong in dezelfde megapolitieke omstandigheden die het voor
overheden mogelijk maakten om 40 procent van de economische productie te
belasten, of meer.

\subsection{\texorpdfstring{\textbf{Afpersing op de werkvloer vóór de
twintigste
eeuw}}{Afpersing op de werkvloer vóór de twintigste eeuw}}\label{afpersing-op-de-werkvloer-vuxf3uxf3r-de-twintigste-eeuw}

De opkomst en ondergang van de afpersing van kapitalisten door vakbonden
is makkelijk te verklaren door te kijken naar de veranderende
megapolitieke omstandigheden binnen het productieproces. In 1776, toen
Adam Smith \emph{The Wealth of Nations} publiceerde, waren de condities
voor afpersing op de werkvloer zo ongunstig dat het vormen van
``combinaties'' van arbeiders om ``de prijs van hun arbeid te verhogen''
zelden uitvoerbaar bleek. De meeste productiebedrijven waren klein en
door families gerund, en grootschalige industriële activiteiten stonden
nog in de kinderschoenen. Hoewel geweld niet uitgesloten werd, was het
niet heel effectief. In de tijd van Smith en tot ver in de negentiende
eeuw beschouwde men vakbonden als illegale combinaties in
Groot-Brittannië, de Verenigde Staten en andere landen met common-law.
Adam Smith omschreef de stakingen als volgt: `Hun gebruikelijke
voorwendselen zijn soms de hoge prijzen van levensmiddelen, soms de
enorme winst die hun meester behaalt met hun werk\ldots{} Zij grijpen
altijd terug op luid tumult, en soms zelfs op schokkend geweld en de
meest gruwelijke beledigingen.'{[}14{]} Toch behalen arbeiders `zeer
zelden enig voordeel uit die tumultueuze combinaties,' behalve `de
bestraffing of ondergang van de aanvoerders.'{[}15{]}

In de negentiende eeuw genoot de industrie van steeds meer
schaalvoordelen en de omvang van ondernemingen nam toe. Toch bleven de
meeste individuen voor zichzelf werken als boeren of kleine ondernemers,
en pogingen tot vakbondsvorming, zoals beschreven door Adam Smith,
``eindigden over het algemeen in niets.''²⁰⁴ De juridische en politieke
positie van vakbonden veranderde pas toen de schaal van ondernemingen
toenam. De eerste vakbonden die zich succesvol organiseerden, waren
ambachtsvakbonden van hooggeschoolde arbeiders, die zich over het
algemeen zonder veel geweld organiseerden. Ze stelden zich tevreden met
loonsverhogingen die overeenkwamen met de potentiële kosten van hun
vervanging. Vakbonden voor ongeschoolde arbeiders waren een ander
verhaal. Zij maakten gebruik van de verschuiving naar grotere
ondernemingen door zich te richten op precies die industrieën die
bijzonder kwetsbaar waren voor dwang, hetzij omdat ze op grotere schaal
opereerden, hetzij omdat het karakter van de werkzaamheden hun eigenaars
blootstelde aan fysieke sabotage. Dit patroon werd bevestigd van
Newcastle tot Argentinië.²⁰⁵

Een vroeg voorbeeld van gewelddadige arbeidersbewegingen in de Verenigde
Staten was een aanval op het Chesapeake and Ohio Canal in 1834. In
tegenstelling tot de meeste bedrijven van het begin van de negentiende
eeuw, was het C\&O Canal geen afgesloten en gemakkelijk te beschermen
onderneming. Oorspronkelijk zou het 550 kilometer lang worden, met een
hoogteverschil van ruim 900 meter van de lagere Potomac tot de bovenloop
van de Ohio.²⁰⁶ Het graven van zo'n kanaal was een enorme klus die nooit
helemaal zou worden voltooid. Toch waren er talrijke arbeiders mee
bezig, waarbij enkelen snel inzagen dat het kanaal eenvoudig onklaar te
maken was. Zonder regelmatig onderhoud kon het door muskusratten die
onder het jaagpad groeven, worden gesaboteerd. Daarnaast konden de
sluizen en kanalen tijdens het gebruik makkelijk beschadigd raken door
onzorgvuldig gebruik, overstromingen door zware regenval of door
botsingen met onbeheerde boten. Het was dan ook niet moeilijk voor
stakers om de waterweg te blokkeren met gezonken boten of ander puin.
Begin 1834 leidde een rel tussen rivaliserende groepen Ierse arbeiders
op het C\&O tot een poging om dit potentieel te benutten en het kanaal
over te nemen. De poging mislukte echter nadat president Andrew Jackson
federale troepen van Fort McHenry had gestuurd om de arbeiders te
verdrijven. Hierbij kwamen vijf mensen om het leven.

Mijnen en spoorwegen waren ook vroege doelwitten voor vakbondsactivisme
in Amerika. Net als het C\&O Canal waren zij zeer kwetsbaar voor
sabotage: mijnen konden bijvoorbeeld worden overstroomd of bij de ingang
worden geblokkeerd; het doden van de muildieren die de ertswagentjes uit
ondergrondse mijnen trokken, creëerde een lastige en onaangename
situatie voor de eigenaars; spoorwegbeddingen strekten zich over vele
kilometers uit en konden slechts moeilijk worden bewaakt. Het was
relatief eenvoudig voor vakbondscriminelen om mijnen en spoorwegen aan
te vallen en aanzienlijke economische schade aan te richten. Dergelijke
aanvallen kwamen veelvuldig voor tijdens pogingen om effectieve
vakbonden te organiseren. Deze inspanningen waren over het algemeen het
intensiefst tijdens perioden waarin de reële lonen stegen door deflatie.
Wanneer eigenaars probeerden om de nominale lonen aan te passen, leidde
dit vaak tot protesten die tot geweld escaleerden. Zulke incidenten
waren wijdverspreid tijdens de depressie die volgde op de paniek van
1873.

In december 1874 brak er oorlog uit in de anthracietmijnen van het
oosten van Pennsylvania. De vakbonden organiseerden een gewelddadige
staking onder de dekmantel van een geheime maatschappij genaamd de
Ancient Order of Hibernians, ook bekend als de ``Molly Maguires,'' naar
een Ierse revolutionair. Deze groep stond bekend om het terroriseren van
de kolenmijnen en het verhinderen dat mijnwerkers die wilden werken, hun
werk konden doen. Tegen haar leden werden sabotage en vernietiging van
eigendom, en zelfs moord en liquidaties ten laste gelegd.²⁰⁷

Er was ook terugkerend geweld onder spoorwegmedewerkers. Bijvoorbeeld in
juli 1877 waren er ernstige uitbarstingen gericht op het vernietigen van
eigendommen van zowel de Pennsylvania Railroad als de Baltimore \& Ohio
Railroad. Arbeiders namen wissels over, haalden sporen weg, sloten
rangeerterreinen af, schakelden locomotieven uit, saboteerden en
plunderden treinen, en erger. In Pittsburgh werden de ronde werkplaatsen
van de Pennsylvania Railroad in brand gestoken met honderden mensen
erin. Tientallen werden gedood, tweeduizend wagons werden verbrand en
geplunderd, en de machinewerkplaats werd vernietigd, samen met een
graanlift en 125 locomotieven. Federale troepen grepen in om de orde te
herstellen.²⁰⁸

Hoewel deze vroege stakingen sympathiek werden geïnterpreteerd door
socialistische en vakbondsactivisten, inspireerden ze weinig publieke
steun. Ondanks de inherente kwetsbaarheid van industrieën zoals mijnen
en spoorwegen, waren de algemene megapolitieke omstandigheden nog niet
gunstig voor de uitbuiting van kapitalisten door arbeiders. De schaal
van ondernemingen was te klein om systematische afpersing mogelijk te
maken. Hoewel er kwetsbare industrieën waren, boden zij werk aan een te
klein deel van de bevolking om de voordelen van dwang tegen werkgevers
breed te laten delen. Zonder dergelijke steun waren ze niet houdbaar,
omdat eigenaars op de overheid konden rekenen voor bescherming. Hoewel
vakbonden soms probeerden lokale ambtenaren via dreigementen ervan te
weerhouden rechterlijke bevelen uit te voeren, waren deze pogingen
zelden succesvol. Zelfs de meest gewelddadige stakingen werden meestal
binnen enkele dagen of weken door militaire middelen onderdrukt.

\subsection{\texorpdfstring{\textbf{Chantage eenvoudig
gemaakt}}{Chantage eenvoudig gemaakt}}\label{chantage-eenvoudig-gemaakt}

Het feit dat pogingen van vakbonden om lonen van kleine bedrijven boven
het marktconforme niveau te krijgen zelden succesvol waren, leert ons
een les voor het Informatietijdperk. Zelfs die bedrijfstakken die
duidelijk kwetsbaar waren voor sabotage, zoals kanalen, spoorwegen,
trams en mijnen, waren niet gemakkelijk onder controle te brengen. Dit
kwam niet omdat de vakbonden geweld schuwden. Integendeel, geweld werd
royaal ingezet, soms ook tegen prominente individuen. Bijvoorbeeld, in
een zaak die binnen de Amerikaanse arbeidersbeweging werd gevierd als
een geval van ``wraak van de mijnwerkers,'' werd gouverneur Frank
Steunenberg van Idaho, die een poging van mijnwerkers om eigendommen in
Coeur d'Alene te blokkeren had tegengewerkt, vermoord door een bom die
door een huurmoordenaar van de vakbond was gegooid.²⁰⁹ Zelfs moord en
dreigementen waren vóór de opkomst van grootschalige fabrieken en
massaproductiebedrijven in de twintigste eeuw echter meestal onvoldoende
om vakbondserkenning te verkrijgen.

Om te begrijpen waarom de omstandigheden van vakbonden in de twintigste
eeuw zo veranderden, moet je kijken naar de kenmerken van
productietechnologie. Er veranderde iets met de snelle opkomst van
fabrieksarbeid in de vroege decennia van de twintigste eeuw. Deze
verandering maakte de economische koplopers bijzonder kwetsbaar voor
afpersing. In feite leken de fysieke eigenschappen van industriële
technologie arbeiders bijna uit te nodigen om de kapitalisten af te
persen. Overweeg het volgende:

\begin{enumerate}
\def\labelenumi{\arabic{enumi}.}
\tightlist
\item
  \emph{De meeste industriële producten bevatten een grote hoeveelheid
  \textgreater{} natuurlijke hulpbronnen.} Dit leidde ertoe dat de
  productie \textgreater{} gebonden bleef aan een beperkt aantal
  locaties, net zoals dat \textgreater{} mijnen moeten geplaatst worden
  waar de ertsen zich bevinden. \textgreater{} Fabrieken die nabij
  transportknooppunten lagen met gemakkelijke \textgreater{} toegang tot
  leveranciers van onderdelen en grondstoffen, hadden \textgreater{}
  significante operationele voordelen. Dit maakte het voor dwingende
  \textgreater{} organisaties, zoals overheden en vakbonden,
  gemakkelijker om een \textgreater{} deel van die voordelen voor
  zichzelf te claimen.
\end{enumerate}

\begin{enumerate}
\def\labelenumi{\arabic{enumi}.}
\setcounter{enumi}{1}
\item
  \emph{Toenemende schaalvoordelen leidden tot zeer grote
  ondernemingen.} \textgreater{} Fabrieken aan het begin van de
  negentiende eeuw waren relatief \textgreater{} klein. Met de opkomst
  van de assemblagelijn in de twintigste eeuw \textgreater{} namen de
  schaalvoordelen toe, waardoor de grootte en de kosten van
  \textgreater{} de modernste fabrieken sterk opliepen. Dit maakte ze op
  \textgreater{} verschillende manieren een makkelijk doelwit.
  Bijvoorbeeld, grote \textgreater{} schaalvoordelen gaan vaak gepaard
  met lange productcycli. Lange \textgreater{} productcycli zorgen voor
  stabielere markten. Dit maakt \textgreater{} ondernemingen vatbaar
  voor roofzuchtige acties, omdat het betekent \textgreater{} dat er
  voordelen op de lange termijn te behalen zijn.
\item
  \emph{Het aantal concurrenten in toonaangevende industrieën daalde
  \textgreater{} scherp. Het was tijdens de industriële periode niet
  ongebruikelijk \textgreater{} om slechts een handvol bedrijven te
  vinden dat concurreerde om \textgreater{} miljardenmarkten.} Ook dit
  maakte hen makkelijkere doelwitten voor \textgreater{} afpersing door
  vakbonden. Het is veel eenvoudiger om vijf \textgreater{} bedrijven
  aan te vallen dan vijfduizend. De concentratie van de \textgreater{}
  industrie zelf was een factor die afpersing vergemakkelijkte. Dit
  \textgreater{} voordeel werkte zelfversterkend, omdat bedrijven die
  gedwongen \textgreater{} werden om hogere lonen te betalen, zelden te
  maken kregen met \textgreater{} concurrenten die daarvan waren
  vrijgesteld. Vakbonden konden \textgreater{} daardoor een behoorlijk
  deel van de winst van dergelijke bedrijven \textgreater{} afromen
  zonder dat ze direct failliet gingen. Uiteraard, als \textgreater{}
  werkgevers routinematig failliet zouden gaan telkens wanneer ze
  \textgreater{} lonen moesten betalen die hoger waren dan de
  marktprijs, dan \textgreater{} zouden arbeiders weinig winnen door hen
  daartoe te dwingen.
\item
  \emph{Door de grotere schaal van de ondernemingen, namen de
  \textgreater{} kapitaalvereisten toe.} Dit vergrootte niet alleen de
  \textgreater{} kwetsbaarheid van kapitaal en verhoogde de kosten van
  \textgreater{} fabrieksluitingen; het maakte het ook steeds
  onwaarschijnlijker \textgreater{} dat een moderne fabriek in handen
  van één individu of familie kon \textgreater{} zijn, behalve via
  erfopvolging van iemand die het bedrijf op \textgreater{} kleinere
  schaal was begonnen. Om de enorme kosten van uitrusting \textgreater{}
  en exploitatie van een grote fabriek te financieren, moest het
  \textgreater{} vermogen van honderden of duizenden mensen worden
  samengebracht op \textgreater{} de kapitaalmarkten. Dit maakte het
  moeilijker voor de \textgreater{} gefragmenteerde en bijna anonieme
  eigenaren om hun eigendom te \textgreater{} verdedigen. Ze werden
  vrijwel gedwongen om professionele \textgreater{} bestuurders in te
  schakelen, die doorgaans slechts een fractie van \textgreater{} de
  aandelen van het bedrijf in bezit hadden. Het vertrouwen op
  \textgreater{} ondergeschikte managers verzwakte de weerstand van
  bedrijven tegen \textgreater{} afpersing. De managers hadden weinig
  prikkels om hun leven of \textgreater{} lichamelijke integriteit te
  riskeren ter bescherming van het \textgreater{} eigendom van het
  bedrijf. Hun verzet bereikte zelden de felheid \textgreater{} waarmee
  kleine ondernemers, zoals slijterijhouders, hun eigendom
  \textgreater{} verdedigen.
\item
  \emph{De grotere omvang van bedrijven betekende ook dat een groter
  deel \textgreater{} van de totale beroepsbevolking in minder bedrijven
  werkzaam was \textgreater{} dan ooit tevoren.} In sommige gevallen
  vonden tienduizenden \textgreater{} arbeiders werk bij één enkel
  bedrijf. In militaire termen waren de \textgreater{} eigenaars en
  managers ruimschoots in de minderheid ten opzichte \textgreater{} van
  de personen in ondergeschikte functies. Verhoudingen van
  \textgreater{} dertig tegen één of erger waren gebruikelijk. Dit
  nadeel nam toe \textgreater{} met de bedrijfsgrootte, omdat enorme
  aantallen werknemers die \textgreater{} samenkwamen, gemakkelijker
  geweld op anonieme wijze konden \textgreater{} toepassen. Onder
  dergelijke omstandigheden hadden de werknemers \textgreater{}
  waarschijnlijk geen betekenisvol contact of relaties met de
  \textgreater{} eigenaren van de fabriek. Het anonieme karakter van
  deze relaties \textgreater{} maakte het ongetwijfeld gemakkelijker
  voor werknemers om het \textgreater{} belang van de eigendomsrechten
  van de eigenaren te negeren.
\item
  \emph{Massale tewerkstelling in een klein aantal bedrijven was een
  breed \textgreater{} maatschappelijk fenomeen.} Dit zorgde voor een
  verdere versterking \textgreater{} van de megapolitieke voordelen van
  vakbonden vergeleken met het \textgreater{} Amerika van de negentiende
  eeuw, toen de meeste mensen zelfstandig \textgreater{} waren of in
  kleine bedrijven werkten. In 1940 had 6 procent van de \textgreater{}
  Amerikaanse beroepsbevolking een baan als arbeider.²¹⁰ Als gevolg
  \textgreater{} hiervan vond het afdwingen van hogere lonen onder een
  groot aantal \textgreater{} mensen, die dachten er zelf voordeel uit
  te halen, steeds meer \textgreater{} steun. Dit werd geïllustreerd
  door een studie uit 1938-39 van de \textgreater{} opvattingen van
  1.700 mensen in Akron, Ohio, over het eigendom van \textgreater{}
  corporaties. De enquête wees uit dat 68 procent van de CIO Rubber
  \textgreater{} Workers weinig of geen sympathie had voor het idee dat
  corporaties \textgreater{} eigendomsrecht mogen hebben, terwijl
  slechts één procent sterke \textgreater{} steun voor het corporatieve
  eigendomsrecht toonde.²¹¹ Aan de \textgreater{} andere kant viel geen
  enkele ondernemer, zelfs geen kleine \textgreater{} eigenaar, in de
  categorie ``sterke tegenstand tegen corporatief \textgreater{}
  eigendom''; 94 procent kreeg beoordelingen in de categorie extreem
  \textgreater{} hoge steun voor eigendomsrechten.²¹²
\item
  \emph{Assemblagelijntechnologie was inherent sequentieel.} Het feit
  dat \textgreater{} het hele productieproces afhankelijk was van het
  verplaatsen en \textgreater{} assembleren van onderdelen in een vaste
  volgorde, creëerde extra \textgreater{} kwetsbaarheden voor
  verstoring. In feite was de assemblagelijn als \textgreater{} een
  spoorweg binnen de fabrieksmuren. Als het spoor geblokkeerd
  \textgreater{} kon worden of de beschikbaarheid van één enkel
  onderdeel kon \textgreater{} worden afgekapt, werd het hele
  productieproces stilgelegd.
\item
  \emph{Assemblagelijntechnologie standaardiseerde arbeid.} Dit
  verminderde \textgreater{} de afwijkingen in de output van personen
  met verschillende \textgreater{} vaardigheden die met dezelfde
  gereedschappen werkten. In feite was \textgreater{} een cruciaal doel
  van fabrieksontwerp het creëren van een systeem \textgreater{} waarin
  een genie en een idioot op opeenvolgende ploegen van de \textgreater{}
  assemblagelijn hetzelfde product zouden produceren. Wat men
  \textgreater{} ``domme'' machines zou kunnen noemen, waren ontworpen
  om slechts één \textgreater{} soort output te leveren. Dit maakte het
  zelfs voor de koper van \textgreater{} een Cadillac overbodig om zich
  te informeren welke lijnarbeiders \textgreater{} zijn voertuig
  produceerden. Alle producten waren bedoeld om gelijk \textgreater{} te
  zijn, ongeacht de verschillen in vaardigheden en intelligentie
  \textgreater{} tussen de werknemers die ze produceerden.
\end{enumerate}

Het feit dat ongeschoolde arbeiders op de assemblagelijn hetzelfde
product konden maken als meer bekwame individuen, droeg bij aan de
egalitaire agenda doordat het de indruk wekte dat ieders economische
bijdragen gelijk waren. Ondernemersvaardigheden en mentale inspanning
leken minder belangrijk. De magie van moderne productie leek in de
machines zelf te liggen. Zelfs als ze niet daadwerkelijk door iedereen
ontworpen konden zijn, leken ze toch intellectueel toegankelijk voor
bijna iedereen. Dit gaf meer geloofwaardigheid aan de fictie dat
ongeschoolde arbeid ``uitgebaat'' werd door fabriekseigenaar, die uit
het proces zou kunnen worden gehaald zonder dat iemand er nadeel van zou
ondervinden behalve de fabriekseigenaar zelf. ``We hebben geleerd dat we
de fabriek kunnen overnemen,'' zei een GM-staker. ``We wisten al hoe we
de fabriek moesten runnen. Als General Motors niet oppast, trekken we
vanzelf de logische conclusies.''²¹³

Deze kenmerken van industriële technologie leidden overal tot de
oprichting van vakbonden om de kwetsbaarheid voor afpersing te benutten,
en tot grotere overheden die profiteerden van de hoge belastingen die
opgelegd konden worden aan grootschalige industriële faciliteiten. Dit
gebeurde niet één of twee keer, het gebeurde overal waar grootschalige
industrieën opkwamen. Steeds weer ontstonden vakbonden die geweld
gebruikten om lonen ver boven de marktprijs te realiseren. Ze konden dit
doen omdat industriële fabrieken duur, zichtbaar, immobiel en kostbaar
waren. Ze konden nauwelijks worden verborgen of verplaatst. Elk moment
dat ze buiten gebruik waren, betekende dat hun enorme kosten niet werden
afgeschreven.

Hierdoor waren het makkelijke doelwitten voor dwangmatige afpersing, een
feit dat veel duidelijker is in de geschiedenis van vakbonden dan de
heersende ideologie van de twintigste eeuw doet geloven. De bekende
econoom Henry Simons formuleerde het in 1944 als volgt:

\begin{quote}
``Arbeidsorganisatie zonder grote macht tot dwang en intimidatie is een
onrealistische abstractie. Vakbonden hebben nu zulke machten; ze hebben
die altijd gehad en zullen die altijd hebben, zolang ze in de huidige
vorm blijven bestaan. Waar de macht klein of onzeker is, moet deze
openlijk en uitgebreid worden uitgeoefend; macht die groot en
onaangevochten is, functioneert als overheid: zelfverzekerd,
gerespecteerd en zelden opvallend tentoongesteld.''²¹⁴
\end{quote}

Hoewel Simons' analyse precies is, zat hij verkeerd op een cruciaal
punt. Hij ging ervan uit dat vakbonden ``altijd zullen hebben'' wat hij
beschreef als een ``grote macht tot dwang en intimidatie.'' In
werkelijkheid verdwijnen vakbonden, niet alleen in de Verenigde Staten
en Groot-Brittannië, maar ook in andere volwassen industriële
samenlevingen. De reden dat ze verdwijnen, wat Simons miste en wat zelfs
veel vakbondsorganisatoren niet begrijpen, is dat de verschuiving naar
een Informatiesamenleving de megapolitieke omstandigheden op cruciale
manieren heeft veranderd, waardoor de veiligheid van eigendom sterk
toeneemt. Microtechnologie heeft al bewezen de afpersing die de
verzorgingsstaat ondersteunt te ondermijnen, omdat ze zelfs in de
commerciële sfeer geheel andere prikkels creëert dan in de industriële
periode.

\begin{enumerate}
\def\labelenumi{\arabic{enumi}.}
\item
  \emph{Informatie­technologie bevat nauwelijks natuurlijke hulpbronnen.}
  Ze \textgreater{} biedt weinig of geen inherente locatievoordelen. De
  meeste \textgreater{} informatietechnologie is zeer verplaatsbaar.
  Omdat het \textgreater{} onafhankelijk van plaats kan functioneren,
  vergroot \textgreater{} informatie­technologie de mobiliteit van
  ideeën, mensen en \textgreater{} kapitaal. General Motors kon zijn
  drie assemblagelijnen in Flint, \textgreater{} Michigan, niet inpakken
  en op het vliegtuig zetten. Een \textgreater{} softwarebedrijf kan dat
  wel. De eigenaren kunnen hun algoritmen op \textgreater{} draagbare
  computers downloaden en het eerstvolgende vliegtuig \textgreater{}
  nemen. Dergelijke bedrijven hebben bovendien een extra prikkel om
  \textgreater{} te ontsnappen aan hoge belastingen of vakbondseisen
  voor \textgreater{} onredelijk hoge lonen. Kleinere bedrijven hebben
  meestal meer \textgreater{} concurrenten. Als je tientallen of zelfs
  honderden concurrenten \textgreater{} hebt die je klanten proberen te
  verleiden, kun je het je niet \textgreater{} veroorloven om politici
  of werknemers veel meer te betalen dan ze \textgreater{} daadwerkelijk
  waard zijn. Als jij dat alleen zou proberen, zouden \textgreater{} je
  kosten hoger zijn dan die van je concurrenten en zou je \textgreater{}
  failliet gaan. Het ontbreken van significante operationele
  \textgreater{} voordelen op een bepaalde locatie betekent dat
  dwingende \textgreater{} organisaties, zoals overheden en vakbonden,
  onvermijdelijk minder \textgreater{} de mogelijkheid krijgen om deze
  voordelen voor zichzelf te \textgreater{} benutten.
\item
  \emph{Informatie­technologie verkleint de omvang van ondernemingen.}
  Dit \textgreater{} leidt tot kleinere bedrijven, wat een groter aantal
  concurrenten \textgreater{} betekent. Hoe meer concurrentie, hoe
  moeilijker afpersing wordt, \textgreater{} omdat er meer doelwitten
  zijn die onder controle gehouden moeten \textgreater{} worden om lonen
  of belastingen boven het competitieve niveau te \textgreater{}
  brengen. De scherpe daling van de gemiddelde bedrijfsgrootte,
  \textgreater{} gefaciliteerd door informatietechnologie, heeft het
  aantal \textgreater{} personen in ondergeschikte posities al
  verminderd. In de Verenigde \textgreater{} Staten, bijvoorbeeld,
  suggereren breed gerapporteerde schattingen \textgreater{} dat in 1996
  ongeveer 30 miljoen mensen zelfstandig in hun eigen \textgreater{}
  bedrijf werkten. Uiteraard is het onwaarschijnlijk dat deze 30
  \textgreater{} miljoen tegen zichzelf in staking zullen gaan. Het is
  slechts iets \textgreater{} minder aannemelijk dat de miljoenen mensen
  die werken in kleine \textgreater{} bedrijven met een handvol
  werknemers hun werkgevers zouden \textgreater{} proberen te dwingen
  tot lonen die hoger dan de marktprijs \textgreater{} zijn.Werknemers
  die in het Informatietijdperk hun loon via dwang \textgreater{} willen
  verhogen, missen het militaire voordeel van hun grote \textgreater{}
  aantallen dat hen vroeger in de fabriek sterk maakte. Hoe minder
  \textgreater{} mensen in een bedrijf werken, hoe minder mogelijkheden
  er zijn \textgreater{} voor anoniem geweld. Door deze reden alleen al
  zouden tienduizend \textgreater{} werknemers verdeeld over vijfhonderd
  bedrijven een kleinere \textgreater{} bedreiging vormen voor het
  eigendom van die bedrijven dan \textgreater{} tienduizend werknemers
  in één bedrijf, zelfs als de verhouding \textgreater{}
  werknemers/eigenaren/managers exact hetzelfde was.
\item
  \emph{Een kleinere omvang van ondernemingen betekent ook dat pogingen
  om \textgreater{} lonen boven de marktprijs af te dwingen minder vaak
  brede \textgreater{} maatschappelijke steun genieten, dan in de
  industriële periode.} \textgreater{} Vakbonden die werkgevers proberen
  af te persen, zullen zich veel \textgreater{} vaker in een
  vergelijkbare situatie bevinden als de kanaal-, \textgreater{}
  spoorweg- en mijnwerkers van de negentiende eeuw. Zelfs waar
  \textgreater{} enkele grootschalige bedrijven als overblijfselen uit
  het \textgreater{} industriële tijdperk blijven bestaan, zullen ze dit
  doen in een \textgreater{} context van wijdverspreide werkgelegenheid
  in kleine bedrijven. \textgreater{} Het overwicht van kleine bedrijven
  en zelfstandigen duidt op \textgreater{} bredere sociale steun voor
  eigendomsrechten, zelfs als de wens om \textgreater{} inkomen te
  herverdelen onveranderd blijft.
\end{enumerate}

\begin{enumerate}
\def\labelenumi{\arabic{enumi}.}
\setcounter{enumi}{3}
\item
  \emph{Informatie­technologie verlaagt de kapitaalkosten, wat ook de
  \textgreater{} concurrentie bevordert door ondernemerschap te
  vergemakkelijken en \textgreater{} meer mensen in staat te stellen
  zelfstandig te werken.} Lagere \textgreater{} kapitaaleisen
  verminderen niet alleen de toetredingsdrempels, maar \textgreater{}
  ook de ``uitgangsdrempels.'' Met andere woorden: bedrijven zullen
  \textgreater{} waarschijnlijk minder activa hebben ten opzichte van
  hun inkomsten \textgreater{} en daardoor een kleiner vermogen om
  verliezen te dragen. Bedrijven \textgreater{} in het
  informatietijdperk zullen niet alleen minder vaak bij \textgreater{}
  banken hoeven lenen, maar zullen waarschijnlijk ook minder fysieke
  \textgreater{} activa hebben om te benutten.
\item
  \emph{Informatie­technologie verkort de productcyclus.} Dit leidt tot
  \textgreater{} snellere productveroudering. Ook hierdoor blijven
  winsten die \textgreater{} voortkomen uit het afpersen van hogere
  lonen boven het markt­niveau \textgreater{} niet lang bestaan. In sterk
  concurrerende markten kunnen te hoge \textgreater{} lonen direct
  leiden tot het verlies van banen en zelfs het \textgreater{}
  faillissement van het bedrijf. Pogingen om tijdelijk hogere lonen
  \textgreater{} te verkrijgen ten koste van het behoud van je baan zijn
  als het \textgreater{} verbranden van je meubels om het huis een paar
  graden warmer te \textgreater{} maken.
\item
  \emph{Informatie­technologie is niet sequentieel maar gelijktijdig en
  \textgreater{} verspreid.} In tegenstelling tot de assemblagelijn kan
  \textgreater{} informatietechnologie meerdere processen tegelijkertijd
  \textgreater{} accommoderen. Het verdeelt activiteiten over netwerken,
  waardoor \textgreater{} redundantie en substitutie mogelijk zijn
  tussen de duizenden of \textgreater{} zelfs miljoenen werkstations
  overal ter wereld. In een toenemend \textgreater{} aantal activiteiten
  kunnen mensen samenwerken zonder ooit fysiek \textgreater{} contact
  met elkaar te hebben. Naarmate virtual reality en \textgreater{}
  videoconferencing zich verder ontwikkelen, zal de trend naar het
  \textgreater{} decentraliseren van functies en telewerken versnellen.
  Dit is het \textgreater{} informatietijdperk-equivalent van
  ``huisnijverheid,'' dat de macht \textgreater{} van de middeleeuwse
  gilden brak. Het feit dat steeds minder mensen \textgreater{} samen in
  rokerige fabrieken werken, neemt niet alleen een \textgreater{}
  belangrijk voordeel weg dat werknemers vroeger hadden bij het
  \textgreater{} afdwingen van geld van kapitalisten; het maakt het ook
  steeds \textgreater{} moeilijker om het soort afpersing dat acceptabel
  was op de \textgreater{} werkvloer nog te onderscheiden van
  georganiseerde misdaad. Tot nu \textgreater{} toe mochten alleen
  personen die samenwerkten en door een bedrijf \textgreater{} in een
  gemeenschappelijke setting werden tewerkgesteld, geweld \textgreater{}
  gebruiken om hun inkomen te verhogen. Als de ``werkplek'' echter
  \textgreater{} niet bestaat als centrale locatie, en de meeste
  functies zijn \textgreater{} verspreid over onderaannemers en
  telewerkers, lijken hun pogingen \textgreater{} om geld te eisen van
  hun klanten of ``werkgevers'' toch sterk op een \textgreater{}
  afpersingspraktijk. Bijvoorbeeld, is een telewerker die onder
  \textgreater{} dreiging van een virus extra geld eist een stakende
  werknemer? Of \textgreater{} een internetcrimineel?\\
  \textgreater{}\\
  \textgreater{} Of hij het ene of het andere is, zal uiteindelijk
  weinig verschil \textgreater{} maken. De reactie van de getroffen
  bedrijven zal in elk geval \textgreater{} grotendeels hetzelfde zijn.
  Technische oplossingen tegen \textgreater{} informaticasabotage die
  hackers buiten houden, zoals sterke \textgreater{} encryptie en
  beveiligde netwerken, zullen eveneens verhinderen dat \textgreater{}
  ontevreden werknemers of onderaannemers schade aanrichten aan hun
  \textgreater{} klanten of partners. Natuurlijk kan worden gesteld dat
  de \textgreater{} werknemer of telewerker altijd naar kantoor kan
  komen om een meer \textgreater{} traditionele staking uit te voeren,
  maar zelfs dit is misschien \textgreater{} niet zo eenvoudig als het
  lijkt in het informatietijdperk. Het \textgreater{} vermogen van
  informatietechnologie om lokale beperkingen te \textgreater{}
  overstijgen en economische functies te verspreiden betekent dat
  \textgreater{} werknemers en werkgevers voor het eerst niet eens in
  dezelfde \textgreater{} rechtsgebieden hoeven te wonen. Hier gaat het
  niet om het verschil \textgreater{} tussen wijken als Mayfair en
  Peckham, maar om werkgevers in \textgreater{} Bermuda en telewerkers
  in New Delhi.\\
  \textgreater{}\\
  \textgreater{} Mocht de fascinatie van de Indiërs voor de grote
  GM-stakingen van \textgreater{} 1936-37 hen naar Bermuda leiden om te
  demonstreren, dan zouden ze \textgreater{} bij aankomst wellicht
  totaal geen fysiek kantoor aantreffen. \textgreater{} Chiat/Day, een
  groot reclamebedrijf, heeft haar hoofdkantoor al \textgreater{}
  grotendeels ontmanteld. Werknemers of onderaannemers blijven in
  \textgreater{} contact via call-forwarding en het internet. Wanneer
  het nodig is \textgreater{} om teams samen te stellen voor projecten,
  huren ze vergaderruimtes \textgreater{} in hotels. Na afloop van het
  project checken ze uit. Het feit dat \textgreater{} microprocessing
  helpt met het bevrijden van het productieproces en \textgreater{} het
  verspreiden van de vaste volgorde van de assemblagelijn,
  \textgreater{} vermindert de macht die dwingende organisaties zoals
  vakbonden en \textgreater{} overheden vroeger hadden aanzienlijk. Als
  de assemblagelijn een \textgreater{} spoorweg binnen fabrieksmuren
  voorstelde die gemakkelijk door een \textgreater{} zitstaking kon
  worden veroverd, is cyberspace een grenzeloos \textgreater{} domein
  zonder fysieke aanwezigheid. Het kan niet met geweld worden
  \textgreater{} bezet of worden gegijzeld. De positie van werknemers
  die geweld \textgreater{} willen gebruiken als pressiemiddel om een
  hoger inkomen af te \textgreater{} dwingen, zal in het
  informatietijdperk veel zwakker zijn dan ten \textgreater{} tijde van
  de zitstakingen bij General Motors in 1936-37.
\end{enumerate}

\begin{enumerate}
\def\labelenumi{\arabic{enumi}.}
\setcounter{enumi}{6}
\tightlist
\item
  \emph{Microprocessing individualiseert werk. Industriële technologie
  \textgreater{} standaardiseerde werk.} iedereen die dezelfde
  gereedschappen \textgreater{} gebruikte, produceerde hetzelfde
  resultaat. Microtechnologie \textgreater{} vervangt ``domme'' machines
  door intelligentere machines die een \textgreater{} sterk variabele
  output kunnen leveren. Deze toegenomen \textgreater{} variabiliteit
  van output heeft ingrijpende gevolgen. Belangrijk is \textgreater{}
  dat waar de output varieert, ook de inkomens variëren. In markten
  \textgreater{} waar talent en vaardigheden uiteenlopen, zal een klein
  aantal \textgreater{} mensen meestal het leeuwendeel van de waarde
  produceren, wat \textgreater{} typisch is voor de meest concurrerende
  sectoren.\\
  \textgreater{} Dit is vrij duidelijk te zien in bijvoorbeeld sport.
  Miljoenen \textgreater{} jongeren spelen wereldwijd voetbal, maar 99\%
  van het geld dat \textgreater{} wordt uitgegeven aan het kijken van
  voetbal gaat naar een zeer \textgreater{} klein aantal spelers. Ook
  bereiken slechts enkele acteurs een \textgreater{} sterrenstatus en
  ontvangen een klein aantal auteurs van de \textgreater{} tienduizenden
  boeken die jaarlijks uitgegeven worden het grootste \textgreater{}
  deel van de royalty's. Jammer genoeg maken wij deel uit van die
  \textgreater{} groep.\\
  \textgreater{}\\
  \textgreater{} De enorme variabiliteit in output tussen personen die
  hetzelfde \textgreater{} materiaal gebruiken, vormt opnieuw een
  obstakel voor afpersing. \textgreater{} Het veroorzaakt een groot
  onderhandelingsprobleem over hoe de \textgreater{} opbrengst verdeeld
  dient te worden. Wanneer een relatief klein \textgreater{} deel van de
  deelnemers aan een bepaalde activiteit het grootste \textgreater{}
  deel van de waarde creëert, is het vrijwel wiskundig onmogelijk
  \textgreater{} dat zij erop vooruitgaan als de inkomens onder dwang
  worden \textgreater{} gemiddeld. De ene softwareprogrammeur kan een
  algoritme voor het \textgreater{} aansturen van een robot ontwikkelen
  dat miljoenen waard blijkt te \textgreater{} zijn. Een ander, die met
  exact dezelfde spullen werkt, kan een \textgreater{} programma
  schrijven dat niets waard is. De productievere \textgreater{}
  programmeur zal net zo min willen dat zijn inkomen gekoppeld wordt
  \textgreater{} aan dat van zijn collega als dat Tom Clancy zou
  instemmen met het \textgreater{} middelen van zijn boekroyalties met
  de onze.\\
  \textgreater{}\\
  \textgreater{} Al in de beginfase van de Informatiesamenleving werd
  duidelijker \textgreater{} dan in 1975 dat economische output sterk
  afhankelijk is van \textgreater{} vaardigheden en intellectuele
  bekwaamheid. Dit heeft de ooit \textgreater{} trotse rechtvaardiging
  voor afpersing van kapitalisten door \textgreater{} arbeiders uit de
  industriële periode al tenietgedaan. De gedachte \textgreater{} dat
  ongeschoolde arbeid werkelijk de waarde creëerde die kennelijk
  \textgreater{} in onevenredig grote mate door kapitalisten en
  ondernemers werd \textgreater{} opgeëist, is al een anachronisme. Het
  is niet eens een plausibele \textgreater{} fictie in het geval van
  informatietechnologie. Bij een programmeur \textgreater{} is er een te
  direct verband tussen zijn vaardigheid en zijn \textgreater{} product
  om veel twijfel te laten over wie verantwoordelijk is. Het
  \textgreater{} is onbetwistbaar duidelijk dat een analfabeet of
  semi-geletterde \textgreater{} geen computer kan programmeren. Het is
  daarom even duidelijk dat \textgreater{} waarde in programma's die
  door anderen zijn samengesteld, niet van \textgreater{} hem gestolen
  kan zijn. Daarom wordt het geroep over ``uitbuiting'' \textgreater{}
  van arbeiders nu vooral aanwezig onder schoonmakers.\\
  \textgreater{}\\
  \textgreater{} Informatietechnologie maakt duidelijk dat het probleem
  van mensen \textgreater{} met lage vaardigheden niet is dat hun
  productieve capaciteiten \textgreater{} oneerlijk worden uitgebuit,
  maar eerder de angst dat zij mogelijk \textgreater{} niet in staat
  zijn een echte economische bijdrage te leveren. \textgreater{} Zoals
  Kevin Kelly suggereert in \emph{Out Of Control}, kan het
  \textgreater{} ``Upstart''-autobedrijf van het Informatietijdperk het
  geesteskind \textgreater{} zijn van ``een dozijn mensen'', die het
  merendeel van hun onderdelen \textgreater{} uitbesteden en toch auto's
  produceren die zorgvuldiger zijn \textgreater{} aangepast aan de
  wensen van hun kopers dan alles wat tot nu toe \textgreater{} uit
  Detroit of Tokio is gekomen: ``Auto's, elk op maat gemaakt voor
  \textgreater{} de klant, worden besteld via een netwerk van klanten en
  verzonden \textgreater{} zodra ze klaar zijn. Mallen voor de
  carrosserie worden snel \textgreater{} gevormd door computergestuurde
  lasers en gevoed met ontwerpen die \textgreater{} zijn gegenereerd op
  basis van communicatie met de klant en \textgreater{} doelgerichte
  marketing. Een flexibele productielijn van robots \textgreater{}
  assembleert de auto's. De reparatie en verbetering van de robots
  \textgreater{} wordt uitbesteed aan een robotbedrijf.''215
\end{enumerate}

\subsection{\texorpdfstring{\textbf{``Gereedschap met een
stem''}}{``Gereedschap met een stem''}}\label{gereedschap-met-een-stem}

In toenemende mate kan ongeschoold werk worden gedaan door
geautomatiseerde machines, robots en computersystemen zoals digitale
assistenten. Toen Aristoteles slaven beschreef als ``werktuigen met een
stem'', doelde hij op mensen. In de nabije toekomst zal ``Gereedschap
met een stem'', zoals de geesten uit sprookjes, kunnen spreken,
instructies opvolgen en zelfs complexe opdrachten uitvoeren. De snel
toenemende rekenkracht heeft al geleid tot een aantal primitieve
toepassingen van spraakherkenning, zoals handsfree-telefoons en
computers die wiskundige berekeningen uitvoeren op basis van mondelinge
instructies. Computers die spraak omzetten naar tekst werden eind 1996,
toen wij dit schreven, al op de markt gebracht. Naarmate
patroonherkenning beter wordt, zullen computers gekoppeld aan
stem­synthesizers via netwerken talloze functies uitvoeren die vroeger
werden gedaan door mensen die werkten als telefonisten, secretaresses,
reisagenten, administratieve assistenten, schaakkampioenen,
schade-experts, componisten, obligatiehandelaren,
cyberoorlogsspecialisten, wapenanalisten, of zelfs gewiekste
flirtmachines die de telefoon aannemen bij 0900-lijnen.

Michael Mauldin van Carnegie-Mellon University heeft een kunstmatige
persoonlijkheid geprogrammeerd, genaamd Julia, die bijna iedereen kan
misleiden met wie zij online een gesprek voert. Volgens persberichten is
Julia een gevatte dame die haar leven leidt in een rollenspel op
internet. Ze is slim, grappig en houdt van flirten. Ze is ook een beetje
een hockeyfanaat en kan op elk moment de perfecte sarcastische opmerking
maken. Julia is echter geen dame. Ze is een bot, een kunstmatige
intelligentie die alleen bestaat in de ether van het internet.216 De
opmerkelijke vooruitgang die al is geboekt in het programmeren van
kunstmatige intelligentie en digitale dienaren laat weinig twijfel
bestaan dat er nog vele praktische toepassingen zullen volgen. Dit heeft
ingrijpende megapolitieke gevolgen.

\subsection{\texorpdfstring{\textbf{Het individu als
ensemble}}{Het individu als ensemble}}\label{het-individu-als-ensemble}

De ontwikkeling van ``gereedschap met een stem'' voor meerdere
toepassingen schept de mogelijkheid om het individu te verdelen over
meerdere gelijktijdige activiteiten. Het individu zal wellicht niet
langer enkelvoudig zijn, maar een ensemble van tientallen of zelfs
duizenden activiteiten die worden uitgevoerd via intelligente
assistenten. Dit zal niet alleen de productieve capaciteit van de meest
begaafde individuen enorm vergroten, maar heeft ook de potentie om het
Soevereine Individu veel krachtiger te maken in militair opzicht dan
ooit tevoren.

Eén individu zal zijn activiteiten zichtbaar kunnen vermenigvuldigen
door een vrijwel onbeperkt aantal intelligente assistenten in te zetten.
Hij of zij zal zelfs na de dood kunnen handelen. Voor het eerst zal een
individu in staat zijn uitgebreide taken uit te voeren, zelfs als hij
biologisch dood is. Het doden van een individu zal voor een crimineel of
een vijand in oorlogstijd niet langer de mogelijkheid tot wraak volledig
wegnemen. Dit is een van de meest revolutionaire innovaties in de logica
van geweld in de hele geschiedenis.

\subsection{\texorpdfstring{\textbf{Inzichten voor het
Informatietijdperk}}{Inzichten voor het Informatietijdperk}}\label{inzichten-voor-het-informatietijdperk-1}

De grootste veranderingen in het leven doen zich voor bij variabelen die
niemand in de gaten houdt. Of anders gezegd: we nemen variabelen als
vanzelfsprekend aan die eeuwenlang of zelfs honderden generaties
nauwelijks zijn veranderd. Gedurende het grootste deel van de
geschiedenis, zo niet van het menselijk bestaan, bewoog het evenwicht
tussen bescherming en afpersing zich binnen een smalle marge, waarbij
afpersing altijd de overhand had. Dat staat nu op het punt te
veranderen. Informatietechnologie legt de basis voor een fundamentele
verschuiving in de factoren die de kosten en baten van het toepassen van
geweld bepalen. Het feit dat intelligente assistenten beschikbaar zullen
zijn om onderzoek te doen en mogelijk op de een of andere manier wraak
te nemen op degenen die geweld initiëren, is slechts een voorproefje van
dit nieuwe perspectief op bescherming. Vijfentwintig jaar geleden zou de
volgende uitspraak niet meer zijn geweest dan het geraas van een
zonderling: ``Als je mij doodt, zal ik het geld van je bankrekeningen
wegsluizen en het schenken aan liefdadigheid in Nepal.'' Na de
eeuwwisseling hoeft dat niet langer ondenkbaar te zijn. Of het een
praktische dreiging vormt, zal afhangen van tijd en plaats. Maar zelfs
als de rekeningen van de misdadiger ontoegankelijk blijken, zal er
ongetwijfeld ander kostbaar onheil zijn dat een leger van intelligente
assistenten kan aanrichten als vergelding voor een misdrijf. Denk daar
maar eens over na.

\subsection{\texorpdfstring{\textbf{Nieuwe alternatieven voor
bescherming}}{Nieuwe alternatieven voor bescherming}}\label{nieuwe-alternatieven-voor-bescherming}

Dit is slechts één van de vele manieren waarop de technologie van het
Informatietijdperk nieuwe mogelijkheden opent voor bescherming. De
meeste daarvan ondermijnen de bijna-monopoliepositie van overheden op
bescherming en afpersing die zij de afgelopen twee eeuwen hebben
genoten. Zelfs zonder de nieuwe technologische toeters en bellen hebben
er altijd alternatieven bestaan voor bescherming die niet door de
overheid werden gemonopoliseerd.

Iemand die zich bedreigd voelt, kan simpelweg wegrennen. Toen de wereld
nog jong was en de horizon openlag, werd deze optie vaak benut. Wie zich
zorgen maakt over diefstal of vandalisme kan besluiten een
verzekeringspolis af te sluiten om dergelijke risico's af te dekken.
Vervloekingen en bezweringen, hoewel zwakke vormen van bescherming,
hebben ook levens gered en diefstal voorkomen in samenlevingen met
bijgelovige criminelen. Waardevolle spullen kunnen daarnaast, waar
mogelijk, met relatief succes beschermd worden door ze te verbergen.
Bezittingen kunnen worden begraven, beveiligd met sloten, achter hoge
muren geplaatst, of voorzien van sirenes en elektronische bewaking. Het
verbergen van personen en eigendommen is echter niet altijd praktisch
gebleken.

Ondanks de verscheidenheid aan beschermingsmiddelen die in de
geschiedenis zijn gebruikt, heeft één methode alle andere gedomineerd:
het vermogen om geweld met geweld te beantwoorden, door een grotere
macht in te zetten om elke potentiële aanvaller of dief te
overmeesteren. De vraag is tot wie je je kunt wenden voor zo'n dienst,
en hoe je iemand kunt motiveren zijn hele hebben en houden te riskeren
om je te helpen tegen boeven die geweld tegen je willen gebruiken. Soms
hebben naaste familieleden gehoor gegeven aan die roep. Soms hebben
tribale ­groepen of clans als onofficiële politie gefungeerd, door geweld
tegen een van hun leden te beantwoorden met bloedwraak. Soms werden
huurlingen of privéwachten ingehuurd om aanvallen af te slaan, maar niet
altijd met het gewenste resultaat.

De nieuwe intelligente assistenten van het Informatietijdperk voegen een
nieuw alternatief toe, hoewel hun activiteiten grotendeels beperkt
zullen blijven tot de cyberspace. Bij hen is er, in tegenstelling tot
die van huurlingen, privéwachten of zelfs verre verwanten, geen twijfel
over hun loyaliteit.

\subsection{\texorpdfstring{\textbf{De paradoxen van de
macht}}{De paradoxen van de macht}}\label{de-paradoxen-van-de-macht}

Het gebruik van geweld ter bescherming tegen geweld zit vol paradoxen.
Onder de omstandigheden die tot nu toe hebben bestaan, gold dat elke
groep of instantie die je kon inzetten om je leven en bezit met succes
te beschermen tegen een aanval, noodzakelijkerwijs ook de capaciteit had
om die af te nemen. Dat is een nadeel waarvoor geen eenvoudige oplossing
bestaat. Normaal gesproken kun je rekenen op concurrentie om aanbieders
van een economische dienst ervan te weerhouden de wensen van hun klanten
te negeren. Maar waar het geweld betreft, heeft directe concurrentie
vaak averechtse gevolgen. In het verleden leidde dit meestal tot meer
geweld. Wanneer twee potentiële beschermingsinstanties hun troepen
uitzonden om elkaar te arresteren, resulteerde dat meer in burgeroorlog
dan in bescherming. Wanneer je bescherming zoekt tegen geweld, wil je
normaal gesproken niet dat de hoeveelheid geweld toeneemt, maar dat deze
wordt onderdrukt. En wel op voorwaarden die niet toestaan dat de klanten
die voor de beschermingsdienst betalen, zelf geplunderd worden.

\begin{quote}
\emph{``\ldots wanneer mensen leven zonder een gemeenschappelijke macht
die hen allen ontzag inboezemt, bevinden zij zich in de toestand die
oorlog wordt genoemd: en wel een oorlog van ieder mens tegen ieder mens,
waarin mensen geen andere zekerheid hebben dan wat hun eigen kracht en
vindingrijkheid hen kan verschaffen.'' -- THOMAS HOBBES}
\end{quote}

\subsection{Monopolie en anarchie}\label{monopolie-en-anarchie}

Dit is de reden waarom anarchie, of ``de oorlog van allen tegen allen,''
zoals Hobbes het beschreef, zelden een bevredigende toestand is geweest.
Lokale concurrentie in het gebruik van geweld betekende meestal hogere
kosten voor minder feitelijke bescherming. Af en toe hebben enthousiaste
vrijemarktdenkers gesuggereerd dat marktmechanismen alleen voldoende
zouden zijn om eigendomsrechten en levens te beschermen, zonder dat er
enige soevereiniteit nodig zou zijn.217 Sommige van de analyses waren
elegant, maar feit blijft dat de voorziening van politiediensten en
justitie door de vrije markt onder de megapolitieke voorwaarden van het
industrialisme niet levensvatbaar is gebleken. Alleen primitieve
samenlevingen met kleine en homogene populaties waarin gedrag sterk
gestandaardiseerd was, konden overleven zonder overheden die via geweld
de gemonopoliseerde lokale beschermingsdienst leverden.

Voorbeelden van anarchistische samenlevingen boven het niveau van de
jager-verzamelaarsstam zijn schaars en oud. Het waren allemaal erg
eenvoudige economieën zoals regenwaterboeren in afgelegen gebieden, de
Kafirs in pre-islamitisch Afghanistan, sommige Ierse stammen in de
donkere middeleeuwen, enkele indianenstammen in Brazilië, Venezuela en
Paraguay, en andere inheemse groepen verspreid over de wereld. Hun
methoden om bescherming zonder overheid te organiseren, zijn alleen
bekend bij kenners van extreme gevallen. Wie daar meer over wil weten,
kan in onze voetnoten meerdere boeken vinden met verdere details.218
Primitieve groepen konden functioneren zonder een aparte organisatie die
gespecialiseerd was in geweld, enkel omdat zij kleine, gesloten en
geïsoleerde samenlevingen waren. Ze konden om zich te verdedigen tegen
de meeste gewelddreigingen op kleine schaal, de enige soort waarmee zij
waarschijnlijk te maken kregen, terugvallen op hechte familiebanden.
Wanneer zij grotere bedreigingen tegenkwamen, georganiseerd door staten,
werden zij overmeesterd en onderworpen aan een door externe groepen
gemonopoliseerde heerschappij. Dit gebeurde steeds opnieuw. Overal waar
samenlevingen ontstonden boven het niveau van clans en stammen, vooral
waar handelsroutes verschillende volkeren met elkaar in contact
brachten, kwamen er altijd specialisten in geweld op om de
surplusproductie van vreedzamere mensen te plunderen. Wanneer
technologische omstandigheden het rendement van geweld verhoogden, waren
samenlevingen die niet georganiseerd waren om grote middelen in
oorlogvoering te steken gedoemd te verdwijnen.

\begin{quote}
\emph{``Welke vorsten leverden de dienst van politie? Welke waren
afpersers of zelfs plunderaars? Een plunderaar kon in feite de
politiechef worden zodra hij zijn `opbrengst' met regelmaat inde, deze
afstemde op het betalingsvermogen, zijn gebied verdedigde tegen andere
plunderaars en zijn territoriale monopolie lang genoeg handhaafde totdat
gewoonte het legitiem maakte.''219 -- FREDERIC C. LANE}
\end{quote}

\subsection{Overheid als verkoper van
bescherming}\label{overheid-als-verkoper-van-bescherming}

Zoals we al meerdere keren hebben gesteld, is de voornaamste economische
functie van de overheid, vanuit het perspectief van degenen die
belastingen betalen, het bieden van bescherming van leven en eigendom.
Toch functioneert de overheid vaak als georganiseerde misdaad, door
middelen af te romen van mensen binnen haar invloedssfeer in de vorm van
een heffing of gewoon ordinaire diefstal. De overheid is niet alleen een
beschermingsdienst; het is een beschermingsmaffia. Terwijl de overheid
bescherming biedt tegen geweld van buitenaf, vraagt zij, net als de
maffia, ook geld voor bescherming tegen schade die zij anders zelf zou
toebrengen. De eerste handeling is een economische dienst. De tweede is
afpersing. In de praktijk kan het lastig zijn om deze twee vormen van
``bescherming'' te onderscheiden. Overheden, merkte Charles Tilly op,
kunnen misschien het best worden begrepen als ``onze grootste
voorbeelden van georganiseerde misdaad.''220

Zelfs de beste overheid leverde meestal een mengvorm van bescherming
gecombineerd met afpersing. Historisch gezien konden beiden het meest
effectief worden uitgevoerd wanneer de overheid nagenoeg een monopolie
op dwang kon opleggen binnen de gebieden waar zij actief was. In
gevallen waar één gewapende groep de overhand kon vestigen in het
gebruik van geweld, was de kwaliteit van de bescherming die zij kon
bieden doorgaans veel beter dan die van meerdere concurrerende
leveranciers van beschermingsdiensten die om hetzelfde territorium
vochten.

\subsection{Een natuurlijk monopolie op
land}\label{een-natuurlijk-monopolie-op-land}

Het bereiken van een lokaal monopolie op dwang stelde een overheid niet
alleen in staat haar potentiële klanten effectiever te beschermen tegen
geweld van buitenaf; het verlaagde ook sterk de operationele kosten.
Zoals Lane het formuleerde: ``De industrie die geweld toepast en geweld
beheerst was een natuurlijk monopolie, althans op land. Binnen
territoriale grenzen kon de dienst die zij leverde veel goedkoper worden
geproduceerd door een monopolie.''221 Zo stelde ``een monopolie op het
gebruik van geweld binnen een aaneengesloten gebied een bescherming
producerende onderneming in staat haar product te verbeteren en de
kosten te verlagen.''222 Een dergelijke organisatie kon meer bescherming
bieden tegen lagere kosten, zolang zij niet voortdurend militaire acties
hoefde te voeren om concurrerende groepen af te weren die eveneens geld
wilden innen bij haar klanten.

Het vooruitzicht dat informatietechnologie zal bijdragen aan het
``versoepelen'' van de aanname dat soevereiniteit gebaseerd moet zijn op
een territoriaal monopolie, heeft al de aandacht getrokken van politieke
theoretici. Het is het centrale thema van \emph{Beyond Sovereignty:
Territory and Political Economy in the Twenty-First Century}, van David
J. Elkins. Elkins sluit zich aan bij onze these dat monopolistische
overheden dezelfde desintermediatie tegemoet gaan als religieuze
monopolies na 1500. Hij schrijft: ``We namen vroeger aan dat religies
hun eigen territorium of `domein' moesten hebben. Toen naties universele
religies vervingen als soevereine scheidsrechters over leven en dood,
maakten de `compactheid' en de `afgebakendheid' van religie plaats voor
het ons inmiddels vertrouwde vermengen van verscheidene gelovigen in
hetzelfde gebied. Daarentegen weigeren we het vermengen van naties, of
zelfs provincies, te accepteren, hoewel ik geloof dat deze aanname bezig
is te verdwijnen.''223

Hij betoogt verder, geheel in lijn met onze visie, dat territoriale
monopolies op soevereiniteit kunnen worden afgebroken zonder dat dit tot
anarchie hoeft te leiden. Bewijzen daarvoor zijn de verdeling van
soevereiniteit tussen nationale en provinciale regeringen in een
federaal systeem zoals dat van Canada, en het condominium-bestuur met
gezamenlijke Franse en Britse soevereiniteit dat een aantal eilanden in
de Stille Oceaan gedurende een groot deel van deze eeuw kenmerkte.
Territoriale monopolies op soevereiniteit zijn dus zelden met geweld
ontmanteld, maar kunnen wel door middel van overeenstemming worden
ontbundeld. Volgens Elkins, en wij zijn het daarmee eens, ``is de
territoriale natiestaat een bundel of mand waarin andere aspecten van
ons leven passen. Het is vergelijkbaar met het economische concept van
een `mand' goederen; je kunt de goederen niet afzonderlijk verkrijgen,
maar neemt het geheel. In een restaurant kan men à la carte bestellen,
maar wat betreft onze identiteiten moeten we nemen wat naties
samenbundelen, wat neerkomt op `table d'hôte.' \ldots{} Een overheid
\emph{à la carte} zal voor burgers in de eenentwintigste eeuw
vanzelfsprekend lijken.''224 Geen ontwikkeling zal meer bijdragen aan de
ontbundeling van soevereiniteit en de opkomst van de overheid à la carte
dan de opkomst van een cyber­economie die fysieke grenzen volledig
overstijgt.

\begin{quote}
\emph{``Wanneer frequenties stijgen en golflengtes dalen, verbetert
digitale prestatie exponentieel. Bandbreedte stijgt, energieverbruik
daalt, antenneformaten krimpen, interferentie stort in, foutenmarges
kelderen.'' -- GEORGE GILDER}
\end{quote}

\section{\texorpdfstring{\textbf{De wet van de telekosmos heft de wetten
van naties
op}}{De wet van de telekosmos heft de wetten van naties op}}\label{de-wet-van-de-telekosmos-heft-de-wetten-van-naties-op}

Wij zijn niet de enigen die inzien dat bandbreedte (of het draagvermogen
van communicatiemedia) het territoriale staatsmodel zal overschaduwen.
Jim Taylor en Watts Wacker, auteurs van \emph{The 500-Year Delta: What
Happens After What Comes Next}, formuleren hun betoog anders dan wij,
maar erkennen dat ``toegang globalisme creëert, en globalisme politieke
systemen ontwricht, waardoor het concept van grenzen achterhaald raakt.
Naarmate grenzen verdwijnen, wordt het concept van belastingheffing, dat
overheden ondersteunt, steeds fragieler\ldots{} Naarmate grenzen
verdwijnen, valt ook het concept van aanspraak uiteen, het geloof dat
je, omdat je op een bepaalde plek bent geboren, recht hebt op de
economische voordelen van die plek, en daarmee verdwijnen de privileges
van het staatsburgerschap. En terwijl dat gebeurt, worden de idealen die
nationaliteit onderbouwen, namelijk patriottisme, democratie, de staat,
de smeltkroes, eenheid, verantwoord burgerschap, of welke idealen in
welk land dan ook, verwezen naar de geschiedkundige vuilnisbelt.''225
Zonder het expliciet te zeggen, voelen ook zij blijkbaar dat de
geschiedenis richting de bevrijding van het Soevereine Individu beweegt.
Zoals ze stellen: ``Aan de horizon wacht een veel zuiverdere vorm van
individualisme dan democratie, zoals wij die nu begrijpen, toelaat.''226

Hoe zal dit gebeuren? Taylor en Wacker zien een krachtige dynamiek aan
het werk. Het simpele feit is dat het bredere gevoel van patriottisme,
de liefde voor de natie en een gevoel van kinderlijke plicht eraan, niet
langer een bijzonder nuttige instelling is. Burgers die gedijen in de
mondiale samenleving zullen hun keuzes niet baseren op nationale
identiteit, maar op politieke, sociale en economische identiteiten die
direct verbonden zijn met hun eigen belangen. Zij zullen zich daarnaar
organiseren. Ze zullen hun vrijheid om te weten, te reizen, te handelen
en te zijn, maximaliseren. Naties en bedrijven die dat niet doen, die
achterhoedegevechten blijven voeren uit nostalgie, zullen wegkwijnen.227

De ontwaarding van fysieke grenzen, die wordt geïmpliceerd door de
jaarlijkse verdrievoudiging van de bandbreedte en de exponentiële groei
van het internet en het World Wide Web, zal het proces van
desintermediatie van overheden versnellen. Een voortzetting van de
jaarlijkse verdrievoudiging van de bandbreedte tot 2012 zou een
miljardvoudige groei betekenen sinds 1993, toen George Gilder voor het
eerst suggereerde dat bandbreedte nog sneller zou verdubbelen dan de
capaciteit van microprocessors. Indien dit werkelijkheid wordt, en daar
rekenen wij op gezien recente doorbraken in geïntegreerde optica, zal de
overvloed aan communicatiemogelijkheden leiden tot een enorme toename
van cybercommerce. Met wave division multiplexing kan een enkele
glasvezel, zo dun als een mensenhaar, één biljoen bits per seconde
vervoeren.228 Met andere woorden: een enkele glasvezelkabel zou
vijfentwintig keer meer bits kunnen verwerken dan de totale capaciteit
van alle communicatienetwerken ter wereld samen. De
uitbreidingsmogelijkheden zijn verbijsterend. Met zoveel
communicatiemogelijkheden, en de extreem lage kosten ervan, zal er enorm
veel meer geld naar communicatie vloeien. Traditionele media zoals vaste
telefonie en televisie zullen anachronismen worden. Het World Wide Web
zal een rijkere mix van signalen naar elke computer brengen dan
consumenten vandaag met televisienetwerken ervaren. Naarmate de
bandbreedterevolutie zich ontvouwt, zullen mensen steeds verder worden
meegezogen in de grenzeloze virtuele wereld van online gemeenschappen en
cybercommerce, een wereld met voldoende grafische dichtheid om de
``metaverse'' te worden, de alternatieve cyberspace-realiteit zoals
beschreven door sciencefictionauteur Neal Stephenson. Stephensons
``metaverse'' is een virtuele gemeenschap met eigen wetten, machthebbers
en schurken.229

Naarmate steeds meer economische activiteit wordt verplaatst naar de
cyberspace, zal de waarde van het monopolie van staten binnen fysieke
grenzen afnemen, waardoor ze steeds meer geprikkeld zullen worden om hun
soevereiniteit uit te besteden en te fragmenteren. Zoals natiestaten nu
prikkels hebben om vrijhavens, vrijhandelszones en zona francas te
hosten, zo zullen zij prikkels hebben om hun soevereiniteit te verhuren.
Wij hebben al gesproken over de vergevorderde onderhandelingen tussen de
negenhonderd jaar oude Soevereine Militaire Hospitaalorde van Sint-Jan
van Jeruzalem, van Rhodos en van Malta, beter bekend als de Orde van
Malta, en de Republiek Malta om de soevereiniteit over Fort Sint-Angelo
terug te geven aan de Orde. Wij verwachten dat deze onderhandelingen
succesvol zullen worden afgerond. Anderen zullen volgen. Sommige staten
zullen soevereiniteit afstaan over kleine enclaves en afgelegen gebieden
aan geheel nieuwe groeperingen en virtuele gemeenschappen. Het is zelfs
niet onwaarschijnlijk dat commerciële partijen, zoals
beveiligingsbedrijven en hotelketens, zullen bieden op soevereiniteit
over kleine stukken grond. Wackenhut, Pinkerton en Argenbright zouden in
de toekomst hybride omheinde gemeenschappen voor gepensioneerden en
belastingvrije zones kunnen aanbieden in aantrekkelijke klimaten
wereldwijd. Religieuze ordes, zoals de Orde van Malta, maar dan van elke
denkbare denominatie, zullen op hun manier proberen een hemel op aarde
te creëren in afgelegen uithoeken. Zelfs rijke individuen en families
zullen hun eigen domeinen bezitten waar zij beperkte soevereiniteit
uitoefenen, eigen postzegels en paspoorten uitgeven en een website
onderhouden.

\section{\texorpdfstring{\textbf{Monopolie en
plundering}}{Monopolie en plundering}}\label{monopolie-en-plundering}

Het is belangrijk op te merken dat de motieven voor het delen of
verhuren van soevereiniteit tegen betaling anders zijn dan de prikkels
waarmee heersers in het verleden te maken hadden, toen hun monopolie op
geweld door rivalen werd uitgedaagd. Verhuurde soevereiniteit is net zo
weinig destabiliserend als het vestigen van een vrijhandelszone.
Daarentegen beïnvloedt militaire machtsstrijd, zoals die van strijdende
krijgsheren en guerrillabendes, rechtstreeks of een aspirant-overheid
meer prikkels heeft om de bevolking te beschermen of juist te plunderen.
Waar rivaliserende groepen in een wankel evenwicht manoeuvreren, nemen
de prikkels tot roofzuchtig geweld toe. Plundering wordt
aantrekkelijker, omdat de macht minder stabiel is en het lokale
geweldsmonopolie minder zeker. Het verkort de tijdshorizonten van wie
geweld kan inzetten. De ``koning van de berg'' staat op zo'n glibberige
helling dat hij er niet op kan rekenen lang genoeg te overleven om een
aandeel te realiseren in de grote winsten die uiteindelijk voortvloeien
uit het indammen van geweld. In dat geval is er weinig dat de
gezaghebbenden ervan weerhoudt om hun macht te gebruiken om de
samenleving te terroriseren en te plunderen.

De logica van dwang leert dus dat hoe meer gewapende groepen er in een
gebied actief zijn, des te groter de kans is dat zij roofzuchtig geweld
zullen inzetten. Zonder een overweldigende macht die freelance-geweld
onderdrukt, heeft het de neiging zich te verspreiden, en gaan veel van
de baten van economische en sociale samenwerking in rook op.

De schade die kan ontstaan wanneer geweld vrij spel krijgt in een
toestand van anarchie, wordt geïllustreerd door het lot van China onder
de krijgsheren in de jaren twintig. Dit verhaal vertelden wij in
\emph{The Great Reckoning}. De rivaliserende krijgsheren richtten grote
schade aan in gebieden waar geen enkele overweldigende macht hen in toom
kon houden. Soortgelijke verhalen, die hetzelfde punt duidelijk maken,
zijn wereldwijd in levendige kleuren uitgezonden door CNN-reportages
vanuit de straten van Mogadishu, Somalië. De strijdkrachten van de
Somalische krijgsheren, bijgenaamd de ``technicals,'' brachten anarchie
naar dat gehavende land voordat de Verenigde Staten een massale
militaire interventie leidden om hen in te dammen. Toen de Amerikaanse
troepenmacht zich terugtrok, kwamen de technicals opnieuw met hun wapens
tevoorschijn, en hervatte de anarchie zich. Een verslag in de
\emph{Washington Post} merkte op:

\begin{quote}
\emph{``{[}P{]}ick-uptrucks met luchtafweerkanonnen rijden opnieuw door
de stoffige, met puin bezaaide straten. Ook zijn de zelfverzekerde jonge
mannen terug, in T-shirts met Kalasjnikovs nonchalant over hun
schouders, die geld afpersen van passerende auto's en bussen bij
geïmproviseerde wegversperringen. Eén wijk hier wordt bestuurd door zo'n
zwaar gewapende militie dat de lokale bevolking ernaar verwijst als
`Bosnië-Herzegovina.' Rondrijden door de gevaarlijke straten van deze
stad doet vandaag sterk denken aan 1992, toen chaotische gevechten
tussen rivaliserende milities Somalië in anarchie en een hongersnood
stortten die een door de VS geleide interventie uitlokte. Opnieuw moeten
reizigers om Mogadishu te doorkruisen een wagen vol gewapende
knokploegen inhuren, in de hoop dat zij bescherming bieden voor honderd
dollar per dag, plus lunchpauze.''230}
\end{quote}

De voorbeelden van Somalië, Rwanda en anderen die u binnenkort op
televisie zult zien, leveren in kleur het bewijs dat een gewelddadige
strijd om territoriale controle niet dezelfde directe economische
voordelen oplevert als andere vormen van competitie. Integendeel. De
rondtrekkende bendes en plunderaars die onder anarchie opereren, missen
zelfs de zwakke prikkels om productieve activiteit te beschermen die
dictators soms nog wel hebben zodra hun heerschappij veiliggesteld is.

\begin{quote}
\emph{``De samenleving die wij de moderne tijd noemen, wordt in het
Westen bovenal gekenmerkt door een bepaald niveau van monopolisering.
Vrij gebruik van militaire wapens is de individuele burger ontzegd en
voorbehouden aan een centrale autoriteit van welke aard dan ook, en
evenzo is de belastingheffing op eigendom of inkomen van individuen
geconcentreerd in de handen van een centrale maatschappelijke
autoriteit. De financiële middelen die zo naar deze centrale autoriteit
vloeien, onderhouden haar monopolie op militaire macht, terwijl dit op
zijn beurt het belastingmonopolie in stand houdt. Geen van beide heeft
in enige zin voorrang op de ander; het zijn twee kanten van hetzelfde
monopolie. Als het ene verdwijnt, volgt het andere automatisch; het
monopolie kan af en toe harder wankelen aan de ene zijde dan aan de
andere.''231\\
-- NORBERT ELIAS}
\end{quote}

\section{\texorpdfstring{\textbf{De evolutie van
bescherming}}{De evolutie van bescherming}}\label{de-evolutie-van-bescherming}

Lane ontwikkelde een argument dat wij ons voor onze analyse van hoe het
Informatie­tijdperk zich mogelijk zal ontvouwen eigen hebben gemaakt. Hij
stelde dat de geschiedenis van de westerse economieën sinds de Donkere
Middeleeuwen kan worden geïnterpreteerd op basis van vier stadia van
competitie en monopolie in de organisatie van geweld. Hoewel Lane
grotendeels zwijgt over de megapolitieke factoren die wij aanwijzen als
bepalend voor de schaal waarop regeringen opereren, sluit zijn
verkenning van de economie van geweld nauw aan bij het betoog dat wij
uiteen hebben gezet in \emph{Blood in the Streets}, \emph{The Great
Reckoning} en elders in dit boek.

We hebben reeds enkele van de megapolitieke factoren geanalyseerd die
een rol speelden in de evolutie van de westerse samenleving na de val
van Rome. Lane onderzocht eveneens deze periode, met de nadruk op de
economische gevolgen van de strijd om het monopolie op geweld. Hij
onderscheidde vier belangrijke stadia in het functioneren van economieën
in de afgelopen duizend jaar, elk gekenmerkt door een andere fase in de
organisatie van geweld.232

\subsection{\texorpdfstring{\textbf{Uit de} D\textbf{onkere}
M\textbf{iddeleeuwen}}{Uit de Donkere Middeleeuwen}}\label{uit-de-donkere-middeleeuwen}

De eerste fase kenmerkt zich door `anarchie en plundering', zoals
tijdens de feodale revolutie van duizend jaar geleden. Lane geeft geen
specifieke data voor de door hem onderscheiden perioden, maar eenvoudige
berekeningen maken duidelijk waar zijn eerste periode begint. Zijn
beschrijving van de fase van `anarchie en plundering' lijkt te passen
bij de omstandigheden tijdens de transitie uit de Donkere Middeleeuwen,
toen het gebruik van geweld `zeer competitief was, zelfs op
land'.{[}42{]} Hij licht niet toe waarom, maar wanneer geweld `zeer
competitief' is. Het wijst doorgaans op aanzienlijke belemmeringen voor
het uitoefenen van macht over afstand. In militaire begrippen: de
verdediging overheerst de aanval.

Zoals we in hoofdstuk 3 hebben toegelicht, viel deze fase van `anarchie
en plundering' samen met een daling van de landbouwproductiviteit als
gevolg van ongunstige klimaatveranderingen. Omdat de technologie
destijds nauwelijks effectieve schaalvoordelen opleverde waarmee een
monopolie op geweld kon worden gevestigd, woedde de concurrentie tussen
potentiële machthebbers hevig. Hierdoor kwam de economische activiteit
ernstig in het gedrang.

De zwakte van de economie verergerde het probleem om een stabiele orde
tot stand te brengen. Het vestigen van een lokaal geweldsmonopolie ging
gepaard met hoge militaire kosten in verhouding tot de beperkte
economische opbrengsten. De bewapende ridders te paard slaagden er niet
in om een effectief monopolie te handhaven over een economisch
levensvatbaar gebied, en terroriseerden en plunderden erop los terwijl
ze nauwelijks echte `bescherming' boden aan hun klanten.

\subsection{\texorpdfstring{\textbf{Feodalisme}}{Feodalisme}}\label{feodalisme}

``De tweede fase begint wanneer kleine regionale of provinciale
monopolies worden gevestigd. De landbouwproductie stijgt dan, en het
grootste deel van het overschot wordt geïnd door de pas opgekomen
monopolisten van geweld.''234: Toch blijft het overschot relatief gering
in deze tweede fase, die wij identificeren met de vroege Middeleeuwen.
Economische groei blijft beperkt door het ontbreken van schaalvoordelen
in de organisatie van geweld, waardoor de militaire kosten voor het
afdwingen van lokale monopolies hoog blijven. Maar hoewel de kosten hoog
zijn, stijgt de prijs die minisouvereinen voor bescherming kunnen
vragen, aangezien economische activiteit toeneemt zodra de anarchie
wordt ingeperkt.

In een latere fase van de tweede periode proberen veel heffinginners
klanten te lokken met speciale aanbiedingen voor landbouw en handel. Zij
bieden bescherming tegen lage prijzen aan wie nieuwe gronden cultiveert,
en leveren extra politiediensten ter bevordering van de handel, zoals de
graven van Champagne deden voor kooplieden die hun jaarmarkten
bezochten.235: Met andere woorden: zodra zij voldoende territoriale
controle hadden om geloofwaardig te kunnen onderhandelen, deden lokale
krijgsheren hetzelfde als kooplieden die hun marktaandeel willen
vergroten, ze boden korting om klanten te trekken. Later gebruikten ze
de extra middelen uit de toegenomen economische activiteit om hun greep
op grotere gebieden te consolideren. Zodra die controle stevig gevestigd
was, begonnen ze meer van de voordelen van een monopolie te genieten. De
militaire kosten voor hun politietaken daalden, en ze konden ook hun
prijzen verhogen zonder zich zorgen te maken dat hun dienst daardoor
minder aantrekkelijk zou worden.

In deze complexe fase van de westerse geschiedenis nemen zij die geweld
gebruiken, de middeleeuwse heren en vorsten, het grootste deel van de
inkomsten boven het bestaansminimum. Er zijn weinig kooplieden, en de
meest succesvolle zijn degenen die het beste in staat zijn belastingen,
heffingen en andere kosten die worden opgelegd als betaling voor
``beschermingsdiensten'' te ontwijken of te minimaliseren.

\subsection{\texorpdfstring{\textbf{De} Vroeg\textbf{moderne}
P\textbf{eriode}}{De Vroegmoderne Periode}}\label{de-vroegmoderne-periode}

Een derde fase wordt bereikt wanneer de kooplieden en grondbezitters die
geen specialisten in geweld zijn ``meer van het economische surplus
overhouden dan leenheren en vorsten\ldots{} In deze derde fase ontvangen
de ondernemingen die gespecialiseerd zijn in het gebruik van geweld
minder van het surplus dan de ondernemingen die bescherming van de
regeringen kopen.''236 Omdat succesvolle kooplieden hun winsten eerder
herinvesteren dan consumeren, leidden de hogere winsten van kooplieden
in die fase van de geschiedenis tot zelfversterkende groei.

\subsection{\texorpdfstring{\textbf{Het tijdperk van de
fabrieken}}{Het tijdperk van de fabrieken}}\label{het-tijdperk-van-de-fabrieken}

Lane identificeert de overgang van de derde naar de vierde fase met het
ontstaan van technologische en industriële innovaties als belangrijkere
factoren voor winst dan het verlagen van de kosten voor bescherming.
Daarmee lijkt Lane te doelen op de periode sinds 1750. Vanaf dat moment
begon de aard van technologie duidelijk een dominante rol te spelen in
de regionale welvaart. Zelfs in gebieden zonder overheid, zoals bepaalde
streken in Nieuw-Zeeland vóór 1840, leidde het feit dat er geen
belastingen waren niet automatisch tot grote welvaart. Op dat punt in de
geschiedenis waren innovaties in industriële technologie belangrijker
voor het behalen van winst dan kostenbesparingen op bescherming, zelfs
als de kosten tot nul zouden zakken. Naarmate de schaal van overheden
toenam, kwamen de krediet- en financieringsmechanismen die
oorspronkelijk door regeringen waren ontwikkeld om middelen voor
militaire operaties te verkrijgen, beschikbaar voor de financiering van
grootschalige ondernemingen.

Hoewel Lane dit niet expliciet zegt, zorgde de concentratie van
technologische voordelen in een bepaald gebied voor minder concurrentie
tussen jurisdicties en stelde dit ``gespecialiseerde ondernemingen in
het gebruik van geweld,'' oftewel overheden, in staat hogere prijzen te
vragen. Wanneer er grote technologische verschillen bestonden tussen de
concurrenten in de ene jurisdictie en die in een andere, zoals tijdens
het Industriële Tijdperk, verdienden ondernemers in de jurisdicties met
de beste technologie doorgaans meer geld, ook al moesten zij hogere
belastingen en andere kosten aan hun overheden betalen.

\subsection{\texorpdfstring{Roof \textbf{met een
glimlach}}{Roof met een glimlach}}\label{roof-met-een-glimlach}

Overheden in het Industriële Tijdperk genoten van een monopolie dat ze
naar hartelust konden uitbuiten. De werkelijke kosten voor het
beschermen van lijf en goed waren verwaarloosbaar klein in verhouding
tot de prijzen (belastingen) die zij hieven. Toch bevonden zij zich in
een situatie waarin de concurrentie zo pervers was dat zij zich veel
meer konden bezighouden met plundering dan met bescherming, zonder dat
dit feit vrijwel werd opgemerkt. Het was een zeldzaam moment in de
geschiedenis.

De nadelen van anarchie onder de megapolitieke omstandigheden van het
industrialisme maakten concurrentie in beschermingsdiensten binnen
hetzelfde grondgebied technologisch onuitvoerbaar. De enige manier om
effectieve bescherming onder die omstandigheden te bereiken, was door te
beschikken over een groter vermogen tot geweld. Daarom viel er weinig te
winnen met een poging om beter te onderscheiden welk deel van de
belastingen, in Lane's woorden, ``als betaling voor de geleverde
dienst'' gold en welk deel ``roof genoemd kan worden.''237 Het
onderscheid was er wel degelijk. Aangezien men echter hoe dan ook
vastzat aan het betalen van belastingen, had het onderzoeken hiervan
weinig toegevoegde waarde, behalve om morbide nieuwsgierigheid te
bevredigen. Zoals Lane zei: ongeacht hoe groot het deel van de
belastingen roof was, was het een prijs die men moest betalen ``om
ernstigere verliezen te vermijden.''238

\subsection{\texorpdfstring{\textbf{De stijging van inkomens onder het
industrialisme}}{De stijging van inkomens onder het industrialisme}}\label{de-stijging-van-inkomens-onder-het-industrialisme}

Het feit dat de inkomens fors stegen, vooral in de landen waar de meeste
industriële ontwikkeling plaatsvond, verklaart deels waarom dit dilemma
in de afgelopen twee eeuwen van natiestaat-heerschappij aanvaardbaar
bleef. De overheden van de OESO-landen belastten bijna elk jaar een
hoger percentage van de inkomens, maar de toename van de plundering ging
desalniettemin gepaard met een veel grotere welvaart, en een grotere
ongelijkheid van rijkdom ten opzichte van de rest van de wereld. Onder
zulke omstandigheden waren bezwaren tegen de stijgende belastingdruk
onvermijdelijk marginaal en onvoldoende om een ander verloop van
gebeurtenissen te realiseren dan de meest logische koers. Sterker nog,
om redenen die in eerdere hoofdstukken uiteen zijn gezet, hing het
militaire voortbestaan van een industriële natiestaat grotendeels af van
haar onbegrensde aanspraken op de middelen van haar burgers.

In elke industriële staat bewogen de beleidslijnen zich min of meer in
dezelfde richting. Op het hoogtepunt van het industrialisme na de Tweede
Wereldoorlog bereikte het marginale inkomstenbelastingtarief 90 procent
of hoger. Dit was een veel agressievere aanspraak van de staat om
middelen af te romen dan zelfs de oosterse despoten van de vroege
hydraulische beschavingen gewoonlijk maakten. Toch volgde de industriële
variant van plundering zijn eigen logica. Veel ervan werd bepaald door
het karakter van de industriële technologie in de eerste helft van de
twintigste eeuw, zoals eerder beschreven.

Deze technologie maakte het vrijwel onvermijdelijk dat de staat een
groot deel van het inkomen zou opeisen en herverdelen, waarbij een groot
deel van de lasten van de roof op een kleine groep kapitalisten viel. De
meeste industriële processen waren sterk afhankelijk van natuurlijke
hulpbronnen, en dus gebonden aan de locaties waar die zich bevonden. Een
staalfabriek, een mijn of een haven kon slechts tegen astronomische
kosten worden verplaatst, of helemaal niet. Zulke faciliteiten waren
daarom stationaire doelwitten die eenvoudig belast konden worden.
Onroerendgoed-, vennootschaps- en winstbelastingen stegen sterk in deze
eeuw. Dat gold ook voor de inkomstenbelasting, aanvankelijk voor de
kapitalisten, maar uiteindelijk ook voor de arbeiders zelf. De opkomst
van grootschalige industriële werkgelegenheid maakte een brede
inkomstenbelasting praktisch uitvoerbaar. Lonen konden aan de bron
worden ingehouden, waarbij de belastingdiensten de inning coördineerden
met de boekhoudafdelingen van industriële ondernemingen. Dat beschouwen
we vandaag als vanzelfsprekend, maar het innen van een
inkomstenbelasting bij de fabriekspoort was een veel eenvoudigere taak
dan het afpersen van een deel van de winsten van miljoenen zelfstandige
ambachtslieden en boeren over het uitgestrekte platteland.

Kortom, industriële technologie maakte belastingen meer geroutineerd,
voorspelbaarder en minder persoonlijk riskant dan in veel eerdere
perioden. Desalniettemin slaagde het erin om een hoger percentage van de
middelen van de maatschappij af te romen dan welke andere vorm van
soevereiniteit tot dan toe had gedaan.

\subsection{\texorpdfstring{\textbf{Wat wordt
beschermd?}}{Wat wordt beschermd?}}\label{wat-wordt-beschermd}

Het feit dat samenlevingen rijker konden worden terwijl het totale
percentage van het inkomen dat in belastingen werd afgeroomd aanzienlijk
steeg, roept de vraag op naar het karakter van de bescherming die
regeringen aan industriële economieën boden. Wat beschermden zij? Ons
antwoord: voornamelijk industriële installaties met hoge kapitaalkosten
en een grote kwetsbaarheid voor aanvallen. Het bestaan van grootschalige
industriële ondernemingen zou onmogelijk zijn geweest in een chaotische
omgeving met meer competitief geweld, zelfs als het resultaat van die
concurrentie een lager belastingpercentage zou zijn geweest.

Dit is waarom kapitaalintensieve activiteiten oneconomisch zijn in de
Amerikaanse sloppenwijken, evenals in derdewereldsamenlevingen waar
ad-hocgeweld endemisch is. De industriële samenleving als geheel kon
zich ontwikkelen omdat er een bepaald soort orde werd gevestigd en
gehandhaafd. Ondernemingen werden blootgesteld aan regelmatige,
voorspelbare afpersingen in plaats van grillig geweld.

Zelfs op het hoogtepunt van het industrialisme was het altijd overdreven
om te spreken van een overheid die een ``monopolie op dwang''
uitoefende. Alle overheden proberen een dergelijk monopolie te
handhaven, maar zoals we hebben gezien, ontdekten werknemers van
industriële bedrijven meestal dat zij in staat waren om geweld tegen hun
werkgevers te gebruiken. Zolang het grote publiek toegang heeft tot enig
wapen, of een wanordelijke menigte fysiek in staat is een bus omver te
werpen of stenen naar de politie te gooien, monopoliseren degenen die
controle hebben over de overheid het geweld niet volledig. Zij beheersen
slechts het overheersende geweld, dominant in die mate dat het voor de
meeste mensen economisch onaantrekkelijk wordt om met hen te concurreren
onder de bestaande omstandigheden.

\begin{quote}
\emph{Een internetregering kan alleen functioneren met toestemming van
de geregeerden. Elke internetoverheid moet haar burgers dus echte
voordelen bieden als zij wil dat zij blijven. Die voordelen hoeven niet
alleen persoonlijke goederen of diensten te zijn, maar kunnen ook de
bredere voordelen van een regulerend regime omvatten: een schoon,
transparant marktplein met gedefinieerde regels en consequenties, of een
toezichthoudende gemeenschap waar kinderen de mensen die zij tegenkomen
kunnen vertrouwen en waar de privacy van individuen wordt
beschermd.''239 -- Esther Dyson}
\end{quote}

\subsection{\texorpdfstring{\textbf{Het}
I\textbf{nformatietijdperk}}{Het Informatietijdperk}}\label{het-informatietijdperk}

Het Informatietijdperk markeert de komst van een vijfde fase in de
ontwikkeling van de manier waarop in het Westen om het gebruik van
geweld wordt geconcurreerd. Deze fase werd niet door Lane voorzien. Het
gaat hier om competitie in de cyberspace, een arena die niet kan worden
gemonopoliseerd door wat voor ``geweldsgebruikende onderneming'' dan
ook. Dat is onmogelijk omdat de cyberspace geen territorium is.

Lane's analyse vertrok nog vanuit de conventionele naoorlogse aanname
van het onvermijdelijke bestaan van de natiestaat. Toch wees hij op een
punt dat vandaag veel crucialer blijkt te zijn dan het vijftig jaar
geleden leek: overheden hebben nooit stabiele monopolies van dwang
gevestigd op de open zee. Geen enkele staat kon haar wetten daar
exclusief handhaven. Dit gegeven is van groot belang om te kunnen
begrijpen hoe de organisatie van geweld en bescherming zich zal
ontwikkelen nu de economie naar de cyberspace migreert, een domein
zonder fysieke grenzen of tastbare ruimte. Om dezelfde redenen die Lane
opmerkte voor het feit dat geen enkele overheid er ooit in slaagde het
geweld op zee te monopoliseren, zo is het nog minder aannemelijk dat een
overheid een grenzeloos, onbegrensd domein effectief zou kunnen
monopoliseren.

\section{\texorpdfstring{\textbf{Concurrentie zonder
anarchie}}{Concurrentie zonder anarchie}}\label{concurrentie-zonder-anarchie}

In het verleden, wanneer de omstandigheden het voor een enkel
gewelddadig machtsapparaat moeilijk maakten om een monopolie te
vestigen, volgde anarchie en plundering. Het Informatie­tijdperk
daarentegen heeft de technologische voorwaarden voor de organisatie van
geweld op een diepgaande manier veranderd. Anders dan vroeger, toen het
onvermogen om bescherming in een regio te monopoliseren hogere militaire
kosten en lagere economische opbrengsten betekende, impliceert het feit
dat overheden de cyberspace niet kunnen monopoliseren juist lagere
militaire kosten en hogere economische opbrengsten. Dit komt doordat
informatietechnologie een nieuwe dimensie van bescherming creëert. Voor
het eerst in de geschiedenis maakt informatietechnologie het mogelijk om
activa te scheppen en te beschermen die volledig buiten het domein van
het territoriale geweldsmonopolie van welke individuele overheid dan ook
liggen.

\begin{quote}
\emph{``Landen met een gefragmenteerde politieke macht en bestuur
waarbij een centrale, stabiele en onbetwiste toezichthoudende bron van
jurisdictie en macht ontbreekt, moeten hun eigen werkbare oplossingen
bedenken om om te gaan met de problemen die door zulke grenzen worden
opgeworpen.''240\\
-- REES DAVIES}
\end{quote}

\subsection{\texorpdfstring{\textbf{De analogie met de
grens}}{De analogie met de grens}}\label{de-analogie-met-de-grens}

De cyberspace is in zekere zin het equivalent van een technologisch
beschermde \emph{march}-regio zoals die bestond in grensgebieden tijdens
de Middeleeuwen. In het verleden, toen de macht van heren en koningen
zwak was en de aanspraken van één of meer elkaar overlapten aan een
grens, bestond er iets dat leek op concurrentie tussen de overheden. Een
blik op hoe de \emph{march}-regio's functioneerden kan inzicht geven in
hoe wetten van de \emph{march} of iets dergelijks mogelijk ook in de
cyberspace zullen ontstaan.

Andorra overleeft als een soort versteende \emph{march}-regio tussen
Frankrijk en Spanje, een overblijfsel van megapolitieke omstandigheden
die het voor beide koninkrijken moeilijk maakten om de overhand te
krijgen in dat koude en bijna ontoegankelijke gebied in de Pyreneeën van
zo'n 490 vierkante kilometer. In 1278 werd een overeenkomst gesloten
waarbij de suzereiniteit over Andorra werd verdeeld tussen lokale Franse
en Spaanse feodale heren, de Franse graaf van Foix en de Spaanse
bisschop van Urgel. Beiden benoemden een van de twee ``viquiers'' die de
minimale overheidsmacht in Andorra spaarzaam uitoefenden, voornamelijk
door het bevel te voeren over de kleine Andorrese militie, nu een
politie-eenheid. De rol van de graaf is in de loop van de geschiedenis
allang achterhaald. De Franse regering vertegenwoordigt hem nu vanuit
Parijs. Eén van haar taken is het innen van de helft van het jaarlijkse
tribuut dat Andorra betaalt, een bedrag dat lager is dan de maandhuur
van een bouwvallig appartement. De bisschop van Urgel blijft zijn deel
van het tribuut ontvangen, net als zijn voorgangers in de Middeleeuwen.

Zoals het gedeelde tribuut impliceert, waren er in Andorra twee bronnen
van ``toezichthoudende jurisdictie en macht'' in plaats van één.
Beroepen tegen civiele vonnissen in Andorra werden traditioneel
ingediend bij het bisschoppelijk college van Urgel of bij het Hof van
Cassatie in Parijs.

Een gevolg van Andorra's dubbelzinnige positie was dat er vrijwel geen
wetten werden uitgevaardigd. Andorra heeft al meer dan zevenhonderd jaar
een verwaarloosbaar kleine overheid en geen belastingen. Vandaag de dag
heeft het land daardoor een groeiende aantrekkingskracht als
belastingparadijs. Tot een generatie geleden stond Andorra echter bekend
als arm. Wat eens een dicht bos was, werd over de eeuwen heen ontbost
door de bewoners die warm wilden blijven in de strenge winters; elk jaar
ligt het van november tot april volledig onder de sneeuw. Zelfs in de
zomer is Andorra zo koud dat gewassen alleen op de zuidelijke hellingen
groeien. Deze beschrijving doet het misschien als onprettig overkomen,
maar dat is net het geheim van haar succes. Andorra overleefde als
feodale enclave in het tijdperk van de natiestaat omdat het afgelegen en
straatarm was.

Er waren ooit talrijke middeleeuwse grens- of \emph{march}-regio's waar
soevereiniteiten met elkaar vermengd raakten. Deze gewelddadige, vaak
arme grenzen bleven tientallen of soms honderden jaren bestaan in de
grensgebieden van Europa. Zoals eerder vermeld, waren er \emph{marches}
tussen gebieden onder Keltische en Engelse controle in Ierland, tussen
Wales en Engeland, Schotland en Engeland, Italië en Frankrijk, Frankrijk
en Spanje, Duitsland en de Slavische grenzen van Midden-Europa, en
tussen de christelijke koninkrijken van Spanje en het islamitische
koninkrijk Granada. Net als Andorra ontwikkelden deze
\emph{march}-regio's eigen, specifieke institutionele en juridische
vormen van een soort die we waarschijnlijk opnieuw zullen zien in het
volgende millennium.

Door de zwakke positie van de twee concurrerende autoriteiten, riepen
heersers soms zelfs vrijwilligers uit hun onderdanen op om zich in
\emph{march}-regio's te vestigen om zo hun invloed te vergroten. Bijna
vanzelfsprekend werden de onderdanen gelokt met vrijstelling van
belastingen. Aangezien ze opereerden op flinterdunne marges, zou het
invoeren van belastingen door een van de autoriteiten in een grensregio,
de eigen volgelingen financieel onder druk zetten en ze een reden geven
om zich met de tegenpartij te verbinden. Daarom hadden inwoners van een
\emph{march} meestal de keuze wiens wetten zij zouden gehoorzamen. Deze
keuze was gebaseerd op de zwakke positie van de concurrerende
autoriteiten; het was geen ideologisch gebaar.

Toch ontstonden er praktische moeilijkheden die moesten worden opgelost.
Onder het feodalisme werden landeigenaren met bezittingen aan beide
zijden van een nominale grens geconfronteerd met een ernstig conflict
wat betreft hun plichten. Bijvoorbeeld: een heer aan de grens van
Schotland en Engeland die eigendommen in beide koninkrijken bezat, kon
theoretisch militaire dienst aan beide kanten verschuldigd zijn in geval
van oorlog. Om deze tegenstrijdige verplichting op te lossen kon vrijwel
iedereen in de feodale hiërarchie kiezen wiens wetten hij gehoorzaamde
via een juridisch proces dat \emph{avowal} werd genoemd.

Door informatietechnologie zullen er ook mogelijkheden ontstaan om
economische activiteit te vestigen op concurrerende locaties, maar met
belangrijke verschillen. Een daarvan is dat de cyberspace naar
verwachting, anders dan de middeleeuwse grensregio's, uiteindelijk het
rijkste economische terrein zal vormen. Het zal dus een groeiend domein
zijn, in plaats van een die langzaam wegkwijnt. Zonder sterke, vaak
religieuze prikkels zouden maar weinig mensen uit de kerngebieden van de
middeleeuwse samenleving de wens hebben gehad om naar de grenzen te
verhuizen, omdat deze regio's doorgaans gewelddadig en arm waren. Ze
trokken dus geen middelen weg uit de macht van de autoriteiten. De
cyberspace zal dat wel doen.

Ten tweede zal het nieuwe grensgebied geen duopolie zijn, dat
samenwerking tussen twee autoriteiten uitlokt om tot een compromis te
komen over hun aanspraken erop. Zulke compromissen waren in de
middeleeuwen meestal niet effectief, deels vanwege de grote culturele
verschillen tussen de rivalen, en vooral omdat zij de fysieke macht
misten om een regeling af te dwingen. In het tijdperk van de natiestaat
verdwenen de meeste \emph{marches} en vage grenzen zodra nationale
autoriteiten wél over voldoende militaire macht beschikten om
oplossingen af te dwingen. Grensafbakening werd de norm, wat een
stabiele oplossing is wanneer een duopolie het gezag over het geweld in
twee aaneengesloten gebieden moet verdelen. Maar de concurrentie om
transacties te vestigen in de cybereconomie zal niet plaatsvinden tussen
twee autoriteiten, maar tussen honderden autoriteiten wereldwijd. Voor
territoriale staten zal het vrijwel onmogelijk zijn om de
belastingtarieven hoog te houden door het vormen van een kartel, net als
dat kartelvorming om monopoliewinsten te behouden niet werkt in markten
met honderden concurrenten.

De Seychellen, een klein land in de Indische Oceaan, bewijst dit punt.
Ze voerden een investeringswet in die door Amerikaanse functionarissen
werd beschreven als de wet ``Welkom Criminelen''. Volgens die wet krijgt
iedereen die \$10 miljoen investeert niet alleen bescherming tegen
uitlevering, maar ook een diplomatiek paspoort. In tegenstelling tot de
beweringen van de Amerikaanse regering waren de beoogde begunstigden
echter geen drugsdealers, die doorgaans toch al onder de bescherming van
belangrijkere overheden vallen, maar onafhankelijke ondernemers die
politiek incorrect waren geworden. De eerste potentiële begunstigde van
de Seychellen-wet was een blanke Zuid-Afrikaan die rijk werd door de
economische sancties tegen het voormalige apartheidsregime te omzeilen.
Nu loopt hij het risico op economische represailles van de nieuwe
Zuid-Afrikaanse regering en is hij bereid de Seychellen te betalen voor
bescherming.241

Wat de merites van individuele gevallen ook zijn, het voorbeeld laat
zien waarom pogingen van overheden om een kartel voor bescherming te
vormen, gedoemd zijn te mislukken. Anders dan in de middeleeuwse
grensgebieden, waar de concurrentie slechts tussen twee autoriteiten
plaatsvond, zullen de grenzen in de cyberhandel tussen honderden
jurisdicties liggen, en waarschijnlijk zal het zelfs tot in de duizenden
oplopen.

In het tijdperk van de virtuele onderneming zullen individuen ervoor
kiezen hun inkomensgenererende activiteiten te vestigen in de
jurisdictie die de beste dienst levert tegen de laagste kosten. Met
andere woorden, soevereiniteit zal worden gecommercialiseerd. Anders dan
de meeste middeleeuwse \emph{march}-samenlevingen, die arm en
gewelddadig waren, zal de cyberspace dat niet zijn. De concurrentie
waarin informatietechnologie overheden dwingt zich te begeven, is geen
militaire concurrentie, maar concurrentie in de kwaliteit en prijs van
een economische dienst: echte bescherming. Kortom, overheden zullen hun
klanten moeten geven wat ze willen.

\subsection{\texorpdfstring{De verminderde effectiviteit \textbf{van
geweld}}{De verminderde effectiviteit van geweld}}\label{de-verminderde-effectiviteit-van-geweld}

Dit wil natuurlijk niet zeggen dat overheden zullen ophouden geweld te
gebruiken. Integendeel, wij stellen dat geweld een groot deel van zijn
effectiviteit verliest. Een mogelijke reactie van overheden zou kunnen
zijn om hun gebruik van geweld in lokale contexten te intensiveren, als
poging om het verlies aan hun mondiale betekenis te compenseren. Wat
overheden ook doen, zij zullen de cyberspace echter niet kunnen
verzadigen met geweld zoals ze de gebieden die ze in de moderne wereld
monopoliseerden met geweld konden verzadigen. Hoeveel overheden ook
proberen de cyberspace te betreden, zij zullen daar niet capabeler of
machtiger zijn dan anderen.

Ironisch genoeg zouden pogingen van natiestaten om
``informatie-oorlogen'' te voeren om de toegang tot de cyberspace te
domineren of te blokkeren, waarschijnlijk alleen hun eigen ondergang
versnellen. De neiging tot de devolutie van grote systemen is al sterk
aanwezig door het wegvallen van schaalvoordelen en de stijgende kosten
om uiteenvallende sociale groepen bijeen te houden. De ironie van
informatie-oorlogen is dat ze mogelijk de broze systemen die uit het
Industriële Tijdperk zijn overgebleven meer schaden dan de opkomende
informatiemaatschappij zelf.

Zolang essentiële informatietechnologie blijft functioneren, kan
cyberhandel parallel doorgaan met de strijd van een informatie-oorlog,
op een manier die in een territoriale oorlog onmogelijk zou zijn. Het is
ondenkbaar dat miljoenen commerciële transacties plaatsvinden aan het
front in een oorlog uit de twintigste eeuw. Virtuele oorlogen hoeven
echter de capaciteit van de cyberspace om meerdere activiteiten te
huisvesten niet uit te putten, en omdat de virtuele realiteit niet
fysiek bestaat, levert nabijheid tot het front nauwelijks gevaar op, en
is er al helemaal geen risico op exploderende virtuele granaatscherven.

\subsection{\texorpdfstring{\textbf{Kwetsbaarheid van grootschalige
systemen}}{Kwetsbaarheid van grootschalige systemen}}\label{kwetsbaarheid-van-grootschalige-systemen}

In een informatieoorlog lopen voornamelijk de grootschalige industriële
systemen die draaien op centraal bevel en controle gevaar. Militaire
autoriteiten in de Verenigde Staten en andere toonaangevende landen
maken zich zorgen over en bereiden zich voor op informatiesabotage,
waarbij grote systemen kunnen worden uitgeschakeld. Cyberaanvallen
kunnen bijvoorbeeld een telefoonwisselstation stilleggen, de
luchtverkeersleiding ontregelen of een pompsysteem dat de watertoevoer
naar een stad regelt saboteren. Een computervirus kan zelfs
conventionele of nucleaire energiebronnen doen uitvallen, waardoor delen
van het elektriciteitsnet tot stilstand komen. Zogenaamde logic bombs
zijn in staat om enorme hoeveelheden informatie te vernietigen,
waaronder de uiterst gevoelige gegevens in de centrale controlesystemen
die de kwetsbare, grote industriële installaties aansturen. Tenzij men
daadwerkelijk alle informatietechnologie massaal vernietigt, wat de
wereldeconomie letterlijk tot stilstand zou brengen, zullen overheden er
niet in slagen om de internethandel en de virtuele realiteit te
onderdrukken of monopoliseren.

Zelfs een belangrijk nadeel van informatietechnologie, namelijk de
vermeende kwetsbaarheid voor de verval en vernietiging van
data-opslagsystemen, wordt grotendeels door de recente
archiveringstechnologie opgelost. Een nieuw systeem, High-Density
Read-Only Memory (HD-ROM), maakt gebruik van een ionenfrees,
vergelijkbaar met die in computergestuurde productie, om archieven in
vacuüm te vervaardigen. De opslagcapaciteit bereikt maar liefst 10.000
megabytes per vierkante centimeter. Waar oudere systemen snel
achteruitgingen door bederf en schokken, verzekert HD-ROM dat de
opgeslagen data de volledige levensduur beschikbaar blijft. Eén van de
ontwikkelaars van HD-ROM, Bruce Lamartine, verklaart: `Het is vrijwel
ondoordringbaar voor de verwoestende invloeden van tijd, thermische en
mechanische schokken, en elektromagnetische velden die andere
opslagmedia beschadigen.'{[}49{]} Zelfs een nucleaire explosie leidt
niet per se tot de verstoring of vernietiging van vitale informatie,
zoals de codes voor digitaal geld, waarop het soepele functioneren van
een cybereconomie steunt.

\begin{quote}
\emph{``Moderne legers zijn zo afhankelijk van informatie dat je ze kunt
verblinden en doof maken, om zo een overwinning te behalen zonder op de
conventionele wijze te vechten.''{[}50{]}\\
- COL. ALAN CAMPEN, U.S.A.F (Gep.)}
\end{quote}

\section{\texorpdfstring{\textbf{Supermachten in de virtuele
oorlogsvoering}}{Supermachten in de virtuele oorlogsvoering}}\label{supermachten-in-de-virtuele-oorlogsvoering}

De veronderstellingen van de oorlogvoerende natiestaat zullen steeds
minder logisch zijn op megapolitiek vlak naarmate informatie
belangrijker wordt in de oorlogsvoering. Omdat de cyberspace geen
fysieke vorm heeft, zijn parameters zoals wij die in de fysieke wereld
kennen niet bepalend. Het maakt niet uit hoeveel programmeurs betrokken
waren bij het opstellen van een reeks commando's. Het enige dat telt is
of het programma functioneert. Het Soevereine Individu kan in de
cyberspace werkelijk net zoveel betekenen als een natiestaat met een
zetel in de VN, een eigen vlag en een leger op de grond. In puur
economische termen beschikken sommige Soevereine Individuen al over
investeerbare inkomens van honderden miljoenen per jaar, bedragen die
het discretionaire bestedingsvermogen van sommige failliete natiestaten
overtreffen. Maar dat is niet alles. Op het gebied van virtuele
oorlogsvoering via informatie kunnen sommige individuen net zo machtig
of machtiger zijn dan veel staten. Één bizar genie, die met digitale
helpers werkt, zou in theorie dezelfde impact in een cyberoorlog kunnen
hebben als een natiestaat. Bill Gates zou dat zeker kunnen.

In die zin is het tijdperk van het Soevereine Individu niet slechts een
slogan. Een hacker, of een klein groepje wiskundigen, laat staan een
bedrijf als Microsoft, of vrijwel elk softwarebedrijf, kan in principe
alles doen wat het Pentagon's Cyber War Task Force kan. Honderden
bedrijven in Silicon Valley en elders hebben al een grotere capaciteit
om een cyberoorlog te voeren dan 90 procent van de bestaande
natiestaten. De aanname dat overheden het leven op de grond zullen
blijven monopoliseren terwijl alternatieve beschermingsmiddelen overal
opkomen, is een anachronisme. het is waarschijnlijker dat natiestaten
opnieuw ingericht moeten worden om hun kwetsbaarheid te verminderen voor
computervirussen, logische bommen, geïnfecteerde kabels en
trapdoor-programma's onder controle van de Amerikaanse National Security
Agency, of van een tienerhacker.

De megapolitieke logica van de cyberspace suggereert dat centrale
command-and-control-systemen die momenteel de wereldwijde infrastructuur
domineren, vervangen moeten worden door multicentrische
beveiligingsmodellen met gedistribueerde capaciteiten zodat ze niet
eenvoudig door een virus kunnen worden overgenomen of geblokkeerd.
Nieuwe soorten software, bekend als agoric open systems, zullen
command-and-control-software uit het Industriële Tijdperk vervangen. Die
oudere software verdeelde rekenkracht volgens rigide prioriteiten,
vergelijkbaar met hoe centrale planners bij Gosplan in de voormalige
Sovjet-Unie goederen aan wagons toewezen volgens vaste regels. De nieuwe
systemen worden bestuurd door algoritmen die marktmechanismen nabootsen
zodat middelen efficiënter toegewezen worden via een intern biedproces
dat de competitieve processen in de hersenen imiteert. In plaats van
gigantische computermonopolies die belangrijke
command-and-control-functies uitvoeren, zullen ze in het nieuwe
millennium gedecentraliseerd zijn.

Er is geen beter voorbeeld van de veerkracht van gedistribueerde
netwerken vergeleken met command-and-control-systemen dan dat van
Digital Equipment in het onderzoekscentrum in Palo Alto. Een ingenieur
opende een kast met het bedrijfseigen computernetwerk. Zoals Kevin Kelly
vertelt, ``trok de ingenieur dramatisch een kabel uit het netwerk,
waarop het rond de schade routeerde en geen moment haperde.''244 :

Het Informatietijdperk zal niet alleen concurrentie zonder anarchie in
de cyberspace mogelijk maken, het zal onvermijdelijk leiden tot het
herontwerp van belangrijke systemen uit het industrialisme. Een
dergelijke herconfiguratie is essentieel om ze minder kwetsbaar te maken
voor schade van wie of waar dan ook. Net zoals het Industriële Tijdperk
onvermijdelijk leidde tot de herconfiguratie van overgebleven
middeleeuwse instituties zoals scholen en universiteiten, zullen de
overgebleven instituties van het Industriële Tijdperk waarschijnlijk in
miniatuurvorm evolueren, volgens de logica van de microtechnologie.

De noodzaak tot bescherming tegen bandieten op de Informatiesnelweg zal
de wijdverspreide toepassing van public key-private key-encryptie
vereisen. Deze maken het nu mogelijk voor elke gebruiker van een
personal computer om een bericht veiliger te versleutelen dan het
Pentagon haar lanceercodes een generatie geleden had kunnen beveiligen.
Deze krachtige, onbreekbare vormen van encryptie zullen nodig zijn om
financiële transacties tegen hackers en dieven te beschermen.

Ze zijn ook om een andere reden noodzakelijk. Particuliere financiële
instellingen en centrale banken zullen onbreekbare encryptie-algoritmen
toepassen zodra ze beseffen dat de Amerikaanse overheid, en mogelijk
niet alleen zij, in staat is om de huidige banksoftware en
computersystemen te penetreren en een land letterlijk failliet te laten
gaan of de bankrekening van bijna iedereen leeg te halen. Er is geen
technologische reden waarom een individu of land zijn financiële
tegoeden of transacties, legaal of illegaal, overgeleverd zou moeten
laten aan de Amerikaanse National Security Agency of opvolgers van de
KGB, of een soortgelijke organisatie.

Encryptie-algoritmen die door overheden niet te doorbreken zijn, zijn
geen droombeelden. Ze zijn al als shareware via het internet
beschikbaar. Wanneer satellietsystemen in een lage baan rond de aarde
volledig operationeel zijn, zullen individuen met geavanceerde personal
computers en antennes niet groter dan die van mobiele telefoons overal
ter wereld kunnen communiceren zonder het telefoonnetwerk te gebruiken.
Het zal even onmogelijk zijn voor een overheid om de cyberspace, een
niet-fysieke werkelijkheid, te monopoliseren als het voor middeleeuwse
ridders zou zijn geweest om de transacties in het industriële tijdperk
te paard te controleren.

\subsection{\texorpdfstring{\textbf{Bescherming door
onopvallendheid}}{Bescherming door onopvallendheid}}\label{bescherming-door-onopvallendheid}

Informatiemaatschappijen zullen een enorme hoeveelheid middelen buiten
het bereik van rovers plaatsen. Wanneer de cyberspace steeds meer
financiële transacties en andere vormen van handel gaat huisvesten,
zullen de daar gebruikte middelen min of meer immuun zijn voor gewone
afpersing en diefstal. Rovers zullen daardoor niet in staat zijn om een
zo groot deel van de middelen te veroveren als ze vandaag de dag doen en
hebben gedaan gedurende het grootste deel van de twintigste eeuw.

Daarom zal overheidsbescherming van een groot deel van de wereldrijkdom
onnodig worden. De overheid zal niet beter in staat zijn om een
banktegoed in de cyberspace te beschermen dan uzelf. Omdat de overheid
minder noodzakelijk zal zijn, zal haar relatieve prijs waarschijnlijk
alleen al om die reden dalen, maar er zijn nog andere.

Nu een groot en groeiend aandeel van de financiële transacties in dit
nieuwe millennium in de cyberspace plaatsvindt, zullen individuen een
keuze hebben in welke jurisdictie zij hun transacties onderbrengen. Dit
zal intense concurrentie creëren om de prijs van overheidsdiensten (de
heffingen die zij oplegt) op een niet-monopolistische wijze te bepalen.
Dit is revolutionair. Zoals George Melloan betoogde in \emph{The Wall
Street Journal}, `de instelling die veruit het meest succesvol de
krachten van wereldwijde concurrentie heeft weerstaan, is de
verzorgingsstaat.' Een studie door onderzoekers van de Wharton School en
de Australian National University besprak de krachten die van invloed
zijn op inkomensherverdeling." Garrett en Mitchell stelden vast dat er
`vrijwel geen aanwijzingen bestaan dat meer marktintegratie de
belangrijkste sociale vangnetten heeft verzwakt.' Integendeel, schrijven
zij, 'overheden hebben altijd gereageerd op toenemende integratie in
internationale markten door inkomensherverdeling te vergroten.'245 De
opkomst van de cybereconomie zal eindelijk de verzorgingsstaat
blootstellen aan echte concurrentie. Het zal de aard van soevereiniteit
veranderen en economieën transformeren, omdat de balans tussen
bescherming en afpersing meer dan ooit in het voordeel van bescherming
zal verschuiven.

{[}1{]} Neil Munro, `The Pentagon's new nightmare: an electronic Pearl
Harbor', \emph{Washington Post}, 16 juli 1995, p.~C3.

{[}2{]} Thomas Hobbes, Leviathan, hoofdstuk 13, `De natuurlijke toestand
van de mens betreffende hun gelukzaligheid en ellende'.

{[}3{]} Thomas Schelling, Arms and Influence (New Haven: Yale University
Press, 1966).

{[}4{]} Kevin Kelly, Out of Control: The New Biology of Machines, Social
Systems, and the Economic World (Reading, Mass.: Addison-Wesley, 1995),
pp.~45--46.

{[}5{]} Zie hoofdstuk 8 van \emph{The Great Reckoning}, `Lineaire
verwachtingen in een niet-lineaire wereld: hoe de telescoop ons leerde
rekenen; hoe de computer ons kan helpen te zien.'

{[}6{]} Ibid., pp.~2--4.

{[}7{]} Ibid., p.~4.

{[}8{]} Heinz Pagels, \emph{the dreams of reason} (New York: Bantam
Books, 1989), geciteerd in Roger Lewin, \emph{complexity: life at the
edge of chaos} (New York: Macmillan, 1992), p.~10.

{[}9{]} Frederic C. Lane, `the economic meaning of war and protection'
in \emph{Venice and history: the collected papers of Frederic C. Lane}
(Baltimore: The Johns Hopkins Press, 1966), pp.~383--384.

{[}10{]} Shi Mai'an and Lao Guanzhong, \emph{outlaws of the marsh},
vertaald door Sidney Shapiro (Bloomington: Indiana University Press,
1981), p. 12.

{[}11{]} George E Will, `Vaarwel aan de welvaartsstaten,'
\emph{Washington Post}, 17 december 1995, p.~C7.

{[}12{]} Robert S. MeElvaine, De grote depressie: Amerika, 1929--1941
(New York: \emph{Times Books}, 1984), p.~292.

{[}13{]} Ibid., p.~293.

{[}14{]} Smith, op. cit., p.~75.

{[}15{]} Ibid., p.~76.

{[}16{]} Ibid.

{[}17{]} Een van de eerste vakbonden in Argentinië die zich
daadwerkelijk organiseerden, was de spoorwegvakbond, opgericht in 1887.
Zie Carmelo Mesa-Lago, \emph{Social Security in Latin America: Pressure
Groups, Stratification, and Inequality} (Pittsburgh: University of
Pittsburgh Press, 1978), p.~161.

{[}18{]} Sloan, op. cit., p.~202. Zie ook S. S. Boynton, `Miners'
vengeance', \emph{`Overland Monthly'}, vol.~22 (1893), pp.~303--307

{[}19{]} Benjamin Schwartz, `American inequality: its history and scary
future', \emph{`New York Times'}, 19 december 1995, p.~A25

{[}20{]} MeElvaine, op. cit., p.~293

{[}21{]} Ibid.

{[}22{]} Ibid.

{[}23{]} Henry C. Simons, `Enkele overpeinzingen over syndicalisme',
\emph{Journal of Political Economy}, maart 1944, p.~22.

{[}24{]} Kelly, op. cit., pp.~191--192.

{[}25{]} Gayle N. Hanson, `Een meeslepend verslag van ``het leven'' in
de postmodernistische cyberspace', \emph{Washington Times}, 24 december
1995, p. B7.

{[}26{]} Een beknopte inleiding op het academische onderzoek naar
anarchie vindt u in Gordon Tullock (red.), \emph{Explorations in the
Theory of Anarchy} (Blacksburg, Va.: Virginia Polytechnic Institute and
State University, 1972). Zie ook Murray N. Rothbard, \emph{Power and
Market: Government and the Economy} (Menlo Park, Calif., 1970) en Robert
Nozick, \emph{Anarchy, State and Utopia} (New York: Basic Books, 1974).

{[}27{]} Zie Pierre Clastres, \emph{Society Against the State: The
Leader as Servant and the Humane Uses of Power Among the Indians of the
Americas} (New York: Urizen Books, 1977) en Jones, op. cit.

{[}28{]} Lane, \emph{economische gevolgen van georganiseerd geweld}, op.
cit., p.~403.

{[}29{]} Charles Tilly, \emph{oorlogsvoering en staatsvorming als
georganiseerde misdaad}, in Peter B. Evans, Dietrich Rueschemeyer en
Theda Skoepol, \emph{de staat weer betrekken} (Cambridge:
\emph{Cambridge University Press}, 1985), p.~169.

{[}30{]} Ibid.

{[}31{]} Lane, \emph{economische gevolgen van georganiseerd geweld}, op.
cit., p.~402.

{[}32{]} David J. Elkins, \emph{voorbij soevereiniteit: territorium en
politieke economie in de eenentwintigste eeuw}. Toronto:
\emph{University of Toronto Press}, 1995, pp.~13--14.

{[}33{]} Ibid., p.~29

{[}34{]} Jim Taylor en Watts Wacker, \emph{The 500-year delta: what
happens after what comes next.} New York: \emph{HarperCollins}, 1997,
p.~40.

{[}35{]} Ibid., p.~67

{[}36{]} Ibid., pp.~41--42

{[}37{]} George Gilder, \emph{Fiber keeps its promise: get ready.
Bandwidth will triple each year for the next 25, creating trillions in
new wealth.} \emph{Forbes ASAP}, 7 april 1997

{[}38{]} Zie Neal Stephenson, Snow Crash. New York: \emph{Bantam Books},
1993

{[}39{]} Keith B. Richburg, `Twee jaar na de landing van de VS in
Somalië is het weer chaotisch', \emph{Washington Post}, 4 december 1994,
p.~A1.

{[}40{]} Geciteerd in Tilly, Coercion, Capital and European States, op.
cit., p.~85.

{[}41{]} Merk op dat Lane's vier fasen van concurrentie en monopolie in
het gebruik van geweld afwijken van de vier fasen in de organisatie van
het economisch leven die wij onderscheiden, namelijk: jagen en
verzamelen, landbouw, industrialisme en het informatietijdperk.

{[}42{]} Lane, `Economische gevolgen van georganiseerd geweld', op.
cit., p.~411.

{[}43{]} Ibid.

{[}44{]} Ibid., p.~403.

{[}45{]} Ibid., p.~404.

{[}46{]} Esther Dyson, \emph{Release 2.1: een ontwerp voor het leven in
het digitale tijdperk}. New York: \emph{Broadway Books}, 1998, p.~131

{[}47{]} Rees Davies, \emph{Frontier arrangements in fragmented
societies: Ireland and Wales}, in Robert Bartlett en Angus MacKay
(red.), \emph{Medieval frontier societies} (Oxford: \emph{Oxford
University Press}, 1992), p.~80.

{[}48{]} Zie Thomas W. Lippman, \emph{Seychelles offers investors safe
haven for \$10 million}, \emph{Washington Post}, 31 december 1995,
p.~A27.

{[}49{]} Zie \emph{ROM of ages}, \emph{Wired}, januari 1996, p.~52.

{[}50{]} Geciteerd in James Adams, \emph{Dawn of the cyber soldiers},
\emph{The Sunday Times} (London), 15 oktober 1995, pp.~3--5.

\bookmarksetup{startatroot}

\chapter{Het overstijgen van
plaatsgebondenheid}\label{het-overstijgen-van-plaatsgebondenheid}

\emph{De opkomst van de cybereconomie}

\begin{quote}
`Controle is waar het werkelijk om draait. Het internet is zo
wijdverspreid dat geen enkele overheid het makkelijk kan beheersen. Door
een naadloze, wereldwijde, niet te reguleren economische zone te creëren
die losstaat van nationale soevereiniteit, stelt het internet het idee
van een natiestaat ter discussie.'\footnote{Perry Barlow, `Thinking
  Locally, Acting Globally', \emph{Time}, 15 januari 1996, p.~57.} -
John Perry Barlow
\end{quote}

De Informatiesnelweg is een van de bekendere metaforen uit de vroege
dagen van het digitale tijdperk geworden. Het is opmerkelijk, niet
alleen door de alomtegenwoordigheid, maar ook door het veel voorkomende
misverstand dat het blootlegt over de cybereconomie. Een snelweg is
immers een industriële versie van een voetpad, een netwerk voor het
fysieke transport van mensen en goederen. De informatiemaatschappij is
niet zoals een snelweg, spoorlijn of pijpleiding. Het vervoert
informatie niet van punt naar punt zoals de Trans-Canada Highway zware
vrachtwagens van Alberta naar New Brunswick brengt. Wat de wereld de
``Informatiesnelweg'' noemt, is niet slechts een transportverbinding.
Het is de bestemming zelf.

De cyberspace overstijgt plaatsgebondenheid. Het is niets minder dan het
onmiddellijke delen van data overal en nergens tegelijk. De opkomende
informatiemaatschappij is gebaseerd op de onderlinge verbindingen die
miljoenen gebruikers van miljoenen computers koppelen. De essentie ligt
in de nieuwe mogelijkheden die uit deze verbindingen ontstaan. Zoals
John Perry Barlow het formuleerde: ``Wat het Net biedt is de belofte van
een nieuwe sociale ruimte, globaal en anti-soeverein, waarin iedereen,
overal, aan de rest van de mensheid zonder angst kan delen wat hij of
zij gelooft. In deze nieuwe media is een voorbode te vinden van de
intellectuele en economische vrijheid die alle autoritaire machten op
aarde zou kunnen ontdoen.''247 :

De cyberspace, net als het denkbeeldige rijk van de goden van Homerus,
is een andere wereld dan de bekende aardse wereld van boerderij en
fabriek. Toch zullen de gevolgen niet denkbeeldig, maar reëel zijn. In
veel grotere mate dan velen nu begrijpen, zal het onmiddellijke delen
van informatie als een oplosmiddel werken dat grote instellingen uiteen
doet vallen. Het zal niet alleen de logica van geweld veranderen, zoals
we al hebben besproken, maar zal de informatie- en transactiekosten, die
bepalen hoe bedrijven zijn georganiseerd en hoe de economie
functioneert, radicaal veranderen. We verwachten dat microprocessing de
economische organisatie van de wereld zal veranderen.

\begin{quote}
\emph{``Het is vandaag, in grotere mate dan ooit in de geschiedenis van
de wereld, mogelijk voor een bedrijf om zich overal te vestigen,
middelen van overal te gebruiken om een product te produceren dat overal
verkocht kan worden.'' -- MILTON FRIEDMAN}
\end{quote}

\section{De tirannie van de plaats}\label{de-tirannie-van-de-plaats}

Dat de eerste benadering van de informatiemaatschappij in het vervagende
industriële tijdperk werd voorgesteld als een enorm publiek
infrastructuurproject, toont aan hoe sterk ons denken nog door
paradigma's van het verleden wordt bepaald. Het is vergelijkbaar met
boeren aan het einde van de achttiende eeuw die een fabriek omschrijven
als ``een boerderij met een dak.'' Toch is de metafoor van de
``snelweg'' onthullender. Ze toont ook hoezeer we nog gevangen zitten in
de tirannie van plaats. Zelfs wanneer technologie ons in staat stelt
locatie te overstijgen, krijgt het instrument van onze bevrijding een
bijnaam die het omschrijft als een route van plek naar plek. Het concept
van lokaliteit zit, net als het scherpe navigatievermogen van zalmen,
nog stevig verankerd in ons bewustzijn.

Gedurende de hele geschiedenis waren economieën tot nu toe gebonden aan
een lokale geografische regio. De meeste mensen die vóór de twintigste
eeuw leefden, brachten hun dagen in feite door als gevangenen onder
huisarrest, zelden verder dan een paar dagen lopen van hun
geboorteplaats. Een reis van welke afstand dan ook kostte generaties.
Alleen zelden leidde een crisis, oorlog, pest, of een ongunstige
klimaatverandering tot brede migratie. Het vereiste iets opmerkelijks en
urgents om de inwoners van een treurig dorp te laten vertrekken; alleen
dat kon hen motiveren hun spullen te verzamelen en elders geluk te
zoeken.

Tot voor kort werden degenen die buiten hun eigen regio naar kansen
keken, vaak beroemd. Denk aan Marco Polo, nog steeds bekend om zijn
reizen over het Euraziatische continent naar het hof van de Grote Khan.
Hij was een uitzondering in zijn tijd. Weinig andere reisverslagen uit
de premoderne periode zijn bewaard gebleven. Onder de bekendere werken
valt \emph{Mandeville's Travels} op, geschreven in het Frans in 1357,
omdat het is geschreven door iemand die waarschijnlijk nooit Europa
verliet. Mandeville beschrijft leuke en vaak fantasierijke details over
het leven wereldwijd, waaronder de suggestie dat veel Ethiopiërs maar
één voet hebben: ``{[}D{]}e voet is zo enorm dat het lichaam volledig in
de schaduw komt te liggen wanneer men rust.''248 : Duidelijk wisten de
weinige tijdgenoten die zijn populaire verhaal lazen waarschijnlijk niet
dat zijn Ethiopische ``Bigfoot'' niet bestond.

Pas met het begin van het moderne tijdperk, aan het einde van de
vijftiende eeuw, ontstonden duurzame contacten tussen de continenten.
Onverschrokken kapiteins als Christopher Columbus en Vasco da Gama, die
de specerijenhandel wilden domineren, waren zo buitengewoon dat zij in
bijna elk geletterd huishouden vijf eeuwen lang werden herinnerd.

Vanaf het begin van de landbouw tot recente generaties werd het leven
gekenmerkt door immobiliteit. Dit is tegenwoordig vrijwel vergeten,
vooral in de Europese kolonies van de ``Nieuwe Wereld,'' waar beweging
vloeiender verloopt en iedereen zich vanuit het perspectief van een
immigrant oriënteert. Een thema in het lager onderwijs in Noord-Amerika
is dat de kolonisten uit Europa op zoek waren naar vrijheid en kansen,
wat waar is. Wat echter zelden wordt verteld, is hoe terughoudend de
meeste mensen waren om de reis te maken, zelfs terwijl ze thuis met
armoede te maken hadden. De weinigen die migreerden, ondergingen naar
huidige maatstaven onvoorstelbare ontberingen om zich daar te vestigen.
Alleen de meest ondernemende mensen of wanhopige armen kwamen. Midden in
de zeventiende eeuw kwamen gevangenen, in Londens beruchte gevangenis
Bridewell, in opstand om ``hun weigering om naar Virginia te gaan'' te
tonen.249 In 1720 waren er rellen in de straten van Parijs om
landlopers, dieven en moordenaars, bestemd voor deportatie naar
Louisiana, te bevrijden.

\subsection{Nauwe horizonten}\label{nauwe-horizonten}

Fysieke communicatie- en transportproblemen, vaak verergerd door
beperkte taalvaardigheden, hielden menselijke activiteiten weinig divers
en lokaal. Nog aan het begin van de twintigste eeuw was het gebruikelijk
dat Chinese dorpen slechts acht kilometer uit elkaar lagen en voor
elkaar onverstaanbare dialecten spraken, zelfs langs de kust. Bijna alle
economieën waren lokaal georganiseerd, wat leidde tot nauwe markten en
gemiste mogelijkheden. De prijzen van productiefactoren bleven hoog door
de geringe concurrentie. Toegang tot gespecialiseerde vaardigheden was
minimaal. Met inkomens zo laag dat ze aan de rand van armoede lagen, en
zonder toegang tot buitenlands kapitaal of efficiënte
verzekeringsmarkten, zaten kleine boeren in veel delen van de wereld
vast in armoede.

We hebben enkele van de moeilijkheden verkend die aan boeren werden
opgelegd door het gesloten dorpsleven. Zelfs nu, terwijl we dit
schrijven, worstelt minstens een miljard mensen, voornamelijk in Azië en
Afrika, om te overleven van minder dan een dollar per dag.

\section{`Alle politiek is locaal'}\label{alle-politiek-is-locaal}

Veel meer dan men zich vaak realiseert, heeft de beperkte mobiliteit van
mensen en hun eigendommen onze kijk op de wereld gevormd. Zelfs degenen
die volhouden dat de aarde inmiddels klein is, denken nog steeds in
verouderde termen over industriële politiek. Dit illustreert een slogan
die in de jaren tachtig populair was onder milieubewuste mensen: `Denk
globaal maar handel lokaal.' De slogan weerspiegelt immers een politiek
die altijd draaide om lokale machtsvoordelen.

Lokale denkwijzen werden altijd bepaald door de megapolitiek van
vroegere samenlevingen. Alle topografische kenmerken die als obstakels
of hefboom voor machtsuitoefening fungeerden, zijn inherent lokaal. Elke
rivier, elke berg en elk eiland heeft zijn eigen betekenis op lokaal
niveau. Ook het klimaat kent lokale variaties: temperatuur, neerslag en
de groeicondities van gewassen veranderen naarmate je een berg beklimt
of oversteekt. Elk micro-organisme beweegt zich binnen een bepaalde
omgeving en niet zomaar overal.

Het verbaast dus niet dat de tirannie van de plaats ons denken over de
organisatie en werking van de samenleving doordringt. De machtsvoordelen
die sommige groepen behaalden met een lokaal geweldsmonopolie, hadden
altijd hun oorsprong ergens in de lokale context en vervaagden aan de
megapolitieke marge, waar de grenzen werden getrokken. Daarom heeft een
wereldregering nooit bestaan.

Hoewel men zelden expliciet benoemt hoe belangrijk locatie is voor het
uitoefenen van macht, merkten enkele voorstanders van een gedwongen
herverdeling al in de jaren dertig dat de invloed van plaats afnam. Zij
herkenden in het moderne vervoer een scheiding van sociale ruimtes
tussen hoge en lage inkomensgroepen. John Dos Passos verwoordde deze
zorg treffend in \emph{The Big Money}: `De zwerver zit aan de rand van
de snelweg, gebroken en hongerig. Boven hem vliegt een transcontinentaal
vliegtuig vol hoogbetaalde leidinggevenden. De hogere klasse heeft zich
in de lucht begeven, de lagere klasse op de weg: er is geen band meer
tussen hen, het zijn twee naties.'\footnote{John Dos Passos, \emph{The
  Big Money} (New York: \emph{Harcourt, Brace \& Co.}, 1936).} Hiermee
geeft hij aan dat de verbeterde vervoersmogelijkheden de effectiviteit
van afpersing verminderden, doordat succesvolle mensen simpelweg meer
mogelijkheden hadden om te kiezen waar ze wilden zijn. De zwerver op de
weg had in ieder geval geen enkel middel om steun af te dwingen van zij
die boven hem vlogen. De tendensen die Dos Passos zestig jaar geleden
opmerkte, zijn sindsdien alleen maar sterker geworden.

\subsection{Massatransport}\label{massatransport}

In 1995 staken dagelijks ongeveer een miljoen mensen ergens in de wereld
een grens over, een opmerkelijke verandering vergeleken met vroeger.
Vóór de twintigste eeuw was reizen zo zeldzaam dat grenzen vooral als
randgebieden werden gezien en nauwelijks een belemmering vormden voor
doorgang. Paspoorten bestonden nog niet. De opkomst van grote zeeboten,
treinen en andere verbeterde vervoermiddelen leidde tot een
spectaculaire toename van het aantal verplaatsingen. Tegelijkertijd
werden deze verplaatsingen steeds strenger gereguleerd door staten, wier
macht was toegenomen dankzij dezelfde verbeterde transport- en
communicatiemiddelen die het reizen voor burgers goedkoper en
eenvoudiger maakte. Films en, vooral, televisie speelden daarnaast een
belangrijke rol in het verbreden van de horizonten en het stimuleren van
reizen en immigratie. Desondanks bleven de fundamentele beginselen van
de sociale en economische organisatie tot nu toe verankerd in de lokale
context.

\begin{quote}
\emph{`...om te voorkomen dat we ons lef verliezen, waarvoor de
geschiedenis zo meedogenloos straft. We moeten de moed hebben om alle
technische extrapolaties tot hun logische conclusie te
volgen.'}\footnote{Clarke, op. cit., p.~29.}\emph{: - ARTHUR C. CLARKE}
\end{quote}

\section{De fout van minimale
verwachtingen}\label{de-fout-van-minimale-verwachtingen}

De geografische greep op de verbeelding blijft nog altijd zo sterk dat
een aantal experts, die in 1995 het internet onder de loep namen,
concludeerden dat het weinig commercieel potentieel bezit en vrijwel
geen betekenis heeft, behalve als medium voor chat en pornografie. De
vele sceptici over het economische belang van de cyberspace vormen de
Colonel Blimps van het informatietijdperk. Hun zelfgenoegzaamheid is
vergelijkbaar met die van de Britse elite in de jaren 1930,
geconfronteerd met de neergang van het rijk. Elites reageren steevast
met ontkenning zodra hun positie in gevaar komt. Dit blijkt uit hun
verwachting dat het internet nooit meer dan een bijzaak zal blijven, een
visie die soms zelfs door autoriteiten wordt gedeeld die beter hadden
moeten weten. We verwezen eerder naar het werk van David Kline en Daniel
Burstein, \emph{Road Warriors: Dreams and Nightmares Along the
Information Highway}. Hun ontkenning van het economische potentieel van
het internet levert extra bewijs dat technische onderlegdheid niet
synoniem staat met het doorgronden van technologische
gevolgen.\footnote{Geciteerd in Kline en Burstein, op. cit., p.~05.}

Zelfs de technisch meest deskundige waarnemers hebben in het verleden
vaak de implicaties van nieuwe technologieën niet begrepen. Een Brits
parlementair comité, bijeengekomen in 1878 om de vooruitzichten van
Thomas Edison's gloeilamp te onderzoeken, beoordeelde Edison's ideeën
als ``goed genoeg voor onze trans-Atlantische vrienden, \ldots{} maar
onwaardig voor de aandacht van praktische of wetenschappelijke
mensen.''253 : Thomas Edison zelf was een man van grote visie, maar hij
dacht dat de fonograaf die hij uitvond voornamelijk door zakenmensen zou
worden gebruikt, om te dicteren. Kort voordat de gebroeders Wright
bewezen dat vliegtuigen konden vliegen, demonstreerde de vooraanstaande
Amerikaanse astronoom Simon Newcomb met gezag waarom vlucht voor
objecten zwaarder dan lucht onmogelijk was. Hij concludeerde: ``De
demonstratie dat geen enkele mogelijke combinatie van bekende stoffen,
bekende machinetypes en bekende krachten kan worden samengebracht in een
praktische machine waarmee mensen lange afstanden door de lucht kunnen
vliegen, lijkt voor de schrijver net zo sluitend te zijn als de
demonstratie van eender welk fysisch gegeven kan zijn.''254 Kort nadat
vliegtuigen gingen vliegen, legde een ander gerenommeerd astronoom,
William H. Pickering, uit aan het publiek waarom commercieel reizen
nooit van de grond zou komen: ``De gemiddelde mens stelt zich vaak
gigantische vliegmachines voor die over de Atlantische Oceaan razen en
talloze passagiers vervoeren, op een manier analoog aan onze moderne
stoomschepen. \ldots{} {[}H{]}et is duidelijk dat met onze huidige
apparaten geen hoop bestaat om qua snelheid te concurreren met onze
locomotieven of onze auto's.''255 : Eerder herinnerden we ons een andere
totaal onjuiste voorspelling over de potentie van een nieuwe
technologie: de voorspelling aan het begin van de twintigste eeuw door
de makers van Mercedes dat er wereldwijd nooit meer dan een miljoen
auto's zouden zijn. Ook hier wisten zij meer over auto's dan bijna wie
dan ook, maar ze konden niet verder van de waarheid zitten wat betreft
de impact van auto's op de samenleving.

Gezien deze traditie van misvattingen is het nauwelijks verrassend dat
veel waarnemers de belangrijkste implicaties van de nieuwe
informatietechnologie pas laat zullen begrijpen, namelijk het feit dat
zij de tirannie van plaats overstijgt. De nieuwe technologie creëert
voor het eerst een oneindig, niet-aards domein voor economische
activiteit. Het biedt de mogelijkheid om de nieuwe grenzen van de
cybereconomie te verkennen, om ``globaal te denken en globaal te
handelen.'' Dit hoofdstuk legt uit waarom.

\section{Voorbij plaatsgebondenheid}\label{voorbij-plaatsgebondenheid}

Het verwerken en gebruiken van informatie vervangt en wijzigt snel
fysieke producten als belangrijkste bron van winst. Dit heeft
ingrijpende gevolgen. Informatietechnologie scheidt het vermogen om
inkomen te genereren van een specifieke geografische locatie. Aangezien
een steeds groter deel van de waarde van producten en diensten wordt
gecreëerd door ideeën en kennis toe te voegen, zal een steeds kleiner
deel van de toegevoegde waarde onder lokale jurisdicties vallen. Ideeën
kunnen overal worden bedacht en wereldwijd met de snelheid van het licht
worden verspreid. Dit betekent onvermijdelijk dat de
informatiemaatschappij drastisch zal verschillen van de economie van het
Fabriekstijdperk.

We geven toe aan de critici dat een opsomming van taken die je in 1996
via het internet had kunnen uitvoeren, misschien banaal lijkt. Er is
immers niets revolutionairs aan het lezen van een artikel over tuinieren
op het Net, of het op afstand kopen van een doos wijn. De potentie van
de cybereconomie kan echter niet uitsluitend worden beoordeeld aan de
hand van het prille begin, net zo min als dat de impact van de auto op
de samenleving in 1900 kon worden ingeschat op basis van wat men toen om
zich heen zag. Wij verwachten dat de cybereconomie zich in meerdere
stadia zal ontwikkelen.

\begin{enumerate}
\def\labelenumi{\arabic{enumi}.}
\item
  De meest primitieve verschijningsvormen van het Informatietijdperk
  \textgreater{} gebruiken het Net eenvoudig als informatiedrager om
  gewone \textgreater{} transacties uit het industriële tijdperk te
  vergemakkelijken. Op \textgreater{} dit punt is het Net niet meer dan
  een exotisch bezorgsysteem voor \textgreater{} catalogi. Virtual
  Vineyards, bijvoorbeeld, een van de eerste \textgreater{}
  cyberhandelaren, verkoopt simpelweg wijn via een pagina op het
  \textgreater{} World Wide Web. Dergelijke transacties ondermijnen de
  oude \textgreater{} instituties nog niet direct. Ze gebruiken
  industriële valuta en \textgreater{} vinden plaats binnen
  identificeerbare jurisdicties. Dit gebruik \textgreater{} van het
  internet heeft weinig megapolitieke impact.
\item
  Een tussenstadium van internethandel zal informatietechnologie
  \textgreater{} gebruiken op manieren die in het industriële tijdperk
  onmogelijk \textgreater{} zouden zijn geweest, zoals bij
  langeafstandsboekhouding of \textgreater{} medische diagnose. Meer
  voorbeelden van deze nieuwe toepassingen \textgreater{} van
  geavanceerde rekenkracht worden hieronder toegelicht. Het
  \textgreater{} tweede stadium van Internethandel zal nog steeds
  functioneren \textgreater{} binnen het oude institutionele kader, met
  gebruik van nationale \textgreater{} valuta en onderworpen aan de
  jurisdictie van natiestaten. De \textgreater{} handelaren die het Net
  gebruiken voor hun verkoop, zullen het nog \textgreater{} niet
  gebruiken om hun winst veilig te stellen, maar alleen om
  \textgreater{} inkomsten te genereren. Deze winsten uit
  internettransacties \textgreater{} zullen nog steeds aan
  belastingheffing onderworpen zijn.
\item
  Een meer gevorderd stadium zal de overgang naar echte cyberhandel
  \textgreater{} markeren. Transacties zullen niet alleen via het Net
  plaatsvinden, \textgreater{} maar ook buiten de jurisdictie van
  natiestaten migreren. \textgreater{} Betalingen zullen plaatsvinden in
  cybervaluta, winst zal worden \textgreater{} geboekt in cyberbanken,
  investeringen zullen worden gedaan via \textgreater{} cybermakelaars,
  veel transacties zullen niet aan belastingheffing \textgreater{}
  onderhevig zijn. In dit stadium zal cyberhandel aanzienlijke
  \textgreater{} megapolitieke gevolgen krijgen, zoals we eerder hebben
  geschetst. \textgreater{} De macht van overheden over traditionele
  delen van de economie zal \textgreater{} worden getransformeerd door
  de nieuwe logica van het Net. \textgreater{} Extraterritoriale
  regelgevende macht zal instorten, jurisdicties \textgreater{} zullen
  verzwakken, de structuur van bedrijven zal veranderen, \textgreater{}
  evenals de aard van werk en arbeid. Deze schets van de stadia van
  \textgreater{} de Informatierevolutie is slechts een summiere weergave
  van wat de \textgreater{} meest ingrijpende economische transformatie
  ooit zou kunnen \textgreater{} worden.
\end{enumerate}

\section{De globalisering van handel}\label{de-globalisering-van-handel}

In het informatietijdperk zullen technologische ontwikkelingen de meeste
traditionele jurisdictievoordelen snel tenietdoen. Tegelijkertijd
ontstaan er nieuwe voordelen. Lagere communicatiekosten hebben de
noodzaak om fysiek aanwezig te zijn voor het doen van zaken al sterk
verminderd. In 1946 kon een investeerder in Londen via een makelaar in
New York een order plaatsen, maar alleen de grootste en meest
overtuigende transacties rechtvaardigden dat: een drie minuten durend
telefoongesprek tussen New York en Londen kostte toen \$650.
Tegenwoordig betaal je daar slechts \$0,91 voor. In een halve eeuw is de
prijs van een intercontinentaal telefoongesprek met meer dan 99 procent
gedaald.

\subsection{Convergente communicatie}\label{convergente-communicatie}

Binnenkort merk je nauwelijks verschil tussen intercontinentale chat en
een lokaal telefoongesprek. Ook vervagen de verschillen tussen je
telefoon, computer en televisie steeds meer; je onderscheidt ze immers
vooral op basis van ergonomie in plaats van functionaliteit. Met je
persoonlijke computer voer je spraakgesprekken via het internet, door
gebruik te maken van de ingebouwde microfoon en luidsprekers, en bekijk
je films. Daarnaast kun je met je televisie communiceren en grote
hoeveelheden data uitwisselen via netwerken die door de
televisie-entertainmentmedia worden aangeboden. Naarmate het onderscheid
tussen de verschillende vormen van communicatie in het industriële
tijdperk verdwijnt en de kosten kelderen, zullen steeds meer diensten
worden aangerekend op basis van de gebruiksduur in plaats van de
bestemming van je berichten. Al met al betaal je straks voor gesprekken
en datatransmissies wereldwijd nauwelijks meer dan wat je in 1985 voor
een lokaal telefoongesprek betaalde.

\subsection{Draadloos Internet}\label{draadloos-internet}

Satellieten in een lage baan rond de aarde en andere vormen van
draadloze technologie zullen gegevens rechtstreeks heen en weer zenden
naar een pieper in je zak, een draagbare computer of een werkstation,
zonder enige aansluiting op een lokaal telefoon- of tv-kabelsysteem.
Kortom, het internet zal draadloos worden. De eerste stappen in die
richting zullen waarschijnlijk aarzelend zijn vanwege de relatief lage
datasnelheid van de vroege draadloze media en de moeilijkheden om zwakke
signalen van gebruiksapparaten te ``horen'', waarvan sommige mobiel
zullen zijn en op batterijen zullen werken. Desalniettemin zullen deze
technische problemen worden aangepakt en opgelost.

\subsection{Zakendoen zonder grenzen}\label{zakendoen-zonder-grenzen}

De voortdurende toename van de rekencapaciteit leidt tot geavanceerdere
compressietechnieken, wat de doorstroming van data versnelt. Door
bestaande algoritmen voor encryptie met publieke en private sleutels op
grote schaal toe te passen, kunnen aanbieders, zoals satellietsystemen,
de facturering naadloos in hun dienst integreren en zo kosten besparen.
Tegelijkertijd krijgen leveranciers de mogelijkheid om rekeningen die op
pc's zijn geladen, direct te belasten, net zoals \emph{France Telecom}
de `smartcards' in de telefoonhokjes in Parijs debiteert.

\subsection{De telefoon wordt een
bank}\label{de-telefoon-wordt-een-bank}

Het verschil is dat je in de nabije toekomst credits op je account kunt
verdienen met allerlei transacties en je telefoon overal mee naartoe
kunt nemen. Je pc zal het filiaal van je bank en mondiale geldmakelaar
zijn, en het equivalent van de kiosk in Parijs waar je je anonieme
telefoonkaart koopt. En net als de 150 smartcardtelefoons die voor
dieven nutteloos zijn als ze met een koevoet worden opengebroken, kan je
computer alleen worden geplunderd door iemand die in staat is
geavanceerde computercode te breken of te manipuleren. Dat sluit veel
tuig, dat wel met een koevoet om kan gaan, uit. Met de juiste encryptie
kan niets in je computer worden ontcijferd of misbruikt.

Tegen de eeuwwisseling zul je bijna overal ten noorden van Antarctica
zaken kunnen doen. Overal waar vaste of digitale mobiele telefoons
beschikbaar zijn, overal waar interactieve kabeltelevisiesystemen worden
gebruikt, overal waar een satelliet zich boven je bevindt of andere
draadloze transmissiesystemen aanwezig zijn. Je zult over grenzen heen,
wanneer je wilt, kunnen spreken, gegevens verzenden en reizen via
virtual reality. Telefoon­nummers die de locatie van de beller aangeven
via netnummers zullen waarschijnlijk worden vervangen door universele
toegangsnummers, die overal op de planeet de persoon zullen bereiken met
wie je wilt communiceren.

\subsection{Chinees begrijpen}\label{chinees-begrijpen}

Je zult niet alleen kunnen praten en faxen. Na verloop van tijd zul je
het jarenlange leertraject kunnen overslaan, waardoor je in het Chinees
kunt converseren met een voorman in een fabriek in Shanghai. Het maakt
dan nog weinig uit dat je zijn taal of dialect niet spreekt. Hoewel hij
in het Chinees communiceert, zullen zijn woorden de `plaatsgebondenheid
overstijgen'. Jij hoort zijn woorden in je eigen taal en hij hoort het
gesprek in het Chinees. Spoedig zal je vermogen tot onmiddellijke
vertaling de concurrentiekracht verhogen in regio's waar taal- en
uitdrukkingsbarrières voorheen een struikelblok vormden. Op dat moment
maakt het nauwelijks of helemaal niet uit dat de Chinese regering
misschien bezwaar heeft tegen het gesprek.

\subsection{Gepersonaliseerde media}\label{gepersonaliseerde-media}

Naarmate de wereld steeds dichter bij elkaar komt, zul je meer
mogelijkheden dan ooit krijgen om je eigen positie vorm te geven. Ook de
informatie die je via de media binnen zult krijgen, zul je zelf kiezen.
De traditionele massamedia zullen plaats maken voor gepersonaliseerde
media. Ben je een fanatieke schaker of een fervent kattenliefhebber? Dan
kun je jouw avondnieuws zo inrichten dat het uitsluitend nieuws bevat
over jouw favoriete onderwerpen. je hoeft voor de nieuwsvoorziening niet
langer afhankelijk te zijn van Dan Rather of van de \emph{BBC}. Je
selecteert zelf het nieuws dat volledig is afgestemd op jouw wensen.

\subsection{Van massa naar op maat gemaakte
productie}\label{van-massa-naar-op-maat-gemaakte-productie}

Als het komkommertijd is, kun je een virtuele catalogus raadplegen op
het \emph{World Wide Web}. Als je een broek ziet die je bijna bevalt,
kun je bij je bestelling de breedte van de pijpen aanpassen. De broek
wordt dan op maat gesneden en door robots in Maleisië nauwkeurig
afgestemd op je lichaam, op basis van foto's die je via je computer
scant en over het internet verstuurt.

\subsection{Cyberbroking}\label{cyberbroking}

Je kunt cybergeld gebruiken om te investeren en voor diensten en
producten te betalen. Als je in een rechtsgebied woont zoals de
Verenigde Staten, waar investeringsmogelijkheden streng gereguleerd
zijn, kies je er bewust voor je activiteiten onder te brengen in een
omgeving die volop vrijheid biedt op vlak van investeringsmogelijkheden.
Of je nu in Cleveland of in Belo Horizonte woont, je kunt je
investeringszaken regelen in Bermuda, op de Kaaimaneilanden, in Rio de
Janeiro of in Buenos Aires. Waar je ook bent, digitale middelen zullen
steeds meer gebruikt worden naarmate de cybereconomie floreert. Je zult
slimme systemen in kunnen zetten om je investeringen te selecteren en
zult cyberaccountants en -boekhouders in kunnen schakelen om de
voortgang van je portefeuille in realtime te volgen.

\subsection{Virtuele cultuur}\label{virtuele-cultuur}

Wanneer je even niet bezig bent met winst- en verliescijfers, kun je een
virtueel bezoek brengen aan het Louvre. Voordat je op pad gaat, moet je
mogelijk een royalty betalen ter waarde van een derde van een cent aan
Bill Gates of aan iemand met een vergelijkbare vooruitziende blik die de
rechten op virtuele realiteit voor museumbezoeken heeft verworven.
Terwijl je je afvraagt of de Mona Lisa ooit problemen met haar tanden
had, downloadt je computer ondertussen S.~I.~Hsiung's vertaling van
\emph{The Romance of the Western Chamber}. Op het moment dat jij dat
wilt, leest je persoonlijke communicatiesysteem de tekst voor, als een
bard uit weleer. Dankzij multitaskingprogramma's kun je meerdere
functies gelijktijdig uitvoeren.

\subsection{Shoppen voor rechtsgebieden op het
Net}\label{shoppen-voor-rechtsgebieden-op-het-net}

Als je geïnspireerd bent door de klassiekers, kun je een virtueel
bedrijf oprichten om dramatische producties van beroemde literatuur te
verkopen om op driedimensionale retinale displays weer te geven. In
plaats van geprojecteerd in de lucht, worden de beelden direct op het
netvlies van kijkers geprojecteerd met laag-energetische lasers die
vijftigduizend keer per seconde fluctueren. Deze technologie, al in
ontwikkeling bij MicroVision in Seattle, Washington, zal veel
slechtzienden weer kunnen laten zien.

Voordat je het project onderneemt, kun je jouw digitale assistent
instrueren om de huidige contractaanbiedingen voor de bescherming van
productiefaciliteiten in Maleisië, China, Peru, Brazilië en Tsjechië te
inventariseren. Zodra je een locatie kiest, kun je je bedrijf binnen één
uur laten oprichten op de Bahama's, via de St.~George's Trust Company.
Je instructies plaatsen alle liquide activa van het bedrijf in een
cyberaccount bij een cyberbank die gelijktijdig is gevestigd in
Newfoundland, de Kaaimaneilanden, Uruguay, Argentinië en Liechtenstein.
Als een van deze rechtsgebieden probeert om de operationele bevoegdheid
in te trekken of de activa van rekeninghouders in beslag te nemen,
worden de activa automatisch overgebracht naar een andere jurisdictie
met de snelheid van het licht.

\section{Kwalitatieve vooruitgang}\label{kwalitatieve-vooruitgang}

Veel transacties die je binnenkort in de cyberspace kunt uitvoeren,
waren in het industriële tijdperk ondenkbaar, en niet alleen doordat ze
een taalbarrière overschrijden. Het inzetten van digitale assistenten om
onvertaalde artikelen uit Hongaarse wetenschappelijke tijdschriften te
verzamelen, onderscheidt zich qua kwaliteit van een gesprek met een
bibliothecaris. Het deelnemen aan een Oxford-tutorial op een afstand van
achtduizend kilometer is niet te vergelijken met het volgen van
diezelfde tutorial terwijl je binnen tien kilometer van Carfax slaapt.
En roulette spelen in het \emph{Hotel de Paris} in Monte Carlo biedt een
totaal andere ervaring wanneer je dit via \emph{virtual reality} vanuit
een feest in Punte del Este, Uruguay beleeft.

\subsection{Een cyberbezoek aan de
cyberdokter}\label{een-cyberbezoek-aan-de-cyberdokter}

Binnen korte tijd, misschien wel sneller dan veel deskundigen
verwachten, zal de economische activiteit naar de cybereconomie
migreren. Hierbij combineer je technologieën op vernieuwende wijze om de
beperkingen van locatiegebondenheid en de achterhaalde instituties van
de industriële economie te doorbreken. Binnenkort zul je wanneer je
buikpijn krijgt een digitale dokter raadplegen, een digitale expert met
encyclopedisch inzicht in symptomen, kwalen en tegengif. Dit systeem
doorzoekt, in versleutelde vorm, je medische geschiedenis en vraagt of
je pijn ervaart na of vóór de maaltijd, of de pijn scherp of dof,
constant of sporadisch is. De digitale dokter stelt alle vragen die een
arts zou stellen. Hij kan daarbij vaststellen dat je te veel of juist te
weinig wijn drinkt en je eventueel doorverwijzen naar een
cyberspecialist. Heb je een operatie nodig, dan verricht een
cyberschirurg in Bermuda de ingreep op afstand met behulp van
gespecialiseerde apparatuur die micro-incisies maakt.

\subsection{Leven-en-dood
informatieverwerking}\label{leven-en-dood-informatieverwerking}

Dit klinkt misschien als sciencefiction, maar veel componenten van
cyberchirurgie zijn al aanwezig. Andere zullen operationeel zijn tegen
de tijd dat je dit boek leest. General Electric heeft een nieuwe
magnetic-resonance-treatment-machine (MRT) geïntroduceerd in vijftien
ziekenhuizen wereldwijd. De machine zal een onderzoeks- en
ontwikkelingsfase van drie jaar doorlopen, maar daarna zal ze
waarschijnlijk snel verspreid raken en de norm worden voor veel soorten
chirurgie. Dit is één voorbeeld, maar een goed voorbeeld, van hoe
technologie de samenleving verandert.

De meesten van ons zijn bekend met magnetic-resonance-imaging-apparaten
(MRI), waarbij magnetische resonantietechnieken worden gebruikt om
artsen beelden van zachte weefsels te verschaffen voor diagnostische
doeleinden. Ze leveren betere beelden van zachte weefsels dan
röntgenstraling of echografie en zijn een essentieel onderdeel geworden
van moderne diagnostische technieken, met name bij kanker. Ze hebben
echter momenteel twee belangrijke beperkingen: de buis biedt geen vrije
toegang tot de patiënt en de machines hebben een beperkt vermogen.

\subsection{Cyberchirurgie}\label{cyberchirurgie}

\emph{General Electric} heeft de magnetische resonantiemachines zo
aangepast dat ze zowel voor diagnostiek als voor behandelingen ingezet
kunnen worden. Ze hebben de kracht van de machines met een factor vijf
verhoogd en de buis in twee helften verdeeld, waardoor de patiënt niet
langer volledig wordt omsloten, maar tussen twee donutvormige
compartimenten komt te liggen. In plaats van eerst een beeld vast te
leggen waarop later de operatie wordt gebaseerd, ziet de chirurg direct
wat hij doet tijdens de ingreep. Het systeem koppelt de MRT aan
microchirurgische technieken die minder invasief zijn. De chirurg hoeft
geen grote sneden met een scalpel te maken, maar zet kleine incisies met
sonderingsinstrumenten, waarbij hij in real‑time observeert wat deze
onthullen. Hij voert de operatie uit op basis van het beeld, in plaats
van er met eigen ogen in te kijken. Bovendien kan hij de instrumenten in
principe op afstand bedienen. Zo kunnen tumoren met uiterste precisie
worden vernietigd, bijvoorbeeld met behulp van laserapparatuur of via
cryogene warmte‑ en vriesbehandelingen.

Hiermee worden operaties mogelijk die tot nu toe onmogelijk leken,
vooral in de neurochirurgie, waar tumoren zich vaak zeer dicht bij
vitale hersengebieden bevinden. Ook is het met deze technologie mogelijk
om operaties meermaals uit te voeren, terwijl het trauma van de
traditionele ingreep niet herhaald kan worden zonder onaanvaardbare
schade. Sommige onderzoekers zijn van mening dat het mes voor chirurgie
aan zacht weefsel tegen 2010 een verouderd relikwie zal zijn. Als dat
standhoudt, verminderen zowel de angst als de naschokken die bij
traditionele operaties horen. Uiteraard is dit uitstekend nieuws voor de
patiënt. Terwijl operaties tegenwoordig uren duren en gevolgd worden
door dagen of weken ziekenhuisopname, kan de ingreep in slechts een half
uur worden voltooid en is een opname mogelijk overbodig. In feite is het
zelfs mogelijk dat de chirurg en de patiënt nooit in dezelfde ruimte
aanwezig zijn. Maar wat betekent dit voor ziekenhuizen en chirurgen?

\subsection{Minder microchirurgen die meer operaties
uitvoeren}\label{minder-microchirurgen-die-meer-operaties-uitvoeren}

Er zal een revolutie plaatsvinden in de chirurgie. Een derde van de
jonge chirurgen lukt het tijdens de opleiding niet om de vaardigheden
voor microscopische chirurgie te verwerven. Een derde kan het net
uitvoeren, en een derde wordt uitstekend. Vergelijkbare verhoudingen
worden gevonden in omscholingen voor oudere chirurgen. Minder chirurgen
zullen in staat zijn om meer operaties in kortere tijd uit te voeren.
Verzekeraars en mensen die een operatie ondergaan, zullen waarschijnlijk
resultaten per chirurg willen zien, die sterk uiteenlopen. Patiënten
zullen naar chirurgen willen gaan die de beste resultaten leveren,
vooral als hun aandoeningen levensbedreigend zijn. In sommige gevallen
kunnen de beste chirurgen operaties op afstand uitvoeren. Ze kunnen de
hele operatie uitvoeren vanuit een andere jurisdictie waar belastingen
lager zijn en rechtbanken exorbitante schadeclaims niet erkennen.

\subsection{Digitale juristen}\label{digitale-juristen}

Voordat een ervaren chirurg akkoord gaat met een operatie, schakelt hij
of zij waarschijnlijk een digitale jurist in om onmiddellijk een
contract op te stellen. Dit contract specificeert en beperkt de
aansprakelijkheid op basis van de grootte en kenmerken van de tumor,
zoals die zichtbaar zijn in de beelden van de magnetische
resonantiemachine. Digitale juristen zijn informatieverwerkende systemen
die met kunstmatige intelligentie, waaronder neurale netwerken,
contracten automatisch aanpassen zodat ze voldoen aan de transnationale
wetgeving. Deelnemers aan belangrijke en waardevolle transacties zoeken
niet alleen geschikte zakenpartners, maar kiezen ook een passend
vestigingsadres voor hun transacties.

\subsection{Spoedconsultatie}\label{spoedconsultatie}

Om het voorbeeld van cybersurgery voort te zetten: de technologie van
het Informatietijdperk zal een premie leggen op de hoogste vaardigheden
in de chirurgie, zoals in bijna elk ander vakgebied. Patiënten waren al
bereid om zo'n premie te betalen sinds messen werden uitgevonden. Maar
beperkingen in informatie en de moeilijkheid om chirurgen in
noodsituaties in een bepaalde regio te vinden, maakten de markt voor
chirurgie behoorlijk imperfect. In het Informatietijdperk zal deze
minder imperfect zijn. Een patiënt die binnen vierentwintig uur, of
misschien zelfs binnen vijfenveertig minuten, een operatie nodig heeft,
zou digitale assistenten kunnen inzetten om de tien beste chirurgen
wereldwijd te vinden die beschikbaar zijn voor een operatie op afstand,
om hun slagingspercentages in vergelijkbare gevallen te beoordelen, en
om offertes voor het specifieke geval op te vragen bij hun digitale
vertegenwoordigers. Dit alles kan in een oogwenk worden uitgevoerd. Als
gevolg hiervan zal de meest gewilde 10 procent van de chirurgen een veel
groter wereldwijd marktaandeel hebben in de chirurgie. De MRT-machine,
plus microchirurgische technieken, zal de premie voor hun werk verhogen.
Chirurgen met minder vaardigheden zullen zich richten op de overgebleven
lokale markten.

Dit leven-en-doodvoorbeeld illustreert enkele revolutionaire gevolgen
van de bevrijding van economieën uit de tirannie van plaats. Sommigen
zullen misschien aanvoeren dat de MRT-machine van General Electric niet
bedoeld was voor gebruik op afstand. Misschien, maar dat mist het punt.
Deze of soortgelijke apparatuur zal dat binnenkort wel zijn. Wanneer
operaties beter kunnen worden uitgevoerd door chirurgen die naar een
scherm kijken dan direct naar de patiënt, zal het minder uitmaken waar
de chirurg en zijn scherm zich bevinden. Een toenemend aantal diensten
zal worden heringericht om te profiteren van het feit dat
informatietechnologie mensen overal ter wereld in staat stelt te
handelen, zelfs in zulke delicate zaken als chirurgie. Bij activiteiten
die minder precieze apparatuur vereisen en lagere faalkansen kennen, zal
de cybereconomie nog sneller floreren.

\begin{quote}
\emph{``Het financiële beleid van de verzorgingsstaat vereist dat er
geen manier is voor vermogenden om zichzelf te beschermen.'' -- ALAN
GREENSPAN}
\end{quote}

\section{De devaluatie van dwang}\label{de-devaluatie-van-dwang}

In bijna elk competitief domein, inclusief het merendeel van de
wereldwijde investeringen ter waarde van biljoenen dollars, zal de
verhuizing van transacties naar de cyberspace worden aangedreven door
een bijna onstuitbare kracht, de drang om roofbelasting te vermijden,
waaronder de inflatiebelasting die iedereen die zijn vermogen in een
nationale munt aanhoudt, treft.

\subsection{Ontsnappen aan het
beschermingsmaffia}\label{ontsnappen-aan-het-beschermingsmaffia}

Je hoeft niet lang na te denken over de megapolitiek van het
Informatietijdperk om te beseffen dat roofbelastingen en inflatie, zoals
die door de rijkste industriële landen als recht aan hun burgers worden
opgelegd, volstrekt onconcurrerend zullen zijn in de nieuwe wereld van
de cyberspace. Kort na de eeuwwisseling zal iedereen die
inkomstenbelasting betaalt tegen de huidige tarieven dat voor 50 procent
vrijwillig doen. Zoals Frederic C. Lane opmerkte, laat de geschiedenis
zien dat ``aan de grenzen en op de hoge zeeën, waar niemand een duurzaam
monopolie op geweld had, handelaars heffingen vermeden die zo hoog waren
dat bescherming goedkoper op andere manieren kon worden verkregen.''256
:

De cybereconomie biedt precies zo'n alternatief. Geen enkele overheid
zal deze kunnen monopoliseren. En de informatietechnologieën die daarbij
horen, zullen goedkopere en effectievere bescherming van financiële
activa bieden dan de meeste overheden ooit moesten leveren.

\subsection{De zwarte magie van samengestelde
rente}\label{de-zwarte-magie-van-samengestelde-rente}

Onthoud: als je elk jaar €5.000 betaalt gedurende veertig jaar, slijt
dat je nettovermogen met €2,2 miljoen, uitgaande van een jaarlijks
rendement van slechts 10 procent op je kapitaal. Bij een rendement van
20 procent loopt het samengestelde verlies op tot ongeveer €44 miljoen.
Voor mensen die veel verdienen in landen met hoge belastingen zijn de
totale verliezen door roofzuchtige belastingheffing over een heel leven
ronduit adembenemend. De meesten verliezen uiteindelijk meer dan ze ooit
bezaten.

Dit klinkt onmogelijk, maar de wiskunde staat als een paal boven water.
Je kunt dit zelf gemakkelijk nagaan met een simpele rekenmachine. De top
1 procent van de belastingbetalers in de Verenigde Staten betaalt
gemiddeld meer dan \$125.000 aan federale inkomstenbelasting per jaar.
Voor een fractie daarvan, namelijk \$45.000 per jaar, kom je in
aanmerking voor een particuliere belastingregeling in Zwitserland, waar
je profiteert van orde en veiligheid, gewaarborgd door wat wordt
beschouwd als het eerlijkste politie- en rechtssysteem ter wereld.
Vanuit dit perspectief kun je de extra \$80.000 aan jaarlijkse
inkomstenbelasting boven dat royale niveau als een ware tribuut of zelfs
als roof bestempelen. Een betaling van \$45.000 is best substantieel
voor de instandhouding van orde en veiligheid, vooral omdat
politiebescherming in principe een collectief goed behoort te zijn. In
theorie kunnen publieke goederen aan extra gebruikers geleverd worden
zonder bijkomende kosten. De Zwitsers zijn dan ook tevreden dat je een
overeengekomen vaste belasting van \$45.000 (oftewel 50.000 Zwitserse
frank) per jaar betaalt, want per aangemelde miljonair boeken zij
daarmee jaarlijks \$45.000 winst.

Als je de Zwitserse regeling vergelijkt, lijdt een belegger die
gemiddeld 20 procent rendement behaalt en federale inkomstenbelasting
betaalt volgens de Amerikaanse tarieven, over een hele levensloop een
verlies van ongeveer \$705 miljoen. Houd er wel rekening mee dat dit
uitgaat van een jaarlijkse belasting van \$45.000. Vergelijk dat eens
met een belastingparadijs als Bermuda, waar vrijwel geen
inkomstenbelasting geldt. Daar loopt het belastingverschil maar liefst
op tot zo'n \$1,1 miljard ten opzichte van de Amerikaanse tarieven.

Je zou kunnen stellen dat een jaarlijks rendement van 20 procent
buitengewoon hoog is -- daar heb je zeker een punt. Maar dankzij de
indrukwekkende groei in Azië gedurende de afgelopen decennia hebben veel
beleggers wereldwijd dat rendement, of zelfs meer, weten te behalen.
Sinds 1950 ligt het samengestelde rendement op vastgoedbeleggingen in
Hongkong op meer dan 20 procent per jaar. Zelfs in economieën die niet
bekendstaan om hun sterke groei, boden zich vaak makkelijke kansen op
hoge winsten. In de afgelopen drie decennia had je met deposito's in
Amerikaanse dollars bij Paraguayaanse banken een reëel gemiddeld
rendement van meer dan 30 procent per jaar kunnen boeken. Hoge
investeringsrendementen zijn in sommige plekken makkelijker te
realiseren dan in andere, maar ervaren beleggers kunnen in goede jaren
absoluut winsten van 20 procent of meer behalen, ook al evenaren zij
niet altijd de prestaties van George Soros of Warren Buffett.

Het ligt natuurlijk voor de hand dat hoe hoger het rendement op je
kapitaal, hoe groter de opportuniteitskosten zijn die ontstaan door
buitensporige inkomsten- en vermogenswinstbelastingen. De vaststelling
dat het verlies gigantisch is, zelfs groter dan het totale vermogen dat
je wellicht ooit had kunnen opbouwen, vereist echter niet dat je
uitzonderlijk hoge rendementen behaalt. Sommige Amerikaanse
beleggingsfondsen boeken al langer dan een halve eeuw een gemiddeld
jaarlijks rendement van meer dan 10 procent. Als dit voor jou het best
haalbare resultaat zou zijn en je behoort tot de top 1 procent van de
Amerikaanse inkomens, dan betekent dat een daling van je nettovermogen
met meer dan \$33 miljoen, enkel door de inkomstenbelasting die je
betaalt over je inkomen boven de \$45.000 per jaar. Vergeleken met een
jurisdictie zonder inkomstenbelasting bedraagt dat verlies zelfs \$55
miljoen.

\subsection{\$55 in plaats van \$55
miljoen}\label{in-plaats-van-55-miljoen}

Als de aannames van economen over winstmaximalisatie correct zijn, en
wij geloven dat dit doorgaans zo is, dan kun je met zekerheid
voorspellen dat de meeste mensen \$55 miljoen zouden proberen veilig te
stellen als ze dat zouden kunnen. Dat is onze voorspelling. Wanneer de
zwarte magie van samengestelde rente duidelijker wordt voor succesvolle
mensen in landen met een hoge belastingdruk, zullen zij serieus beginnen
te shoppen tussen jurisdicties, net zoals ze nu auto's kopen of
verzekeringspolissen vergelijken. Als je twijfelt, vraag dan
willekeurige mensen op straat in New York of Toronto of ze voor \$55
miljoen naar Bermuda zouden verhuizen. De vraag stellen, is hem
beantwoorden. Het dilemma doet denken aan dat van Mark Twain, die zich
afvroeg of hij liever een nacht zou doorbrengen met een naakte Lillian
Russell of met generaal Grant in zijn stijlvolle uniform. Hij hoefde er
niet lang over na te denken. Inwoners van volwassen verzorgingsstaten,
vooral in de Verenigde Staten, zullen trager reageren, maar alleen omdat
ze zich nog niet bewust zijn van de keuze waar ze voor staan. Er komt
een moment waarop ze dat wel zullen zijn. Wie streeft naar een beter
leven, zal het nut zien van het minimaliseren van de verliezen die
roofbelastingen veroorzaken. Je hoeft slechts je transacties in de
cyberspace onder te brengen. Dit zal natuurlijk in veel jurisdicties
illegaal zijn. Maar oude wetten kunnen zelden weerstand bieden tegen
nieuwe technologie. In de jaren 1980 was het in de Verenigde Staten
illegaal om een faxbericht te verzenden. Het Amerikaanse postkantoor
beschouwde faxen als eersteklas post, waarover het een eeuwenoud
monopolie claimde. Er werd een decreet uitgevaardigd dat alle
faxberichten moesten worden doorgestuurd naar het dichtstbijzijnde
postkantoor voor bezorging met de gewone post. Miljarden faxen later is
het onduidelijk of iemand ooit aan die wet heeft voldaan. Als dat al zo
was, dan was dat van korte duur. De voordelen die de opkomende
cybereconomie biedt, zijn nog overtuigender dan het omzeilen van het
postkantoor met een fax.

Een brede adoptie van public-key/private-key-encryptietechnologieën zal
binnenkort veel economische activiteiten overal ter wereld mogelijk
maken. Zoals James Bennet, technologie-redacteur van \emph{Strategic
Investment}, schreef:

\begin{quote}
\emph{``De handhaving van wetten, en met name belastingwetgeving, is
sterk afhankelijk geworden van toezicht op communicatie en transacties.
Zodra de volgende logische stappen zijn gezet, en offshorebanken
communicatie aanbieden via sterke RSA-versleutelde e-mail met
accountnummers afgeleid van public-key-systemen, zullen financiële
transacties vrijwel onmogelijk te monitoren zijn, zowel in de bank als
in communicatie. Zelfs als de belastingautoriteiten een infiltrant in de
offshorebank zouden plaatsen of de bankgegevens zouden stelen, zouden
zij de rekeninghouders niet kunnen identificeren.''257:}
\end{quote}

In een mate die nooit eerder mogelijk was, zullen individuen kunnen
bepalen waar zij hun economische activiteiten onderbrengen en hoeveel
inkomstenbelasting zij bereid zijn te betalen. Veel transacties in het
Informatietijdperk hoeven helemaal niet binnen een territoriale
soevereiniteit te worden ondergebracht. Degenen die dat wel doen, zullen
steeds vaker terechtkomen in plaatsen zoals Bermuda, de Kaaimaneilanden,
Uruguay of vergelijkbare jurisdicties die geen inkomstenbelasting of
andere kostbare transactielasten opleggen.

\subsection{Van monopolie naar
concurrentie}\label{van-monopolie-naar-concurrentie}

Overheden zijn gewend geraakt aan het opleggen van
``beschermingsdiensten'' die, in de woorden van Frederic C. Lane, ``van
slechte kwaliteit en belachelijk duur'' zijn.258 : Deze gewoonte om veel
meer te vragen dan de werkelijke waarde van de overheidsdiensten
ontwikkelde zich in de loop van het eeuwenlang durende monopolie.
Iedereen die in staat leek om ze te betalen, werd meedogenloos belast,
juist omdat overheden een monopolie of bijna-monopolie op dwang hadden.
Deze monopolietraditie zal op fundamentele wijze botsen met de nieuwe
megapolitieke mogelijkheden die cyberhandel biedt.

Encryptie zal het gemakkelijk maken om transacties in de cyberspace te
beschermen. De kosten van een effectief encryptieprogramma, zoals PGP,
zijn lager dan de commissie die een full-service broker rekent voor de
aankoop van honderd aandelen. Toch maakt het vrijwel elke transactie
voor nog vele jaren onzichtbaar en onaantastbaar voor overheden en
dieven. In het Informatietijdperk maakt de nieuwe technologie het
mogelijk om digitale bezittingen bijna kosteloos te beschermen. Voor
\$55 in plaats van \$55 miljoen zullen deelnemers aan de cybereconomie
een betere daadwerkelijke bescherming van hun activa genieten dan
tijdens het industriële tijdperk of op enig ander moment in de
geschiedenis. Gemakkelijk te gebruiken encryptie-algoritmen en de
mogelijkheid om transacties tussen verschillende landen te verplaatsen,
zullen effectieve bescherming bieden tegen de grootste bron van roof: de
natiestaten zelf.

Dat wil niet zeggen dat territoriale overheden volledig zullen worden
uitgeschakeld. Ze zullen nog steeds persoonlijke kwetsbaarheden kunnen
exploiteren om hoofdelijke belastingen te innen, of mogelijk zelfs rijke
individuen gijzelen voor losgeld. Ze zullen ook de inning van
consumptiebelastingen kunnen afdwingen. Toch zal bescherming, de
belangrijkste dienst van overheden, bijna competitief worden. Een
kleiner deel van de kosten die productieve mensen betalen voor
bescherming zal door politieke autoriteiten kunnen worden ingenomen en
herverdeeld. Technologische innovaties zullen een groot en groeiend deel
van de rijkdom in de wereld buiten het bereik van overheden plaatsen.
Dit zal de risico's van handel verminderen en, in de woorden van de
historicus Janet Abu-Lughod, ``het aandeel van alle kosten'' scherp
verlagen dat anders ``aan transitrechten, heffingen of eenvoudige
afpersing'' zou zijn uitgegeven.259

Het kwam zelden in de geschiedenis voor dat overheden werkelijk door
concurrentie werden beperkt. In de weinige gevallen dat iets dergelijks
voorkwam, waren overheden zwak en waren technologieën vergelijkbaar
tussen de rechtsgebieden. Zoals Lane suggereerde, is de belangrijkste
factor die de winstgevendheid onder zulke omstandigheden beïnvloedt het
verschil in beschermingskosten die door verschillende ondernemers worden
betaald. De middeleeuwse handelaar die twintig tolrechten moest betalen
om zijn goederen op de markt te brengen, kon niet concurreren met een
handelaar die slechts vier tolrechten moest betalen om dezelfde goederen
bij de klant te leveren. Vergelijkbare omstandigheden zullen terugkeren
in het Informatietijdperk. Winstgevendheid zal opnieuw niet zozeer
bepaald worden door technologische voordelen, maar door het succesvol
minimaliseren van de kosten die voor bescherming betaald moeten worden.

Deze nieuwe economische dynamiek staat lijnrecht tegenover het verlangen
van overheden uit het industriële tijdperk om monopolieprijzen voor hun
beschermingsdiensten op te leggen. Maar, of ze het nou leuk vinden of
niet, het oude systeem zal niet levensvatbaar zijn in de nieuwe
competitieve omgeving van het Informatietijdperk. Elke overheid die
steevast haar burgers zware belastingen oplegt, die ze bij concurrenten
niet hoeven betalen, zal er enkel voor zorgen dat winst en rijkdom
ergens anders naartoe migreren. Zo zal het onvermogen van volwassen
verzorgingsstaten om de belastingen langdurig te verlagen, op den duur
zichzelf corrigeren. Overheden die overmatig belasten, zorgen er
simpelweg voor dat wonen binnen hun macht een financieel onhoudbare
aangelegenheid wordt.

\begin{quote}
\emph{``\ldots{} zoals de vorst door zijn voorrecht geld kan scheppen
uit eender welke materie en vorm, en de standaard kan bepalen, kan hij
eveneens de samenstelling en het ontwerp van het geld wijzigen, de
waarde opdrijven of verlagen, of het geheel afschaffen en ongeldig
verklaren.''260 : -- UIT EEN ENGELSE RECHTSBESLISSING, 1604}
\end{quote}

\section{De dood van seigniorage}\label{de-dood-van-seigniorage}

Overheden zullen niet alleen hun macht verliezen om verscheidene vormen
van inkomen en kapitaal te belasten, ze zijn ook gedoemd om de macht
over geld te verliezen. In het verleden gingen megapolitieke transities
vaak gepaard met veranderingen in de aard van geld.

\begin{itemize}
\item
  De introductie van muntgeld droeg bij aan het op gang brengen van de
  \textgreater{} vijfhonderdjarige expansiecyclus van de economie in de
  oudheid, \textgreater{} die eindigde met de geboorte van Christus en
  de laagste \textgreater{} rentestanden vóór het moderne tijdperk.
\item
  Het begin van de Donkere Middeleeuwen viel samen met de vrijwel
  \textgreater{} volledige sluiting van de muntslagende instituten.
  Hoewel Romeins \textgreater{} muntgeld bleef circuleren, kromp de
  geldhoeveelheid samen met de \textgreater{} handel in een zichzelf
  versterkende neerwaartse spiraal.
\item
  De feodale revolutie viel samen met een herintroductie van geld, het
  \textgreater{} slaan van muntgeld, wisselbrieven en andere middelen om
  \textgreater{} commerciële transacties af te wikkelen. Vooral een
  stijging in de \textgreater{} Europese zilverproductie uit nieuwe
  mijnen in Rammelsberg, \textgreater{} Duitsland, maakte een grotere
  circulatie van muntgeld mogelijk, \textgreater{} wat de handel
  vergemakkelijkte.
\item
  De grootste revolutie in geld vóór het Informatietijdperk kwam met
  \textgreater{} de opkomst van het industrialisme. De vroegmoderne
  staat \textgreater{} consolideerde haar macht in de Buskruitrevolutie.
  Naarmate haar \textgreater{} controle toenam, eiste de staat ook macht
  over geld op, en ging \textgreater{} zwaar leunen op de kenmerkende
  technologie van het industrialisme: \textgreater{} de drukpers. Het
  eerste instrument voor massaproductie, de \textgreater{} drukpers,
  werd in de moderne periode breed door overheden gebruikt
  \textgreater{} om grote hoeveelheden papiergeld te produceren.
\end{itemize}

Papiergeld is een typisch industrieel product. Vóór de drukpers was het
onpraktisch om kwitanties of certificaten te dupliceren die tot
papiergeld konden worden omgezet. Monniken in de scriptoria zouden hun
tijd zeker niet zinvol besteed hebben met het natekenen van biljetten
van vijftig pond. Papiergeld droeg bovendien aanzienlijk bij aan de
macht van de staat, niet alleen door winst te genereren via devaluatie
van de munt, maar ook door de staat invloed te geven op wie rijkdom kon
opbouwen. Zoals Abu-Lughod stelde: "Toen papiergeld, gesteund door de
staat, de erkende valuta werd, werd het moeilijk om kapitaal te vergaren
in oppositie tot of onafhankelijk van het staatsapparaat.''261 :

\section{Cybercash}\label{cybercash}

De opkomst van het Informatietijdperk impliceert ook een nieuwe
revolutie in het karakter van geld. Naarmate cyberhandel op gang komt,
zal dit onvermijdelijk leiden tot cybergeld. Deze nieuwe vorm van geld
zal de kansen herschikken en de macht van natiestaten om te bepalen wie
een Soeverein Individu wordt, verminderen. Een cruciaal onderdeel van
deze verandering zal voortkomen uit het bevrijdende effect van
informatietechnologie tegen onteigening van vermogen door inflatie.
Binnenkort zal je voor bijna elke transactie via het Net of World Wide
Web betalen op het moment dat deze plaatsvindt, met cybergeld.

Deze nieuwe digitale vorm van geld zal ongetwijfeld een centrale rol
spelen in de cyberhandel. Het zal bestaan uit versleutelde reeksen
priemgetallen met honderden cijfers. Dit geld, uniek, anoniem en
verifieerbaar, zal de grootste transacties mogelijk maken, terwijkl het
ook deelbaar zal zijn in de kleinste fractie van waarde. Het zal
verhandelbaar zijn met een druk op de knop in een grenzeloze
groothandelsmarkt ter waarde van biljoenen dollars.

\subsection{Handelen zonder dollars}\label{handelen-zonder-dollars}

Het nieuwe cybergeld zal zich onvermijdelijk onttrekken aan nationale
beperkingen. Zodra mensen grensoverschrijdend handelen in een virtuele
wereld, weigeren ze de ouderwetse praktijk waarmee regeringen de waarde
van hun geld kunstmatig doen dalen door inflatie. Waarom zouden ze dat
nog accepteren? De controle over geld verschuift van de machtscentra
naar de wereldwijde marktplaats. Iedereen met toegang tot de cyberspace,
zowel een individu als een onderneming, kan eenvoudig overstappen naar
elke andere valuta wanneer de waarde van de ene dreigt te kelderen. In
tegenstelling tot nu hoef je transacties niet langer met een wettig
betaalmiddel te verrichten. Sterker nog, wanneer transacties wereldwijd
plaatsvinden, rekent ten minste één partij in elke transactie in een
valuta die voor die partij niet als wettig betaalmiddel geldt.

\subsection{Verminderde nadelen van
ruilhandel}\label{verminderde-nadelen-van-ruilhandel}

In de cybereconomie kun je in elk gewenst medium handelen. Zoals de
inmiddels overleden Nobelprijswinnende econoom \emph{E. A. Hayek}
betoogde, bestaat er `geen duidelijk onderscheid tussen geld en
niet-geld.' Hij schreef: ``Hoewel we er doorgaans van uitgaan dat er een
scherpe scheidslijn is tussen wat geld is en wat niet, en de wet vaak
probeert dit onderscheid te handhaven, bestaat er, wat betreft de
causale effecten van monetaire gebeurtenissen, geen dergelijk duidelijk
verschil. Wat we zien, is eerder een continuüm waarin objecten met
verschillende mate van liquiditeit, of met waarden die onafhankelijk van
elkaar kunnen schommelen, in elkaar overlopen in de mate waarin ze als
geld functioneren.'' \footnote{Friedrich A. von Hayek, de
  denationalisatie van geld (London: \emph{Institute of Economic
  Affairs}, 1976), p.~47.} Digitale valuta op wereldwijde
computernetwerken maken elk object op Hayeks continuüm van liquiditeit
extra liquide, met uitzondering van overheidsgeld. Een direct gevolg
hiervan is dat ruilhandel praktischer wordt. Steeds meer goederen en
diensten komen via specifieke ruilaanbiedingen beschikbaar in ruil voor
andere goederen en diensten. Deze potentiële transacties verschijnen
wereldwijd op het Net, waardoor hun liquiditeit enorm toeneemt.

Een van de grootste nadelen van ruilhandel is dat het lastig is om
precies de juiste mensen bij elkaar te brengen, waarbij iemand met een
specifieke behoefte wordt gekoppeld aan iemand die exact kan leveren wat
gevraagd is. Vroeger stond ruilhandel voor de moeilijke uitdaging om
twee partijen te vinden die met elkaar wilden handelen op de lokale
markt. Contant geld oversteeg de beperkingen van ruilhandel en blijft
dat in veel transacties doen. Tegelijkertijd verkleinen de enorme groei
van computerkracht en de globalisering van de handel via cyberspace ook
de nadelen van directe ruilhandel. De kans dat je iemand vindt met
wensen die precies bij die van jou aansluiten, neemt enorm toe wanneer
je je niet hoeft te beperken tot de lokale omgeving, maar wereldwijd
kunt zoeken.

\subsection{Niet vatbaar voor
vervalsing}\label{niet-vatbaar-voor-vervalsing}

Hoewel papiergeld ongetwijfeld over zal blijven als circulerend
ruilmiddel voor de armen en computeranalfabeten, zal geld voor
transacties met hoge waarde geprivatiseerd worden. Cybergeld zal niet
langer uitsluitend worden uitgedrukt in nationale eenheden zoals het
papiergeld van het industriële tijdperk. Het zal waarschijnlijk worden
gedefinieerd in grammen of ounces goud, en zal net zo deelbaar zijn als
goud zelf. Of het kan worden gedefinieerd in termen van andere echt
waardevaste middelen. Zelfs op plekken waar verschillende
prijsmaatstaven worden gebruikt, of bepaalde transacties in nationale
valuta geprijsd blijven, zal cybergeld de consument veel beter dienen
dan genationaliseerd geld ooit deed. Het voortdurende bijstellen van
prijzen, wat noodzakelijk is bij het gebruik van verschillende
ruilmiddelen, zal door de snelle groei van rekenkracht vrijwel volledig
probleemloos verlopen. Elke transactie zal een versleutelde reeksen
priemgetallen met honderden cijfers overdragen. In tegenstelling tot de
papieren geldbewijzen die door overheden tijdens de goudstandaard werden
uitgegeven, en die naar believen konden worden gedupliceerd, zullen de
nieuwe digitale goudstandaard of haar ruil­handelequivalenten vrijwel
onmogelijk te vervalsen zijn, om de fundamentele wiskundige reden dat
het praktisch onmogelijk is om het product van zulke grote priemgetallen
te ontleden. Alle bewijzen zullen verifieerbaar uniek zijn.

De namen van traditionele valuta zoals de ``pond'' en de ``peso''
weerspiegelen het feit dat ze ooit ontstonden als gewichtseenheden van
specifieke hoeveelheden edelmetaal. Het pond sterling was ooit
letterlijk een pond sterlingzilver. Papiergeld in het Westen begon als
bewijs voor de opslag van hoeveelheden edelmetaal in kluizen of andere
opslagplaatsen. Overheden die deze bewijzen uitgaven, ontdekten al snel
dat ze er veel meer van konden drukken dan mensen geneigd waren om
daadwerkelijk weer voor hun edelmetaal in te wisselen. Dit was
eenvoudig. Geen enkele houder van een goud- of zilvercertificaat kon, op
basis van dat bewijs, inschatten hoe groot de werkelijke voorraad
edelmetaal was. Afgezien van de serienummers zagen alle bewijzen er
hetzelfde uit, een feit dat aansprak bij zowel valsmunters als politici
en bankiers die wilden profiteren van inflatie.

Cybergeld zal op deze manier vrijwel onmogelijk te vervalsen zijn, zowel
officieel als officieus. De verifieerbaarheid van de digitale bewijzen
sluit dit klassieke middel tot onteigening van vermogen via inflatie
uit. Het nieuwe digitale geld van het Informatietijdperk zal de controle
over het ruilmiddel teruggeven aan de bezitters van vermogen, die het
willen behouden, in plaats van aan natiestaten die het willen afromen.

\subsection{De transactiekosten van `vrij'
geld}\label{de-transactiekosten-van-vrij-geld}

Het gebruik van dit nieuwe cybergeld zal je in belangrijke mate
bevrijden van de macht van de staat. Eerder hebben we de sombere staat
van dienst van natiestaten in de afgelopen halve eeuw besproken op vlak
van het behoud van de waarde van hun valuta. Geen enkele munt heeft
sinds de Tweede Wereldoorlog een kleiner verlies door inflatie geleden
dan de Duitse mark. Toch verdween zelfs daarvan 71 procent van de waarde
tussen 1 januari 1949 en eind juni 1995. De wereldreservemunt in die
periode, de Amerikaanse dollar, verloor 84 procent van zijn waarde.263
Dit is een maat voor de rijkdom die overheden hebben onteigend via de
uitbuiting van hun territoriale monopolie op een wettig betaalmiddel.

Er is overigens geen enkele intrinsieke noodzaak voor een valuta om in
waarde te dalen of voor een jaarlijkse toename van de nominale kosten
voor levensonderhoud. Integendeel, de technische uitdaging om de
koopkracht van spaargeld te behouden is minimaal. Dit valt duidelijk op
als je de langetermijnkoopkracht van goud bekijkt.

Tussen 1 januari 1949 en eind juni 1995, terwijl de beste nationale
valuta bijna driekwart van hun waarde verloren, steeg de koopkracht van
goud juist. Zoals professor Roy W. Jastrom heeft aangetoond in \emph{The
Golden Constant,} heeft goud, afgezien van kleine fluctuaties, zijn
koopkracht behouden sinds de vroegste betrouwbare prijsgegevens, die in
Engeland teruggaan tot 1560.

Nationale valuta die aan goud waren gekoppeld, behielden hun koopkracht
ook zolang er geen militaire noodsituaties waren. De waarde van het
Britse pond sterling steeg zelfs in de relatief vreedzame negentiende
eeuw, ondanks dat het slechts zwak aan goud gekoppeld was. De nieuwe
megapolitieke omstandigheden van het Informatietijdperk maken voor het
eerst juist een sterke koppeling mogelijk, geen zwakke koppeling zoals
de goudstandaard, door de enorm verbeterde informatie- en rekenmiddelen
in handen van consumenten.

\begin{quote}
\emph{``Het gevaar van het snelle verlies van hun hele bedrijf als ze
niet aan de verwachtingen voldoen (en de zekerheid dat een
overheidsorganisatie de kans zou misbruiken om grondstofprijzen te
manipuleren!) zou een veel sterkere waarborg tegen een staatsmonopolie
bieden dan welke ook maar bedacht zou kunnen worden.''264\\
--- FRIEDRICH A. VON HAYEK}
\end{quote}

\subsection{Privatisering van geld}\label{privatisering-van-geld}

Friedrich von Hayek betoogde in 1976 dat het gebruik van concurrerende,
private valuta inflatie zou uitbannen.265 Zonder de verplichting om een
inflatoire munt als wettig betaalmiddel binnen een jurisdictie te
accepteren, zo stelde Hayek, zou marktconcurrentie de private uitgevers
van valuta dwingen de waarde van hun ruilmiddelen te behouden. Elke
uitgever die er niet in zou slagen om de waarde te handhaven, zou snel
zijn klanten verliezen. De ontwikkeling van versleuteld cybergeld zal
Hayeks logica duidelijk nieuw leven inblazen.

De theorie van ``free banking'', zoals die wordt genoemd, is niet
slechts een hypothetische academische gedachte. Concurrerende private
valuta circuleerden in Schotland vanaf het begin van de achttiende eeuw
tot 1844. In die periode had Schotland geen centrale bank. Er waren
weinig regels of beperkingen om tot de bankensector toe te treden.
Private banken namen deposito's aan en gaven hun eigen valuta uit,
gedekt door goud. Zoals professor Lawrence White heeft gedocumenteerd,
werkte dit systeem goed. Het was stabieler, met minder inflatie dan het
zwaarder gereguleerde en gepolitiseerde bankensysteem dat in diezelfde
periode in Engeland werd gehanteerd.266 Michael Prowse van de
\emph{Financial Times} vatte de Schotse ervaring met free banking als
volgt samen: ``Er was weinig fraude. Er was geen bewijs van de
overmatige uitgifte van biljetten. Banken hielden doorgaans noch
buitensporige, noch ontoereikende reserves aan. Bankruns waren zeldzaam
en niet besmettelijk. De vrije banken genoten het respect van de burgers
en boden een solide basis voor economische groei die gedurende het
grootste deel van die periode groter was dan die in Engeland.''267

Wat in de achttiende en negentiende eeuw al goed werkte, zal nog beter
functioneren met de technologie van de eenentwintigste eeuw. Je zult
binnenkort in staat zijn om te handelen in digitaal geld van een private
onderneming, uitgegeven op een manier die vergelijkbaar is met hoe
American Express travelcheques uitgeeft als bewijs van contant geld. Een
instelling met meer aanzien dan welke regering dan ook, zoals een
toonaangevend mijnbedrijf of de Swiss Bank Corporation, zou versleutelde
bewijzen kunnen creëren voor hoeveelheden goud of zelfs voor unieke
goudstaven, geïdentificeerd door moleculaire kenmerken en mogelijk zelfs
voorzien van hologrammen. Deze bewijzen zullen vervolgens als geld
circuleren, vrijwel zonder de kans om vervalst of opgeblazen te worden.

Het nieuwe digitale goud zal veel van de praktische problemen overwinnen
die het directe gebruik van goud als geld in het verleden hebben
belemmerd. Het zal niet langer onhandig, omslachtig of gevaarlijk zijn
om in grote bedragen goud te handelen. Digitale bewijzen zullen niet te
zwaar zijn om te dragen; hun enige fysieke bestaan zal in feite complexe
patronen van computercode zijn. Ook zal het eenvoudig zijn om digitale
bewijzen te delen in eenheden die klein genoeg zijn om zelfs
microbetalingen mogelijk te maken. Een stukje fysiek goud dat klein
genoeg is om een kauwgompje te betalen, zou snel kwijt raken of verward
worden met een stukje dat groot genoeg is om twee kauwgompjes te
betalen. Maar voor een computer zal het even makkelijk zijn om deze
denominaties van digitaal geld te onderscheiden alsof ze zo groot waren
als een eekhoorn of een neushoorn.

Het vermogen van digitaal geld om microbetalingen mogelijk te maken, zal
de opkomst faciliteren van nieuwe soorten bedrijven die tot nu toe niet
konden bestaan, gespecialiseerd in het organiseren van de distributie
van laagwaardige informatie. De aanbieders van deze informatie zullen nu
worden beloond via directe betalingsregelingen die de vroegere
ontmoedigende transactiekosten overwinnen. Wanneer de kosten van
facturering hoger zijn dan de waarde van een transactie, zal deze
waarschijnlijk niet plaatsvinden. Het gebruik van cybergeld maakt zeer
goedkope onmiddelijke afrekening mogelijk, waarbij rekeningen direct bij
gebruik worden afgewikkeld. We noemden eerder al het voorbeeld waarbij
je misschien een royalty van een derde van een cent betaalt aan Bill
Gates, of aan wie de virtuele rechten dan ook toebehoren om het Louvre
te bezoeken. Vermenigvuldig dit met duizend toepassingen. Virtual
reality zal bijna onbeperkte licentiemogelijkheden creëren die
desondanks slechts microroyalty's zullen opleveren. Op een dag zul je
bijvoorbeeld het derde duel van de World Series van 1969 opnieuw kunnen
beleven en microroyalty's betalen aan de spelers wier beelden worden
gebruikt om jouw virtuele realiteit echt te laten lijken.

\section{Inflatie uitroeien}\label{inflatie-uitroeien}

Ondanks al deze mogelijkheden zal de meest ingrijpende consequentie van
het nieuwe digitale geld ongetwijfeld het einde van inflatie en de
afbouw van schuld in het financiële systeem zijn. De economische
implicaties zijn diepgaand. De opkomst van inflatie in de twintigste
eeuw, zoals we betoogden in \emph{Blood in the Streets} en \emph{The
Great Reckoning}, hing nauw samen met de machtsverhoudingen in de
wereld. De toenemende opbrengsten van geweld forceerden veel hogere
militaire uitgaven, wat op zijn beurt steeds agressievere pogingen
vereiste om vermogen te onteigenen. Overheden ontdekten dat ze in feite
een jaarlijkse vermogensbelasting konden opleggen aan iedereen die
tegoeden in hun nationale valuta aanhield. Deze jaarlijkse
vermogensbelasting voor eigenaars van valuta kon ook worden gezien als
een transactiekost voor het gemak om vermogen te bewaren in een handige
vorm, geleverd door de uitgevers.268

Het idee dat inflatie een transactiekost is die voor het gebruiksgemak
van valuta wordt aangerekend, is ongewoon, maar denk er goed over na.
Tijdens het Industriële Tijdperk raakten we zo gewend om valuta te zien
als een dienst waarvoor men niet direct betaalt, dat het makkelijk werd
vergeten dat de uitgevers van dollars, peso's, ponden en francs,
overheden dus, wel degelijk betaling eisten, en wel fors, namelijk via
inflatie. Het tarief van deze inflatoire transactiekost varieerde in de
afgelopen halve eeuw van 2,7 procent per jaar als laagste percentage
voor de Duitse mark tot percentages die gevaarlijk dicht bij 100 procent
lagen. Zo verloor de Argentijnse munt tussen 1960 en 1991, toen
president Menem de currency-boardhervorming doorvoerde, zeventien nullen
door opeenvolgende golven van inflatie. Als al het vermogen van de
wereld in 1960 in Argentijnse peso's zou zijn omgezet en begraven, dan
zou het in 1991 niet meer de moeite waard zijn geweest om het op te
graven.

Het voorbeeld van Argentinië is een voorloper voor het volgende
millennium. Geld zal niet meer geïnflateerd worden omdat andere
natiestaten er niet langer mee weg zullen komen, net zoals dat
Argentinië dat niet meer kan. Inflatie had in het industriële tijdperk
nog een aantrekkingskracht omdat prijzen en lonen niet gemakkelijk
neerwaarts bewogen. Lichte inflatie verhoogde de productie door de reële
lonen te verlagen, terwijl prijzen kwetsbaar waren voor een ingevoerde
kredietkrimp uit het buitenland. Privaatgeld zal niet inflatoir zijn
door de druk van concurrentie.

Het einde van inflatie zal de verborgen winsten wegnemen die inflatie
eerder verschafte aan de monopolistische uitgevers van geld. Als al deze
verborgen winsten verdwijnen, zal een nieuwe methode van betaling nodig
zijn om de uitgevers rechtstreeks te compenseren. Het gebruik van het
nieuwe monetaire systeem zal daarom waarschijnlijk een expliciete
transactiekost inhouden, mogelijk een vergoeding van ongeveer 1 procent
per jaar. Dit is een kleine prijs in vergelijking met de jaarlijkse
inflatieboete van 2,7 tot 99 procent die door natiestaten werd opgelegd.
Zeker omdat de kans groot is dat de algemene prijzen in de toekomst
zullen dalen naarmate monopolies afbrokkelen en concurrentie wereldwijd
toeneemt.

\subsection{Afnemende hefboomwerking}\label{afnemende-hefboomwerking}

De opkomst van digitaal geld zal niet alleen inflatie definitief
bestrijden, maar ook de hefboomwerking in het wereldwijde bankensysteem
verminderen. Dat mensen wereldwijd regelgeving kunnen omzeilen en hun
vermogen rechtstreeks via internet kunnen verplaatsen, is een totaal
nieuwe uitkomst van de geglobaliseerde markten. Het zal buiten de macht
van welke regering dan ook liggen om dit te reguleren. Wanneer
regeringen valuta niet langer kunnen devalueren door geld bij te drukken
of het bedriegen van spaarders via de uitbreiding van krediet in een
gevangen bankensysteem, verliezen ze een groot deel van hun indirecte
vermogen om middelen op te eisen.

\subsection{Hogere rentetarieven}\label{hogere-rentetarieven}

Dit zal een duidelijk dilemma creëren voor de meeste westerse
regeringen. Ze zullen te maken krijgen met scherpe dalingen in hun
inkomsten uit belastingen en de bijna volledige eliminatie van
hefboomwerking in het monetaire systeem. Tegelijkertijd behouden ze de
ongedekte schulden en opgeblazen verwachtingen voor sociale uitgaven die
ze uit het industriële tijdperk hebben geërfd. Het te verwachten
resultaat is een intense fiscale crisis met veel onaangename sociale
bijeffecten, die in latere hoofdstukken worden besproken. De economische
consequentie van deze overgangsperiode zal waarschijnlijk een eenmalige
piek in de reële rentetarieven inhouden. Schuldenaren zullen onder druk
komen te staan naarmate langlopende verplichtingen uit het oude systeem
worden afgelost en gunstige kredieten opdrogen.

\subsection{Veranderd door
concurrentie}\label{veranderd-door-concurrentie}

Wanneer overheden worden geconfronteerd met serieuze concurrentie voor
hun monopolies op geld, zullen ze waarschijnlijk cybervaluta proberen te
onderwaarderen door krediet te versmallen en spaarders betere
rendementen op contanten in de nationale valuta te geven. Sommige
overheden kunnen zelfs proberen goud opnieuw als geld te introduceren om
de concurrentie met private valuta aan te gaan. Ze zullen mogelijk
denken dat ze hogere seignioragewinsten zullen behalen met een soepele
negentiende eeuwse goudstandaard dan wanneer ze hun nationale valuta
volledig zouden laten verdrukken door commercieel cybergeld. Niet alle
overheden zullen echter op dezelfde manier reageren. Overheden in
regio's waar computergebruik en Net-deelname laag zijn, kunnen in de
vroege fase van de cybereconomie kiezen voor ouderwetse hyperinflatie.
Dit stelt deze regeringen niet in staat om de contante tegoeden van
rijken af te pakken, maar het zal middelen afromen van degenen met
weinig rijkdom of toegang tot de cybereconomie. Dergelijke overheden
kunnen desalniettemin internationaal lenen in cybergeld.

Andere overheden kunnen zich aanpassen aan de kansen die de
informatiemaatschappij biedt en lokale transacties in cybergeld
faciliteren. De rechtsgebieden die als eerste de geldigheid van digitale
handtekeningen erkennen en lokale gerechtelijke handhaving bij
niet-betaling van cyberschulden bieden, zullen profiteren van een
onevenredige toename in langetermijnkapitaalverstrekking. Uiteraard zal
er geen cybergeld beschikbaar zijn voor langlopende kredieten in
gebieden waar lokale rechtbanken straffen oplegden of schuldenaren
zonder consequenties toestonden in gebreke te blijven.

\subsection{Rentekloof}\label{rentekloof}

De combinatie van kredietcrisissen, competitieve aanpassingen door
nationale muntautoriteiten en vroege overgangsbelemmeringen bij het
verstrekken van kredieten in cybergeld zorgt in de beginfase van de
informatie-economie voor een rentekloof. Naar verwachting biedt
cybergeld lagere rentetarieven dan de nationale valuta en brengt het
waarschijnlijk ook expliciete transactiekosten met zich mee. Een
verbeterde bescherming tegen verliezen door buitensporige belastingen en
inflatie zorgt ervoor dat deze nadelen worden gecompenseerd. Aangezien
cybergeld vermoedelijk aan goud gekoppeld wordt, profiteert het tevens
van een stijgende goudprijs. De prijs van goud zal naar verwachting
sterk stijgen ten opzichte van andere grondstoffen, ongeacht welke
alternatieve overheidsmaatregel de doorslag krijgt. Waarom? De reële
prijs van goud stijgt vrijwel altijd tijdens een deflatie, want deflatie
duidt immers op een tekort aan liquiditeit. Goud blijft uiteindelijk de
ultieme vorm van liquiditeit.

\subsection{De deflatie van het industriële
tijdperk}\label{de-deflatie-van-het-industriuxeble-tijdperk}

Hogere reële rentes dwingen overal tot het liquideren van dure,
onproductieve activiteiten en drukken de consumptie tijdelijk omlaag. We
hebben de logica van de kredietcyclus en de daaropvolgende afwikkeling
ervan behandeld in \emph{Blood in the Streets} en \emph{The Great
Reckoning}, dus herhalen we die argumenten hier niet. Het volstaat te
stellen dat de deflatoire periode enige tijd kan aanhouden, waarbij de
dure industriële economieën in Noord-Amerika en West-Europa harder onder
de nadelige gevolgen zullen lijden dan de goedkope economieën in Azië en
Latijns-Amerika.

\subsection{Lagere rentes op lange
termijn}\label{lagere-rentes-op-lange-termijn}

Hoewel de opkomst van de cybereconomie aanvankelijk tot hogere
rentetarieven leidt, keert het effect op de lange termijn precies om. De
rendementen na belasting zullen voor spaarders fors stijgen zodra
middelen ontsnappen aan de greep van overheden. Dramatische
verbeteringen in de efficiëntie van het gebruik van hulpbronnen en de
bevrijding van kapitaal om wereldwijd de hoogste rendementen te behalen,
compenseren naar verwachting de aanvankelijk verloren productie tijdens
de transitiecrisis snel.

\subsection{Investeerderscontrole over
kapitaal}\label{investeerderscontrole-over-kapitaal}

Conventionele denkers die ons betoog op dit punt beoordelen, zouden
concluderen dat het wegvallen van inkomensherverdeling in de leidende
natiestaten de wereld voor een economische ineenstorting zou zorgen.
Geloof dat niet. We ontkennen niet dat een overgangscrisis
waarschijnlijk is, maar de opvatting dat de staat de economie verbetert
door massale herverdeling van middelen, is een anachronisme, een
geloofspunt vergelijkbaar met de wijdverspreide bijgeloven aan het einde
van de Middeleeuwen dat vasten en geseling nuttig zouden zijn voor een
gemeenschap. Het mag niet vergeten worden dat overheden op grote schaal
middelen verspillen. Middelen verspillen maakt je arm. Een dramatische
verbetering in de efficiëntie van het gebruik van middelen zal ontstaan
wanneer inkomsten die historisch door overheden werden beheerd, in
handen komen van personen met werkelijk talent.

Tientallen miljarden, en uiteindelijk honderden miljarden dollars,
zullen worden beheerd door honderden duizenden, en later miljoenen
Soevereine Individuen. Deze nieuwe beheerders van 's werelds rijkdom
zullen waarschijnlijk veel capabeler blijken dan politici wanneer het
aankomt op het benutten van middelen en het inzetten van investeringen.
Voor het eerst in de geschiedenis zullen megapolitieke omstandigheden de
meest bekwame investeerders en ondernemers, in plaats van specialisten
in geweld, ultieme controle over kapitaal geven. Het is niet onredelijk
om te verwachten dat de rendementen op deze verspreide, marktgedreven
investeringen het dubbele of drievoudige kunnen zijn van de magere
rendementen van de door politiek gedreven budgetallocaties uit het
tijdperk van de natiestaat. In de laatste decennia van de twintigste
eeuw waren voorbeelden van overheidsinvesteringen met substantieel
negatieve opbrengsten in geen enkel land een uitzondering. We citeerden
officiële Russische statistieken in de herziene versie van \emph{The
Great Reckoning} van november 1992, waaruit bleek dat de hele Russische
economie ``slechts \$30 miljard waard was, minder dan een derde van de
waarde van haar gebruikte ruwe grondstoffen. Dit impliceert dat de
output van Rusland meer dan verdrievoudigen zou als de binnenlandse
productie- en diensteneconomie volledig zou worden stilgelegd. In plaats
van waarde toe te voegen, trekken ze waarde af.''269

Het voorbeeld van Rusland na de val van het communisme is extreem, maar
er is ruim bewijs dat het verminderen van staatscontrole over middelen
de economische efficiëntie verbetert. Groei­cijfers vermeld door
\emph{The Economist} suggereren dat economische vrijheid sterk
correleert met economische groei, waarbij de snelst groeiende landen ook
de meest vrije zijn. De cybereconomie van het Informatietijdperk zal
vrijer zijn dan enig ander commercieel domein in de geschiedenis. Het is
daarom redelijk om te verwachten dat de cybereconomie snel de
belangrijkste nieuwe economie van het nieuwe millennium zal worden. Het
succes zal nieuwe deelnemers van over de hele wereld aantrekken, net
zoals het brede gebruik van faxmachines telecopiëren steeds
aantrekkelijker maakte voor niet-gebruikers. Maar nog belangrijker,
vrijheid van roofzuchtig geweld zal de cybereconomie in staat stellen te
groeien met veel hogere samengestelde groeipercentages dan conventionele
economieën die door natiestaten worden gedomineerd.

Dit is mogelijk het belangrijkste aspect bij het inschatten van de
economische impact van het waarschijnlijke falen van het
overheidsmonopolie op belastingheffing en inflatie. Afgezien van
overgangsproblemen, die decennia kunnen duren, zijn de
langetermijnvooruitzichten voor de wereldeconomie zeer gunstig. Wanneer
omstandigheden het voor mensen mogelijk maken om kosten voor bescherming
te verlagen en tribuut aan degenen die georganiseerd geweld beheersen te
minimaliseren, groeit de economie meestal dramatisch. Zoals Lane zei:
``Ik wil suggereren dat de meest gewichtige factor in de meeste
groeiperioden, als één factor het belangrijkst was, een vermindering is
van het aandeel van middelen dat aan oorlog en politie wordt
besteed.''270

Er kunnen grote efficiëntiewinsten ontstaan door een vermindering van
middelen die aan roof en het leven van de buit daarvan worden besteed.
Als de prijsstelling van bescherming competitief zou worden gemaakt,
waarbij lokale monopolies concurreren op prijs en kwaliteit, zouden
potentieel enorme efficiëntiewinsten mogelijk zijn. Het te verwachten
resultaat zouden veel lagere belastingtarieven zijn en minder verlies
van middelen en inspanningen in politieke activiteiten, die niet langer
hun eerdere enorme opbrengsten zouden opleveren.

Zouden kiezers bereid zijn om de politieke cadeautjes op te geven waar
ze aan gewend zijn geraakt? Dat is een kwestie die elders uitgebreid
wordt besproken. Maar een eenvoudig antwoord is dat we mogelijk geen
keuze hebben. Niemand demonstreert nu tegen regen of droogte, hoe
economisch schadelijk of onaangenaam die ook zijn. Niemand, hoe
crimineel ook ingesteld, houdt een arme als gijzelaar vast voor een
enorm bedrag aan losgeld, op straffe van de dood. Als het voor politici
onmogelijk wordt om middelen te verkrijgen voor herverdeling, kan het
publiek rationeel reageren en politiek naast zich neer leggen, net zoals
dat goedbedoelende mensen stopten met het organiseren van boetemarsen
toen de Middeleeuwen eindigden.

\bookmarksetup{startatroot}

\chapter{\texorpdfstring{\textbf{HET EINDE VAN EGALITAIRE
ECONOMIE}}{HET EINDE VAN EGALITAIRE ECONOMIE}}\label{het-einde-van-egalitaire-economie}

\emph{De revolutie van het vermogen om geld te verdienen in een wereld
zonder banen}

`God laat zich niet bespotten: want wat een mens zaait, zal hij ook
oogsten.' - Galaten 6:7

Grote veranderingen in de dominante productie- of defensiemethoden
veranderen de structuur van de samenleving en de verdeling van rijkdom
en macht. Het Informatietijdperk betekent meer dan alleen het groeiende
gebruik van krachtige computers: het betekent een revolutie in
levensstijlen, instituties en de verdeling van middelen. Met de sterke
terugval van verborgen geweld als middel om hulpbronnen te beheersen,
zal de verdeling van vermogen zich herordenen, vrij van de gedwongen
inmenging van overheden zoals die in de twintigste eeuw gebruikelijk
was. Locatie zal veel minder betekenen in de Informatiesamenleving,
waardoor organisaties die gebonden zijn aan geografie, zoals politici,
vakbonden, gereguleerde beroepen, lobbyisten en regeringen zelf, aan
belang verliezen. De gunsten en handelsbeperkingen die overheden
mogelijk maakten, verliezen hun waarde, waardoor minder middelen worden
verspild aan lobbies of het verzet daartegen.

Degenen die dwang en lokale voordelen gebruikten om het inkomen te
herverdelen, verliezen veel macht. Dit verandert de controle over
middelen: vermogen zal bij de mensen blijven die het creëren, in plaats
van te worden onteigend door de natiestaat. Steeds meer vermogen vloeit
naar de meest bekwame ondernemers en durfkapitalisten wereldwijd.
Globalisering en de kenmerken van de informatie-economie vergroten de
inkomsten van de meest getalenteerden in elk veld. Omdat de marginale
waarde van uitzonderlijke prestaties enorm is, zal de verdeling van het
vermogen om geld te verdienen wereldwijd steeds meer lijken op die in
prestatieberoepen zoals sport en opera.

\section{\texorpdfstring{\textbf{Een omvang voorbij de wet van Pareto}
De wet van Pareto stelt dat 80 procent van het voordeel afhankelijk is
van of toekomt aan 20 procent van de betrokkenen. Dit is ongeveer waar,
maar opvallender is dat 1 procent van de bevolking van de Verenigde
Staten 28,7 procent van de inkomstenbelasting betaalt. Dit suggereert
dat samenlevingen die het informatietijdperk ingaan een nog schevere
verdeling van inkomens en bekwaamheid zullen ervaren dan Vilfredo Pareto
aan het einde van de negentiende eeuw waarnam. Mensen zijn gewend aan
substantiële vermogensongelijkheid. In 1828 bezat 4 procent van de New
Yorkers naar schatting 62 procent van alle rijkdom in de stad. In 1845
bezat de top 4 procent ongeveer 81 procent van al het bedrijfs- en
niet-bedrijfsvermogen in New York City. Breder gezien bezat de top 10
procent van de bevolking in 1860 ongeveer 40 procent van het vermogen in
de hele VS. Tegen 1890 suggereren gegevens dat de rijkste 12 procent
ongeveer 86 procent van Amerika's vermogen
bezat.}{Een omvang voorbij de wet van Pareto De wet van Pareto stelt dat 80 procent van het voordeel afhankelijk is van of toekomt aan 20 procent van de betrokkenen. Dit is ongeveer waar, maar opvallender is dat 1 procent van de bevolking van de Verenigde Staten 28,7 procent van de inkomstenbelasting betaalt. Dit suggereert dat samenlevingen die het informatietijdperk ingaan een nog schevere verdeling van inkomens en bekwaamheid zullen ervaren dan Vilfredo Pareto aan het einde van de negentiende eeuw waarnam. Mensen zijn gewend aan substantiële vermogensongelijkheid. In 1828 bezat 4 procent van de New Yorkers naar schatting 62 procent van alle rijkdom in de stad. In 1845 bezat de top 4 procent ongeveer 81 procent van al het bedrijfs- en niet-bedrijfsvermogen in New York City. Breder gezien bezat de top 10 procent van de bevolking in 1860 ongeveer 40 procent van het vermogen in de hele VS. Tegen 1890 suggereren gegevens dat de rijkste 12 procent ongeveer 86 procent van Amerika's vermogen bezat.}}\label{een-omvang-voorbij-de-wet-van-pareto-de-wet-van-pareto-stelt-dat-80-procent-van-het-voordeel-afhankelijk-is-van-of-toekomt-aan-20-procent-van-de-betrokkenen.-dit-is-ongeveer-waar-maar-opvallender-is-dat-1-procent-van-de-bevolking-van-de-verenigde-staten-287-procent-van-de-inkomstenbelasting-betaalt.-dit-suggereert-dat-samenlevingen-die-het-informatietijdperk-ingaan-een-nog-schevere-verdeling-van-inkomens-en-bekwaamheid-zullen-ervaren-dan-vilfredo-pareto-aan-het-einde-van-de-negentiende-eeuw-waarnam.-mensen-zijn-gewend-aan-substantiuxeble-vermogensongelijkheid.-in-1828-bezat-4-procent-van-de-new-yorkers-naar-schatting-62-procent-van-alle-rijkdom-in-de-stad.-in-1845-bezat-de-top-4-procent-ongeveer-81-procent-van-al-het-bedrijfs--en-niet-bedrijfsvermogen-in-new-york-city.-breder-gezien-bezat-de-top-10-procent-van-de-bevolking-in-1860-ongeveer-40-procent-van-het-vermogen-in-de-hele-vs.-tegen-1890-suggereren-gegevens-dat-de-rijkste-12-procent-ongeveer-86-procent-van-amerikas-vermogen-bezat.}

De cijfers van 1890 sluiten goed aan bij Pareto's idee. Ze wijken vooral
af door de enorme toestroom van straatarme immigranten aan het einde van
de negentiende eeuw. Hun aandeel in de totale rijkdom was te
verwaarlozen, waardoor hun komst de ongelijkheid automatisch vergrootte.
Dit illustreert hoe een echte toename van kansen vrijwel onvermijdelijk
leidt tot een tijdelijke golf van ongelijkheid. In 1890 bestond zo'n 15
procent van de Amerikaanse bevolking uit immigranten, en in sommige
noordoostelijke staten, waar veel inkomen en rijkdom werd gegenereerd,
zelfs meer dan 40 procent. Gecorrigeerd voor deze immigratiegolf paste
het Amerika van de late negentiende eeuw goed bij Pareto's formule, net
als het Zwitserland van die tijd waar Pareto zelf woonde.

Het Informatietijdperk heeft de verdeling van rijkdom al veranderd,
vooral in de VS, en is een van de oorzaken van de bitterheid in de
moderne Amerikaanse politiek, waar we in het volgende hoofdstuk verder
op ingaan. Succes in de informatiesamenleving vereist een hoog niveau
van geletterdheid en rekenvaardigheid. Een grootschalig onderzoek van
het Amerikaanse ministerie van onderwijs, \emph{Adult Literacy in
America}, wees uit dat wel 90 miljoen Amerikanen ouder dan vijftien
ernstig incompetent zijn. Of, zoals de Amerikaanse emigrant Bill Bryson
het kleurrijk verwoordde: ``Ze zijn zo dom als varkenssnot.'' Concreet
werden 90 miljoen Amerikaanse volwassenen beoordeeld als niet in staat
om een brief te schrijven, een busdienstregeling te begrijpen, of op te
tellen en af te trekken, zelfs met een rekenmachine. Wie al moeite heeft
met een bustijdenoverzicht, zal weinig kunnen met de Informatiesnelweg.
Uit dit derde deel van de Amerikanen dat zich niet heeft voorbereid op
de elektronische informatiewereld, ontstaat een boze onderklasse. Aan de
top staat een kleine groep, misschien 5 procent, van hoogopgeleide
informatie­werkers of kapitaalbezitters: de tegenhangers van de feodale
landadel. Met dit cruciale verschil: de elite van het Informatietijdperk
zijn specialisten in productie, niet in geweld.

\subsection{De megapolitiek van
innovatie}\label{de-megapolitiek-van-innovatie}

Zonder een echte goede reden zijn de meeste twintigste-eeuwse sociologen
ervan uitgegaan dat technologische vooruitgang vanzelf zou leiden tot
steeds meer egalitaire samenlevingen. Dit was vóór ongeveer 1750 niet
waar. Rond die tijd begon nieuwe industriële technologie banen te openen
voor ongeschoolden en maakte het schaalvergroting van ondernemingen
mogelijk. De fabriekstechnologie verhoogde niet alleen de reële inkomens
van de armen zonder speciale inspanning vanuit hunzelf, maar versterkte
ook de macht van politieke systemen, waardoor zij zowel beter in staat
waren om het inkomen te herverdelen als beter bestand tegen onrust. Op
langere termijn is er geen enkele reden om aan te nemen dat technologie
de verschillen in menselijke talenten en motivatie altijd maskeert in
plaats van accentueert. Sommige technologieën waren relatief egalitair
en vroegen om bijdragen van vele onafhankelijke arbeiders van ongeveer
gelijke waarde; andere concentreerden macht of rijkdom in de handen van
enkele meesters, terwijl de rest nauwelijks meer dan lijfeigenen was.
Zowel geschiedenis als technologie hebben verschillende landen op
verschillende manieren gevormd. Het Fabriekstijdperk bracht één vorm
voort, en het Informatietijdperk brengt een andere. Een die minder
gewelddadig is, maar daardoor ook meer elitair en minder egalitair dan
het tijdperk dat het vervangt.

\section{\texorpdfstring{\textbf{Ammon's
raap}}{Ammon's raap}}\label{ammons-raap}

Aan het eind van de negentiende eeuw begonnen verschillende economen,
van wie William Stanley Jevons de meest vooraanstaande was in Engeland,
wiskundige economie te ontwikkelen. Een van de eersten die
kansberekening toepaste op een grote sociale kwestie was de Duitse
econoom Otto Ammon. Zijn werk werd voor het eerst in het Engels vertaald
door Carlos C. Closson in een artikel in de \emph{Journal of Political
Economy} in 1899, getiteld \emph{Some Social Applications of the
Doctrine of Probability.}274 Men zou kunnen veronderstellen dat een
dergelijk artikel nu slechts antieke waarde heeft. In werkelijkheid
behandelt het een economisch vraagstuk dat opnieuw actueel wordt, en het
doet dat nog steeds op bijzondere wijze.

Otto Ammon was geïnteresseerd in de verdeling van bekwaamheid in de
samenleving, en de relatie daarvan tot de verdeling van inkomen en
status. Hij begon met de waarschijnlijkheidsverdeling van uitkomsten van
vier dobbelstenen met zes zijden. Van de 1.296 mogelijke worpen komen
sommige totalen vaker voor dan andere.

\begin{quote}
De som van 24 komt 1 keer voor.\\
De som van 23 komt 4 keer voor.\\
De som van 22 komt 10 keer voor.\\
De som van 21 komt 20 keer voor.\\
De som van 20 komt 35 keer voor.\\
De som van 19 komt 56 keer voor.\\
De som van 18 komt 80 keer voor.\\
De som van 17 komt 104 keer voor.\\
De som van 16 komt 125 keer voor.\\
De som van 15 komt 140 keer voor.\\
De som van 14 komt 146 keer voor.\\
De som van 13 komt 140 keer voor.\\
De som van 12 komt 125 keer voor.\\
De som van 11 komt 104 keer voor.\\
De som van 10 komt 80 keer voor.\\
De som van 9 komt 56 keer voor.\\
De som van 8 komt 35 keer voor.\\
De som van 7 komt 20 keer voor.\\
De som van 6 komt 10 keer voor.\\
De som van 5 komt 4 keer voor.\\
De som van 4 komt 1 keer voor.
\end{quote}

Daaruit blijkt meteen dat zowel hoge als lage scores relatief zeldzaam
zijn. De uiterste totalen komen in totaal slechts 35 keer voor, terwijl
de middelste zeven groepen samen 884 keer voorkomen; het middelste derde
deel van de mogelijke scores komt dus in meer dan tweederde van alle
worpen voor. Dit is de typische concentratie rond het midden die
kansberekening kenmerkt.

Ammon stelde dat deze verdeling overeenkomt met de verdeling van de
menselijke bekwaamheid. Hij schreef dit vóór de ontwikkeling van
intelligentietests en IQ's, en baseerde zich op het eerdere werk van
Francis Galton. Volgens Ammon hing maatschappelijk nut of succes niet
enkel af van intelligentie. Hij onderscheidde ``drie groepen van mentale
eigenschappen die grotendeels bepalend zijn voor de plaats die een mens
in het leven zal innemen.'' Deze waren:

\begin{enumerate}
\def\labelenumi{\arabic{enumi}.}
\item
  Intellectuele eigenschappen; waaronder ik alles reken wat behoort
  \textgreater{} tot de rationele kant van de mens: vlug begrip,
  geheugen, \textgreater{} beoordelingsvermogen, vindingrijkheid en al
  wat verder tot dit \textgreater{} gebied behoort.
\item
  Morele eigenschappen; namelijk zelfbeheersing, wilskracht, ijver,
  \textgreater{} volharding, matiging, familieplichtsgevoel, eerlijkheid
  en \textgreater{} dergelijke.
\item
  Economische eigenschappen; zoals zakelijk vermogen, \textgreater{}
  organisatietalent, technische vaardigheid, voorzichtigheid, slimme
  \textgreater{} berekening, vooruitziendheid, spaarzaamheid enzovoort.
\end{enumerate}

Aan deze geestelijke eigenschappen voegde hij toe:

\begin{enumerate}
\def\labelenumi{\arabic{enumi}.}
\setcounter{enumi}{3}
\tightlist
\item
  Lichamelijke eigenschappen; arbeidskracht, uithoudingsvermogen,
  \textgreater{} vermogen om inspanning te doorstaan en bestand te zijn
  tegen \textgreater{} allerlei prikkels, kracht, goede gezondheid,
  enzovoort.
\end{enumerate}

Volgens Ammon was de verdeling van deze eigenschappen van intelligentie,
karakter, talent en lichaam vergelijkbaar met de dobbelstenen. Hij ging
verder en stelde dat er in werkelijkheid veel meer dan vier variabelen
waren, die bovendien meer dan zes gradaties kenden. Bij acht
dobbelstenen zijn er bijvoorbeeld 1.679.616 mogelijke worpen, maar de
hoogste score, achtenveertig, komt nog steeds slechts één keer voor. Een
man of vrouw die op alle bepalende factoren uitzonderlijk hoog scoort,
is dus veel zeldzamer dan een worp van vier zessen; wellicht zo zeldzaam
als acht zessen. Toch kan een mengeling van hoge en lage scores leiden
tot ``personen met onevenwichtige, disharmonische gaven, die ondanks het
bezit van briljante eigenschappen de beproevingen van het leven niet
aankunnen.''

\begin{quote}
\emph{``Als een eenzame bergtop, of liever als de spits van een
kathedraal, rijzen de mannen van groot talent en genie boven de brede
massa van middelmatigheid uit\ldots{} Het aantal hoogbegaafden is in elk
geval zo klein dat het onmogelijk is dat `velen' van hen in lagere
klassen werden gehouden door onvolmaaktheden in sociale instituties.'' -
OTTO AMMON}
\end{quote}

\subsection{Eigenschappen en inkomens}\label{eigenschappen-en-inkomens}

Ammon richt zich vervolgens op de inkomensverdeling. Hoewel de
statistieken uit de jaren 1890 natuurlijk veel minder betrouwbaar waren
dan die van nu, was de Duitse bureaucratie al sterk ontwikkeld; Otto
Ammon vond in Saksen, Pruisen, Baden en andere Duitse staten
inkomensverdelingen die, naar zijn idee, zowel de verdeling van
menselijke bekwaamheid als de dobbelsteenkansen weerspiegelden. Hij vond
soortgelijke cijfers in Charles Booths \emph{Life and Labour of the
People of London} (1892). Inderdaad, Booths sociale verdeling ziet eruit
zoals men op basis van Ammons waarschijnlijkheidstheorie zou verwachten.
Uit Booths onderzoek bleek dat in Londen 25 procent arm of nog slechter
af was, 51,5 procent die comfortabel leefde, en 15 procent die
welgesteld of beter was; neemt men de twee laagste categorieën van Booth
samen, dan komt dat neer op 9,5 procent. Vóór de komst van de
verzorgingsstaten in de twintigste eeuw was het gebruikelijk om over de
minstbedeelden te spreken als het ``ondergedompelde tiende.''275 De twee
hoogste categorieën van Booth komen samen neer op 7 procent.

Uit dit alles trok Otto Ammon een aantal interessante conclusies. Hij
meende dat de bekwaamheid van mensen, ruim gedefinieerd, hun plaats in
de samenleving en hun inkomen bepaalden. Hij geloofde dat grote
bekwaamheid er vanzelf toe leidde dat mensen stegen in inkomen en
sociale positie. ``Als een eenzame bergtop, of liever, als de spits van
een kathedraal, rijzen de mannen van groot talent en genie boven de
brede massa van middelmatigheid uit\ldots{}'' Hij geloofde ook dat de
``ware vorm van de zogenaamde sociale piramide die van een enigszins
platte ui of knolraap is.'' Deze knolraap heeft een smalle stengel
bovenaan en een smalle wortel onderaan. Zo'n sociale knolraap is een
betere metafoor dan de sociale piramide, omdat zij, net als de moderne
industriële samenleving, haar massa in het midden heeft, terwijl de
piramide haar massa aan de onderkant heeft.

\subsection{De vorm van de raap}\label{de-vorm-van-de-raap}

Moderne industriële samenlevingen zijn inderdaad allemaal knollen, met
een kleine rijke, hogere professionele klasse aan de top, een grotere
middenklasse, en een arme minderheidsklasse aan de onderkant. In
verhouding tot het midden zijn beide uitersten klein. In het moderne
Londen, zo niet in Washington, zijn er zeker meer miljonairs dan
daklozen.

Dit alles is interessant, maar de directe relevantie van Ammons werk
ligt in de grote langetermijnverschuiving die we nu meemaken in de
financiële en politieke verhoudingen tussen de top en het midden. De
vaardigheden die nodig waren in het Fabriekstijdperk, dat nu ten einde
komt, zijn onmiskenbaar anders dan die worden vereist in het
Informatietijdperk. De meeste mensen konden de vaardigheden beheersen
die vereist waren om de machines van het midden van de twintigste eeuw
te bedienen, maar die banen zijn inmiddels vervangen door slimme
machines die zichzelf besturen. Een hoop werkgelegenheid van laag of
middelhoog vaardigheidsniveau is al verdwenen. Als wij gelijk hebben, is
dit een voorbode van het verdwijnen van het merendeel van de banen en de
herinrichting van werk via de spotmarkt.

\begin{quote}
\emph{``Toch is het een feit dat officieel, maar stilzwijgend, wordt
erkend: de meeste werkloze jongeren hebben totaal geen
kwalificaties\ldots{}''276\\
--- CLIVE JENKINS EN BARRIE SHERMAN}
\end{quote}

\section{\texorpdfstring{\textbf{Minder mensen zullen meer werk
verrichten}}{Minder mensen zullen meer werk verrichten}}\label{minder-mensen-zullen-meer-werk-verrichten}

Laten we uitgaan van de simpele vierdobbelsteenverdeling van menselijke
bekwaamheid en veronderstellen dat men in het fabriekstijdperk een score
kon halen van 4 × 2 of hoger. Dat zou betekenen dat meer dan 95 procent
van de bevolking boven wat Charles Booth ``de laagste grens van
positieve sociale bruikbaarheid'' noemde, zou zitten. In de jaren 1940
en 1950 werd 3 procent vastgesteld als de norm voor volledige
werkgelegenheid. Stel dat in het Informatietijdperk de vereiste score is
gestegen naar een 4 x 3, en dat de vereiste minimumscore is opgetrokken
van 8 naar 12. Dat zou betekenen dat bijna 24 procent onder deze grens
van ``sociale bruikbaarheid'' zou vallen.

Iets soortgelijks zou gebeuren aan de bovenkant van de schaal. In het
Fabriekstijdperk was het vereiste niveau voor een hoge bekwaamheid
misschien 4 x 4. Stel dat dit in het Informatietijdperk zou stijgen naar
4 x 5, dan zou het aandeel van mensen dat in aanmerking komt voor de
topbanen, die ook het best betaald zijn, dalen van 34 procent naar 5
procent.

Deze cijfers zijn puur hypothetisch. Uiteraard weten we niet wat de
verschuiving in de bekwaamheidseisen zal zijn of al is geweest, maar ze
zijn zeker gestegen. Door de vorm van de knol zou een vrij bescheiden
stijging in de minimale bekwaamheidseis grote aantallen mensen buiten
een betekenisvolle economische rol plaatsen. Evenzo zou een vrij kleine
stijging in de hogere bekwaamheidseisen het aantal mensen dat in
aanmerking komt voor de hogere banen zeer sterk verminderen. Er vindt
een verschuiving plaats: we weten alleen nog niet hoe groot die zal
zijn.

Er is inderdaad geen gebrek aan sociaal en politiek bewijs dat deze
verschuiving plaatsvindt in alle geavanceerde industriële samenlevingen,
dat het tempo ervan toeneemt, en dat de beweging al groot is. De
beloningen voor zeldzame vaardigheden zijn gestegen en stijgen nog
steeds. Dit is met ongenoegen opgemerkt door conventionele denkers. Neem
bijvoorbeeld \emph{The Winner-Take-All Society van} Robert H. Frank en
Philip J. Cook.277 Het documenteert de groeiende tendens dat de meest
getalenteerde concurrenten in veel sectoren in de Verenigde Staten zeer
hoge inkomens verdienen. De kansen voor middelmatige bekwaamheid nemen
daarnaast af; een aanzienlijk aantal laagwaardige vaardigheden valt nu
buiten het niveau dat een comfortabel inkomen biedt, al kunnen ze nog
steeds een plek vinden in de kleinschalige dienstverlening.

Als het Informatietijdperk hogere bekwaamheid vereist, zowel aan de
boven- als onderkant, zal iedereen behalve de bovenste 5 procent
relatief in het nadeel zijn, maar de bovenste 5 procent zal enorm
winnen. Zij zullen zowel een groter aandeel van het inkomen verdienen
als een groter aandeel behouden van wat zij verdienen. Tegelijkertijd
zullen ze een groter deel van het werk in de wereld doen dan ooit
tevoren. Velen zullen opkomen als Soevereine Individuen. In het
Informatietijdperk zal de knolraap van de inkomensverdeling er meer
uitzien zoals in 1750 dan zoals in 1950.

Samenlevingen die zijn geïndoctrineerd met de verwachting van
inkomensgelijkheid en hoge consumptieniveaus voor mensen met lage of
bescheiden bekwaamheden zullen demotivatie en onzekerheid ervaren.
Naarmate meer landen de informatietechnologie dieper integreren in de
economie, zullen ze de opkomst zien, zoals nu al duidelijk zichtbaar is
in Noord-Amerika, van een min of meer onbruikbare onderklasse. Dit is
precies wat er gebeurt. Dit zal leiden tot een reactie met een
nationalistische, antitechnologische inslag, zoals we in het volgende
hoofdstuk uiteenzetten.

Het Fabriekstijdperk kan wel eens een unieke periode zijn geweest waarin
halfdomme machines een zeer winstgevende niche aan ongeschoolde mensen
verschaften. Nu machines zelfstandig functioneren, komen de vruchten van
het Informatietijdperk terecht bij de top 5 procent van Otto Ammon's
knol. Het Informatietijdperk zag er al veel beter uit voor de bovenste
10 procent, de zogenaamde cognitieve elite. Het zal echter het meest
gunstig zijn voor de bovenste 10 procent van de bovenste 10 procent, de
cognitieve dubbele top. In het feodale tijdperk waren er honderd
halfgeschoolde boeren nodig om één hooggeschoolde krijgsheer (of ridder)
te onderhouden. De Soevereine Individuen van de informatiemaatschappij
zullen geen krijgsheren zijn, maar meesters van gespecialiseerde
vaardigheden, waaronder ondernemerschap en investeringen. Toch lijkt de
feodale verhouding van honderd tegen één terug te keren. Hoe dan ook, de
samenlevingen van de eenentwintigste eeuw zullen waarschijnlijk
ongelijker zijn dan die waarin we in de twintigste eeuw hebben geleefd.

\section{\texorpdfstring{\textbf{De meeste mensen zullen profiteren van
de dood van de
politiek}}{De meeste mensen zullen profiteren van de dood van de politiek}}\label{de-meeste-mensen-zullen-profiteren-van-de-dood-van-de-politiek}

Het is onwaarschijnlijk dat de egalitaire economie, en daarmee de naties
die haar ondersteunen, zonder crisis zal verdwijnen. Hoewel een crisis
van nature kort duurt, verwachten we desalniettemin dat het trauma dat
gepaard gaat met het verdwijnen van naties nog jarenlang zal voortduren.
Zonder dit trauma, waarvan we de omvang later uitgebreid behandelen, uit
het oog verliezen, mogen we niet vergeten dat de overgang naar de
informatie-economie in veel delen van de wereld tot een enorme
productieboost zal leiden, met hogere inkomens voor iedereen. Inderdaad,
in regio's die nooit ten volle hebben kunnen profiteren van de voordelen
van het industrialisme maar nu de vrije markt omarmen, stijgen de
inkomens in alle lagen van de bevolking al, of zullen gaan stijgen.

Het afnemen van dwang als kenmerk van het economische leven zal
producenten in staat stellen de activa te behouden die voorheen werden
onteigend en herverdeeld. Herverdeling betekende meestal dat activa
werden ingezet voor minder waardevolle doeleinden, waardoor de
productiviteit van kapitaal afnam. Vermogen werd disproportioneel
ontnomen van de personen die het meest bedreven waren in het investeren
van middelen en door politici herverdeeld aan degenen die minder
bedreven waren. In de meeste gevallen werd het herverdeeld inkomen
ingezet in economische activiteit van lagere waarde. Het bevrijden van
middelen van deze systematische dwang zal afhankelijk van de jurisdictie
sterk andere gevolgen hebben. Het zal het faillissement van de
verzorgingsstaten betekenen, en de schaalnadelen versterken die grote
overheden en alle door hen gesubsidieerde instellingen ondermijnen. Aan
de andere kant zal de overgang naar de cybereconomie de economische
nadelen verminderen voor mensen die opereren onder soevereiniteiten in
regio's die traditioneel gezien moeite hadden met grootschalige
organisatie.

\begin{quote}
\emph{``Als de wereld als één grote markt opereert, zal iedere werknemer
concurreren met iedere persoon in de wereld die hetzelfde werk kan doen.
Er zijn er veel en velen hebben honger.''278\\
--- ANDREW S. GROVE, PRESIDENT, INTEL CORP.}
\end{quote}

\section{\texorpdfstring{\textbf{Verschuivende
locatievoordelen}}{Verschuivende locatievoordelen}}\label{verschuivende-locatievoordelen}

Doordat de opbrengsten van geweld niet langer stijgen, verdwijnt ook het
nut om onder een overheid te leven die ze zou kunnen opeisen. Ooit
bekwame overheden zullen niet langer vermogensopbouw bevorderen, maar er
juist vijandig tegenover staan. Hoge belastingen, forse
reguleringskosten en ambitieuze verplichtingen tot inkomensherverdeling
maken de gebieden onder hun controle onaantrekkelijk voor
ondernemerschap.

Degenen die wonen in jurisdicties die arm of onderontwikkeld bleven
tijdens de industriële periode hebben het meest te winnen bij de
bevrijding van de geografische beperkingen op de economie. Dit druist in
tegen wat je vaak hoort. Het grootste debat rond de opkomst van de
informatie-economie en het Soevereine Individu zal draaien om de
vermeende negatieve effecten op de ``eerlijkheid'' die voortkomen uit de
dood van de politiek. Het is waar dat de wereldwijde informatie-economie
een dodelijke klap zal toebrengen aan grootschalige
inkomensherverdeling. De belangrijkste begunstigden van
inkomensherverdeling in het industriële tijdperk waren inwoners van
rijke jurisdicties, waar het consumptieniveau twintig keer hoger lag dan
het wereldgemiddelde. Alleen binnen de OESO-landen verhoogde
inkomensherverdeling merkbaar de inkomens van ongeschoolden.

De grootste inkomensongelijkheid is echter altijd waargenomen tussen
jurisdicties. Herverdeling heeft daar weinig aan veranderd. Sterker nog,
buitenlandse hulp en internationale ontwikkelingsprogramma's hebben
volgens deze visie vaak een pervers effect gehad: ze verlaagden de reële
inkomens van arme mensen in arme landen door incompetente regeringen in
stand te houden. Dit thema wordt later verder uitgediept bij de morele
implicaties van de Informatierevolutie.

\subsection{Een eeuw van stijgende
inkomensongelijkheid}\label{een-eeuw-van-stijgende-inkomensongelijkheid}

In de industriële periode bepaalde het rechtsgebied waarin iemand woonde
grotendeels zijn levenslange inkomen. In tegenstelling tot de gangbare
indruk in de welvarende economieën van tegenwoordig, nam de
inkomensongelijkheid in die tijd sterk toe. Volgens een schatting van de
Wereldbank lag het gemiddelde inkomen per inwoner in de rijkste landen
in 1870 elf maal hoger dan in de armste landen, en in 1985 was dit
gestegen tot tweeënvijftig maal. Hoewel de wereldwijde ongelijkheid
drastisch toenam, merkten de inwoners van de rijke industriële landen
dat nauwelijks op: de inkomensverschillen namen vooral toe tussen
rechtsgebieden en niet binnen de rechtsgebieden.

Zoals we eerder al bespraken, zorgde de aard van de industriële
technologie ervoor dat de inkomenskloof afnam in jurisdicties waar
capabele overheden op grote schaal de macht uitoefenden. Naarmate de
effecten van geweld toenamen, zoals tijdens het industriële tijdperk
gebeurde, werden overheden die op grote schaal opereerden vaak door hun
werknemers bestuurd. Daardoor bleek het praktisch onmogelijk om de
aanspraken, die deze overheden op middelen maakten, in te perken. Hun
onbelemmerde controle over middelen leverde een aanzienlijk militair
voordeel op, zolang de omvang van hun macht zwaarder woog dan de
efficiëntie om deze te benutten. Niet toevallig versnelde de
inkomensherverdeling sterk dankzij het overheidsbestuur door haar
medewerkers. Bijna iedere samenleving kent wel een vorm van
inkomensherverdeling, al is dit soms slechts tijdelijk in uitzonderlijke
omstandigheden. Als je de geschiedenis van steun aan de armen goed
bestudeert, blijkt echter dat uitkeringen meestal het gulst zijn in
tijden dat er weinig armoede heerst. Een beperking van de herverdeling
van inkomens komt eerder voor wanneer inkomens voor grote groepen mensen
dalen. In de welvarende industriële samenlevingen van de tweede helft
van de twintigste eeuw waren de omstandigheden vrijwel ideaal voor
inkomensherverdeling. Dat resulteerde in veel hogere beloningen voor
laaggeschoolde arbeid in deze bevoorrechte rechtsgebieden, waardoor
zelfs mensen die helemaal niet werkten van een hoog consumptieniveau
konden genieten.

\subsection{De paradox van industriële
rijkdom}\label{de-paradox-van-industriuxeble-rijkdom}

De ironie is dat juist in deze jurisdicties ook meer mensen rijk werden.
Deze schijnbare paradox wordt begrijpelijk zodra je de dynamiek van
megapolitiek uit eerdere hoofdstukken begrijpt. De leidende sectoren van
de industriële economie vereisten grootschalige ordehandhaving om
optimaal te functioneren. Dit maakte ze bijzonder kwetsbaar voor
chantage door vakbonden en overheden die hun invloed zo veel mogelijk
wilden uitbreiden. Toch verstikte de brede inkomensherverdeling het
functioneren van de industriële economie niet volledig. Wie tijdens de
bloeiperiode van het industrialisme het geluk had geboren te worden in
West-Europa, de voormalige Britse koloniën of Japan, had daardoor
waarschijnlijk een veel hoger inkomen dan iemand met gelijkwaardige
vaardigheden in Zuid-Amerika, Oost-Europa, de late Sovjet-Unie, Afrika
of Azië. De gunstige impact van informatietechnologie zal onder meer
bestaan uit het wegnemen van veel obstakels die het grootste deel van de
wereldbevolking hebben verhinderd om tijdens een groot deel van de
moderne periode te ontwikkelen en te profiteren van de voordelen van
vrije markten.

\begin{quote}
\emph{``De inheemse kenmerken van arme landen zijn opvallend ongeschikt
voor effectieve grootschalige organisatie, vooral voor grootschalige
organisaties die (zoals overheden) over een groot geografisch gebied
moeten opereren.''280 --- MANCUR OLSON}
\end{quote}

\section{\texorpdfstring{\textbf{Schaalnadelen en vertraagde
groei}}{Schaalnadelen en vertraagde groei}}\label{schaalnadelen-en-vertraagde-groei}

Zoals Mancur Olson heeft aangetoond, lag de achterstand in de twintigste
eeuw niet aan een gebrek aan kapitaal of gespecialiseerde vaardigheden
op zich. In \emph{Diseconomies of Scale and Development}, een essay uit
1987, twee jaar vóór de val van de Berlijnse Muur, schreef Olson:

\begin{quote}
``Als kapitaal in arme landen daadwerkelijk schaars zou zijn geweest,
dan had de `marginale productiviteit' ervan, en dus de winstgevendheid
van het gebruik, groter moeten zijn dan in de welvarende landen. De lage
groeicijfers van veel landen die significante hoeveelheden buitenlandse
hulp ontvingen en de lage productiviteit van sommige moderne fabrieken
die in arme landen werden gebouwd, hebben de overtuigingskracht van de
these van `kapitaalschaarste' als oorzaak van onderontwikkeling verder
verzwakt.''281
\end{quote}

Dit kan niet anders dan kloppen. Als een tekort aan kapitaal of
vaardigheden het belangrijkste probleem zou zijn geweest, zouden de
opbrengsten van beide hoger zijn geweest in arme jurisdicties dan in
ontwikkelde landen. Zowel geschoold personeel als kapitaal zouden
massaal naar deze regio's zijn gestroomd totdat de opbrengsten gelijk
zouden zijn geworden. In werkelijkheid gebeurde vaak het
tegenovergestelde: er vond een aanzienlijke emigratie plaats van
opgeleide mensen uit achtergebleven jurisdicties. De weinigen die er wel
in slaagden om kapitaal op te bouwen, brachten dat zo snel mogelijk
onder in Zwitserland en andere ontwikkelde landen.

\subsection{Beter bestuur kon niet geïmporteerd
worden}\label{beter-bestuur-kon-niet-geuxefmporteerd-worden}

Olson stelt, en wij zijn het daarmee eens, dat het werkelijke obstakel
voor ontwikkeling in achtergebleven landen de enige productiefactor was
die niet gemakkelijk geleend of geïmporteerd kon worden: de overheid.
Dit probleem werd groter naarmate de twintigste eeuw vorderde. In 1900
hielden Groot-Brittannië en Frankrijk, samen met enkele andere Europese
landen, zich bezig met het exporteren van competente bestuurssystemen
naar regio's waar lokale machten niet in staat waren om succesvol op
grote schaal te functioneren. Veranderende megapolitieke omstandigheden
in de twintigste eeuw verhoogden echter de kosten en verlaagden de
opbrengsten van deze activiteit. Kolonialisme, of imperialisme, zoals
het minder vriendelijk werd genoemd, was geen rendabele onderneming
meer. Technologische verschuivingen verhoogden de kosten van
machtsprojectie van het centrum naar de periferie en verlaagden de
militaire kosten van effectieve weerstand. Als gevolg trokken imperiale
machten zich terug of bleven slechts aanwezig in kleine enclaves zoals
Bermuda of de Kaaimaneilanden.

\begin{quote}
\emph{``Als de postkoloniale natiestaat een belemmering voor vooruitgang
was geworden, zoals tegen het einde van de jaren tachtig steeds meer
critici in Afrika leken te erkennen, dan kon er weinig twijfel bestaan
over de voornaamste reden. De staat bevrijdde en beschermde zijn burgers
niet, ongeacht wat de propaganda claimde. Integendeel, het netto-effect
was beperkend en uitbuitend, anders functioneerde de staat eenvoudigweg
helemaal niet in sociale zin.''282 --- BASIL DAVIDSON}
\end{quote}

De inheemse overheden die het koloniale bestuur vervingen in landen waar
Europeanen zich niet hadden gevestigd, rekruteerden hun leiders en
bestuurders uit bevolkingsgroepen met weinig ervaring of bekwaamheid in
het leiden van grootschalige ondernemingen. In veel gevallen, vooral in
Afrika, werd de infrastructuur die door de vertrekkende koloniale
machten was achtergelaten, snel geplunderd, vernietigd of verwaarloosd.
Telefoonlijnen werden neergehaald en omgesmeed tot armbanden, wegen
werden niet meer onderhouden, spoorlijnen raakten onbruikbaar doordat
wegbedden instortten en locomotieven stuk gingen. In Zaïre was de
uitgebreide transportinfrastructuur die door de Belgen was aangelegd
tegen 1990 vrijwel volledig verdwenen. Alleen een paar gammele
rivierboten functioneerden nog, waarvan er één door de dictator werd
overgenomen als een soort drijvend paleis.

Onbetrouwbare communicatie- en transportinfrastructuur weerspiegelen de
incompetentie van achtergebleven natiestaten in het handhaven van orde.
Ze hebben prijzen hoog gehouden en kansen voor het grootste deel van de
wereldbevolking geminimaliseerd. Zoals Olson benadrukt:

\begin{quote}
\emph{``Ten eerste dwingen slechte transport- en communicatiesystemen
een bedrijf om vooral te vertrouwen op lokale productiefactoren.
Naarmate een bedrijf opschaalt, moet het verder weg zoeken naar
productiefactoren, en hoe slechter de transport- en de
communicatie-infrastructuur, hoe sneller deze kosten zullen stijgen bij
een groeiende output. Het tweede en belangrijkere punt waarom slechte
transport- en communicatiesystemen nadelig zijn voor effectieve
grootschalige ondernemingen, is dat ze het veel moeilijker maken om
zulke ondernemingen effectief te coördineren.''283}
\end{quote}

\subsection{Het verlichten van de last van slecht
bestuur}\label{het-verlichten-van-de-last-van-slecht-bestuur}

De ambitieuze armen in de wereld hebben, meer dan wie dan ook, baat bij
het feit dat informatietechnologie het vermogen om inkomen te verdienen
loskoppelt van de plek waar men woont. Nieuwe technologieën, zoals de
digitale mobiele telefoon, maken communicatie mogelijk onafhankelijk van
het vermogen van de lokale politie om elke telefoonpaal te beschermen
tegen koperdieven. Naarmate draadloze fax- en internetverbindingen
beschikbaar komen, wordt het minder relevant of wanhopig arme
postmedewerkers de post achterhouden om de postzegel te stelen.

In veel gevallen vervangt effectieve communicatie zelfs de noodzaak van
fysiek transport van goederen en diensten. Betere communicatie en sterk
toegenomen rekenkracht maken niet alleen de coördinatie van complexe
activiteiten goedkoper en efficiënter, ze verlagen ook schaalvoordelen
en ontbinden grote organisaties. Al deze veranderingen verminderen de
nadelen die mensen in achtergebleven landen hebben ondervonden door
incompetente overheden. De Informatierevolutie zal het belang van
capabele overheden flink doen afnemen. Hierdoor wordt het voor mensen in
traditioneel arme landen gemakkelijker om de obstakels te overwinnen die
hun overheden tot nu toe in de weg van economische groei hebben
geplaatst.

\subsection{Gelijke kansen in het
Informatietijdperk}\label{gelijke-kansen-in-het-informatietijdperk}

In het Informatietijdperk zullen bekende locatievoordelen snel door
technologie worden getransformeerd. Het verdienvermogen van mensen met
vergelijkbare vaardigheden zal veel gelijker worden, ongeacht de
jurisdictie waar ze in wonen. Dit proces is al begonnen. Omdat
instellingen die dwang en lokale voordelen gebruikten om inkomen te
herverdelen macht verliezen, zal de inkomensongelijkheid binnen
jurisdicties toenemen. Tegelijk zal wereldwijde concurrentie de
inkomsten van de meest getalenteerde individuen in elk vakgebied
verhogen, waar ze ook wonen, vergelijkbaar met wat nu gebeurt in de
professionele sport. De marginale waarde van superieure prestaties op
een wereldmarkt zal enorm zijn.

Hoewel het publieke debat zal focussen op groeiende ``ongelijkheid'' in
de OESO-landen, zullen individuen overal veel gelijkere kansen hebben.
Ze hoeven om succes te hebben niet langer in een jurisdictie te wonen
die goed functioneert op grote schaal. Aangeboren talenten en de
bereidheid om deze te ontwikkelen, zullen op een eerlijker speelveld dan
ooit tevoren worden gemeten. Jurisdictionele voordelen die tijdens het
industriële tijdperk leidden tot toenemende ongelijkheid tussen rijke en
arme economieën, zullen ingrijpend veranderen.

\subsection{Hogere rendementen in arme
gebieden}\label{hogere-rendementen-in-arme-gebieden}

De belemmeringen op het functioneren van vrije markten, die overheden in
armere regio's opwerpen, zullen sterk afnemen naarmate de cybereconomie
op gang komt. ALs gevolg zullen schaars kapitaal en vaardigheden in veel
van de huidige arme regio's hogere rendementen opleveren, precies zoals
ontwikkelingsdeskundigen in de jaren vijftig voorspelden. Bovendien
zullen kapitaal en vaardigheden veel makkelijker worden geïmporteerd.
Opkomende economieën zullen niet langer zo sterk als in het Industriële
Tijdperk afhankelijk zijn van lokale productiefactoren. Hun verbeterde
vermogen om op afstand kapitaal en expertise aan te boren, zal leiden
tot hogere groeicijfers. Dit zal gebeuren, of incompetente overheden
eerlijker worden of beter in staat zijn eigendomsrechten te beschermen,
of niet. Zonder controle over de cyberspace zullen slechte overheden
simpelweg minder in staat zijn om mensen binnen hun jurisdicties te
beletten te profiteren van economische vrijheid.

\subsection{Positieve versterking}\label{positieve-versterking}

In de nieuwe cybereconomie zal de vrijwel totale draagbaarheid van
informatietechnologie het hamsteren van veel van de voordelen van
jurisdicties uit het Industriële Tijdperk verhinderen. Verhoogde
concurrentie tussen een groeiend aantal jurisdicties zal afhangen van
nieuwe soorten lokale voordelen. Soevereiniteit zal commercieel worden
in plaats van roofzuchtig. Overheden zullen door de kracht van
concurrentie verplicht zijn om beleid te voeren dat aantrekkelijk is
voor degenen die de grootste bijdrage leveren aan het economische
welzijn, niet voor degenen die weinig bijdragen of wiens economische
bijdrage negatief is.

Dit betekent een enorme verandering ten opzichte van de gangbare
praktijk in de twintigste eeuw. De ideologie van de natiestaat was dat
het leven positief gereguleerd kan en moet worden door het subsidiëren
van ongewenste uitkomsten en het bestraffen van de wenselijke. Arm zijn
is ongewenst, daarom werden de armen gesubsidieerd. Rijk worden is
wenselijk, daarom werden strenge belastingen geheven op de rijken, zodat
het leven ``eerlijker'' werd.

Omdat deze hele beleidsaanpak geworteld was in een megapolitieke basis
die aan elke kritiek weerstand bood, deed het weinig ter zake wat de
perverse gevolgen waren van het subsidiëren van disfunctioneren. Er werd
ook nauwelijks rekening gehouden met de vaardigheden, het harde werk of
de vindingrijkheid die nodig waren om het herverdeelde vermogen in
eerste instantie te verwerven. Uitkomsten werden beoordeeld op basis van
aanspraken. Volgens het twintigste-eeuwse politieke denken moesten
uitkomsten gelijk zijn om als ``rechtvaardig'' te gelden.

\subsection{Het nieuwe paradigma}\label{het-nieuwe-paradigma}

De nieuwe megapolitieke omstandigheden van de eenentwintigste eeuw
zullen het mogelijk maken dat markttesten de uitkomsten reguleren in
gebieden die voorheen door de politiek werden gedomineerd. Het
marktparadigma veronderstelt dat resultaten beter kunnen worden
gereguleerd door wenselijke uitkomsten te belonen en ongewenste te
bestraffen. Arm zijn is ongewenst, rijk worden is wenselijk. Daarom
zouden prikkels rijkdom moeten belonen en mensen aanmoedigen te betalen
voor de middelen die ze gebruiken. Het leven is ``eerlijker'' wanneer
mensen meer van hun verdiende inkomen mogen behouden.

Deze visie zal in het nieuwe millennium vaker gehoord worden dan in de
eeuw die nu ten einde komt. Bovendien zal het overtuigender zijn dan
ooit, omdat het een megapolitieke basis heeft. Kapitaal in het
Informatietijdperk wordt met de dag mobieler. Het vermogen om hoge
inkomens te verdienen is niet langer gebonden aan het verblijf op
specifieke locaties, zoals vroeger het geval was toen het meeste
vermogen werd gecreëerd door de exploitatie van natuurlijke hulpbronnen.
Elke dag wordt het makkelijker voor mensen die gebruikmaken van
draagbare informatietechnologie om activa te creëren die veel minder
onderhevig zijn aan dwang dan enig ander type van vermogen ooit was.

Arbitraire politieke regels die kosten opleggen zonder daar
marktvoordelen tegenover te stellen, zullen binnenkort onhoudbaar zijn.
Krachtige concurrentiedruk heeft de neiging om prijzen van goederen,
diensten, arbeid en kapitaal wereldwijd gelijk te trekken. Overheden
zullen minder speelruimte hebben dan ze gewend zijn om willekeurige
beleidsmaatregelen op te leggen. Elke overheid die probeert om zwaardere
regels op te leggen dan andere soevereinen, zal die activiteit
eenvoudigweg verdrijven. In sommige gevallen zal het verdrijven van
ongewenste activiteiten de markt echter juist tevreden stellen en die
jurisdicties populairder en welvarender maken. In dit opzicht kunnen
bepaalde regels worden vergeleken met de huisregels die eigenaren van
een hotelketen opleggen. Als ze verbieden om op blote voeten te lopen of
in de lobby te roken, zullen ze ongetwijfeld sommige klanten verliezen.
Maar het weren van die klanten hoeft de jurisdictie als geheel geen
inkomsten te kosten. Niet-rokers met goed schoeisel betalen mogelijk
zelfs meer omdat rokers op blote voeten worden uitgesloten. Op dezelfde
manier kunnen regels die het kostbaar of onmogelijk maken om een
verwerkingsinstallatie in een bepaalde jurisdictie te exploiteren, de
activiteit elders brengen zonder de jurisdictie als geheel van inkomen
te beroven.

Deze voorbeelden laten zien dat regels in zeldzame gevallen een
positieve in plaats van een negatieve marktwaarde kunnen hebben, vooral
in een wereld met een snel toenemend aantal jurisdicties. Regels die
hoge normen voor volksgezondheid, schone lucht en schoon water
handhaven, zullen in veel gebieden zeer gewaardeerd worden. Zo zullen
andere, soms meer exotische regels en convenanten, zoals opgelegd door
vastgoedontwikkelaars of hotels die zich op bepaalde marktsegmenten
richten, ook waardevol blijken.

\subsection{Geen douanehuis in
cyberspace}\label{geen-douanehuis-in-cyberspace}

We verwachten dat de commercialisering van soevereiniteit snel zal
leiden tot de devolutie van veel grote territoriale soevereinen. Het
feit dat informatietechnologie niet onderworpen kan worden aan
grenscontroles zoals bij de handel in industriële en landbouwproducten
nog wel het geval is, heeft belangrijke implicaties. Het betekent dat
protectionisme na verloop van tijd minder effectief zal zijn naarmate
handel in informatie de plaats van fysieke producten inneemt bij het
genereren van vermogen. Het betekent ook dat kleinere regio's, voor de
toegang tot markten, steeds minder afhankelijk zullen zijn van het in
stand houden van uitgebreide politieke jurisdicties.

Informatietechnologie stelt mensen die werkzaam zijn in voorheen
beschermde dienstensectoren bloot aan buitenlandse concurrentie. Als een
bedrijf in Toronto twintig jaar geleden een boekhouder wilde aannemen,
moest die persoon fysiek in Toronto of een nabijgelegen woonplaats
binnen woon-werkafstand zijn. In het Informatietijdperk kan een
boekhouder in Boedapest of Bangalore, India, het werk doen en al het
benodigde materiaal versleuteld via het internet downloaden. Directe
communicatie via satellietverbindingen maakt elk deel van de wereld
slechts een moment verwijderd via modem en fax. Iemand die behoefte
heeft aan aandelenanalisten, kan er zevenentwintig in India aannemen
voor de prijs van één op Wall Street. Naarmate informatietechnologie
elke achttien maanden een orde van grootte verbetert (Wet van Moore),
zullen steeds meer dienstverleners worden blootgesteld aan
prijsconcurrentie die politici in wezen onmogelijk kunnen belemmeren.
Deze concurrentie zal uiteindelijk even volledig van toepassing zijn op
de hogere beroepsgroepen als op boekhouders. Digitale advocaten en
cyberartsen zullen zich vermenigvuldigen in de Informatie-economie.

\subsection{Doodvonnis voor
natiestaten}\label{doodvonnis-voor-natiestaten}

Naarmate de economische voordelen die voorheen binnen de grenzen van
natiestaten werden vastgehouden verdwijnen, zullen de natiestaten zelf
uiteindelijk instorten onder hun zware verplichtingen. Het feit dat alle
natiestaten op een doodvonnis staan, betekent echter niet dat ze
allemaal tegelijk zullen verdwijnen. Integendeel. Devolutionaire druk
zal het sterkst zijn in grote politieke entiteiten waar het inkomen van
de meeste mensen stagneert of daalt. Jurisdicties in Latijns-Amerika en
Azië waar het inkomen per hoofd van de bevolking snel stijgt, kunnen
generaties lang standhouden, of totdat het levenslange
inkomensperspectief daar gelijk is aan dat in de voorheen rijke
industriële landen. Op dat moment zullen er geen eenvoudige winsten door
kostenbesparing meer te behalen zijn, en wordt de politiek van groei
uitdagender.

We vermoeden ook dat natiestaten met één grote metropool langer coherent
blijven dan staten met meerdere grote steden, en dus meerdere
machtscentra en hun achterlanden.

Een andere stimulans voor devolutie zal de hoge schuldenlast van de
centrale overheid zijn. De drie rijke industriële landen met de hoogste
relatieve schuldenlast, Canada, België en Italië, hebben niet toevallig
ook geavanceerde separatistische bewegingen. Alle drie hebben te maken
gehad met chronische begrotingstekorten en hebben nu nationale schulden
die boven 100 procent van het BBP uitstijgen. Naarmate de nationale
schuld in elk land toenam, groeide ook de aantrekkingskracht van
separatistische bewegingen. In Italië is de Lega Nord opgekomen als een
dynamische en populaire regionale politieke beweging, gebaseerd op een
eenvoudige wiskundige observatie: Noord-Italië, of ``Padania'', zou
rijker zijn dan Zwitserland als grote delen van het inkomen niet naar
Rome en het armere zuiden zouden worden afgetapt. De Lega Nord stelt een
voor de hand liggende oplossing voor: afscheiden van Italië en zo
ontsnappen aan sommige van de ernstige gevolgen van samengestelde rente.
Zo manoeuvreren in België, waar de nationale schuld boven 130 procent
van het BBP ligt, ook de Vlamingen en Walen als een vijandig echtpaar
voor een scheiding. Een groeiende minderheid van de Vlamingen stelt dat
zij de Walen oneerlijk subsidiëren en hun economische situatie kunnen
verbeteren door België in tweeën te splitsen.

In Canada verschilt de situatie doordat Frans Canada, de belangrijkste
regio die nu separatistische ideeën aanhangt, historisch gesubsidieerd
werd door Engels Canada. Maar naarmate de federale schuld en het tekort
toenemen, dringt het besef door in Quebec dat deze vorm van
inkomensherverdeling zal afnemen. Het Bloc Québécois flirt daarom met
een aantrekkingskracht die het tien jaar geleden nog niet had: de
belofte om het inkomen na belasting te verhogen door de betaling van
federale belastingen af te schaffen. Separatistische leiders suggereren
ook dat Quebec Canada zou moeten verlaten zonder een evenredig deel van
de federale schuld te dragen.

Engels-Canadezen verzetten zich tegen dit argument en zijn het vaak
oneens met de implicaties, omdat ze zich bewust zijn van de grote
transfers naar Quebec door de jaren heen. Desondanks is de
aantrekkingskracht van het Parti Québécois sterk, en het lijkt slechts
een kwestie van tijd voordat een afscheidingsreferendum Canada doet
uiteenvallen. Een soortgelijk lot wacht andere natiestaten wanneer hun
financiële situatie verslechtert.

Een andere nadelige factor voor het overleven van Canada op de
lange-termijn is dat het een dunbevolkt land is met een omvangrijke
industriële infrastructuur die onderhouden moet worden. De overgang naar
het Informatietijdperk ontwaardt fysieke infrastructuur onvermijdelijk.
Naarmate telewerkers fabrieks- en kantoormedewerkers vervangen, wordt
het minder belangrijk dat snelwegen en andere transportwegen opnieuw
worden aangelegd en goed onderhouden. Met fiscale crises aan alle kanten
zullen steeds meer groepen in Canada teruggrijpen naar het
achttiende-eeuwse perspectief op de financiering van openbare goederen
zoals bepleit door Adam Smith. Hij schreef in \emph{The Wealth of
Nations}:

\begin{quote}
\emph{``Stel dat de straten van Londen verlicht en geplaveid zouden
worden ten laste van de {[}nationale{]} schatkist, is er dan enige kans
dat ze even goed verlicht en geplaveid zouden zijn als nu, of zelfs
tegen zo'n lage kosten? De kosten zouden bovendien, in plaats van te
worden geheven via een lokale belasting van de bewoners van elke
specifieke straat, parochie of wijk in Londen, in dit geval worden
betaald uit de algemene staatsinkomsten, en zouden dus worden geheven
via een belasting op alle inwoners van het koninkrijk, van wie het
grootste deel geen enkel voordeel trok uit het verlichten en verharden
van de straten van Londen.''284}
\end{quote}

Vervang London door Toronto, en je zit midden in een redenering die
velen in Alberta en British Columbia zullen maken. De logica van
devolutie zal besmettelijk blijken.

Wanneer Canada uiteenvalt, zal dit leiden tot een duidelijke toename van
afscheidingsbewegingen in het noordwesten van de Verenigde Staten.
Inwoners van Alaska, Washington, Oregon, Idaho en Montana zouden een
duidelijk nadeel ervaren in concurrentie met Alberta en British Columbia
als onafhankelijke soevereiniteiten.

\section{\texorpdfstring{\textbf{Na de
natiestaat}}{Na de natiestaat}}\label{na-de-natiestaat}

In plaats van natiestaten zul je aanvankelijk kleinere rechtsgebieden op
provinciaal niveau zien en uiteindelijk nog kleinere soevereiniteiten,
enclaves van diverse aard die doen denken aan middeleeuwse stadstaten,
omgeven door hun achterland. Hoewel dit vreemd kan overkomen op mensen
die van jongs af aan geleerd hebben dat politiek allesbepalend is,
kiezen de leiders van deze nieuwe ministaten er in veel gevallen voor om
hun beleid te laten vormgeven door ondernemerschap en strategische
positionering, in plaats van door politieke conflicten. Deze
gefragmenteerde soevereiniteiten bedienen een breed scala aan smaken,
net zoals hotels en restaurants dat doen, door in hun openbare ruimten
specifieke regels vast te stellen die naadloos aansluiten bij de wensen
van de marktsegmenten waaruit zij hun klanten halen. Dat betekent echter
niet dat er geen bijzondere uitdagingen ontstaan bij het organiseren van
bescherming op nomadische basis. Die pakken we in het volgende hoofdstuk
aan.

\begin{quote}
\emph{``Dorpslucht brengt vrijheid.'' - MIDDELEEUWSE GEWOONHEID}
\end{quote}

\subsection{Niet-burgers van de Pale}\label{niet-burgers-van-de-pale}

Ondanks deze moeilijkheden vindt menselijke vindingrijkheid meestal een
manier om instellingen te creëren die winstgevende kansen benutten,
zelfs wanneer de vraag van personen die weinig kunnen betalen, komt.
Waar de potentiële klanten behoren tot de rijkste mensen op aarde, moet
die neiging des te sterker zijn. Een exit, of ``stemmen met de voeten,''
is altijd een optie wanneer verouderde producten, organisaties of zelfs
overheden hun aantrekkingskracht verliezen en weinig vooruitzicht bieden
op directe verbetering. Denk bijvoorbeeld aan de groei van middeleeuwse
die dienden als veilige toevluchtsoorden voor horigen die aan feodale
onderdrukking ontsnapten. Zij zouden een treffende analogie kunnen
blijken voor de rol die nieuwe jurisdicties zullen spelen bij het
faciliteren van de exit uit natiestaten. De acceptatie van vreemdelingen
die aan een heer ontkwamen als ``burgers van de Pale'' tartte de
geldende conventies van het feodale recht en de bisschoppelijke macht.
Toch was het over het algemeen een succesvol alternatief voor degenen
die het toepasten, en het droeg aanzienlijk bij aan het verzwakken van
de greep van het feodalisme. Zoals de middeleeuwse historicus Fritz
Rorig stelde, zou de horige van een seculiere heer ``na een jaar en een
dag een vrije burger van de stad zijn.''285 Het is redelijk te
verwachten dat er nieuwe institutionele toevluchtsoorden zullen
ontstaan, op basis van ``nieuwe juridische principes,'' om fiscale
toevlucht te bieden aan burgers van de staat, net zoals de middeleeuwse
stad toevlucht bood aan feodale onderdanen die binnen de schaduw van
haar muren leefden.

Econoom Albert O. Hirschman, die de theoretische subtiliteiten van
``stemmen met de voeten'' onderzocht in \emph{Exit, Voice, and Loyalty},
voor het eerst gepubliceerd in 1969, voorzag dat technologische
vooruitgang de kans zou vergroten dat men een exit-strategie zou
aannemen wanneer staten in verval raken. Hij schreef: ``Pas wanneer
landen op elkaar beginnen te lijken door de vooruitgang in communicatie
en algehele modernisering zal het gevaar van voortijdige en
buitensporige exits ontstaan \ldots{}''286 Dat is precies wat er
gebeurt. Informatietechnologie vermindert snel veel van de verschillen
tussen jurisdicties, waardoor vertrekken een veel aantrekkelijkere optie
wordt. Uiteraard verwijzen Hirschmans woorden ``voortijdige en
buitensporige exits'' naar het standpunt van wat het beste is voor de
staat die men achterlaat. Ongetwijfeld geloofden heren in middeleeuws
Europa dat zij leden onder ``voortijdige en buitensporige exits'' van
hun horigen naar steden die hen de vrijheid brachten die ze zochten.

Om terug te komen op ons eerdere voorbeeld: het lijkt misschien
vergezocht om aan te nemen dat er een aantal ministaten zullen zijn die
toevlucht bieden aan ballingen die de stervende natiestaten ontvluchten,
maar dat is het niet. Deze soevereiniteiten zullen concurreren op de
algemene voorwaarden voor ballingschap. Sommige ervan, bijvoorbeeld aan
de westkust van Noord-Amerika, zouden zich kunnen richten op mensen die
niet roken en een hekel hebben aan passief roken. Uiteraard zouden
dergelijke regimes niet populair zijn bij rokers. Het verbieden van hun
gewoonte zal voor veel rokers overkomen als een arbitraire oplegging.

In het industriële tijdperk van massapolitiek werden dergelijke
meningsverschillen uitgevochten in politieke campagnes die uiteindelijk
een van de partijen ertoe dwong om zich neer te leggen bij de wensen van
de machtigere partij. Het is echter niet noodzakelijk dat geschillen
over elkaar uitsluitende keuzes worden beslecht op een manier die
vereist dat de voorkeuren van grote aantallen mensen worden onderdrukt.

Sommige mensen eten graag foie gras, anderen hotdogs, en weer anderen
tofu. Ze hoeven meestal niet te twisten over hun voorkeuren voor eten,
omdat hun culinaire keuzes niet aan elkaar gebonden zijn. Niemand dwingt
iedereen om dezelfde maaltijd te eten. Megapolitieke omstandigheden
dwongen echter wel tot gemeenschappelijke consumptie van vele soorten
collectieve en zelfs private goederen die door regeringen in het
industriële tijdperk werden geleverd. Waarom? Omdat er grote economische
voordelen te behalen waren door op grote schaal te opereren. Het was
daarom onpraktisch om uitgestrekte jurisdicties op te delen in enclaves
waar iedereen zijn zin kon krijgen, zelfs wanneer het om belangrijke
zaken ging. De benadering die Adam Smith bepleitte om het aanbieden van
publieke goederen te beperken, kan veel gemakkelijker worden toegepast
wanneer het aantal jurisdicties tien- of zelfs honderdvoudig toeneemt.
In het Informatietijdperk zullen steeds meer soevereiniteiten kleine
enclaves zijn in plaats van continentale rijken. Sommigen kunnen
Noord-Amerikaanse indianenstammen zijn die een belastingjurisdictie
zullen claimen over hun reservaten, net zoals zij nu aanspraak maken op
het recht om casino's te exploiteren of om zonder beperkingen te vissen.

Omdat informatietechnologie veel nadelen van het opdelen van
handelsgebieden wegneemt, zal het voor de nieuwe soevereiniteiten
eenvoudiger zijn om meer volgens de principes van clubs of
affiniteitgroepen te opereren dan volgens die van territoriale
natiestaten. Net zoals dat het niet absoluut noodzakelijk is dat elke
potentiële klant dezelfde smaak in kleding heeft of dezelfde
televisieprogramma's bekijkt, zullen de affiniteitspunten die de
bestuursstijl van gefragmenteerde soevereiniteiten bepalen ook niet
langer door iedereen goed bevonden dienen te worden.

Een breed palet van smaken zal leiden tot sterk uiteenlopende stijlen
van gefragmenteerde soevereiniteit, net zoals er steeds bredere keuzes
zijn in kledingstijlen of televisie-uitzendingen. Sommige microstaten
kunnen zelfs met elkaar verbonden zijn zoals hotelketens in franchises,
of samenwerken om voordelen te behalen op het gebied van politietaken en
andere overblijvende overheidsdiensten. Degenen die van schone straten
houden en een hekel hebben aan kauwgom onder een tafel, zullen Singapore
aantrekkelijk vinden, fans van \emph{Beavis and Butthead} niet. Degenen
die van wild nachtleven houden, zullen Macao of Panama of een
soortgelijke plek verkiezen. Klanten die zich ongemakkelijk voelen met
de zeden in de ene jurisdictie zullen welkom zijn in andere. Terwijl
Salt Lake City wellicht rookvrij is, zal de nieuwe stadstaat in Havana,
misschien herdoopt tot Monte Cristo, waarschijnlijk gehuld zijn in een
wolk van sigarenrook.

\begin{quote}
\emph{``Het betekent dat alle monopolies, hiërarchieën, piramides en
machtsnetwerken van de industriële samenleving zullen breken onder deze
constante druk, veroorzaakt door het verspreiden van intelligentie naar
de randen van alle netwerken. Bovenal zal de wet van Moore de
belangrijkste concentratie, de belangrijkste fysieke samenballing van
macht in Amerika van vandaag omverwerpen: de grote stad, dat grote
geheel van industriële steden dat nu leeft op levensondersteunende
systemen, zo'n 360 miljard aan directe subsidies van ons allemaal elk
jaar. Grote steden zijn achtergebleven bagage uit het industriële
tijdperk.''287 -- GEORGE GILDER}
\end{quote}

Een bijzondere ironie van de heropkomst van micro-soevereiniteiten of
``stadstaten'' is dat dit kan samenvallen met het leeglopen van veel
steden. De grote stad was grotendeels een artefact van het
industrialisme in het Westen. Het fabriekssysteem deed steden opbloeien
om schaalvoordelen in de productie van goederen die gemaakt werden van
een grote hoeveelheid natuurlijke hulpbronnen te benutten.

Toen de negentiende eeuw begon, werden steden van meer dan 100.000 als
enorm beschouwd, en buiten Azië, waar bevolkingsstatistieken
twijfelachtig waren, waren er geen steden van meer dan een miljoen
inwoners. De grootste stad in de Verenigde Staten in 1800 was
Philadelphia, met een populatie van 69.403 mensen. New York had er
slechts 60.489. Baltimore was de derde grootste stad in Amerika met
26.114 inwoners.288 De meeste van wat later de grote metropolen van
Europa zouden worden, hadden bevolkingen die minuscuul zijn naar
twintigste-eeuwse maatstaven. Londen, met een bevolking van 864.845, was
waarschijnlijk de grootste stad ter wereld. Parijs, met 547.756, was de
enige andere stad in Europa met meer dan een half miljoen inwoners in
1801.289 Lissabon had 350.000 inwoners.290 Wenen had een bevolking van
252.000.291 Berlijn kwam pas in 1819 net boven de 200.000 uit.292 Madrid
was de thuisbasis van 156.670.293 In 1802 was de bevolking van Brussel
66.297. Boedapest had slechts 61.000 inwoners.294

Er is een duidelijke verleiding om te denken dat de groei van grote
steden een directe functie is van bevolkingsgroei. Maar dat is niet
noodzakelijk zo. Elke mens op aarde zou in Texas kunnen worden
ondergebracht, waarbij elk gezin in zijn eigen vrijstaand huis met tuin
zou wonen, en er zou nog steeds een deel van Texas overblijven. Zoals
Adna Weber betoogde in de klassieke studie \emph{The Growth of Cities in
the Nineteenth Century}, verklaart bevolkingsgroei op zichzelf niet
waarom mensen in stedelijke omgevingen wonen in plaats van verspreid op
het platteland. In 1890 had Bengalen (het huidige Bangladesh) ongeveer
dezelfde bevolkingsdichtheid als Engeland. Toch was de stedelijke
bevolking van Bengalen slechts 4,8 procent, terwijl die van Engeland
61,7 procent was.295

Historisch gezien waren steden door muren afgescheiden van het
platteland om plunderaars en de lagere klassen buiten te houden. De
groei van industriële werkgelegenheid in de negentiende en twintigste
eeuw creëerde grote steden. Nu industrialisme begint te vervagen, is de
grote stad uiterst kwetsbaar geworden voor ontwrichting. Deze
ontwikkeling wordt bij uitstek gesymboliseerd door Detroit, ooit de
belangrijkste industriële stad van de twintigste eeuw. Ooit passeerde
een groot deel van de industriële wereldproductie door Detroit. Nu is
het een uitgeholde schelp, geteisterd door misdaad en wanorde. In veel
straten van het centrum van Detroit zijn een of meer vervallen gebouwen
afgebrand of gesloopt, waardoor de indruk ontstaat dat de stad een reeks
bombardementen uit de Tweede Wereldoorlog heeft overleefd.

Detroit herinnert ons aan het feit dat veel industriële steden niet
langer levensvatbaar zijn. Nu informatie en ideeën belangrijker worden
voor waardecreatie dan fabricage uit natuurlijke hulpbronnen, zullen ze
in verval raken. In veel gevallen is de grote stad al te groot geworden
om haar eigen gewicht te dragen. Om een metropool te laten functioneren,
moeten een aanzienlijk aantal ondersteunende systemen efficiënt op grote
schaal kunnen opereren. Het samenbrengen van miljoenen mensen impliceert
een enorm verhoogde kwetsbaarheid voor misdaad, sabotage en willekeurig
geweld. Tijdens het industriële tijdperk werden de kosten om zich tegen
deze risico\textquotesingle s te beschermen terugbetaald met de
voordelen die productie op grote schaal opleverden.

In het Informatietijdperk overleven slechts steden die hun
onderhoudskosten rechtvaardigen door het bieden van een hoge kwaliteit
van leven. Personen op afstand zullen niet langer verplicht zijn hen te
subsidiëren. Een goede graadmeter voor de levensvatbaarheid van steden
is of degenen die in de kern van de stad wonen rijker zijn dan degenen
aan de rand. Buenos Aires, Londen en Parijs zullen aantrekkelijke
plaatsen blijven om te wonen en zaken te doen, lang nadat het laatste
goede restaurant sluit in South Bend, Louisville en Philadelphia.

\subsection{Country states}\label{country-states}

Sommige stadstaten zullen wellicht slechts enclaves blijken te zijn
zonder bijbehorende steden. Misschien kunnen ze beter worden beschouwd
als dorpstaten of landstaten.

Natuurlijke hulpbronnen zullen ook op verschillende manieren worden
gewaardeerd. Wanneer je overal zaken kunt doen, zou je ervoor kunnen
kiezen om dit te doen op een mooie plek waar je diep kunt ademhalen
zonder al te veel kankerverwekkende luchtvervuiling.
Communicatietechnologieën die taaldrempels minimaliseren, zullen het
steeds gemakkelijker maken om in bijna elke aantrekkelijke omgeving te
verblijven. Dunbevolkte regio's met gematigde klimaten en een grote
hoeveelheid landbouwgrond per hoofd van de bevolking, zoals
Nieuw-Zeeland en Argentinië, zullen ook een comparatief voordeel
genieten omdat zij hoge standaarden van volksgezondheid hebben en
goedkope producenten zijn van voedsel en hernieuwbare producten.
Dergelijke producten zullen profiteren van toenemende vraag naarmate de
levensstandaard van miljarden mensen in Oost-Azië en Latijns-Amerika
stijgt.

\subsection{Het
inequivalentietheorema}\label{het-inequivalentietheorema}

Veel aannames van economen over menselijk gedrag zijn geworteld in de
tirannie van plaats. Een duidelijk voorbeeld is
Ricardo\textquotesingle s "Equivalentietheorema", dat stelt dat burgers
in een land met hoge begrotingstekorten hun verwachtingen aanpassen in
afwachting van hogere belastingen die in de toekomst nodig zullen zijn
om de schuld af te lossen. In die zin bestaat er een ``equivalentie''
tussen het financieren van uitgaven via belastingen en via schulden. Dat
was althans zo in het begin van de negentiende eeuw, toen Ricardo dit
schreef. In het Informatie­tijdperk zal de rationele mens echter niet
reageren op het vooruitzicht van hogere belastingen door meer te sparen,
maar hij zal zijn domicilie verplaatsen of zijn transacties elders
onderbrengen. Om dezelfde reden dat producenten leveranciers selecteren
op basis van de laagste kosten, zullen zij nog sterker gemotiveerd zijn
om alternatieve aanbieders van bescherming te zoeken. De baten hiervan
zullen de marges van een overstap naar een nieuwe leverancier van
bijvoorbeeld kunststofbuizen verre overtreffen. Het te verwachten gevolg
is dat \emph{Soevereine Individuen} en andere rationele mensen zullen
vluchten uit jurisdicties met grote, niet-gedekte verplichtingen.

Goedkope overheden met weinig verplichtingen en lage lasten voor hun
``klanten'' zullen in het Informatie­tijdperk bij uitstek de domicilies
voor vermogensvorming zijn. Dit impliceert veel aantrekkelijkere
vooruitzichten voor het ondernemen in gebieden waar de schuldenlast laag
is en waar overheden al zijn hervormd, zoals Nieuw-Zeeland, Argentinië,
Chili, Peru, Singapore en andere delen van Azië en Latijns-Amerika. Deze
regio's zullen bovendien superieure platforms zijn voor ondernemerschap
in vergelijking met niet-hervormde, dure economieën in Noord-Amerika en
West-Europa.

\subsection{De afname van lokale
prijsverschillen}\label{de-afname-van-lokale-prijsverschillen}

Zeer lage informatiekosten doen de meeste lokale prijsvoordelen
verdwijnen. Kopers doorzoeken immers talloze verkooppunten om de laagste
prijzen voor verhandelbare goederen te vinden en schakelen daarbij
externe diensten in om over landsgrenzen heen te winkelen. Hierdoor zal
het voor mensen veel eenvoudiger zijn om de eigenschappen van complexe
producten, zoals verzekeringen, met elkaar te vergelijken. Daarbij
omzeilt dit de handelsbeperkingen die lokale vergunningsprocedures met
zich meebrengen. Naar verwachting zorgen extra informatie en
intensievere concurrentie ervoor dat lokale prijsverschillen verdwijnen,
wat resulteert in lagere winstmarges.

\section{\texorpdfstring{\textbf{Nieuwe organisatorische
verplichtingen}}{Nieuwe organisatorische verplichtingen}}\label{nieuwe-organisatorische-verplichtingen}

De cybereconomie zal wezenlijk van de industriële economie verschillen
in de manier waarop de deelnemers met elkaar omgaan.
Informatie­technologie zal veel van de langdurige organisatorische
voordelen van bedrijven, die voortkomen uit hoge transactie- en
informatiekosten, doen verdwijnen. Het Informatie­tijdperk zal het
tijdperk van de ``virtuele onderneming'' zijn.

Veel analisten, die meer kennis van informatie­technologie hebben dan
wij, hebben volledig gemist dat deze technologie de logica van
economische organisatie zal transformeren. De nieuwe technologie
overschrijdt niet alleen grenzen en barrières, ze betekent ook een
revolutie voor de ``interne'' kosten van rekenkracht. Zelfs het kleine
aantal bedrijven dat niet direct wordt beïnvloed door toenemende
grensoverschrijdende concurrentie dankzij verbeterde informatie- en
communicatietechnologie zal worden geconfronteerd met nieuwe
organisatorische eisen. Snel dalende informatie- en transactiekosten
zullen de schaalvoordelen drastisch verlagen, waardoor veel van de
prikkels die tijdens de industriële periode leidden tot langlevende
bedrijven en carrièremogelijkheden, teniet worden gedaan.

\subsection{Waarom bedrijven?}\label{waarom-bedrijven}

Klassieke economen als Adam Smith besteedden nauwelijks aandacht aan de
vraag welke omvang bedrijven idealiter zouden moeten hebben. Ze gingen
niet in op wat de optimale grootte bepaalt, waarom ondernemingen de
vormen aannemen die zij aannemen, of zelfs waarom ze überhaupt bestaan.
Waarom kiezen ondernemers ervoor om medewerkers in dienst te nemen in
plaats van elke benodigde taak via een veiling aan onafhankelijke
aannemers uit te besteden? De Nobelprijswinnaar Ronald Coase stippelde
een nieuwe koers in de economie uit door enkele van deze fundamentele
vragen te adresseren. Zijn bevindingen tonen duidelijk de revolutionaire
impact van informatietechnologie op de structuur van bedrijven. Coase
betoogde dat ondernemingen een efficiënte manier waren om het gebrek aan
informatie en hoge transactiekosten te overbruggen.296

\subsection{Informatie- en
transactiekosten}\label{informatie--en-transactiekosten}

Om te begrijpen waarom, stel je eens voor met welke obstakels je te
maken krijgt als je een assemblagelijn uit het industriële tijdperk in
gang wilt zetten zonder dat één onderneming de activiteiten coördineert.
In principe had een auto geproduceerd kunnen worden zonder dat de
productie werd gecentraliseerd onder toezicht van één firma. Econoom
Oliver Williamson, samen met Coase, was een pionier in de ontwikkeling
van de firmatheorie. Williamson definieerde zes verschillende methoden
van werking en controle. Een daarvan is de ``ondernemersmodus,'' waarin
elk werkstation wordt beheerd en geëxploiteerd door een specialist. Een
andere noemt Williamson de ``gefedereerde werkstations,'' waarbij een
tussenproduct door elke werknemer van de ene naar de andere fase wordt
overgedragen. Er is geen fysieke reden waarom de duizenden werknemers
niet vervangen hadden kunnen worden door een groep onafhankelijke
aannemers, die elk een ruimte op de fabrieksvloer huren, bieden op
onderdelen, en aanbieden om de as te monteren of de spatborden aan het
chassis te lassen. Toch zou je tevergeefs zoeken naar een voorbeeld van
een autofabriek uit het industriële tijdperk die door onafhankelijke
aannemers georganiseerd en gerund werd.

\subsection{Coördinatieproblemen}\label{couxf6rdinatieproblemen}

Het runnen van een industriële faciliteit zonder de coördinatievoordelen
van één centrale organisatie zou vrijwel alle schaalvoordelen die bij
grootschalige exploitatie mogelijk zijn, doen verdwijnen. Het
coördineren van een wirwar aan kleine bedrijven brengt enorme
transactionele problemen met zich mee, waardoor de assemblagelijn
praktisch onbruikbaar wordt. Om het systeem überhaupt werkend te houden,
zou het nodig zijn om voortdurend te onderhandelen met de diverse
aannemers. In plaats van zich op de productie te concentreren, zouden ze
hun tijd moeten besteden aan het bepalen van de prijzen voor onderdelen
en het vastleggen van de voorwaarden voor hun voortdurend wisselende
samenwerkingen. Alleen al het monitoren van de productie zou een
moeilijk probleem geweest zijn.

\subsection{De bevoegdheid om op te
treden}\label{de-bevoegdheid-om-op-te-treden}

Als zo'n netwerk van onafhankelijke organisaties al moeite heeft om een
auto in elkaar te zetten, dan zou het ontwikkelen en herontwerpen van
modellen een regelrechte nachtmerrie vormen. Stel je eens voor met welke
problemen een ontwerper zou worden geconfronteerd wanneer hij de
honderden zelfstandige aannemers moet overtuigen van de noodzakelijke
wijzigingen om een nieuw model te introduceren. In de praktijk vereist
dat vrijwel unanieme instemming. Wie terughoudend is of bezwaar maakt
tegen een enkele aanpassing in de productspecificaties, zou de
modelverbetering praktisch kunnen blokkeren of de introductiekosten doen
stijgen, waardoor de voordelen van grootschalige exploitatie nog verder
in gevaar komen.

\subsection{Onnodige onderhandelingen}\label{onnodige-onderhandelingen}

Een assemblagelijn die door onafhankelijke aannemers werd gehuurd (of in
afzonderlijk bezit was) zou onderhevig zijn geweest aan tal van
kwetsbaarheden die vermeden werden door binnen één enkel bedrijf te
opereren. Het overlijden, de ziekte of het financiële falen van
individuele aannemers zou een veel te vaak voorkomend probleem zijn
geweest in operaties die de samenwerking van duizenden mensen vereisten
om onder één dak een enkel product te bouwen. De veilingmarkt zou deze
aannemers zeker hebben kunnen vervangen, maar elke opvolging zou een
onderhandelde regeling hebben vereist, zoals een uitkoop van de vorige
exploitant door zijn vervanger. Ook zou het een overeenkomst hebben
vereist over de overname van de huur van de fabrieksruimte, en misschien
een nieuw leasecontract voor de lasmachine of de pers die werd gebruikt
voor het stansen van de achterlichtfittingen. Dit alles zou ingewikkeld
zijn geweest.

\subsection{Perverse prikkels}\label{perverse-prikkels}

Een andere cruciale moeilijkheid bij een assemblagelijn van
onafhankelijke aannemers onder de omstandigheden van het industriële
tijdperk was dat de kapitaalbehoeften voor de individuele aannemers
sterk zouden verschillen. Een mal die nodig was om een kunststoffen
schakelaar van het dashboard te produceren zou bijvoorbeeld relatief
goedkoop kunnen zijn, terwijl de apparatuur die nodig is om een
motorblok te gieten of het plaatwerk van een spatbord te stansen
miljoenen zou kunnen kosten. De grote hoeveelheid grondstoffen en de
sequentiële aard van assemblagelijnproductie maakten problemen door hoge
kapitaalkosten onvermijdelijk om redenen die in het vorige hoofdstuk
zijn geanalyseerd. Aannemers met kapitaalintensieve taken zouden in
wezen afhankelijk zijn geweest van de medewerking van anderen om hun
investeringen af te schrijven. Aannemers met grote kapitaaleisen konden
alleen winst maken en financiering krijgen als zij de samenwerking van
andere, minder kapitaalintensieve deelnemers wisten te verzekeren. In
veel gevallen zouden ze die medewerking niet hebben gekregen.

Er zou een substantiële prikkel zijn geweest voor de kleintjes om de
groten uit te buiten. Wie minder kapitaal nodig had voor hun taak op de
assemblagelijn, kon voordeel behalen door op belangrijke momenten niet
mee te werken. Net als stakende arbeiders konden ze de productie
stilleggen, waardoor zij nauwelijks verlies leden maar de degenen met
grote investeringen wel. Het productieproces zou voortdurend worden
beïnvloed door strategisch gedrag, waarbij kleine aannemers degenen met
hogere kapitaalkosten blootstelden aan gijzeling via hun vermogen om de
productie te dwarsbomen. Het gedrag van kleinere aannemers om extra geld
van de groten af te dwingen zou de efficiëntie van het systeem hebben
verminderd.

\subsection{The firm solution}\label{the-firm-solution}

Kortom, een groot deel van de efficiëntie die tijdens het industriële
tijdperk werd behaald door een assemblagelijn op grote schaal te
exploiteren, zou zijn verloren gegaan als de productie was verdeeld over
talloze individuele aannemers. Het enkele grote bedrijf was een
efficiënte manier om deze nadelen te overwinnen, ondanks de andere
beperkingen. Grote bedrijven waren bureaucratisch, maar tot op zekere
hoogte waren bureaucratie en hiërarchie precies wat nodig was tijdens
het Industriële Tijdperk. Administratieve en managementteams hielden
toezicht op en coördineerden de productie, waarbij talrijke
middenmanagers opdrachten naar beneden gaven en andere informatie terug
de hiërarchie in stuurden. De corporatieve bureaucratie zorgde ook voor
boekhoudkundige controles en minimaliseerde problemen, waarbij
werknemers niet handelen in het beste belang van het bedrijf dat hen in
dienst heeft. Om geavanceerde boekhouding te realiseren onder de
omstandigheden van het industriële tijdperk, was het werk van vele
mensen nodig. Het hebben van een dergelijke administratieve bureaucratie
was kostbaar. Deze moest betaald worden, of er nou actief geproduceerd
werd of niet. Omdat zulke beheerders over cruciale kennis beschikten die
nodig was om het bedrijf te runnen, kregen zij doorgaans een hoger
salaris dan hun vaardigheden op de vrije markt zouden opleveren.

\subsection{Organisational slack}\label{organisational-slack}

Het grote aantal professionele managers en administrateurs had ook het
nadeel dat zij de neiging hadden het bedrijf ``over te nemen'' en in hun
eigen belang te laten opereren in plaats van in dat van de
aandeelhouders. Het was in het industriële tijdperk bijvoorbeeld niet
ongewoon om bedrijven te zien die royale bedragen spendeerden aan
kantoorinrichting, lidmaatschappen van clubs en andere voordelen die
door het management konden worden genoten, maar die mogelijk geen direct
rendement voor investeerders opleverden.

In een complex bedrijf was het van buitenaf onmogelijk om gemakkelijk te
controleren welke overheadkosten essentieel waren en welke louter luxe
waren voor werknemers. Ook was het moeilijk te voorkomen dat een soms
aanzienlijk deel van de bedrijfsmedewerkers de kantjes er vanaf liep.
Het feit dat het technologisch lastig was om prestaties te monitoren,
maakte een groot middenmanagement noodzakelijk, en tegelijk maakte het
het moeilijk om de toezichthouders zelf te controleren.

Al deze omstandigheden droegen bij aan wat bekend werd als
``organisational slack,'' een term bedacht in 1963 door Richard Cyert en
James March in \emph{A Behavioral Theory of the Firm}. Zorgvuldig
onderzoek wees uit dat talrijke echte bedrijven substantieel
onderpresteerden.

\begin{quote}
\emph{``Of je nu resultaten behaalt of niet, het loon is hetzelfde. Of
je nu hard werkt of niet, het loon is hetzelfde. Of het je iets kan
schelen of niet, het loon is hetzelfde.'' -- CHRIS DRAY}
\end{quote}

\subsection{``Dat is mijn werk niet''}\label{dat-is-mijn-werk-niet}

Als een naar permanentie strevende entiteit had het grote industriële
bedrijf het nadeel dat het, zoals we al zagen, kwetsbaar was voor
afpersing door vakbonden. Het deelde ook enkele kenmerken van
bureaucratie, zoals we in nog sterkere mate zagen in overheidskantoren.
Bevelen kwamen van bovenaf. Taken waren gestandaardiseerd en in
compartimenten verdeeld. Deze taken waren vaak rigide gedefinieerd. Er
ontstonden grenzen tussen functiecategorieën, vergelijkbaar met degene
die werden afgedwongen door de kartels die de vrije beroepen
reguleerden. Het idee dat een boekhouder zelf het kapotte lampje in zijn
bureaulamp zou vervangen, leek in het industriële tijdperk voor velen
even vreemd als het inschakelen van een advocaat om je van griep te
genezen. Er werd niet van werknemers verwacht, en in veel gevallen was
het zelfs niet toegestaan om de grenzen tussen strikt afgebakende
functies te overschrijden.

``Dat is mijn werk niet'' was een veelgehoorde slogan die de
``organisational slack'' van het industriële tijdperk onderstreepte.
Ieders werk was exact omschreven in termen van gestandaardiseerde taken
waar niet van mocht worden afgeweken, hoezeer dat de productiviteit ook
had kunnen verbeteren. Elke werknemer in de bedrijfsbureaucratie werd
aangenomen op basis van ``kwalificaties'' die geacht werden prestaties
in zijn specifieke functie te voorspellen. Met weinig uitzonderingen
werd iedereen betaald volgens zijn functie, met min of meer uniforme
beloning binnen de hele organisatie. Omdat specifieke prestaties binnen
de administratieve hiërarchie van Big Business vaak niet werden gemeten,
net als in staatsbureaucratieën, verliep het werk in een rustig tempo.
Het bedrijf wist dus wel de schaalvoordelen van massaproductie te
benutten, maar droeg daarvoor wel de lasten van andere inefficiënties.

\begin{quote}
\emph{``In een markt doe je iets niet omdat iemand je dat opdraagt of
omdat het op pagina dertig van het strategisch plan staat. Een markt
kent geen taakomschrijvingen. \ldots{} Er zijn geen bevelen, geen
signalen van bovenaf die vertaald moeten worden, niemand die het werk in
pakketjes verdeelt. In een markt heb je klanten, en de relatie tussen
leverancier en klant is fundamenteel niet-organisatorisch, omdat het
gaat om twee onafhankelijke entiteiten.'' -- WILLIAM BRIDGES}
\end{quote}

\subsection{Nieuwe verplichtingen}\label{nieuwe-verplichtingen}

De nieuwe megapolitieke omstandigheden van het Informatietijdperk zullen
de logica van bedrijfsorganisatie aanzienlijk veranderen. Een deel
daarvan is duidelijk. Als informatietechnologie iets doet, dan is het
wel het drastisch verlagen van de kosten van het verwerken, berekenen en
analyseren van informatie. Eén gevolg van deze technologie is dat de
noodzaak om grote aantallen middenmanagers in te huren om
productieprocessen te monitoren, sterk afneemt. Geavanceerde
computergestuurde werktuigen vervangen in veel gevallen al werknemers
die per uur betaald krijgen. En waar het productieproces nog steeds door
mensen wordt bemand, is het controle- en coördinatieproces grotendeels
geautomatiseerd. Apparatuur met microprocessors kan de voortgang van de
lopende band veel beter monitoren dan managers ooit konden. Niet alleen
kan de nieuwe apparatuur de snelheid en nauwkeurigheid meten waarmee
mensen werken, het kan ook automatisch de boekhouding bijhouden en
onderdelen opnieuw bestellen zodra ze uit de voorraad worden gehaald.
Zelfs de kleinste bedrijven kunnen nu financiële controleprogramma's
betalen die hun financiën sneller en geavanceerder beheren dan zelfs de
grootste ondernemingen enkele decennia geleden konden bereiken via hun
productiehiërarchieën.

Het feit dat informatietechnologie verspreide, niet-sequentiële
productie met minder grondstoffen mogelijk maakt, verkleint de
kwetsbaarheid voor manipulatie en afpersing drastisch, zoals we al
hebben gezien. Maar dit zijn niet de enige kenmerken van
informatietechnologie die het steeds aantrekkelijker maken om functies
uit te besteden die voorheen door werknemers werden uitgevoerd.
Kapitaalkosten zijn lager en productcycli zijn korter. De onafhankelijke
contractanten zelf, inclusief de eenpersoonsbedrijven, hebben de
beschikking over veel geavanceerdere informatienetwerken. Binnenkort
zullen zij kunnen vertrouwen op een reeks digitale assistenten om een
breed scala aan kantoorfuncties uit te voeren, van telefoondiensten tot
secretariaatswerk. Digitale assistenten zullen optreden als
secretaresses, reclameadviseurs, reisagenten, bankloketten en
bureaucraten.

\subsection{Het verdwijnen van goede
banen}\label{het-verdwijnen-van-goede-banen}

In toenemende mate zullen individuen die in staat zijn om een grote
hoeveelheid economische waarde te creëren, het grootste deel van die
waarde voor zichzelf kunnen behouden. Ondersteunend personeel dat
voorheen een groot deel van de opbrengsten van de belangrijkste
waardecreërende krachten in een onderneming opslokte, zal worden
vervangen door goedkope geautomatiseerde informatiesystemen. Dit houdt
in dat een organisatie de hoogste kwaliteit van dienstverlening beter
kan garanaderen door deze uit te besteden, in plaats van de functie
binnen het bedrijf te behouden, waar het relatief moeilijker zal zijn om
individuen goed te belonen voor hun prestaties. Een virtuele onderneming
zal de meeste ``organisational slack'' elimineren door de organisatie
zelf te elimineren.

``Goede banen'' zullen tot het verleden behoren. Een ``goede baan,''
zoals Princeton-econoom Orly Ashenfelter het stelde, ``is een baan die
meer betaalt dan je waard bent.'' In het Industriële Tijdperk bestonden
er veel ``goede banen'' vanwege hoge informatie- en transactiekosten.
Bedrijven groeiden en internaliseerden een bredere waaier aan functies
omdat ze zo schaalvoordelen konden benutten. Corporatief overgewicht
werd eveneens gesubsidieerd door belastingwetten. In de latere
industriële periode versterkten hoge belastingen kunstmatig de
aantrekkelijkheid om langlevende bedrijven te vormen met vaste
werknemers. In de meeste landen verhoogden belastingwetten en -regels
aanzienlijk de kosten voor het oprichten en ontbinden van bedrijven op
projectbasis. Ze dwongen ondernemers er bovendien vaak toe om
onafhankelijke contractanten als werknemers aan te nemen. Juridische
interventies bliezen tijdelijk het aanbod van ``goede banen'' verder op
door het kostbaar en moeilijk te maken een werknemer te ontslaan,
ongeacht hoe weinig hij bijdroeg aan de productiviteit van het bedrijf.

Het was dan ook onvermijdelijk en logisch dat de aard van de
bedrijfsorganisatie in het industriële tijdperk ervoor zorgde dat de
meest bekwame en getalenteerde mensen, die een onevenredig groot deel
van de toegevoegde waarde in een onderneming creëerden, relatief minder
werden betaald dan hun bijdrage waard was. Dit zal veranderen in het
Informatietijdperk.

De microprocessorrevolutie verbetert de beschikbaarheid van informatie
enorm en verlaagt transactiekosten. Dit devolueert de firma. In plaats
van permanente bureaucratie zullen activiteiten rond projecten worden
georganiseerd, zoals filmmaatschappijen dat al doen. De meeste voorheen
``interne'' functies van het bedrijf zullen worden uitbesteed aan
onafhankelijke contractanten. Werknemers uit het industriële tijdperk
die ``goede banen'' hadden maar weinig bijdroegen en op collega's
vertrouwden om hen te ``dekken,'' zullen zichzelf spoedig aanbieden voor
contracten op de spotmarkt. En hetzelfde geldt voor vele loyale,
ijverige werknemers. ``Goede banen'' zullen een anachronisme zijn omdat
banen in het algemeen een anachronisme worden.

In het extreme geval van grote Japanse ondernemingen verwachtten
werknemers een baan voor het leven te hebben. Zelfs wanneer zij geen
productieve taak te vervullen hadden, bleven zij in dienst, soms slechts
aanwezig achter ``een kaal bureau in de hoek van een fabriek.'' Nu wordt
zelfs in Japan het opgeblazen witteboordenapparaat afgeslankt. De kop
van een artikel in de \emph{International Herald-Tribune} vertelde het
verhaal: ``Bitter Afscheid: Het Pijnlijke Verval van Japans
Baan-voor-het-Leven-Cultuur.''

In de postindustriële periode zullen banen taken zijn die je doet, niet
iets dat je ``hebt.'' Voor het industriële tijdperk was permanente
werkgelegenheid vrijwel onbekend. Zoals William Bridges stelde: ``Voor
1800, en in veel gevallen nog lang daarna, verwees een job altijd naar
een bepaalde taak of onderneming, nooit naar een rol of positie in een
organisatie. \ldots{} Tussen 1700 en 1890 registreert de Oxford English
Dictionary veelvuldig het gebruik van termen als \emph{job-coachman},
\emph{job-doctor} en \emph{job-gardener}, allemaal verwijzend naar
mensen die op eenmalige basis werden ingehuurd. \emph{Jobwork} (een
andere veel voorkomende term) was occasioneel werk, geen vaste
werkgelegenheid.'' In het Informatietijdperk zullen de meeste taken die
voorheen binnen bedrijven werden opgevangen als middel om informatie- en
transactiekosten te verlagen, terug migreren naar de spotmarkt. ``Just
in time''-voorraadbeheer en outsourcing zijn beide praktisch geworden
door informatietechnologie. Het zijn stappen in de richting van de dood
van banen. Grote ondernemingen zoals AT\&T hebben inmiddels al alle
vaste functiecategorieën afgeschaft. Posities in dat grote bedrijf zijn
nu voorwaardelijk. In de woorden van Bridges: ``Werkgelegenheid wordt
opnieuw tijdelijk en situationeel, en categorieën verliezen hun
grenzen.'' In de nieuwe cybereconomie zullen ``onafhankelijke
contractanten'' via telewerken continenten overstijgen en zich groeperen
rond het equivalent van de lopende band van het Informatietijdperk.

\subsection{Hollywood neemt het over}\label{hollywood-neemt-het-over}

Het model voor de organisatie van bedrijven zou in de nieuwe
informatiemaatschappij een filmproductiemaatschappij kunnen zijn.
Dergelijke ondernemingen kunnen zeer geavanceerd zijn, met budgetten van
honderden miljoenen dollars. Hoewel het vaak grote operaties zijn, zijn
ze tijdelijk van aard. Een filmmaatschappij die een film produceert voor
\$100 miljoen kan zich een jaar lang vormen en daarna weer ontbinden. De
mensen die aan de productie werken zijn getalenteerd, maar ze verwachten
niet dat werk aan het project gelijkstaat aan een ``vaste baan.''
Wanneer het project voorbij is, gaan de lichttechnici, cameramensen,
geluidstechnici en kostuumspecialisten hun eigen weg. Ze kunnen in een
volgend project weer bij elkaar komen, of ook niet.

Naarmate schaalvoordelen afnemen en kapitaalvereisten voor veel soorten
informatie-intensieve activiteiten tegelijkertijd dalen, zal er een
sterke prikkel zijn voor bedrijven om te ontbinden. Bedrijfsactiviteiten
zullen ad-hoc en tijdelijk zijn. Bedrijven zullen over het algemeen
korter bestaan. Virtuele ondernemingen die talenten voor specifieke
doelen samenbrengen zullen efficiënter zijn dan langdurige bedrijven.
Naarmate encryptie breed gebruikt wordt en de belasting op kapitaal door
concurrentie wordt gedrukt, zullen kunstmatige schaalvoordelen die het
bestaan van ``permanente'' bedrijven ondersteunen, verdwijnen. Dit zal
gebeuren, ongeacht of belastingen snel of langzaam worden verlaagd. Als
dit snel gebeurt, verdwijnen de kunstmatige kosten van projectmatig
functioneren sneller. Bij een trage verlaging dragen bestaande bedrijven
nog steeds de last van anachronistisch hoge belastingen, terwijl nieuwe
ondernemingen als virtuele bedrijven opereren en beter in staat zijn om
dure lasten van de stervende natiestaat te vermijden.

Hoewel speciale vaardigheden en talenten belangrijker dan ooit zullen
zijn in de informatiemaatschappij, zullen de meeste kunstmatige grenzen
tussen beroepen verdwijnen. Geavanceerde informatie- en
opslagtechnologieën zullen de bedrijfsgeheimen en gespecialiseerde
informatie van beroepen zoals recht, geneeskunde en boekhouding
toegankelijk maken voor iedereen. De economische waarde van memorisatie
als vaardigheid zal afnemen, terwijl het belang van synthese en
creatieve toepassing van informatie zal toenemen.

De volledige implicaties van deze verandering zullen worden vertraagd
door verouderde regelgeving. Maar op de lange termijn zal de macht van
overheden om de cybereconomie te reguleren tot vrijwel nul afnemen. Elke
kunstmatige regulering van professionele monopolies die kosten verhoogt
zonder marktwaarde te leveren, zal uiteindelijk worden genegeerd.

Er zijn nog andere implicaties van de verschuiving naar een
informatiemaatschappij:

\begin{itemize}
\item
  Lokale regulering die hogere kosten oplegt, zal worden \textgreater{}
  getransformeerd naar marktconforme systemen.
\item
  Concurrentie tussen rechtsgebieden zal toenemen om hoogwaardige
  \textgreater{} activiteiten aan te trekken die in principe overal
  gevestigd \textgreater{} kunnen worden. Geen enkele locatie is
  noodzakelijkerwijs \textgreater{} aantrekkelijker dan de volgende.
\item
  Zakelijke relaties zullen steeds meer steunen op \textgreater{}
  ``vertrouwenskringen.'' Doordat encryptie individuen in staat stelt
  \textgreater{} om onopgemerkt te stelen, zal eerlijkheid een hoog
  gewaardeerde \textgreater{} eigenschap van zakenpartners worden.
\item
  Octrooi- en auteursrechtssystemen zullen veranderen door de
  \textgreater{} gemakkelijke toegang tot bepaalde informatie.
\item
  Bescherming zal steeds meer technologisch dan juridisch zijn. Lagere
  \textgreater{} klassen zullen worden buitengesloten. De beweging naar
  afgesloten \textgreater{} gemeenschappen is vrijwel onvermijdelijk.
  Mensen die last \textgreater{} veroorzaken buitenhouden is een
  effectieve, en traditionele, \textgreater{} manier om crimineel geweld
  te minimaliseren in tijden met zwak \textgreater{} centraal gezag
\item
  Bulkgoederen zullen zwaar belast worden en lokaal worden vervoerd,
  \textgreater{} terwijl luxegoederen licht belast zullen worden en over
  grote \textgreater{} afstand zullen worden getransporteerd.
\item
  Politiefuncties zullen steeds meer door particuliere beveiligers
  \textgreater{} worden uitgevoerd, gekoppeld aan handelsverenigingen.
\item
  Private bedrijven kunnen tijdelijk voordeel hebben tegenover
  \textgreater{} beursgenoteerde bedrijven omdat zij meer vrijheid
  genieten om \textgreater{} kosten van overheden te ontwijken.
\item
  Banen voor het leven zullen verdwijnen, aangezien ``banen'' steeds
  \textgreater{} vaker taakgericht of projectwerk worden in plaats van
  posities \textgreater{} binnen een organisatie.
\item
  Controle over economische middelen zal verschuiven van de staat naar
  \textgreater{} personen met superieure vaardigheden en intelligentie,
  naarmate \textgreater{} het steeds makkelijker wordt om vermogen te
  creëren door kennis \textgreater{} aan producten toe te voegen.
\item
  Veel leden van geleerde beroepen zullen worden vervangen door
  \textgreater{} interactieve informatie-opzoeksystemen.
\item
  Naarmate inkomensongelijkheid binnen rechtsgebieden toeneemt, zullen
  \textgreater{} voor mensen met lagere intelligentie nieuwe
  overlevingsstrategieën \textgreater{} ontstaan, gericht op
  vrijetijdsvaardigheden, sportieve \textgreater{} bekwaamheden en
  criminaliteit, evenals dienstverlening aan de \textgreater{} groeiende
  groep Soevereine Individuen.
\end{itemize}

Politieke systemen die ontstonden toen geweld hoge rendementen
opleverde, moeten ingrijpende aanpassingen ondergaan. Nu efficiëntie
belangrijker wordt in verhouding tot de macht van een systeem, zullen
kleine, efficiënte soevereiniteiten, die meer bescherming bieden tegen
lagere kosten, steeds beter houdbaar zijn.

Net als in de middeleeuwen ontstaan er opnieuw toenemende schaalnadelen
bij het uitoefenen van geweld. Dit blijkt al uit het groeiende aantal
soevereine entiteiten sinds de val van het communisme. Het aantal
soevereiniteiten in de wereld zal naar verwachting snel toenemen
naarmate de logica van het Informatietijdperk door ervaring wordt
bevestigd.

Macht zal opnieuw op kleine schaal worden uitgeoefend. Bij het bieden
van aantrekkelijke voorwaarden voor soevereiniteit aan hun ``afnemers'',
zullen enclaves en provincies zelfs substantieel voordeel kunnen ervaren
ten opzichte van uitgestrekte landen. Dit zal heel anders zijn dan de
snel stervende moderne periode, waarin geen enkele entiteit kon
overleven zonder militaire macht, krachtig genoeg om een koninkrijk te
beheersen. Vroeger, toen er schaalnadelen gepaard gingen met het
uitoefenen van macht, hadden degenen die het meest profiteerden van
bescherming, zoals de rijke kooplieden in de late middeleeuwse
stadstaten, controle over de overheid. Naar onze mening kan iets
dergelijks opnieuw gebeuren. De verlaging van roofzuchtige lasten en
efficiëntere verdeling van middelen zal leiden tot snelle groei in
gebieden waar klanten daadwerkelijk controle uitoefenen over de lokale
soevereiniteiten.

Of deze ontwikkelingen kunnen of moeten doorgaan ondanks oppositie van
vele verliezers, zal, zoals we hierna verkennen, tot een van de
belangrijkste controverses van het Informatietijdperk behoren.

{[}1{]} Benjamin Schwarz, `American Inequality: Its History and Scary
Future', \emph{New York Times}, 19 december 1995, p.~A25.

{[}2{]} Adna Ferrin Weber, `The Growth of Cities in the Nineteenth
Century' (New York: \emph{Macmillan}, 1899; herdrukt door \emph{Cornell
University Press}, 1963), p.~249.

{[}3{]} Bill Bryson, `The Lost Continent' (New York: \emph{Harper
Perennial}, 1989), p.~72.

{[}4{]} Dit artikel is herdrukt in vol.~4 van de verzameling `Early
Mathematical Economists' van Adrian Darnell (6 delen, London:
\emph{Pickering \& Chatto}, 1991).

{[}5{]} Clive Jenkins en Barrie Sherman, The Collapse of Work (London:
Methuen, 1979), p.~103.

{[}6{]} Robert H. Frank en Philip J. Cook, The Winner-Take-All Society
(New York: The Free Press, 1995).

{[}7{]} Clay Chandler, `Buchanans succes schrikt het bedrijfsleven af',
\emph{Washington Post}, 22 februari 1996, p.~D12.

{[}8{]} Zie Mancur Olson, `Diseconomies of scale and development,'
\emph{Cato Journal}, vol.~7, nr. 1 (lente/zomer 1987).

{[}9{]} Ibid.

{[}10{]} Basil Davidson, \emph{The Black Mans' Burden: Africa and the
Curse of the Nation State} (New York: Times Books, 1992), p.~290.

{[}11{]} Olson, op. cit.

{[}12{]} Adam Smith, \emph{The Wealth of Nations}, p.~724. Dit punt werd
gesuggereerd door een argument van Edwin G. West in \emph{Adam Smith and
Modern Economics} (Aldershot, Engeland: Edward Elgar Publishing, 1990),
pp.~88--89.

{[}13{]} Fritz Rorig, \emph{The Medieval Town} (Berkeley: University of
California Press, 1967), p.~28.

{[}14{]} Albert O. Hirschman, `Exit, Voice, and Loyalty' (Cambridge:
\emph{Harvard University Press}, 1969), p.~81.

{[}15{]} Tom Peters en George Gilder, `City vs.~country: debat van Tom
Peters en George Gilder over de impact van technologie op locatie',
\emph{Forbes}, februari 1995.

{[}16{]} Weber, op. cit., p.~21.

{[}17{]} Ibid., p.~46 voor Londen, p.~73 voor Parijs.

{[}18{]} Ibid., p.~120.

{[}19{]} Ibid., p.~95.

{[}20{]} Ibid., p.~84.

{[}21{]} Ibid., p.~119.

{[}22{]} Ibid., p.~101.

{[}23{]} Ibid., p.~5.

{[}24{]} Zie Ronald Coase, \emph{The Nature of the Firm}, herdrukt in
Louis Putterman en Randall S. Kroszner (red.), \emph{The Economic Nature
of the Firm: A Reader}, 2e editie (Cambridge: \emph{Cambridge University
Press}, 1996), pp.~89--104.

{[}25{]} Geciteerd door West, op. cit., p.~58; zie ook Oliver E.
Williamson, \emph{The Organization of Work: A Comparative Institutional
Assessment}, \emph{Journal of Economic Behaviour and Organisation},
vol.~1, nr. 1.

{[}26{]} Geciteerd door West, op. cit., p.~59; zie eveneens Williamson,
op. cit.

{[}27{]} Richard Cyert en James March, \emph{A Behavioral Theory of the
Firm} (Englewood Cliffs, N.J.: \emph{Prentice-Hall}, 1983).

{[}28{]} Chris Dray, `Civil servants lead lives of quiet collusion',
\emph{Globe and Mail}, 2 februari 1996, p.~A14.

{[}29{]} William Bridges, Jobsh ft: How to Prosper in a Workplace
Without Jobs (Reading, Mass.: \emph{Addison-Wesley}, 1994), pp.~62, 64.

{[}30{]} Zie Al Ehrbar, `Re-engineering geeft bedrijven nieuwe
efficiëntie, werknemers de ontslagbrief,' \emph{Wall Street Journal}, 22
juli 1992, p.~A14, geciteerd door Bridges, op. cit., p.~39.

{[}31{]} Sheryl WuDunn, `Parting Is Such Sour Sorrow: Japan's
Job-for-Life Culture Painfully Expires,' \emph{International Herald
Tribune}, 13 juni 1996, p.~13.

{[}32{]} Bridges, op. cit., pp.~31--32.

{[}33{]} Ibid., p.~58.

{[}34{]} Abu-Lughod, op. cit., p.~186.

\bookmarksetup{startatroot}

\chapter{Nationalisme, reactie en de nieuwe
ludieten}\label{nationalisme-reactie-en-de-nieuwe-ludieten}

\begin{quote}
`Nationalisme is uiteraard van nature absurd. Waarom zou het toeval --
of het lot om geboren te worden als Amerikaan, Albanees, Schot of
eilandbewoner van \emph{Fiji} -- loyaliteiten toewijzen die een individu
volledig domineren en een samenleving zo inrichten dat ze formeel met
anderen in conflict komt? Vroeger leefden mensen met lokale
loyaliteiten, gehecht aan een plaats, clan of stam, en hadden zij
verplichtingen jegens een heer of landeigenaar, wat leidde tot
dynastieke of territoriale oorlogen. De voornaamste loyaliteiten waren
echter gericht op religie, op God of op de godkoning, eventueel op een
keizer of op een beschaving als geheel. Er was geen natie. Men voelde
wel een band met de patria, het land van je voorouders, of kende
patriottisme, maar over nationalisme spreken vóór de moderne tijd is een
anachronisme.'\footnote{William Pfaff, \emph{The Wrath of Nations:
  Civilization and the Furies of Nationalism} (New York:
  \emph{Simon~\&~Schuster}, 1993), p.~17.} - WILLIAM PFAFF
\end{quote}

\url{http://www.ibm.com} Zeggen dat `de wereld kleiner wordt' is een
treffende beeldspraak, versterkt door autoriteiten zo gerenommeerd als
het reclamebureau van \emph{IBM}. Hun multiculturele reclamecampagnes
voor `\emph{Solutions for a small planet}' op het Internet herinneren
sportliefhebbers -- die dit zelf wellicht niet doorhebben -- eraan dat
de verhoudingen tussen individuen in wijd verspreide rechtsgebieden door
technologische ontwikkelingen ingrijpend zijn veranderd. We verwijzen
naar de vooraanstaande historicus William McNeill voor een waardevolle
voetnoot over de implicaties. Hij schrijft: ``De voortdurende
intensivering van communicatie en transport, in plaats van de nationale
consolidatie te bevorderen, begint in te werken in een tegenovergesteld
verlopend proces, daar de reikwijdte ervan de bestaande politieke en
etnische grenzen overstijgt.''\footnote{William H. McNeill,
  `Herbevestiging van de poly-etnische norm', in John Hutchinson en
  Anthony D. Smith (red.), \emph{Nationalism} (Oxford: \emph{Oxford
  University Press}, 1994), p.~300.} Nu de wereld steeds `kleiner wordt'
en de communicatie verbetert, zullen de willekeurige en inherent absurde
aanspraken van naties en van het nationalisme onvermijdelijk verzwakken.

\section{De grote transformatie}\label{de-grote-transformatie}

Het probleem met deze redelijke verwachting blijkt uit de gehele
geschiedenis: zij toont aan dat deze verwachtingen op geen enkele wijze
vervuld kunnen worden. De noodzakelijke overgang gaat gepaard met een
crisis. Het vraagt om een radicaal andere denkwijze en om een nieuwe
voorstelling van gemeenschap die verder reikt dan het nationalisme en de
natiestaat. Zoals Michael Billig benadrukt, zijn onze opvattingen over
de natiestaat en de vanzelfsprekendheid van nationaal behoren simpelweg
het resultaat van een specifieke historische periode.\footnote{Michael
  Billig, \emph{Banal Nationalism} (Londen: \emph{Sage Publications},
  1995), p.~16.} Die periode -- het Moderne Tijdperk -- behoort mogelijk
inmiddels tot het verleden. De dominante instituties -- de natiestaten
-- bestaan nog, maar zij staan op een wankele, geërodeerde basis.
Wanneer het onvermijdelijke zich voltrekt en natiestaten instorten,
verwachten we een felle reactie, met name in die welvarende landen waar
de `nationale economie' in de twintigste eeuw ongeschoolde arbeid goed
betaalde. Wij zijn ervan overtuigd dat de veranderingen in de
megapolitieke omstandigheden, dankzij de opkomst van
informatietechnologie, zullen resulteren in een ingrijpende
institutionele transformatie. De stelling in dit boek luidt dat de
geconcentreerde macht van de natiestaat gedoemd is te worden
geprivatiseerd en gecommercialiseerd. Net als bij andere ingrijpende
institutionele veranderingen zal de privatisering en commercialisering
van soevereiniteit een revolutie ontketenen in ons `gezond verstand', in
de wijze waarop we de wereld interpreteren. Dergelijke veranderingen
verlopen zelden geleidelijk of lineair.

Integendeel. Zoals we in \emph{The Great Reckoning} hebben aangetoond,
sluiten we dat in de praktijk vrijwel altijd uit. Wij verwachten dat het
informatietijdperk ingrijpende discontinuïteiten teweegbrengt -- dat wil
zeggen, scherpe breuken met de bestaande instituties en met het
bewustzijn van het verleden. Let op het volgende naarmate het proces
vordert:

\begin{enumerate}
\def\labelenumi{\arabic{enumi}.}
\item
  Wijzigingen in de economische organisatie, zoals in voorgaande
  hoofdstukken besproken, door de invloed van microverwerking.
\item
  Een tamelijk snelle daling in het belang van organisaties die zich
  beperken tot binnenlandse operaties. Overheden, vakbonden,
  gereglementeerde beroepsgroepen en lobbyisten spelen in het
  informatietijdperk een minder prominente rol dan in het industriële
  tijdperk. Doordat overheden hun gunsten en handelsbeperkingen niet
  meer zo effectief kunnen inzetten, daalt ook de verspilling aan
  lobbyactiviteiten.\footnote{Gordon Tullock, \emph{Rent-Seeking}
    (Aldershot, Engeland: \emph{Edward Elgar}, 1993).}
\item
  Een groeiend besef dat de natiestaat verouderd is, wat wijdverspreid
  leidt tot afscheidingsbewegingen in diverse delen van de wereld.
\item
  Een afname in status en invloed van traditionele elites, met een
  verminderd respect voor de symbolen en idealen waarop de natiestaat is
  gebaseerd.
\item
  Een heftige, soms zelfs gewelddadige nationalistische reactie die
  vooral voorkomt bij mensen die hun status, inkomen en macht verliezen
  wanneer hun `normale leven' verstoord wordt door politieke
  decentralisatie en de opkomst van nieuwe marktsystemen. Onder de
  kenmerkende uitingen van deze reactie vallen onder meer:

  \begin{enumerate}
  \def\labelenumii{\alph{enumii}.}
  \item
    een diepgeworteld wantrouwen en verzet tegen globalisering,
    vrijhandel, `buitenlands' eigendom en het binnendringen van lokale
    economieën;
  \item
    een uitgesproken vijandigheid tegenover immigratie, in het bijzonder
    als het gaat om groepen die duidelijk afwijken van de traditionele
    nationale samenstelling;
  \item
    een sterke afkeer van de informatie-elite, de rijken en
    hoogopgeleiden, gecombineerd met klachten over kapitaalvlucht en het
    verdwijnen van banen;
  \item
    extreme maatregelen die erop gericht zijn de afscheiding van
    individuen en regio's uit wankelende natiestaten te voorkomen, zoals
    het inzetten van oorlogen en het uitvoeren van `etnische
    zuiveringen', waarmee de nationale verbondenheid met de staat
    versterkt en de aanspraken van de staat op haar burgers en hun
    middelen wordt legitimeerd.
  \end{enumerate}
\item
  Omdat het duidelijk wordt dat informatietechnologieën soevereine
  individuen in staat stellen zich los te maken van staatsmacht,
  reageert men op het wegvallen van dwang met een neo-Luddistische
  aanval op zowel deze nieuwe technologieën als op hun gebruikers.
\item
  De nationalistisch-luddistische reactie verschilt per regio en
  bevolkingsgroep:

  \begin{enumerate}
  \def\labelenumii{\alph{enumii}.}
  \tightlist
  \item
    In snelgroeiende economieën, waar in het industriële tijdperk het
    inkomen per hoofd nog laag was en waar een verdere ontwikkeling van
    de markten leidde tot inkomensstijging op alle vaardigheidsniveaus,
    valt deze reactie minder intens op.
  \item
    Reactionaire gevoelens worden het sterkst ervaren in de huidige
    rijke landen, met name in gemeenschappen met een hoog percentage
    waardeloze en vaardigheidsarme mensen die voorheen hoge inkomens
    genoten.\footnote{Lawrence E. Harrison bespreekt in zijn boek
      \emph{Who Prospers? How Cultural Values Shape Economic and
      Political Success} (New York: Basic Books, 1992) uitvoerig de
      nauwe relatie tussen vaardigheden en waarden en het daaruit
      voortvloeiende economische succes.}
  \end{enumerate}
\end{enumerate}

\begin{enumerate}
\def\labelenumi{\alph{enumi}.}
\setcounter{enumi}{2}
\item
  Met uitzondering van de Unabomber trekken de neo-Luddieten vooral
  aanhangers uit het onderste tweederde van de verdiencapaciteit binnen
  de bevolking van de toonaangevende natiestaten.
\item
  De nationalistische en Luddistische respons zal zich het sterkst tonen
  niet bij de top, maar juist bij mensen met gemiddelde vaardigheden --
  `underachievers' met diploma's die volwassen werden in het industriële
  tijdperk en nu geconfronteerd worden met neerwaartse mobiliteit.
\end{enumerate}

\begin{enumerate}
\def\labelenumi{\arabic{enumi}.}
\setcounter{enumi}{7}
\item
  Naarmate nieuwe megapolitieke omstandigheden een vernieuwd
  identiteitsbesef oproepen, vergezeld van aanvullende ideologieën en
  morele waarden, verliezen de traditionele imperatieven van het
  nationalisme geleidelijk aan hun aantrekkingskracht.
\item
  De nationalistische reactie bereikt zijn hoogtepunt in de vroege
  decennia van het nieuwe millennium en neemt daarna af wanneer
  gefragmenteerde soevereiniteiten efficiënter blijken dan de
  gecentraliseerde macht van de natiestaat. Wij vermoeden dat het
  aangeboren vijandige gedrag van natiestaten tegenover alternatieve
  rechtsgebieden -- zoals geïllustreerd door de Russische invasie van
  Tsjetsjenië -- ertoe leidt dat naties en nationalistische
  fanatiekelingen de sympathie van de nieuwe generaties missen, die
  opgroeien onder de megapolitieke voorwaarden van het
  informatietijdperk.
\item
  Uiteindelijk stort de natiestaat ineen door een fiscale crisis.
  Systemische crises ontstaan doorgaans wanneer zwakke instituties
  kampen met stijgende kosten en dalende inkomsten -- een ontwikkeling
  die de toonaangevende natiestaten onvermijdelijk treft nu pensioenen
  en zorguitgaven begin eenentwintigste eeuw de hoogte in schieten.
  Terwijl wij dit schrijven worstelen zowel het Verenigd Koninkrijk als
  de Verenigde Staten met onverzekerde pensioenverplichtingen ter waarde
  van meerdere biljoenen dollars (per hoofd van de bevolking
  vergelijkbaar), die waarschijnlijk niet beteugeld kunnen worden. Ook
  andere leidende natiestaten kampen met vergelijkbare,
  faillissementsverwekkende lasten.
\end{enumerate}

\section{Parallels with the
Renaissance}\label{parallels-with-the-renaissance}

Eerder hebben we uiteen gezet waarom velen ervan overtuigd zijn dat de
ondergang van de `nanny state' gevolgen zal hebben die sterk lijken op
het verval van het institutionele monopolie van de Heilige Moederkerk
vijf eeuwen geleden. De Kerk oefende, net als de moderne natiestaat, al
eeuwenlang een onbetwiste dominantie uit. Op sommige punten zat de Kerk
zelfs dieper verankerd dan de staat dat vijfhonderd jaar later ooit zou
worden. Al jarenlang profileerde de Kerk zich als `de universele
autoriteit aan het hoofd van de christelijke samenleving,' aldus de
middeleeuwse intellectueel historicus John~B. Morrall.\footnote{John B.
  Morrall, \emph{Political Thought in Medieval Times} (New York:
  \emph{Harper}, 1958), p.~48.} Nog voor de technologische revolutie van
de jaren 1490 zouden nauwelijks Europeanen de aanspraak van de Kerk op
suprematie binnen het christendom in twijfel trekken, maar daarna hield
de traditionele rol van de Kerk nauwelijks langer stand.

\subsection{The privatization of
conscience}\label{the-privatization-of-conscience}

Aan het begin van de jaren 1520 verwierpen miljoenen Europeanen de
universele autoriteit van de Rooms-Katholieke Kerk -- een daad die als
ketterij werd beschouwd, waarvoor enkele decennia eerder foltering en de
doodstraf waren uitgerekend. Inderdaad, talloze middeleeuwse kathedralen
en kerken in Europa pronkten met leerzame gravures waarop ketters
afgebeeld stonden, als bewijsmateriaal dat demonen hun tong
uittrokken.\footnote{Voorbeeld: de gevel van de kathedraal in Angoulême
  (Frankrijk).}

Die martelingen maakten ongetwijfeld diepe indruk op de analfabete
parochianen, die de slachtoffers uitsluitend herkende aan het feit dat
zij als ketters bestraft werden. De beeldspraak liet geen twijfel
bestaan: ketters waren degenen van wie de tong werd verminkt. Hoe streng
de straf ook was, het was slechts een voorproefje van de ultieme
bestraffing voor ketterij -- de dood op de brandstapel. Tot
teleurstelling van de Kerk schrok deze waarschuwing echter niet genoeg
af. Met de opkomst van de drukpers nam het aanbod aan ketterse
argumenten zo sterk toe dat zelfs de dreiging van gruwelijke straffen
potentiële ketters niet langer afschrok. Inderdaad, geen enkele pionier
voor religieuze vrijheid in het vroegmoderne Europa betaalde de tol van
het laten afhakken van zijn tong. Anderen werden op de brandstapel
verbrand. De reactionaire agenten van de Inquisitie verbrandden
letterlijk mensen omdat zij uitspraken deden die wij als gewone
gewetensuitingen beschouwen. Al met al hebben de Reformatie en de
daaropvolgende reactie miljoenen levens geëist. Alleen al in de tweede
helft van de Dertigjarige Oorlog kwamen er op het slagveld 1.151.000
doden aan te pas.\footnote{Karen A. Rasler en William R. Thompson,
  \emph{Oorlog en staatsvorming: de vorming van de wereldmachten.
  Studies in internationale conflicten}, vol.~2 (Boston: Unwin Hyman,
  1989), p.~13.} Veel meer mensen stierven door honger, ziekte en door
de hand van de Inquisitie en andere autoriteiten. Niet al het geweld
kwam van katholieke machthebbers. De botten van meer dan duizend
vooraanstaande Engelse katholieken -- waarvan men dacht dat koning
Hendrik~VIII ze bruut had vermoord -- kwamen aan het licht in de Tower
of London. Sommigen, waaronder Sir Thomas More en bisschop St.~John
Fisher, werden openlijk geëxecuteerd omdat zij weigerden het oude geloof
op te geven.\footnote{Julian Large, `Bisschop stierf omdat hij
  standvastig bleef tegen Hendrik VIII', \emph{Daily Telegraph}, 16 juni
  1996, p.~2.} Aan de andere kant liet koning Hendrik~VIII's katholieke
dochter, koningin Maria -- die door syfilis, geërfd van haar vader,
waanzinnig was geworden -- in de laatste twee jaren van haar
regeerperiode maar liefst driehonderd protestantse ketters op de
brandstapel verbranden.

Men eiste een enorme tol van mensen met uiteenlopende overtuigingen toen
zij in de praktijk hun geloofsovertuigingen en het lang ontzegde recht
om zelf de door henzelf gesteunde kerk te kiezen omarmden. Vanuit ons
perspectief aan het einde van de twintigste eeuw vallen deze
persoonlijke uitingen ruim binnen het bereik van wat de vrijheid van
godsdienst en meningsuiting zou moeten beschermen. Maar aan het begin
van de zestiende eeuw bestonden noch de vrijheid van religie, noch de
vrijheid van meningsuiting. De autoriteiten hanteerden nog steeds een
vervagende middeleeuwse wereldbeschouwing. Zij zagen elke uiting van
individuele autonomie -- zeker wanneer die in opstand kwam tegen de
`plentitude potestatis' (volheid van macht) van de paus -- als
schandalig en subversief. Theologische historicus Euan Cameron merkte op
dat religieuze hervormers zoals Maarten~Luther overtuigingen omarmden
die `een bewuste en beslissende breuk betekenden met de institutionele
en geestelijke continuïteit van de oude Kerk.'\footnote{Cameron,
  \emph{op. cit.}, p.~97.}

\subsection{Ketterij en verraad}\label{ketterij-en-verraad}

In deze geest verwachten we `een weloverwogen en beslissende breuk' met
de institutionele en ideologische continuïteit van de natiestaat. Tegen
het einde van het eerste kwartaal van de volgende eeuw zullen miljoenen
rechtschapen burgers het seculiere equivalent van de ketterij uit de
zestiende eeuw hebben begaan -- een soort klein verraad. Zij trekken hun
loyaliteit aan de wankele natiestaat in om hun eigen soevereiniteit te
bevestigen, namelijk het recht om als burgers niet blindelings hun
bisschoppen of gebedshuizen te accepteren, maar zelf de vorm van bestuur
te kiezen. De privatisering van soevereiniteit gaat hand in hand met de
privatisering van het geweten van vijf eeuwen geleden. Beide vormen een
massale defectie van voormalig trouwe aanhangers van dominante
instellingen. Zoals Albert~O.~Hirschman, een expert op het gebied van
`reacties op verval in bedrijven, organisaties en staten', schreef, is
dit type exit zwaarwegend, omdat `exit vaak als crimineel wordt
bestempeld, aangezien het wordt gezien als desertie, defectie en
verraad.'\footnote{Hirschman, \emph{op. cit.}, p.~17.}

Soevereine individuen laten zich niet langer behandelen als louter
menselijke hulpbronnen van de staat. Miljoenen zullen de verplichtingen
van het burgerschap afwijzen om klanten te worden van de
overheidsdiensten. Zij zullen parallelle instellingen oprichten en
ondersteunen die vrijwel alle diensten die met burgerschap samenhangen
volledig op commerciële basis aanbieden. Gedurende het grootste deel van
de twintigste eeuw behandelde de staat werkenden als activa, net zoals
een melkveehouder zijn koeien inzet. Zij werden steeds harder uitgebuit
-- en nu krijgen de koeien vleugels.

\subsection{Defectie van het
burgerschap}\label{defectie-van-het-burgerschap}

Net zoals in de zestiende eeuw nieuwe megapolitieke ontwikkelingen het
kerkelijke monopolie ondermijnden, verwachten wij dat de megapolitiek
van het informatietijdperk uiteindelijk de voorwaarden voor het bestuur
in de eenentwintigste eeuw zal bepalen -- ongeacht hoe extreem haar
nieuwe voorschriften ook lijken voor degenen die de waarden van de
moderne politiek als de hunne beschouwen. De transformatie van de status
van `burger' naar die van `klant' betekent een breuk met het verleden,
zo ingrijpend als de overgang van ridderlijkheid naar burgerschap in de
vroegmoderne tijd. Wanneer de informatie-elite haar band met het
burgerschap verbreekt, levert dat een stimulans die vergelijkbaar is met
de reden waarom vijfhonderd jaar geleden miljoenen Europeanen hun
vertrouwen in de paus verloren.

Als de vergelijking met de Reformatie niet overtuigend overkomt, ligt
dat misschien deels aan het feit dat men tegenwoordig niet direct inziet
dat het afzweren van loyaliteit aan religieuze instituties ooit zo
ingrijpend was als het verraad dat in de twintigste eeuw streng werd
bestraft. Buiten enkele islamitische landen beschouwt men ketterij aan
het einde van de twintigste eeuw als een geestelijke overtreding die
iemands reputatie net zo min aantast als een bekeuring voor het rijden
met 45 mijl per uur in een 30-mijlzone. Sterker nog, het komt regelmatig
voor in Europa en Noord-Amerika dat geestelijken -- en zelfs bisschoppen
-- openlijk aangeven niet in God te geloven of cruciale fundamenten van
het geloof dat zij belijden af te wijzen. Tegenwoordig zal men ketterij
bijna alleen nog constateren wanneer zij neerkomt op flagrante
duivelaanbidding. In de meeste westerse landen zijn de religieuze
doctrines zo onsamenhangend en losjes vastgelegd dat weinigen nog de
theologische kernelementen kunnen aanwijzen die ooit het middelpunt
vormden van ketterijgeschillen. Dit weerspiegelt de algemene
verschuiving van de aandacht weg van religieuze kwesties.

Religieuze leiders hebben er in zekere mate toe bijgedragen dat men in
de late twintigste eeuw spirituele onderwerpen niet langer serieus nam.
Zij verlegden hun energie van de spirituele sferen naar het worden van
lobbyisten en agitatoren. Als losse individuen, aangetrokken door de
kracht van de macht, richten zij zich voornamelijk op het onder druk
zetten van politieke leiders om herverdelingsmaatregelen in te voeren
die essentieel blijken voor het nationalistische compromis. Denk
bijvoorbeeld aan de luidruchtige acties van de katholieke kerk in
Argentinië, die de regering van president Carlos Menem probeert te
dwingen economische hervormingen te laten varen ten gunste van
conventionele, inflatoire monetaire en Keynesiaanse fiscale maatregelen.
Vergelijkbare klachten hebben religieuze leiders ook geuit over pogingen
de opgeblazen begrotingen in Nieuw-Zeeland -- en in vele andere landen
-- te hervormen. Katholieke bisschoppen lobbyden fel tegen de hervorming
van de sociale zekerheid in de Verenigde Staten.

\subsection{Een fiscale inquisitie?}\label{een-fiscale-inquisitie}

Simpel gezegd richten hedendaagse religieuze leiders -- die hun moreel
gezag zien afnemen -- zich vooral op seculaire verlossing en op het
beïnvloeden van de staat via agitatie, in plaats van op spirituele
redding. In deze situatie kun je verwachten dat zij als medeplichtigen
meewerken aan de tegenreactie op de opkomende seculaire reformatie.
Wanneer de natiestaat wordt uitgedaagd en begint te wankelen, slaagt zij
er niet langer in de materiële beloftes na te komen die de kern van haar
populaire steun vormen. Het compromis dat tijdens de Franse Revolutie
werd gesloten, vervalt. De staat kan haar burgers niet langer goedkoop
of gratis onderwijs, laat staan medische zorg, werkloosheidsverzekering
en pensioen bieden als ruil voor toch al laag betaalde militaire dienst.
Hoewel de veranderende eisen van oorlogsvoering regeringen in staat
stellen zich en hun grondgebied te verdedigen zonder massale legers te
mobiliseren, nodigt het doorbreken van dit verouderde akkoord vrijwel
onvermijdelijk tot kritiek uit.

Inderdaad, zodra de nieuwe megapolitieke logica de overhand krijgt,
zullen de verliezers in de nieuwe informatie-economie de gevolgen als
buitengewoon onwelgevallig ervaren. Het is vrijwel zeker dat talloze
religieuze leiders, samen met de voornaamste begunstigden van
overheidsuitgaven, de leiding nemen in een nostalgische reactie die het
nationalisme opnieuw kracht bijzet. Zij zullen stellen dat geen enkele
Amerikaan, Fransman, Canadees of iemand van een andere nationaliteit
naar bed zou mogen gaan zolang hij honger heeft. Zelfs landen die
vooroplopen in hervormingen en onevenredig veel profiteren van
`marktvriendelijk globalisme', zoals Nieuw-Zeeland, zullen geteisterd
worden door reactionaire verliezers. Zij zullen proberen de
grensoverschrijdende beweging van kapitaal en mensen te belemmeren -- en
daarmee stoppen ze beslist niet.

Demagogen zoals Winston Peters, leider van de \emph{New Zealand First
Party}, hebben simpelweg geen zin om op originele wijze na te denken
over hoe de nieuwe wereld gaat functioneren. Maar na verloop van tijd
zullen Winston en zijn volgelingen de logica van de informatie-economie
doorgronden. Zij zullen trachten de verspreiding van computers,
robotica, telecommunicatie, encryptie en andere technologieën uit het
informatietijdperk -- die de verdringing van arbeiders in vrijwel elke
sector van de wereldeconomie bevorderen -- te blokkeren. Overal ontmoet
je politici die de vooruitzichten op langdurige welvaart actief willen
dwarsbomen, uitsluitend om te voorkomen dat individuen hun
onafhankelijkheid van de politiek gaan verkondigen.

\subsection{20/20-visie}\label{visie}

Tegen 2020 -- ofwel ongeveer vijf eeuwen nadat Martin Luther zijn 95
subversieve stellingen op de kerkdeur in Wittenberg had vastgenageld --
zal de opvatting over de kosten en baten van burgerschap een even
subversieve verheldering hebben doorgemaakt. De visie op de natiestaat
bij mensen met talent en vermogen -- de soevereine individuen van de
toekomst -- zal een politieke transformatie ondergaan die te vergelijken
is met laserchirurgie. Zij zullen 20/20 zien. In de twintigste eeuw --
evenals gedurende het hele moderne tijdperk -- zorgde het voortdurende
succes van geweld ervoor dat een grote overheid rendabel werd. De kracht
van geconcentreerde macht wist de loyaliteit van welgestelden en
ambitieuze individuen te winnen voor de \emph{OECD}-natiestaten, ondanks
de roofzuchtige belastingen op inkomen en kapitaal. Politici slaagden
erin om in elk \emph{OECD}-land marginale belastingtarieven op te leggen
die in het decennium direct na de Tweede Wereldoorlog de 90 procent
naderden of zelfs overschreden.

Zoals we al bespraken, hadden de rijken nauwelijks een andere keus dan
zich neer te leggen bij dergelijke heffingen. De omstandigheden dwongen
hen immers hun veiligheid toe te vertrouwen aan regeringen die
grootschalig geweld konden beheersen. Het deed nauwelijks ertoe --
behoudens wellicht Britse agenten die de kans kregen in Hongkong te gaan
werken -- dat \emph{OECD}-regeringen monopolistische belastingen
oplegden. Mensen met een hoog verdienvermogen, die tijdens het
Industriële Tijdperk toonaangevende kansen wilden benutten, hadden bijna
geen andere keuze dan in een economie met hoge belastingen te blijven
wonen. Dat betekende dat zij een belastingdruk moesten dragen die
onevenredig was ten opzichte van de geleverde diensten.

\subsection{De rekenkunde van de
politiek}\label{de-rekenkunde-van-de-politiek}

De Amerikaanse vicepresident John~J.~Calhoun uit de negentiende eeuw
beschreef op slimme wijze de `rekenkunde' van de moderne politiek. In
zijn formule verdeelde hij de volledige bevolking van de natiestaat in
twee groepen: enerzijds de belastingbetalers, die meer bijdragen aan
overheidsuitgaven dan zij ervan profiteren, en anderzijds de
belastingconsumenten, die meer overheidsvoordelen ontvangen dan zij
bijdragen. Met enkele opvallende uitzonderingen waren tegen het einde
van de twintigste eeuw de meeste ondernemers in \emph{OECD}-landen in
hoge mate netto belastingbetalers. In 1996 droeg bijvoorbeeld het top 1
procent van de Britse belastingplichtigen 17 procent bij aan de totale
inkomstenbelastingdruk. Zij betaalden 30 procent meer dan de onderste 50
procent van de verdieners, die slechts 13 procent van de
inkomstenbelasting voor hun rekening namen. In de Verenigde Staten droeg
de elite zelfs nog zwaarder bij, waarbij het top 1 procent in 1994 28
procent van de totale inkomstenbelastingontvangsten betaalde.\footnote{David
  Smith, `Wat Clarke van Reagan zou kunnen leren,' \emph{The Sunday
  Times} (Londen), 16 juni 1996, p.~6.} De rijken moesten niet alleen
betalen voor overheidsdiensten die -- zoals Frederic C. Lane ons
herinnert -- `van slechte kwaliteit en belachelijk overgeprijsd' waren,
maar hun betalingen stonden vaak totaal los van de geleverde
diensten.\footnote{Lane, `Economische gevolgen van georganiseerd
  geweld,' p.~404.} Voordelen waarvoor zij betaalden, kwamen vaak
volledig bij anderen terecht. Meestal waren de welgestelden er zelfs mee
tevreden dat zij minder gebruik maakten van overheidsdiensten, die
doorgaans van lage kwaliteit bleken te zijn. Overheidsinstanties in
bijna elk land stonden erom bekend buitengewoon inefficiënt te werken,
vooral omdat zij vaak werden bestuurd door medewerkers die geen
aansporing hadden om de productiviteit te verbeteren. Volgens vrijwel
alle maatstaven betaalden de grootste belastingbetalers tijdens het
industriële tijdperk vele malen meer voor overheidsdiensten dan deze in
een concurrerende markt waard zouden zijn.

Dat bleef nauwelijks onopgemerkt. Helaas leverde de erkenning dat
betalingen aan de overheid voor bescherming -- in Lanes woorden
`verspilling volgens ideale normen' -- zijn, halverwege de twintigste
eeuw zelden bruikbare inzichten op. Integendeel, men accepteerde het
eenvoudigweg als een onvermijdelijk gebrek, als `een van de
verschillende vormen van verspilling die inherent zijn aan de sociale
organisatie.'\footnote{Ibid.}

Het alternatief voor de ontevredenen was niet om bijvoorbeeld van
Groot-Brittannië naar Frankrijk te verhuizen of van de Verenigde Staten
naar Canada. Behalve in uitzonderlijke gevallen zou zo'n verhuizing
weinig verschil maken. Alle leidende natiestaten kampten immers met
hetzelfde nadeel: zij voerden min of meer confiscatoire
belastingregimes. Om aanzienlijk meer autonomie te verkrijgen, moest men
zich volledig losmaken van de kernlanden in Europa en Noord-Amerika en
op zoek gaan naar de periferie. In delen van Azië, Zuid-Amerika en op
diverse afgelegen eilanden lag de belastingdruk aanzienlijk lager. Toch
moest men doorgaans een prijs betalen voor het ontlopen van roofzuchtige
belastingheffing, namelijk een verlies van economische kansen en vaak
een daling van de levensstandaard. Zoals we hebben aangetoond, waren de
economische mogelijkheden beperkt en bleken de levensstandaarden
ondermaats in de meeste rechtsgebieden buiten de
kernindustrie‑natiestaten die zich schuldig maakten aan confiscatoire
belastingen.

Neem de communistische systemen als voorbeeld. Net als veel
Derde-Wereldregimes legden zij doorgaans geen hoge inkomstenbelastingen
op -- soms zelfs helemaal niet.\footnote{Cuba introduceerde pas in 1996
  een inkomstenbelasting als noodmaatregel, als reactie op de
  economische depressie die ontstond nadat subsidies wegvielen door de
  ineenstorting van het communisme in Europa.} Desondanks zochten over
de bijna drie kwart eeuw dat de Sovjet-Unie bestond vrijwel geen
ondernemer naar belastingontwijking. Hoewel de
inkomstenbelastingtarieven in de Sovjet-Unie niet hoog waren, boden zij
geen voordeel, omdat de Sovjets er een deugd van maakten dat zij
eigendomsrechten niet erkenden. Dit legde een nog zwaardere last op dan
de gewone belastingdruk. De communistische systemen maakten het bijna
onmogelijk om een bedrijf op te zetten en er serieus geld mee te
verdienen. In feite nam de communistische staat het inkomen al in beslag
vóórdat er belastingen werden geheven.

Bovendien, als iemand -- om welke eigenzinnige reden dan ook -- met een
vast inkomen had besloten om in Moskou of Havana te gaan wonen, zou hij
moeite hebben gehad om een fatsoenlijke levensstandaard te bekostigen.
Buiten de toegang tot goede sigaren, kaviaar, voortreffelijke orkesten
en ballet bood het leven in de voormalige communistische systemen
nauwelijks consumentelijk genot. De schaarse geneugten bleven vaak
ontoegankelijk of werden streng gerantsoeneerd op basis van politieke
invloed in plaats van via vrije uitwisseling. Hoewel critici van het
postmoderne leven het `belang van consumptie in de postmoderne ervaring'
benadrukken, heeft de gestegen standaard van goederen en diensten die
sinds de val van het communisme wereldwijd beschikbaar is, er
ongetwijfeld toe bijgedragen dat de concurrentie tussen rechtsgebieden
levendiger werd, waardoor de banden met natie en plaats
verzwakten.\footnote{Ni. Featherstone, \emph{Consumer Culture and
  Postmodernism} (Londen: Sage, 1991), en J. F. Sherry, `Postmodern
  alternatief: de interpretatieve wending in consumentonderzoek,' in T.
  Robertson en H. Kassarjian (red.), \emph{Handbook of Consumer
  Research} (Englewood Cliffs, N.J.: Prentice-Hall, 1991), besproken in
  Billig, \emph{op. cit.}}

Onder het oude regime waren de keuzevrijheden voor consumenten zo
beperkt dat zelfs Castro moeite had kunnen hebben om een pakje
fatsoenlijke flosdraad te bemachtigen als hij cohiba-resten uit zijn
tanden wilde verwijderen. Tot voor kort mochten zelfs de rijken in vele
delen van de wereld niet genieten van de levensstandaard waaraan de
middenklasse in West-Europa of Noord-Amerika gewend was. Geconfronteerd
met deze benarde situatie kozen de meeste getalenteerde mensen tijdens
het industriële tijdperk voor de nationalistische koers. Ze bleven ter
plaatse en betaalden schandalig hoge belastingen voor de twijfelachtige
bescherming die hun natiestaat bood, een staat die het gebruik van
geweld als monopolie hanteerde binnen het grondgebied waarin zij geboren
werden.

\begin{quote}
`Het paradijs is nu gesloten en vergrendeld, buitengesloten door
engelen, dus moeten we nu verder, de wereld rondreizen en kijken of er
op de een of andere manier een achterdeur is.' HEINRICH VON KLEIST
\end{quote}

De val van het communisme heeft het `ijzeren gordijn' weggenomen dat het
reizen ooit belemmerde en de globalisering van de handel effectief
blokkeerde, waardoor de wereld kunstmatig `groot' bleef. Door de komst
van straalvliegtuigen en informatietechnologieën die het communisme
ondermijnden, is de wedijver om de welvarende reisdollars toegenomen. De
stoet bankiers die zelfs via de meest afgelegen provincies stroomden,
zorgde voor een enorme impuls aan zowel de woonstandaarden als de
culinaire scene wereldwijd. Hiermee bedoelen we niet de opmars van
\emph{McDonald's}-hamburgers en \emph{Kentucky Fried
Chicken}-franchises, zelfs niet in voorheen onherbergzame steden als
Moskou en Boekarest. Wat minder in het oog valt, maar des te
belangrijker is, is de opkomst van vooraanstaande hotelketens en
kwaliteitsrestaurants waar men in alle rust kan genieten van grand
cru-clarets in plaats van wodka met cola. Dankzij deze transformatie kan
iedereen die het zich kan veroorloven, bijna overal op de planeet
genieten van een hoge materiële levensstandaard. Het is tegenwoordig
immers zeldzaam een land aan te treffen waar niet ten minste één
eersteklas hotel en een restaurant is te vinden dat zelfs een
Michelin-inspecteur zou bekoren.

Zoals Hirschman al een kwart eeuw geleden voorzag, heeft technologische
vooruitgang de aantrekkingskracht van `exit' als oplossing voor
ontoereikende dienstverlening en prijsstelling aanzienlijk versterkt.
Hij schreef: `Loyaliteit aan je land is daarentegen iets waarvan we ook
zonder kunnen. \ldots{} Pas wanneer landen door de vooruitgang in
communicatie en algehele modernisering steeds meer op elkaar gaan
lijken, zal het gevaar van voortijdige en buitensporige exits ontstaan,
waarbij de 'brain drain' een actueel voorbeeld is.' Merk daarbij op,
zoals we in hoofdstuk 8 opmerkten, dat Hirschmans norm voor `voortijdige
en buitensporige exits' wordt benaderd vanuit het perspectief van de
ontdoening van de natiestaat en niet vanuit het oogpunt van het individu
dat op zoek is naar een beter bestaan.

Toch is zijn conclusie onmiskenbaar: hoe meer landen op elkaar gaan
lijken, des te aantrekkelijker defectie en exit worden. Dat het
tegenwoordig overal eenvoudiger is om goed te leven, maakt het
aantrekkelijk om je te vestigen op plekken waar de kosten het laagst
liggen. Nog belangrijker dan dat je vrijwel overal comfortabel kunt
wonen, is dat je tegenwoordig overal een hoog inkomen kunt verdienen. Je
hoeft niet langer in een dure jurisdictie te wonen om voldoende vermogen
op te bouwen om, zoals Lord Keynes adviseerde, `wijs, aangenaam en goed'
te leven. Zoals we eerder betoogden, verandert microtechnologie de
fundamentele megapolitieke basis waarop de natiestaat rust. In het
informatietijdperk ontstaat een nieuwe cybereconomie die geheel buiten
het bereik valt van elke overheid om te monopoliseren. Voor het eerst
stelt technologie individuen in staat rijkdom op te bouwen in een domein
dat zich niet zomaar laat onderwerpen aan de eisen van systematische
dwang.

De nieuwe samenleving -- en daarmee de bijbehorende cultuur -- krijgt
vorm door twee krachten. Enerzijds spelen machines een bepalende rol
doordat zij taken uitvoeren waar mensen niet meer tegenop kunnen, zoals
automatisering die steeds meer laaggeschoolde werkzaamheden overbodig
maakt. Anderzijds biedt informatietechnologie krachtige mogelijkheden
aan mensen met het talent om er optimaal van te profiteren. In zo'n
samenleving ontstaan grotere spanningen tussen een kleine elite -- die
je eventueel `informatiesterren' of `informatiaristocraten' kunt noemen
-- en een groeiende achterstandsklasse, de `informatiearmen'. Een
onderscheid tussen beide groepen is dat de informatiearmen vaak
vastzitten aan een specifieke locatie of weinig baat hebben bij een
verhuizing, terwijl informatiesterren, zoals elders al toegelicht,
buitengewoon mobiel zijn en overal kunnen verdienen waar zij zich
aangetrokken voelen -- vergelijkbaar met populaire romanschrijvers.
Honderd jaar geleden verdiende Robert Louis Stevenson zijn kost op een
eiland in de Stille Oceaan; tegenwoordig doet de informatiearistocratie
dat net zo goed.

\subsection{Marktconcurrentie tussen
rechtsgebieden}\label{marktconcurrentie-tussen-rechtsgebieden}

Doordat informatietechnologie de beperkingen van een vaste locatie
overschrijdt, stelt zij automatisch alle rechtsgebieden bloot aan reële
wereldwijde concurrentie op basis van kwaliteit en prijs. Overheden die
als lokale monopolen opereren, komen uiteindelijk de harde concurrentie
van de markt te lijf, afhankelijk van hoe goed zij hun klanten bedienen.
Zo wordt al snel duidelijk dat de oude logica -- waarbij in het
industriële tijdperk hoge kostenregimes voordelig waren -- inmiddels op
zijn kop staat. Vooraanstaande natiestaten, met hun roofzuchtige,
herverdelende belastingstelsels en strenge regelgeving, gelden niet
langer als aantrekkelijke rechtsgebieden. Onpartijdig gezien leveren zij
immers bescherming van mindere kwaliteit en beperkte economische kansen
tegen monopolistische prijzen. In de komende jaren kunnen zij sociaal
zelfs minder ontvankelijk en gewelddadiger uitpakken dan regio's in Azië
en Latijns-Amerika, waar de inkomenshistorie traditioneel ongelijker is.
De vooraanstaande verzorgingsstaten zullen hun meest getalenteerde
burgers verliezen doordat zij massaal het land verlaten.

\subsection{Het ``extranationale'' tijdperk dat voor ons
ligt}\label{het-extranationale-tijdperk-dat-voor-ons-ligt}

Naarmate het tijdperk van het `soevereine individu' vorm krijgt, zullen
steeds meer bekwame mensen stoppen met zichzelf te definiëren als
onderdeel van een natie -- als `brits', `amerikaans' of `canadees'. In
het nieuwe millennium ontdekken we een transnationaal of extranationaal
wereldbeeld en een geheel nieuwe manier om iemands plek daarin te
bepalen. Deze nieuwe identiteit komt niet voort uit de systematische
dwang die in de twintigste eeuw leidde tot de universele invoering van
natiestaten en het statelijke systeem.

Het feit dat we wereldomspannende ontwikkelingen vaak als
`internationaal' bestempelen, laat zien hoezeer het nationalistische
paradigma onze kijk op de wereld heeft geïnternaliseerd. Na twee eeuwen
waarin men werd onderwezen in de geheimen van `internationale
betrekkingen' en `internationaal recht', vergeet men al snel dat het
begrip `internationaal' geen eeuwenoud Westers concept is. Jeremy
Bentham bedacht het woord in 1789 en introduceerde het in zijn boek
\emph{An Introduction to the Principles of Morals and Legislation}. Hij
schreef: `Het woord internationaal, dat moet worden erkend, is nieuw, al
is gehoopt dat het voldoende analoog en begrijpelijk zal
zijn.'\footnote{Jeremy Bentham, \emph{An Introduction to the Principles
  of Morals and Legislation}, geredigeerd door J. H. Burns en H. L. A.
  Hart (London: Methuen, 1982), p.~296, geciteerd door Billig, \emph{op.
  cit.}, p.~84.} Het begrip sloeg snel aan en ver ging al snel verder
dan wat Bentham oorspronkelijk bedoelde; `internationaal' werd een los
synoniem voor alles wat zich wereldwijd afspeelt.

Het internationale tijdperk ging in 1789 van start, het jaar van de
Franse Revolutie, en hield twee eeuwen stand tot 1989, toen in Europa de
opstand tegen het communisme losbarstte. Wij zijn ervan overtuigd dat
die tweede revolutie het internationale tijdperk daadwerkelijk
beëindigde -- en niet louter omdat het in diskrediet geraakte
communistische volkslied `The International' op de voorgrond trad. De
centrale planeconomie, waarin de staat het eigendom beheerste, vormde de
meest ambitieuze uiting van de natiestaat. De hechte band tussen
staatsmacht en nationalisme kwam duidelijk tot uiting in de taal.
`Nationaliseren' werd het meest geladen werkwoord van de moderne tijd,
wat inhield dat iets onder staatsbezit en -controle kwam. Demagogen in
bijna alle delen van de wereld gebruikten het woord moeiteloos, maar
tegenwoordig hoort het enkel nog bij het vocabulaire van vervlogen
tijden. Nationalisatie past niet meer in onze tijd, omdat ook de
traditionele staatsmacht tot het verleden behoort.

Aan het einde van het moderne tijdperk verzwakten technologische
innovatie en marktkrachten stelselmatig de gecentraliseerde macht van de
staat. Nu breekt een nieuwe fase aan, waarin de markt zijn triomf
voltooit. Individuele natiestaten verdwijnen geleidelijk en wij
verwachten zelfs dat de club der natiestaten -- de Verenigde Naties --
spoedig failliet gaat. We zouden ons dan ook niet verbazen als men de VN
kort na de millenniumwisseling ontbindt.

Als we `internationaal' als een aandeel beschouwen, is het nu tijd om te
verkopen. In het nieuwe millennium zal men het begrip waarschijnlijk
vervangen of op zijn minst terugbrengen naar de oorspronkelijke
betekenis, want de wereld laat zich niet langer regeren door een systeem
van onderling verbonden soevereine natiestaten. Relaties krijgen de
nieuwe `extranationale' vorm, mede door het groeiende belang van
microjurisdicties en soevereine individuen. Een conflict tussen een
enclave aan de kust van Labrador en een soeverein individu zal dan niet
als een `internationaal' geschil worden bestempeld, maar als
extranationaal.

In de tijd die komt, bepalen territoriale grenzen niet langer de
samenstelling van gemeenschappen en loyaliteiten. Mensen voelen zich
vooral verbonden door oprechte affiniteiten, gedeelde overtuigingen,
gemeenschappelijke belangen en zelfs gezamenlijke genetica, in plaats
van door de kunstmatige band die nationalisten zo nadrukkelijk
waarderen. Ook organiseren we onze bescherming op nieuwe manieren die
niet langer af te bakenen zijn met een sextant, een loodlijn of andere
instrumenten uit de vroegmoderne tijd die ooit territoriale grenzen
markeerden.

\section{Uitgevonden gemeenschappen en
tradities}\label{uitgevonden-gemeenschappen-en-tradities}

De kosmopolitische elite bestempelt in de komende eeuw het idee dat
mensen zichzelf van nature in een `uitgevonden' gemeenschap -- oftewel
een natie -- plaatsen als excentriek en onredelijk, net zoals men dit
gedurende het grootste deel van de menselijke geschiedenis al deed.
Zoals socioloog Anthony Giddens schreef, kent de natiestaat `geen
precedent in de geschiedenis.'\footnote{Anthony Giddens, \emph{Social
  Theory and Modern Sociology} (Cambridge: Polity Press, 1987), p.~166,
  geciteerd in Billig, \emph{op. cit.}} Michael Billig, een autoriteit
op het gebied van nationalisme, versterkte dat punt:

\begin{quote}
In vroegere tijden hechtten mensen niet dezelfde betekenis aan begrippen
als taal en dialect, laat staan aan die van grondgebied en
soevereiniteit -- begrippen die tegenwoordig zo vanzelfsprekend en
tastbaar lijken. Zulke ideeën liggen zo diep verankerd in het
hedendaagse gezond verstand dat men gemakkelijk vergeet dat ze feitelijk
verzonnen constructies zijn. De middeleeuwse schoenmakers in de
werkplaatsen van Montaillou of San Mateo zouden ons, na een
tussenperiode van 700 jaar, als beperkte, door bijgeloof doordrenkte
figuren zien. Zij zouden onze opvattingen over taal en natie als vreemd
mystiek beschouwen en zich verbazen dat zo'n mystiek een kwestie van
leven en dood kan zijn.\footnote{Billig, \emph{op. cit.}, p.~36.}
\end{quote}

\begin{quote}
Wij vermoeden dat ook denkers in een extranationale toekomst evenzeer
verbaasd zullen zijn. Zoals Benedict Anderson opmerkte, zijn naties
`ingebeelde gemeenschappen.'\footnote{Benedict Anderson, \emph{Imagined
  Communities} (London: Verso, 1983), geciteerd door Billig, \emph{op.
  cit.}, p.~10.} Dit wil niet zeggen dat wat ingebeeld is per se
onbeduidend is. Zoals Dr.~Johnson zei: zonder verbeelding zou een man
net zo graag `bij een kamermeisje liggen als bij een hertogin.' Toch
lijken naties voor degenen die in de twintigste eeuw opgroeiden zo'n
onvermijdelijke organisatie-eenheid te vormen dat het moeilijk te
bevatten is dat ze in feite louter ingebeeld zijn en niet natuurlijk. Om
te begrijpen hoe anders de toekomst kan zijn dan de wereld die ons
vertrouwd is, moeten we onderzoeken hoe het nationalisme eens aan het
`gezond verstand' van het Industriële Tijdperk werd opgelegd.
\end{quote}

\begin{quote}
Het valt gemakkelijk over het hoofd te zien in hoeverre de `nationale
gemeenschap' tot stand komt door een voortdurende inzet van verbeelding.
Er bestaan immers geen objectieve criteria om nauwkeurig vast te stellen
welke groep wel of geen `natie' behoort te zijn. Ook bestaan er, strikt
genomen, geen `natuurlijke grenzen', zoals de vooraanstaande historici
Owen Lattimore en C.~R. Whittaker hebben aangetoond. `Een belangrijke
keizerlijke grens is, aldus Lattimore, niet louter een lijn die
geografische gebieden en menselijke samenlevingen scheidt. Zij
vertegenwoordigt ook de optimale limiet van de groei van een bepaalde
samenleving.'\footnote{Owen Lattimore, \emph{Inner Asian Frontiers of
  China} (New York: Beacon Press, 1960), p.~60. Geciteerd door Ronald
  Findlay in `Towards a Model of Territorial Expansion and the Limits of
  Empire', opgenomen in Michelle R. Garfinkel en Stergios Skaperdas
  (red.), \emph{The Political Economy of Conflict and Appropriation}
  (Cambridge: Cambridge University Press, 1996), p.~54.} Of, zoals
econoom Ronald Findlay van \emph{Columbia University} het verwoordde:
`Voor zover men er in de economie überhaupt aandacht aan besteedt,
worden de grenzen van een bepaald economisch systeem of `land' doorgaans
als gegeven beschouwd, evenals de bevolking die binnen die grenzen
leeft. Toch is het overduidelijk dat, hoe gewijde deze grenzen ook
mochten zijn geworden in het internationale recht, zij op een gegeven
moment altijd betwist werden door rivaliserende claimanten en
uiteindelijk bepaald werden door het evenwicht tussen economische en
militaire macht van de betrokken partijen.'\footnote{Findlay, \emph{op.
  cit.}, p.~41.}
\end{quote}

\begin{quote}
Zelfs iemand die over alle gegevens beschikt over de helft van de
natiestaten wereldwijd en een indrukwekkende verzameling
satellietbeelden bezit, zou er niet in slagen te voorspellen waar de
grenzen van de overige staten liggen. Er bestaat bovendien geen
wetenschappelijke methode waarmee men op basis van biologische of
taalkundige criteria de leden van de ene nationaliteit van die van een
andere kan onderscheiden. Zelfs de meest geavanceerde autopsietechnieken
slagen er niet in om genetisch onderscheid te maken tussen de
overblijfselen van Amerikanen, Canadezen en Soedanese na een
vliegtuigongeluk. De grenslijnen tussen staten en nationaliteiten zijn
niet natuurlijk, zoals dat wel geldt voor de grenzen tussen soorten of
de fysieke verschillen tussen dierlijke rassen. Integendeel: zij vormen
artefacten van zowel vroegere als voortdurende pogingen tot
machtsprojectie.
\end{quote}

\begin{quote}
``Een taal is een dialect met een leger en een marine'' - MARIO PEI
\end{quote}

\section{talen als artefacten van
macht}\label{talen-als-artefacten-van-macht}

Opmerkelijk genoeg geldt voor talen ongeveer hetzelfde principe. Na
eeuwenlange dominantie door natiestaten lijkt het onbezonnen -- of zelfs
absurd -- te beweren dat `taal' geen objectieve basis biedt om volkeren
van elkaar te onderscheiden. Maar bekijk het eens goed: de geschiedenis
van de moderne talen toont duidelijk hoe bewust ze zijn vormgegeven om
de nationalistische identiteit te versterken. De westerse `talen', zoals
wij ze vandaag de dag spreken en begrijpen, hebben zich niet op
natuurlijke wijze ontwikkeld tot hun huidige vorm. Men kan ze bovendien
niet objectief onderscheiden van `dialecten'. Tegenwoordig kiest vrijwel
niemand ervoor een `dialect' te spreken; bijna iedereen verkiest dat
zijn moedertaal als authentiek wordt gezien -- als een volwaardige
`taal'.

\begin{quote}
`Laat niemand zeggen dat het woord in zulke momenten weinig nut heeft.
Woord en daad vormen samen één. De krachtige, energieke bevestiging die
harten geruststelt, schept daden -- dat wat gezegd wordt, wordt
gerealiseerd. De daad is hier de dienaar van het woord; hij volgt
onderdanig, zoals op de eerste dag van de wereld: Hij zei, en de wereld
was.' -MICHELET, augustus 1792
\end{quote}

\subsection{`woord en daad vormen samen
één'}\label{woord-en-daad-vormen-samen-uxe9uxe9n}

Alvorens de Franse Revolutie uitbrak, sprak men in Zuid-Frankrijk een
mengvorm van het Latijn, de la langue d'oc -- oftewel het Occitaans --
die meer weg had van de volkstaal in Catalonië dan van la langue d'oil,
de Parijse taal waaruit het moderne `Frans' is ontstaan. Toen de
`Verklaring van de Rechten van de Mens en de Burger' in Parijse stijl
werd gepubliceerd, begreep de meerderheid van de mensen die binnen de
huidige landsgrenzen van Frankrijk leefde de tekst
nauwelijks.\footnote{Billig, \emph{op. cit.}, p.~25.} De revolutionairen
stonden voor de uitdaging hun pamfletten en edicten te vertalen naar het
patois van talloze dorpen, die elkaar slechts vaag konden verstaan.

De inwoners van het gebied dat later Frankrijk zou worden, hanteerden
zeer uiteenlopende spreekwijzen. Zij kozen er bewust voor deze te
bundelen tot één officiële taal als beleidsoptie. Sinds Franciscus I in
1539 het Edict van Villers-Cotterets uitvaardigde en daarmee het
geschreven Frans als de officiële taal van de rechtbanken
instelde,\footnote{Anderson, \emph{op. cit.}, p.~93.} betekende dit nog
lang niet dat iedereen de taal kon begrijpen -- net zoals het `juridisch
Frans' in Engeland, dat na 1200 als officiële rechtbankstaal werd
gebruikt, niet breed werd verstaan. Beide vormen dienden immers louter
als administratieve volkstalen, niet als een gestandaardiseerde taal die
over het gehele grondgebied leefde.

De revolutionairen wilden meer: zij streefden naar een echte nationale
taal. Historicus Janis Langins merkt in \emph{The Social Histor,' of
Language} op: `een invloedrijke groep revolutionairen was ervan
overtuigd dat het succes van de Revolutie en de verlichting zou
bevorderen als men opzettelijk een standaard Frans oplegde in het
grondgebied van de Republiek.'\footnote{Janis Langins, `woorden en
  instellingen tijdens de Franse revolutie: het geval van
  ``revolutionair'' wetenschappelijk en technisch onderwijs', opgenomen
  in Peter Burke en Roy Porter, \emph{de sociale geschiedenis van de
  taal} (Cambridge: Cambridge University Press, 1987), p.~137.} Deze
bewuste inspanning leidde tot verhitte discussies over het gebruik van
specifieke woorden. Zo ontstond bijvoorbeeld het aansprekende
bijvoeglijk naamwoord `revolutionair', dat in 1789 voor het eerst door
Marabou werd ingezet. Na een periode van `enigszins wijd en
ongedifferentieerd gebruik', zoals Langins beschrijft, volgde tijdens de
Terreur een decennialange fase waarin het woord onderdrukt en vergeten
werd. Op 12 juni 1795 besloot de Conventie de taal -- evenals de
instellingen die onze voormalige tirannen {[}dat wil zeggen de verslagen
Robespierre-aanhangers{]} hadden opgericht -- te hervormen door het
woord `revolutionair' in officiële benamingen te vervangen.\footnote{Ibid.,
  pp.~140, 142.} Tot op de dag van vandaag blijkt deze traditie van
taalinterventie in de kieskeurige houding van de Franse autoriteiten
tegenover leenwoorden als `weekend', die vanuit het Engels hun weg naar
het Frans hebben gevonden.

Twee eeuwen geleden richtten de nationale taalplanners in Frankrijk zich
niet alleen tegen woorden uit het overkantse Engeland; zij kregen de
veel grotere taak om alle lokale spraakvarianten binnen het grondgebied
van de Republiek uit te bannen. Hun inspanning beperkte zich niet louter
tot het onderdrukken van la langue d'oc. Het `Frans' dat destijds aan de
Rivièra werd gesproken, deed meer denken aan het `Italiaans' dat in
oostelijke gebieden werd gehanteerd dan aan het Parijse Frans. Evenzo
had men de taal van de Elzas makkelijk kunnen beschouwen als een vorm
van Duits, dat zelf weer boordevol regionale varianten zat. In de
Pyreneeën hoorde men Baskisch, en net als het Breton dat langs de
noordwestkust van Frankrijk gesproken werd, had het Baskisch weinig weg
van de volkstaaldialecten van het Latijn waaruit het `Frans' is
voortgekomen. Ook in het noordoosten woonde een aanzienlijk aantal
Vlamingstaligen. `De Parijse spreekstijl', zoals Michael Billig ons
herinnert, verspreidde zich niet via spontane marktprocessen, maar werd
wettelijk en cultureel opgelegd als `Frans'.\footnote{Billig, \emph{op.
  cit.}, p.~27.}

Wat in Frankrijk gold, bleek ook elders geldig te zijn bij het ontstaan
van natiestaten. Talen verspreidden zich vaak via militaire campagnes en
werden door koloniale machten opgelegd. Zo werd bijvoorbeeld, na de
onafhankelijkheid, de kaart van Afrika getekend op basis van de gebieden
waar de administratieve talen van de Europese mogendheden domineerden.
Lokale dialecten kwamen zelden aan bod op scholen. Het verschil tussen
erkende `talen' -- die vaak dienen als basis voor de definitie van
`naties', zelfs bij natiestaten met willekeurige koloniale grenzen -- en
`dialecten' had in wezen een politieke lading.

Kortom, het opleggen van een `nationale taal' maakte deel uit van een
mondiaal proces om de staatsmacht te versterken. Door alle inwoners
binnen een grondgebied, waar de staat het geweld monopoleerde, te
verplichten of aan te moedigen de `moedertaal' te gebruiken, verwierven
de machthebbers aanzienlijke voordelen bij het uitoefenen van hun macht.

\subsection{De militaire dimensie van
taaluniformiteit}\label{de-militaire-dimensie-van-taaluniformiteit}

In een wereld waarin geweld steeds lonender werd, leverde het invoeren
van een nationale taal duidelijke militaire voordelen op. Een nationale
taal vormde vrijwel de noodzakelijke basis voor het versterken van de
centrale macht in natiestaten. Door hun burgers te stimuleren dezelfde
taal te spreken, konden centrale autoriteiten de militaire kracht van
lokale potentaten effectief ondermijnen. De standaardisering van de taal
na de Franse Revolutie maakte de inzet van nationale dienstlegers -- de
goedkoopste en meest efficiënte vorm van moderne militaire macht --
mogelijk. Een gemeenschappelijke taal stelde troepen uit alle delen van
de `natie' in staat om vlot met elkaar te communiceren. Dit bleek
essentieel voordat nationale dienstlegers de onafhankelijke bataljons,
die door machtige lokale potentaten in plaats van door centrale
autoriteiten werden gerekruteerd en aangestuurd, konden verdringen.

Voor de Franse Revolutie, zoals we in hoofdstuk 5 bespraken, werden
troepen opgevoerd en geleid door lokale potentaten, die al dan niet
reageerden op strijdoproepen uit Parijs of een andere hoofdstad. Hoe dan
ook, hun positie werd na grondige onderhandelingen vastgesteld. Zoals
Charles Tilly opmerkt: `het vermogen om steun te geven of te
onthouden\ldots{} grote onderhandelingsmacht' \footnote{Tilly,
  \emph{Coercion, Capital, and European States}, p.~22.}. Bovendien
vormden de onafhankelijke militaire eenheden een extra uitdaging voor
centrale autoriteiten, omdat zij zich konden verzetten tegen pogingen
van de overheid om binnenlandse middelen in beslag te nemen. Het is
duidelijk dat centrale autoriteiten -- of het nu de koning of de
revolutionaire conventie was -- voor een zware opgave stonden om
belastingen te innen of op andere manieren de middelen van lokale
potentaten, die met privélegers hun bezittingen konden verdedigen, te
bemachtigen.

Nationale legers vergrootten de macht van de centrale overheid
aanzienlijk, omdat zij haar in staat stelden haar wil over een
uitgestrekt grondgebied door te voeren. Het invoeren van een nationale
taal speelde daarbij een sleutelrol, aangezien het de totstandbrenging
van zulke legers mogelijk maakte. Voordat deze legers konden ontstaan en
daadwerkelijk effectief opereren, moesten hun leden elkaar immers
vloeiend verstaan.

Het bood een strategisch voordeel als iedereen binnen een rechtsgebied
bevelen en instructies begreep en bovendien via de hiërarchische keten
essentiële informatie kon terugkoppelen. De Franse revolutionairen
lieten vrijwel direct zien hoe waardevol dit was. Naast het runnen van
een soort taalschool organiseerden zij intensieve, een maand durende
cursussen, waarin -- volgens Langins -- `honderden studenten uit heel
Frankrijk werden opgeleid in de technieken van buskruit- en
kanonfabricage.'\footnote{Langins, \emph{op. cit.}, p.~143.}

Het militaire voordeel van de Franse aanpak bleek niet alleen uit hun
successen in de Napoleontische periode, maar ook uit de ervaringen van
regimes die tijdens oorlog geen baat hadden bij een gemeenschappelijke
taal. Zo droeg het aristocratische officierenkorps van de tsaar, dat
doorgaans in het Duits communiceerde (de andere hoftaal van de Romanovs
was Frans), er mede toe dat gewone soldaten en de burgerij de bevelen
niet begrepen -- een van de vele factoren die in de vroege dagen van de
Eerste Wereldoorlog leidden tot desastreuze nederlagen en een
demoraliserende stemming onder de Russische troepen.

Dit benadrukt een ander belangrijk militair voordeel van een
gemeenschappelijke taal: het verlaagd de drempel om oorlog te voeren.
Propaganda verliest immers zijn effect als de boodschap onbegrijpelijk
is. Ook op dit vlak twijfelden de Franse revolutionairen niet aan de
mogelijkheden. Volgens Langins lag hun overkoepelende gedachte in `de
wil der mensen'. Zij zorgden ervoor dat de volkse wil tot uiting kwam in
de taal van het volk. Vóór 1789 belemmerde het gebrek aan onderlinge
taalbegrip tussen burgers het gezamenlijk uiten van hun wil, wat de
uitoefening van nationale macht remde. Meertalige staten en rijken
ondervonden tijdens de industriële periode bovendien extra obstakels bij
de mobilisatie voor oorlog.

Op zij­niveau vervingen landen die intern meertalig waren, vaak
natiestaten die hun burgers beter konden enthousiasmeren voor de strijd
en efficiënter middelen konden mobiliseren. Dit blijkt uit
nationalistische consolidatie, zoals de vorming van Frankrijk en de
Fransen aan het eind van de achttiende eeuw, en uit voorbeelden van
nationalistische decentralisatie, zoals de ineenstorting van het
Oostenrijks-Hongaarse Rijk na de Eerste Wereldoorlog. De nieuwe
natiestaten die na het Habsburgse Rijk ontstonden -- Oostenrijk,
Hongarije, Tsjechoslowakije en Joegoslavië -- bestempelde Keynes als
`onvolledig en onvolwassen.' Toch overtuigden de argumenten voor
onafhankelijke natiestaten, gebaseerd op nationale identiteiten die voor
een deel door taal werden gedefinieerd, onder andere Woodrow Wilson en
andere geallieerde leiders bij de totstandkoming van het Verdrag van
Versailles.

De opdeling van Centraal-Europa na de Eerste Wereldoorlog laat zien hoe
taal als tweesnijdend zwaard werd ingezet bij staatsvorming. Toen het
geweld toenam, maakte een gemeenschappelijke taal de machtsuitoefening
eenvoudiger en verstevigde het de rechtsorde. Maar wanneer de prikkels
voor consolidatie verzwakten, veroorzaakten taalkundige minderheden vaak
dat meertalige staten uiteenvielen.

In het midden van de negentiende eeuw sloegen separatistische gevoelens
in de steden van het Oostenrijks-Hongaarse Rijk aan, nadat epidemieën de
Duitssprekende bevolking zwaar hadden getroffen. Praag was aan het begin
van de negentiende eeuw een Duitssprekende stad. Net als andere
stedelijke centra groeide Praag de loop van de eeuw snel, voornamelijk
door migratie: enorme aantallen landloze, Tsjechischsprekende boeren
trokken de stad binnen en werden geassimileerd. Aanvankelijk leerde men
vanuit noodzaak Duits, maar halverwege de eeuw, toen hongersnoden en
ziekte-uitbraken grote groepen Duitssprekende stadsbewoners wegvaagden,
werden zij opgevolgd door Tsjechischsprekenden. Plotseling woonden er
zoveel Tsjechischsprekenden in de stad dat de nieuwkomers niet langer
afhankelijk waren van de Duitse taal. Praag transformeerde tot een
overwegend Tsjechisch sprekende stad en werd een broedplaats voor
Tsjechisch nationalisme.

Tegenwoordig ontstaan separatistische bewegingen vaak rond taalkwesties
in meertalige landen. Dit is duidelijk in België en Canada, twee landen
die -- zoals eerder opgemerkt -- waarschijnlijk tot de eerste van de
OESO-lande behoren die in het nieuwe millennium uiteenvallen.

Weinige regeringen kunnen tippen aan de harde maatregelen die door de
\emph{Parti Québécois} in Québec worden opgelegd om taaleenheid af te
dwingen.\footnote{Rheal Seguin, `PQ staat op het punt de taalwetten aan
  te scherpen: in Quebec worden Engelse borden verboden', \emph{Globe
  and Mail}, 29 augustus 1996, p.~A1.} Wat nog verrassender is, is dat
taalklachten ook een rol speelden bij de opstart van de activiteiten van
de noordelijke separatisten in Italië, een land dat eveneens met
desintegratie kampt. Begin jaren tachtig verklaarde de Lombardische Liga
-- zoals ze destijds heette -- dat Lombardisch een zelfstandige taal
was, los van het Italiaans. Billig merkt op: `Als het programma van de
Liga succesvol was geweest in de vroege jaren tachtig en Lombardije zich
had afgescheiden van Italië en eigen staatsgrenzen had vastgesteld, had
men kunnen voorspellen dat Lombardisch steeds meer als verschillend van
het Italiaans zou worden erkend.'\footnote{Billig, \emph{op. cit.},
  p.~35.} Deze bewering is niet willekeurig, maar weerspiegelt wat er in
soortgelijke situaties is gebeurd. Zo gingen Noorse nationalisten, toen
Noorwegen in 1905 onafhankelijk werd, doelbewust te werk om kenmerken
van de `Noorse taal' te benoemen en te benadrukken die afweken van het
Deens en het Zweeds. Evenzo vervingen activisten die pleitten voor een
onafhankelijk Belarus de verkeersborden door `Witrussisch' te hanteren,
al slaagden zij er kennelijk niet in duidelijk te maken dat Witrussisch
een eigen taal is en geen dialect van het Russisch.

Nu de militaire drijfveren voor taaleenheid grotendeels tot het verleden
behoren, verwachten we dat nationale talen langzaam zullen wegkwijnen --
maar dan niet zonder weerstand. We mogen verwachten dat het bekende
gezegde `oorlog is de gezondheid van de staat' opnieuw zal worden
aangewend als middel tot herstel. Naarmate de relevantie van de
natiestaat afneemt, zullen demagogen en reactionairen oorlogen en
conflicten aanwakkeren langs etnische en tribale lijnen, zoals ooit in
het voormalige Joegoslavië en in talloze Afrikaanse landen, van Burundi
tot Somalië. Conflicten blijken nuttig, omdat zij een dekmantel bieden
voor hen die de trend naar de commercialisering van soevereiniteit
willen tegenhouden. Oorlogen vergemakkelijken tevens het handhaven van
strengere belastingregimes en het opleggen van zwaardere straffen voor
het ontlopen van de burgerlijke plichten en lasten. Daarbij versterken
zij de `wij-versus-hij'-mentaliteit binnen het nationalisme. Voor de
aanhangers van systematische dwang lijkt de commerciële soevereiniteit
-- waarbij individuen soevereiniteitsdiensten kunnen kiezen op basis van
prijs en kwaliteit -- even zondig als de opvatting dat mensen het recht
hebben de uitspraken van de paus te vetoën en tijdens de Reformatie hun
eigen weg naar redding te kiezen.

De gelijkenis valt op doordat zowel de nieuwe druktechnologie aan het
eind van de vijftiende eeuw als de informatietechnologie aan het eind
van de twintigste eeuw voorheen verborgen kennis op een bevrijdende
manier toegankelijk maakten voor individuen. Dankzij de boekdrukkunst
kwamen de Schrift en andere heilige teksten rechtstreeks beschikbaar
voor mensen, die tot dan toe afhankelijk waren geweest van priesters en
de kerkhierarchie om het Woord van God te interpreteren. De
informatietechnologie maakt het iedereen met een computerverbinding
mogelijk om informatie op te vragen over handel, investeringen en
actuele gebeurtenissen -- informatie die voorheen enkel voor personen
aan de top van overheids- en bedrijfsstructuren beschikbaar was.

\begin{quote}
`{[}T{]}e ontwikkeling van de drukkunst en de uitgeefkunst maakte het
nieuwe nationale bewustzijn mogelijk en stimuleerde de opkomst van
moderne natiestaten.'\footnote{Jack Weatherford, \emph{Savages and
  Civilization: Who Will Survive?} (New York: Fawcett Columbine, 1994),
  p.~143.} - Jack Weatherford
\end{quote}

\subsection{Rock and roll in
cyberspace}\label{rock-and-roll-in-cyberspace}

Vergeet niet dat de komst van het internet en het \emph{World Wide Web}
het nationalisme net zo hard zal ondermijnen als de introductie van
buskruit en de drukpers destijds deden. Wereldwijd verbonden computers
brengen het Latijn niet terug als universele taal, maar zorgen er wel
voor dat de handel verschuift van lokale dialecten -- zoals het Frans in
Quebec -- naar de nieuwe, wereldwijde taal van het internet en het
\emph{World Wide Web}: de taal van rock-'n-roll, Engels, zoals Otis
Redding en Tina Turner aan de wereld hebben laten zien.

Deze nieuwe media ondermijnen het nationalisme door grensoverschrijdende
verbondenheden te scheppen. Ze spreken brede doelgroepen aan, vooral op
die plaatsen waar hoogopgeleiden zich hebben verzameld. Zulke
niet-territoriale verbondenheden kunnen floreren en daarmee een basis
leggen voor een nieuw soort `patriottisme'. Of, om het anders te
verwoorden, ze vormen nieuwe `in-groepen' waarmee mensen zich kunnen
identificeren zonder daarbij hun economische rationaliteit op te
offeren. De geschiedenis van de joden in de afgelopen tweeduizend jaar
leert dat dit op de lange termijn mogelijk is, zelfs in vijandige lokale
omgevingen. Zoals William Pfaff aan het begin van dit hoofdstuk
opmerkte, is het historisch onjuist te veronderstellen dat loyaliteit
aan het land van de vaderen -- de patria -- per definitie trouw betekent
aan een instelling die op een natiestaat lijkt. Geoffrey Parker en
Lesley M. Smith geven in \emph{The General Crisis of the Seventeenth
Century} nog een duidelijker beeld: wat op het eerste gezicht
voorbeelden lijken van vroegmodern nationalisme, betreft in
werkelijkheid vaak patriotten die een veel nauwere patria verdedigen --
vaak tegen de binnendringing van een staat. Zij schrijven: `Al te vaak
blijkt een vermeende verbondenheid met een nationale gemeenschap, bij
nadere inspectie, niets dergelijks te zijn. De patria is minstens even
goed een stad of provincie als de gehele natie.'\footnote{Geoffrey
  Parker en Lesley M. Smith, \emph{The General Crisis of the Seventeenth
  Century} (London: Routledge \& Kegan Paul, 1985), p.~122.}

Zoals Jack Weatherford duidelijk toelicht in \emph{Savages and
Civilization} leidde de opkomst van de drukpers -- de eerste
massaproductietechnologie -- tot ingrijpende veranderingen bij het
ontstaan van de politiek, waarbij men steeds meer loyaliteit eiste aan
een bredere natiestaat. In 1500 waren in Europa drukpersen actief op 236
locaties, waarmee samen ongeveer 20 miljoen boeken werden
gedrukt.\footnote{Weatherford, \emph{op. cit.}, p.~144.} Gutenberg's
eerste gedrukte boek was een Latijnse Bijbeluitgave; daarop volgden
edities van andere populaire middeleeuwse teksten in het Latijn.
Weatherford laat zien dat de ontwikkeling van de drukkunst niet verliep
zoals men aanvankelijk verwachtte, namelijk dat de gemakkelijke
beschikbaarheid van teksten automatisch het gebruik van het Latijn -- en
zelfs het Grieks -- zou verspreiden. Integendeel, twee belangrijke
oorzaken verklaren waarom de drukpers het Latijngebruik niet bevorderde.

Ten eerste maakte de drukpers massaproductie mogelijk. Zoals Benedict
Anderson opmerkt: `{[}A{]}ls manuscriptkennis schaars was en als arcane
wijsheid gold, leefde gedrukte kennis van reproduceerbaarheid en
verspreiding.'\footnote{Anderson, \emph{op. cit.}, p.~90.} In 1500
spraken weinigen in Europa meerdere talen, waardoor de doelgroep voor
werken in het Latijn beperkt bleef. De overgrote meerderheid, die
slechts één taal beheerste, vormde juist een veel grotere markt van
potentiële lezers. Wat hierbij voor lezers geldt, geldt nog sterker voor
schrijvers: uitgevers hadden continu producten nodig om te verkopen.
Omdat er in de vijftiende- en zestiende eeuw maar weinig auteurs waren
die bevredigende nieuwe werken in het Latijn konden samenstellen, dwong
de marktvraag uitgevers ertoe over te stappen op publicaties in de
volkstaal. Zo droeg de drukkunst bij aan een taalkundige differentiatie
in Europa. Dit proces werd gestimuleerd door zowel de publicatie van
nieuwe werken die de identiteit van opkomende talen -- zoals het Spaans
en Italiaans -- bevestigden als door de introductie van karakteristieke
lettertypen, waaronder Romeins, Italic en de zware Gotische schriftstijl
die in Duitse uitgaven tot ver in de twintigste eeuw gangbaar bleef. De
opkomst van volkstaalpublicaties -- wat Anderson aanduidt als
`printkapitalisme' -- boekte immens succes. Opvallend daarbij was dat de
drukpers de ketterij de impuls gaf die we tegenwoordig associëren met de
denationalisering van het individu via het internet. Zo werd Luther `de
eerste bestverkopende auteur zoals we die kennen. Of om het anders te
zeggen, de eerste schrijver die zijn nieuwe boeken op basis van zijn
naam kon verkopen.'\footnote{Ibid., p.~91.} Opmerkelijk is dat Luther's
werken `niet minder dan een derde van alle Duitstalige boeken die tussen
1518 en 1525 werden verkocht' vertegenwoordigden.\footnote{Ibid.}

In veel opzichten zal de informatietechnologie van het digitale tijdperk
een deel van de megapolitieke impact van de vijftiende-eeuwse drukpers
tenietdoen bij het stimuleren en ondersteunen van de opkomst van
natiestaten. Het \emph{World Wide Web} schept een commerciële arena
waarin Engels dienstdoet als wereldtaal. Uiteindelijk zal
simultaan-vertalingssoftware deze ontwikkeling versterken, zodat bijna
iedereen effectief meertalig wordt -- wat bijdraagt aan de
denationalisering van taal en verbeelding. Net zoals de drukpers de
loyaliteit aan de dominante instelling van de middeleeuwen, de Heilige
Moederkerk, ondermijnde, verwachten wij dat de moderne
communicatietechnologie de autoriteit van de `nannystaat' zal aantasten.
Op den duur zal vrijwel elk gebied meertalig worden, lokale dialecten
zullen belangrijker worden en de samenhang van centrale propaganda
verzwakt, naarmate immigranten en sprekers van minderheidstalen de moed
vinden zich tegen assimilatie in te zetten.

\section{Militaire mystiek}\label{militaire-mystiek}

Naties vormen bij lange na geen objectieve gemeenschappen zoals
jager-verzamelaarsgroepen dat wel doen; in plaats daarvan ontstaan
naties uit een mystiek dat geworteld is in een inmiddels verdwenen
militair imperatief. Dat imperatief wilde iedere inwoner van een bepaald
grondgebied verbinden aan een identiteit die men als belangrijker
beschouwde dan het eigen leven. Zoals Kantorowicz opmerkte, is het geen
toeval dat de staat op een gegeven moment in de geschiedenis -- in
abstracte vorm of als corporatie -- verscheen als een corpusmystiek, en
dat de dood van dit nieuwe mystieke lichaam dezelfde betekenis leek te
hebben als de dood van een kruisvaarder voor de goddelijke
zaak!\footnote{E. H. Kantorowicz, geciteerd door Llobera, \emph{op.
  cit.}, p.~83.} In die zin kun je de natiestaat beschouwen als een
mystieke constructie. Maar zoals Billig aangeeft, betreft nationalisme
`een banale mystiek, zo alledaags dat het mystieke schijnbaar al lang is
verdampt.' Het bindt `ons' aan het vaderland -- die bijzondere plek die
meer is dan slechts een geografisch gebied. Het vaderland wordt
voorgesteld als iets huiselijks, vanzelfsprekend en, indien nodig, zelfs
opofferingswaardig. Vooral mannen koesteren hun eigen, met vreugde
doordrenkte herinneringen aan de mogelijkheden tot opoffering.

Nationalisten benadrukken steevast de denkbeeldige band tussen de natie
en het thuis. Volgens Billig zien zij de natie als een huiselijke
omgeving, knus binnen haar grenzen en beschermd tegen een bedreigende
buitenwereld. Wij -- als leden van het vaderland -- stellen onszelf
hierdoor al snel voor als een hechte familie. De clichés van het
nationalisme, onvermoeibaar en routinematig herhaald, omvatten talloze
alledaagse metaforen van verwantschap en identiteit. Ze verbinden de
natie met het persoonlijke gevoel van `inclusieve fitheid', een
krachtige drijfveer voor altruïsme en opoffering.

\begin{quote}
`Dat opofferingsaltruïsme daadwerkelijk bestaat bij sociale insecten,
andere niet-menselijke dieren en mensen, impliceert dat maximalisatie
van eigenbelang niet louter kan worden gedefinieerd in termen van de
wensen en behoeften van een individueel organisme. Inderdaad, de
aanwezigheid van altruïsme, met name jegens verwanten, heeft een
volledige heroverweging van de traditionele opvattingen over het
overleven van de best aangepaste in de biowetenschappen noodzakelijk
gemaakt. Dit heeft geleid tot een groeiende overtuiging dat natuurlijke
selectie uiteindelijk niet op het individu opereert.'\footnote{Shaw en
  Wong, \emph{op. cit.}, pp.~26--27.} - R. Paul Shaw en Yuwa Wong
\end{quote}

\section{Nationalisme en inclusieve
fitheid}\label{nationalisme-en-inclusieve-fitheid}

In dit boek richten we ons vooral op de objectieve ``megapolitieke''
factoren die de kosten en beloningen van menselijke keuzes beïnvloeden.
We gaan ervan uit dat mensen voortdurend op zoek zijn naar
aantrekkelijke beloningen en kosten proberen te vermijden -- een
fundamentele waarheid zoals Charles Darwin `de economie van de natuur'
noemde. Hoewel deze visie veel inzicht biedt, verklaart het najagen van
beloningen niet alles in het leven. Wel verheldert het twee van de drie
belangrijkste vormen van menselijke socialiteit, die Pierre Van Den
Berghe aanduidt als `wederkerigheid' en `dwang'. Met wederkerigheid
bedoelt hij samenwerking die voor beide partijen gunstig is. De meest
complexe en ingrijpende voorbeelden hiervan zie je bij marktinteracties:
handelen, kopen, verkopen, produceren en andere economische
activiteiten. Dwang betreft het inzetten van geweld voor eenzijdig
voordeel -- oftewel intra-specifiek parasitisme of predatie. In dit boek
en twee voorgaande werken hebben we aangetoond dat dwang een cruciaal
element van de menselijke samenleving vormt, vaak belangrijker dan men
doorgaans erkent. Dwang beïnvloedt de veiligheid van eigendom en beperkt
de mogelijkheden voor wederzijds voordelige samenwerking; het vormt
immers de basis van alle politiek. Het derde aspect binnen Van Den
Berghe's typologie van menselijke socialiteit is `verwantschapsselectie'
-- het coöperatieve gedrag dat dieren tegenover hun naaste verwanten
tonen. We bespreken verwantschapsselectie verderop uitgebreid, maar ook
dit fenomeen vormt een essentieel kenmerk van `de economie van de
natuur.'

Zoals Jack Hirshleifer verwoordde: ``De heropleving van de
darwinistische evolutietheorie van selectie, toegepast op problemen van
sociaal gedrag -- wat bekendstaat als sociobiologie -- heeft een
duidelijk economisch aspect.''

Wanneer we het volledige spectrum van het leven bekijken, streeft de
sociobiologie ernaar de universele wetten te achterhalen die de
uiteenlopende vormen van associatie tussen organismen verklaren. Waarom
zie je soms zowel geslacht als families, soms enkel geslacht, en soms
helemaal geen van beide? Waarom zwermen sommige dieren terwijl anderen
liever alleen leven? Binnen groepen ontstaan soms duidelijke
hiërarchieën, terwijl dat in andere groepen ontbreekt. Waarom delen
bepaalde soorten hun territoria, terwijl andere dat niet doen? Wat
motiveert het onbaatzuchtige gedrag van sociale insecten, en waarom komt
dat fenomeen zo zelden voor in de natuur? En wanneer worden hulpbronnen
vreedzaam toegekend en wanneer wordt er geweld ingezet? Deze vragen
formuleer je in vertrouwde economische termen. Sociobiologen onderzoeken
welke netto voordelen de waargenomen associatiepatronen bieden en welke
mechanismen ervoor zorgen dat deze sociale structuren in evenwicht
blijven. Misschien verklaart juist de opvatting over de continuïteit in
economisch gedrag tussen mens en andere levensvormen -- door een
criticus `genetisch kapitalisme' genoemd -- de vijandige houding van
sommige ideologen tegenover de sociobiologie\ldots{}

Wij integreren sociobiologie in onze analyse van het nationalisme omdat
dit perspectief inzicht geeft in die aspecten van de menselijke natuur
die systematische dwang rechtvaardigen. Wij delen de overtuiging van
natuurwetenschapper Cohn Tudge, auteur van \emph{The Time Before
History}, dat we -- voordat we de hedendaagse wereld doorgronden, laat
staan een toekomstvisie ontwikkelen -- allereerst de inleiding op de
geschiedenis moeten begrijpen. Dat houdt in dat we onszelf moeten
plaatsen in een veel bredere tijdsschaal. Tudge wijst ons erop dat
achter de alledaagse beroeringen van ons leven talrijke diepgewortelde
en krachtige krachten schuilgaan die uiteindelijk ons en al onze
medebewoners beïnvloeden\ldots{} Wij vermoeden dat in deze diepere,
krachtige krachten een genetisch bepaalde motivatie schuilgaat die het
nationalisme fundeert. Zoals Hirshleifer opmerkt, in navolging van Adam
Smith en R. H. Coase, `zijn menselijke verlangens uiteindelijk adaptieve
reacties die gevormd zijn door de biologische aard van de mens en zijn
positie op aarde.' Dit komt duidelijk naar voren in de overwegend
biologische verwijzingen binnen de meeste discussies over nationalisme.
Zelfs in de Verenigde Staten, een opvallend multietnische natie,
personifieert men de overheid in familiale termen als `Oom Sam.'

De biologische erfenis

Kortom, als we de voortdurende evolutie van de menselijke samenleving
willen begrijpen, moeten we rekening houden met de menselijke natuur, de
oorsprong van soorten en hun ontwikkeling door natuurlijke selectie. In
dit geval onderzoeken we de waarschijnlijke reactie van de mens op de
nieuwe omstandigheden die de informatietechnologie met zich meebrengt.
We richten ons vooral op de reactie op de opkomst van de cybereconomie
en de vele gevolgen daarvan, waaronder een economische ongelijkheid die
uitgesprokener is dan ooit tevoren. Sleutels tot een deel van die
verwachte reactie vind je in onze genetische erfenis.

Wanneer een nieuwe soort verschijnt, verwerpt zij niet al het DNA dat
zij in haar vorige vorm bezat, maar bouwt zij er juist op voort. Het
genetische verschil tussen mens en chimpansee betreft minder dan twee
procent van hun DNA -- we delen dus ruim 98 procent -- en een deel
daarvan herleiden we tot oerprimitieve organismen, diep verankerd in de
evolutionaire geschiedenis.

\section{Genetische inertie}\label{genetische-inertie}

Menselijke culturen bevatten ook universele elementen, waarvan sommige
daadwerkelijk van premenselijke voorouders zijn overgeërfd. Hoe we
voedsel zoeken, ons voortplanten, gezinnen stichten, met onbekenden
omgaan en ons verdedigen, vormt samen een complexe mix van instinct en
cultuur met zeer primitieve wortels. Tegelijkertijd kunnen deze
gedragingen zich op een moderne manier aanpassen, zoals kenmerkend is
voor de hedendaagse natiestaat. Als we culturen op deze wijze bekijken,
trekken we parallellen met de genetische evolutie. Drie belangrijke
verschillen springen hierbij eruit. Ten eerste worden culturen
overgedragen via de uitwisseling van informatie tussen mensen in plaats
van via de genetische keten van generatie op generatie. Ten tweede
kunnen culturen -- wellicht in mindere mate dan vaak gedacht --
aangepast worden door doelbewust en intelligent handelen. Ten derde
passen ze zich aan aan een snel veranderende omgeving van kosten en
beloningen, die veel sneller muteert dan genetische eigenschappen.
Fysiek lijken we nog sterk op onze voorouders van 30.000 jaar geleden,
maar cultureel hebben we een enorme sprong voorwaarts gemaakt.

\subsection{Evolutionaire modellen}\label{evolutionaire-modellen}

Er bestaan twee biologische modellen die de evolutie van soorten
proberen te verklaren. De heersende wetenschappelijke consensus volgt de
neo-darwinistische benadering: willekeurige genetische mutaties leiden
tot diverse fysieke varianten. De meeste van deze vormen bieden geen
overlevingsvoordeel -- zoals bijvoorbeeld te zien is bij de albino merel
-- en sterven daarom doorgaans uit. Slechts enkele varianten bevorderen
overleving en verspreiden zich binnen de soort. Hoewel deze theorie nog
gepaard gaat met veel moeilijkheden -- problemen die wetenschappers in
de komende eeuw wellicht oplossen -- vormt het idee dat willekeur en het
overleven van voordelige aanpassingen centraal staan wel de basis van de
huidige wetenschappelijke orthodoxie.

Een alternatief model is een variant op de theorie van de vroege
twintigste-eeuwse Franse filosoof Henri Bergson, die stelde dat de
natuur een niet-willekeurig, creatief doel nastreeft -- een intelligente
kracht die voortdurend op zoek is naar oplossingen. Dit idee vindt
weerklank in het werk van hedendaagse denkers zoals \emph{David Layzer}
en \emph{Stephen Jay Gould}, die benadrukken dat genetische variatie
niet louter willekeurig verloopt, maar duidelijke tendensen
vertoont.\footnote{Zie Stephen Jay Gould, `Evolutionaire biologie van
  beperkingen,' \emph{Daedalus}, lente 1980, en David Layzer, `Altruïsme
  en natuurlijke selectie,' \emph{Journal of Social and Biological
  Structures} (1978), geciteerd door Howard Margolis, \emph{Egoïsme,
  altruïsme en rationaliteit} (Chicago: \emph{University of Chicago
  Press}, 1984).} Dit betreft geen kreationisme in de strikte bijbelse
zin, maar het omzeilt wel veel van de problemen die het orthodoxe
darwinisme met zich meebrengt.

„De belangrijkste theoretische bijdrage van de sociobiologie ligt in de
uitbreiding van het `fitness'-begrip naar dat van `inclusieve fitness.'
Een dier kan immers zijn genen rechtstreeks doorgeven via zijn eigen
nageslacht of indirect via de voortplanting van verwanten met wie het
bepaalde proporties genen deelt. Daarom mag men verwachten dat dieren,
naarmate zij genetisch nauwer verwant zijn, zich coöperatiever opstellen
en zo elkaars overlevingskansen vergroten. Dit principe noemt men
verwantschapsselectie. Kortom, dieren vertonen nepotisme: zij geven de
voorkeur aan familieleden boven aan niet-verwanten en aan naaste
familieleden boven aan verre verwanten. Bij mensen kan dit bewust
gebeuren, maar vaker verloopt het onbewust.``\footnote{Van den Berghe,
  op. cit., p.~96.}

--- Pierre van den Berghe

\section{Genetisch beïnvloede motivationele
factoren}\label{genetisch-beuxefnvloede-motivationele-factoren}

De introductie van het concept `inclusieve fitness' door W. D. Hamilton
in 1963 in \emph{The Evolution of Altruistic Behavior} gaf een flinke
impuls aan het biologische perspectief op menselijk gedrag. Hamilton
merkte op dat mensen van nature vaak op eigenbelang gericht zijn, maar
dat zij ook wel eens altruïstische of zelfopofferende handelingen
verrichten die op het eerste gezicht weinig voordeel bieden. Hij
probeerde deze schijnbare tegenstrijdigheden te verklaren door te
stellen dat de fundamentele eenheid van maximalisatie niet het individu
zelf is, maar het gen. Volgens hem streeft elk organisme ernaar niet
alleen het eigen welzijn te bevorderen, maar ook wat hij `inclusieve
fitness' noemt. Daarmee bedoelde hij niet enkel het veiligstellen van de
eigen overleving in darwinistische zin, maar ook het vergroten van de
voortplantingskansen en overleving van naaste familieleden die dezelfde
genen delen.\footnote{Zie W. D. Hamilton, `The Evolution of Altruistic
  Behavior', \emph{American Naturalist}, 1963, pp.~346--54.} Hamiltons
these over `inclusieve fitness' werpt licht op vele anderszins
raadselachtige kenmerken van menselijke samenlevingen, waaronder
politieke fenomenen binnen natiestaten.

\subsection{Altruïsme: misbenaming of fossiele
verwantschapsselectie?}\label{altruuxefsme-misbenaming-of-fossiele-verwantschapsselectie}

Volgens Van Den Berghe richt `altruïsme' zich in de praktijk
hoofdzakelijk op familie -- en dan vooral op naaste familie -- waardoor
de term eigenlijk misleidend blijkt te zijn. Het vertegenwoordigt in
feite het ultieme genetische egoïsme en vormt een blinde poging om
inclusieve fitness te maximaliseren.\footnote{Van den Berghe, op. cit.,
  p.~96.} Dit betekent echter niet dat alle vormen van hulpvaardigheid
uitsluitend afhangen van nauwe genetische verwantschap, zoals Hamilton
en Van Den Berghe veronderstellen. De onzekerheden die voortkomen uit
het feit dat mensen zich seksueel voortplanten in plaats van asexueel te
klonen, zorgen er bijna voor dat een drang tot het vergroten van de
inclusieve fitness ook een behoorlijke mate van `altruïstisch' gedrag
stimuleert -- hetgeen soms zelfs allelen bevordert die niet met het
`egoïstische gen' geassocieerd worden. Zo kan het gebeuren dat mensen
hulp verlenen in de veronderstelling dat zij naaste familie
ondersteunen, terwijl dit in werkelijkheid niet altijd het geval is. De
vader die een opofferende daad verricht voor zijn nakomelingen is
mogelijk niet zijn biologische ouder, maar hij gedraagt zich er wel
naar.\footnote{Dezelfde redenering geldt uiteraard voor de zoon of
  dochter die zich opoffert voor degenen die hij als zijn broers en
  zussen beschouwt, ook al is dat in werkelijkheid niet zo.} Dit is niet
louter een onderwerp voor soapseries, maar raakt een fundamenteel
vraagstuk: het voortbestaan van `egoïstische genen' wordt waarschijnlijk
juist bevorderd als iedere vermeende ouder zich gedraagt alsof hij
daadwerkelijk de biologische ouder is, ook al bestaat de kans dat dit
niet zo is.

Wanneer we dit in de juiste context plaatsen, valt -- zoals Hirshleifer
opmerkt -- op dat veel van de ogenschijnlijke paradoxen rond `altruïsme'
eigenlijk semantische verwarringen zijn. Mensen raken daardoor de
competitieve context uit het oog waarin hulpverlening juist een
overlevingsvoordeel kan opleveren. Hirshleifer stelt: `Als een keuze
voor een altruïstische strategie levensvatbaar moet zijn in de
concurrentie met niet-altruïsme, dan moet altruïsme meer bijdragen aan
het eigen overleven dan niet-altruïsme, en kan het dus eigenlijk niet
echt altruïsme zijn.' We kunnen al deze verwarring voorkomen door de
term `altruïsme' achter ons te laten en in plaats daarvan de vraag te
stellen: wat bepaalt eigenlijk het objectieve verschijnsel dat wij
hulpverlening noemen?\footnote{Hirshleifer, op. cit., p.~179.}

Deze vraag is met name intrigerend bij hulpverlening binnen
familiebanden. Hamiltons fundamentele formulering van inclusieve fitness
bevatte een biologische kosten-batenanalyse, waarin een individu --
ofwel `het gen dat het hulpverleningsgedrag beheerst' -- de overleving
van een identieke kopie van zichzelf net zo waardeert als zijn eigen
overleving. Daarom hangt de bereidheid om hulp te verlenen -- of zelfs
offers te brengen -- af van de kans dat een ander individu dat identieke
gen draagt. Concreet dicteert een gen voor hulpverlening binnen
familiebanden dat een man (onder gelijke omstandigheden) zijn leven mag
opofferen als hij daarmee twee broers of zussen, vier halfbroers of
halfzussen, acht neven of nichten, enzovoort kan redden.\footnote{Ibid.}

\section{Kansproblemen bij inclusieve
fitness}\label{kansproblemen-bij-inclusieve-fitness}

Hoewel dit biologische concept in eerste instantie helder lijkt, onthult
een nadere beschouwing enkele complicaties. Zo impliceert het
bijvoorbeeld niet per se dat het feit dat broers, zussen en kinderen een
kans van 50 procent hebben om een identiek gen te erven, inhoudt dat dit
gen ook daadwerkelijk tot expressie komt. Elk individu draagt twee
kopieën van elk gen -- één van de vader en één van de moeder. Dit
betekent dat nakomelingen altijd slechts de helft van de genen van een
ouder erven. Bovendien bestaat er altijd het risico op mutatie tijdens
de voortplanting, wat -- hoe onwaarschijnlijk ook -- de zekerheid van de
genetische kosten-batenanalyse ondermijnt. Als we de metafoor `het gen
als optimalisator' serieus nemen, vormt het geval van een vader die niet
de biologische ouder blijkt te zijn slechts het duidelijkste voorbeeld
van een breder probleem. Als het immers daadwerkelijk gaat om de
optimalisatie van de overleving van het `egoïstische gen' door offers te
brengen voor naaste familie, geldt iedere vervanging van een identieke
kopie van dit gen door een ander allel als één van die complexe
mechanismen die Moeder Natuur hanteert.

Onzekere gevolgen

Altruïsme dat zich op verwanten richt, brengt dan ook zijn eigen
problemen met zich mee. Niet alleen speelt het
waarschijnlijkheidsprobleem rond het `egoïstische gen' een rol --
schijnverwanten van de gastheer dragen mogelijk niet exact dezelfde
kopieën -- maar bovendien is het lastig om, onder omstandigheden van
onzekerheid, vast te stellen of een bepaalde opofferingshandeling
primair voordelig is voor verwanten in plaats van voor anderen. (Een
opoffering die vooral anderen bevoordeelt, kan immers de inclusieve
fitness van het `egoïstische gen' schaden doordat de kans dat het in
volgende populaties voorkomt, afneemt.)

Overweeg eens een afschuwelijk voorbeeld, geïnspireerd door het nieuws
dat we tijdens het schrijven meemaakten. Stel je voor dat een ouder in
Dunblane, Schotland, op het laatste moment hoort dat een gewapende
maniak op weg is naar een lokale school, met als doel er schade aan toe
te richten. Door direct in te grijpen, kon die ouder de heroïsche --
maar wellicht gedoemde -- actie ondernemen om de maniac te confronteren
en zo mogelijk de schoolkinderen te redden.

Of misschien niet.

Zelfs een meedogenloze maniak, die vastbesloten is elk kind op aarde te
vermoorden, kan maar beperkte schade aanrichten voordat hij zonder
munitie komt of door anderen wordt overmeesterd. Als de ouder er echter
voor kiest niet in te grijpen, overleven zijn of haar kinderen
hoogstwaarschijnlijk toch, net als de meeste schoolkinderen. De schade
die door een moedige daad had kunnen worden voorkomen, zou anders
waarschijnlijk de kinderen van anderen treffen. Door zijn of haar leven
te riskeren om in essentie andermans kinderen te redden, verlaagt de
ouder feitelijk zijn `inclusieve fitness'. Daarmee berooft hij zijn
eigen nakomelingen van de volledige genetische bijdrage van beide
ouders, waardoor zij in de darwinistische strijd in een nadeligere
positie komen te staan.

Hoewel dit voorbeeld wat geforceerd lijkt, blijft het realistisch. Het
illustreert dat er talloze situaties bestaan waarin zowel grootschalige
als bescheiden hulpbetonen positieve effecten hebben. Vaak beperken de
directe begunstigden van zulke daden zich niet tot de naaste familie.
En, ironisch genoeg, zoals we hierna zullen zien, kan dit zelfs
bijdragen aan het overlevingsvoordeel dat mensen met minder kieskeurige
hulpgenen in staat stelde alle eeuwen van ontberingen tot nu toe door te
komen.

Altruïsme en genetische inertie

Als we aannemen dat de `selfish gene'-these een nauwkeurige verklaring
biedt voor wat menselijk gedrag motiveert, is het te simpel te
veronderstellen dat het daaruit voortvloeiende hulpzame of offerende
gedrag zich uitsluitend richt op echte verwanten. Beperkte kennis maakt
het in sommige situaties lastig om nauwkeurig vast te stellen wie tot de
familie behoort. Zelfs wanneer de verwantschap duidelijk is, kan de
aanwezigheid van een specifieke `selfish gene' binnen de groep slechts
op basis van waarschijnlijkheden worden vastgesteld. Tot voor kort kon
men echte genetische markers bij individuen niet onderscheiden, en we
zijn nog ver verwijderd van het praktisch kunnen vaststellen welke
naaste verwanten daadwerkelijk de `selfish gene' tot uiting brengen die
hun overleving bevordert. Daarbij speelt ook het probleem mee dat
voordelen strikt beperkt worden tot verwanten en niet worden doorgegeven
aan anderen.

Uit ervaring blijkt bovendien dat mensen hun `zorginstinct' soms richten
op niet-verwanten wanneer geschikte familieleden ontbreken. Een
duidelijk voorbeeld hiervan is het gedrag van ouders tegenover
geadopteerde kinderen, maar ook kinderloze personen tonen vergelijkbare
betrokkenheid bij hun huisdieren. Het is niet ongewoon dat zulke mensen
zichzelf ernstig verwonden of zelfs hun eigen leven riskeren om
bijvoorbeeld een kat te redden die in een boom vastzit. Elk jaar komen
bovendien aanzienlijke aantallen mensen om bij huishoudelijke ongevallen
die op de een of andere manier samenhangen met het gevaar voor
huisdieren. Wat geldt voor huisdieren geldt in nog grotere mate voor
adoptiekinderen. Het is zeker niet overdreven te stellen dat
adoptieouders hun kinderen vaak behandelen `alsof' het eigen bloed
betreft, waardoor het begrip `verwantschapsselectie' een andere lading
krijgt.

Dergelijke gevallen weerleggen de `selfish gene'-theorie niet zozeer als
sommigen beweren. Integendeel, we zien voorbeelden van mensen die zich
gedragen alsof zij offers brengen voor nauwe familiebanden om zo hun
eigen inclusieve fitness te vergroten -- wat we `genetische inertie'
noemen. Met andere woorden, hun gedrag weerspiegelt het feit, zoals
Howard Margolis opmerkt in \emph{Selfishness, Altruism and Rationality},
dat `de menselijke samenleving sneller verandert' dan onze genetische
samenstelling. Daarom gedragen mensen zich nog steeds `in wezen alsof ze
tot een kleine groep jagers-verzamelaars behoren.'\footnote{Margolis,
  op. cit., p.~32.}\\
\textgreater{} Het waren kleine, door inbroring gekenmerkte populaties
van enkele honderden individuen. \ldots{} Leden van de stam, hoewel zij
waren onderverdeeld in kleinere verwantengroepen, zagen zichzelf als één
volk, geïsoleerd van de buitenwereld en met elkaar verbonden door een
web van verwantschap en huwelijksbanden, waardoor de stam feitelijk een
superfamilie vormde. Een hoge mate van inbroring verzekerde dat de
meeste echtgenoten tevens verwanten waren.\footnote{Van Den Berghe, op.
  cit., p.~98.}

Samenvattend waren etnische groepen gedurende het hele menselijke
bestaan vóór de komst van de landbouw `inbroedende superfamilies.' Door
de hechte band tussen familie en ingroep kan er een genetisch bepaalde
neiging ontstaan om de eigen groep als verwanten te zien. Het is goed
voor te stellen dat dergelijk gedrag in het verleden overlevingsvoordeel
bood, toen elk lid van zo'n superfamilie nauw verwant was. Zoals
Margolis suggereert, is het aannemelijk dat binnen zulke kleine, hecht
verwante groepen alomvattend eigenbelang -- los van enige verwachting
van wederkerigheid of vergelding -- op zichzelf al voldoende motivatie
bood om bij te dragen aan het groepsbelang. We kunnen dan stellen dat
een zekere drang tot groepsgericht handelen voortleeft als een fossiel
van verwantenaltruïsme.\footnote{Margolis, op. cit., p.~32.} Met andere
woorden, doordat we het genetische profiel van jagers-verzamelaars in
ons dragen, weerspiegelt ons gedrag tegenover ingroepen de vorm van
`altruïsme' die nodig is om het overlevingssucces van groepen, gevormd
door inbroedende superfamilies, optimaal te ondersteunen.

Naar verluidt, zoals Margolis speculeert, heeft de neiging tot
groepsbelanggericht gedrag -- voortkomend uit wat sommigen `fossiel
verwantenaltruïsme' of genetische traagheid noemen -- bijgedragen aan
het overleven van Homo sapiens, terwijl andere humanoïde soorten
uitstierven.\footnote{Ibid.}

\subsection{Epigenese}\label{epigenese}

Wij beschouwen dit `alsof'-gedrag als een treffend voorbeeld van
epigenese, oftewel de aangeboren neiging van genetisch beïnvloede
motivatiefactoren om de voorkeur te geven aan de ene keuze boven de
andere. Met andere woorden, onze geest is geen onbeschreven blad, maar
functioneert als een harde schijf met vooraf ingestelde circuits die
bepaalde reacties sneller laten iften en aantrekkelijker maken dan
andere. Volgens deze stelling denken wij automatisch in termen van een
out‑groep -- die we associëren met vijandigheid of onverschilligheid --
en een in‑groep, waaraan wij doorgaans sterke gevoelens van
vriendelijkheid en loyaliteit koesteren, meestal gericht op
verwanten.\footnote{Shaw en Wong, op. cit., pp.~68-74.}

Deze epigenetische aanleg om binnen een groep te handelen alsof het om
naaste familieleden gaat, schept een kwetsbaarheid voor manipulatie.
Nationalisten hebben hier vaak op ingespeeld om de bereidheid tot
opoffering voor de staat aan te wakkeren. In dat opzicht valt het dus
niet te verwonderen dat nationalistische propaganda overal doordrenkt is
met de taal van verwantschap.

\begin{quote}
``Door de stem van haar schrikwekkende kanon roept het mooie Frankrijk
haar kinderen op. Soldaten om ons heen bewapenen zich. Vooruit, vooruit,
want het is onze moeder die roept.''\footnote{Geciteerd door Shaw en
  Wong, op. cit., p.~91.} - lied van Franse soldaten
\end{quote}

\subsection{vals verwantschap}\label{vals-verwantschap}

Overweeg de sterke neiging van politici om de staat te beschrijven met
termen die zij uit de verwantschap ontlenen. Zo noemt men de natie `ons
vaderland' of `ons moederland' en wordt over haar burgers gesproken als
over `wij', `familieleden' of onze `broers en zussen'.\footnote{Zie
  Billig, op. cit., p.~71.} Dat staten als Frankrijk, China en Egypte --
ondanks hun culturele verschillen -- zulke vergelijkingen hanteren, is
volgens ons geen toevalligheid in de retoriek, maar een duidelijk
voorbeeld van `epigenese', oftewel de invloed van genetisch bepaalde
motivatiefactoren die de mens van nature bepaalde voorkeuren laten
ontwikkelen.

Hoe werkt deze epigenese? Het identificatiemechanisme dat we inzetten om
de emotionele verbondenheid met de natiestaat te stimuleren, maakt
gebruik van middelen die in de oeroude tijden als herkenningspunten voor
verwantschap dienden -- `om de zorgen van het individu over zijn
inclusieve fitheid te koppelen' aan de belangen van de staat.\footnote{Shaw
  en Wong, op. cit., p.~106.} Shaw en Wong benoemen bijvoorbeeld vijf
middelen die moderne natiestaten aanwenden om hun bevolking tegen
buitenstaanders te mobiliseren:

\begin{enumerate}
\def\labelenumi{\arabic{enumi}.}
\tightlist
\item
  een gemeenschappelijke taal,
\item
  een gedeeld thuisland,
\item
  overeenkomende fenotypische kenmerken,
\item
  een gemeenschappelijke religieuze erfenis en
\item
  het geloof in een gezamenlijke afstamming.\footnote{Ibid.}
\end{enumerate}

Deze kenmerken hebben in primitieve tijden de etnische kern
waarschijnlijk onderscheiden. Veel van de aantrekkingskracht van het
nationalisme ontleent men aan de wijze waarop men deze
identificatiemiddelen heeft overgenomen en verpakt in de taal van
verwantschap -- zoals geïllustreerd door het eerder geciteerde Franse
soldatengezang. Zulke mobilisatie-instrumenten, die de staat aanduiden
als `het vaderland' of `het moederland', zijn wereldwijd gangbaar omdat
ze effectief blijken te zijn.

\subsection{Genetische boekhouding}\label{genetische-boekhouding}

Het imaginaire karakter van de vermeende verwantschapsbanden tussen
burgers en de staat blijkt uit het gebrek aan variatie dat je bij echte
familiebanden wel aantreft. Zelfs binnen grote families, waar iedereen
met elkaar vernocht is, verschilt de mate van verwantschap: ouders,
broers en zussen vormen de naaste kring, terwijl grootouders, neven en
nichten minder nauw verbonden zijn en zelfs verre, kussende neven en
nichten nauwelijks meer kans hebben dan vreemden om een bepaald gen met
elkaar te delen. Tegenwoordig delen echtgenoten doorgaans geen nauwe
verwantschap, in tegenstelling tot wat in de steentijd het geval was.
Men definieert echte verwantschap wiskundig via de
`verwantschapscoëfficiënt', een maat voor genetische overlap die
Hamilton berekende.\footnote{Zie Hamilton, op. cit., en W. D. Hamilton,
  \emph{The Genetical Evolution of Social Behavior, land!!}, Theoretical
  Biology vol.~7, pp.~1--16, 17--52.}

Daarentegen presenteert men de nationale `familie' als volledig en
flexibel samenvallend met de territoriale grenzen van de staat.
Nationaliteit verspreidt zich als een vloeistof gelijkmatig over iedere
spleet binnen die strikt afgebakende perken. Benedict Anderson stelt:
`In de moderne opvatting is staatssouvereiniteit volledig, vlak en
gelijkmatig van kracht over elke vierkante centimeter van een wettelijk
afgebakend grondgebied.'\footnote{Anderson, op. cit.} Daarbij komt dat
de coëfficiënt van de denkbeeldige verwantschap altijd één bedraagt
wanneer het opoffering voor de staat betreft.

Het samenvoegen van inclusieve fitheid met de natiestaat biedt
interessante inzichten in de neiging van mensen om de veranderingen van
het nieuwe millennium te omarmen of zich ertegen te verzetten. Zoals we
eerder toonden, berustten alle samenlevingsvormen vóór het
informatietijdperk op territoriale grenzen. Men organiseerde zich ofwel
rondom het thuisgebied van de etnische kern, of -- zoals bij de
natiestaat -- op basis van groepssolidariteit om gezamenlijk kracht te
mobiliseren voor de verdediging van een lokaal grondgebied tegen
buitenstaanders. Buiten het eigen directe grondgebied beschouwde men
vreemdelingen dan als vijanden, wat vanuit de aannames van kinselectie
in de oertijd volstrekt logisch was. Toen de mens in zijn huidige
genetische vorm opkwam, waren stamleden nauw verwant en behoorden zij
tot een etnische kern, een `inteeltsuperfamilie.'

Bovendien bood de logica van verwantschapsselectie het individu een
concrete economische reden om de voorspoed en overleving van directe
verwanten gelijk te stellen aan die van de hele stam of superfamilie.
Een lid van een jager-verzamelaarsstam was immers sterk afhankelijk van
het succes van de gehele stam voor zijn eigen welzijn. Er bestond geen
sprake van onafhankelijk bezit, noch had een individu of familie
daadwerkelijk de mogelijkheid om los van de stam te overleven en te
floreren. Hierdoor raakte het eigenbelang van het individu nauw verweven
met dat van de groep. Zoals Hirshleifer verwoordde: `Voor zover de leden
van een groep een gemeenschappelijk lot delen, wordt het elkaar helpen
gezien als een vorm van eigenhulp.'\footnote{Hirshleifer, op. cit.,
  p.~188.}

``Blijkbaar gaat de primitieve mens ervan uit -- en de Lovedu, die als
afspiegeling van honderden vergelijkbare volkeren kunnen worden
beschouwd -- dat een samenleving er altijd zo uitziet dat ieders positie
exact gelijk is.''

\subsection{Nieuwe omstandigheden, oude
genen}\label{nieuwe-omstandigheden-oude-genen}

Tegenwoordig stelt microtechnologie ons in staat om omstandigheden te
creëren die sterk afwijken van de omgevingen waartegen we genetisch
gezien gewend raakten in het Stenen Tijdperk. Tegelijkertijd zorgt
informatietechnologie voor economische ongelijkheid op een schaal die
onze voorouders in het ongerepte, egalitaire Stenen Tijdperk nooit
hebben ervaren. Bovendien schept informatietechnologie supraterritoriale
activa die het gewicht van de traditionele ingroep -- de natiestaat --
doen afnemen. Ironisch genoeg kunnen deze nieuwe cyberactiva juist meer
waard blijken te zijn, juist doordat ze ver van huis verblijven. Dit
effect neemt des te meer toe als er, als reactie op de groeiende
economische ongelijkheid in de welvarende industriële landen, een
afgunstige tegenreactie ontstaat. Activa die ver weg worden gehouden
stijgen vervolgens in waarde, omdat ze niet alleen minder vatbaar
blijken voor jaloezie, maar ook buiten het bereik blijven van diegenen
die je het meest kwaad willen doen -- namelijk je eigen natiestaat.

\subsection{Diseconomieën van de natuur en
nationalisme}\label{diseconomieuxebn-van-de-natuur-en-nationalisme}

Misschien toont het belang van epigenese bij het vormen van houdingen
wel dat men tot nu toe nauwelijks stil heeft gestaan bij de ironie van
ingroepidentificatie in relatie tot de moderne natiestaat. In de moderne
periode heeft de logica van geweld de oorspronkelijke drijfveer, die de
neiging veroorzaakte om de inclusieve fitheid met de eigen groep te
identificeren, doen verwateren. Waarom? Omdat door de identificatie van
de inclusieve fitheid van het individu met een nationale groep, in
plaats van in te zetten op het overleven en de voorspoed van naaste
familie, de waarde van iedere daad van opoffering voor die naasten
vrijwel onbeduidend is geworden. De typische moderne natiestaat is
simpelweg te groot om een statistisch significante
`verwantschapscoëfficiënt' tussen een individu en de overige burgers --
die aanspraak op hem maken -- te realiseren. Niet alleen is het aandeel
van naaste verwanten binnen de groep sterk afgenomen -- van bijna
volledige samenhang in het Stenen Tijdperk tot een vrijwel
verwaarloosbaar biologisch spoor in de twintigste eeuw --, ook is de
`verwantschapscoëfficiënt' tussen een individuele burger en de rest van
de mensheid doorgaans niet hoger dan wanneer men naar de gehele
wereldbevolking kijkt. Een ingroep met tientallen of zelfs honderden
miljoenen leden (of, bij de Chinezen, meer dan een miljard) wordt
zodanig gigantisch dat het effect van elke opoffering of elk voordeel
als druppel in de oceaan wordt verwaterd. Logisch gezien kan een moderne
nationalist zich dan -- in tegenstelling tot de jager-verzamelaar uit
het Stenen Tijdperk -- niet redelijkerwijs beroepen op het idee dat een
daad van opoffering of hulp aan zijn groep de overlevingskansen van zijn
familie op een betekenisvolle wijze zou vergroten.

Hoewel nationale economieën in de moderne tijd als de fundamentele
maatstaven voor het meten van welzijn fungeerden, vormde de last die
namens de natie -- de eigen ingroep -- werd opgelegd de grootste
belemmering voor het succes van het talentvolle individu -- en daarmee
ook voor dat van zijn naasten. Dit gold althans voor hen die zich
primair bezighielden met een op wederkerigheid gebaseerde, in plaats van
met een dwingend opgelegde, socialiteit -- als we de
indelingscategorieën van menselijk gedrag volgens Van Den Berghe
herzien.\footnote{Van Den Berghe, op. cit., p.~97.}

De logica van de natiestaat impliceert dat de hoogste prijs van
burgerschap opoffering -- zelfs de dood -- is. Zoals Jane Bethke
Elshtain opmerkte, indoctrineren natiestaten hun burgers meer tot
opoffering dan tot agressie: `De jongeman gaat de oorlog in niet zozeer
om te doden als wel om te sterven, om zijn specifieke lichaam op te
offeren voor dat van het grote lichaam, het politieke
lichaam.'\footnote{J. B. Elshtain, `Sovereignty, Identity, Sacrifice',
  in M. Ringrove en A. J. Lerner (red.), \emph{Reimaging the Nation}
  (Buckingham, Engeland: \emph{Open University Press}, 1993), zoals
  opgemerkt door Billig, op. cit.}

Ook van belastingbetalers wordt verwacht dat zij opofferingen brengen.
Het betalen van belastingen -- net als het dragen van wapens -- geldt
als een plicht en is geen ruiltransactie waarbij men geld inruilt voor
een product of een dienst van gelijke of hogere waarde. Dat wordt in de
volksmond erkend. Men spreekt van een `belastinglast', terwijl men nooit
spreekt van een `voedsellast' bij boodschappen, van een `autolast' bij
de aanschaf van een auto of van een `vakantielast' voor reizen;
commerciële aankopen vormen immers doorgaans een eerlijke ruil, anders
zouden kopers daar niet mee instemmen.

Op deze wijze laat het nationalisme zien hoe epigenese de logica van de
darwinistische `natuureconomie' kan omkeren. De natiestaat maakte
systematische, territoriaal gebaseerde roofzucht mogelijk. In
tegenstelling tot de uitdagingen waarmee jagers-verzamelaars in het
Stenen Tijdperk werden geconfronteerd, bleek aan het eind van de
twintigste eeuw de voornaamste parasiet én het voornaamste roofdier van
het individu niet de `buitenstaander', de buitenlandse vijand, maar
juist de vermeende belichaming van de eigen ingroep -- de lokale
natiestaat. Het grootste voordeel van de opkomst van activa die de
grenzen van territorium overstijgen in het informatietijdperk lag er
juist in dat deze buiten het bereik vielen van de systematische dwang
die de lokale natiestaat -- binnen wiens grondgebied het zogenaamd
soevereine individu leefde -- uitoefende.

Als onze visie klopt, maakt microtechnologie het technisch mogelijk voor
individuen grotendeels te ontsnappen aan de lasten van een ondergeschikt
burgerschap. Zij zullen als extranationale soevereinen over hun eigen
bestaan opereren -- geen onderdanen meer in de nieuwe `Virtual City' --
en hun loyaliteit vastleggen via contracten of particuliere
overeenkomsten, op een wijze die doet denken aan het premoderne Europa,
waar kooplieden commerciële verdragen en charters veiligstelden om zich
te beschermen tegen willekeurige inbeslagnames van eigendommen en om
vrijstelling van feodaal recht te verkrijgen.\footnote{Zie Abu-Lughod,
  op. cit., p.~90.}

Binnen de cybercultuur krijgen succesvolle personen vrijstelling van de
burgerschapsverplichtingen die voortkomen uit een toevallige geboorte.
Zij zullen zichzelf niet langer primair als Brits of Amerikaan zien,
maar als extranationale bewoners van de hele wereld, die toevallig in
één of meerdere lokale gemeenschappen wonen.

\section{De cybereconomie en ons genetisch
erfgoed}\label{de-cybereconomie-en-ons-genetisch-erfgoed}

Het addertje onder het gras is overigens dat zowel dit technologische
wonder als het economische fenomeen -- namelijk het ontsnappen aan de
tirannie van plaatsgebondenheid -- afhankelijk is van de bereidheid van
individuen om een aanzienlijk deel van hun rijkdom en toekomst in handen
van vreemden te leggen. Volgens een strikte genetische boekhouding
zouden deze vreemden immers niet per se genetisch minder nauw met ons
verwant zijn dan de meeste `mede-burgers', op wie we de afgelopen eeuwen
hebben vertrouwd.

De vraag is of de verontrustende gevolgen van de onderlinge
verbondenheid binnen een natiestaat negatieve of juist positieve
signalen geven voor de cybereconomie. Zullen de `achterblijvers', die
dreigen de voordelen van dwangmatige herverdeling te verliezen, de
ondergang van de natiestaat ervaren als een aanval op hun verwanten? De
eerste 25 jaar van dit nieuwe millennium zullen laten zien hoe het
afloopt. De emotionele reacties kunnen complex worden. Het feit dat 115
miljoen mensen in de twintigste eeuw hun leven gaven in de strijd voor
natiestaten, toont schrijnend de kracht van onze genen aan.\footnote{Charles
  Tilly, `Collective Violence in European Perspective,' in T. R. Gurr
  (red.), \emph{Violence in America}, vol.~2, Protest, Rebellion, Reform
  (Newbury Park, Calif.: Sage Publications, 1989), p.~93.} Het maakt
duidelijk dat velen het voortbestaan van hun natie als een kwestie van
leven en dood zagen. De vraag blijft of deze houding doorwerkt in een
nieuw tijdperk met andere megapolitieke prioriteiten.

Dat offers, gemotiveerd door genetische verbondenheid, voor de
natiestaat vaak ingingen tegen het evolutionaire principe van
verwantschapsselectie, bewijst bovendien dat mensen zich flexibel genoeg
aanpassen aan omstandigheden waarvoor we in de steentijd niet genetisch
geprogrammeerd waren. Zoals Tudge opmerkt bij het beschrijven van de
`extreme generaliteit' van de mens: `Wij zijn het dierlijke equivalent
van de Turingmachine: het universele apparaat dat op iedere taak kan
worden ingezet.'\footnote{Tudge, op. cit., p.~168.} Welke trend tijdens
de komende overgangscrisis de overhand zal krijgen? Waarschijnlijk
beide.

De commercialisering van soevereiniteit hangt ervan af of
honderdduizenden soevereine individuen en miljoenen anderen bereid zijn
hun activa in te zetten in de `First Bank of Nowhere' om zo immuniteit
tegen directe dwang te verkrijgen. Dergelijk vertrouwen kent in het
oeroude verleden geen duidelijk equivalent. In de steentijd beschikte
men over nauwelijks activa; de weinigen die er waren, werden opgepot
door een stam -- een hechte, maar vaak ook paranoïde groep die
buitenstaanders wantrouwde. Desondanks biedt de cybereconomie, ondanks
haar evolutionaire nieuwheid, de mens de kans om ons meest baanbrekende
genetische erfgoed tot uiting te brengen: de intelligentie die onze
buitengewone hersenen kenmerkt. Degenen die tot de informatie-elite
behoren, herkennen ongetwijfeld een goede zaak zodra ze die tegenkomen.

Daarnaast kan de creatie van activa die vrijwel immuun zijn voor
predatie een opwaartse impuls geven aan de `inclusieve fitness' van
soevereine individuen. Hoewel de economische logica achter deelname aan
de cybereconomie de fundamenten van de natiestaat ondermijnt, spreekt
deze redenering vooral tot mensen met een hoog vaardigheidsniveau.

Om ten volle te profiteren van de kansen om tussen verschillende
jurisdicties te kiezen, moeten mensen bereid zijn hun natiestaat achter
zich te laten en hun persoonlijke veiligheid toe te vertrouwen aan
beveiligingsdiensten die hoofdzakelijk door marktprikkels worden
gedreven, ook al bevinden deze zich ver van hun geboorte- en
opvoedplaats. Dit biedt een duidelijk voordeel voor iedereen die
meerdere talen beheerst en een kosmopolitische cultuur omarmt in plaats
van jingoïstische neigingen te tonen. Bovendien betekent het dat
iedereen die serieus te werk gaat om het bevrijdende potentieel van de
cybereconomie voor zichzelf en zijn gezin te benutten, er verstandig aan
doet om in jurisdicties buiten de plek waar hij zijn hoofdcarrière heeft
opgebouwd een nieuwe, gunstige positie voor zichzelf te veroveren. Voor
meer details verwijs ik je naar onze bespreking van strategieën voor het
bereiken van onafhankelijkheid in de bijlagen.

\subsection{Echte affiniteiten}\label{echte-affiniteiten}

Een vernieuwd, extranationaal wereldbeeld en een andere manier om je
plek in de wereld te bepalen, kunnen onze culturele gewoonten -- zo niet
zelfs onze aangeboren neigingen -- ingrijpend veranderen. De
extranationale invulling van identiteit die in het nieuwe millennium
opkomt, maakt de aanpassing aan een veranderde wereld wellicht
eenvoudiger dan je zou denken. In tegenstelling tot nationaliteit
ontstaan deze nieuwe identiteiten niet door de systematische dwang die
in de twintigste eeuw natiestaten en het natiestaatssysteem universeel
oplegde. In het nieuwe tijdperk binden gemeenschappen en loyaliteiten
zich niet langer aan vaste territoria. Je identiteit zal steeds meer
gebaseerd zijn op oprechte affiniteiten, gedeelde belangen of werkelijke
verwantschap, in plaats van op de gefabriceerde banden van burgerschap
die de traditionele politiek onvermoeibaar nastreeft. Bescherming komt
op geheel nieuwe wijze tot stand, ver verwijderd van het meetgereedschap
waarmee een landmeter territoriale grenzen afbakent. Activa komen steeds
vaker in de cyberspace terecht in plaats van op een vaste locatie, wat
een nieuwe vorm van concurrentie stimuleert om de `beschermingskosten'
-- oftewel de belastingen -- in de meeste territoriale jurisdicties
omlaag te brengen.

\begin{quote}
`Ambitieuze mensen beseffen dat een migratoire levensstijl de prijs is
die je betaalt om vooruit te komen.'\footnote{Christopher Lasch,
  \emph{The Revolt of the Elites and the Betrayal of Democracy} (New
  York: W W Norton \& Company, 1995), p.~S.} - Christopher Lasch
\end{quote}

\section{Ontsnappen aan de
natiestaat}\label{ontsnappen-aan-de-natiestaat}

Ondanks de sterke invloed die de natiestaat als `in-group' op onze
moderne verbeelding uitoefent, zullen getalenteerde mensen -- die
vandaag nog vol vertrouwen profiteren van aansluiting bij zo'n
buitengewoon kostbare, ingebeelde gemeenschap -- dat nut spoedig in
twijfel trekken.

Inderdaad, de aanhangers van de natiestaat klagen inmiddels al over de
groeiende onthechting van de cognitieve elite.

De overleden Christopher Lasch valt in zijn tirade \emph{De opstand van
de elites en het verraad van de democratie} degenen aan `wiens
levensonderhoud niet zozeer berust op het bezit van eigendom als op de
manipulatie van informatie.'\footnote{Ibid., p.~34.} Lasch betreurt het
extranationale karakter van de opkomende informatie-economie. Hij
schrijft:

\begin{quote}
De markten waarin de nieuwe elites actief zijn, hebben inmiddels een
internationale reikwijdte. Hun fortuin is nauw verweven met
ondernemingen die grensoverschrijdend opereren. Ze richten zich meer op
het vlot functioneren van het geheel dan op de afzonderlijke onderdelen
ervan. Hun loyaliteit -- als we die term in deze context niet als
anachronistisch beschouwen -- is internationaal in plaats van regionaal,
nationaal of lokaal. Ze hebben meer gemeen met hun tegenhangers in
Brussel of Hong Kong dan met de massa Amerikanen die nog niet deel
uitmaken van het wereldwijde communicatienetwerk.\footnote{Ibid.,
  pp.~34--35.}
\end{quote}

Hoewel Lasch allesbehalve een onpartijdige waarnemer was en duidelijk
wilde maken dat zijn beeld van de informatie-elite allesbehalve vleiend
moest zijn, baseerde hij zijn minachting voor degenen die zich
onthechten van de plaatsgebonden tirannie op observaties van
ontwikkelingen die in dit boek centraal staan.

Lees je de kritieken van Lasch, Mickey Kaus (\emph{The End of
Equality}), Michael Walzer (\emph{Spheres of Justice}) en Robert Reich
(\emph{The Work of Nations}), dan bevestigen zij -- zij het op vaak
ongelukkige wijze -- delen van onze analyse. Deze auteurs tonen weinig
begrip voor de talrijke consequenties van een verdiepte marktwerking, om
nog maar te spreken van de denationalisatie van soevereine individuen.

Lasch hekelt fel degenen met extranationale ambities -- `die het
lidmaatschap van de nieuwe aristocratie van hersenen begeren' -- omdat
zij nauwe banden onderhouden met de internationale markt voor snel
bewegend geld, glamour, mode en populaire cultuur. Hij vervolgt:

Of zij zichzelf eigenlijk als Amerikanen beschouwen, valt te
betwijfelen. Patriottisme scoort duidelijk laag op hun ranglijst van
deugden. Daarentegen past `multiculturalisme' perfect bij hen en roept
het het charmante beeld op van een wereldbazaar waar men volop kan
genieten van exotische keukens, opvallende kleding, bijzondere muziek en
authentieke tradities -- zonder dat er vragen gesteld worden of
verplichtingen ontstaan. De nieuwe elites voelen zich alleen op hun
gemak wanneer ze onderweg zijn naar een conferentie op hoog niveau, de
grootse opening van een nieuwe franchise, een internationaal
filmfestival of een nog onontdekt vakantieoord. Hun blik op de wereld is
in wezen toeristisch, een houding die weinig bijdraagt aan een oprechte
betrokkenheid bij de democratie.\footnote{Ibid., p.~6.}

Economisch nationalisme

In de kritiek op de `transienten' die de virtuele gemeenschappen van het
informatietijdperk kenmerken, schuilt de erkenning dat voor velen in de
elite de baten van vergankelijkheid de kosten ruimschoots overtreffen.
Critici als Lasch en Walzer betwisten niet dat een nuchtere
kosten-batenanalyse het burgerschap voor hooggekwalificeerden overbodig
maakt, noch stellen zij dat de leden van de informatie‑elite -- waarvan
zij de houdingen afkeuren -- ten onrechte inschatten wat in hun belang
is. Evenmin doen zij alsof tabellen voor samengestelde rente aantonen
dat het voortdurend investeren in een nationaal
socialezekerheidsprogramma -- om nog maar te zwijgen over de
inkomstenbelastingen -- beter rendeert dan particuliere investeringen.
Integendeel, zij doorgronden de rekenkunde en hebben de berekeningen
doorgerekend tot de voor de hand liggende conclusies. In plaats van de
subversieve logica van economische rationaliteit te erkennen, keren ze
zich ervan af en zien ze het als `verraad' dat de informatie‑elite de
plaatsgebonden tirannie overstijgt en `de onwetenden'
achterlaat.\footnote{Ibid., p.~21.}

Net als Pat Buchanan behoren de sociaaldemocraten tot de economische
nationalisten die de dominantie van markten boven de politiek verwerpen.
Zij bekritiseren `de nieuwe aristocratie van hersenen' omdat die geen
band heeft met een specifieke locatie en zich niet oprecht bekommert om
wat volgens hen in het belang is van de massa. Hoewel zij de
denationalisatie van het individu niet expliciet erkennen, verzetten zij
zich tegen de vroege signalen en uitingen daarvan -- wat Walzer aanduidt
als `het imperialisme van de markt' -- en tegen de neiging van geld om
`grensoverschrijdend' te circuleren voor de aankoop van zaken die, zoals
Lasch betoogt, `niet koopbaar zouden moeten zijn', bijvoorbeeld
vrijstelling van de militaire dienst. Merk op hoe men reactionair
terugvalt op de militaire eisen van de natiestaat, als een heilige zone
waar geld en markten geen toegang mogen krijgen.\footnote{Ibid., p.~21.}

Deze kritiek op de informatie‑elite schetst de contouren van een
populaire tegenreactie tegen de opkomst van onafhankelijke individuen in
het komende millennium. Naarmate er nieuwe, meer marktgerichte
beschermingsvormen beschikbaar komen, zal voor talloze bekwame personen
steeds duidelijker worden dat de vermeende voordelen van nationaliteit
vooral illusoir zijn. Dit leidt niet alleen tot een zorgvuldiger
afweging van de alternatieve kosten van burgerschap, maar opent ook de
weg naar nieuwe manieren om zogenaamd `politieke' en zelfs `economische'
vraagstukken te benaderen. Voor het eerst kan `een zelfstandige
ondernemer' zijn eigen beschermingskosten aanpassen door zich tussen
verschillende rechtsgebieden te bewegen, zonder te hoeven wachten op
besluiten via groepsbesluiten en -actie, zoals een oud dilemma van
Frederic C. Lane ooit verwoordde.\footnote{Lane, `De economische
  betekenis van oorlog', in Venetië en geschiedenis. De verzamelde
  papieren van Frederic C. Lane, p.~385.}

Naarmate de prijs voor bescherming steeds meer volgens het
substitutieprincipe wordt vastgesteld, komen de rekenregels van dwang
aan het licht en verergert dit de conflicten tussen de nieuwe
kosmopolitische elite van het informatietijdperk en de `informatiearmen'
-- de rest van de bevolking die grotendeels eentalig is en niet
uitblinkt in probleemoplossend vermogen of beschikt over wereldwijd
verhandelbare vaardigheden. Deze `verliezers' of `achterblijvers', zoals
Thomas L. Friedman ze omschrijft, zullen onvermijdelijk afhankelijk
blijven van het politieke leven binnen de bestaande
natiestaten.\footnote{Zie Thomas L. Friedman, `Laat de verliezers van de
  globalisering niet buiten beschouwing', \emph{International Herald
  Tribune}, 18 juli 1996, p.~8.}

\section{De meeste politieke agenda's zullen reactionair
zijn}\label{de-meeste-politieke-agendas-zullen-reactionair-zijn}

De meesten met een vurige politieke agenda -- of ze nu nationalistisch,
milieuactivistisch of socialistisch zijn -- zullen, zodra de
eenentwintigste eeuw aanbreekt, massaal opkomen voor de wankelende
natiestaat. In de loop der tijd wordt steeds duidelijker dat het
voortbestaan van de natiestaat en de nationalistische gevoelens
essentieel zijn om een domein voor politieke dwang te behouden. Zoals
Billig opmerkt, vormt nationalisme `de voorwaarde voor conventionele
(politieke) strategieën, wat de specifieke politiek ook moge
zijn.'\footnote{Billig, op. cit., p.~99.} Daarom zal het
nationalistische element in alle politieke programma's de komende jaren
als de buik van een gulzigaard opzwellen. Milieuactivisten gaan zich
bijvoorbeeld minder richten op de bescherming van `Moeder Aarde' en meer
op die van `het vaderland.' Om redenen die we later behandelen, zullen
natie en burgerschap bijzonder heilig worden voor degenen die veel
waarde hechten aan gelijkheid. Meer dan ze nu wellicht beseffen, zullen
zij het eens worden met Christopher Lasch, die op Hannah Arendt volgde
met de uitspraak: `Het is het burgerschap dat gelijkheid verleent, niet
gelijkheid die een recht op burgerschap schept.'\footnote{Lasch, op.
  cit., p.~88.}

De privatisering van soevereiniteit doet de nadruk op gelijkheid, zoals
die in het industriële tijdperk aanwezig was, afnemen doordat het de
band tussen de welvaartsscheppers enerzijds en natie en plaats
anderzijds doorsnijdt. Het burgerschap vervult daarmee niet langer de
functie van een mechanisme voor inkomensherverdeling op basis van
gelijke stemrechten binnen een afgebakend grondgebied. De gevolgen
zullen opnieuw een deuk slaan in het progressieve historische
wereldbeeld. In tegenstelling tot de verwachtingen van zogenaamd
vooruitstrevende denkers aan het begin van de twintigste eeuw bleef de
vrije markt door de decennia heen triomfantelijk overeind. De marxisten
voorspelden de ondergang van het kapitalisme -- wat echter nooit
gebeurde -- in de hoop dat dit zou leiden tot het vervagen van
natiestaten en het ontstaan van een universeel klassenbewustzijn onder
arbeiders. In werkelijkheid zal de staat in de schaduw treden, maar op
een geheel andere manier. Er gebeurt bijna het tegenovergestelde van wat
men verwachtte. De triomf van het kapitalisme leidt tot het ontstaan van
een nieuw, wereldwijd of extranationaal bewustzijn onder kapitalisten,
waarvan velen als zelfstandige, soevereine individuen gaan opereren. In
plaats van afhankelijk te zijn van de staat om arbeiders te
disciplineren, zoals de marxisten voorspelden, blijken de meest bekwame
en rijkste personen uiteindelijk netto verliezers te worden door de
acties van de natiestaat. Het is duidelijk dat zij het meeste winnen
door het nationalisme achter zich te laten, nu de markten de overhand
krijgen ten koste van dwang.

Misschien niet onmiddellijk, maar binnen een generatie zal bijna
iedereen uit de informatie-elite besluiten zijn inkomensgenererende
activiteiten te verplaatsen naar lagebelastings- of belastingvrije
jurisdicties. Naarmate het informatietijdperk de wereld transformeert,
illustreert dit onmiskenbaar de kracht van samengestelde rente. Binnen
enkele jaren -- om nog maar te zwijgen van decennia -- wordt algemeen
duidelijk dat vrijwel iedereen met talent een aanzienlijk hoger
nettovermogen kan opbouwen en een beter leven kan leiden door afstand te
nemen van hoogbelaste natiestaten. We hebben al gewezen op de torenhoge
kosten die de vooraanstaande natiestaten met zich meebrengen, maar
aangezien dit de kern vormt van een kwestie die vaak onbegrepen blijft,
is het de moeite waard om nog eens de alternatieve kosten van
nationaliteit te benadrukken.

\subsection{Alternatieve kosten}\label{alternatieve-kosten}

In plaats van te lijden onder het verlies of de beperking van
overheidsdiensten die momenteel met hoge belastingen worden
gefinancierd, zal de informatie-elite ongekende bloei ervaren. Door
simpelweg de overmatige belastingdruk die zij nu dragen achter zich te
laten, creëren zij een enorme marge om het materiële welzijn van hun
gezinnen te verbeteren.

Zoals eerder vermeld, verkleint elke \$5.000 die je jaarlijks aan
belasting betaalt, je levenslange nettowaarde met \$2,4 miljoen als je
een jaarlijks rendement van 10 procent behaalt op je investeringen.
Verdient je portefeuille echter 20 procent, dan zorgt een jaarlijkse
belasting van \$5.000 over veertig jaar voor een verlies van maar liefst
\$44 miljoen. Cumulatief betekent dit dat je met \$5.000 per jaar meer
dan een miljoen dollar per jaar misloopt. Bij deze werkwijze leidt een
jaarlijkse belasting van \$250.000 al snel tot een verlies van meer dan
\$50 miljoen per jaar -- ofwel \$2,2 miljard over een heel leven. Zelfs
incidenteel hogere inkomsten, bijvoorbeeld in de vroege fase van je
leven, zorgen voor een nog schokkender vermogensverlies door
roofzuchtige belastingheffing.

Onze auteurs hebben tot onze tevredenheid vastgesteld dat rendementen
van meer dan 20 procent mogelijk zijn. Zo behaalden onze collega's bij
\emph{Lines Overseas Management} in Bermuda, gedurende de periode waarin
wij dit boek schreven, driemaal hun investering -- een gemiddeld
jaarlijks rendement van 226 procent. Hun ervaring bevestigt wat de
berekeningen suggereren: voor veel hoogverdieners en kapitaalbezitters
betekent roofzuchtige belastingheffing een levenslange kostenpost die
neerkomt op een aanzienlijk fortuin.

Iemand met een hoge verdiencapaciteit die belasting betaalt volgens de
tarieven van Hongkong kan uiteindelijk duizend keer meer vermogen
opbouwen dan iemand met vergelijkbare bruto-prestaties die belastingen
afdraagt volgens Noord-Amerikaanse of Europese standaarden. Je kapitaal
herhaaldelijk blootstellen aan de druk van een hoogbelastingsregime is
als deelnemen aan een race waarin je bij elke stap wordt tegengehouden.
Zou je diezelfde race met de juiste bescherming en zonder belemmeringen
kunnen lopen, dan zou je uiteraard veel verder en sneller komen.

De onafhankelijke individuen van de toekomst zullen profiteren van die
`vergankelijke' neigingen waar \emph{Christopher Lasch} en andere
critici van de informatie-elite zo fel op wijzen. Zij gaan actief op
zoek naar de rechtsgebieden met de meest gunstige fiscale voorwaarden om
zich te vestigen. Hoewel dit in strijd lijkt met het nationalistische
denken, is het economisch volstrekt logisch. Zelfs een verschil in
nettowinst van 10 procent -- laat staan een tienvoudig verschil -- zal
winstmaximaliserende mensen ertoe aanzetten om hun levensstijl,
productiemethoden én woonplaats te wijzigen. De geschiedenis van de
westerse beschaving is immers een aaneenschakeling van voortdurende
veranderingen, waarin mensen en welvaart keer op keer migreerden op zoek
naar nieuwe kansen, aangewakkerd door grillige, grootschalige politieke
omstandigheden. Een duizendvoudig verschil in nettowinst zou wel eens de
krachtigste stimulans kunnen zijn die ooit rationele mensen in beweging
heeft gezet. Of anders gezegd: de meeste mensen -- in het bijzonder
degenen die \emph{Thomas L. Friedman} aanduidt als de `verliezers en
achterblijvers' -- zouden, als ze de keuze hadden, elke natiestaat
verruilen voor \$50 miljoen, om nog maar te zwijgen van de nog hogere
belastinglasten die staten opleggen, vooral voor de top 1 procent. De
opkomst van onafhankelijke individuen die actief op zoek gaan naar
gunstige rechtsgebieden behoort dan ook tot de meest zekere
voorspellingen.

\section{De commercialisering van
soevereiniteit}\label{de-commercialisering-van-soevereiniteit}

Als we het burgerschap tegen het einde van de twintigste eeuw in een
kosten-batenanalyse meewegen, bleek het een buitengewoon koopje te zijn.
Dat werd treffend geïllustreerd door een humoristisch bedoelde
parlementaire onderzoeksnota, getiteld `Is the Queen an Australian
Citizen?', opgesteld in augustus 1995 door Ian Ireland van de
\emph{Australian Parliamentary Research Service}.\footnote{Ian Ireland,
  `Is de koningin een Australische staatsburger?', \emph{Parliamentary
  Research Service}, Australië, nr. 6, 28 augustus 1995.} Ireland
bestudeert de Australische staatsburgerschapswet van 1948 en bespreekt
de vier manieren om Australisch burgerschap te verkrijgen. Deze methoden
komen overeen met die in andere toonaangevende landen, namelijk:

\begin{itemize}
\tightlist
\item
  burgerschap door geboorte
\item
  burgerschap door adoptie
\item
  burgerschap door afstamming
\item
  burgerschap door verlening
\end{itemize}

Op zichzelf valt dit niet op, maar het legt wel het verschil bloot
tussen soevereiniteit en burgerschap. Zoals Ireland stelt: `Volgens
traditionele juridische en politieke opvattingen is de vorst soeverein
en zijn de mensen onderdanen. Onderdanen zijn gebonden door trouw en
onderwerping aan de vorst.' Gezien het voor de hand liggende feit dat
koningin Elizabeth~II soeverein is, concludeert hij dat `er een argument
bestaat dat de koningin geen Australisch staatsburger is.'\footnote{Ibid.,
  p.~2.}

En dat klopt ook: dat is zij immers niet. De koningin hoeft zich geen
zorgen te maken over haar staatsburgerschap, want zij is soeverein -- de
soeverein over haar onderdanen. Net als enkele andere monarchen bezit
zij van geboorte soevereiniteit, een status die zij heeft geërfd volgens
een oud gebruik dat de moderne tijd ver vooruit is. Het concept van
monarchie is immers al eeuwenoud en gaat terug tot de vroegste
historische verslagen van het menselijk bestaan. Landen die hun
monarchie hebben behouden, danken hun grondwet vaak hun rijke
geschiedenis, wat nog steeds bijdraagt aan de vorm van hun samenleving
-- al is dat vooral op sociaal vlak, zo niet ook wat betreft de
politieke macht. Postmoderne burgers, die niet over de voordelen van de
koningin beschikken, zullen nieuwe juridische fundamenten moeten
bedenken waarop zij de feitelijke soevereiniteit kunnen baseren die de
informatietechnologie hen biedt.

Soevereine individuen moeten ook de corrosieve effecten van afgunst
onder ogen zien -- een hinderpaal die monarchen soms teisteren, maar
voor mensen die niet traditioneel vereerd worden en zelf hun
soevereiniteit claimen, nog sterker zal doorwerken. Zoals Helmut Schoeck
in zijn uitgebreide onderzoek \emph{Envy} opmerkte: `Als er slechts één
koning -- of één president van de Verenigde Staten, met andere woorden,
slechts één persoon met een bepaalde status -- is, kan hij vrijwel
ongestraft leven op een manier die, zelfs op een veel kleinere schaal,
in dezelfde samenleving tot grote verontwaardiging zou leiden indien
succesvolle leden van bredere professionele of sociale groepen zich op
die wijze zouden gedragen.'\footnote{Schoeck, op. cit., p.~265.}
Monarchen, als verpersoonlijking van de natie, genieten een zekere
immuniteit tegen afgunst -- een bescherming die niet geldt voor andere
soevereine individuen.

De `verliezers en achterblijvers' in de informatiesamenleving zullen
ongetwijfeld het succes van de winnaars benijden en met afgunst toezien,
zeker nu de verdere ontwikkeling van de markten er op wijst dat we
steeds meer een `winnaars nemen alles'-wereld tegemoetzien. Steeds vaker
hangt de beloning af van relatieve prestaties in plaats van absolute
prestaties, zoals dat in de industriële productie het geval was. Vroeger
kreeg een fabrieksarbeider loon op basis van het aantal gewerkte uren,
gemeten met een tijdklok, of volgens een vastgestelde outputnorm --
bijvoorbeeld het aantal geproduceerde stuks of gemonteerde
eenheden.\footnote{Voor een kritische kijk op vergoedingssystemen
  gebaseerd op relatieve prestaties zie Robert H.~Frank en Philip
  J.~Cook, \emph{The winner-take-all society}, pp.~24f.} Die
gestandaardiseerde beloning was mogelijk omdat de output voor iedereen
die met dezelfde gereedschappen werkte, vergelijkbaar was. Het creëren
van conceptuele rijkdom -- denk aan artistieke prestaties -- verschilt
echter enorm tussen mensen die met dezelfde middelen werken. In dat
opzicht lijkt de hele economie steeds meer op een operawereld, waarbij
de beste stemmen de hoogste beloningen krijgen en degenen die vals
zingen, hoe oprecht ook, doorgaans weinig verdienen. Naarmate meer
terreinen opengaan voor echte wereldwijde concurrentie, zal de beloning
voor gemiddelde prestaties onvermijdelijk dalen. Gemiddelde talenten
komen in overvloed voor, mede dankzij mensen die hun tijd kunnen
verhuren voor een fractie van de tarieven die in toonaangevende
industriële landen gebruikelijk zijn. De verliezers worden de
buitenvelders in de lagere competitie, met `slider speed bats', waarvan
de reflexen een halve seconde te traag zijn om een fastball uit de major
leagues te raken. In plaats van een miljoen dollar per jaar te verdienen
met homeruns, krijgen zij slechts \$25.000 binnen, zonder bijkomende
inkomsten uit beroemdheidssponsoring. Anderen zullen totaal mislukken.

\begin{quote}
`Zodra een land zich openstelt voor de wereldmarkt, ontstaan er winnaars
onder de burgers die beschikken over de juiste vaardigheden, terwijl
degenen zonder die talenten als verliezers of achterblijvers uit de bus
gaan. Vaak beweert een partij in staat te zijn de globalisering te
weerstaan of haar impact te verzachten. Denk hierbij aan Pat Buchanan in
Amerika, de communisten in Rusland en nu de \emph{Islamitische
Welzijnspartij} hier in Turkije. Wat er in Turkije gebeurt, is dan ook
veel complexer dan louter een fundamentalistische overname. Het is
precies het gevolg van een verdere globalisering die steeds meer
verliezers voortbrengt, terwijl een toenemende democratisering hen een
stem geeft -- en religieuze partijen deze samenloop van omstandigheden
effectief misbruiken om de macht te grijpen.'\footnote{Friedman,
  \emph{op. cit.}}
\end{quote}

--- THOMAS L. FRIEDMAN

Wie worden de verliezers in het informatietijdperk? In de meeste
gevallen blijken dit de belastingbetalers te zijn. Zij vergroten hun
vermogen doorgaans niet doordat verhuizen naar een andere jurisdictie
voor hen geen optie is. Een groot deel van hun inkomen is immers stevig
verankerd in de regels van hun nationale politieke systeem, in plaats
van tot stand te komen via marktwaarderingen. Daarom maakt het
afschaffen of drastisch verlagen van belastingen die negatief op hun
nettovermogen inwerken, hen niet per se beter af -- want een lagere
belastingdruk gaat altijd gepaard met minder overdrachtsuitkeringen. Zij
zullen inkomsten mislopen, omdat zij niet langer op politieke dwang
kunnen rekenen om geld af te tappen van mensen die productiever zijn.
Degenen zonder spaargeld, die afhankelijk zijn van de overheid voor hun
pensioen en zorg, zullen vrijwel zeker een terugval in hun
levensstandaard ondervinden.

Dit verlies in inkomen vertaalt zich in een waardevermindering van wat
de financieel schrijver Scott Burns `transcendentale' -- oftewel
politieke -- kapitaal noemt.\footnote{James Dale Davidson, \emph{The
  squeeze} (New York: Summit Books, 1980), pp.~38--55.} Dit
`transcendentale' of denkbeeldige kapitaal berust niet op economisch
eigendom van activa, maar op de feitelijke aanspraak op de
inkomensstroom die voortvloeit uit politieke regels en voorschriften. Zo
kan het verwachte inkomen uit overheidsuitkeringen worden omgezet in een
obligatie, waarvan de waarde wordt berekend aan de hand van de geldende
rentetarieven. Deze denkbeeldige obligatie, gefinancierd door de
ingebeelde gemeenschap, vormt het transcendente kapitaal. Door de `grote
transformatie', gericht op het verzwakken van de greep van politieke
autoriteiten op de kasstroom die nodig is om hun beloften waar te maken,
zal dit kapitaal plotseling in waarde dalen.

\begin{quote}
`Aan de grenzen en op de hoge zeeën, waar niemand blijvend een monopolie
op het gebruik van geweld had, konden handelaren de hoge afpersingen
ontwijken, waardoor alternatieve vormen van bescherming goedkoper werden
verkregen.'\footnote{Lane, `Economic consequences of organized
  violence', p.~404.}
\end{quote}

--- FREDERIC C. LANE

Het is niet moeilijk voor te stellen dat de informatie‑elite de kansen
zal benutten die de nieuwe cybereconomie biedt voor bevrijding en
persoonlijke soevereiniteit. Ook mag men verwachten dat de
achterblijvers steeds jingoïstischer en onaangenamer worden, naarmate de
impact van informatietechnologie in dit millennium toeneemt. Het blijft
lastig precies te voorspellen wanneer de reactie lelijk zal worden. Wij
vermoeden dat de wederzijdse beschuldigingen zullen intensiveren zodra
de westerse landen onmiskenbaar uiteenvallen, vergelijkbaar met wat
destijds bij de voormalige Sovjet‑Unie gebeurde.

Telkens als een natiestaat uiteenvalt, versnelt dat de decentralisatie
en vergroot het de autonomie van soevereine individuen. Wij verwachten
een aanzienlijke toename van soevereine entiteiten, waarbij tientallen
enclaves en rechtsgebieden die meer op stadstaten lijken, herrijzen uit
het puin van natiestaten. Deze nieuwe eenheden kiezen er vaak voor om
beschermingsdiensten tegen zeer concurrerende tarieven aan te bieden,
vaak met lage of zelfs geen belastingen op inkomen en kapitaal. Bijna
onvermijdelijk prijzen zij hun beschermingsdiensten aantrekkelijker dan
de vooraanstaande OESO‑natiestaten. Als men het louter vanuit
marktsegmentatie bekijkt, geldt dat het segment dat momenteel het
slechtst bediend wordt, op een efficiënte en goedkope wijze kan worden
aangeleverd. Iedereen die bereid is hoge belastingen te betalen in ruil
voor een ingewikkeld pakket aan staatsuitgaven, krijgt daarvoor volop de
gelegenheid. Daarom ligt de meest voordelige en winstgevende strategie
voor een nieuwe mini‑soevereiniteit vrijwel onvermijdelijk in een hoog
efficiënte, laaggeprijsde variant. Zo'n mini‑soevereiniteit kan met
grote moeite verwachten een uitgebreider dienstenpakket te bieden dan de
overgebleven natiestaten. Omdat niet alle natiestaten gelijktijdig
instorten, zal er in de beginfase van de transitie ruim voldoende keuze
zijn. Bovendien kan een sobere invulling van orde en veiligheid relatief
goedkoop worden gerealiseerd. Als sociale onrust en criminaliteit zich
in de traditionele kernindustrieën uitbreiden tot het niveau dat wij
voorspellen, zal een toereikende orde en veiligheid in een rechtsgebied
veel aantrekkelijker blijken dan een nationaal ruimteprogramma, een door
de staat gesponsord vrouwenmuseum of gesubsidieerde
omscholingsprogramma's voor ontslagen leidinggevenden.

\section{De denationalisatie van het
individu}\label{de-denationalisatie-van-het-individu}

Het burgerschap wordt minder aantrekkelijk en houdbaar naarmate nieuwe
instituties ontstaan die de keuze in de diensten, waarover de overheid
momenteel de touwtjes in handen heeft -- te beginnen met bescherming --
vergemakkelijken. Dit maakt het voor individuen praktisch om te stoppen
met zichzelf in nationale termen te identificeren. Toch verloopt de
demystificatie van het burgerschap traag. Je wordt voortdurend
geconfronteerd met een lawine van alledaagse boodschappen in je
dagelijkse routine, die erop gericht zijn je band met je lokale
natiestaat te versterken. Daardoor is het vrijwel onmogelijk om je
nationaliteit te vergeten. Voor veel mensen vormt nationaliteit immers
een essentieel identiteitssymbool. `Wij' leren de wereld in termen van
nationaliteit te zien. Het is ons land, `onze' sporters strijden op de
Olympische Spelen en wanneer zij winnen, wappert `onze' vlag tijdens de
ceremonie. Ons volkslied trekt tijdens de prijsuitreiking de aandacht
van juryleden en medekandidaten. Wij geloven dat de overwinning van ons
is, ook al blijft het onduidelijk wat onze bijdrage precies inhoudt,
afgezien van het feit dat we als burgers binnen hetzelfde grondgebied
verblijven.

\subsection{Van de eerste persoon meervoud naar het
enkelvoud}\label{van-de-eerste-persoon-meervoud-naar-het-enkelvoud}

Naarmate informatietechnologie steeds meer in de schijnwerpers komt te
staan, zal zij bijdragen aan het vormen van een mondiale blik en tevens
nieuwe manieren bieden voor onafhankelijke individuen om het verborgen
potentieel van deze technologie te benutten en zo te ontsnappen aan de
nationale lasten van belastingen.

Binnen enkele decennia zal narrowcasting het traditionele broadcasten
vervangen als methode om nieuws te vergaren. Dit heeft ingrijpende
gevolgen: het verandert de manier waarop miljoenen mensen zichzelf zien,
namelijk door een verschuiving van het collectieve `wij' naar het
individuele `ik'. Wanneer mensen zelf als hun eigen nieuwsredacteur
optreden en zelf bepalen welke onderwerpen en nieuwsberichten voor hen
belangrijk zijn, verkleint de kans dat zij zich laten indoctrineren in
de opofferingsplicht voor de natiestaat.

Ook door de privatisering van het onderwijs -- mede mogelijk gemaakt
door technologische ontwikkelingen -- zal een vergelijkbaar effect
optreden. In de middeleeuwen hield de Kerk het onderwijs stevig in eigen
hand, waarna in latere tijden de staat het overnam. Zoals Eric Hobsbawm
opmerkt: `staatsonderwijs transformeerde mensen in burgers van een
specifiek land: ``boeren in Fransen''\,'. In het informatietijdperk
krijgt onderwijs een private en individuele inslag en zal het niet
langer beladen zijn met de zware politieke bagage die het onderwijs
tijdens de industriële periode kenmerkte. Nationalisme zal niet meer in
elk hoekje van ons denken worden ingeprent.

De opkomst van het internet en het \emph{World Wide Web} zal bovendien
het belang van locatie in de handel doen afnemen. Hierdoor ontstaan er
individuele adressen die niet aan een specifiek territorium zijn
gebonden. Digitale satelliettelefoondiensten evolueren zó dat zij meer
bieden dan louter locatiegebonden vaste-lijnsystemen met een
gemeenschappelijke internationale belcode. Ieder individu krijgt zo een
uniek, wereldwijd telefoonadres -- vergelijkbaar met een internetadres
-- waarmee hij overal te bereiken is. Na verloop van tijd vallen
nationale postmonopolies in duigen, zodat wereldwijd opererende,
geprivatiseerde postbezorgsystemen mogelijk worden, zonder band met een
specifieke natiestaat.

Deze en andere schijnbaar kleinigheden zullen zowel de doorsnee
consument als de intellectuele elite bevrijden van de vastgeroeste
identificatie met de natiestaat. De ontmythologisering van het
burgerschap zal op spectaculaire wijze versnellen door de opkomst van
praktische alternatieven voor zakendoen binnen de door staten
gemonopoliseerde, afgebakende territoria. De fundamenten van de
cybereconomie -- denk aan cybergeld, cyberbankieren en een wereldwijde,
ongereguleerde cybermarkt voor effecten -- zullen vrijwel onvermijdelijk
op grote schaal vorm krijgen. Zodra dit gebeurt, neemt het vermogen van
hebzuchtige regeringen om de rijkdom van `burgers' te confisqueren,
aanzienlijk af.

Hoewel de hoofdmachten ongetwijfeld zullen proberen een kartel op te
leggen dat hoge belastingen en fiatgeld handhaaft -- door samen te
werken om encryptie te beperken en burgers te verhinderen hun
rechtsgebied te ontvluchten -- zullen zij uiteindelijk falen. De meest
ondernemende mensen op de planeet vinden immers hun weg naar economische
vrijheid. Het is onwaarschijnlijk dat de staat überhaupt in staat zal
zijn mensen op te sluiten op plekken waar zij als gijzelaars kunnen
dienen. Het ineffectieve karakter van pogingen om illegale immigranten
buiten te houden, bewijst dat natiestaten hun grenzen niet kunnen
afsluiten tegen het vertrek van succesvolle individuen. De rijken zullen
even daadkrachtig hun vertrek regelen als taxichauffeurs of obers dat
doen bij hun intrede.

Voor het eerst sinds de middeleeuwen, toen soevereiniteit nog
gefragmenteerd was, zullen grenzen niet scherp afgebakend zijn. Zoals
eerder aangetoond, ontstaat er geen centraal gebied waar de meeste
toekomstige financiële transacties plaatsvinden. In plaats van een reeks
verplichtingen op basis van geboorte te accepteren, zullen steeds meer
onafhankelijke individuen deze ambiguïteit benutten om aan hun
belastingplichten te ontsnappen en daarmee het traditionele burgerschap
achter zich te laten. Zij gaan als `klanten' privé-belastingverdragen
onderhandelen, vergelijkbaar met de regelingen die nu in Zwitserland
bestaan, zoals in hoofdstuk 8 is geanalyseerd. Een typisch
privé-belastingverdrag dat met de Franstalige Zwitserse kantons wordt
gesloten, stelt een individu of familie in staat om in het grondgebied
te verblijven in ruil voor een vaste jaarlijkse belasting van 50.000
Zwitserse frank (momenteel ongeveer \$45.000). Let op: dit betreft geen
belasting op basis van een vast percentage, maar een door de overheid
vastgesteld vast bedrag, onafhankelijk van het inkomen. Verdient u
jaarlijks 50.000 Zwitserse frank, zou zo'n overeenkomst resulteren in
een belastingtarief van 100 procent. Bij een inkomen van 500.000
Zwitserse frank bedraagt het tarief dan 10 procent, bij 5.000.000
slechts 1 procent en bij 50 miljoen slechts 0,1 procent. Als dit een
buitengewoon aantrekkelijke deal lijkt ten opzichte van een marginale
belasting van 58 procent in New York City, illustreert dat slechts hoe
uitbuitend en monopolistisch de overheidsdiensten tijdens de industriële
periode hun tarieven bepaalden.

In feite is 50.000 Zwitserse frank per jaar ruimschoots voldoende om de
noodzakelijke en nuttige overheidsdiensten te bekostigen. De Zwitsers
profiteren er ongetwijfeld van: iedere miljonair die zich hier vestigt,
moet voor dat voorrecht jaarlijks 50.000 Zwitserse frank betalen. In
veel gevallen bedragen de bijkomende kosten voor de overheid om een
extra miljonair binnen haar jurisdictie te ontvangen bijna niets,
waardoor de jaarlijkse opbrengst per transactie bijna op 50.000
Zwitserse frank uitkomt. Elke overheidsdienst die ver onder de
marktprijs wordt aangeboden, terwijl de goedkoopste aanbieder er toch
circa 100 procent winst mee behaalt, getuigt van monopolievorming en
extreme overprijsing. Opvallend is niet dat het belastingtarief als
percentage van het inkomen daalt, maar dat men in de twintigste eeuw
ooit als `eerlijk' beschouwde dat mensen radicaal verschillende bedragen
voor overheidsdiensten betaalden. Dat is des te opmerkelijker, want
degenen die het meest van overheidsdiensten profiteren, betalen het
minste, terwijl degenen die er nauwelijks gebruik van maken, het hoogste
bedrag moeten betalen. Al deze regelingen leveren iedere hoogverdienende
Amerikaan een domicilievoordeel op ten opzichte van de Verenigde Staten,
dat over een heel leven kan oplopen tot tientallen miljoenen. Tenzij de
Amerikaanse belastingen zodanig hervormd worden dat ze concurrerender
zijn met die van andere jurisdicties en niet langer op basis van
nationaliteit worden geheven, zullen verstandige burgers hun Amerikaans
burgerschap opgeven -- ondanks de obstakels van Clintons exitbelasting
-- om paspoorten te bemachtigen met minder zware verplichtingen.

Overheden in het industriële tijdperk schreven de prijs voor hun
diensten op basis van het succes van de belastingbetaler, in plaats van
op basis van de werkelijke kosten of de waarde van de geleverde
diensten. De verschuiving naar een commerciële prijszetting voor
overheidsdiensten zorgt voor een aantrekkelijkere bescherming tegen een
aanzienlijk lagere prijs dan die door conventionele natiestaten wordt
afgedwongen.

\subsection{Burgerschap gaat de weg van de
ridderlijkheid}\label{burgerschap-gaat-de-weg-van-de-ridderlijkheid}

Kortom, burgerschap zal een pad volgen dat vergelijkbaar is met dat van
de ridderlijkheid. Nu we de basis voor bescherming opnieuw inrichten,
veranderen vanzelf ook de rechtvaardigingen en motiverende ideologieën
die het systeem ondersteunen. Een half millennium geleden, aan het einde
van de middeleeuwen, reageerden mensen voorspelbaar: toen het verlenen
van bescherming in ruil voor persoonlijke diensten doorgaans niet meer
rendabel bleek, lieten zij de ridderlijkheid achter zich. Gezwoer en
persoonlijke trouw werden niet meer zo gewaardeerd als in de voorgaande
vijf eeuwen.

Informatietechnologie belooft het burgerschap op een vergelijkbaar
subversieve manier te transformeren. De natiestaat en de aanspraken van
het nationalisme verliezen geleidelijk hun mystieke allure, net zoals
vijf eeuwen geleden de aanspraken van de monopolistische kerk werden
ontmaskerd. Hoewel reactionairen vernieuwers proberen te demoniseren en
het nationalistische sentiment nieuw leven in te blazen, betwijfelen wij
of de politiek verouderde natiestaat nog in staat is voldoende
loyaliteit op te roepen om de door informatietechnologie veroorzaakte
competitieve druk te weerstaan. De meeste kritische individuen in een
wereld vol failliete overheden kiezen er eerder voor als klanten van
beschermingsdiensten goed behandeld te worden dan als burgers van
natiestaten geplunderd te worden.

De welvarende OECD-landen leggen een zware fiscale en regelgevende last
op aan iedereen die binnen hun grenzen onderneemt. Deze kosten waren
wellicht nog draaglijk toen de OECD-landen de enige rechtsgebieden waren
waar men zowel redelijk kon ondernemen als prettig kon wonen, maar die
tijden zijn voorbij. De extra heffingen die inwoners betalen omdat ze in
de rijkste natiestaten wonen, wegen niet meer op tegen de voordelen. Nu
de concurrentie tussen rechtsgebieden toeneemt, wordt dit steeds
onacceptabeler. Wie beschikt over de middelen en het kapitaal om de
uitdagingen van het informatietijdperk het hoofd te bieden, kan zich
overal vestigen en zaken doen. Als men kan kiezen uit meerdere
domicilies, blijven uiteindelijk alleen de meest patriotische of domme
mensen wonen in landen met hoge belastingen.

Om die reden moet men verwachten dat één of meer natiestaten stiekem
maatregelen gaan treffen om de aantrekkingskracht van een tijdelijk
verblijf te verminderen. Reizen kan bijvoorbeeld actief ontmoedigd
worden door biologische oorlogsvoering -- denk aan het opzettelijk laten
uitbreken van een dodelijke epidemie. Dit zou niet alleen de reislust
temperen, maar landen wereldwijd ook een argument geven om hun grenzen
te sluiten en de immigratie te beperken.

\subsection{Het nadeel van
nationaliteitsbelasting}\label{het-nadeel-van-nationaliteitsbelasting}

Tenzij er een baanbrekende en bijna wonderbaarlijke beleidswijziging
plaatsvindt, krijgt de succesvolle belegger of ondernemer in het
informatietijdperk een levenslange financiële last opgelegd -- die kan
oplopen tot tientallen miljoenen, honderden miljoenen of zelfs miljarden
dollars -- wanneer hij in landen wil wonen die een fiscaal beleid
hanteren als dat van de landen die in de twintigste eeuw de hoogste
levensstandaard hadden.

Zonder een radicale verandering valt deze last voor Amerikanen het
zwaarst uit. De Verenigde Staten behoren tot de weinige rechtsgebieden
ter wereld die belastingen heffen op basis van nationaliteit in plaats
van op basis van woonplaats. De overige twee landen die op deze manier
te werk gaan, zijn de Filippijnen -- een voormalige Amerikaanse kolonie
-- en Eritrea, waar tijdens de langdurige opstand tegen het Ethiopische
bewind één van de verbannen leiders in de ban raakte van de \emph{IRS}.
Tegenwoordig heft Eritrea een nationaliteitsbelasting van 3 procent.
Hoewel dit een schamele imitatie is van de Amerikaanse tarieven,
betekent zelfs die geringe last dat het Eritrese staatsburgerschap in
het informatietijdperk een zware prijs verlangt. De huidige regelgeving
zorgt er bovendien voor dat het Amerikaanse staatsburgerschap nog
zwaarder belast wordt. De \emph{IRS} is inmiddels uitgegroeid tot een
van de toonaangevende exportproducten van Amerika. Meer dan enig ander
land reikt de Verenigde Staten wereldwijd uit om inkomsten te innen van
hun staatsburgers.

Stel je voor dat een 747-jet, gevuld met één belegger uit ieder
rechtsgebied ter wereld, in een net onafhankelijk geworden land aan land
gaat en dat iedere belegger daarnaast \$1.000 op het spel zet in een
start-up binnen de nieuwe economie. In dat scenario betaalt een
Amerikaan op eventuele winsten aanzienlijk meer belasting dan iedere
andere belegger. De bijzondere, in feite strafbelastende heffing op
buitenlandse investeringen -- zoals blijkt uit de zogenoemde
PFIC-belasting -- in combinatie met de Amerikaanse
nationaliteitsbelasting kan ertoe leiden dat belastingverplichtingen
oplopen tot 200 procent of meer op langetermijnactiva die buiten de
Verenigde Staten worden aangehouden. Een succesvolle Amerikaan kan zijn
levenslange belastingdruk verlagen door staatsburger te worden van een
van meer dan 280 andere rechtsgebieden wereldwijd.

De Verenigde Staten beschikken over het meest roofzuchtige, de rijken
uitpersende belastingstelsel ter wereld. Zowel binnen als buiten de VS
worden Amerikanen meer als een numerieke last behandeld en minder als
gewaardeerde klanten vergeleken met staatsburgers uit andere landen.
Daardoor blijkt het Amerikaanse belastingregime niet alleen achterhaald,
maar ook minder geschikt voor succes in het informatietijdperk dan zelfs
de berucht hoogbelaste verzorgingsstaten in Scandinavië. Burgers in
Denemarken of Zweden ondervinden nauwelijks juridische belemmeringen bij
het realiseren van hun toenemende technologische autonomie als individu.
Wil men zijn eigen belastingtarief bepalen, dan kan men er bijvoorbeeld
voor kiezen om in Zwitserland belast te worden op basis van een
privéverdrag of naar Bermuda te verhuizen, zodat men helemaal geen
inkomstenbelasting betaalt. Een Zweed of Deen die bereid is hoge
belastingen te betalen omdat hij overtuigd is van de meerwaarde van de
Scandinavische verzorgingsstaat, maakt daar bewust voor zijn keuze. Hij
kan belasting betalen tegen elk tarief dat in een ander rechtsgebied
geldt -- of dat nu in de beschaafde of in de onbeschaafde wereld
gebruikelijk is. Om zijn belastingtarief aan te passen, hoeft hij enkel
te verhuizen. Technologie maakt zo'n beslissing met de dag makkelijker,
maar Amerikanen wordt die mogelijkheid ontzegd. Het bezit van een
Amerikaans paspoort staat immers op het punt een groot nadeel te worden
bij het benutten van de kansen op individuele autonomie die de
informatie-revolutie biedt. In de industriële periode werd het als een
fortuinlijke omstandigheid beschouwd om als Amerikaan geboren te worden
-- maar in de vroege fase van het informatietijdperk blijkt dat
inmiddels een last ter waarde van meerdere miljoenen dollars te worden.

Om te illustreren hoe zwaar deze last weegt, laten we een vergelijking
maken. Onder redelijke aannames betaalt een Nieuw‑Zeelander met dezelfde
pre-belaste inkomensprestaties als het gemiddelde van de top 1 procent
van de Amerikaanse belastingplichtigen zoveel minder belasting dat de
samengestelde besparingen hem op lange termijn rijker maken dan welke
Amerikaan dan ook. Aan het einde van zijn leven zou de Nieuw‑Zeelander
maar liefst \$73 miljoen extra overhouden om door te geven aan zijn
kinderen of kleinkinderen. En Nieuw‑Zeeland is niet eens officieel een
belastingparadijs. Meer dan veertig andere rechtsgebieden hebben immers
lagere belastingtarieven op inkomen en vermogen. Als deze redenering
klopt, zal het aantal landen met lage belastingen waarschijnlijk
toenemen in plaats van afnemen. Al deze jurispudenties bieden als
vestigingsplaats een voordeel ten opzichte van de Verenigde Staten dat
over een heel leven gezien tientallen, zo niet honderden miljoenen
dollars kan opleveren. Tenzij de Amerikaanse belastingen zodanig
hervormd worden dat zij concurrerender worden met die van andere landen
en niet langer op nationaliteitsbasis worden geheven, zullen verstandige
mensen uiteindelijk afstand doen van hun Amerikaans staatsburgerschap --
ondanks de obstakels die de exitbelasting van Clinton met zich
meebrengt.

De competitieve condities van het informatietijdperk maken het mogelijk
bijna overal hoge inkomens te verdienen. In feite zullen de
locatiegebonden monopolies die natiestaten inzetten om extreem hoge
belastingen af te dwingen door technologische vernieuwing teniet worden
gedaan. Ze brokkelen nu al af en naarmate ze verder verzwakken, zal de
druk van de concurrentie de meest ondernemende en capabele mensen
vrijwel onvermijdelijk dwingen landen met exorbitante belastingtarieven
achter zich te laten. Zoals voormalig hoofdredacteur van \emph{`The
Economist'}, Norman Macrae, het verwoordde: zulke landen `zullen
residueel bewoond worden, voornamelijk door dummies.'

\begin{quote}
`{[}B{]}ij het jaar 2012 zullen de geraamde uitgaven voor aanspraken en
rente op de nationale schuld alle door de federale overheid geïnde
belastinginkomsten opslokken. \ldots{} Er zal geen cent overblijven voor
onderwijs, kinderprogramma's, snelwegen, nationale verdediging of enig
ander discretionair programma.' - \emph{`BIPARTISAN U.S. COMMISSION ON
ENTITLEMENT AND TAX REFORM'}
\end{quote}

De uittocht van de rijken uit hoogontwikkelde verzorgingsstaten valt
precies samen met het demografisch ongelijke moment. Aan het begin van
de eenentwintigste eeuw krijgen de vergrijzende bevolkingsgroepen in
Europa en Noord-Amerika te maken met een schrale spaarsom, waardoor zij
hun medische kosten en levensstijl tijdens de pensioenjaren niet kunnen
bekostigen. Zo blijkt dat maar liefst 65 procent van de Amerikanen
helemaal geen spaargeld voor hun pensioen heeft -- helemaal niets. En
degene die wel sparen, sparen ver te weinig. De gemiddelde Amerikaan
bereikt de leeftijd van vijfenzestig, maar rekent ermee dat hij voor
meer dan \$200.000 aan medische rekeningen komt te zitten vóór zijn
overlijden en een nettovermogen heeft van minder dan \$75.000. Zelfs de
weinigen met privépensioenen zullen waarschijnlijk niet in comfortabele
omstandigheden leven, want het gemiddelde pensioen dekt slechts 20
procent van het inkomen dat men vóór pensionering verdiende. Daarbij
vertegenwoordigen de meeste bezittingen van de typische gepensioneerde
geen werkelijk vermogen, maar `transcendent kapitaal' -- de te
verwachten waarde van toekomstige overdrachtsuitkeringen. Mensen raken
er inmiddels aan gewend dat men op deze uitkeringen vertrouwt om de
kloof in hun eigen middelen op te vullen, terwijl de addert onder het
gras is dat deze uitkeringen waarschijnlijk niet gerealiseerd worden.
Pay‑as‑you‑go-systemen beschikken simpelweg niet over de benodigde
cashflow of middelen om aan deze verplichtingen te voldoen. Uit een
onderzoek van Neil Howe blijkt dat, zelfs als de Amerikaanse
voorbelastinginkomsten sneller zouden stijgen dan in de afgelopen
twintig jaar, de gemiddelde na‑belastinginkomsten tegen 2040 met 59
procent moeten dalen om de huidige niveaus van \emph{Social Security} en
overheidsgezondheidsprogramma's te bekostigen.

Je kunt dit probleem niet zomaar omzeilen. De verzorgingsstaat staat op
de rand van insolventie en haar financieringskwesties zijn zelfs
nijpender in Europa dan in Noord-Amerika. Italië vormt wellicht het
ergste voorbeeld, gevolgd door Zweden en andere Noordse
verzorgingsstaten die bekend staan om hun royale inkomensondersteuning.
\emph{Financial Times} schat dat, wanneer de contante waarde van de
Italiaanse staats pensioenen wordt meegerekend, de publieke schuld van
het land meer dan 200 procent van het BBP zal bedragen.

Een schuldenlast van dergelijk niveau is volgens de cijfers vrijwel
onhoudbaar. Een diepgravend onderzoek naar de commerciële schuldenlast
van bedrijven op de Toronto Stock Exchange, uitgevoerd enkele jaren
geleden, toonde aan dat nauwelijks iemand een schuldratio kan overleven
die al een kwart zo extreem is als die waarmee de toonaangevende
verzorgingsstaten vandaag worstelen.

Kortom, zij zijn blut. Wanneer men deze onvermijdelijke realiteit -- zij
het met tegenzin -- eindelijk onder ogen ziet, volgt er letterlijk een
afschrijving van biljoenen aan niet-gefinancierde aanspraken.

Zo werkt de logica van de cybereconomie. Een mogelijke belemmering is
simpelweg traagheid -- dat nestinstinct waardoor mensen huiverig zijn
hun vertrouwde stek op te geven. Als er andere obstakels zijn, zitten
die mogelijk diep in de menselijke aard verankerd. De economische logica
achter het inzetten van activa in cyberspace kan echter botsen met de
biologische basis, die zich uit in een diepgeworteld wantrouwen jegens
buitenstaanders. In elke cultuur tonen kinderen van nature een afkeer
van vreemden. Tegenstanders van de commercialisering van soevereiniteit
doen er alles aan om twijfel te zaaien over de nieuwe mondiale cultuur
van het informatiestijdperk en over het verval van de natiestaat dat
daarmee gepaard gaat. Een andere mogelijke belemmering, die voortkomt
uit epigenese -- oftewel genetisch beïnvloede motivatiefactoren -- is
het vooruitzicht dat de `verliezers en achterblijvers' zullen reageren
op ontwikkelingen die de natiestaat ondermijnen met de woede van
jager-verzamelaars die hun familie beschermen. In een omgeving waarin
gedesoriënteerde en vervreemde mensen meer ruimte krijgen om te
verstoren en te vernietigen, kan een terugslag tegen de
informatie-economie leiden tot gewelddadige en onaangename gevolgen.

\begin{quote}
`Historisch gezien is collectief geweld regelmatig voortgekomen uit de
centrale politieke processen van Westerse landen. Mensen die erop uit
zijn om de machtshefbomen te grijpen, te behouden of te herschikken,
hebben in hun strijd steeds gebruikgemaakt van collectief geweld. De
onderdrukten hebben in naam van gerechtigheid geslagen, de bevoorrechten
in naam van orde, en de daartussen degenen in naam van angst. Grote
verschuivingen in de machtsverhoudingen hebben doorgaans geleid -- en
waren vaak afhankelijk van -- uitzonderlijke momenten van collectief
geweld.'\footnote{Tilly, `collective violence in European perspective,'
  p.~62.}
\end{quote}

\begin{quote}
CHARLES TILLY
\end{quote}

\section{Geweld in perspectief}\label{geweld-in-perspectief}

Er bestaan minstens twee concurrerende theorieën over de oorzaken van
geweld in veranderende omstandigheden. Historicus Charles Tilly vat een
van deze ideeën als volgt samen: `{[}D{]}e prikkel voor collectief
geweld komt grotendeels voort uit de angsten die mensen ervaren wanneer
gevestigde instituties uiteenvallen. Als ellende of gevaar de angst
verergert, wordt de reactie des te gewelddadiger.' Tilly betoogt dat
geweld niet primair uit angst voortkomt, maar een weloverwogen poging is
om de autoriteiten te dwingen hun verantwoordelijkheden na te komen --
gedreven door een `gevoel van ontzegde rechtvaardigheid.' Volgens Tillys
interpretatie leiden `grote structurele veranderingen' vaak tot
collectief geweld met een politieke inslag. In plaats van een abrupte
breuk met het `normale' politieke leven te markeren, gaan gewelddadige
opstanden vaak gepaard met georganiseerde, vreedzame inspanningen van
dezelfde mensen om hun doelstellingen te bereiken; ze vullen deze
aanvullen en versterken ze. Ze maken immers deel uit van dezelfde
dynamiek als geweldloos verzet.\footnote{Ibid., p.~68.}

Ongeacht welke theorie over de oorzaken van geweld het meest klopt,
lijken de vooruitzichten op sociale vrede tijdens de Grote Transformatie
beperkt. De ineenstorting van de natiestaat vormt immers een opvallend
voorbeeld van een gevestigde orde die uiteenvalt. Hierdoor zullen de
angsten waarschijnlijk sterk toenemen, net als de politieke stimulans
voor geweld. Dit geldt vooral in vooraanstaande verzorgingsstaten, waar
de bevolking gewend is aan een zekere mate van inkomensgelijkheid. Omdat
de mensen die nu volwassen worden in de vroege fase van de
informatie-economie zijn opgegroeid in een industriële tijd waarin
politieke autoriteiten klachten met materiële voordelen konden
beslechten, mag men verwachten dat de `achterblijvers' blijven
aandringen op materiële voorzieningen. Waarschijnlijk vergt het een
langzame en pijnlijke kennismaking met de realiteit van de cybereconomie
voordat de bevolking van de OESO loskomt van de verwachting dat
grootschalige inkomensherverdeling afgedwongen kan worden. Hoe dan ook,
of geweld nu voortvloeit uit angst of het resultaat is van een
weloverwogen poging om te profiteren van systematische dwang, de
omstandigheden maken geweld bijna onvermijdelijk.

\subsection{Achterban van verliezers}\label{achterban-van-verliezers}

De ondergang van de gedwongen inkomensherverdeling zal ongetwijfeld
degenen van streek maken die gewend zijn te rekenen op de biljoenen die
via overdrachtprogramma's worden verdeeld. Dit betreft doorgaans `de
verliezers' of `de achterblijvers', mensen die niet over de vaardigheden
beschikken om op de wereldmarkt te concurreren. Net zoals de
gepensioneerden in de voormalige Sovjet-Unie, die de kern vormden van
Zuganovs communistische aanhang, zullen de teleurgestelde
gepensioneerden van de vervalende verzorgingsstaten een reactionaire
achterban vormen. Zij zullen fel protesteren tegen de privatisering van
natiestaatelijke soevereiniteit, waardoor de staat zijn vergunning om te
stelen verliest. Naarmate zij beseffen dat de regeringen die zij eens
beheersten hun controle over hulpbronnen en hun vermogen om
grootschalige inkomensoverdrachten af te dwingen kwijtraken, worden zij
net zo vastberaden als Franse ambtenaren in hun strijd tegen de
rekenkunde.

U herinnert zich vast nog de heftige reactie die volgde op de tamelijk
bescheiden voorstellen van premier Alain Juppe om de `demografisch
onhoudbare' pensioenregelingen voor staatsmedewerkers in te krimpen en
de bedrijfsvoering van het genationaliseerde spoorwegsysteem efficiënter
in te richten. Een treffend symbool van de absurditeit van l'État
Providence, zoals de Fransen hun sociale zekerheidsstaat noemen, blijkt
uit de regel die ingenieurs op de geavanceerde hogesnelheidstreinen
\emph{TGV} toestaat om al op vijftigjarige leeftijd met pensioen te gaan
-- net als hun voorgangers die op door kolengestookte locomotieven
ploeterden.\footnote{Dick Howard, `french toast: can politicians
  anywhere tangle with entitlements without getting burned?' \emph{The
  New Democrat}, juli/augustus 1996, p.~39f.} Een heftig verzet tegen de
bezuinigingen op zulke onhoudbare voordelen is in elk OESO-land reëel.
Zelfs in landen waar de bevolking minder boos reageert, kunt u erop
rekenen dat de waarschijnlijke verliezers er alles aan zullen doen om de
inperking van staatsdwang te voorkomen.

Dit alles zal leiden tot enkele onverwachte wendingen. In de Verenigde
Staten heeft het nativistische sentiment van oudsher duidelijke
racistische ondertonen -- een traditie die begon met de \emph{White
Caps} en de \emph{Ku Klux Klan} uit de negentiende eeuw. Toch profiteren
zwarte mensen als groep in de eerste plaats van inkomensoverdrachten,
positieve discriminatie en andere vruchten van politieke dwang.
Bovendien zijn zij onevenredig vaak vertegenwoordigd in het Amerikaanse
leger. Daarom zullen zij, samen met de blanke arbeidersklasse,
waarschijnlijk opstaan als enkele van de meest vurige aanhangers van het
Amerikaanse nationalisme.

Politici die inspelen op de onzekerheden van degenen wiens talenten
relatief ver achterblijven bij Ammon's rapen, zullen in vrijwel elk land
luidruchtig de kop op de voorgrond zetten. Van Slobodan Milosevic in
Servië tot Pat Buchanan in de Verenigde Staten, van Winston Peters in
Nieuw-Zeeland tot Necmettin Erbakan van de fundamentalistische
Islamitische Welvaartspartij in Turkije -- demagogen zullen fel
protesteren tegen de globalisering van markten, immigratie en de
vrijheid om te investeren.

Degenen die zichzelf zien als `slachtoffers van de wereldeconomie'
richten hun specifieke vijandigheid op zowel de rijken als de
immigranten. Volgens Andrew Heal zullen zij `de binnenkomst van
immigranten verachten, waarvan het voornaamste toelatingscriterium
schijnbaar hun rijkdom of juist het ontbreken daarvan is, wat hen
volgens de schijnbare logica tot een last voor de verzorgingsstaat
maakt.'

\subsection{Angst voor vrijheid}\label{angst-voor-vrijheid}

Het vooruitzicht dat de natiestaat aan het begin van het nieuwe
millennium verdwijnt, lijkt zo getimed dat het voor de meest kwetsbaren
in hun leven de maximale ontwrichting veroorzaakt. Dit leidt tot
wijdverspreide onvrede. Velen merken op dat mensen die zich
buitengesloten voelen door de gedachte aan een wereld zonder grenzen
vaak op een herkenbare manier reageren. Naarmate de brede, inclusieve
nationale groep uiteenvalt en de meer mobiele `informatie-elite' haar
zaken globaliseert, zoeken de achterblijvers houvast in het lidmaatschap
van etnische subgroepen, stammen, bendes of religieuze en taalkundige
minderheden. Dit is deels een praktische reactie op het wegvallen van
diensten -- zoals de handhaving van orde -- die de staat vroeger
verzorgde. Voor mensen met weinig verhandelbare middelen is het vaak
lastig om toegang te krijgen tot marktalternatieven wanneer de publieke
voorzieningen falen.

Wat ooit als publieke goederen werden gezien -- onderwijs, schoon water
en buurtpolitiezorg -- verandert nu in particuliere producten. Degenen
met voldoende middelen om kwalitatief hoogwaardige alternatieven aan te
schaffen, weten hier hun voordeel mee te doen. Voor wie op contant geld
wil vertrouwen, is het praktisch om op familie te rekenen of zich aan te
sluiten bij een etnisch gebaseerde hulpgroep, zoals de oude etnische
Chinese `Hokkien' in Zuidoost-Azië, of bij een religieuze gemeenschap.
In regio's waar dynamische, missionerende religies actief zijn, winnen
hun programma's mede aan populariteit doordat ze terugvallen op
traditionele methoden voor het leveren van sociale voorzieningen en
publieke diensten. Zo namen bijvoorbeeld door moslims geleide
waakhondgroepen een voortrekkersrol in de bestrijding van gewelddadige
bendes in Kaapstad, Zuid-Afrika.\footnote{Roger Matthews, `Zuid-Afrika
  roept troepen op voor oorlog tegen criminaliteit,' \emph{Financial
  Times}, 31 augustus/1 september 1996, p.~1.}

Hoe nuttig en effectief die etnische en religieuze hulporganisaties ook
blijken te zijn, er speelt meer mee in de reactionaire respons op het
afbrokkelen van de staat. Daarbij speelt ook een sterk psychologisch
element een belangrijke rol in de tegenreactie op globalisering.

Dit standpunt sluit nauw aan bij de psychologische verklaring voor de
aantrekkingskracht van het fascisme, zoals Erich Fromm beschreef in zijn
beroemde werk `vrees voor vrijheid', dat voor het eerst verscheen in
1941.\footnote{Erich Fromm, \emph{Angst voor vrijheid} (Londen:
  Routledge \& Kegan Paul, 1942).} Fromm betoogt dat de sociale
mobiliteit die het kapitalisme met zich meebracht, de stabiele
identiteiten in het traditionele dorpsleven verbrijzelde. De zoon van
een boer wist niet meer zeker dat hij onvermijdelijk het boerenambacht
zou overnemen of gedoemd was te ploeteren op de magere grond waarop zijn
vader werkte. Plotseling had hij keuzevrijheid: hij kon leraar,
handelaar of soldaat worden, geneeskunde studeren of de zee opgaan.
Zelfs als boer bleef hij niet langer gebonden aan de oude gewoonten; hij
kon naar de Verenigde Staten, Canada of Argentinië emigreren en ver van
het ouderlijk huis een nieuw bestaan opbouwen. Deze vrijheid om eigen
identiteit te creëren, bood het kapitalisme zeker kansen, maar maakte
het ook angstaanjagend voor hen die niet bereid waren die vrijheid
creatief te benutten. Zoals Billig opmerkte, hunkerden zij naar `de
zekerheid van een solide identiteit' en voelden zij zich aangetrokken
tot `de eenvoud van nationalistische en fascistische
propaganda.'\footnote{Billig, \emph{op. cit.}, p.~137.} Eveneens
omschrijft Billig de schemering van het industriële tijdperk als volgt:
`Er is een mondiale psychologie die de natie van bovenaf treft en
loyaliteiten doet verwelken door het vrije spel van identiteiten. En dan
is er de vurige psychologie van kaste of stam, die met een krachtige
intolerante inzet en emotionele fellerheid de kwetsbare onderbuik van de
staat raakt.'\footnote{Ibid., p.~135.}

Andrew Heal benadert dit fenomeen vanuit een ander perspectief. Hij
onderscheidt twee grote wereldwijde politieke en economische trends:
enerzijds groeit de wereldeconomie, en anderzijds nemen
nationalistische, etnische en regionalistische sentimenten toe -- of het
nu gaat om de Maori, Schotten, Welsh of anti-immigrantgroepen -- die,
zelfs wanneer hun regeringen hen naar nieuwe, grenzeloze horizonten
drijven, zichzelf juist nog sterker in de tegenovergestelde richting
oriënteren. Hoe men er ook naar kijkt -- of men deze ontwikkelingen
beschouwt als belangrijke trends of als psychologische thema's -- het
valt niet te ontkennen dat het reactionaire sentiment dat het
nationalisme omarmt en zich verzet tegen het verdwijnen van grenzen en
de verdere integratie van markten wereldwijd, steeds sterker wordt.

\section{multiculturalisme en
slachtofferschap}\label{multiculturalisme-en-slachtofferschap}

Toen de verzorgingsstaat in haar laatste fase kwam en haar vermogen om
loze beloftes waar te maken al wankeler werd, ging ze nieuwe mythen over
discriminatie koesteren. Overal in Noord-Amerika benoemde men diverse
groepen als officieel `onderdrukte' mensen. Men vertelde individuen die
tot deze zogenaamde slachtoffergroepen behoorden dat zij niet
verantwoordelijk waren voor de gebreken in hun eigen leven. Integendeel,
men wees `dode blanke mannen' van Europese afkomst aan als schuldigen en
stelde dat een onderdrukkende machtsstructuur de uitgesloten groepen
benadeelde. Als je zwart, vrouwelijk, homoseksueel, Latino, Franstalig
of gehandicapt was, betekende dat dat je recht had op compensatie voor
vroegere discriminatie en onderdrukking.

Volgens Lasch had het versterken van het slachtoffergesin als doel
naties te verzwakken, zodat de nieuwe, ongebonden informatielobby
gemakkelijker aan de verplichtingen van het burgerschap kon ontsnappen.
Wij geloven echter niet dat de nieuwe elite, vooral de mensen in de
massamedia, zo sluw is dat zij zo'n houding zouden aannemen. Het zou
bijna geruststellend klinken als dat wel zo was. Wij zien de groei van
slachtofferschap vooral als een poging om de sociale vrede te kopen,
niet alleen door -- zoals Lasch betoogt -- de instroom in de
meritocratie te vergroten, maar ook door de rechtvaardigingen voor
inkomensherverdeling te herstructureren. In Noord-Amerika bereikte
victimologie haar overdreven vorm, mede doordat de informatietechnologie
daar dieper doordrong. Wij vermoeden echter dat nieuwe mythen over
discriminatie, zij het in een ietwat verouderde variant, uiteindelijk in
alle geïndustrialiseerde samenlevingen zullen voorkomen. De
multietnische verzorgingsstaten in Noord-Amerika gaven sneller toe aan
de verleiding om de kosten van inkomensherverdeling op de privésector af
te schuiven. Ze slaagden erin om een gevoel van benadeeldheid en van
recht op compensatie wakker te maken door de gehele maatschappelijke
structuur -- en in het bijzonder blanke mannen -- de schuld te geven
voor de economische tekortkomingen binnen verschillende subculturen.

De megapolitiek van innovatie

Nog voordat informatietechnologie dreigde met de `creatieve
vernietiging' van de industriële economie, weerlegde zij al duidelijk
veel van de gekoesterde mythen van marxisten en socialisten. In een
eerder hoofdstuk onderzochten we de megapolitiek van innovatie. Wat we
daar benadrukten helpt om de maatschappelijke impact van de
informatierevolutie in perspectief te plaatsen. Hoewel technologie in de
afgelopen eeuwen de arbeidskansen aanzienlijk heeft vergroot, garandeert
dat geen blijvende economische regelmaat. Het is namelijk mogelijk dat
de opbrengsten zich concentreren in de handen van een welvarende
minderheid.

\section{Reële lonen dalen met 50
procent}\label{reuxeble-lonen-dalen-met-50-procent}

Precies dat gebeurde in de eerste twee eeuwen -- of zelfs langer -- van
de moderne periode. Rond de buskruitrevolutie, zo'n periode van ongeveer
1500 tot 1700, zakten de reële inkomens van de onderste 60 tot 80
procent in West-Europa met vijf decimale procenten of meer. In veel
streken bleef het reële inkomen dalen tot ongeveer 1750 en keerde het
pas in 1850 weer terug op het niveau van 1500.

In tegenstelling tot de ontwikkelingen van de afgelopen 250 jaar gingen
de inkomenswinsten in de eerste helft van de moderne periode -- een tijd
waarin de West-Europese economieën dramatisch groeiden -- enkel naar een
kleine minderheid. De hedendaagse doorbraak in informatietechnologie
verschilt fundamenteel van de industriële vernieuwingen van voorgaande
eeuwen. Moderne, arbeidsbesparende innovaties leiden er juist toe dat
werk sterk gespecificeerd raakt en schaalvoordelen afnemen -- precies
het tegenovergestelde van wat we sinds circa 1750 zagen.

Industriële vernieuwing bood doorgaans werkgelegenheid aan
laaggeschoolde arbeiders en vergrootte de schaalvoordelen voor
bedrijven. Hierdoor stegen de inkomens van de minder bedeelden, vaak
zonder dat zij er veel voor hoefden te doen, en versterkten deze
veranderingen tevens de macht van politieke systemen, waardoor zij beter
bestand waren tegen sociale onrust. Wie in de vroege fase van de
Industriële Revolutie door mechanisatie en automatisering zijn baan
verloor, betrof meestal vakbekwame ambachtslieden en gezellen -- niet de
ongeschoolde werknemers. Dit kwam duidelijk naar voren in de
textielindustrie, de eerste sector die op grote schaal mechanisatie en
krachtapparatuur toepaste. De opkomst van deze machines leidde tot een
hevige opstand van de Luddieten, die textielmachines vernietigden en
fabriekseigenaren vermoordden in een razernij aan het begin van de
negentiende eeuw. Aan de andere kant behoorden dagloners tot de
volgelingen van Captain Swing, de legendarische leider van een opstand
in 1830 in zuidoostelijk Engeland. Zij eisten dat lokale welgestelden
een belasting invoerden om hen geld of bier te verschaffen, dat de
loonsverhoging bij dagloners bij werkgevers werd afgedwongen en dat
nieuwe landbouwmachines -- met name dorsmachines -- werden vernietigd,
omdat deze de vraag naar landelijke dagarbeiders bij boeren deden
afnemen.

In tegenstelling tot het romantische gepraat van marxisten en anderen,
die de tegenstanders van arbeidsbesparende technologie als helden
opdroegen, bestond dit gezelschap uit onaangename en gewelddadige
figuren die zich uitsluitend uit eigenbelang verzette tegen de
introductie van technologie die wereldwijd de levensstandaard
verbeterde.

Terwijl de gewelddadige volgelingen van Ned Ludd en Captain Swing
maandenlang de openbare orde in Engeland verstoorden, was het
onvermijdelijk dat hun opstanden mislukten zodra een centrale autoriteit
hen de kop indrukte. Bovendien sloeg de arme, laaggeschoolde meerderheid
al snel de oproep af om machines te vernietigen, omdat diezelfde
machines niet alleen banen creëerden, maar ook de kosten van
basisgoederen zoals warme kleding en brood verlaagden en zo hun
levensstandaard verbeterden.

\subsection{Hogere inkomsten voor de
laaggeschoolden}\label{hogere-inkomsten-voor-de-laaggeschoolden}

Na verloop van tijd ontdekten de minderbedeelden dat industriële en
agrarische automatisering nieuwe inkomenskansen bood en hun kosten van
levensonderhoud verlaagde. Dankzij technische vernieuwingen kon ook
iemand zonder specialistische kennis goederen produceren van dezelfde
kwaliteit als die door hoogopgeleiden werden vervaardigd. Op een lopende
band zou zowel een genie als een idioot tot hetzelfde product komen en
evenveel verdienen.

In de afgelopen twee eeuwen heeft industriële automatisering de lonen
voor ongeschoolde arbeid drastisch doen stijgen, vooral in dat kleine
deel van de wereld waar als eerste de omstandigheden voor het
kapitalisme aanwezig waren. De grootschaligheid van geavanceerde
industriële ondernemingen beloonde ongeschoolde werknemers niet alleen
met ongekende lonen, maar zorgde ook voor een herverdeling van inkomens.

De welvaartsstaat ontstond als logische uitkomst van de technologische
ontwikkelingen in het industrialisme. Door hun enorme schaal en hoge
kapitaalkosten vormden de leidende industriële werkgevers de meest voor
de hand liggende doelwitten voor belastingen. Men ging ervan uit dat
deze ondernemingen een nauwkeurige administratie voerden en
looninhoudingen beheersten, waardoor de invoering van de
inkomstenbelasting technisch realiseerbaar werd -- iets wat in
voorgaande eeuwen, toen de economieën nog gedecentraliseerder waren,
niet mogelijk was. Uiteindelijk zorgde de door industriële innovatie
versnelde groei van schaalvoordelen ervoor dat overheden rijker werden
en daardoor vermoedelijk beter in staat waren de orde te handhaven.

\subsection{Het proces is omgekeerd}\label{het-proces-is-omgekeerd}

Naar onze mening gebeurt er tegenwoordig juist het tegenovergestelde.
Informatietechnologie vergroot de verdienkansen van geschoolden en
ondermijnt grootschalige instituties, waaronder de natiestaat.

Dit wijst op nog een ironie van ons informatietijdperk: de schijnbaar
tegenstrijdige en fundamenteel belemmerende houding van vrije
marktcritici ten aanzien van zowel de opkomst als het verdwijnen van
industriële banen. In de beginfase van het industrialisme waren deze
critici geraakt door wat zij zagen als het kwaad van fabrieksbanen, die
landloze boeren weg trokken uit `de wereld die we verloren hebben.' Zij
beschouwden de komst van fabrieksbanen als een ongekend kwaad en als
`uitbuiting' van de arbeidersklasse. Nu lijkt het echter zo dat het
echte probleem niet de komst, maar juist het verdwijnen van
fabrieksbanen is. Zelfs de achterkleinkinderen van degenen die ooit
mopperden over de introductie ervan, klagen nu over het tekort aan banen
die een hoog loon bieden voor laaggeschoold werk.

De rode draad in al deze klachten is een onverzettelijke weerstand tegen
technologische vernieuwing en marktveranderingen. In de beginfase van
het fabrieksysteem leidde die weerstand zelfs tot geweld. Dat zou
opnieuw kunnen gebeuren.

En dat niet omdat kapitalisten de arbeiders `uitbuiten'. De opkomst van
de computer als paradigma-technologie toonde immers de belachelijkheid
van die stelling. Voor de onoplettene zou het nog enigszins aannemelijk
kunnen lijken dat een nauwelijks geletterde automedewerker op de een of
andere manier `uitbuit' werd bij de assemblage van een auto door
eigenaren die de ondernemingen hadden bedacht en gefinancierd waarin
arbeiders in dienst waren. De cruciale rol van conceptueel kapitaal bij
de productie en marketing van fysieke producten kwam minder duidelijk
naar voren dan bij de output van het informatietijdperk, waarin mentaal
werk duidelijk de boventoon voerde. Hierdoor won de bewering dat
ondernemers op de een of andere wijze de waarde van informatieproducten
-- in werkelijkheid door werknemers gecreëerd -- toe-eigenden,
grotendeels aan betekenis in. Wanneer de waarde onmiskenbaar voortkwam
uit mentaal werk -- zoals bij de ontwikkeling van consumentensoftware --
wordt het ronduit belachelijk geacht te stellen dat dat werk niet door
de experts is verricht die het hebben bedacht. Sterker nog, de
duidelijke en toenemende verschuiving weg van ongeschoolde arbeid riep
juist de vraag op of ongeschoolde arbeiders nog wel een economische
bijdrage kunnen leveren.\footnote{William~Julius~Wilson, \emph{Wanneer
  werk verdwijnt: de wereld van de nieuwe stedelijke armen} (New York:
  \emph{Alfred A. Knopf}, 1996).}

Hierdoor verandert de basis waarop inkomens wordt herverdeeld: vroeger
ging men uit van uitbuiting -- wat inhield dat mensen met lage inkomens
toch productief konden zijn -- terwijl men nu uitgaat van discriminatie,
die dat niet veronderstelt. Men betoogde bovendien dat discriminatie
verklaarde waarom mensen met beperkte vaardigheden er niet in slaagden
om waardevollere capaciteiten te ontwikkelen.

Deze vorm van discriminatie diende eveneens als rechtvaardiging voor het
opleggen van suboptimale aanwervingscriteria en andere maatstaven die
zogenaamd `kansen' creëren -- of, om het preciezer te zeggen, voor het
herverdelen van inkomen ten gunste van achterblijvende groepen. In de
Verenigde Staten zorgden rasgebonden aanpassingen in prestatie- en
bekwaamheidstesten er bijvoorbeeld voor dat zwarte kandidaten hogere
scores behaalden dan witte en Aziatische sollicitanten, terwijl zij op
objectieve maten lager scoorden. Via deze en andere methoden
verplichtten overheden werkgevers om meer zwarte mensen en andere
officieel als `benadeeld' bestempelde groepen in dienst te nemen tegen
hoger loon; wie de regels negeerde, kreeg te maken met kostbare
rechtszaken en procedures met hoge strafschadevergoedingen.

Het aanwijzen van slachtoffers had nooit tot doel paranoïde
achtervolgingswanen te creëren binnen belangrijke subgroepen van de
industrie, noch om het verspreiden van contraproductieve waarden te
bevorderen. Men wilde de falende staat immers ontlasten van de fiscale
druk die samenhangt met inkomensherverdeling. Het ontstaan van
achtervolgingswanen bleek echter een ongelukkige bijwerking van dit
streven. Ironisch genoeg viel de opkomst van de bezorgdheid over
`discriminatie' samen met de beginfase van een technologische revolutie
die ervoor zal zorgen dat willekeurige discriminatie in de toekomst veel
minder een probleem vormt. Op het internet bekommert niemand zich erover
of de auteur van een nieuw softwareprogramma zwart, wit, man, vrouw,
homoseksueel of een vegetarische dwerg is.

Hoewel toekomstige vormen van discriminatie vermoedelijk minder
onderdrukkend zullen zijn, neemt dat de roep om `reparaties' -- om
diverse reële of ingebeelde onrechtvaardigheden te compenseren -- niet
per se weg. Iedere samenleving, ongeacht de objectieve omstandigheden,
levert immers altijd één of meer rechtvaardigingen voor
inkomensherverdeling op. Deze motiveringen lopen uiteen van subtiele tot
absurde argumenten, van het bijbelse gebod om uw naaste lief te hebben
als uzelf tot het aanroepen van zwarte magie. Hekserij, tovenarij en het
kwaad oog vormen de keerzijde van religieuze gevoelens, het spirituele
equivalent van de Belastingdienst of de \emph{IRS}. Als mensen niet uit
liefde de armen steunen, zorgen de armen er via angst zelf voor dat ze
in actie komen. Soms uit zich dat in regelrechte afpersing -- een mes
aan de keel, een pistool aan het hoofd --, terwijl op andere momenten de
dreiging zich in vermomde of vergezochte gedaante toont. Het valt
daarbij niet te merken dat in de vroegmoderne periode de meeste `heksen'
weduwen of ongetrouwde vrouwen met beperkte middelen waren. Zij
terroriseerden hun buren met vloeken, waardoor die buren er niet zelden
toe overgingen te betalen. Je kunt dan ook niet zomaar aannemen dat zij
puur bijgelovig waren, want de kwaadaardige intentie achter het kwaad
oog bleek reëel. Zelfs een arme vrouw kon vee loslaten of iemands huis
in brand steken. In dat opzicht waren de heksenprocessen van weleer niet
zo belachelijk als ze op het eerste gezicht lijken, hoe wreed de
straffen ook waren en hoeveel onschuldigen er ten prooi vielen aan de
waanbeelden die ontstonden door schimmelvergiftiging.

Wij voorzien dat afpersing weer oplaait, gedreven door de wens mee te
profiteren van de vruchten van prestaties, naarmate het
informatietijdperk zich verder ontwikkelt. Groepen die zich benadeeld
voelen door vroegere discriminatie laten hun schijnbaar waardevolle
status als slachtoffer niet zomaar los, simpelweg omdat hun aanspraken
op de samenleving minder onderbouwd of moeilijker af te dwingen blijken.
Zij blijven hun rechten handhaven totdat het bewijs in hun directe
omgeving geen enkele twijfel meer toelaat dat zij niet de beloningen
ontvangen die hen toekomen.

De toename van sociopathisch gedrag onder Afro-Amerikanen en
Afro-Canadezen bevestigt dit punt. Het suggereert dat er nauwelijks
balans bestaat tussen de zwarte woede en een realistische inschatting
van in hoeverre de problemen binnen de zwarte gemeenschap het gevolg
zijn van antisociaal gedrag. De woede is gegroeid, terwijl de manier van
leven steeds disfunctioneler wordt. Het aantal kinderen dat buiten het
huwelijk wordt geboren, is explosief toegenomen, het opleidingsniveau
daalt en een groeiend percentage jonge zwarten raakt betrokken bij
criminele activiteiten, waardoor er nu meer zwarte mannen in
penitentiaires zitten dan op universiteiten.

Deze perverse ontwikkelingen hadden mogelijk tijdelijk tot gevolg dat er
tijdens de schemering van het industrialisme extra middelen naar de
onderklassegemeenschappen stroomden, doordat de druk die op de
samenleving werd uitgeoefend, toenam. Dit effect blijkt echter slechts
van korte duur te zijn. Doordat de positieve werking van concurrentie --
die mensen die onderpresteren aanspoort zich aan productieve normen te
houden -- wegvalt, heeft de verzorgingsstaat bijgedragen aan de opkomst
van legioenen disfunctionele, paranoïde en slecht aangepaste mensen, het
sociale equivalent van een kruitvat. De ondergang van de natiestaat en
het verdwijnen van grootschalige inkomensherverdelingen leiden er
ongetwijfeld toe dat enkele van deze psychopatische zielen zich keren
tegen iedereen die er welvarender uitziet. Daarom mogen we ervan uitgaan
dat de sociale vrede onder druk komt te staan naarmate het
informatietijdperk zich verder ontvouwt, met name in Noord-Amerika en in
multietnische enclaves in West‑Europa.

\begin{quote}
``Wij zullen onze wapens nooit neerleggen {[}totdat{]} het Lagerhuis een
wet aanneemt om alle machinewerktuigen die schadelijk zijn voor de
gemeenschap neer te leggen, en die wet niet teniet te maken om
framebrekers op te hangen. Maar wij. Wij doen geen smeekbedes meer ---
dat schikt niet --- er moet worden gevochten.''\\
Ondertekend door de generaal van het leger der rechtzetters, Ned
Ludd-klerk, ``Rechtzetter-vooraltijd, amen'' \footnote{Tilly,
  `Collectief geweld', p.~78.}
\end{quote}

Neo-Luddiet

Gezien de antitechnologische opstanden in het begin van de negentiende
eeuw en de lange traditie van collectief geweld in zowel Europa als
Noord‑Amerika, valt het niet te verwonderen als we tegenwoordig een
neo-luddietische aanval op de informatietechnologie en haar gebruikers
zien. De Luddites, zoals eerder genoemd, waren textielarbeiders die zich
in West Yorkshire, Engeland, hadden verzameld en in 1811--1812 een
terroristische campagne voerden tegen geautomatiseerde snoeimachines en
tegen de fabriekseigenaren die deze in gebruik namen.\footnote{Robert~Reid,
  \emph{Land van verloren inhoud: de ludditenopstand 1812} (London:
  \emph{Penguin}, 1986), p.~44.} Met zwartgeblakste gezichten raasden de
Luddites door West Yorkshire, staken fabrieken in brand en vermoordden
fabriekseigenaren die de nieuwe technologie durfden te omarmen.

Het grootste deel van het geweld kwam van de zogenaamde `croppers',
hooggekwalificeerde ambachtslieden wier meesterlijke beheersing van
enorme scharen -- die tot wel vijftig pond wogen -- ooit een cruciaal
onderdeel was van de productie van wollen stoffen.

Het afwerkingsproces dat de croppers toepasten -- het opwekken van
stofpluis met distels en het snoeien van de stof met scharen -- was
volgens Robert Reid, de auteur van `Land of Lost Content: The Luddite
Revolt 1812', `te eenvoudig om niet gemachinaliseerd te
worden.'\footnote{Ibid., p.~45.}

Leonardo da Vinci had al een ontwerp voor zo'n mechanische snoeimachine
uitgetekend, maar zijn idee voor automatische snoei bleef eeuwen
onbenut. Uiteindelijk heruitvonden ingenieurs in 1787 een apparaat naar
het model van Leonardo's ontwerp, en brachten zij het in Engeland in
productie.

Reid merkt daarbij op: `Alle bouwstenen van de technologie waren al
geruime tijd bekend, waardoor het opmerkelijk is dat men deze niet
eerder had doorgevoerd\ldots{} De nieuwe machines uit de Industriële
Revolutie vroegen zo weinig kracht en vaardigheid dat veel vacatures
werden ingevuld door vrouwen en jonge kinderen, aanvankelijk tegen lage
lonen. Één van deze apparaten kon -- zelfs bediend door relatief
ongeschoolden -- in achttien uur verrichten wat een ervaren cropper met
handscharen in achtentachtig uur voor elkaar kreeg.'\footnote{Ibid.,
  p.~26.}

Opmerkelijk is dat werknemers die fel tegen mechanisatie opstandig
werden, zich selectief verzetten tegen technologieën die hun eigen banen
wegvaarden of de vraag naar vakbekwame arbeid deden afnemen.

Toen ondernemer William Cooke tapijtweefmachines invoerde in West
Yorkshire, brak er geenszins geweld uit. Men trachtte zijn molen niet in
brand te steken, zijn machines niet te vernielen en zeker niet hem --
laat staan zijn leven -- te ruïneren. Volgens Reid, die de
Luddite-opstanden uitvoerig beschrijft, riep Cooke's nieuwe technologie
geen verzet op, omdat niemand in het dal tot dan toe gespecialiseerd was
in tapijtproductie. Hij vervolgt: `Doordat Cooke een nieuw product
introduceerde en banen creëerde die niet op traditionele methoden waren
gebaseerd, kende zijn molen een ware bloeiperiode.'

Dit voorbeeld heeft grote implicaties voor de toekomst. Het duidt erop
dat vooruitziende ondernemers in het komende millennium ingrijpende,
arbeidsefficiënte automatisering zullen introduceren in regio's waar men
nog geen traditie had in de productie van het betreffende product of de
dienst.

Als de geschiedenis ons iets leert, dan zullen de meest radicale
terroristen in de eerste jaren van het nieuwe millennium waarschijnlijk
niet uit dakloze verarmden bestaan, maar uit werknemers die ooit tot de
middenklasse behoorden en aanzien genoten, maar nu hun baan kwijt zijn
geraakt. Dat bleek immers tijdens de Luddite-opstand van 1812, waar het
merendeel van de betrokken Luddites geen verarmd proletariaat vormde,
maar hoogopgeleide ambachtslieden waren die gewend waren een inkomen te
verdienen dat ten minste vijfmaal hoger lag dan dat van een gemiddelde
arbeider. De hedendaagse variant zou vermoedelijk bestaan uit verdrongen
fabrieksarbeiders. Helaas toont een blik op de demografie van de meeste
OESO-landen dat er op meerdere plaatsen potentiële knelpunten liggen
voor gewelddadige reacties.

Wereldwijd zullen staten proberen de opkomst van de cybereconomie én die
onafhankelijke individuen tegen te gaan die hiervan profiteren om
rijkdom te vergaren. Deze pogingen roepen een woedende nationalistische
reactie op, waarin onvermijdelijk ook een antitechnologische stroom
meespeelt -- vergelijkbaar met de Luddieten- en andere
antitechnologie-opstanden in Groot-Brittannië tijdens de Industriële
Revolutie. Dit scenario vereist nauwgezette overweging, want het zou wel
eens de sleutel kunnen vormen tot de toekomstige evolutie van bestuur in
het nieuwe millennium. Een van de grootste uitdagingen van de aanstaande
transformatie zal het handhaven van de orde zijn te midden van
escalerend geweld, of het voorkomen van een rampzalige uitbarsting.
Personen en ondernemingen die nauw verbonden zijn met het ontstaan van
het informatietijdperk -- waaronder de spelers in \emph{Silicon Valley}
en zelfs de leveranciers van de elektriciteit die de nieuwe technologie
aandrijft -- moeten extra waakzaam zijn voor freelance, neo-luddietisch
terrorisme.

Een krankzinnige zoals de Unabomber zal helaas waarschijnlijk brigades
van na-ijmpers aanwakkeren, nu de frustratie over dalende inkomens en de
toegenomen wrok tegen prestaties de kop opsteken. Wij vermoeden dat een
groot deel van het toekomstige geweld uit bomaanvallen zal bestaan.
Volgens de \emph{New York Times} is het binnenlands terrorisme in de
Verenigde Staten in de jaren negentig fors toegenomen. `Ze zijn de
afgelopen vijf jaar met meer dan vijftig procent gegroeid en in het
afgelopen decennium bijna verdrievoudigd. Het aantal criminele explosies
en pogingen steeg van 1.103 in 1985 tot 3.163 in 1994\ldots. In kleine
steden en voorstedelijke buurten, evenals onder stedelijke straatbendes,
zien we een toename van wat op een alledaagse bommenwerper lijkt.'
\footnote{Timothy Egan, `Terrorisme wordt nu huisgemaakt naarmate
  bomaanslagen in de V.S. toenemen,' \emph{New York Times}, 25 augustus
  1996, p.~1.}

\subsection{Defensie wordt een privaat
goed}\label{defensie-wordt-een-privaat-goed}

Hoe hoog de belastingen voor bescherming ook zijn, natiestaten zullen in
de komende jaren waarschijnlijk niet in staat zijn om deze op effectieve
wijze te leveren. De afnemende schaal van geweld -- mede beïnvloed door
de nieuwe informatietechnologie -- maakt het opbouwen van een massief
militair apparaat steeds minder zinvol. Dit betekent niet alleen dat de
oorlogvoering minder beslissend wordt en staten hun burgers minder goed
kunnen beschermen, maar ook dat de schijnbare extraterritoriale
hegemonie van de Verenigde Staten als 's werelds supermacht in de
volgende eeuw waarschijnlijk minder effectief zal blijken dan de
hegemonie van Groot-Brittannië in de negentiende eeuw. Tot het uitbreken
van de Eerste Wereldoorlog oefende het centrum tegen relatief lage
kosten effectief en doorslaggevend zijn macht uit op de periferie. In de
eenentwintigste eeuw nemen de bedreigingen van grootmachten voor het
leven en eigendom onvermijdelijk af, nu geweld steeds vaker in kleinere,
minder grootschalige vormen voorkomt. De dalende opbrengsten van geweld
wijzen erop dat natiestaten of rijken die grootschalig militaire macht
inzetten, mogelijk niet zullen overleven of zelfs niet zullen ontstaan
in het informatietijdperk.

Nu de overheidsuitgaven voor een adequate verdediging dalen, wordt het
steeds aannemelijker om beschermingsdiensten als private goederen te
beschouwen. Beveiligingsbedreigingen op kleinere schaal bestrijd je
immers steeds beter met commerciële beveiligingstroepen -- denk aan
muren, hekken en beveiligingsperimeters om onruststokers buiten de boot
te houden. Bovendien kan een welvarend individu of een onderneming zich
doorgaans zelf de nodige bescherming veroorloven tegen de meeste
bedreigingen die in het informatietijdperk kunnen ontstaan. Daarnaast
vergroot de afnemende schaal van militaire bedreigingen het risico op
anarchie of op interne gewelddadige conflicten binnen een enkel
grondgebied. Tegelijkertijd neemt de concurrentie tussen rechtsgebieden
op het gebied van bescherming onder marktconforme voorwaarden toe. Dit
houdt in dat rechtsgebieden steeds intensiever zullen concurreren op het
terrein van beschermingsdiensten, paspoort- en consulaire diensten en de
uitvoering van justitie.

Op de lange termijn kunnen soevereine individuen waarschijnlijk reizen
met niet-gouvernementele documenten, uitgegeven door private
agentschappen en affiniteitsgroepen in de vorm van kredietbrieven. Het
is niet vergezocht te denken dat zich een groep zal vormen, een soort
koopmansrepubliek in cyberspace à la middeleeuwse Hanze, die het sluiten
van private verdragen en contracten tussen rechtsgebieden
vergemakkelijkt en tegelijkertijd haar leden bescherming biedt. Stel je
een speciaal paspoort voor, uitgegeven door de Liga van Soevereine
Individuen, dat de drager erkent als iemand die onder de bescherming van
de liga valt.

Als zo'n document ooit tot stand komt, blijft het slechts een tijdelijk
aandenken aan de overgang van de natiestaat en het bureaucratische
tijdperk dat deze ooit in stand hield.

In vroegere tijden had men doorgaans geen paspoort nodig om de vaak vaag
afgebakende grenzen over te steken. In middeleeuwse grensgebieden
hanteerde men soms veiligheidsbrieven, maar de lokale autoriteiten gaven
deze immers af voor het gebied dat bezocht moest worden, in plaats van
vanuit de jurisdictie waar de reiziger vandaan kwam. Veel belangrijker
dan een paspoort waren introductie- en kredietbrieven, die een reiziger
hielpen onderdak te vinden en zakelijke deals te sluiten. Die tijd komt
weer, want binnenkort zullen prominente personen zonder enige documenten
reizen. Ze identificeren zich dan via waterdichte biometrische methoden,
zoals spraakherkenningssystemen of netvliesscanners die hen uniek
herkennen.

Wij verwachten dat de wereld ergens in de eerste helft van de volgende
eeuw daadwerkelijk de privatisering van soevereiniteit zal meemaken. Dit
zal ervoor zorgen dat het bereik van dwang tot het logische minimum
krimpt. Toch zullen de seculiere inquisiteurs en reactionairen van het
volgende millennium in verontwaardiging en met een gevoel van bedreiging
reageren wanneer de eens `heilige' kenmerken van nationaliteit
verhandelbaar worden en als een kwestie van kosten-batenberekening
kunnen worden gekocht en verkocht.

In dit boek stellen wij dat een natiestaat niet langer nodig is om een
informatieoorlog te voeren. Computerprogrammeurs kunnen zulke oorlogen
uitvechten door talloze `bots' of digitale bedienden in te zetten. Bill
Gates beschikt inmiddels over een veel grotere capaciteit dan de meeste
natiestaten om logische bommen in kwetsbare systemen wereldwijd te laten
ontploffen. In het tijdperk van de informatieoorlog vormt iedere
softwaremaatschappij -- of zelfs de \emph{Scientology}kerk -- een
geduchtere tegenstander dan de gecombineerde dreiging van de meeste
staten met een zetel in de Verenigde Naties.

Het machtverlies van natiestaten is een logische consequentie van de
opkomst van goedkope, geavanceerde rekenkracht.
Microprocessortechnologie vermindert niet alleen de baten van geweld,
maar zorgt er ook voor dat er voor het eerst een concurrerende markt
ontstaat voor beschermingsdiensten, waarvoor overheden in het
industriële tijdperk monopolistische prijzen rekenden.

In de nieuwe wereld waarin soevereiniteit commercieel wordt
georganiseerd, kiezen mensen hun rechtsgebieden, net zoals velen
tegenwoordig hun verzekeringsmaatschappijen of religies
selecteren.\footnote{Zie Stephen J. Duhner, \emph{Choosing My Religion},
  \emph{New York Times Magazine}, 31 maart 1996, p.~36 ff.}
Rechtsgebieden die er niet in slagen een passende mix van diensten te
leveren -- welke diensten dat ook mogen zijn -- zullen failliet gaan en
geliquideerd worden, zoals incompetente bedrijven of mislukte religieuze
gemeenschappen dat ook doen. Concurrentie zet lokale rechtsgebieden
ertoe aan hun vermogen om diensten efficiënt en economisch te leveren te
verbeteren. Op die manier levert de concurrentie tussen rechtsgebieden
bij het aanreiken van publieke goederen een effect op dat vergelijkbaar
is met wat we in andere levenssferen zien, want concurrentie verhoogt
doorgaans de klanttevredenheid.

\section{Concurrentie en anarchie}\label{concurrentie-en-anarchie}

Belangrijk is te beseffen dat de rivaliteit tussen de jurisdicties die
wij voor ogen hebben niet primair draait om organisaties die binnen
hetzelfde grondgebied met geweld opereren. Zoals eerder is opgemerkt,
hebben gewelddragende organisaties vaak de neiging om geweld verder de
dagelijkse sfeer in te laten sijpelen, waardoor de economische kansen
afnemen. Zoals Lane het verwoordde,

\begin{quote}
Bij het gebruik van geweld waren er uiteraard grote schaalvoordelen bij
het concurreren met rivaliserende gewelddragende ondernemingen of bij
het vestigen van een territoriaal monopolie. Dit gegeven vormt de basis
voor de economische analyse van een aspect van de overheid: de
gewelddragende, geweldbeheersende industrie was een natuurlijk
monopolie, althans op land. Binnen territoriale grenzen kon de geleverde
dienst veel goedkoper door een monopolie worden geproduceerd. Natuurlijk
zijn er momenten geweest waarop gewelddragende ondernemingen wedijverden
om betalingen voor bescherming af te dwingen in vrijwel hetzelfde
grondgebied, bijvoorbeeld tijdens de Dertigjarige Oorlog in Duitsland.
Maar zo'n situatie was nog oneconomischer dan concurrentie in dezelfde
gebieden tussen rivaliserende telefoonsystemen.\footnote{Lane,
  `Economische gevolgen van georganiseerd geweld', p.~402.}
\end{quote}

Lanes opmerking levert op twee vlakken waardevolle inzichten. Ten eerste
onderschrijven we zijn algemene conclusie dat soevereiniteiten de
neiging hebben territoriale monopolieën te vestigen, zodat zij
goedkopere en effectievere beschermingsdiensten kunnen leveren. Ten
tweede valt het op dat hij een inmiddels verouderde vergelijking maakt
met het telefoonmonopolie, terwijl we inmiddels weten dat
telefoonsystemen niet per se een monopolie hoeven te vormen. Deze
constatering vraagt om enige voorzichtigheid in de analyse.
Technologische veranderingen kunnen de algemene conclusie ondermijnen
dat anarchie binnen territoriale grenzen onhoudbaar is. Stel
bijvoorbeeld dat cyberactiva massaal groeien in een domein dat buiten
het bereik van dwang valt; dan zal de prijsbepaling van
beschermingsdiensten minder door de `vraag' worden bepaald en meer door
onderhandelingsprocessen op de markt.

We hebben het hier niet over alomtegenwoordige anarchie, maar over de
concurrentie tussen jurisdicties die elk een monopolie op geweld binnen
hun eigen grondgebied hebben. Zulke jurisdicties wedijveren er immers
hard voor hun `klanten' de meest waardevolle en kosteneffectieve
beschermingsdiensten te leveren. Toegegeven, in het informatietijdperk
zullen er ongetwijfeld meer onzekerheden ontstaan rondom de levering van
deze diensten en zal de particuliere voorziening van politie- en
defensiediensten uitgebreider worden dan we vroeger gewend waren. Maar
de concurrentie waar wij over spreken, verschilt wezenlijk van een
strijd waarbij meerdere beschermingsinstanties op hetzelfde grondgebied
vechten om verschillende klanten te bedienen -- want dat zou pure
anarchie betekenen.

Hoe het ook zij, de groei van soevereiniteiten -- waarbij individuen
steeds vaker als soevereinen optreden zodra zij over voldoende middelen
beschikken -- zorgt er onvermijdelijk voor dat de ruimte voor anarchie
zal toenemen. De relaties tussen soevereiniteiten verlopen namelijk
altijd anarchaïsch. Er bestaat, en heeft nooit bestaan, een
wereldregering die het gedrag van alle individuele soevereiniteiten
reguleert, of het nu gaat om ministaten, natiestaten of rijken. Zoals
Jack Hirshleifer opmerkt: ``{[}W{]}aarin samenlevingen, variërend van
primitieve stammen tot moderne natiestaten, intern door een vorm van wet
worden bestuurd, blijven hun externe betrekkingen met elkaar
voornamelijk anarchaïsch.''\footnote{Jack Hirshleifer, `Anarchie en de
  ineenstorting', in Michelle R. Garfinkel en Stergios Skaperdas, red.,
  \emph{De politieke economie van conflict en toe-eigening} (Cambridge:
  \emph{Cambridge University Press}, 1996), p.~15.} Met meer soevereine
entiteiten ontstaan er daarom ook onvermijdelijk relaties die zich over
meerdere rechtsgebieden uitstrekken en dus anarchaïsch zijn.

Belangrijk is op te merken dat anarchie -- met andere woorden, het
ontbreken van een overkoepelende macht om geschillen te beslechten --
niet gelijkstaat aan totale chaos of aan het gebrek aan structuur.
Hirshleifer stelt dat zelfs interstam- of internationale systemen
regelmatigheden en systematisch analyseerbare patronen
kennen.\footnote{Ibid., p.~15.} Met andere woorden: net zoals `chaos' in
de wiskunde kan wijzen op een complexe en zeer geordende structuur, is
ook anarchie niet per definitie vormeloos of wanordelijk.

Hirshleifer bestudeert diverse anarchaïsche situaties. Hij bespreekt
daarbij niet alleen de relaties tussen soevereiniteiten, maar ook
bendeoorlogvoering in het Chicago van de drooglegging en de
confrontaties tussen mijnwerkers en claimjumpers tijdens de
Californische goudkoorts. Let op: hoewel Californië in 1849 al deel
uitmaakte van de Verenigde Staten, werden de omstandigheden in de
goudvelden terecht als anarchistisch beschreven. Zoals Hirshleifer
opmerkt: ``{[}D{]}e officiële organen van de wet bleken
machteloos.''\footnote{Ibid., p.~34.} Hij betoogt dat de barre
omstandigheden in de bergkampen, gecombineerd met de effectieve
organisatie van mijnwerkers die zichzelf als rechtshandhavers hadden
aangesteld tegen claimjumpers, het voor bendes van buitenstaanders
vrijwel onmogelijk maakten om de goudmijnen te veroveren -- ondanks het
gebrek aan een degelijke wetshandhaving. Met andere woorden, onder
bepaalde omstandigheden kan waardevol eigendom zelfs in een
anarchistische situatie goed beschermd worden.

De vraag rijst of Hirshleifer's theoretische analyse van de dynamiek
binnen de spontane orde van de darwinistische `natuurlijke economie' ook
relevant is voor de economie in het informatietijdperk. Wij denken van
wel. Hoewel we geen alomtegenwoordige anarchie -- of goudveldachtige
omstandigheden -- overal verwachten, zien we zeker een toename van
anarchaïsche relaties in het wereldsysteem. In dat licht is
Hirshleifer's betoog over de condities waaronder `twee of meer
anarchaïsche deelnemers' in evenwicht hun leefbare aandeel van de
maatschappelijk beschikbare middelen weten te behouden bijzonder
suggestief.\footnote{Ibid., p.~17.} Vooral onderzoekt hij wanneer
anarchie overgaat in tirannie of dominantiehiërarchieën, wat gebeurt
wanneer anarchaïsche partijen ondergeschikt raken aan een overweldigende
autoriteit.

Deze kwesties blijken in het informatietijdperk vaak nog cruciaalere
kennisgebieden dan in het industriële tijdperk. Een belangrijke reden
dat de fijne nuances in de dynamiek van anarchie in voorgaande eeuwen
minder doorslaggevend leken, is dat het rendement op geweld in de
moderne tijd is toegenomen. Daardoor leidde het samenbrengen van steeds
grotere militaire machten -- zoals natiestaten dat in recente eeuwen
deden -- tot beslissende oorlogvoering. Bijna per definitie zorgen zulke
oorlogsvoering ervoor dat strijdende partijen in hun strijd om
hulpbronnen onder de heerschappij van een sterkere autoriteit komen te
staan, wat de anarchie tempert. Tegelijkertijd bevordert een afnemende
beslissendheid in gevechten -- mede dankzij de technologische
superioriteit van de verdediging -- de dynamisch stabiele positie van
anarchie. Hierom zorgt informatietechnologie, die de beslissendheid van
militaire acties verlaagt, ervoor dat de anarchie tussen ministaten
stabieler blijft en dat een grote overheid deze situatie minder
gemakkelijk kan overnemen. Bovendien betekent een afname van de
beslissendheid ook minder gevechten, wat een hoopvolle uitkomst biedt
voor de wereld van vandaag.

\subsection{Levensvatbaarheid}\label{levensvatbaarheid}

Een andere belangrijke voorwaarde voor het voortbestaan van anarchie is
levensvatbaarheid, oftewel de mate waarin mensen beschikken over een
voldoende inkomen. Degenen die niet genoeg verdienen om in hun
levensonderhoud te voorzien, zullen waarschijnlijk óf (1) enorme
inspanningen leveren om te strijden en zo de benodigde middelen voor hun
overleving te bemachtigen, óf (2) zich onderwerpen aan een andere
uitdager in ruil voor voedsel en levensonderhoud. Een dergelijk fenomeen
deed zich voor tijdens de opkomst van het feodalisme rond het jaar 1000.
We verwachten dat steeds meer mensen met een laag inkomen in westerse
landen -- die voorheen afhankelijk waren van overheidsuitkeringen --
zich als hofdienaren bij welvarende huishoudens zullen voegen. Toch
geeft het feit dat sommige uitdager in een anarchistische samenleving
een laag inkomen hebben op zichzelf nog geen duidelijk signaal over hoe
de situatie zich verder zal ontwikkelen. Zoals Hirshleifer stelt:
`{[}H{]}et loutere feit van een laag inkomen onder anarchie \ldots{}
geeft op zichzelf geen duidelijk signaal als toespraak op wat er hierna
zal gebeuren.'\footnote{Ibid., p.~37.}

\subsection{De aard van activa}\label{de-aard-van-activa}

Een andere interessante voorwaarde voor de duurzaamheid van anarchie is
dat de middelen voorspelbaar en verdedigbaar zijn. In zijn analyse merkt
Hirshleifer op: `{[}A{]}narchie is een sociale ordening waarin uitdager
strijden om duurzame middelen te veroveren en te verdedigen.'\footnote{Ibid.,
  p.~16.} Hij definieert `duurzame middelen' als landgebieden of
verplaatsbare kapitaalgoederen.\footnote{Ibid.} In het
informatietijdperk kunnen digitale middelen voorspelbaar blijken, maar
ze vallen niet in de categorie `duurzame middelen' die Hirshleifer
koppelt aan territorialiteit en anarchie. Immers, wanneer digitaal geld
vrijwel onmiddellijk over de hele wereld wordt verstuurd, is het
veroveren van het grondgebied waar een cyberbank opereert zinloos. Als
natiestaten soevereine individuen willen onderdrukken, moeten zij
tegelijkertijd zowel de bankparadijzen als de dataparadijzen van de
wereld veroveren. Zelfs dan zijn zij -- mits de versleutelde systemen
naar behoren zijn ontworpen -- slechts in staat om een beperkte
hoeveelheid digitaal geld te saboteren of te vernietigen; het
daadwerkelijk in bezit krijgen blijft buiten hun bereik.

De conclusie komt erop neer dat in het komende informatietijdperk de
meest voorspelbare en kwetsbare bezittingen van de rijken wellicht hun
eigen lichamen -- met andere woorden, hun levens -- zijn. Daarom
verwachten we de komende decennia terreur in luddietenstijl, waarbij een
deel mogelijk heimelijk door agents provocateurs in dienst van
natiestaten wordt aangewakkerd.

Op lange termijn betwijfelen we of toonaangevende natiestaten soevereine
individuen daadwerkelijk kunnen onderdrukken. Vooral staten in
kapitaalarme regio's zullen inzien dat zij meer baat hebben bij het
huisvesten van soevereine individuen dan door vast te houden aan
solidariteit met Noord-Atlantische natiestaten en het in ere houden van
het `internationale' systeem. Dat failliete, zwaar belaste
welvaartsstaten erop aandringen om `hun burgers' en `hun kapitaal'
binnen eigen landsgrenzen te houden, biedt elders namelijk geen
overtuigende stimulans voor de honderden gefragmenteerde
soevereiniteiten.

Dit stellen we, ook al bestaan er duizenden multinationale organisaties
die proberen het gedrag van diverse soevereiniteiten wereldwijd te
beïnvloeden. We twijfelen er nauwelijks aan dat sommige van deze
organisaties, zoals de Europese Unie en de Wereldbank, grote invloed
uitoefenen. Maar bedenk dat rechtsgebieden die soevereine individuen
verwelkomen, aanzienlijk profiteren van hun aanwezigheid. Zelfs de
koppige Verenigde Staten, die volgens de huidige trends gedwongen worden
hard op te treden om te voorkomen dat een cybereconomie buiten hun
controle ontstaat, zullen uiteindelijk niet toestaan dat mensen met
positieve banksaldi -- mensen die bewust geen Amerikanen willen zijn --
wereldwijd buitengesloten raken. Dit wordt des te waarschijnlijker nu
winkelen tegenwoordig een grote aantrekkingskracht op reizigers
uitoefent. Uiteindelijk nemen ook de Verenigde Staten -- zij het later
dan anderen -- of delen daarvan deel aan de commercialisering van
soevereiniteit als gevolg van toenemende concurrentiedruk.

\subsection{Vraag creëert aanbod}\label{vraag-creuxebert-aanbod}

Die druk zal in een vroeg stadium vooral voelbaar zijn in natiestaten
met zwakke balansen. Binnen de nieuwe `offshore'-centra zullen
fragmenten en enclaves van de huidige natiestaten opduiken --
bijvoorbeeld Canada en Italië -- die vrijwel zeker al ruim voor het
einde van het eerste kwartaal van de eenentwintigste eeuw uiteen vallen.
Het ontstaan van een mondiale markt voor hoogwaardige en
kostenefficiënte rechtsgebieden speelt daarbij een belangrijke rol. Net
als in de reguliere handel kunnen kleinschalige concurrenten leniger
opereren en daardoor beter concurreren. Een rechtsgebied met weinig
inwoners organiseert zich immers eenvoudiger zodat het efficiënt
functioneert.

De informatie-elite zoekt op contractbasis hoogwaardige bescherming
tegen een redelijke vergoeding. Hoewel die vergoeding ruim onder het
niveau blijft dat nodig is om een merkbaar voordeel over de gehele
bevolking van natiestaten -- zoals zij nu functioneren, met tientallen
tot honderden miljoenen burgers -- te verspreiden, is dit in
rechtsgebieden met tienduizenden of honderdduizenden inwoners zeker niet
onbeduidend. De belastinginkomsten en andere economische voordelen, die
voortvloeien uit de aanwezigheid van een klein aantal buitengewoon rijke
individuen, leveren per hoofd van de bevolking in rechtsgebieden met een
kleine populatie een veel groter voordeel op.

Aangezien het in de praktijk vrijwel onbelangrijk wordt waar men zijn
onderneming vestigt -- afgezien van het negatieve feit dat bepaalde
adressen hogere verplichtingen met zich meebrengen -- kunnen kleinere
rechtsgebieden gemakkelijker commercieel aantrekkelijke
beschermingsvoorwaarden vaststellen. Daarom genieten rechtsgebieden met
een kleine bevolking een duidelijk voordeel bij het opstellen van een
fiscaal beleid dat soevereine individuen aanspreekt.

Wij zijn ervan overtuigd dat het tijdperk van de natiestaat voorbij is,
maar dat betekent niet dat de aantrekkingskracht van nationalisme, als
beroep op menselijke emoties, meteen zal verdwijnen. Nationalisme speelt
als ideologie uitstekend in op universele emotionele behoeften. We
hebben allemaal wel eens ontzag ervaren, bijvoorbeeld bij het voor het
eerst aanschouwen van een enorme waterval of bij het binnenstappen in
een indrukwekkende kathedraal. Eveneens hebben we het gevoel van
saamhorigheid ervaren, zoals tijdens een familiekerstfeest of als lid
van een succesvol sportteam. Onze cultuur roept reacties op beide
krachtige emoties op. We laten ons inspireren door de historische
tradities van ons eigen land, dat weer deel uitmaakt van de bredere
menselijke cultuur. We vinden troost in het besef dat we tot een
culturele gemeenschap behoren, die ons zowel een gevoel van
verbondenheid als van identiteit geeft.

De invloed van deze culturele symbolen kan het meest ingrijpende
emotionele effect hebben. Amerikaanse associaties met de vlag, het
volkslied of het familiediner op Thanksgiving, en Engelse associaties
met de monarchie of met cricket, grijpen respectievelijk Amerikanen en
Engelsen stevig in de verbeelding -- een greep die door herhaling steeds
versterkt wordt en diep in het onderbewustzijn doordringt. Zulke
symbolen helpen ons te begrijpen wat voor soort mensen we zijn en doen
ons herinneren aan een nationale cultuur. Toen
anti-Vietnamoorlogdemonstranten de rest van de Verenigde Staten wilden
choqueren, verbrandden zij de vlag. Vervreemden Engelsen vallen de
monarchie aan en er gaan zelfs verhalen rond dat zij gaten in
cricketbanen graven.

Deze prikkelende symbolen lijken misschien oppervlakkig, maar ze zijn
zeker niet onbelangrijk. Het zijn associaties waardoor ons bloed sneller
gaat stromen. Ongeacht de veranderingen in megapolitieke omstandigheden
of de daaruit voortvloeiende aanpassingen in instituties, zullen zij
waarschijnlijk een blijvende rol spelen in de verbeelding van mensen die
-- net als wij -- in de twintigste eeuw volwassen werden.

De inhoud van het te vertalen boek is:

\bookmarksetup{startatroot}

\chapter{De schemering van de
democratie}\label{de-schemering-van-de-democratie}

\begin{quote}
`Democratische politieke systemen vormen in historische termen een
recent fenomeen. Ze bestonden kort in Griekenland en Rome en keken
minder dan 200 jaar geleden in de 18e eeuw weer terug. Een cyclus van
afwijzing lijkt nu opnieuw aan te breken.'\footnote{John Dunn,
  \emph{Western Political Theory in the Face of the Future} (Cambridge,
  Eng.: Cambridge University Press, 1979, p.~2).} - William Pfaff
\end{quote}

Het valt niet te ontkennen dat democratie als bestuursvorm in de loop
der tijd vrij zeldzaam en van korte duur is geweest. In de perioden --
zowel in de oudheid als in modernere tijden -- waarin democratie
dominant was, bepaalde het succes ervan in sterke mate de aanwezigheid
van grootschalige omstandigheden die zowel de militaire macht als het
draagvlak van de massa versterkten. Historicus Carroll Quigley
onderzocht deze bepalende kenmerken in \emph{Weapons Systems and
Political Stability}.\footnote{Carroll Quigley, \emph{Weapons Systems
  and Political Stability} (Washington, D.C.: University Press of
  America, 1983).}

Hiervoor behoren:

\begin{enumerate}
\def\labelenumi{\arabic{enumi}.}
\item
  \textbf{Betaalbare en breed beschikbare wapens}. Democratieën bloeien
  doorgaans wanneer het aanschaffen van doeltreffende wapens weinig
  kost.
\item
  \textbf{Wapens die effectief door amateurs ingezet kunnen worden}. De
  kans op democratie neemt toe wanneer iedereen zonder langdurige
  training effectieve wapens kan hanteren.
\item
  \textbf{Een militair voordeel voor een groot aantal infanteristen in
  de strijd}. Zoals Quigley opmerkt: `{[}P{]}erioden waarin de
  infanterie de overhand had, gingen gepaard met een bredere spreiding
  van politieke macht binnen de gemeenschap, waardoor de democratie
  betere kansen had om te zegevieren.'\footnote{Ibid., p.~56.}
\end{enumerate}

Dit vormt geenszins een volledige opsomming van alle voorwaarden
waaronder democratie kan bestaan. Als dat wel zo was, had de democratie
aan het einde van de twintigste eeuw niet als een triomfantelijk systeem
doorgebroken. In de schemering van het industriële tijdperk waren wapens
vermoedelijk duurder dan ooit, en veel van de meest krachtige wapens
vereisten ongetwijfeld specialisten om hun potentie daadwerkelijk te
benutten. Bovendien toont de Golfoorlog tussen de Verenigde Staten, hun
bondgenoten en Irak aan hoe kwetsbaar grote contingenten infanterie
zijn, zelfs wanneer zij zich in loopgraven en diep ingegraven
verdedigingswerken bevinden. Dus waarom lijkt de democratie juist te
floreren in deze omstandigheden, nu de twintigste eeuw ten einde loopt?

\section{Democratie, de broederlijke tweeling van het
communisme?}\label{democratie-de-broederlijke-tweeling-van-het-communisme}

In hoofdstuk 5 gaven we een paradoxale verklaring: de democratie bloeide
als broederlijke tweeling van het communisme juist omdat de staat
onbelemmerd de controle over de middelen kon uitoefenen. Deze conclusie
lijkt wellicht absurd volgens het `gezond verstand' van het industriële
tijdperk. We ontkennen niet dat in die samenleving democratische
systemen en het communisme bittere tegenpolen vormden. Maar als je beide
systemen vanuit een megapolitiek perspectief bekijkt -- zoals het
informatietijdperk dat waarschijnlijker benadert -- hebben ze meer
gemeen dan je zou vermoeden.

In tijden waarin wapens grotesk duur waren, fungeerde de democratie als
het besluitvormingsmechanisme dat de staatscontrole over de middelen tot
het uiterste dreef. Net als staatssocialistische systemen investeerden
democratische regeringen enorme bedragen om een massale militaire opbouw
te financieren. Het verschil was dat de democratische verzorgingsstaat
de staat in staat stelde om nog meer middelen in handen te krijgen dan
de staatssocialistische systemen -- die immers praktisch elk waardevol
bezit opeisten.

Als zuiver instrument om financiering bijeen te drijven, bleek de
democratische staat objectief superieur te zijn aan het staatssocialisme
in haar vermogen de staatskas te verrijken. Zoals we eerder toelichtten,
kon de democratie haar militairen aanzienlijk meer geld verschaffen
dankzij de aansluiting op particulier eigendom en de productiviteit van
het kapitalisme.

Het staatssocialistische systeem baseerde zich op de doctrine dat de
staat alles bezat. Daarentegen beperkte de democratische
verzorgingsstaat haar initiële aanspraken en deed zij alsof particulier
eigendom was toegestaan -- zij het in een voorwaardelijke vorm --
waardoor zij betere prikkels bood om de productie te stimuleren. In
plaats van van meet af aan alles verkeerd te regelen, gaven westerse
regeringen burgers de kans om eigendom te vergaren en vermogen op te
bouwen. Pas nadat dat vermogen was ontstaan, grepen de democratische
natiestaten in en belastten zij er een Large fractie mee.

Het woord `Large' schrijven we met een hoofdletter. In 1996 bedroeg het
levenslange federale belastingtarief in de Verenigde Staten bijvoorbeeld
73 cent per dollar. Voor ondernemers die hun inkomen via dividenden
ontvingen, lag het tarief op 83 cent per dollar, terwijl het voor mensen
die geld wilden nalaten of aan kleinkinderen wilden schenken, 93 cent
per dollar bedroeg. Wanneer men ook staats- en lokale belastingen
meerekent, confisqueert de democratische overheid op alle niveaus het
merendeel van elke verdiende dollar. Deze roofdierachtige
belastingtarieven maakten de democratische staat in feite tot een
partner die tussen drie-kwarten en negen-tenths van alle inkomsten
opeiste -- al is dat niet hetzelfde als staatssocialisme, het vertoont
er wel duidelijke gelijkenissen.

De democratische staat overleefde langer omdat zij flexibeler was en
enorme middelen bijeenbracht, vergeleken met wat beschikbaar was in
Moskou of Oost-Berlijn.

Inefficiëntie, waar het telde

Wij hebben de megapolitieke troef van democratie als
besluitvormingsmethode binnen een krachtige regering verwoord met de
slagzin `inefficiëntie, waar het telde.' Vergeleken met het communisme
werkte de verzorgingsstaat immers efficiënter. Maar wanneer je de
verzorgingsstaat afzet tegen een zuivere laissez‑faire-enclave als
Hongkong, blijkt zij in alle eenvoud inefficiënt te zijn. De
groeicijfers in Hongkong waren ronduit indrukwekkend, vooral omdat de
inwoners -- en niet de overheid -- in staat waren om 85 procent van de
vruchten van de razendsnelle groei voor zichzelf te houden.

Hongkong is natuurlijk geen democratie. Integendeel, het
vertegenwoordigt juist een mentaal model van het type rechtsgebied dat
in het informatietijdperk zal floreren. In het industriële tijdperk had
Hongkong immers geen behoefte aan democratie, omdat men werd behoed voor
de lastige verplichting om middelen te verzamelen voor de ondersteuning
van een formidabele militaire macht. Aangezien Hongkong van buitenaf
werd verdedigd, kon men zich een werkelijk vrije economie permitteren.

Juist dat vermogen om eigen middelen te vergaren plaatste de democratie
in de megapolitieke context van het industriële tijdperk boven andere
systemen. Massademocratie en industrialisme gingen als een
onafscheidelijk duo hand in hand. Zoals \emph{Alvin Toffler} opmerkte,
is massademocratie `de politieke uitdrukking van massaproductie,
massadistributie, massaconsumptie, massaal onderwijs, massamedia,
massavermaak en alles daartussenin.'\footnote{Geciteerd in Kelly,
  \emph{op. cit.}, p.~46.}

Nu informatietechnologie de massaproductie grotendeels vervangt, mogen
we verwachten dat het tijdperk van massademocratie ten einde komt. Het
fundamentele megapolitieke imperatief dat massademocratie in het
industriële tijdperk deed zegevieren, is verdwenen. Het is daarom
slechts een kwestie van tijd voordat massademocratie hetzelfde lot
ondergaat als haar broederlijke tweeling, het communisme.

Massademocratie onverenigbaar met het informatietijdperk

Bij nader inzien blijkt dat de technologie van het informatietijdperk op
zichzelf geen massatechnologie vormt. In militaire context -- zoals al
eerder aangegeven -- maakt deze technologie de ontwikkeling van `slimme
wapens' en een `informatieoorlog' mogelijk, waarbij zogenaamde `logische
bommen' centraal aangestuurde commando- en controlesystemen kunnen
uitschakelen. Informatietechnologie illustreert niet alleen de
verfijning van wapens die door specialisten worden bediend, maar
vermindert tevens de impact van offensieve operaties, waardoor de
positie van de verdediging relatief versterkt wordt. Dankzij
microtechnologie behaalt een individu een aanzienlijk grotere militaire
macht, waardoor het strategisch belang van massale infanterieformaties
afneemt. Volgens \emph{`Rand Corporation'} rapporteerde aan de minister
van Defensie: `Onderling verbonden netwerken kunnen worden aangevallen
en verstoord, niet alleen door staten, maar ook door niet-statelijke
actoren, waaronder verspreide groepen en zelfs individuen.'\footnote{Molander
  et al., \emph{Strategic Information Warfare}, op. cit., p.~xv.}
Bovendien suggereert dit dat een cyberoorlog de inherente nadelen van
sterk gecentraliseerde systemen teniet zal doen.

Volgens \emph{`Rand-experts'} maken informatiegebaseerde technieken
geografische afstanden onbelangrijk; doelwitten binnen de continentale
Verenigde Staten lopen net zo'n risico als agenten in operationele
gebieden.\footnote{Ibid., p.~xiv.} Waar men vroeger aannam dat het
veilig was om binnen de grenzen van grote supermachten als de Verenigde
Staten te verblijven, keert in het informatietijdperk de logica van
machtsconcentratie om. Zelfs een plaats als Peoria, hoewel ver van een
mogelijk militair front, biedt geen bescherming meer tegen
cyberaanvallen van bijna elke tegenstander. Wonen binnen de grenzen van
een supermacht maakt je immers tot een direct doelwit. In plaats van te
centraliseren, kunnen lokale gemeenschappen hun veiligheid beter
waarborgen door te decentraliseren. Door de opkomst van cyberoorlog
groeit de kwetsbaarheid van centraal georganiseerde commando- en
controlesystemen, terwijl verspreide systemen steeds concurrerender
worden.

De op gang gebrachte feedbackmechanismen kunnen het
decentralisatieproces verder versnellen. Volgens \emph{`Rand-experts'}
zal de overheid, om de kwetsbaarheid van de in de latere fases van de
natiestaat ontwikkelde commandostructuren voor cyberaanvallen te
verkleinen, gedwongen worden de inzet van nieuwe
softwareversleutelingstechnieken te intensiveren. Daarmee worden deze
systemen, die grotendeels tot de privésector behoren, veel minder
vatbaar voor sabotage. Tegelijkertijd versnelt dit de commerciële
verspreiding van sterke encryptie, waardoor zij zich beter kunnen
losmaken van staatscontrole. Ook dit stimuleert de decentralisatie en
bevordert de spreiding van middelen in cyberspace, zodat zij buiten het
bereik van de politiek blijven.

Op de lange termijn leidt dit tot het einde van massademocratie, met
name in haar overheersende vorm van representatief wanbestuur -- zowel
in een congres- als in een parlementair stelsel.

\section{De megapolitiek van
misvertegenwoordiging}\label{de-megapolitiek-van-misvertegenwoordiging}

Wanneer de megapolitieke omstandigheden drastisch veranderen -- zoals we
nu duidelijk zien -- past de organisatie van de overheid zich
onvermijdelijk aan. Sterker nog, de representatieve overheid
weerspiegelt al altijd de verdeling van ruwe macht, want men kiest de
vertegenwoordigers op basis van geografie in plaats van volgens andere
criteria.

Stel je voor: een wetgevend orgaan zou net zo democratisch zijn als
wanneer de leden willekeurig uit de bevolking werden gekozen.
Parlementswijken of kiesdistricten zouden bijvoorbeeld gebaseerd kunnen
worden op geboortedata of zelfs op een alfabetische indeling. Iedereen
die op 1 januari geboren is, zou via één kandidatenlijst stemmen,
terwijl degenen die op 2 januari geboren zijn, via een andere lijst
zouden kiezen. Of iedere persoon wiens naam begint met `Aa' tot `Af' zou
uit één kandidatenlijst kiezen, en wie met `Ag' begint, zou via een
andere lijst gaan stemmen. En zo verder.

Zo'n systeem bestaat tegenwoordig om verschillende redenen niet. Ten
eerste was dit in de achttiende eeuw technologisch simpelweg niet
uitvoerbaar. Nog belangrijker is dat kiesdistricten gebaseerd op
geboortedata of alphabetische indelingen de werkelijke verdeling van
ruwe macht -- zoals die tijdens verkiezingen tot uiting komt -- niet
hadden kunnen weerspiegelen of benaderen. Mensen die slechts dezelfde
geboortedatum of de eerste paar letters van hun naam delen, vormen
immers nauwelijks een samenhangende machtsbasis.

\subsection{Waarom wegen geografische indelingen
zwaarder?}\label{waarom-wegen-geografische-indelingen-zwaarder}

Stemming ontstond oorspronkelijk als een alternatief voor militaire
strijd -- en dat is nog steeds het geval, zij het op een minder directe
wijze. Dergelijke conflicten worden doorgaans langs geografische lijnen
georganiseerd, en minder vaak op basis van verwantschap of religie. Je
kunt zulke conflicten immers niet inrichten op basis van verjaardagen of
de eerste letters van namen. Ook kan men conflicten op basis van
beroepen niet effectief organiseren, tenzij die beroepsgroepen beperkt
blijven tot erfelijke gilden -- zoals de kasten in India -- of wanneer
ze zich lokaal concentreren, bijvoorbeeld onder boeren in Iowa.

Het uitgangspunt van de huidige representatieprincipes is dat zij
belangen vertegenwoordigen die geografisch verankerd zijn, in plaats van
langs een andere dimensie. Historisch hing militair succes nauw samen
met het beheersen van grondgebied; alle militaire dreigingen ontstonden
op lokaal niveau. Vertegenwoordigingssystemen bieden een alternatieve
arena om die macht tot uiting te brengen. Het bevoordelen van gevestigde
lokale belangen is dan ook een onvermijdelijk gevolg van deze formule.
Geografisch ingedeelde kiesdistricten zorgen ervoor dat
vertegenwoordigers specifieke groepen gaan bevoordelen, vaak ten koste
van de gemeenschappelijke belangen die alle inwoners delen.

\subsection{Nieuwe mogelijkheden in het
verschiet}\label{nieuwe-mogelijkheden-in-het-verschiet}

Zoals analyses door \emph{Public Choice}-economen aantonen, veroorzaken
schijnbaar kleine aanpassingen in de structuur van een verkiezing of in
de wijze waarop de stemming wordt geteld grote en voorspelbare effecten
op de uitslag.\footnote{Dennis C. Mueller, \emph{Public Choice}, vol.~2
  (Cambridge: \emph{Cambridge University Press}, 1989), pp.~43--226.}
Daarom moeten serieuze politicastudenten tegenwoordig ook grondige
kennis van grondwetten bezitten. Het is bovendien een van de redenen om
verder te kijken dan traditionele grondwetten en in plaats daarvan de
ultieme metagrondwet te overwegen, zoals gevormd door de overheersende
megapolitieke factoren in een bepaalde omgeving.

Technologische vernieuwingen hebben enkele van de fundamenten weggevaagd
die ervoor zorgden dat stemmen vroeger uitsluitend binnen geografische
kiesdistricten plaatsvond. In de achttiende en negentiende eeuw verliep
bijna alle communicatie lokaal. De meeste mensen leefden en stierfden
binnen enkele mijlen van hun geboorteplaats, en al hun
handelsactiviteiten en communicatie vonden in de directe omgeving
plaats. Tegenwoordig verloopt de communicatie wereldwijd vrijwel
onmiddellijk. Je kunt net zo eenvoudig zaken doen met iemand die
vijfduizend mijl ver weg woont als met je buurman. Steeds vaker
overstijgt de economie geografische grenzen, en is de samenleving veel
mobieler geworden.

Dit geldt ook voor rijkdom in het informatietijdperk. In tegenstelling
tot een staalfabriek, die politici als drukmiddel kunnen inzetten, kan
men een computerprogramma niet zomaar gijzelen via het lokale politieke
systeem. Een staalfabriek blijft hardnekkig op één plek staan, zelfs
wanneer wetgevers besluiten extra belastingen te heffen of strengere
regels in te voeren. Daarentegen laat een computerprogramma zich via een
modem bliksemsnel wereldwijd overdragen. De eigenaar kan simpelweg zijn
laptop inpakken en letterlijk wegvliegen. Dit ondermijnt tevens de
megapolitieke basis van geografisch bepaalde kiesdistricten.

Een aanzienlijk probleem dat volgens onze analyse alle representatieve
democratieën teistert, is dat geografische kiesdistricten van nature de
gevestigde belangen uit het industrieel tijdperk bevoordelen.
`Achterblijvers' vormen ideale kiezers doordat zij geografisch
geconcentreerd wonen en politiek behoeftig zijn. Ook de geschiedenis van
de industriële democratie bevestigt dit. Tijdens de hoogtijdagen van het
industriële tijdperk in de jaren dertig kregen `winnaars' uit nieuwe
industrieën structureel te weinig vertegenwoordiging in de wetgevende
debatten.\footnote{Michael A. Bernstein, \emph{De grote depressie:
  uitgesteld herstel en economische verandering in Amerika, 1929--1939}
  (Cambridge: \emph{Cambridge University Press}, 1987).} Politici kiezen
er van nature voor de gevestigde spelers te ondersteunen in plaats van
de nieuwe ondernemingen of hun potentiële klanten -- een kenmerk dat het
representatieve bestuur inherent typeert. Zoals \emph{The Rise and
Decline of Nations} betoogt, smeden langlevende industrieën doorgaans
effectievere `distributieve coalities' om te lobbyen en te strijden voor
de politieke buit.\footnote{Mancur Olson, \emph{De opkomst en ondergang
  van naties: economische groei, stagflatie en sociale rigiditeiten}
  (New Haven: \emph{Yale University Press}, 1982).}

Dit probleem neemt in de economie van het informatietijdperk enorm toe.
De creatieve krachten van de nieuwe economie verspreiden zich over de
hele wereld, waardoor zij vrijwel nooit een voldoende geconcentreerde
groep vormen die de aandacht van wetgevers weet te trekken, zoals
zalmvissers in Schotland of tarweboeren in Saskatchewan dat doen.
Sterker nog, veel van de dynamische talenten achter de nieuwe economie
slagen er waarschijnlijk niet in burger te worden binnen zelfs de
ruimste jurisdicties. Daardoor hebben zij nauwelijks stem in de
wetgevende besluitvorming van representatieve democratieën. Als treffend
voorbeeld noemen we de dubieuze pogingen van Amerikaanse wiskunde-PhD's
om buitenlandse wiskundigen de kans op een baan in de Verenigde Staten
te ontnemen.\footnote{Michael M. Phillips, `Wiskunde-ph.d.'s dragen bij
  aan de anti-vreemdelingengolf: geleerden die, geconfronteerd met hoge
  werkloosheid, pleiten voor immigratiebeperkingen,' \emph{Wall Street
  Journal}, 4 september 1996, p.~A2.} Hun xenofobische betogen aan het
Congres om werkgevers op basis van verdienste te blokkeren, zijn
inmiddels al lang achterhaald. De achterhaalde geografische
vertegenwoordiging, een erfenis uit het industrieel tijdperk, houdt geen
rekening met buitenlandse wiskundigen of andere cruciale
welvaartsbijdragers die geen kiesrecht hebben.

`Waarom geloven mensen in de legitimiteit van democratische instituties?
Het beantwoorden van die vraag is bijna net zo ingewikkeld als verklaren
waarom mensen in bepaalde religieuze dogma's geloven, aangezien het
niveau van begrip, scepsis en geloof binnen een samenleving en in de
loop der tijd sterk uiteenloopt.'\footnote{Juan J. Linz en Alfred Stepan
  (red.), \emph{de ineenstorting van democratische regimes} (Baltimore,
  Md.: \emph{The Johns Hopkins University Press}, 1978), p.~18.} JUAN J.
LINZ

Slechts weinigen hebben systematisch stilgestaan bij de impact van
technologische veranderingen die het industrialisme ondermijnen en de
inkomensverdeling radicaal herschikken. Het lijkt erop dat democratie
dan vooral een recept voor wettelijk parasitisme vormt, indien de
inkomens zo ver uiteenlopen als in de informatie-economie mogelijk is.
Bijna niemand heeft doorgrond dat sommige instituties van de industriële
overheid inherent onverenigbaar zijn met de megapolitieke realiteit van
een postindustriële samenleving. Of we deze tegenstellingen nu expliciet
erkennen of niet, hun gevolgen worden steeds duidelijker naarmate er
wereldwijd meer voorbeelden van politiek falen optreden. De
overheidsinstellingen die in onze moderne tijd zijn gevormd,
weerspiegelen de geopolitieke condities van eeuwen geleden en hebben de
omslag van een agrarische samenleving naar stedelijk industrialisme
doorstaan. Het informatietijdperk kan echter om nieuwe
representatiemechanismen vragen om chronische disfunctie -- of zelfs een
ineenstorting in Sovjetstijl -- te voorkomen.

U kunt verwachten dat in veel landen crisis van slecht bestuur zal
uitbreken, omdat politieke beloften leeg blijken te zijn en regeringen
hun krediet en institutionele steun verliezen. Uiteindelijk zullen
nieuwe institutionele vormen ontstaan die de vrijheid in deze nieuwe
technologische omstandigheden waarborgen en tegelijkertijd de
gemeenschappelijke belangen van individuen tot uiting brengen.

Dit alles wijst op het einde van de massademocratie zoals wij die in de
twintigste eeuw kenden. De vraag luidt: wat neemt haar plaats in? Als
het enige alternatief voor massademocratie een dictatuur zou zijn
waarbij het individu geen zeggenschap heeft over zijn eigen lot, dan
zouden sommigen in de verleiding kunnen komen zich bij de neo-Luddieten,
oftewel de `opstand tegen de toekomst', aan te sluiten.

\subsection{Nieuwe instellingen}\label{nieuwe-instellingen}

Gelukkig is dictatuur niet het enige alternatief voor massademocratie.
Informatietechnologie maakt het mogelijk eigen keuzes te maken. In
plaats van een collectieve beslissing binnen de beperkende context van
`massaproductie, massaconsumptie, massaal onderwijs, massamedia, massaal
vermaak en al het andere', faciliteert informatietechnologie een
werkelijk consumentgerichte keuze voor op maat gemaakte
soevereiniteitsdiensten. Dit wordt mogelijk doordat grootschalig
opereren niet langer noodzakelijk is. Wij zijn ervan overtuigd dat de
technologie van het informatietijdperk zal leiden tot nieuwe vormen van
bestuur -- net zoals de agrarische revolutie en later het industriële
tijdperk elk hun eigen specifieke vormen van sociale organisatie
voortbrachten.

Wat voor nieuwe instellingen zouden dat kunnen zijn? Laat daarbij alle
verkeerd benoemde politicologieteksten voor wat ze zijn. De nieuwe
bestuursvormen voor het informatietijdperk zullen de grenzen van
conventioneel denken overschrijden. De overgang naar zulke instellingen
is al in gang gezet. Het gaat om nauwelijks erkende improvisaties die
gericht zijn op het herstructureren van onderbenutte voordelen van
soevereiniteit. De natiestaten van de wereld, bezorgd over
afscheidingsbewegingen en de ingrijpende effecten van devolutie, hebben
zich verenigd tot het krachtigste grensbewaringskartel dat zij kunnen
vormen. Hoewel het aantal nieuwe staten in de jaren negentig wereldwijd
is toegenomen, gebeurde dit vooral in twee clusters dankzij de
ineenstorting van multietnische, communistische dictaturen in de
voormalige Sovjet-Unie en Joegoslavië. Het valt op dat andere
vooraanstaande natiestaten -- waaronder de Verenigde Staten --
strategieën hanteerden om de Sovjet-Unie zo lang mogelijk in stand te
houden. En maar weinig regeringen stonden positief tegenover de
ontbinding van Joegoslavië. Pas nadat afscheidingsbewegingen,
ondersteund door eigen militaire inspanningen, de controle overnamen,
werd de onafhankelijkheid van de voormalige Joegoslavische republieken
erkend. De wereldmachten schikten zich uiteindelijk bij het aanschouwen
van ongewapende of slecht bewapende separatisten, die massaal door hun
Servische onderdrukkers werden afgeslacht. Zelfs het verre China, een
machtige natiestaat zonder direct belang bij het behoud van het resten
van Joegoslavië, verzette zich fel tegen de pogingen tot zelfbeschikking
van de onderdrukte etnische Albanezen in Kosovo. Opmerkelijk genoeg zal
deze fixatie op het vastleggen van grenzen eerder leiden tot een
gefragmenteerde soevereiniteit dan de devolutie daadwerkelijk tegengaan.
De felle weerstand van kwetsbare natiestaten wereldwijd tegen open
afscheiding en politieke desintegratie maakt erkende soevereiniteit tot
een waardevolle vorm van transcendent kapitaal, dat staten vrijwillig
kunnen fragmenteren en onderverhuren.

Een voorbeeld van het vrijwillig fragmenteren van soevereiniteit,
waardoor in feite een particuliere, belastingvrije jurisdictie ontstaat,
is de Agulhas Bay Concession Free Zone, die vijftig vierkante kilometer
van het eiland Principe voor de kust van West-Afrika omvat. Hoewel dit
grondgebied binnen de grenzen van de Democratische Republiek São Tomé en
Principe valt, wordt de zone administratief geprivatiseerd. Het bestuur
wordt contractueel geregeld en verzorgd door WADCO, de \emph{West
African Development Corporation Ltd.}, een particulier bedrijf dat in
Zuid-Afrika is opgericht. In de zone is de voertaal Engels in plaats van
het officiële Portugees van São Tomé en Principe, en wordt de handel
gevoerd in de dollar -- het wereldgeld, de Amerikaanse dollar. De
veiligheid wordt niet gewaarborgd door de nationale politie van São Tomé
en Principe, maar door private politiediensten die door WADCO in dienst
zijn genomen. Op de handel binnen de zone is het commerciële recht van
São Tomé niet van toepassing en hebben de rechtbanken van São Tomé geen
rechtsmacht. Alle geschillen lossen de partijen op via transnationale
arbitrage conform de Parijse ICC-regels. Behalve een paar strikt
gecontroleerde en triviale uitzonderingen geldt de belastingwetgeving
van São Tomé niet en functioneren officiële monopolies daar eveneens
niet. Zo worden bijvoorbeeld telecommunicatiediensten binnen de zone
automatisch gedereguleerd. Onder voorbehoud van tijdige huurbetaling en
naleving van de overige concessievoorwaarden mag WADCO haar huurcontract
op deze particuliere, gefragmenteerde soevereiniteit automatisch en
telkens met vijftig jaar verlengen, gerekend vanaf de eerste
verlengingsdatum in 2047.

Wat WADCO in São Tomé en Principe heeft bereikt, kan door anderen in
uiteenlopende rechtsgebieden worden gekopieerd -- en dat zal ook
gebeuren. Een ware pionier van de ontwikkeling in de eenentwintigste
eeuw, Joaquin Aguirre, heeft in de Centrale Aguirre Portuaria in
oostelijk Bolivia een soortgelijke zone van particuliere soevereiniteit
opgericht. Aguirre -- multimiljonair, romanschrijver, uitvinder,
medeoprichter van de Verenigde Naties én voormalig senator van de
Boliviaanse Republiek -- is op vele vlakken een echte pionier. Halve
eeuw nadat hij een belangrijke rol speelde bij de oprichting van de
Verenigde Naties, belichaamt hij nu het archetype van het soevereine
individu in de eenentwintigste eeuw. Zijn Zona Franca, vrij van de
meeste Boliviaanse belastingen, heffingen en regelgevende beperkingen,
wijst de weg naar een nieuwe soort geprivatiseerde stadstaten die door
succesvolle individuen in het informatietijdperk steeds vaker
gerealiseerd zullen worden. Het bewijst bovendien onmiskenbaar dat het
bestaan van de massa's -- zo vaak besingevuld in lofzangen door de
voorstanders van een sterke overheid -- ingrijpend kan verbeteren
dankzij de economische ontwikkeling die vrijkomt via vrijhandelszones,
zoals die gestart door Señor Aguirre. Geleidelijk zal het aantal de
facto stadsstaten wereldwijd aanzienlijk toenemen. Inderdaad, als je als
individu voldoende financiële onafhankelijkheid bereikt, kun je de
ultieme mate van onafhankelijkheid verwerven, net zoals Joaquin Aguirre.
Mocht een gefragmenteerde, gecommercialiseerde soevereiniteit van een
ander je geen comfortabele thuisbasis bieden, dan kun je altijd je eigen
ministaat opzetten -- net zo onafhankelijk als een middeleeuws
hertogdom. In plaats van met demagogen en politieke opportunisten in de
strijd te gaan om te voorkomen dat je bezittingen worden weggeharkt en
verdeeld onder de talloze kravende handen van de massademocratie, kun je
je eigen particuliere domein van bestuur creëren.

De ingrijpende overgang van massademocratie naar het soevereine individu
als ultieme vorm van zelfbestuur hoeft niet gepaard te gaan met een
radicale verandering in de publieke opinie, noch met een wonderbaarlijke
stemming waardoor ontgoochelde kiezers massademocratie afschaffen. Zo'n
revolutie kan namelijk al onopgemerkt zijn ingezet door het verhuren van
soeverein grondgebied voor belastingvrije zones, `Zona Francas', en
vrije havens. Op termijn zal soevereiniteit steeds verder worden
gefragmenteerd, tot het punt waarop verdere opsplitsing niet meer genoeg
waarde oplevert om de transactiekosten van de devolutie te compenseren.
Gelet op de wet van Moore en Gilder's corollarium -- dat stelt dat de
bandbreedte elk jaar verdrievoudigt -- is er momenteel geen reden om te
verwachten dat de devolutietrend vroegtijdig zal stoppen. Integendeel,
wij voorzien dat de schijnbaar solide macht van de natiestaten die
vandaag de dag massademocratie belichamen, zal uiteenvallen in
tienduizenden fragmenten, in een systeem dat meer doet denken aan de
middeleeuwen dan aan het moderne industriële tijdperk.

Op termijn zullen zelfs natiestaten die nog restanten van
massademocratie in zich dragen, een ingrijpende beleidsverschuiving
doormaken om in lijn te komen met de nieuwe metaconstitutionele
realiteiten. Zoals William Keech, een vurig voorstander van democratie,
in \emph{Economic Politics: The Costs of Democracy} betoogt:\footnote{William
  Keech, \emph{economische politiek: de kosten van de democratie}
  (Cambridge: \emph{Cambridge University Press}, 1995), p.~221.} `Mensen
leren te willen wat ze zien dat ze kunnen krijgen, maar ze kunnen ook
van gedachten veranderen als ze merken dat ze niet houden van wat ze
wilden en wat ze kregen.' Met andere woorden, dat massademocratie met
haar conventionele instituties overal wordt geprezen nu de twintigste
eeuw ten einde loopt, zou wel eens een verkoopsignaal kunnen zijn. Dit
garandeert echter geenszins dat deze besluitvormingsregels de tand des
tijds zullen doorstaan, zelfs niet volgens hun eigen maatstaven. Bedenk
dat, als je buiten de politiek kijkt, er nauwelijks bewijs is dat
bestuurders, beheerders, coaches of andere professionele leiders via
democratische keuzes worden aangesteld. Integendeel, de meest
succesvolle leiders worden routinematig door eigenaars voorgedragen via
selectieprocessen, waarbij diegenen met het grootste belang een
onevenredig grote stem hebben. Als democratische selectie werkelijk een
universele methode zou zijn om capabele leiders te vinden, zou je
verwachten dat dit vrijwel uitsluitend in de politieke sfeer gebeurt.
Kortom, op basis van de huidige gegevens lijkt het aannemelijker dat de
levering van soevereine diensten wordt belemmerd door de overheersing
van democratische besluitvorming, dan dat bedrijven en
handelsorganisaties eronder lijden dat zij geleid worden door door
eigenaars aangestelde bestuurders in plaats van door een stemming.

Tegen de helft van de eenentwintigste eeuw zal de toename van
particuliere jurisdicties, gebaseerd op gefragmenteerde soevereiniteit,
onomstotelijk de voordelen van particuliere administratie bewijzen.
Kiezers zullen merken dat zij de lasten van massademocratie dragen.
Daarom zullen zij -- zoals professor Keech suggereert -- concluderen dat
de voordelen van werknemerscontrole over de overheid de kosten niet
compenseren en zetten zij de stap richting hervorming. Zelfs kieskringen
in Europa en Noord-Amerika, die nu schijnbaar fel tegen hervorming zijn,
zouden uiteindelijk kunnen stemmen om hun regio's meer open te stellen
voor particuliere bestuursvormen. Meerderheden zullen wellicht
vrijwillig -- ja, zelfs graag -- de politieke schijnvertoning opgeven
ten gunste van een particuliere bestuursvorm die er werkelijk op gericht
is de optimale voorwaarden te scheppen voor het sluiten en afdwingen van
contracten.

Zolang de overheid met al haar vertrouwde toebehoren nog bestaat, kan
zij op totaal nieuwe manieren worden geïnformeerd. Ergens, in een
bepaald rechtsgebied -- nog voordat het te laat is -- zal iemand het
potentieel van computertechnologie doorgronden, waardoor een werkelijk
representatieve overheid mogelijk wordt. We kunnen het probleem van
buitensporige campagne-uitgaven en de onmiskenbare irritatie over
voortdurende politieke campagnes binnen een oogwenk oplossen. In plaats
van verkozen te worden, selecteren we de vertegenwoordigers volledig
willekeurig via trekking, zodat hun talenten en zienswijzen nauw
aansluiten bij die van de algemene bevolking.

Dit is in feite een moderne variant van het oude Griekse systeem waarbij
men door loting functies vervulde. Zoals E. S. Staveley beschrijft in
zijn gezaghebbende geschiedenis \emph{Greek and Roman Voting and
Elections} selecteerde Athene talrijke ambtspersonen -- van magistraten
tot archons -- via trekking in plaats van via verkiezingen.\footnote{E.S.
  Staveley, \emph{Grieks en Romeins stemmen en verkiezingen} (Ithaca,
  N.Y.: \emph{Cornell University Press}, 1972), p.~62.} Ze deden dit op
een slimme manier, ondanks de mechanische beperkingen bij het
willekeurig genereren van kansen, door gebruik te maken van een
sorteermachine -- of, zoals de Atheners het noemden, de cleroterion.

Men gebruikte een rij zwart-witte bonen als willekeurige tellers om te
bepalen wie voor diverse ambten in aanmerking kwam en om de volgorde
vast te stellen waarin de tribale afdelingen als prytaneis in de Raad
aan de beurt waren.\footnote{Ibid., p.~65.} Het feit dat dit idee uit de
klassieke oudheid stamt, geeft het extra geloofwaardigheid. Het meest
aantrekkelijke aan dit systeem is echter dat het de nadelen van
zelfselectie in de politiek voorkomt -- statistisch gezien domineren
hierdoor minder advocaten en egomaniakken de publieke aangelegenheden.

Wetgevende lichamen kunnen zo worden gevormd uit ware
vertegenwoordigers. Doordat zij niet worden samengebracht door
machtsbejag en vrijwel nooit opnieuw per lot worden gekozen, hebben zij
de vrijheid om staatszaken te beheren en beleid te formuleren op basis
van een rationele analyse van de kwesties.

\subsection{Rechtstreekse commissie}\label{rechtstreekse-commissie}

Tegenwoordig hebben politici -- die er alles aan doen om stemmen te
winnen -- nauwelijks motivatie om problemen op een samenhangende wijze
te onderzoeken. Het is dan ook weinig verrassend dat hun staat van
dienst in het daadwerkelijk oplossen van problemen zo armzalig is
vergeleken met die van ondernemers, bedrijfsleiders en sportcoaches, die
op basis van prestaties beloond worden. Een beloningssysteem dat
wetgevers -- die willekeurig gekozen worden -- naar hun resultaten
beloont, zou nooit leiden tot het niveau van effectiviteit dat iemand
als Lee Kuan Yew kenmerkt. Toch is er alle reden te geloven dat hun
prestaties sterk zouden verbeteren als het salaris gekoppeld werd aan
een objectieve maatstaf, bijvoorbeeld de groei van het inkomen per hoofd
van de bevolking na belastingen. Beloon hen op basis van prestaties en
de kans dat zij goed presteren zal duizendmaal toenemen.

Het voordeel voor de samenleving van beleid dat het reële inkomen na
belastingen verhoogt, kan enorm zijn. Waarom zouden premiers en
presidenten niet zelfs een klein deel van die winst ontvangen? De
benodigde financiering voor dergelijke betalingen zou opgehaald kunnen
worden via een kleine, onopvallende belasting. Zo'n regeling zou de
samenleving bevrijden van de bedreiging waar zij momenteel mee te maken
heeft door ambitieuze politici met specialistisch politiek talent, zoals
Richard Nixon en Bill Clinton.

\begin{quote}
`Ze brachten hem goud, zilver en kleding; maar de ``Christus'' deelde al
deze zaken uit aan de armen. Wanneer er geschenken werden aangeboden,
wierpen hij en zijn vrouwelijke metgezel zich op de knieën en spraken
zij gebeden uit; maar zodra hij weer overeind kwam, beval hij de menigte
hem te aanbidden. Later organiseerde hij een gewapende bende, die hij
door het platteland leidde, reizigers op de loer liggend en beroofde hen
onderweg. Ook hier was zijn ambitie niet om rijk te worden, maar om
aanbeden te worden. Hij verdeelde de hele buit onder hen die niets
hadden -- waaronder, naar men aanneemt, ook zijn eigen volgelingen.'
\footnote{Norman Cohn, \emph{De zoektocht naar het millennium} (Oxford:
  \emph{Oxford University Press}, 1970), p.~41.} NORMAN COHN
\end{quote}

\subsection{Messiaanse
persoonlijkheden}\label{messiaanse-persoonlijkheden}

Er wordt te weinig stilgestaan bij het feit dat verkiezingspolitiek
chaotische, messiaanse persoonlijkheden aantrekt. Zulke figuren hebben
altijd bestaan en vormden vaak een serieuze bedreiging voor de sociale
orde, zelfs in agrarische samenlevingen lang voordat democratische
systemen ontstonden. Kijk je naar de loopbanen van Eudo de Stell, de
Bretonse Christus, Adelbert in de achtste eeuw, Eon in de elfde eeuw,
Tanchelm uit Antwerpen, Melchior Hoffman, Bernt Rothmann en
soortgenoten, valt één ding op: hoe duidelijker hun politieke talenten
leken, hoe groter de schade die zij aanrichtten. Omdat de staat nog niet
de handen had uitgeslagen in het organiseren van grootschalige,
systematische dwang, namen deze vroege protopolitici vaak zelf het
initiatief om te plunderen en te beroven. Zo vergaarden zij contant
geld, dat ze vervolgens onder hun volgelingen -- de armen -- verdeelden.

\subsection{Protopolitici in actie}\label{protopolitici-in-actie}

De verhalen over hun capriolen wekken de indruk van talenten die ver
achter hun tijd liggen, alsof je leest over mannen van zeven voet die
over een veld rennen, lang voordat basketbal in opkomst was.
Tegenwoordig verdienen deze torenhoge spelers dankzij het \emph{NBA}
miljoenen met dribbelen en dunkelen. Mocht basketbal verdwijnen, dan
zouden zij weer in de kieren van de samenleving verdwijnen --
hoogstwaarschijnlijk als circusacten en in bijvoorstellingen.

Toen formele politiek nog niet bestond, sloegen demagogen de weg in van
rondtrekkende prediking, de beste politieke invulling die de agrarische
wereld te bieden had. Ze richtten hun preken tot grote menigten en
beloofden, net als hedendaagse politici, op welbespraakte wijze een
beter leven aan hun volgelingen. Zowel vroeger als nu richten demagogen
zich primair op de armen. Norman Cohns grootse geschiedenis van
millenaristische bewegingen, \emph{The Pursuit of the Millennium},
beschrijft de loopbanen van talloze messiaanse leiders in een tijd nog
vóór de invoering van verkiezingen. Wat daarbij opvalt, is hoe sterk hun
persoonlijkheid overeenkomt met die van de charismatische politicus van
vandaag.

\begin{quote}
De leider bezit -- net als farao's en vele andere `goddelijke koningen'
-- alle eigenschappen van een ideale vader: hij straalt volmaakte
wijsheid en rechtvaardigheid uit en beschermt de zwakken. Tegelijkertijd
fungeert hij als de zoon die belast is met het transformeren van de
wereld, als de Messias die een nieuw hemelrijk en een herboren aarde zal
stichten en die trots kan verkondigen: `Zie, ik maak alle dingen nieuw!'
Zowel in de rol van vader als die van zoon is deze figuur kolossaal,
bovenmenselijk en almachtig. Men schrijft hem een overvloed aan
bovennatuurlijke krachten toe, waardoor hij wordt voorgesteld als licht
dat in stralen naar buiten breekt. Bovendien, door de goddelijke geest
die hem vervult, beschikt de eschatologische leider over unieke
wonderkrachten. Zijn legers zegevieren steevast en triomfantelijk, zijn
aanwezigheid maakt de aarde vruchtbaar en zijn heerschappij leidt tot
een tijdperk van harmonie zoals de oude, corrupte wereld nog nooit heeft
gekend.
\end{quote}

\begin{quote}
Deze voorstelling is uiteraard puur fantasie, omdat zij niets zegt over
de werkelijke aard en capaciteiten van enig menselijk wezen dat ooit
heeft geleefd of ooit zou kunnen leven. Toch kon men dit beeld op een
levend mens projecteren; er waren altijd mannen die er meer dan in
meegingen, die vurig verlangden om als onfeilbare, wonderbaarlijke
redders gezien te worden. Het geheim van hun opmars lag nooit in hun
afkomst of opleiding, maar altijd in hun persoonlijkheid. Ook in moderne
verslagen van deze messiaanse figuren die de armen vertegenwoordigden,
staat vaak hun welbespraaktheid, hun indrukwekkende uitstraling én hun
magnetische persoonlijkheid centraal. Men krijgt zo de indruk dat, zelfs
als sommigen zich bewust voordeden als bedriegers, de meesten zichzelf
werkelijk zagen als belichaamde goden. En die overtuiging werkte
moeiteloos over op menigten waarvan het diepste verlangen juist uitging
naar een eschatologische redder. \footnote{Ibid., pp.~84--85.}
\end{quote}

Hoewel deze passage op bewonderenswaardige wijze de vermeende
millenaristische redders beknopt schetst -- zij die de middeleeuwse
samenleving zo vaak in beroering brachten -- slaagt zij er niet in de
volledige reikwijdte van Cohns gezaghebbende overzicht weer te geven.
Wie het gehele werk leest, kan niet anders dan in de capriolen van deze
profeten de vertrouwde kenmerken van de moderne demagoog opmerken:
welbespraaktheid, `persoonlijk magnetisme', messiaanse pretenties en het
terugkerende verlangen om als vertegenwoordiger van de armen aanbeden te
worden.

Het grootste verschil tussen hoe de middeleeuwse samenleving met deze
bedriegers omging en hoe de democratie hen aan het einde van de
twintigste eeuw benaderde, bestaat eruit dat men hen in de middeleeuwen
vrijwel altijd ter dood veroordeelde, terwijl de hedendaagse
democratische politiek via een open kanaal legitieme kansen biedt om de
macht in de natiestaat te grijpen.

Als een systeem routinematig de controle over de grootste en meest
dodelijke ondernemingen wereldwijd overdraagt aan de winnaar van
populariteitswedstrijden tussen charismatische demagogen, zal het op den
duur onvermijdelijk de negatieve gevolgen ervan ervaren.

\subsection{Betaal leiders om goed werk te
leveren}\label{betaal-leiders-om-goed-werk-te-leveren}

Zoals hierboven al werd aangegeven, is het vrij eenvoudig om een betere
manier te vinden om talentvol leiderschap binnen een organisatie veilig
te stellen, namelijk door deze leiders daadwerkelijk aan te stellen.
Deze werkwijze wordt in concurrerende economieën op grote schaal en met
succes toegepast.

Een doordacht selectieproces, gecombineerd met een stimulerend
beloningssysteem voor positief leiderschap, zorgt ervoor dat bekwame
mensen de overheid gaan leiden. Het trekt bovendien nieuw talent aan dat
zich anders nauwelijks zou interesseren voor bestuurlijke vraagstukken.

De meest getalenteerde topfunctionarissen wereldwijd zouden
wanpresterende regeringen met enthousiasme overnemen als hun beloning
afhangt van de concrete resultaatsverbetering die zij voor de
samenleving realiseren.

Een leider die het reële inkomen in elk vooraanstaand Westers land fors
weet te verhogen, verdient naar recht veel meer dan Michael Eisner. In
een betere wereld is iedere succesvolle regeringsleider multimiljonair.

\subsection{Elektronische
plebiscieten}\label{elektronische-plebiscieten}

Een andere voor de hand liggende oplossing voor representatief
wanbestuur is het inzetten van elektronische plebiscieten, waarbij
burgers -- bijvoorbeeld een representatieve groep die via een
manipulatiebestendige loting wordt samengesteld -- rechtstreeks stemmen
op wetgevende voorstellen. Dankzij computertechnologie kunnen we op een
democratische wijze besluiten vaststellen via deze plebiscieten. Je kunt
ze bovendien eenvoudig combineren met een toewijzingssysteem, zodat per
specifiek vraagstuk slechts een beperkt aantal stemmers meedoet. Hoe dan
ook, voor potentiële kiezers is het veel eenvoudiger om politieke
vraagstukken te doorgronden dan om te proberen politici te ontcijferen
en hun standpunten over dezelfde kwesties te beoordelen, om nog maar te
spreken van het werkelijk weten wat die politici gaan doen zodra zij de
macht in handen hebben. Dit wordt alleen maar ingewikkelder nu politici
en hun medewerkers steeds bedrevener worden in het verpakken en
manipuleren van de beelden die ze aan het publiek presenteren.

\section{Gecommercialiseerde
soevereiniteit}\label{gecommercialiseerde-soevereiniteit}

Wij voorzien dat er een nieuw systeem opstaat dat de traditionele
politiek vervangt. Hoewel we in de eerder genoemde opties elk wel wat
voordelen zien, verwachten we niet dat hervormingen de politiek nieuw
leven inblazen. Integendeel, zij raakt verouderd en verliest op veel
vlakken haar relevantie. We bedoelen hiermee niet dat wij een dictatuur
voor ogen hebben, maar dat de overheid een ondernemende, commerciële
vorm aanneemt -- een commercialisering van soevereiniteit.

In tegenstelling tot een dictatuur of zelfs democratie sluit
gecommercialiseerde soevereiniteit de individuele keuzevrijheid niet
uit. Integendeel, zij geeft ieder mens meer ruimte om zijn ideeën te
uiten. Bovendien biedt dit systeem diegenen die er hun voordeel mee
weten te doen een praktischere omgeving voor besluitvorming en
zelfbepaling dan alle andere vormen van sociale organisatie die we tot
nu toe kenden.

\subsection{Op maat gemaakte overheid}\label{op-maat-gemaakte-overheid}

Om dit niet als een ouderwets ideaal te laten klinken, bedenkt u dat
microtechnologie objecten miniaturiseert en uit elkaar haalt. Dankzij
deze technologische vooruitgang kunnen we kiezen voor maatwerk in plaats
van massaproductie. U kunt vandaag een winkel binnenlopen en een
spijkerbroek kopen die wordt gesneden volgens een patroon dat perfect op
uw maten is afgestemd en halverwege de wereld genaaid wordt. Zodra
nieuwe instituties zich ontwikkelen om in te spelen op de megapolitieke
realiteiten van het informatietijdperk, krijgt u een bestuur dat net zo
nauw aansluit op uw persoonlijke wensen en voorkeuren als een perfect
zittende spijkerbroek.

Alvin Toffler, van alle mensen, heeft kritiek geleverd op het idee dat
informatietechnologie burgers in klanten kan veranderen. Volgens Toffler
geeft dit model een te beperkte kijk: `Of we het nu willen of niet, er
bestaat een wereld van religie en gevoel die niet zomaar te reduceren
valt tot contractuele relaties.'\footnote{Geciteerd in Kelly, \emph{op.
  cit.}, p.~46.} We hebben eerder betoogd dat `de wereld van
nationalistisch gevoel' niet eenvoudig te reduceren is tot louter
`contractuele relaties.' Dit betekent echter niet dat zo'n benadering
onhaalbaar of onwenselijk is. Een minder irrationeel enthousiasme voor
nationalisme kan immers miljoenen levens redden.

\subsection{Toetreding, uittreding en
inspraak}\label{toetreding-uittreding-en-inspraak}

Voor velen blijft de commercialisering van soevereiniteit een vreemd
concept -- kennelijk zelfs voor Alvin Toffler. Toch drukt het
economische perspectief op soevereiniteit zich dagelijks uit in het
leven aan het einde van de twintigste eeuw. In elke enigszins vrije
economie kunt u uw wensen direct kenbaar maken door diensten en
producten te kopen of juist af te zien van aankopen. Als u ontevreden
bent over de uitvoering van een product of de service van een aanbieder,
toont u uw ongenoegen direct via `exit' -- kortom, u verlegt uw
klandizie naar een andere leverancier.

In de voorgaande hoofdstukken hebben we onderzocht hoe technologische
vooruitgang het binnenkort mogelijk maakt om in cyberspace activa te
creëren die vrijwel immuun zijn voor roofzuchtige inmenging door
natiestaten. Dit leidt tot een \emph{de facto} metaconstitutionele
voorwaarde: overheden moeten u eerst daadwerkelijk bevredigende
dienstverlening leveren voordat u hun rekeningen betaalt. Waarom? Omdat
het betalen van de inkomstenbelasting in de praktijk vrijwel net zo
vrijwillig zal verlopen als men in de theorie aanneemt.

\subsection{Het vermijden van logge politieke
kanalen}\label{het-vermijden-van-logge-politieke-kanalen}

In feite zorgt de verwachte ontwikkeling van de informatietechnologie
ervoor dat klanten overheden daadwerkelijk gaan controleren. Als klant
beschik je al snel over honderden -- later zelfs duizenden --
mogelijkheden om direct je beschermingskosten te verlagen, bijvoorbeeld
door een particuliere belastingafspraak te sluiten met een natiestaat of
door je volledig af te keren van natiestaten en over te stappen op
opkomende mini‑soevereiniteiten. Deze contractuele `toetredings'- en
`exit'-opties weerspiegelen in economische termen jouw wensen als klant.
Stemmen met je voeten en met je geld levert als groot voordeel op dat je
exact de gewenste resultaten behaalt.

Hoe verhouden jouw `toetredings'- en `exit'-opties als klant zich tot de
politieke manier van je mening geven in een democratie? Mensen die
ontevreden zijn over een product of dienst -- vooral als het gaat om
iets dat door de overheid geleverd of streng gereguleerd wordt -- laten
hun stem vaak horen door brieven te sturen naar de president van de
Verenigde Staten of door een afspraak te maken met hun parlementslid of
een andere geschikte gekozen functionaris. Soms hebben zulke
liefdesbrieven effect, maar meestal niet. Als die pogingen aanvankelijk
niet slagen, organiseren mensen die verandering wensen een demonstratie,
adverteren ze een volledige pagina in een krant of proberen ze zelfs
zelf een politieke functie te bekleden.

Hoewel de politieke vorm een platform biedt voor welsprekende
verklaringen en retoriek, heb je er zelden direct voordeel van en
verbeter je je positie er niet mee. Wanneer je te maken krijgt met een
inferieur product of een overheidsdienst, blijf je immers de kosten
betalen totdat je het hele politieke systeem ervan weet te overtuigen in
te stemmen met jouw verzoek tot verandering.

In westerse landen -- en tegenwoordig wereldwijd -- betekent dit dat men
de steun van de meerderheid voor een democratisch systeem moet
waarborgen. Het betrekken van een meerderheid brengt enorme
transactiekosten met zich mee tussen jou en het bereiken van wat
hoogstwaarschijnlijk een relatief eenvoudig en rationeel doel is.

Milton Friedman besprak in \emph{Capitalism and Freedom} de verdiensten
van een economische in plaats van een politieke manier van uiten bij het
formuleren van zijn voorstel voor schoolvouchers.

Ouders zouden veel directer hun mening over scholen kenbaar kunnen maken
door hun kinderen van school te halen en ze naar een andere school over
te plaatsen, in tegenstelling tot nu, waar verhuizen vaak de enige
uitweg is. Momenteel biedt verhuizen immers de enige mogelijkheid om een
standpunt te laten gelden en moeten zij hun ongenoegen anderszins via
omslachtige politieke routes verwoorden.\footnote{Milton Friedman,
  \emph{Capitalism and Freedom} (Chicago: University of Chicago Press,
  1962), p.~91. Besproken door Hirschman, \emph{op. cit.}, pp.~16--17.}

Albert O. Hirschman, een voorvechter van politieke betrokkenheid, sprak
zijn kritiek uit op Friedman's pleidooi voor `exit' als directe manier
om ontevredenheid over een organisatie te uiten. Wie niet vertrouwd is
met economische theorieën, zou gemakkelijk naïef denken dat het
simpelweg kenbaar maken van zijn mening voldoende is.\footnote{Hirschman,
  \emph{op. cit.}, p.~17.}

Of het effectiever is om als consument via marktmechanismen --
bijvoorbeeld door wel of niet zijn steun te verlenen -- zijn mening te
geven, of via de ingewikkelde gang van zaken in de politiek, blijft een
complexe en omstreden kwestie. Meningen hierover lopen uiteen. Voor
mensen die hun politieke invloed vooral uitoefenen door voordelen op de
kosten van anderen af te dwingen, kan de overstap naar een economische
vorm van uitdrukken als een trieste plaatsvervanger overkomen,
vergeleken met het schrijven van een brief aan een politicus om meer
eisen te stellen.

Economische expressie en `wederkerige socialiteit'

Wie ernaar streeft medemensen te betrekken in een wederkerige, in plaats
van dwangmatige of parasitaire, socialiteit ontdekt in de economische
manier van uiten de mogelijkheid om met minder inspanning veel meer
voldoening te behalen. Wat professor Hirschfield ervan beweert, valt
gemakkelijk aan te tonen.

Elke reeks economische handelingen -- zoals toetredingen, lopende
contracten en uittredingen -- kan als een politieke `stem' worden
ingezet door simpelweg een grote groep mensen bij het
besluitvormingsproces te betrekken. Probeer het eens als experiment.
Voor zo'n experiment heb je slechts een paar honderd mensen nodig die
vinden dat er te weinig politiek in hun leven is. In plaats van hun
jaarlijkse beschikbare inkomen aan duizenden losse aankopen te besteden,
zetten ze deze overvloed aan economische besluiten om in een handvol
politieke beslissingen.

Aanvankelijk stemt iedereen ermee in om zijn beschikbare inkomen samen
te voegen en af te zien van individuele aankopen. In plaats van dat
iedereen duizenden dollars op talloze manieren persoonlijk uitgeeft,
krijgt iedere persoon één stem -- of wellicht meerdere stemmen,
afhankelijk van het aantal te vervullen functies. In plaats van direct
geld te verbrassen om op elk gewenst moment precies te krijgen wat men
wil, brengt men zijn stem in op de weinige momenten waarop verkiezingen
plaatsvinden. De gekozen vertegenwoordigers bepalen vervolgens hoe de
inmiddels enorme gezamenlijke geldpot wordt besteed.

Vervolgens neem je, samen met de anderen, deel aan de consumptie van
precies die goederen die door de besturende commissie namens de
meerderheid zijn goedgekeurd.

Lijkt dat al op een `omslachtige politieke uitlaatklep' voor expressie?
Wacht maar. Dit model bezit de potentie om net zo welsprekend en
overtuigend te zijn als de nationale politiek, en vangt tevens een
aanzienlijk deel van de opgebouwde frustratie op.

Stel bijvoorbeeld dat je dol bent op verse broccoli, maar de groep qua
smaakvoorkeuren gemiddeld verdeeld is -- dan zit je waarschijnlijk in de
problemen. De kans is groot dat enkelen, of zelfs de meerderheid, liever
een groter deel van het gezamenlijke voedingsbudget aan rood vlees
besteden dan aan verse verse groenten. Om te voorkomen dat de
kantinecommissie naar een magazijnwinkel trekt en het hele jaarlijkse
groentebudget verkwist op ingeblikte erwten en maïs, moet je wellicht
opstaan en je mening kenbaar maken. Je kunt de aandacht van de groep
vestigen op de voordelen van het consumeren van meer vitamines en
fytonutriënten -- zoals het sulforaphaan in broccoli -- tegenover de
verzadigde vetten en het cholesterol in rood vlees.

Hoe je dit of een ander punt ook maar duidelijk maakt, blijft in dit
geconstrueerde politieke model even een raadsel als voor de voorstanders
van welke politieke zaak of kandidaat dan ook. Je kunt een toespraak
houden, maar daarvoor moet een aanzienlijk deel van je doelgroep zich al
hebben verzameld en bereid zijn te luisteren. Je kunt ook flyers laten
drukken, mits zo'n `campagne-uitgave' volgens de huisregels van jouw
politieke spel is toegestaan. Of je schrijft brieven. Beide opties
hangen ervan af dat de ontvangers voldoende leesvaardigheid bezitten.

\begin{quote}
`Het schetst een beeld van een samenleving waarin de overgrote
meerderheid van de Amerikanen zich er niet van bewust is dat zij niet
beschikken over de vaardigheden die nodig zijn om in onze steeds
technologischer wordende samenleving en op de internationale marktplaats
in hun levensonderhoud te voorzien.'
\end{quote}

\begin{quote}
Richard Riley; VS-secretaris van onderwijs, in \emph{`Adult Literacy in
America'}
\end{quote}

\subsection{Negentig miljoen
Alzheimerpatiënten?}\label{negentig-miljoen-alzheimerpatiuxebnten}

Als de groep waarmee je dit modelpolitieke experiment uitvoert toevallig
uitsluitend uit Amerikanen bestaat, wordt het buitengewoon lastig om een
overtuigende boodschap over te brengen, zeker wanneer de groepsleden
representatief zijn voor het gehele Amerikaanse electoraat. Het idee dat
een onevenredig groot deel van de burgers van 's werelds machtigste
natiestaat onderpresteerders zou zijn, wordt somber bevestigd door het
meest grondige onderzoek naar de competentie van Amerikaanse
volwassenen. De studie \emph{`Adult Literacy in America'} laat zien dat
het vinden van een geletterd publiek voor elk politiek betoog verre van
eenvoudig is. Een aanzienlijk deel, misschien zelfs de meerderheid, van
Amerikanen van vijftien jaar en ouder beschikt niet over de
basisvaardigheden die nodig zijn om ideeën te evalueren en onderbouwde
oordelen te vormen. Volgens het Amerikaanse Ministerie van Onderwijs
kunnen 90 miljoen Amerikanen geen brief schrijven, een bustijdschema
doorgronden of zelfs simpele rekensommen uitvoeren. Dat is precies wat
je zou verwachten als 90 miljoen Amerikanen de diverse stadia van de
ziekte van Alzheimer doormaakten. Bij dertig miljoen bleek men zo
onkundig dat zij zelfs niet in staat waren vragen te beantwoorden.

Mocht je gezondheidsboodschap het tij niet hebben kunnen keren --
terwijl deze zich anders vanzelf op een bepaald niveau zou stabiliseren
-- dan kun je hulp inschakelen van dierenrechtenactivisten. Misschien
krijg je hen zover om je tegenstanders te laten picketten in de
kantinecommissie of opschudding te zaaien over het kwaad van het doden
van koeien bij de huizen van invloedrijke personen.

We zouden dit voorbeeld oneindig kunnen uitwerken -- veel verder dan het
geduld van rationele mensen toelaat. Het laat namelijk twee zaken zien:
(1) dat elke economische handeling -- of het nu een toetreding of een
uittreding betreft -- kan worden omgevormd tot een politieke stemming
wanneer men er een collectieve beslissing van maakt en (2) dat
dergelijke collectieve besluiten, ondanks de ruimte voor welsprekendheid
die zij bieden, juist buitengewoon traag en vaak onbuigzaam zijn.

Precies dit bevestigt de praktijk. Het is allesbehalve eenvoudig de
benodigde inzet te organiseren om de werking van een democratie
fundamenteel te veranderen. Om het nogmaals te benadrukken: dit zou wel
eens de reden kunnen zijn dat democratische verzorgingsstaten eeuwenlang
konden concurreren met alternatieve bestuursvormen en aan het einde van
het industriële tijdperk de overhand kregen. Democratie slaagde als
politiek systeem juist doordat haar werking het voor burgers bijzonder
lastig maakte om de staat effectief te controleren of haar aanspraken op
middelen te beperken.

Maar nu de onbeperkte inmenging van de staat in jouw zaken in het
informatietijdperk geen militair voordeel meer oplevert, vinden
vindingrijke mensen ongetwijfeld betere manieren om de weinige,
waardevolle diensten die overheden daadwerkelijk leveren, te verkrijgen.
Het is waarschijnlijk dat de daadwerkelijke macht verschuift naar
collectieve mechanismen die zichzelf uiteindelijk niet meer kunnen
bekostigen. Wij verwachten dat efficiëntie de overhand krijgt boven
geconcentreerde macht. Zoals Neil Munro beknopt verwoordde: ``{[}H{]}et
is geautomatiseerde informatie, niet mankracht of massaproductie, die de
Amerikaanse economie steeds meer aandrijft en die oorlogen zal winnen in
een wereld die is toegerust op 500 tv-zenders. De geautomatiseerde
informatie bestaat in cyberspace -- de nieuwe dimensie gecreëerd door de
eindeloze reproductie van computernetwerken, satellieten, modems,
databanken en het publieke internet.'' \footnote{Neil Munro, `De nieuwe
  nachtmerrie van het Pentagon: een elektronische Pearl Harbor,'
  \emph{Washington Post}, 16 juli 1995, p.~C3.}

Grote legers zullen in zo'n wereld nauwelijks nog een rol spelen.
Efficiëntie krijgt daarbij een steeds grotere betekenis. Dankzij
microtechnologie, die ons een geheel nieuwe dimensie van bescherming
biedt -- zoals we in hoofdstuk 6 en elders al bespraken -- krijgen
individuen voor het eerst in de geschiedenis de mogelijkheid om activa
te creëren en te beveiligen die volledig buiten het bereik vallen van
het territoriale geweldmonopolie van welke overheid dan ook. Deze activa
kunnen dan namelijk optimaal door individuen worden beheerd.

Het lijkt volkomen logisch dat jij en vele toekomstige soevereine mensen
met `je voeten stemmen' door je af te melden bij dominante natiestaten
en in plaats daarvan een persoonlijke beschermingsdienst af te nemen bij
een perifere natiestaat of een nieuwe mini-soevereiniteit, die slechts
een commercieel betaalbaar tarief hanteert in plaats van het merendeel
van je vermogen opeist. Kortom, je zou waarschijnlijk instemmen met een
aanbod van \$50 miljoen om naar Bermuda te verhuizen.

Eerst uitstappen, later contracteren.

De eerste impuls voor de commercialisering van soevereiniteit moet komen
van mensen die door te vertrekken hun economische voorkeur kenbaar
maken. In de Verenigde Staten blijkt deze optie het moeilijkst
realiseerbaar, terwijl ze daar tegelijkertijd het meest waardevol is. De
`Berlijnse Muur' voor kapitalisten, ingesteld door president Bill
Clinton en het Republikeinse Congres, staat haaks op de slogan `hou
ervan of verlaat het', die in de jaren zestig vol vertrouwen door
Amerikaanse nationalisten werd uitgedragen. Door strafbelastingen te
heffen op degenen die vertrekken, wil de exitbelasting loyaliteit
afdwingen. Niettemin kan deze wraakzuchtige wetgeving -- die doet denken
aan de straffen die in de laatste dagen van het Romeinse Rijk op
vluchttende landeigenaars werden toegepast -- onbedoeld de basis vormen
voor een rationeler beleid in het latere informatietijdperk.

Op een gegeven moment, zodra voldoende bekwame mensen vertrokken zijn en
grote fortuinen offshore hebben opgebouwd, wordt het voor de Amerikaanse
autoriteiten aantrekkelijk om burgers of groenekaarthouders toe te staan
zich uit toekomstige belastingverplichtingen uit te kopen door éénmalig
een exitbelasting te betalen, zonder daadwerkelijk te vertrekken. Met
andere woorden, de exitbelasting kan wel eens als model dienen voor een
eenmalige afkoop. Een overheid die een exitbelasting heft, profiteert
immers aanzienlijk als zij toestaat dat vertrekkers hun verblijf later
hervatten op basis van een privaat verdrag, zoals dat tegenwoordig in
Zwitserland en elders mogelijk is.

Dergelijke maatregelen van de Verenigde Staten of andere regeringen
vormen rationele, inkomensoptimaliserende initiatieven. Uiteindelijk
dwingt de concurrentie op het gebied van beschermingsdiensten de
belastingtarieven omlaag en passen de voorwaarden voor belastingheffing
zich aan meer beschaafde normen aan. In plaats van te rekenen op
wetgevende lichamen om acceptabele belastingregimes in te voeren, zullen
toekomstige soevereine individuen via privaat verdrag in staat zijn om
op maat gemaakte, aanvaardbare beleidsvoorwaarden te bedingen.

\section{Het beledigen van de ware
gelovigen}\label{het-beledigen-van-de-ware-gelovigen}

Natuurlijk beweren we niet dat veel van dit beleid in de smaak zal
vallen. De denationalisering van het individu en de daarmee gepaard
gaande commercialisering van soevereiniteit zullen de achterblijvende
ware gelovigen -- zij die nog vasthouden aan de clichés uit de politiek
van de twintigste eeuw -- diep beledigen. Net als de inmiddels overleden
Christopher Lasch zien zij het verval van de politiek als een bedreiging
voor het welzijn van de meerderheid. Volgens hen zou een heropleving van
de politiek uit het industriële tijdperk, met de nadruk op
inkomensherverdeling, een oplossing kunnen bieden voor de problemen die
velen ondervinden door de competitieve druk van de
informatietechnologie.

E. J. Dionne, Jr.~is een politiek verslaggever voor de \emph{Washington
Post}. Net als Lasch kijkt hij met nostalgie terug op de politiek.
Daarnaast belichaamt hij een sociaal-democratische, egalitaire impuls
die de komende decennia steeds luider zal klinken, nu de nieuwe
megapolitieke realiteiten van het informatietijdperk de overgebleven
instituties in de moderne wereld steeds krachtiger ondermijnen. Dionne
wijt de brede materiële vooruitgang in levensstandaarden, die in de
twintigste eeuw in rijke rechtsgebieden wijdverspreid was, vooral aan
democratische politiek en niet aan technologische of economische
ontwikkelingen. Hij stelt dat hoop voor de toekomst inhoudt dat het
politieke gezag zich ook moet uitstrekken tot de technologieën van het
informatietijdperk:

\begin{quote}
De grootste noodzaak in de Verenigde Staten en in de hele democratische
wereld is een hernieuwde inzet op democratische hervorming, de politieke
drijfveer die het industriële tijdperk zo succesvol maakte. De
technologieën van het informatietijdperk bouwen op zichzelf geen
succesvolle samenleving op, net zoals het industrialisme de wereld,
wanneer het aan zichzelf wordt overgelaten, niet beter heeft gemaakt.
\ldots{} Zelfs de meest baanbrekende technologische doorbraken en de
meest briljante toepassingen van het internet gaan ons niet behoeden
voor sociale ontwrichting, criminaliteit of onrecht. Alleen de politiek
-- de kunst om ons te organiseren -- kan überhaupt beginnen dergelijke
taken op zich te nemen.\footnote{E. J. Dionne, `Waarom rechts het mis
  is,' \emph{Utne Reader}, juni 1996, p.~32.}
\end{quote}

Dionne en anderen zoals hij snappen niet dat de omstandigheden die het
leven in de twintigste eeuw zo bevorderlijk maakten voor systematische
dwang, niet het resultaat waren van bewuste menselijke keuzes. De
uitdrukking `de kunst van hoe wij ons organiseren' zou in vroegere
tijden nauwelijks te begrijpen zijn geweest. Samenlevingen zijn
simpelweg te complex om als het resultaat van een doelbewuste inspanning
tot zelforganisatie te worden gezien. De moderne natiestaten zijn
spontaan ontstaan als een toevallig bijproduct van de industriële
technologie, die de opbrengsten van geweld vergrootte. Tegenwoordig
verlaagt informatietechnologie die opbrengsten. Dit maakt de politiek
anachronistisch en onherstelbaar, hoe oprecht men er ook voor pleit haar
te behouden voor het volgende millennium.

\begin{quote}
'Niet van vandaag, noch van gisteren hetzelfde

Zij leven door alle tijden; en waar zij vandaan kwamen

Dat weet niemand.' - SOPHOCLES, Antigone
\end{quote}

\section{`Ze maken ze niet zoals
vroeger'}\label{ze-maken-ze-niet-zoals-vroeger}

De vurige drang om `wetten te maken' -- een vanzelfsprekend onderdeel
van het gezond verstand in de politiek van de twintigste eeuw -- is
geenszins universeel in alle culturen. Wanneer deze gewoonte in de
toekomst verdwijnt, kan dat wijzen op een cyclus die door de eeuwen heen
op en neer heeft gewogen. Zo meenden de oude Grieken dat wetten niet
door mensen gemaakt konden worden. Zoals de filosoof Ernst Cassirer
opmerkt, geloofden de Grieken namelijk dat `de ongeschreven wetten, de
wetten van de rechtvaardigheid, geen begin in de tijd
hebben.'\footnote{Ernst Cassirer, \emph{The Myth of the State} (New
  Haven: Yale University Press, 1946), p.~81.} Net als andere
pre-politieke volkeren waren zij ervan overtuigd dat niemand de
natuurlijke, `geometrische' wetten van de rechtvaardigheid kon
verbeteren -- wetten die niet door een menselijke macht waren gecreëerd.

Zij geloofden niet in een `wetgever.' Zoals Cassirer het verwoordde:
`Het is door rationeel denken dat wij de normen voor moreel gedrag
vinden, en het is enkel de rede, en alleen de rede, die hen hun
autoriteit kan verlenen.' Elke poging om wetten via wetgeving aan de
samenleving op te leggen, lijkt in die zin even onzinnig als het
proberen de geometrie via wetgeving te veranderen.

\subsection{Wetgeving als
heiligschennis}\label{wetgeving-als-heiligschennis}

Om uiteenlopende redenen bestond gedurende een groot deel van de
middeleeuwen een duidelijke weerstand tegen het `wetten maken.' John~B.
Morrall schrijft: `{[}V{]}oor de Germanen bestond de wet al van
oudsher.' Het vormde een garantie voor de rechten van de individuele
stamleden.\footnote{John B. Morrall, \emph{Politiek denken in de
  middeleeuwen} (New York: Harper Torchbooks, 1962), p.~15.}\\
Koningen en raden

\begin{quote}
hadden vooralsnog niet de intentie om nieuwe wetten te maken. Zo'n
voornemen zou in die vroege middeleeuwse tijden niet alleen overbodig
zijn geweest, maar zelfs enigszins godslasterlijk, want zowel de wet als
het koningschap straalden een heiligheid uit. In plaats daarvan zagen
koning en raadslieden zichzelf louter als uitleggers of verduidelijkers
van de ware betekenis van het reeds bestaande en volledige
wetgevingssysteem. De Germaanse traditie drukte een idee op de
middeleeuwse geest dat men nooit vergat, zelfs als de praktijk er anders
uitzag: goede wetten werden herontdekt of herformuleerd, maar nooit
geheel opnieuw gemaakt.\footnote{Ibid., p.~16.}
\end{quote}

Na alle excessen van de regelgeving in de twintigste eeuw krijgt die
oude houding een zekere, ouderwetse charme. Het verlangen om de staat
haar dwingende macht voor privébelangen in te zetten -- met name voor de
herverdeling van inkomsten -- leek bijna vanzelfsprekend.

\subsection{Regrets}\label{regrets}

Het verbaast dan ook niet dat er verdrietige liederen klinken over de
laatste dagen van de politiek. Ze zijn volledig voorspelbaar, en dat
niet alleen omdat ze de blindheid van de meeste denkers voor de eisen
van megapolitiek blootleggen. Weinig politieke verslaggevers, zoals
Dionne, accepteren de schijnbare verdroging en ondergang van de
politiek, ook al brengt dat hen weer in de nabijheid van misdaadzaken.
Aan het einde van de middeleeuwen riepen mensen op tot een heropleving
van de ridderlijkheid. Denk bijvoorbeeld aan \emph{Il Libro del
Cortegiano} oftewel \emph{The Book of the Courtier}, dat graaf
Baldassare Castiglione in 1514 schreef en in 1528 in Venetië door Aldus
uitgaf.

Castiglione voelde een diep verlangen naar de heropleving van
ridderlijke deugden, maar zo'n sentiment kon in de zestiende eeuw het
verleden dat uitsterfde niet weer tot leven brengen. Evenmin zal dat in
de eenentwintigste eeuw gebeuren.

Zoals wij in onze uitleg van de theorie van megapolitiek probeerden over
te brengen, vormen technologische drijfveren -- en niet de publieke
opinie -- de voornaamste bronnen van verandering. Als onze theorie van
megapolitiek klopt, komt de vervanging van het feodale systeem en de
ridderlijkheid -- die steunden op persoonlijke eden en relaties -- door
het moderne tijdperk, met zijn burgerschapsconcept en staatgerichte
politiek, niet voort uit ideeën. De cruciale drijfveer zijn de
verschuivingen in kosten en baten die nieuwe technologie met zich
meebrengt.

Ridderlijkheid verdween niet omdat Castiglione of anderen er niet in
slaagden een onverschillige bevolking -- die bovendien geen invloed had
op deze zaak -- ervan te overtuigen dat eer en moraliteit in de politiek
overbodig waren.

Integendeel, in \emph{Courtier} bekritiseert Castiglione vorsten en het
gedrag dat zijn tijdgenoot Niccolo Machiavelli in zijn \emph{Il
Principe} (oftewel \emph{The Prince}) juist toejuichtte.

Maar wat gebeurde er daarna? Machiavelli bereikte uiteindelijk een veel
breder publiek met zijn boek, niet doordat zijn betoog in \emph{The
Prince} overtuigender was, maar omdat zijn adviezen beter aansloten bij
het grootschalige politieke landschap van het moderne tijdperk.

Zoals de invloedrijke twintigste-eeuwse filosoof Ernst Cassirer opmerkte
in zijn bespreking van `het morele probleem in Machiavelli',

\begin{quote}
Het boek beschrijft, met volledige onverschilligheid, de manieren en
middelen waarmee politieke macht verkregen en behouden dient te worden.
Over het juiste gebruik van deze macht zegt het geen woord. \ldots{}
Niemand had ooit getwijfeld dat het politieke leven, zoals het ervoor
stond, vol misdaden, verraderij en zware misdrijven zat. Maar geen
enkele denker vóór Machiavelli had ooit de kunst van deze misdaden
onderwezen. Deze daden werden wel verricht, maar zij werden niet
onderwezen. Het feit dat Machiavelli beloofde een leermeester te worden
in de kunst van sluwheid, verraderij en wreedheid, was ongehoord.
\end{quote}

Kortom, \emph{The Prince} presenteerde een radicaal recept waarmee een
aspirant-heerser zijn carrière ten koste van anderen kon bevorderen.
Machiavelli keurde gedragingen goed die perfect aansloten op de harde
werkelijkheid van een machtsgetrokken tijdperk. Tegelijkertijd bleek het
dubbelspel -- dat moderne politici als een slimme strategie waarderen --
ronduit schandalig en ondermijnde het de in voorgaande eeuwen opgebouwde
cultuurnormen van ridderlijkheid.

Zoals we eerder zagen, draaiden de deugden van de ridderlijkheid er
vooral om dat iedereen zich strikt aan zijn eden hield. Dat vereiste men
in een samenleving waar bescherming werd gegarandeerd in ruil voor
persoonlijke diensten. De afspraken waarop het feodale systeem was
gebouwd, zouden niet vanzelf herleven als mensen alleen hun eigen
belangen nastreefden. Daarom moesten de feodale verplichtingen -- de
ruggengraat van de ridderlijkheid -- altijd steunen op een sterk gevoel
van eer. In dat opzicht was Machiavelli's advies, dat een vorst niet zou
moeten aarzelen te liegen, bedriegen en stelen wanneer dat hem in zijn
voordeel werkte, ronduit subversief.

Toen de twintigste eeuw ten einde kwam, bleven Machiavelli's argumenten
in de belangstelling als middel om de moderne politiek en de
uiteenlopende misdaden en tirannieën van die tijd te doorgronden. In
tegenstelling daarmee raakte het werk van Castiglione vrijwel in de
vergetelheid. Over een jaar lezen waarschijnlijk slechts een handjevol
masterstudenten literatuur en enkele kenners van de
etiquettegeschiedenis \emph{Il Libro del Cortegiano} van kaft tot kaft.

Binnen enkele decennia raakt de megapolitiek van het informatietijdperk,
zoals vervat in \emph{De vorst}, duidelijk achterhaald. Het
onafhankelijke individu zal een nieuw succesrecept moeten hanteren --
eentje die sterk inzet op eer en integriteit bij het inzetten van
middelen buiten de greep van de staat. We verwachten dat E. J. Dionne,
Jr.~en de overige nog levende sociaaldemocraten zo'n advies met weinig
enthousiasme zullen oppikken.

\subsection{Door klanten vastgesteld
beleid}\label{door-klanten-vastgesteld-beleid}

Dit geldt vooral in de beginfase van de transitie, wanneer de meeste
rechtsgebieden nog worstelen met de verplichting om een beleid te
ontwikkelen dat de meerderheid van de bevolking weet te bekoren. Later,
wanneer de democratie verzwakt en de markt voor soevereiniteitsdiensten
zich verder ontwikkelt, begrijpen we beter welke marktomstandigheden het
`beleid' beperken.

Wat we tegenwoordig onder `politiek leiderschap' verstaan -- dat we
altijd binnen een natiestaat plaatsen -- wordt steeds meer ondernemend
in plaats van louter politiek. In zulke omstandigheden krimpt het
praktische keuzepalet voor het samenstellen van een `beleidsregime' voor
een rechtsgebied, vergelijkbaar met de beperkte opties waarover
ondernemers beschikken bij het ontwerpen van een eersteklas resorthotel
of een vergelijkbaar product of dienst, bepaald door wat mensen ervoor
willen betalen. Een resorthotel probeert bijvoorbeeld zelden te opereren
onder voorwaarden waarbij van de gasten wordt verlangd dat zij zwaar
lichamelijk werk verrichten om de faciliteiten te repareren en uit te
breiden. Zelfs een resorthotel dat eigendom is van of door zijn
werknemers wordt bestuurd -- vergelijkbaar met de typische moderne
democratie -- zal tevergeefs proberen klanten aan dergelijke eisen te
onderwerpen, vooral wanneer betere accommodaties beschikbaar komen. Als
klanten er de voorkeur aan geven te golfen in plaats van zwaar
lichamelijk werk te verrichten in de hete zon, biedt de markt op dat
vlak weinig ruimte om willekeurige alternatieven op te leggen. In
dergelijke situaties maken de huidige `politieke' kwesties plaats voor
ondernemende inzichten, terwijl rechtsgebieden actief zoeken naar
beleidsbundels die hun klanten aanspreken.

\subsection{Het verval van de
politiek}\label{het-verval-van-de-politiek}

Als dit eenmaal duidelijk is, past men de houdingen ingrijpend aan. In
rechtsgebieden waar bevoegdheden gedeeld worden, verwachten mensen niet
langer te moeten kiezen uit het aanbod aan wensvervullende beleidsopties
dat in de politieke debatten van de twintigste eeuw gangbaar was. Nu de
inkomens ongelijker verdeeld zijn dan in het industriële tijdperk,
stemmen rechtsgebieden steeds meer af op de behoeften van de klanten
wiens economische activiteiten de meeste waarde genereren en die de
grootste keuzevrijheid hebben bij hun bestedingen.

Onder zulke omstandigheden maakt het waarschijnlijk minder uit of
beleidsmaatregelen, die commercieel optimaal blijken voor een
rechtsgebied, ook aanslaan bij de `mediaankiezer' in een focusgroep.
Kortom, de commercialisering van soevereiniteit maakt het voor klanten
mogelijk om overheden effectiever te controleren. Hierdoor worden de
meningen van niet-klanten irrelevant -- of in ieder geval minder
belangrijk -- net zoals de opvattingen van \emph{`Big Mac'-eters} over
foie gras geen enkele invloed hebben op het succes van drie-sterren
Franse restaurants zoals L'Arpege in Parijs.

\section{`De verraad van de
democratie'}\label{de-verraad-van-de-democratie}

Net als de overleden Christopher Lasch klagen tegenstanders er niet
alleen op dat informatietechnologie banen vernietigt, maar stellen zij
ook dat deze technologie de democratie ondermijnt doordat mensen hun
middelen buiten het bereik van politieke dwang kunnen houden. Daarom
ervaren reactionairen in het nieuwe millennium de via
informatietechnologie mogelijk gemaakte financiële privacy als een
ernstige bedreiging. Zij zullen terugdeinzen bij de gedachte dat
inkomsten- en vermogensbelastingen daadwerkelijk op `vrijwillige
naleving' moeten berusten. Ook steunen zij nieuwe, zelfs ingrijpende
methoden om van iedereen die welvarend lijkt middelen af te rommelen,
zoals het invoeren van een `presumptieve belasting' en het letterlijk
gijzelen van vermogende personen.

\subsection{Gemeenschappelijk
eigendom}\label{gemeenschappelijk-eigendom}

Al tijdens het schrijven dringen er hints door over wat nog komen gaat.
De eerste aanwijzingen dat regeringen steeds minder in staat zijn
internationale markten te beheersen, stoten degenen tegen die ervan
uitgaan dat individuen van rechtswege als activa van de natiestaat
behoren. Zij eisen dan ook dat de staat haar middelen gebruikt om
burgers als waardevolle activa te beschouwen in plaats van als klanten.
Volgens de reactionairen moeten alle inkomens als opbrengsten van de
gemeenschap worden gezien, wat inhoudt dat deze volledig ter beschikking
van de staat zouden moeten staan.\footnote{Robert I Shapiro, `Volledig
  mis: nieuwe belastingregelingen kunnen niet tippen aan oude
  progressieve waarheden,' \emph{Washington Post}, 24 maart 1996, p.~C3,
  en Thomas L. Friedman, `Politiek in het tijdperk van NAFTA,' \emph{New
  York Times}, 7 april, p.~Eli.}

We hebben de argumenten van Lasch in \emph{Revolt of the Elites} en
\emph{Betrayal of Democracy} al besproken. Maar zijn tirade is niet de
enige stem ter ondersteuning van de natiestaat. De politicoloog van
\emph{Harvard University}, Michael Sandel, betoogt in \emph{Democracy in
Discontent} dat `democratie van vandaag niet mogelijk is zonder een
politiek die de mondiale economische krachten kan beheersen, want zonder
die controle maakt het niet uit op wie mensen stemmen, de bedrijven
zullen regeren.'\footnote{Geciteerd door Friedman, \emph{op. cit.}} Met
andere woorden moet de staat haar parasitaire macht over burgers
behouden om te waarborgen dat politieke uitkomsten kunnen afwijken van
marktuitkomsten.

Volgens ons is Sandels klaagzang, net als die van Lasch, slechts half
correct. We geven toe dat de democratie veel van haar waarde verliest
wanneer regeringen niet de macht hebben om individuen te verplichten
zich te gedragen zoals politici dat eisen. Dit is vanzelfsprekend.
Inderdaad, de democratie zoals wij die in de negentiende en twintigste
eeuw kenden, zal verdwijnen. Maar Sandel slaat de ware betekenis van de
triomf van de markten boven de dwang mis. Zijn pleidooi voor `corporate
rule', gekoppeld aan het verval van de natiestaat, komt opvallend
anachronistisch over.

Bedrijven zullen nauwelijks in staat zijn de markten van de nieuwe
wereldeconomie te beheersen. Inderdaad, zoals we al suggereerden, is het
allesbehalve zeker dat bedrijven zelfs in hun hedendaagse, moderne vorm
blijven bestaan. Integendeel, zij zullen vrijwel onvermijdelijk
transformeren in de megapolitieke revolutie die de komst van het
informatietijdperk markeert. Zoals we eerder bespraken, zal de opkomst
van microprocessering de `informatiekosten' doen veranderen, wat mede
bepaalt wat het `nexus van contracten' vormt dat bedrijven kenmerkt.
Volgens economen Michael C. Jensen en William H. Meckling vormen
bedrijven slechts één juridische structuur die `een nexus voor een
geheel van contractuele relaties tussen individuen' biedt.\footnote{Zie
  Louis Putterman en Randall S. Kroszner, `The Economic Nature of the
  Firm: A New Introduction,' in Louis Putterman en Randall S. Kroszner
  (red.), \emph{The Economic Nature of the Firm: A Reader} (Cambridge:
  Cambridge University Press, 1996), p.~17.}

Of een onderneming het in zijn geheel kan overleven -- om nog maar te
zwijgen over het besturen ervan als een domein van bureaucratische
sturing dat afgeschermd is van marktkrachten -- hangt waarschijnlijk af
van `de volledigheid van de marktkrachten en het vermogen van die
krachten om door te dringen in intrafirmale relaties', zoals economen
Louis Putterman en Randall S. Kroszncr opmerken.\footnote{Ibid.}

Zoals we eerder betoogden, twijfelen we eraan of bedrijven de toenemende
penetratie van marktkrachten in wat voorheen `intrafirmale relaties'
waren, kunnen doorstaan. Hiervoor zullen bedrijven waarschijnlijk
uiteenvallen, aangezien informatietechnologie het aantrekkelijker maakt
om taken via het prijmechanisme en de veilingsmarkt te laten uitvoeren
in plaats van deze intern te organiseren binnen een formele structuur.
Naarmate informatietechnologie het productieproces steeds verder
automatiseert, zal dit een deel wegnemen van de \emph{raison d'être} van
het bedrijf, namelijk de noodzaak om managers in te zetten en
individuele werknemers te controleren en te motiveren.

\subsection{``Waarom bestaan
bedrijven?''}\label{waarom-bestaan-bedrijven}

Bedenk dat de vraag `Waarom bestaan bedrijven?' veel ingewikkelder is
dan op het eerste gezicht lijkt. De micro-economie veronderstelt immers
dat het prijmechanisme het meest doeltreffende middel is om middelen zo
te coördineren dat ze optimaal worden benut. Volgens Putterman en
Kroszner houdt dit in dat organisaties, zoals bedrijven, geen
fundamentele economische bestaansreden bezitten.\footnote{Ibid., p.~9.}
Op deze manier ontstaan bedrijven in wezen als gevolg van informatie- en
transactiekosten, kosten die door informatietechnologie drastisch worden
teruggedrongen.

Hierdoor zal het informatietijdperk waarschijnlijk het tijdperk worden
van zelfstandige ondernemers zonder vaste banen, maar met duurzame
ondernemingen. Doordat technologie de transactiekosten verlaagt, kunnen
individuen zich losmaken van de zeggenschap van politici en voorkomen we
tegelijkertijd dat bedrijven de markt gaan domineren. Ondernemingen gaan
wedijveren met `virtuele ondernemingen' van over de hele wereld, wat
vrijwel alle gevestigde organisaties -- met enkele uitzonderingen --
flink zal doen schrikken. De meeste ondernemingen zullen als
organisaties gelukkig overleven in een omgeving met toenemende
concurrentie, naarmate de markten completer worden.

De verwachte uitkomst is zeker niet dat individuen hun lot in de handen
van ondernemingen leggen. Integendeel, ondernemingen hebben evenmin meer
invloed op de markt dan politici. Integendeel, mensen zullen eindelijk
de vrijheid hebben hun eigen koers te varen in een echte vrije markt,
die noch door grote overheden, noch door bedrijfsstructuren wordt
aangestuurd.

Deze daling van transactiekosten weerlegt bovendien de recent populaire
ideeën over `stakeholderkapitalisme'. Zulke opvattingen, dierbaar bij
Tony~Blair van de Britse Labour Party en enkele aanhangers van
Bill~Clinton, berusten op het idee dat de staat bedrijven zou kunnen
sturen. Nu het socialisme ingestort is, proberen interventionisten de
doelen van dat systeem te bereiken via efficiëntere marktmiddelen, door
bedrijven streng te reguleren. Volgens deze nieuwe, herverdelende
theorie behoren het management, de aandeelhouders, de werknemers en de
`gemeenschap' allen tot de stakeholders van een onderneming. Het betoog
luidt dat zij immers allemaal profiteren van langdurige
bedrijfsstructuren en zelfs afhankelijk zijn van die voordelen. Daarom
zou regelgeving de belangen moeten beschermen die managers, werknemers
en lokale belastingautoriteiten koesteren met betrekking tot hun
traditionele banden met ondernemingen.

`Stakeholderkapitalisme' is een doctrine die uiteindelijk niet alleen
uitgaat van de veronderstelling dat de staat de besluitvorming binnen
ondernemingen kan beïnvloeden, maar in de kern berust op de overtuiging
dat ondernemingen als duurzame organisaties zelfstandig kunnen
functioneren, ook zonder de prikkels van prijssignalen op de
veilingmarkt.

Wij vermoeden dat het verder ontwikkelen van markten niet alleen de
belastingcapaciteit van de natiestaat zal verlagen, maar ook de
mogelijkheid voor politici beperkt om via regelgeving hun wil arbitrair
op te leggen aan de eigenaren van middelen. In een wereld waarin
jurisdictievoordelen onderhevig raken aan markttesten en lokale markten
overal aan de internationale concurrentie blootstaan, is het nauwelijks
te verwachten dat lokale `gemeenschappen' doeltreffende methoden vinden
om bevoorrechte ondernemingen te beschermen tegen mondiale
concurrentiedruk. Zij zullen nauwelijks in staat zijn ervoor te zorgen
dat ondernemingen, die extra kosten maken -- bijvoorbeeld voor het in
dienst houden van overbodig personeel en management of voor het
openhouden van overtollige faciliteiten ter tegemoetkoming aan lokale
politieke eisen -- op een wijze worden belast die deze kosten dekt en
hen in bedrijf houdt.

Tijdens het industriële tijdperk konden politici de markten afsluiten en
de toetreding beperken tot enkele bevoorrechte ondernemingen om te
voldoen aan werkgelegenheids- en andere doelstellingen. In de toekomst,
wanneer informatie overal ter wereld vrij verhandelbaar is, zullen
overheden nauwelijks de macht hebben om lokale bedrijven af te schermen
van de mondiale concurrentiedruk.

Ook is het onwaarschijnlijk dat oproepen tot een `nieuw sociaal
contract', waarbij een zogenoemde onafhankelijke of vrijwillige sector
de periode opvangt waarin arbeiders anders werkloos of gemarginaliseerd
raken `in de gemeenschap', daadwerkelijk haalbaar blijken te
zijn.\footnote{Zie Jeremy Rifkin, \emph{The End of Work: The Decline of
  the Global Labor Force and the Dawn of the Post-Market Era} (New York:
  G.P. Putnam's Sons, 1995).} Jeremy Rifkin pleit voor een `nieuw
partnerschap tussen de overheid en de derde sector' om de sociale
economie opnieuw op te bouwen. \ldots{} Het doel is om de armen te
voeden, basisgezondheidszorg te bieden, de jeugd te onderwijzen,
betaalbare woningen te realiseren en het milieu te beschermen\ldots{}
\footnote{Ibid., p.~250.}

\subsection{De ondergang van openbare
goederen}\label{de-ondergang-van-openbare-goederen}

Voorstanders van dwang zullen ongetwijfeld stellen dat een afnemend
staatsgezag leidt tot een onvermogen om openbare goederen te leveren of
ervan te genieten. Maar dat lijkt om zowel competitieve als andere
redenen onwaarschijnlijk. Ten eerste verliezen rechtsgebieden -- nu
technologie plaatsgebonden voordelen grotendeels heeft weggenomen -- hun
klanten snel als ze essentiële openbare goederen, zoals het handhaven
van orde, niet kunnen aanbieden. In uiterste gevallen, zoals te zien was
in Somalië, Liberia, Rwanda en het voormalige Joegoslavië, zullen
waarschijnlijk hordes blutvluchtelingen de landsgrenzen overschrijden op
zoek naar een betrouwbaardere waarborging van wet en orde. Deze extreme
vormen van deserteeractie, of `stemmen met de voeten', wijken in
urgentie nauwelijks af van gewoon `rechtsgebiedswinkelen'. Hoe dan ook,
ondernemingen zullen lokale rechtsgebieden ertoe dwingen om aan de
behoeften van hun klanten te voldoen.

\subsection{Competitieve territoriale
clubs}\label{competitieve-territoriale-clubs}

Econoom Charles Tiebout verwoordde dit idee al in 1956 -- het gaat om
meer dan een louter theoretisch concept.\footnote{Zie Charles M.
  Tiebout, `A Pure Theory of Local Expenditure,' Journal of Political
  Economy 64 (1956), pp.~416--424.} Fred Foldvary documenteerde in
\emph{Public Goods and Private Communities: The Market Provision of
Social Services} dat er geen fundamentele reden bestaat om sociale
diensten en andere openbare goederen uitsluitend via politieke kanalen
te leveren. De voorbeelden die Foldvary aandraagt, bevestigen tevens de
controversiële stelling van de Nobelprijswinnende econoom Ronald Coase,
namelijk dat overheidsinterventie niet nodig is om externaliteiten,
zoals vervuilingsproblemen, op te lossen. Ondernemers leveren
collectieve goederen via marktmechanismen en velen hebben dat al in
praktijk gebracht binnen lokale gemeenschappen. Foldvary's casestudies
tonen aan dat het privatiseren van gemeenschappen kan leiden tot
innovatieve manieren om openbare goederen en diensten te leveren en te
financieren.

\subsection{De weg naar voorspoed}\label{de-weg-naar-voorspoed}

De opkomst van microtechnologie opent de deur naar nieuwe financierings-
en reguleringsmethoden voor goederen die tot voor kort als publieke
goederen werden beschouwd. Later blijkt dat sommige van deze goederen in
werkelijkheid private kenmerken hebben. Snelwegen vormen daar een
duidelijk voorbeeld van. Toen congestie nog geen groot probleem was,
behandelde men wegen en snelwegen als publieke goederen. Adam Smith
bekritiseerde dit namelijk omdat hij vond dat deze infrastructuur een
onevenredig groot voordeel biedt aan omwonenden, terwijl bewoners van
afgelegen gebieden gedwongen worden te betalen zonder er veel van te
profiteren.

Dankzij technologische ontwikkelingen kunnen we in het
informatietijdperk tolheffing -- inclusief congestieheffingen --
invoeren die de toegang tot snelwegen, start- en landingsbanen en andere
infrastructuur nauwkeurig prijzen, zonder de verkeersstroom te
verstoren. Op die manier privatiseren we de levering van
transportinfrastructuur discreet en laat de financiering ervan direct
over aan de gebruikers. Volgens econoom Paul Krugman voegt een
marktgerichte prijsstelling van de Amerikaanse transportinfrastructuur,
naar schatting, jaarlijks tussen de 60 en 100 miljard dollar toe aan de
CIDP in de Verenigde Staten, terwijl het de efficiëntie in het gebruik
van hulpbronnen verbetert en de vervuiling vermindert.\footnote{Paul R.
  Krugman, `The Tax-Reform Obsession,' \emph{New York Times Magazine}, 7
  april 1996, p.~37.}

Verder mogen we niet vergeten dat het duurste onderdeel van wat moderne
natiestaten doen -- namelijk inkomensherverdeling -- in feite niet
draait om het leveren van een openbaar goed, maar om het verstrekken van
private goederen, en dat ten koste van de belastingbetalers. Met
`publieke kost' bedoelen we immers `ten koste van de belastingbetalers'.

Hoe zit het dan met een écht openbaar goed, zoals het onderhouden van
een militaire macht die een aanval door een grootmacht kan afschrikken?
Zo'n macht is nu eenmaal traditioneel kostbaar. Zoals we eerder al
aangaven: een regering zonder onbeperkte bevoegdheid om de inkomens en
eigendommen van haar burgers in te nemen, kan nooit een volgende oorlog
tussen grootmachten -- vergelijkbaar met de Tweede Wereldoorlog --
financieren.

Toch vormt deze fiscale beperking minder een bedreiging dan de
reactionairen doen vermoeden, simpelweg omdat toekomstige conflicten
niet meer zullen plaatsvinden zoals in de Tweede Wereldoorlog. De
technologische doorbraak die individuen bevrijdt, zal daarvoor zorgen.

Los van de politiek

In plaats van de kwaliteit en aard van zulke diensten over te laten aan
de grillen van de politiek, kun je `regeringen' op een ondernemende
wijze besturen en omvormen tot wat Foldvary omschrijft als
`concurrerende territoriale clubs.' Wij vermoeden dat uiteindelijk de
manier waarop deze clubs hun besluitvorming inrichten minder belangrijk
zal blijken dan hun succes in de marktproeven op prestatiegebied.
Tegenwoordig maakt het voor de meeste consumenten nauwelijks uit of een
bedrijf dat een product of dienst levert een eenmanszaak is, een
besloten vennootschap of een onderneming die door externe bestuurders --
voorgedragen door pensioenfondsen -- wordt geleid. Wij twijfelen er
bovendien niet aan dat de rationele afnemer van soevereiniteitsdiensten
in dit informatietijdperk zich niet druk zal maken over de vraag of
Singapore een massademocratie vormt of het particuliere bezit is van Lee
Kwan Yew.

De inhoud van het te vertalen boek is:

\bookmarksetup{startatroot}

\chapter{Moraliteit en misdaad in de `natuurlijke economie' van het
informatietijdperk}\label{moraliteit-en-misdaad-in-de-natuurlijke-economie-van-het-informatietijdperk}

\begin{quote}
``Corruptie\ldots{} is veel verder verspreid en universeel dan voorheen
werd gedacht. Bewijs daarvan is overal te vinden, zowel in
ontwikkelingslanden als, steeds vaker, in geïndustrialiseerde landen. .
. . Prominente politieke figuren, waaronder staatshoofden en ministers,
zijn beschuldigd van corruptie. . . . Op een bepaalde manier betekent
dit een privatisering van de staat, waarbij haar macht niet naar de
markt verschuift, zoals privatisering normaliter impliceert, maar naar
overheidsfunctionarissen en bureaucraten.''\footnote{Vito Tanzi,
  `Corruption: Arm's-length Relationships and Markets,' in Gianluca
  Fiorentini en Sam Peltzman, eds., \emph{The Economics of Organized
  Crime} (Cambridge: Cambridge University Press, 1995), pp.167, 170.} --
VIRO TANZI
\end{quote}

Wij menen dat naarmate de moderne natiestaat uiteenvalt, hedendaagse
barbariërs steeds meer invloed achter de schermen gaan uitoefenen.
Groepen zoals de Russische maffia's, die de botten van de voormalige
Sovjet-Unie oprapen, andere etnische misdaadbendes, nomenklaturen,
drugskartels en afvallige geheime diensten hanteren steeds vaker hun
eigen regels. Dat doen ze nu al.

Veel meer dan men doorgaans denkt, hebben moderne barbariërs de
contouren van de natiestaat al geïnfiltreerd, zonder dat haar uiterlijke
verschijning wezenlijk verandert. Het gaat om microparasieten die zich
tegoed doen aan een stervend systeem. Net zo gewelddadig en meedogenloos
als een staat in oorlog zetten deze groepen staatsmethoden op een
kleinere schaal in. Hun groeiende invloed en macht illustreren de
verkleining van de politiek. Efficiënte, kleinschalige
organisatiemethoden verminderen namelijk de omvang die nodig is om
effectief geweld te gebruiken en te beheersen. Naarmate deze
technologische revolutie vordert, groeit de neiging om roofzuchtig
geweld buiten centrale sturing te organiseren. Pogingen om geweld in te
dammen decentraliseren bovendien zodanig dat zij meer leunen op
efficiëntie dan op pure machtsschatten.

De toename van clandestiene criminele activiteiten en corruptie binnen
natiestaten vormt een belangrijke rode draad te midden van de
wereldwijde veranderingen. Wat je zult aanschouwen, lijkt op een
duistere, geheime versie van een slechte film, \emph{Invasion of the
Body Snatchers}. Nog voordat de meeste natiestaten zichtbaar instorten,
zullen hedendaagse barbariërs de dienst uitmaken. Zoals in de beroemde
B-film uit de jaren vijftig treden zij vaak in vermomming op. De `Pod
People' van de toekomst blijken echter geen buitenaardse wezens, maar
criminelen met uiteenlopende achtergronden die officiële functies
bekleden, terwijl hun loyaliteit ten minste gedeeltelijk buiten de
constitutionele orde valt.

Het einde van een tijdperk valt vaak samen met een periode van intense
corruptie. Wanneer de fundamenten van het oude systeem afbrokkelen,
vervaagt ook de sociale ethos, waardoor een klimaat ontstaat waarin
invloedrijke personen publieke doelen met privé criminele activiteiten
vermengen.

Helaas mag je niet altijd rekenen op de reguliere informatiekanalen voor
een nauwkeurig en actueel beeld van de ondergang van de natiestaat. Dit
`aanhoudende verzinsel' -- van het type dat ooit de val van het Romeinse
Rijk verhulde -- blijkt een typisch kenmerk te zijn van de ontbinding
van grote politieke entiteiten. Tegenwoordig verhult en maskeert het de
teloorgang van de natiestaat. Om verschillende redenen kun je de
nieuwsmedia niet altijd vertrouwen om je de waarheid te vertellen. Velen
koesteren een conservatieve visie en vertegenwoordigen daarmee de
nostalgie van het verleden. Sommigen raken verblind door verouderde
ideologische toewijding aan socialisme en de natiestaat. Anderen
schuwen, om meer concrete redenen, om de groeiende corruptie in een
vervalachtig systeem bloot te leggen. Weer anderen missen simpelweg de
fysieke moed die zo'n taak vergt. Daarnaast vrezen sommigen om hun baan
of schromen om zich uit te spreken, omdat ze bang zijn voor
vergeldingsmaatregelen. Uiteraard is er geen reden te vermoeden dat
verslaggevers en redacteuren minder vatbaar zijn voor corruptieve
prikkels dan bouwinspecteurs of Italiaanse bestratingsaannemers. Veel
vaker dan je zou verwachten, blijken belangrijke informatiebronnen --
die zogenaamd alles willen rapporteren -- minder betrouwbaar te zijn dan
men algemeen gelooft. Velen laten hun eigen belangen, zoals het
versterken van steun voor een wankelend systeem, prevaleren boven het
eerlijk informeren van jou. Ze delen weinig mee en geven nog minder
uitleg.

\section{Voorbij de realiteit}\label{voorbij-de-realiteit}

Naarmate de technologieën voor kunstmatige realiteit en computerspellen
blijven verbeteren, zul je zelfs in staat zijn om een avondnieuwsbericht
te bestellen dat exact het nieuws nabootst dat je wilt horen. Wil je
bijvoorbeeld een verslag waarin jij als winnaar van de tienkamp op de
Olympische Spelen centraal staat? Geen probleem. Het kan moeiteloos het
hoofdnieuws van morgen vormen. Je zult elk verhaal dat je verlangt -- of
het nu op waarheid berust of volledig verzonnen is -- op je televisie of
computer zien ontvouwen met een overtuigende realiteitsgetrouwheid die
veel verder gaat dan wat \emph{NBC} of \emph{BBC} momenteel kunnen
leveren.

We bewegen ons in een razendsnel tempo naar een wereld waarin informatie
net zo ongebonden zal zijn van de beperkingen van de werkelijkheid als
dat menselijke vindingrijkheid het mogelijk maakt. Dit zal ongetwijfeld
enorme gevolgen hebben voor zowel de kwaliteit als de aard van de
informatie die je ontvangt. In een wereld waarin kunstmatige realiteit
en de onmiddellijke overdracht van alles en overal samenkomen, wordt het
essentieel om het vermogen te ontwikkelen om het ware van het valse te
onderscheiden en integer te blijven oordelen.

Maar het betekent minder een verandering in onze huidige situatie dan
velen verwachten. De grens tussen waar en onwaar vervaagt vaak door
invloeden die door de technologie nog worden versterkt. Binnen de
wetenschap stellen we dat veel gevolgen van de informatie-revolutie
juist bevrijdend zijn.

Technologie overschrijdt inmiddels grenzen van geografische nabijheid en
politieke overheersing. Overheden mogen wel barrières opwerpen om de
handel in goederen te verstoren, maar zij hebben weinig macht over de
overdracht van informatie. Bijna iedere gast in elk restaurant in Hong
Kong is via een mobiele telefoon verbonden met de hele wereld. De harde
staatsgreepcomplotters in Moskou in augustus 1991 slaagden er niet in
Jeltsins communicatielijnen tot stilstand te brengen, omdat hij
beschikte over mobiele telefoons.

\subsection{Meer informatie, minder
begrip}\label{meer-informatie-minder-begrip}

Nu de obstakels voor het verspreiden van informatie verdwenen zijn,
hebben we toegang tot meer informatie -- wat positief is. Tegelijkertijd
zorgt dat ook voor meer onduidelijkheid over de betekenis ervan. De
moderne technologie die informatie losmaakt van politieke controle en de
beperkingen van tijd en plaats, maakt traditionele oordeelsvorming
bovendien waardevoller. Het inzicht dat ons helpt onderscheiden wat
belangrijk en waar is te midden van de overvloed aan feiten en
fantasieën, wordt vrijwel dagelijks kostbaarder. Dit geldt om ten minste
drie redenen:

\begin{enumerate}
\def\labelenumi{\arabic{enumi}.}
\tightlist
\item
  De enorme hoeveelheid beschikbare informatie vraagt om bondigheid.
\end{enumerate}

Beknoptheid leidt tot afkortingen. Afkortingen slaan vaak cruciale
details over. Als je met talloze feiten te verwerken bent en daarnaast
nog ontelbare telefoontjes moet terugbellen, verlang je er van nature
naar om elk moment van informatieoverdracht zo beknopt mogelijk te
maken. Helaas levert verkorte informatie vaak slechts een zwakke basis
op voor diepgaand begrip. De diepere en rijkere lagen van de
geschiedenis vallen namelijk in de vijfentwintig seconden durende
soundbites vaak weg, terwijl \emph{CNN} ze regelmatig verkeerd
weergeeft. Het overbrengen van een boodschap als variatie op een al
bekend thema gaat veel eenvoudiger dan het verkennen van een geheel
nieuw paradigma van begrip. Je kunt een honkbal- of cricketscore veel
gemakkelijker melden dan uitleggen hoe deze sporten precies worden
gespeeld en wat ze betekenen.

\begin{enumerate}
\def\labelenumi{\arabic{enumi}.}
\setcounter{enumi}{1}
\item
  De razendsnel veranderende technologie ondermijnt de megapolitieke
  fundamenten van onze sociale en economische organisatie. Hierdoor
  verouderen brede paradigma-inzichten -- onuitgesproken theorieën over
  hoe de wereld functioneert -- sneller dan ooit tevoren. Dit vergroot
  de behoefte aan een overkoepelend overzicht en ondermijnt de waarde
  van individuele `feiten' die vrijwel iedereen direct kan opzoeken met
  een zoekmachine.
\item
  De groeiende tribalisering en marginalisering in onze samenleving
  remmen zowel het publieke debat als ons denkvermogen. Veel mensen
  raken er zo aan gewend dat ze conclusies mijden, zelfs als de feiten
  daar ondubbelzinnig voor spreken. Een recent psychologisch onderzoek,
  vermomd als een opiniepeiling, liet zien dat leden van afzonderlijke
  beroepsgroepen vrijwel unaniem geen enkele conclusie accepteerden die
  voor hen tot inkomensverlies zou leiden, hoe sluitend de onderliggende
  logica ook was. Door de toegenomen specialisatie richt de
  interpretatieve informatie over deze beroepsgroepen zich vooral op het
  behartigen van hun belangen. Zij tonen weinig interesse in opvattingen
  die als onbeschoft, onrendabel of politiek incorrect worden ervaren.
  Een treffend voorbeeld van deze neiging is de voortdurende herhaling
  van rooskleurige vooruitzichten voor aandeleninvesteringen. De meeste
  voorspellingen hierover komen van effectenmakelaars, die zelden
  toegeven dat aandelen overgewaardeerd zijn, omdat hun inkomen
  afhankelijk is van transacties waarbij de meerderheid van hun klanten
  koopt. Onafhankelijke, afwijkende stemmen hoor je zelden.
\end{enumerate}

Om deze en andere redenen is het informatietijdperk nog lang geen
tijdperk van begrip geworden. Integendeel, het publieke debat is
aanzienlijk minder streng. Hoewel de wereld tegenwoordig meer informatie
in huis heeft dan ooit tevoren, ontbreekt er vrijwel een publieke stem
die de betekenis van gebeurtenissen kan beoordelen en kan vaststellen
wat waar is. Daarom hebben wij de lauwe belangstelling -- met name in de
Amerikaanse media -- voor het melden van aanwijzingen van
sensatiebeluste corruptie op hoog niveau binnen de Amerikaanse overheid
altijd geboeid.

Een centraal thema waar we in dit boek mee hebben geworsteld, is hoe
veranderende technologie en andere `megapolitieke' factoren de
`natuurlijke economie' transformeren. Met de `natuurlijke economie'
bedoelen we de darwinistische toestand van de natuur, waarin uitkomsten
-- vaak oneerlijk -- bepaald worden door fysieke kracht. In die context
noemen biologen een belangrijke gedragsvorm `interferentieconcurrentie'.

\subsection{Interferentieconcurrentie}\label{interferentieconcurrentie}

«Interference competitors», zoals Jack Hirshleifer het verwoordde,
winnen en behouden de controle over middelen door hun rivalen direct af
te weren of te hinderen.\footnote{Hirshleifer, op. cit., p.176.} Hoezeer
we er ook naar verlangen dat menselijk gedrag altijd geleid wordt door
de rechtsstaat en andere maatschappelijk opgelegde spelregels (ook wel
`politieke economie' genoemd), het bewijs is overweldigend dat veel
mensen `naar de regels spelen' slechts wanneer het hen uitkomt.
Hirshleifer, een autoriteit op het gebied van conflicten, merkte op:
«{[}D{]}e volharding van criminaliteit, oorlog en politiek leert ons dat
de feitelijke menselijke aangelegenheden nog steeds grotendeels
onderhevig zijn aan de drukken vanuit de natuurlijke
economie.»\footnote{Ibid., p.~169.}

Met andere woorden: het vreedzame en wettige handelen van de Homo
economicus -- zoals in de leerboeken wordt beschreven, waarbij
eigendomsrechten worden gerespecteerd en men niet zomaar neemt wat niet
van hem is -- bepaalt slechts een deel van de economische uitkomsten.
Conflicten, waaronder openlijk geweld, spelen eveneens een belangrijke
rol. Zoals econoom Hirshleifer opmerkt: «Zelfs onder gezag van wet en
regering vindt het rationele, eigenbelanghebbende individu een afweging
tussen legale en illegale middelen om zijn doelen te bereiken -- tussen
productie en ruil aan de ene kant en diefstal, fraude en afpersing aan
de andere kant.»\footnote{Hirshlcifer, op. cit., p.~173.}

\section{Overval in het
informatietijdperk}\label{overval-in-het-informatietijdperk}

In een waardevol boek over geweld, misdaad en politiek, \emph{The
Political Economy of Conflict and Appropriation}, stellen Michelle R.
Garfinkel en Stergios Skaperdas dat individuen en groepen óf rijkdom
creëren door te produceren óf de door anderen opgebouwde welvaart in
beslag nemen.\footnote{Garfinkel en Skaperdas, op. cit., p.~1.} Zij
citeren een verhaal over moderne interferentieconcurrentie,
oorspronkelijk gerapporteerd door \emph{The Economist}: «Een Amerikaanse
zakenman, die onlangs in Moskou was aangekomen om een kantoor te openen,
werd bij zijn hotel opgewacht door vijf mannen met gouden horloges,
pistolen en een afdruk van de nettowaarde van zijn bedrijf. Zij eisten
7\% van toekomstige inkomsten. Hij nam de eerstvolgende vlucht naar New
York, waar overvallers minder geraffineerd zijn.»\footnote{Ibid.} Dit
verhaal laat zien dat overvallen in het informatietijdperk vooral te
danken zijn aan de mogelijkheden die nieuwe technologie biedt, en niet
louter het feit dat criminelen in Rusland via het internet toegang
hebben tot financiële profielen en kredietrapporten van hun
slachtoffers.

\subsection{Dalende beslissendheid van militaire
macht}\label{dalende-beslissendheid-van-militaire-macht}

In zowel positieve als negatieve zin verzwakt informatietechnologie de
mogelijkheid van de natiestaat om haar gezag in een onstabiele wereld op
te leggen, doordat grootschalige militaire macht minder doorslaggevend
is. Waar men ooit, zoals Voltaire vermaande, geloofde dat `God aan de
kant van grotere bataljons' stond, lijkt de goddelijke steun voor
grootschalig geweld met de dag af te nemen. Integendeel, alle signalen
wijzen op dalende opbrengsten van geweld, wat erop duidt dat grote
entiteiten zoals de natiestaat hun enorme vaste kosten niet langer
kunnen rechtvaardigen.

Het meest voor de hand liggende bewijs voor de afnemende effectiviteit
van gecentraliseerde macht is de opkomst van terrorisme. De
grootschalige bomaanslagen in de Verenigde Staten midden in de jaren
negentig tonen aan dat zelfs de militaire supermacht van de wereld niet
immuun is voor aanvallen.

Een andere duidelijke uiting van de dalende winstgevendheid van geweld
blijkt uit de wereldwijde toename van gangsterisme en georganiseerde
misdaad, vergezeld van politieke vriendjespolitiek en corruptie. Deze
ontwikkelingen karakteriseren een overwegend amorale sfeer waarin de
staat weliswaar kan dwingen, maar weinig in staat is bescherming te
bieden. Nu haar geweldsmonopolie afbrokkelt, dringen nieuwe spelers zich
op -- denk maar aan de pestkoppen die geprobeerd hebben particuliere
belastingen op te leggen aan de Amerikaanse zakenman in Moskou.

Kleine groeperingen, stammen, triaden, bendes, gangsters, mafia's,
milities en zelfs individuele strijders hebben hun militaire
effectiviteit versterkt. In de `natuurlijke economie' van het komende
millennium zullen zij waarschijnlijk meer reële macht uitoefenen dan in
de twintigste eeuw. Wapens die microchips inzetten, verschuiven de
machtsbalans naar de verdediging, waardoor beslissende agressie minder
winstgevend en daarmee minder waarschijnlijk wordt. Slimme wapens, zoals
Stinger-raketten, neutraliseren een groot deel van het voordeel dat
grote, welvarende staten vroeger haalden met dure luchtmacht om armere,
kleinere groepen aan te vallen.

\subsection{Informatieoorlog in het
verschiet}\label{informatieoorlog-in-het-verschiet}

Wat aan de horizon opdoemt, is de veelbesproken maar nauwelijks begrepen
mogelijkheid van een `informatieoorlog'. Dit duidt tevens op de
afnemende meeropbrengst van traditioneel geweld. De zogenaamde `logische
bommen' kunnen onder meer luchtverkeersleidingssystemen,
wisselmechanismen op spoorwegen, stroomgeneratoren en
distributienetwerken, water- en rioleringssystemen, telefonische relais
en zelfs de communicatiesystemen van het leger ontregelen of saboteren.
Nu samenlevingen steeds meer leunen op geautomatiseerde
besturingssystemen, veroorzaken `logische bommen' bijna evenveel schade
als fysieke explosies.

In tegenstelling tot conventionele bommen brengen vijandige overheden,
groepen freelance programmeurs en zelfs begaafde individuele hackers
`logische bommen' op afstand tot ontploffing. Neem bijvoorbeeld dat in
1996 de autoriteiten een Argentijnse tiener arresteerden omdat hij
herhaaldelijk Pentagoncomputers hackte. Hoewel hackers tot nu toe zelden
kozen voor het destructief saboteren van computergestuurde systemen, is
het gebrek aan effectieve tegenmaatregelen geen reden voor hun
terughoudendheid.

Als het tijdperk van de informatieoorlog eindelijk aanbreekt, zullen de
tegenstanders waarschijnlijk niet uitsluitend overheden betreffen. Een
bedrijf als \emph{Microsoft} beschikt immers over een grotere capaciteit
om een informatieoorlog te voeren dan 90 procent van de natiestaten
wereldwijd.

\subsection{De tijd van het soevereine
individu}\label{de-tijd-van-het-soevereine-individu}

Dat vormt voor een deel de reden waarom we ons boek \emph{het soevereine
individu} zo hebben genoemd. Naarmate de schaal van oorlogsvoering
afneemt, regelen we defensie en bescherming op een kleinere, meer lokale
schaal. Daardoor biedt men deze diensten steeds vaker als particuliere
in plaats van publieke goederen aan, waarbij particuliere aannemers op
winst nastreven. Dat zie je al duidelijk terug in de privatisering van
de politiezorg in Noord-Amerika. In de Verenigde Staten behoort het
beroep van `beveiliger' tot de snelst groeiende beroepen. Prognoses
voorspellen dat het aantal particuliere beveiligers in 2005 met zo'n 24
tot 40 procent zal toenemen ten opzichte van 1990.\footnote{Hamish
  McRae, \emph{The World in 2020} (London: \emph{Harper Collins}, 1995),
  p.~188.}

De privatisering van politietaken vormt inmiddels al een duidelijke
trend. Toch, zoals de Anglo-Ierse guru Hamish McRae opmerkt, komt dit
nauwelijks voort uit doelbewuste overheidsbeslissingen. In \emph{The
World in 2020} schrijft hij het volgende:

\begin{quote}
Geen enkele overheid heeft bewust besloten zich terug te trekken uit
bepaalde politietaken, noch hebben overheden dat daadwerkelijk gedaan;
in plaats daarvan nam de private sector hun plaats in. Zowel vanwege de
vermeende tekortkomingen bij de politie als door andere maatschappelijke
veranderingen hebben particuliere beveiligingsbedrijven in de loop der
jaren geleidelijk het merendeel van de taak overgenomen om burgers in
kantoren en winkelcentra te beschermen. Zoals de ommuurde woonwijken in
Los Angeles aantonen, keert men zelfs enigszins terug naar het
middeleeuwse stadsconcept, waarbij burgers achter stadsmuren leven die
door bewakers patrouilleren en waarvan de toegang uitsluitend via
gecontroleerde poorten verloopt.\footnote{Ibid., pp.~188--89.}
\end{quote}

Wij zijn ervan overtuigd dat dit nog maar het voorproefje is van een
veel bredere privatisering van bijna alle functies die de overheden in
de twintigste eeuw vervulden. Informatietechnologie heeft immers de
capaciteit van gecentraliseerde autoriteiten om macht uit te oefenen en
fysieke veiligheid te waarborgen in grootschalige systemen ernstig
ondermijnd, waardoor de optimale omvang van vrijwel elke onderneming in
de `natuurlijke economie' afneemt.

Om in te spelen op deze technologische verschuiving is een enorme
investering nodig (lees: een kans) om kwetsbare systemen opnieuw in te
richten met gedistribueerde in plaats van geconcentreerde capaciteiten.
Als men de grootschalige kwetsbaarheden niet verhelpt, raken die
systemen vatbaar voor catastrofale storingen.

Vroeg of laat -- bijna onvermijdelijk, al gebeurt het niet opzettelijk
-- ontwikkelen de diensten en producten die grote bureaucratische
instanties en bedrijven leveren zich tot uiterst concurrerende markten.
In plaats van vanuit één centraal hoofdkantoor te werken, organiseren
zij hun activiteiten via een gedistribueerd, decentraal netwerk.

Bedrijven met een hoofdkantoor dat makkelijk omsingeld kan worden door
betogers of gesaboteerd door terroristen lopen het risico kwetsbaar te
zijn totdat zij overstappen op een `virtuele onderneming' zonder vaste
locatie -- oftewel, een organisatie die `tegelijkertijd op tal van
plaatsen aanwezig is', zoals Kevin Kelly, uitvoerend redacteur van
\emph{Wired} magazine, in \emph{Out Of Control} beschrijft.

Kelly wijst erop dat technologische ontwikkelingen de behoefte
verminderen om productieprocessen centraal te organiseren. Tijdens het
grootste deel van de industriële revolutie vergaarde men aanzienlijke
rijkdom doordat alle productie onder één dak werd geconcentreerd -- men
dacht destijds dat `groter efficiënter is.' Tegenwoordig geldt dat
principe niet meer.

Kelly voorspelt dat een auto van de toekomst -- de Upstart Car --
ontworpen en geproduceerd kan worden door slechts een dozijn mensen die
via een virtueel netwerk samenwerken.

In de toekomst kan een te grote schaal niet alleen averechts werken,
maar ook gevaarlijk blijken. Grotere ondernemingen vormen immers
aantrekkelijkere doelwitten. Beoefenaars van de ondergrondse economie
tonen aan dat het vermijden van opsporing een van de geheimen is om
belastingdruk te ontduiken. Dit gelingt kleinschalige, `virtuele
ondernemingen' veel gemakkelijker dan bij gevestigde bedrijven die
vanuit een imposante wolkenkrabber opereren met hun naam in lichten.
Zulke ondernemingen trekken onvermijdelijk de aandacht van `mannen met
gouden horloges, pistolen en een print-out van de nettowaarde van het
bedrijf' -- gangsters die, bijvoorbeeld in Rusland, hun eigen
particuliere vorm van belasting heffen. Bedrijven van elke omvang lopen
daardoor risico op criminele afpersing en extra lasten opgelegd door
georganiseerde misdaadbendes.

\begin{quote}
`Als we iemand een racketeer noemen, bedoelen we iemand die eerst een
dreiging opzet en vervolgens kosten rekent om die weg te nemen. De
bescherming die overheden bieden, komt volgens deze definitie vaak neer
op afpersing.'\footnote{Tilly, \emph{War Making and State Making as
  Organized Crime}, in Evans, Rueschemeyer en Skoepol (op. cit.),
  p.~171.} - CHARLES TILLY
\end{quote}

\subsection{Natuur haat monopolies}\label{natuur-haat-monopolies}

Naarmate het geweldmonopolie waarover de `grotere bataljons' beschikken
verder afbrokkelt, mogen we als eerste gevolg verwachten dat de
georganiseerde misdaad floreert. Immers, georganiseerde misdaad
concurreert met natiestaten als het gaat om het inzetten van geweld voor
roofzuchtige doeleinden. Hoe onbeleefd dat ook klinkt, maar zoals
politicoloog Charles Tilly ons eraan herinnert, kwalificeren regeringen
-- het toonbeeld van beschermingsafpersing met het voordeel van
legitimiteit -- zich eigenlijk als onze grootste voorbeelden van
georganiseerde misdaad.

Als je verder niets wist van de wereld behalve dat een belangrijk
monopolie instortte, kun je er vrijwel zeker van zijn dat de directe
concurrenten het meeste zullen profiteren. Het is dan ook geen toeval
dat drugskartels, criminele bendes, mafia's en triaden wereldwijd in
opmars zijn.

\subsection{Sistema del potere}\label{sistema-del-potere}

Van Rusland tot Japan en de Verenigde Staten beïnvloedt georganiseerde
misdaad economieën veel meer dan economische leerboeken doen vermoeden.
Wat de Sicilianen `sistema del potere' noemen -- oftewel het
machtssysteem van de georganiseerde misdaad -- speelt een steeds
belangrijkere rol in het functioneren van economieën.

Europese politiefunctionarissen melden dat internationale
misdaadsyndicaten, waaronder Russische en Italiaanse mafia's, een
`dominante rol' hebben gespeeld bij het financieren van de genocidale
oorlogen die de Balkan de afgelopen jaren teisterden.

Ook drugssmokkelaars hebben een cruciale rol gespeeld in het financieren
van recente burgeroorlogen en opstanden in andere delen van de wereld.
Julio Fernandez, chef van de drugseenheid van de Spaanse nationale
politie in Catalonië, zegt: `Van 1986 tot 1988 vervoerden Tamil
Tiger-guerrilla's, in samenwerking met Pakistaanse inwoners in Barcelona
en Madrid, 80 procent van de heroïne in Spanje. Zodra we dat netwerk
door arrestaties hadden ontmanteld, namen Koerden uit Turkije het over
en domineerden zij volledig gedurende de daaropvolgende twee
jaar.'\footnote{Frank Viamo, `De nieuwe maffiaorde,' \emph{Mother
  Jones}, mei/juni 1995, p.~55.} Het is waarschijnlijk dat wanneer een
nieuwe burgeroorlog of opstand uitbreekt, wanhopig arme strijders hun
militaire inzet financieren door drugs te verhandelen en drugsgeld wit
te wassen.

\subsection{Door drugs gefinancierde
prijsverlaging}\label{door-drugs-gefinancierde-prijsverlaging}

Georganiseerde criminele syndicaten oefenen een neerwaartse druk uit op
de prijzen van goederen -- met uitzondering van drugs. Op microniveau
financieren deze groepen schijnbaar legitieme ondernemingen met de buit
uit hun illegale activiteiten. Ze witwassen drugswinst en ander onwettig
verkregen gelden door reguliere producten onder de kostprijs te
verkopen, waardoor ze de prijzen van hun eerlijke concurrenten
ondermijnen en talrijke bedrijven in de problemen brengen.

\subsection{Yakuza-deflatie}\label{yakuza-deflatie}

In Japan vervulden invloedrijke Yakuza-bendes een sleutelrol tijdens de
hyperactieve vastgoedzeepbel van de late jaren tachtig. Hoewel
negentigduizend Yakuza jaarlijks tussen de \$10,19 miljard (volgens
officiële schattingen) en \$71,35 miljard (volgens professor Takatsugu
Nato) genereren, gingen Japanse banken voor deals die door de Yakuza
werden gesteund vaak oninbare leningen aan, hetgeen hun solvabiliteit in
gevaar bracht.\footnote{Zie Velisarios Kattoulas, `Japans yakuza nemen
  hun plaats in de criminele elite,' \emph{Washington Times}, 25
  november 1994, p.~A22.} Daardoor ervaart de Japanse economie een
deflatoire druk -- `prijsvernietiging', zoals de Japanners het noemen --
die kenmerkend is voor die periode.

\subsection{Een blind oog}\label{een-blind-oog}

De Russische maffia's, zoals Yeltsin zelf al toegaf, zijn versmolten met
`commerciële organisaties, administratieve instanties, organen van het
ministerie van Binnenlandse Zaken en gemeentelijke overheden'
\footnote{Viamo, op. cit., p.~49.}. Dankzij de immuniteit die zij zich
hadden veiliggesteld door banden met de politie aan te gaan, kunnen zij
met flagrant geweld hun particuliere belastingen afdwingen. Betrouwbare
bronnen stellen dat tegenwoordig vier van de vijf Russische bedrijven
afpersingsgeld betalen. `Volgens sommige rapporten moeten lokale kleine
bedrijven in Rusland 30 tot 50 procent van hun winst afstaan aan
racketeers, en niet slechts de magere 7 procent die van de Amerikaanse
zakenman wordt geëist.'\footnote{Garfinkel en Skaperdas, op. cit., p.~2.}

In 1993 classificeerde men in Rusland officieel 355.500 misdrijven als
afpersing, waaronder bijna 30.000 voorbedachte moorden -- grotendeels
huurmoorden als gevolg van conflicten in de commerciële en financiële
sector. Volgens voormalig minister van Binnenlandse Zaken, generaal
Viktor Yerin, betrof het merendeel huurmoorden. Meestal sloegen de
autoriteiten de ogen dicht. Criminele organisaties, die dankzij hun
beheersing van dwang en corruptie opereren, spelen een sleutelrol in de
economie, zoals economen Gianluca Fiorentini en Sam Peltzman schrijven
in \emph{The Economics of Organized Crime}.\footnote{Fiorentini en
  Peltzman, op. cit., p.~15.} Theoretisch kan hun invloed soms zelfs
positief uitpakken, doordat zij de regelgeving temperen en overheden
aansporen beter publieke goederen te leveren. Een machtige maffia houdt
de monopolistische rol van overheidsinstanties in toom.\footnote{Ibid.}
In gebieden waar georganiseerde misdaadgroepen sterk vertegenwoordigd
zijn, slagen overheden er vaak niet in zelfstandig beleid te voeren,
doordat de maffia hen tegenspreekt.

\subsection{Collusie}\label{collusie}

Het valt op dat overheden de maffia's -- hun voornaamste concurrenten op
het gebied van georganiseerde dwang -- maar al zelden direct aanpakken.
Uit strikt economisch oogpunt is dat niet verrassend. Voor `de gekozen
leden van het openbaar bestuur' blijkt de meest winstgevende regeling
een `collusieakkoord' met de georganiseerde misdaad te zijn. Fiorentini
en Peltzman merken op dat `er aanwijzingen zijn voor grootschalige
overeenkomsten waarbij de georganiseerde misdaad politieke steun
waarborgt voor groepen kandidaten, terwijl laatstgenoemden de gunst
beantwoorden door een gunstig beheer van overheidsopdrachten en de
levering van overheidsdiensten of subsidies.'

In tegenstelling tot de indruk die Hollywood wekt, richten criminele
organisaties zoals de Siciliaanse maffia zich tegenwoordig op het
binnendringen in en oplichten van overheden. `De meeste geleerden menen
dat de kernactiviteit van de Siciliaanse maffia er tegenwoordig precies
in bestaat dat zij zich toe-eigenen van diverse bronnen van
overheidsuitgaven en fraude organiseren met betrekking tot lokale,
nationale en Europese subsidieregelingen.'\footnote{Ibid.}

\subsection{Narco republieken}\label{narco-republieken}

Zoals we in \emph{The Great Reckoning} hebben gewaarschuwd, zijn
talrijke regeringen wereldwijd volledig gecorrumpeerd door drugsherren.
Mexico vormt daar een onbetwistbaar voorbeeld van. Voormalig Mexicaans
federaal adjunct-aanklager Eduardo Valle Espinosa plaatste het
Mexicaanse systeem in perspectief in zijn ontslagverklaring: `Niemand
kan een politiek project opzetten waarin de kopstukken van de
drugshandel en hun geldschieters niet voorkomen. Want als je dat wel
doet, sterf je.' Valle verklaarde dat steekpenningen het dienen als
Mexicaanse politiechef zo lucratief maken dat kandidaten tot \$2 miljoen
betalen om aangenomen te worden. Als je het strikt vanuit een winst- en
verliesberekening bekijkt, kan het kopen van een functie binnen de
lokale politie al gauw een winstgevende investering blijken te zijn.
Drugskartels betalen fortuinen aan zelfs laaggeplaatste Mexicaanse
ambtenaren, omdat dat hen immuniteit biedt tegen vervolging voor hun
misdaden.

Colombia is een ander land waar de hoogste regionen van de overheid
gedomineerd worden door drugsherren. Onlangs trokken de Amerikaanse
autoriteiten het visum van de Colombiaanse president Ernesto Samper voor
de Verenigde Staten in, omdat hij opzettelijk politieke bijdragen van
drugsdealers aannam in ruil voor gunsten.

\subsection{De pot verwijt de ketel dat hij zwart
ziet}\label{de-pot-verwijt-de-ketel-dat-hij-zwart-ziet}

Wie de rapporten in onze nieuwsbrief \emph{Strategic Investment} uit de
jaren negentig heeft gevolgd, herkent direct de ironie in de houding van
de Clinton-administratie tegenover Samper. Geloofwaardig bewijs wijst
erop dat de Amerikaanse president Bill Clinton alles heeft gedaan
waarvan Samper wordt beschuldigd -- en zelfs meer. Zelfs als u ons woord
niet aanneemt, onthullen twee grondig onderzochte boeken, geschreven
door auteurs uit beide uiteinden van het politieke spectrum, Clintons
achtergrond in opvallend detail.

Roger Morris, die doorgaans een linkse invalshoek hanteert, werkte als
veiligheidsfunctionaris in het Nixon-tijdperk en was bovendien een
belangrijke medewerker van Dean Acheson, president Lyndon Johnson en
Walter Mondale. Morris behaalde zijn doctoraat aan \emph{Harvard
University}. In zijn boek \emph{Partners in Power} onthult hij een
smerig verleden van Clinton, waardoor Samper als een onschuldige
padvinder overkomt.

Morris beschrijft Clintons vaderloze jeugd in Hot Springs, Arkansas --
een broeinest van gokken, prostitutie en georganiseerde misdaad waarbij
bijna de gehele familie betrokken was. Bill Clintons stiefoom, Raymond
Clinton, naar wie hij verwees als een `vadersfiguur', zou naar verluidt
een invloedrijke `peetvader' zijn geweest binnen de Dixie-maffia. Morris
stelt dat Bill Clinton als rekruté bij de CIA instroomde en zijn
studietijd aan Oxford besteedde aan het observeren van
anti-Vietnamoorlogactivisten. Volgens Morris diende Clinton tijdens zijn
gouverneurschap als een verlengstuk van de CIA, doordat hij een operatie
faciliteerde voor het smokkelen van drugs en wapens, met als middelpunt
Mena, Arkansas. Hij beschuldigt de gehele CIA ervan betrokken te zijn
bij drugshandel, in plaats van de mogelijkheid te overwegen dat Clinton
zich bij een corrupte factie binnen het agentschap had aangesloten --
een kans die voor ons aannemelijker lijkt. Beide interpretaties
impliceren dat de voornaamste geheime inlichtingendienst van de
Amerikaanse overheid, al dan niet direct, op grote schaal betrokken is
bij georganiseerde drugssmokkel. Als de CIA geen verlengstuk blijkt te
vormen van de georganiseerde misdaad, nadert zij desondanks op
zorgwekkende wijze de grens van 50. \footnote{Voor aanvullend en
  expliciet bewijs van de betrokkenheid van de CIA bij de drugshandel,
  zie Michael Levine, \emph{The Big White Lie: The Deep Cover Operation
  That Exposed the CIA Sabotage of the Drug War} (New York: Thunder's
  Mouth Press, 1994).}

\subsection{Eén kans op 250.000.000}\label{euxe9n-kans-op-250.000.000}

Toch bevat \emph{Partners in Power} details die elke student van de
corruptie in de hedendaagse Amerikaanse politiek zeker zullen boeien.
Morris richt zijn aanwijten niet uitsluitend op Bill Clinton; ook zijn
vrouw krijgt volop kritische aandacht. Neem bijvoorbeeld dit fragment
uit Morris' verslag over Hillary Clintons opmerkelijke handel in
grondstoffen: `In 1995 lieten economen van \emph{Auburn University} en
\emph{North Florida University} een geavanceerd statistisch
computermodel draaien op basis van de handelsactiviteiten van de
eerstedame, ter voorbereiding op publicatie in het \emph{Journal of
Economics and Statistics}, waarbij zij alle beschikbare gegevens en
marktinformatie uit de \emph{Wall Street Journal} benutten. Zij
berekenden dat de kans dat Hillary Rodham haar transacties op legitieme
wijze uitvoerde, minder was dan één op 250.000.000.' \footnote{Roger
  Morris, \emph{Partners in Power} (New York: Henry Holt, 1996), p.~233.}

Morris verzamelt talloze belastende details over de drugssmokkel- en
witwasoperatie die in Arkansas onder Clinton floreerde. `Door de enorme
hoeveelheden drugs en geld die met zijn vluchten werden gegenereerd,
groeide het kleine Mena, Arkansas, in de jaren tachtig uit tot één van
's werelds centra van de narcoticahandel.' \footnote{Ibid., p.~393.}
Morris citeert een vertrouweling die over Clinton getuigde: `Hij wist
het.'

Clinton wist van de cocaïnesmokkel en vertelde staatspolitieagent L. D.
Brown -- een voormalig lijfwacht aan wie hij hielp een functie bij de
\emph{CIA} te bemachtigen -- dat het drugsvervoer geen
\emph{CIA}-operatie betrof. `Oh, nee,' zei Clinton, `dat is Lasater's
zaak.' \footnote{Ibid., p.~411.}

Dan Lasater, een veroordeelde cocaïnedistributeur, behoorde tot
Clinton's belangrijkste financiële supporters. Hij verdiende miljoenen
met het staatsbedrijfsleven in Arkansas en overhandigde naar verluidt
eens \$300.000 contant -- in een bruine papieren zak -- aan de
toenmalige gouverneur van Kentucky, John Y. Brown. Volgens Morris was
Lasater nooit louter een grote donor die speciale eerbied verdiende,
maar een buitengewoon vertrouweling die Clinton geregeld opzocht in zijn
makelaarskantoor en spontaan naar het landhuis kwam. \footnote{Ibid.,
  p.~418.} Morris meldt dat Lasater's chauffeur -- die hem vaak naar het
landhuis bracht -- een veroordeelde moordenaar was die altijd een wapen
bij zich droeg en erom bekendstond ook drugs te verhandelen. \footnote{Ibid.}
Volgens Morris had de president van de Verenigde Staten een warmere band
met een drugshandelaar dan de vermeende connectie tussen de Colombiaanse
president Ernesto Samper en het Cali-cartel.

\begin{quote}
`Poeh! Bob zegt dingen over Bill Clinton die zelfs Hillary niet zou
zeggen' -- P.J. O'ROURKE
\end{quote}

R. Emmett Tyrell, Jr., hoofdredacteur van \emph{The American Spectator},
is geen links-liberaal zoals Morris. Toch bevat zijn verslag \emph{Boy
Clinton} veel van dezelfde details die Morris aandroeg om Clinton af te
schilderen als een corrupte politicus, nauw verweven met de drugshandel
en andere misdaden. Sterker nog, de proloog van \emph{Boy Clinton}
citeert L. D. Brown, Clinton's voormalige lijfwacht, die sensationeel
beweert dat Clinton medeplichtig was aan de activiteiten van een
dodensdienst die was opgezet om getuigen met kennis van de drugshandel
in Mena uit te schakelen.

Brown vertelt dat men hem op 18 juni 1986 persoonlijk naar Puerto
Vallarta, Mexico stuurde en hem daarbij een in België vervaardigd F.A.L.
lichtautomatisch geweer overhandigde. Onder het pseudoniem Michael
Johnson kreeg Brown naar verluidt de opdracht om Terry Reed neer te
schieten en te vermoorden.

U herinnert zich ongetwijfeld dat Reed in 1994 bekend werd als co-auteur
van \emph{Compromised: Clinton, Bush en de CIA}. Volgens dat boek heeft
de CIA het presidentschap in haar greep gekregen en hebben haar `zwarte
operaties, als een kanker, de organen van de overheid gemetastasiseerd'.
Meer specifiek stellen Reed en zijn mede-auteur dat zowel Clinton als
Bush door hun betrokkenheid bij illegale activiteiten in Arkansas --
onder meer drugshandel -- diep gecompromitteerd raakten.

Brown voerde de opdracht tot de moord op Reed niet uit. Hij en Reed
overleefden, waardoor zij in ieder geval delen van hun verhalen konden
vertellen, en het er zo beter op hadden dan degenen die destijds en
later met Clinton in verband kwamen. Neem als voorbeeld de inmiddels
overleden Jerry Parks, die in 1992 de beveiliging verzorgde voor het
hoofdkwartier van Clinton-Gore en in september 1993 slachtoffer werd van
een ganglandachtige aanslag.

In een andere bizarre wending onthulde de Londense \emph{Sunday
Telegraph}, op basis van exclusieve informatie verstrekt door de weduwe
van Parks, dat de overleden Vincent Foster Parks had ingehuurd als spion
op Bill Clinton.

Waarom Foster een dossier met compromitterende informatie over Clinton
wilde samenstellen, blijft een raadsel. (Hij verklaarde dat hij het voor
Hillary deed.) Maar dit weerlegt in elk geval de officiële beschrijving
van Foster als een naïeve plattelandsjongen, die zo geschokt raakte door
de meedogenloze werkwijzen in Washington dat hij zichzelf uit wanhoop
van het leven beroofde. Dat altijd ongeloofwaardige verhaal verliest met
elke nieuwe onthulling steeds meer aan geloofwaardigheid.\footnote{Voor
  een grondige bespreking van het Foster-verhaal zie \emph{Christopher
  Ruddy, Vincent Foster: The Ruddy Investigation}, dat verkrijgbaar is
  voor \$19,95 via 1-800-711-1968.}

\subsection{De president van de
maffia}\label{de-president-van-de-maffia}

Hoewel de wereld als geheel weigert te accepteren dat nauwe banden met
de georganiseerde misdaad en criminelen de president van de Verenigde
Staten negatief beïnvloeden, wijst het bewijs eenduidig in die richting.
Morris verwijst naar een voormalig Amerikaanse aanklager die criminele
figuren en hun belangen nauwlettend in de gaten hield. Hij stelt dat de
verkiezing van Clinton als gouverneur in 1984 `de verkiezing was waarop
de maffia écht binnenkwam in de politiek van Arkansas, de jongens van de
hondenbaan en de renbaan, de profiteurs die een mooie kans zagen\ldots{}
het ging verder dan onze oude Dixie-maffia, die in vergelijking daarmee
een kleinigheid was. Dit was crimineel geld uit het oosten en de
Westkust dat, net als de legitieme ondernemingen, de kansen zag.'

Blijkbaar blijven andere gelijkgestemden de mogelijkheden rondom Clinton
opmerken. Volgens \emph{New York} magazine -- dat opvolgt op een eerder
artikel in \emph{Readers' Digest} -- meldt men dat `de belangrijkste
bondgenoten van de president binnen de vakbondsbeweging tevens mannen
zijn die gelieerd zijn aan wat naar alle aanzichten enkele van de
viesste, meest door de georganiseerde misdaad geïnfiltreerde vakbonden
in Amerika lijken te zijn.' Van bijzonder belang is de hechte band
tussen Clinton en Arthur Coia. Coia, een van de `prima fondsenwervers'
van Clinton, leidt de Laborers International Union of North America,
`een van de meest opzichtig corrupte vakbonden in de geschiedenis van de
vakbeweging.'

Blijkbaar sloot het ministerie van Justitie in de Clinton-periode een
deal met Coia -- wat \emph{New York} magazine omschrijft als een
`merkwaardig genereuze deal' -- om zijn positie te handhaven, ondanks de
overtuigende beschuldigingen dat hij al lange tijd bevriend is met
figuren uit de georganiseerde misdaad.

Of de these van Terry Reed, dat `de CIA de presidentsfunctie heeft
gekaapt', nu klopt of niet, bestaat er duidelijk een sterke verleiding
voor personen binnen geheime organisaties -- die belast zijn met het
uitvoeren van `zwarte operaties' -- om volgens professor Hirshleifer
bewust onwettige middelen in te zetten om hulpbronnen te verkrijgen.

Gezien technologische veranderingen die de doorslaggevende kracht van
geconcentreerde militaire macht wereldwijd ondermijnen, mag men een
toename van corruptie verwachten, zo niet zelfs een volledige overname
van regeringen door georganiseerde criminele ondernemingen.

Hirshleifer betoogt -- en wij zijn het daarmee eens -- dat de
instituties van de politieke economie nooit zo perfect kunnen
functioneren dat zij de grondslagen van de natuurlijke economie volledig
verdrijven.\footnote{Hirshleifer, op. cit., p.~173.} Binnen de
`natuurlijke economie' verschuift de macht, wat ingrijpende
veranderingen in de machtsverhoudingen binnen de samenleving met zich
meebrengt.

Politieke corruptie, zoals Vito Tanzi scherp opmerkte, vertegenwoordigt
een vorm van privatisering van de staat waarbij de macht niet de markt
betreedt zoals gebruikelijk is bij privatisering, maar in plaats daarvan
bij overheidsfunctionarissen en bureaucraten terechtkomt.\footnote{Tanzi,
  op. cit., pp.~167, 170.} In feite gebeurde dit al bij het \emph{FBI}
en andere politiediensten onder Clinton. Clinton en zijn vriendjes
vormen de `rechtsstaat' naar eigen smaak. Tot dusver blijkt er
nauwelijks bewijs dat deze corrupte connecties enige invloed uitoefenen
op de kiezers, ook al pakken de massamedia de verhalen op en bespreken
ze uitvoerig. Integendeel, er lijkt weinig bezorgdheid te bestaan over
signalen dat de president van de Verenigde Staten medeplichtig is aan
drugshandel, witwaspraktijken en nog ernstigere misdrijven.

Dit roept de vrees op die de inmiddels overleden Walter Lippmann ooit
uitte, namelijk dat kiezers niet in staat zijn om door te dringen tot
wat hij `fictieve persoonlijkheden' noemde. Hij stelde dat kiezers
slecht worden bediend door vleierij en oppervlakkige bewondering en dat
zij verraden worden door een slaafse hypocrisie, die hen voorhoudt dat
hun stem bepalend is voor wat waar en onwaar, juist en onjuist is.

Lippmann voorzag een `inbreuk op de constitutionele orde' als aanleiding
voor de plotselinge en catastrofale achteruitgang van de westerse
samenleving. We zijn in korte tijd ver afgedwaald\ldots{} Wat we
meemaken is niet alleen verval -- een groot deel van de oude structuur
valt uiteen -- maar het lijkt wel of we op de drempel van een
historische catastrofe staan.\footnote{Ibid., p.~15.}

Het probleem schuilt erin dat politieke oordelen minder een reactie zijn
op de werkelijkheid dan op een door het algemene publiek geconstrueerde
schijnrealiteit over fenomenen die buiten hun directe kennis
liggen.\footnote{Paul Roazen, \emph{Inleiding}, in Lippmann, op. cit.,
  p.~xv.} Toch is het een vergissing om je te laten beperken tot wat
anderen waarnemen. Zelfs als je er onverschillig tegenover staat -- of
als Vincent Foster vermoord zou zijn en zijn dood kunstmatig in scène is
gezet door de hoogste politiediensten en verantwoordelijke
functionarissen van de Amerikaanse regering -- inclusief de huidige
speciale aanklager, Kenneth Starr, overweeg je wellicht bewijs van een
breder patroon van banden tussen georganiseerde misdaad en het Witte
Huis.

Op de lange termijn ondermijnt politieke corruptie op het hoogste niveau
het traditionele geloof in de mogelijkheden van de democratie om
publieke problemen doelbewust aan te pakken. In het informatietijdperk
gaat eerlijkheid van de overheid veel meer wegen dan haar omvang of
macht. De meeste diensten die overheden vroeger verleenden, zullen in de
komende eeuw naar de private sector verschuiven. Toch wijzen bewijzen
van over de hele wereld erop dat je op lange termijn nauwelijks kunt
vertrouwen op een systeem met corrupte leiders voor de veiligheid van je
gezin en je investeringen.

Zoals Morris opmerkt, ``{[}D{]}e Clintons zijn niet louter
symptomatisch, maar staan symbool voor het bredere tweepartijenstelsel
dat aan het einde van de eeuw op een doodlopende weg stuit.''\footnote{Morris,
  op. cit., p.~469.} In zijn essay over corruptie betoogt Vito Tanzi dat
``de enige manier om corruptie te weren is door de omvang van
overheidsinterventie aanzienlijk te verminderen.''\footnote{Fiorentini
  en Peltzman, op. cit., p.~16.} De informatierevolutie zal die omvang
drastisch verkleinen en daarmee hoop bieden op een wedergeboorte van
moraliteit en eerlijkheid. Een andere voor de hand liggende consequentie
voor de morele orde is de toegenomen kwetsbaarheid, doordat cyberhandel
en virtuele ondernemingen voortaan met onbreekbare encryptie
communiceren. Dieven binnen een organisatie -- zelfs binnen een virtuele
omgeving -- zullen moeilijker te traceren zijn en het wordt vrijwel
onmogelijk om gestolen geld terug te krijgen, zeker niet wanneer dit in
het geheim wordt vergaard door de verkoop van bedrijfsgeheimen, patenten
of andere waardevolle economische activa.

Misdaad loont, en velen vinden het aantrekkelijk om legale, productieve
activiteiten aan te vullen met illegale, roofzuchtige ondernemingen. In
tegenstelling tot de situatie zoals die in de afgelopen twee eeuwen in
westerse samenlevingen gangbaar was, worden criminelen tegenwoordig niet
meer slechts gezien als buitenbeentjes zonder sociale status. Wanneer
misdaad winstgevend is, duikt er doorgaans een elite van criminelen op,
omdat er steeds minder sociale afkeuring heerst voor crimineel gedrag.
Denk aan de Siciliaanse maffia, die samen met talloze drugsdealers --
die lokaal personeel inhuren tegen torenhoge tarieven -- op eigen
grondgebied zowel respect als populaire steun geniet.

\section{De morele orde en haar
vijanden}\label{de-morele-orde-en-haar-vijanden}

Sterke samenlevingen rusten op een degelijke morele basis. Elk onderzoek
naar de geschiedenis van economische ontwikkeling illustreert de nauwe
samenhang tussen morele en economische factoren. Landen en groepen die
succesvol ontwikkelen, slagen er deels in omdat zij een ethiek hanteren
die zelfredzaamheid, hard werken, gezins- en maatschappelijke
verantwoordelijkheid, een hoge spaarzin en eerlijkheid bevordert. Dit
geldt ook voor kleinere sociale subgroepen. Het zakelijk succes van
Joden -- met name religieuze Joden --, van de Puriteinen in New England,
van de Quakers in de Britse zakenwereld in de achttiende en negentiende
eeuw, en van de Mormonen in het moderne Amerika, toont stuk voor stuk de
economische voordelen aan die voortvloeien uit een cultuur met een
stevig moreel fundament.

Neem als voorbeeld de Quakers. Deze groep werd zakelijk zeer succesvol
en stond om diverse redenen vooral bekend als bankiers. Zij hanteerden
de hoogste norm van betrouwbaarheid: in plaats van eden af te leggen,
beschouwden zij elke zakelijke toezegging als even bindend als een eed.
`Mijn woord is mijn borg' was voor hen een onwrikbaar principe. Ze kozen
voor een rustige, fatsoenlijke en zuinige levensstijl en beschouwden het
als hun religieuze plicht geld niet te verspillen aan de ijdelheden van
deze wereld. Ze vermeden ruzies en waren ervan overtuigd dat oorlog
altijd zondig is. Daarnaast vonden ze dat een zakenman als morele
verplichting eerlijke waarde moet leveren, wat hen als handelaren een
reputatie opleverde voor het bieden van hoge kwaliteit tegen gematigde
prijzen. De uitdrukking `caveat emptor' -- laat de koper maar oppassen
-- volstond voor hen niet. In een tijd waarin de meeste kooplieden hun
handelstheorie baseerden op hoge prijzen en grote marges, leidde de
Quaker-moraliteit onvermijdelijk tot een strategie met lage winstmarges
en een hoge omloopsnelheid. Zoals Henry Ford later aantoonde, kan zo'n
beleid potentieel veel winstgevender zijn. Ze kozen bewust voor deze
benadering omdat zij het als hun plicht zagen hun klanten niet te
bedriegen en ontdekten al snel dat dit tegelijkertijd de beste manier
bleek om hun zaken uit te breiden. Zo bleken de Quakers uitstekende
zakenpartners te zijn: hun klanten keerden keer op keer terug, waardoor
beide partijen voordeel hadden. Bovendien gaven ze als een zuinige
gemeenschap die haar verplichtingen nakwam de Quakers een voorsprong als
bankiers, en het lidmaatschap van de Quakers op zich vormde al een
betrouwbaar bedrijfsinstrument.

Helaas kan het succes waarmee zulke zakelijke voordelen worden behaald,
er uiteindelijk voor zorgen dat ze zichzelf ondermijnen. Landen
doorlopen immers een cyclus -- zoals beschreven in de sociologische
theorie van Adam Ferguson uit de achttiende eeuw -- die begint bij
armoede en hard werken, overgaat in rijkdom, luxe en decadentie, en
uiteindelijk leidt tot achteruitgang. De oude Romeinen keken met heimwee
terug op de deugden van de republikeinse periode waarin het rijk werd
opgebouwd en betreurden de luxe en luiheid die zij als oorzaak van hun
ondergang beschouwden. Deze ondermijning van de ijverige deugden door
welvaart kan verrassend snel plaatsvinden. Hoewel de Duitsers nog steeds
een capabel en efficiënt volk zijn, werken zij onmogelijk meer zo hard
als toen zij hun land herbouwden na de verwoestende nederlaag in 1945.
In slechts twee generaties stapten zij over van lange werkdagen -- bijna
met blote handen in tijden van acute armoede -- naar korte werkdagen in
ruil voor het hoogste loon en de beste sociale voorzieningen op aarde.
In oktober 1995 tekenden zestien Duitse werkgeversverenigingen de
Petersburg-verklaring, een overzicht van goed onderbouwde klachten die
de achteruitgang van de industriële moraal in Duitsland weerspiegelen.

\begin{quote}
In 1995 bereikte de belastingdruk in Duitsland recordhoogten, mede door
de solidariteitstoeslag en de bijdragen aan de langdurige
zorgverzekering. Met een totale ondernemingsbelasting van meer dan 60
procent ligt Duitsland ver boven het internationale gemiddelde van 35
tot 40 procent. Praktijken in de publieke sector, zoals gereguleerde
promoties, een baan voor het leven en hogere pensioenuitkeringen, moeten
wijken voor vrijemarkteisen op het gebied van meritocratische promotie
en beloning. Omdat Duitsland de hoogste arbeidskosten ter wereld kent,
moet het loonbeleid bijdragen aan het terugdringen van de werkloosheid
door de kosten voor ondernemingen te verlagen. Loonstijgingen zouden
gebaseerd moeten worden op concurrentievermogen en productiviteit, en
het gedrag van de vakbonden dient te veranderen. Het jaarlijkse ritueel
van campagnes, eisen, mobilisatie van werknemers, dreigementen en
waarschuwingsstakingen is schadelijk.
\end{quote}

Die bezorgdheid dat vooral de Duitsers -- in het bijzonder de jongeren
en de erfgenamen van de welvaart -- de gewoonte van hard werken dreigen
te verliezen, deelt bondskanselier Kohl.

Het arbeidscontract bij Volkswagen kent autowerkers ter wereld het
hoogste loon toe, waarboven sociale zekerheidsbijdragen komen, in ruil
voor een werkweek van 28 uur -- vier dagen van elk zeven uur. Het
naoorlogse Duitsland exporteert nu massaal banen. Halverwege de
negentiende eeuw stonden de Britten bekend als de meest efficiënte
industriële natie, een reputatie die zij een eeuw later zeker hadden
ingeleverd. De cyclus van welvaart ondermijnt onvermijdelijk de deugden
van hard werken en bescheiden verwachtingen die kenmerkend zijn voor de
vroege fasen van een succesvolle industriële opbouw. Naties slagen er
vaak niet in hun oorspronkelijke deugden te behouden, net zoals
individuen bij een te gemakkelijke vorm van succes hebberig en lui
kunnen worden.

Wereldwijde investeringen prijzen terecht deze ijverige deugden en
straffen wie hebberig en lui worden, zoals het hoort. Men kan stellen
dat een verstandige investering zowel moreel als financieel verantwoord
moet zijn. Die Engelsman uit de achttiende eeuw, die investeerde in het
kapitaal van een Quakerbank, boekte naar alle waarschijnlijkheid groot
succes. In de negentiende eeuw stopten de Quakers hun geld in
chocoladebedrijven, ervan overtuigd dat cacao gezonder was dan alcohol.
Dat was waarschijnlijk zo, maar investeren in Fry's of Cadbury's bleek
eveneens een slimme zet. Investeerders dienen oppassen dat ze niet
verstrikt raken in perioden van decadentie. Hoewel Duitsland op de
Europese markt sterk staat en beschikt over geavanceerde industriële
kennis, beperken hoge arbeidskosten en korte werktijden het toekomstige
potentieel van het land.

Sociale moraliteit en economisch succes gaan hand in hand. Maar welke
factoren versterken juist de sociale moraal en welke ondermijnen die?
Arnold Toynbee, de invloedrijke filosofisch-historicus uit de eerste
helft van de twintigste eeuw, introduceerde de theorie van uitdaging en
respons. Uitdagingen stuwen samenlevingen vooruit en brengen de deugden
naar boven waarvan men niet eerder wist dat ze daar in schuilden.

Mensen beseffen al lang dat moeilijke tijden vaak gezondere reacties
oproepen dan periodes van overvloed. We streven er allemaal naar het
comfortabel te hebben; we dromen van een fijne woning, een baan waar we
voldoening uit halen en een stevige spaarpot. De strijd om deze doelen
te bereiken levert zijn vruchten af. We studeren hard, trainen onszelf
en werken keihard. Tegelijkertijd moet het loonbeleid de werkloosheid
bestrijden door bedrijfs-kosten te verlagen. Loonstijgingen moeten
gebaseerd zijn op werkelijke concurrentiekracht en productiviteit, en
het gedrag van de vakbonden verdient een herziening. Het jaarlijkse
ritueel van campagnes, eisen, werknemersmobilisaties, dreigementen en
waarschuwingstakingen werkt immers schadelijk.

Voor velen blijkt de strijd om die doelen uiteindelijk aantrekkelijker
dan het resultaat zelf. Begin deze eeuw behandelde de gerenommeerde
Zwitserse psycholoog Carl Jung een Amerikaanse zakenman die als jongeman
dezelfde ambities koesterde. Hij werkte onvermoeibaar om een eigen
bedrijf op te zetten en genoeg geld te verdienen zodat hij op
veertigjarige leeftijd met pensioen kon. Hij trouwde met een jonge,
aantrekkelijke vrouw, kocht een prachtig huis, stichtte een gezin en
verkocht uiteindelijk met groot succes zijn bedrijf, zodat hij als
rijke, onafhankelijke man zonder zorgen met pensioen ging. In het begin
omarmde hij zijn vrijheid en deed hij alles wat hij zichzelf had
beloofd. Hij nam zijn gezin mee op reis door Europa, bezocht
kunstgalerijen en dergelijke. Maar na verloop van tijd vervaagden die
interesses en het gevoel van vrijheid zelf. Hij keek terug op de periode
dat hij `niet vrij' was -- toen hij dag en nacht aan zijn bedrijf werkte
en alle gebruikelijke zakelijke zorgen had -- en beschouwde die tijd als
de gelukkigste van zijn leven. Uiteindelijk raakte hij in een depressie,
waarna zijn vrouw hem naar Jung bracht. Jung stelde vast dat hij geen
uitlaatklep had voor zijn creatieve energie, die naar binnen gekeerd hem
verwoestte. Hoewel de diagnose waarschijnlijk klopte, leidde dat niet
tot genezing; de zakenman herstelde nooit van zijn zenuwinzinking.

Voor veel mensen is de strijd belangrijker dan het uiteindelijke
resultaat; wij zijn gemaakt voor actie, en succes kan uiteindelijk
teleurstellend blijken te zijn. Ambitie -- wat voor vorm die ook
aanneemt -- stuwt ons in de strijd, maar die strijd blijkt vaak
bevredigender dan het beoogde eindresultaat, zelfs wanneer we ons doel
bereiken. Natuurlijk slagen de meesten van ons er slechts gedeeltelijk
in onze dromen waar te maken. We hebben niet zoveel geld als we zouden
willen en wonen niet in het perfecte droomhuis. We moeten genoegen nemen
met minder.

Men kwam tot het besef dat deugdzaamheid een dynamisch begrip is -- dat
ze vooral wordt gevormd door de inzet en niet louter door het behaalde
resultaat -- en dit inzicht ontwikkelde zich op uiteenlopende manieren
in de negentiende eeuw. Er circuleert een bekend gedicht van Arthur Hugh
Clough dat tijdens de Tweede Wereldoorlog veel troost bood te midden van
het leven en de dood. Het valt op dat het zelfmoordcijfer in de
strijdende landen in die perioden daalde; zelfs het voeren van oorlog
bleek soms beter dan het wegzinken in de verlamming van inactiviteit.

\begin{quote}
\emph{Zeg niet: de strijd is zinloos,\\
Het zwoegen en de littekens tellen niet voor niets,\\
De vijand geeft niet toe en verliest geen kracht,\\
En, zoals altijd, blijft hij voortbestaan.}
\end{quote}

\begin{quote}
\emph{Als hoop je op het verkeerde been zet, kunnen angsten je eveneens
misleiden;\\
Misschien schuilt er, verborgen in die rook, dat je kameraden de
vluchtenden achterna zitten,\\
En -- ware het niet voor jou -- al het slagveld bezetten.}
\end{quote}

\begin{quote}
\emph{Want terwijl de vermoeide golven tevergeefs beuken,\\
Lijken hier geen duidelijke signalen naar voren te komen;\\
Ver weg, gevormd door kreken en inhammen,\\
stromen ze rustig binnen en overspoelen de zee.}
\end{quote}

\begin{quote}
\emph{En niet uitsluitend via oosterse ramen,\\
Wanneer het daglicht doorbreekt, baant het zich een weg naar binnen,\\
Aan de horizon klimt de zon traag, oh zo traag,\\
Maar kijk naar het westen: het land straalt helderheid uit.}
\end{quote}

Deze energieke competitie spreekt nog steeds tot de moderne
gevoeligheid. Immers, velen van ons leiden hun leven als een
voortdurende strijd om kansen in een vaak vijandige omgeving te
benutten. We bevinden ons allemaal in een concurrerende wereld en de
meesten kiezen er niet voor om er buiten te blijven. Natuurlijk bestaat
er ook een contemplatieve, spirituele inslag, maar die komt zelden voor.

Een vergelijkbare 19e-eeuwse visie op deze dynamische moraal verwoordde
William James, de vooraanstaande Amerikaanse filosoof, in een toespraak
voor de Yale Philosophical Club in 1891:

\begin{quote}
Het fundamentele verschil in het morele leven van de mens ligt in het
onderscheid tussen een ontspannen en een inspannende stemming.\\
Als we in een ontspannen stemming verkeren, staat het afwijken van het
actuele kwaad als doorslaggevende overweging voorop.\\
De inspannende stemming daarentegen zorgt ervoor dat we het actuele
kwaad links laten liggen, mits het hogere ideaal wordt bereikt.\\
Waarschijnlijk bezit iedere mens het vermogen voor een inspannende
stemming, maar bij sommigen komt dat moeilijk tot uiting.\\
Deze stemming verlangt ernaar gewekt te worden door intensere passies,
door grote angsten, hartstocht en verontwaardiging, of door de diep
doordringende roep van hogere trouwswaarden als rechtvaardigheid,
waarheid en vrijheid.\\
Voor de visie ervan is een krachtige verlichting noodzakelijk; en een
wereld waarin bergen instorten en valleien oprijzen, biedt geen
gastvrije omgeving voor die stemming.\\
Daarom kan deze stemming bij een eenzame denker eeuwig sluimeren zonder
ooit ontwaakt te worden.\\
Zijn uiteenlopende idealen, waarvan hij weet dat het slechts
persoonlijke voorkeuren zijn, dragen vrijwel dezelfde symbolische
waarde, waardoor hij er naar hartenlust mee kan jongleren.\\
Dit verklaart bovendien dat in een puur menselijke wereld zonder God de
oproep tot onze morele energie niet zijn maximale stimulerende kracht
bezit.\\
Het leven vormt, zelfs in zo'n wereld, een oprechte ethische symfonie;
maar die speelt zich af binnen het beperkte bereik van een paar arme
octaven, waarbij de oneindige rijkdom aan waarden niet de ruimte krijgt
zich te openbaren.
\end{quote}

William James was ervan overtuigd dat de dynamische moraal -- waarin
doen boven louter zijn en handelen boven afzien van actie centraal staan
-- ook toepasbaar is binnen de religieuze sfeer. Tevens komt in het werk
van Adam Smith (1776) een krachtige ontwikkeling naar voren in de moraal
van competitie en overleving -- de morele doctrine die de basis vormt
van de moderne, wereldwijde economische orde -- een centraal thema dat
zorgvuldig heroverwogen dient te worden.

Het darwinisme gaat ervan uit dat soorten overleven doordat ze zich
aanpassen aan hun omgeving en dat het proces van natuurlijke selectie
hun kenmerken vormgeeft. Bij dieren komt dat proces door willekeurige
mutaties tot stand, waarvan inmiddels bekend is dat ze onderdeel zijn
van een genetisch proces waar Darwin zelf slechts op kon raden. Het
voortbestaan van menselijke samenlevingen berust echter op culturele
keuzes, gebaseerd op menselijke intelligentie. Cultuur transformeert de
samenleving op dezelfde wijze als genen andere soorten veranderen.
Hierdoor kunnen in onze samenlevingen veranderingen veel sneller
plaatsvinden; het draait immers niet om vele generaties, zoals bij
willekeurige genetische mutaties het geval is.

In tegenstelling tot dieren, die via natuurlijke selectie evolueren,
ontwikkelden mensen een vorm van culturele selectie. Sommige
samenlevingen voerden in een bepaald tijdvak baanbrekende technologieën
in die hen een doorslaggevend voordeel gaven bij het vergaren van
welvaart of het verkrijgen van macht.

Het culturele voordeel van nieuwe technologieën blijkt vaak bepalend --
denk maar aan de mens in het ijzertijdperk tegenover die in het bronzen
tijdperk, of de `elektronische' mens tegenover de `mechanische' mens.

Adam Smith was misschien niet de eerste econoom die het welzijn van
naties reduceerde tot individuele handelingen, maar hij verwoordde dit
op de meest bondige en gezaghebbende wijze:

\begin{quote}
Ieder individu streeft er voortdurend naar om het beste uit het kapitaal
te halen dat hij bezit. Hij werkt immers voor zijn eigenbelang -- niet
voor dat van de samenleving. Toch leidt zijn zoektocht naar persoonlijk
gewin er onvermijdelijk toe dat hij kiest voor hetgeen uiteindelijk het
meeste voordeel voor de gemeenschap oplevert.
\end{quote}

Thomas Malthus, de pionier op het gebied van bevolkingsstudies, stelde
vast dat Adam Smiths redenering niet alleen op de economische
ontwikkeling van naties van toepassing is, maar ook op het voortbestaan
van menselijke populaties. Hij is vooral beroemd om zijn stelling dat
`de bevolking, wanneer zij niet wordt ingeperkt, toeneemt volgens een
geometrisch tempo, terwijl het levensonderhoud slechts rekenkundig
groeit. Al een korte kennismaking met getallen toont de gigantische
verschillen tussen deze groeipatronen aan.'

Lang voor Darwin wees Malthus erop dat hetzelfde principe in de gehele
natuur geldt:

\begin{quote}
De natuur verspreidt de zaden van het leven in dieren- en plantenrijken
met een ongekende overvloed en vrijgevigheid. Tegelijkertijd gaat ze
zuinig om met de beschikbare ruimte en de voeding die nodig is om die te
laten groeien. Die levenskiemen, gehuisvest in een klein stukje aarde en
omgeven door overvloedig voedsel en ruime groeimogelijkheden, zouden in
enkele duizenden jaren miljoenen werelden kunnen vormen. Maar de
allesomvattende wet van de natuur houdt hen binnen vaste grenzen.
\end{quote}

Aan het einde van de achttiende eeuw bleek al dat de wereld zich op een
dynamische manier ontwikkelt -- een inzicht dat al in de tijd van
Adam~Smith en Malthus werd opgemerkt. De mens, als slechts één van de
vele levensvormen, moet concurreren vanwege de spanning tussen een
onbeperkte voortplantingscapaciteit en een beperkt vermogen om voedsel
te produceren. Het voortbestaan van menselijke samenlevingen, net als
dat van diersoorten, vraagt om een succesvolle aanpassing aan de
omgeving. Een dynamische moraliteit pakt daarom de uitdagingen van deze
aanpassing actief aan, doordat mensen hun handelen afstemmen op de
kansen die hun omgeving biedt en zo de beschikbare middelen optimaal
benutten.

Malthus merkte op dat de ideeën van Adam~Smith de wereld wezenlijk
hadden veranderd en stelde dat zijn eigen betoog over de bevolking niet
geheel origineel was: `De principes waarop het berust, zijn deels
uitgelegd door Hume en deels door Dr.~Adam~Smith.' Tevens benadrukte hij
dat deze voortdurende strijd om te overleven niet louter een praktische,
maar in essentie een morele zaak is. De afsluitende alinea van het
`Essay' uit 1798 luidt als volgt:

\begin{quote}
Het kwaad bestaat in de wereld niet om wanhoop te zaaien, maar om ons
tot actie te bewegen. Wij mogen het niet passief accepteren, maar moeten
ons inspannen het te vermijden. Het is zowel in ons eigen belang als
onze plicht om met alle macht het kwaad uit onszelf en uit de kring
waarop wij invloed uitoefenen weg te nemen; hoe vastberadener wij deze
taak volbrengen, des te wijzer en succesvoller onze inspanningen worden
en hoe meer wij onze eigen geest weten te verheffen, zodat wij
vollediger lijken te voldoen aan de wil van onze Schepper.
\end{quote}

Men kan het belang van dit betoog verder illustreren met Darwins
samenvatting van hoofdstuk~3 uit zijn baanbrekende boek \emph{On the
Origin of Species}, voor het eerst gepubliceerd in 1859. Hij noemde dit
cruciale hoofdstuk `\emph{Struggle for Existence}.' De onderwerpstitels
luiden als volgt: `Betekenis van natuurlijke selectie -- de term in
brede zin gebruikt -- geometrische vermenigvuldigingskracht -- snelle
toename van ingeburgerde dieren en planten -- de aard van de
groeibelemmeringen -- universele concurrentie -- effecten van het
klimaat -- bescherming door het aantal individuen -- complexe relaties
tussen alle dieren en planten in de natuur -- de strijd om het leven,
die het hevigst woedt tussen individuen en variëteiten van dezelfde
soort -- vaak nog feller tussen soorten van hetzelfde geslacht -- de
relaties tussen organismen, die de belangrijkste van alle relaties
vormen.'

Sinds 1776 blijkt al duidelijk dat naties hun welvaart het best
optimaliseren wanneer zij mensen de vrijheid bieden hun eigen
kapitaalrendement te maximaliseren in een systeem van vrije
concurrentie. Sinds 1798 weten we dat het voortbestaan van populaties
afhangt van samenlevingen die economisch en politiek voldoende succes
boeken om in hun eigen onderhoud te voorzien, zich te beschermen tegen
infectieziekten en hun bevolking in oorlogstijd te verdedigen. Sinds
1859 is gebleken dat het hele drama van het leven -- of het nu om
mensen, dieren of planten gaat -- een voortdurende strijd om te
overleven inhoudt, waarbij nauw verwante soorten of culturen vaak
elkaars grootste rivalen vormen. Deze strijd vraagt om een dynamische
moraliteit die het kwaad actief afweert in plaats van er enkel reactief
mee om te gaan.

Deze ideeën bleken zo krachtig dat, sinds hun ontstaan, niemand nadacht
over de aard van de mensheid of morele vraagstukken zonder er meteen op
in te springen. Karl Marx hechtte evenveel waarde aan de strijd om te
overleven als Charles Darwin, maar hij interpreteerde die als een oorlog
tussen sociale klassen, gevormd door economische krachten. Ook
Adolf~Hitler geloofde in de strijd om te overleven en bekijktte zijn
politieke carrière bijna uitsluitend vanuit dat perspectief. Hij vond
echter dat de strijd tussen verschillende rassen werd uitgevochten.
Marx, Lenin, Stalin, Mao en Hitler vertoonden allemaal kenmerken van
sociaal-Darwinisme, omdat zij de strijd om te overleven -- `Mein Kampf',
zoals Hitler dat noemde -- als de kern van de politieke realiteit
beschouwden. Marxisten zagen sociale klassen als afzonderlijke groepen,
terwijl nazi's rassen op een vergelijkbare wijze benoemden.

Dit leidt echter niet tot een dynamische moraliteit, zoals Malthus had
beoogd, maar juist tot een dynamische immoraliteit. Zowel het marxisme
als het nazisme wilden de strijd om te overleven aanpakken, maar deden
dat door de concurrentie te vernietigen. Ze drongen op onbekend terrein
door conflicten te zaaien tussen klassen die om sociale macht streden of
tussen rassen die zij beschouwden als economische uitbuiters (zoals vaak
door anti-semiten jegens joden wordt beweerd) of als een gevaarlijke
onderklasse (zoals de angst die blanke tegenstanders voor zwarten
koesterden). De Tweede Wereldoorlog vormde voor Adolf~Hitler -- die
mislukte -- een poging om voor het Duitse volk een overlevingsvoordeel
te creëren door potentiële rivalen, met name Slavische volkeren en
joden, uit te schakelen. Paradoxaal genoeg kwam Duitsland met een
nederlaag in de oorlog uiteindelijk beter af dan wanneer een
nazi-overwinning ooit was bewerkstelligd.

Het alternatief voor destructieve interferentie in de concurrentie is
samenwerkende concurrentie -- het kernidee van Adam~Smith, maar ook van
Malthus en William~James. Het model van destructieve concurrentie is de
veroveraar, die zijn rivalen verplettert om hun bezittingen binnen te
halen -- wat soms inhoudt dat hij hun landen overneemt en hun volk tot
slaaf maakt. Daartegenover staat de handelaar, het voorbeeld van
samenwerkende concurrentie. Een handelaar zorgt ervoor dat de klant
tevreden is, want alleen tevreden klanten komen terug voor meer
transacties. Daarnaast is het gunstig als de klant welvarend is,
aangezien een welvarende klant meer kan kopen. Verovering betekent de
vernietiging van de ander, terwijl handel draait om het vervullen van de
behoeften van de ander. Nu moderne technologie verovering tot een
buitengewoon risicovol beleid heeft gemaakt, vormt handel de enige
rationele benadering van de overlevingsvraagstukken.

Deze wederzijdse afhankelijkheid krijgt extra kracht door een ander
belangrijk principe van Adam~Smith, namelijk de specialisatie van
functies. `The Wealth of Nations' opent met een beroemde passage waarin
Smith opmerkt dat de grootste vooruitgang in de productieve krachten van
arbeid en het merendeel van de vaardigheid, behendigheid en het
oordeelsvermogen die ergens wordt toegepast, schijnbaar voortkomt uit
arbeidsdeling. Hij merkt bijvoorbeeld op dat het ambacht van het
vervaardigen van een speld is opgedeeld in zo'n achttien verschillende
handelingen, die in sommige fabrieken door diverse arbeiders worden
uitgevoerd. Hoe verder de specialisatie, hoe efficiënter de productie,
maar zo'n economie is vanzelfsprekend sterk afhankelijk van
samenwerking: succes vraagt om gezamenlijk optreden.

Een succesvolle sociale moraliteit moet aan een aantal voorwaarden
voldoen. Ze moet krachtig zijn -- een zwakke moraliteit is immers
kwetsbaar en ineffectief -- en de strijd om te overleven bevorderen op
een samenwerkende in plaats van een vernietigende manier. Hitler
hanteerde een krachtige overlevingsmoraliteit, maar zijn destructieve
drang heeft bijna zijn eigen samenleving verwoest. Verder moet een
degelijke sociale moraliteit dynamisch zijn, zodat hij meebeweegt met de
voortdurende veranderingen in moderne technologie en alle hedendaagse
sociale systemen, en bovendien economisch efficiënt opereren. De mix van
egalitaire en autoritaire ideeën binnen het Leninistische systeem bleek
simpelweg niet te werken. Dit omvat echter niet alle aspecten van
sociale moraliteit: ze moet ook de samenleving prettig en samenhangend
houden. Bovendien moeten moraliteiten zich flexibel kunnen aanpassen en
overleven -- een fragiele moraliteit mag in onze generatie lijken te
werken, maar in de volgende niet standhouden, terwijl een starre,
traditionele moraliteit te inflexibel blijkt om mee te bewegen met de
veranderende sociale structuur. Tegelijkertijd levert een puur
relativistische benadering geen moraliteit op, omdat zij geen duidelijke
richtlijnen verschaft voor gewenst gedrag.

Allereerst plaatsen we de sociale moraal in een bredere context. Een
hechte gemeenschap, zelfs een virtuele, is afhankelijk van een moraal
die breed wordt gedeeld. De meest succesvolle perioden in de
geschiedenis ontstaan wanneer de collectieve moraal algemeen gedragen
wordt. Zo'n moraal vervult niet alleen specifieke functies -- zoals het
verminderen van criminaliteit en het versterken van familie- en sociale
structuren -- maar geeft burgers ook een gevoel van doelgerichtheid en
richting. Historische eensgezindheid over moraal hangt vaak samen met
het bestaan van een dominante religie, of het nu gaat om een
staatsgodsdienst die in vroege tijden het overleven van een verspreid
volk waarborgde, de islam met haar sociale regels, het middeleeuwse
katholicisme of het protestantisme in het vroege New England. De
begrippen volk, moraal en religie zijn onlosmakelijk verbonden en
versterken elkaar.

In zo'n morele samenleving kan iedere burger binnen een kader van
sociale steun aan persoonlijke doelen werken. Natuurlijk kunnen de
morele voorschriften soms arbitrair overkomen -- of op zijn minst zo
worden ervaren door buitenstaanders. Een orthodoxe jood verliest de
vrijheid om varkensvlees of schelpdieren te eten en op de sabbat te
werken. Een trouwe katholiek wordt beroven van de mogelijkheid
kunstmatige anticonceptie te gebruiken, om nog maar te zwijgen van
abortus. Een moslim mag geen alcohol nuttigen. En een vrome confuciaan
moet mogelijk een langdurige, oncomfortabele rouwperiode doorstaan na
het verlies van zijn eerwaardige vader -- waarbij zelfs Confucius
waarschuwde dat rouwrituelen te ver zouden kunnen gaan. Toch zien de
volgelingen van deze geloofssystemen hun voorschriften als een geringe
opoffering voor de voordelen van een gedeeld en samenhangend
wereldbeeld, waarin ieder individu een vaste plaats heeft. Zo kan een
orthodoxe jood stellen dat het in acht nemen van de sabbat een kleine
opoffering is in ruil voor de kracht van de wet en de saamhorigheid
binnen het joodse gezin.

John Locke en de vroege voorvechters van vrijheid streefden naar een
gedeelde moraal in een tolerante samenleving. Zij waren ervan overtuigd
dat elke maatschappij -- ongeacht haar aard -- regels nodig heeft, maar
dat men enkel verplicht zou moeten worden de essentiële normen te
volgen, mits deze zijn gebaseerd op de hoogste rede. Ze erkenden dat
enige dwang in de sociale moraal onvermijdelijk is -- zeker ter
bescherming van leven en eigendom -- omdat geen enkele samenleving kan
voortbestaan zonder veiligheid. Tegelijkertijd pleitten zij voor vrijwel
absolute tolerantie betreffende persoonlijke keuzes die het welzijn van
anderen niet raken. Een confuciaan die veertig dagen om zijn vader
rouwde, kon naast een jood leven -- die de sabbat eerde -- zonder elkaar
in de weg te lopen of de ander te dwingen de eigen religieuze gebruiken
over te nemen.

Uit de combinatie van een sociale moraal voor de essentiële zaken en de
tolerantie voor persoonlijke beslissingen ontstaat een kernethiek
waaraan alle burgers gebonden zouden moeten zijn, naast een vrijwillige
ethiek die zij als individuen of als leden van subgroepen binnen de
samenleving omarmen. Wanneer een benedictijner monnik geloften van
armoede, kuisheid en gehoorzaamheid aflegt, doet hij dat als lid van
zijn kloostergemeenschap. Hij verwacht niet dat alle katholieken -- laat
staan al zijn medeburgers -- dezelfde geloften afleggen of dezelfde
regels naleven. Hij volgt de bevelen van zijn abt, maar gaat er niet van
uit dat iemand buiten het klooster hier aandacht aan besteedt. Het
naleven van minder essentiële aspecten van de moraal kan vrijwillig
blijven, maar de kernwaarden moet iedereen delen; wie die fundamentele
moraal weigert, schaadt zowel zichzelf als de samenleving. In het
uiterste geval biedt een maatschappij die overspoeld wordt door rovers
die zonder aarzelen vermoorden -- zoals in grote delen van Europa ná de
val van het Romeinse Rijk -- niemand een bevredigend bestaan, zelfs de
rovers zelf niet, omdat zij voortdurend onder druk staan van andere
moordenaars. Dit scenario doet zich ook in bepaalde stedelijke gebieden
in de huidige Verenigde Staten voor. Anarchie vormt immers géén ideaal
model, want zonder handhaving van de wet ontbreekt de menselijke
veiligheid.

Als men kijkt naar de krachten die de moraal van de samenleving
ondermijnen, dient men de kernethiek als uitgangspunt te nemen -- een
moraal die in de meeste moderne religies globaal min of meer hetzelfde
is. Ten minste twee van de Tien Geboden -- `Gij zult niet doodslaan' en
`Gij zult niet stelen' -- gelden als universele normen, zowel voor
christenen als voor joden. Zelfs bijna alle serieuze agnosten zien zowel
moord als diefstal, de ultieme bedreigingen voor leven en eigendom, als
ontoelaatbaar en erkennen dat de samenleving het recht heeft om hen die
beroven of vermoorden te straffen. Over de juiste strafmaat voor
individuele misdrijven kunnen zij wellicht van mening verschillen, maar
zij zijn het er unaniem over eens dat de samenleving een fundamenteel
recht heeft om besluiten te nemen omtrent bestraffing.

John Locke verwoorde het treffend: ieder mens heeft recht op `leven,
vrijheid en bezit.' In 1776 voegde Thomas Jefferson hieraan toe: `het
nastreven van geluk.' Hoewel dit een mooie en nobele uitdrukking is,
spreken `leven, vrijheid en bezit' meer over de praktische basis dan
`het nastreven van geluk.' De maatschappij is immers in hoge mate
afhankelijk van de rechten op leven en eigendom. De geschiedenis leert
ons dat deze fundamentele rechten uitsluitend in een vrije samenleving
beschermd kunnen worden. Wanneer de staat alomtegenwoordig en al te
machtig is, wordt zij de grootste vijand van het leven -- zoals te
merken is in agressieworlogen -- en van individueel eigendom, doordat
zij een onevenredig deel van de nationale rijkdom opeist voor eigen,
vaak ongewenste en verspillerige doeleinden.

De kernmoraal krijgt in de meest vooruitstrevende landen flinke kritiek,
mede dankzij de krachten van de moderniteit die hen hun technologische
voorsprong geven. De Verenigde Staten vormen 's werelds voornaamste
technologische macht. Tot ver in de vroege jaren zestig beschouwden
velen -- waaronder de meeste Amerikanen -- het land als een moreel
voorbeeld voor de rest van de wereld. Vandaag de dag komt zo'n oordeel
zelden meer naar voren, zelfs niet onder Amerikanen die trots op hun
land zijn. Men kon, net als de rest van de wereld, niet luisteren naar
het proces tegen O.J. Simpson en tegelijk de VS zien als de eenvoudige,
deugdzame republiek waarmee het ooit begon.

Als men terugkijkt op de benamingen uit het oude Amerika, bleek dat deze
de noden van een grensmaatschappij weerspiegelden en zelfs in de grote
steden de houding van de burgers bepaalden. De grensgebieden ademen een
intrinsiek democratische sfeer uit: mensen voelen zich gelijkwaardig en
de vroege Amerikanen wierpen de Europese klassehiërarchieën overboord.
Zelfs dienstmannen, die vanuit Engeland als gevangenen naar Amerika
werden gestuurd, vestigden zich als zelfstandige ambachtslieden, boeren
of vrije arbeiders zodra hun contract afliep. De lonen waren hoger dan
in Europa en de kosten van levensonderhoud laag, al waren geïmporteerde
vervaardigde goederen prijzig. Men was in de grensgebieden sterk op
elkaar aangewezen; ondanks de hardheid van het bestaan kende het leven,
in Europese termen, voorspoed. Immigranten begonnen vaak als
laagbetaalde arbeiders in de sloppenwijken van Boston of New York, maar
wisten daar doorgaans al snel aan te ontsnappen en bereikten generatie
na generatie voorspoed. Na de burgeroorlog voelden zwarte Amerikanen
zich als behorende tot een andere groep immigranten, en velen omarmden
de Amerikaanse waarden en idealen. Hieruit groeide uiteindelijk de
zwarte middenklasse.

Die ambitie, gesterkt door de ervaringen aan de grens en de invloed van
zowel protestantse als katholieke kerken, vormde het patriottisme van de
Amerikanen. Ze waren ervan overtuigd dat ze in Gods eigen land woonden
-- een opvatting die volledig werd gevoed door democratische idealen en
het christelijk geloof en die leidde tot 's werelds eerste en meest
succesvolle democratie. Dat beeld kennen we allemaal; het komt tot
uiting in de bijna iconische voorstelling van Abraham Lincoln, hoewel
sommigen in het zuiden hem nog steeds zien als de man die de gruwelen
van de eerste moderne oorlog ontketende om te voorkomen dat vrije staten
een unie verlieten die ze niet langer vertrouwden.

Toch blijft het beeld van Lincoln -- ruig, eenvoudig, eerlijk en
welbespraakt -- het hoogste Amerikaanse ideaal en dient hij in wezen als
moreel voorbeeld. Veel Amerikanen ervaren nog steeds de krachtige
oorspronkelijke tegenstelling tussen de democratische energie van het
nieuwe land en de versleten hiërarchieën uit Europa. Buitenlanders zien
dit ideaal van een intrinsiek dynamische meritocratie nauwelijks terug
in hedendaagse steden als Los Angeles, New York, Houston of Washington,
hoewel er in de grote voorstedelijke gebieden en op het platteland nog
meer dan slechts sporen van te vinden zijn. De Amerikaanse puriteinse
ethiek, met al haar historische betekenis, houdt het best stand ten
noorden van de sneeuwgrens, terwijl het ondernemende dynamisme over het
hele land verspreid is.

Veel Amerikanen wijzen op het verval van de grote steden, die
broedplaatsen zijn geworden voor criminaliteit -- met name in verband
met de narco-industrie -- als het meest ernstige symptoom van een
afnemend gemeenschappelijk moreel besef. Ook erkennen zij dat er een
botsing plaatsvindt tussen uiteenlopende morele culturen die wedijveren
om hun gezag en aanspraak. De `politiek correcte' cultuur wijst vele,
maar niet alle, morele beginselen van de oude cultuur af. Zij legt
sterke nadruk op de rol en rechten van groepen die historisch gezien
uitgebuit werden door een dominante witte mannencultuur en verwerpt die
cultuur, hoewel deze juist de oprichtingscultuur van de Verenigde Staten
vormt.

De dominante mannencultuur in de eerste helft van de twintigste eeuw
draaide om het behoud van het kerngezin. Hierdoor behield de echtgenoot
en vader -- zij het vaak slechts symbolisch -- de leiding in huis,
terwijl de echtgenote of moeder in de praktijk het dagelijkse beheer
voerde en haar man als nominale meester onderdanig accepteerde. Ook op
de werkvloer oefende de man een flinke macht uit, een positie die de
feministische beweging tot op heden weliswaar aan de kaak heeft gesteld,
maar niet heeft weten om te draaien. Het grote belang dat aan het gezin
werd gehecht én de historische christelijke leer schreven abortus ten
stelligste verboden. Vroeger beschouwde de oude moraal abortus als een
onrechtmatige doding, volstrekt ontoelaatbaar -- een visie die haar
aanhangers nog steeds delen, terwijl voorstanders van de nieuwe moraal
juist het tegendeel inzien. In \emph{Roe v. Wade} stelde het
Hooggerechtshof het grondwettelijke recht op abortus -- dat voorheen als
een aangelegenheid voor de individuele staten gold -- vast op basis van
het recht op privacy, een doctrine die ver verwijderd is van enige
expliciete formulering in de Grondwet of haar amendementen.

Men interpreteerde de privacy van een vrouw zodanig dat zij zelf kon
bepalen of zij kinderen wilde krijgen, zonder rekening te houden met de
gevolgen voor het embryo. Het Hooggerechtshof achtte het embryo namelijk
niet als drager van grondwettelijke rechten -- in de late twintigste
eeuw beschouwde men embryo's als extraconstitutionele entiteiten,
vergelijkbaar met de positie van slaven in de eerste helft van de
negentiende eeuw. `Leven, vrijheid en de zoektocht naar geluk' gold
immers niet voor slaven, en de rechters in \emph{Roe v. Wade} pasten
deze bewoordingen dan ook niet toe op embryo's.

Het abortusdebat illustreert op treffende wijze het conflict tussen de
oude en de nieuwe moraal, al daagt een vernieuwde visie in tal van
andere domeinen de traditionele sociale organisatie uit. De traditionele
christelijke moraal, in zowel protestantse als katholieke kerken, legde
strikte normen vast voor seksuele rollen: relaties buiten of vóór het
huwelijk schonk men geen plaats, en ook genitale homoseksuele omgang
werd afgewezen. Lesbianisme kwam minder nadrukkelijk aan bod, daar men
het bestaan ervan nauwelijks erkende. Toen koningin Victoria er voor het
eerst van hoorde, weigerde zij koppig te geloven dat vrouwen zulke
relaties zouden hebben. Politieke correctheid vertegenwoordigt de moraal
van zogenaamd onderdrukte groepen. Homoseksuelen eisten gelijke
erkenning voor hun levensstijl en tartten de traditionele afkeuring van
hun seksuele gedrag. Men bestempelde `homofobie' als een verwerpelijke
vorm van vooroordeel, vergelijkbaar met rassendiscriminatie. Critici op
dit vlak worden door de nieuwe moraal net zo onaanvaardbaar geacht als
kritiek op zwarten, joden of vrouwen.

Tegelijkertijd doorbraken en schafte men ook andere seksuele taboes af.
In de jaren zestig ontwaakte een nieuwe golf van `vrije liefde', deels
dankzij de vermeende betrouwbaarheid van de anticonceptiepil voor
vrouwen, maar ook gestimuleerd door stemmingsveranderende drugs en
popmuziek. Hierdoor werd samenwonen buiten het huwelijk steeds
gangbaarder. Tegen de jaren negentig beschouwde men in Groot-Brittannië
-- een samenleving die vergeleken met de meeste delen van de Verenigde
Staten wat ouderwets was -- als volstrekt normaal dat prins Edward in
Buckingham Palace met zijn vriendin sliep in een stabiele, ongehuwde
relatie, vergelijkbaar met de manier waarop studenten in de jaren zestig
in hun studentenkamers samenleefden. Weinig mensen vonden het raar dat
koningin Elizabeth~II, hoofd van \emph{Church of England}, het gedrag
van haar jongste zoon tolereerde, terwijl de huwelijken van haar drie
oudste kinderen al waren verbroken. Wie hierover klaagde, noemde men
ouderwets en betweterig, terwijl velen nog steeds de traditionele moraal
prefereerden, ook al leefden zij er niet naar en verwachtten zij niet
dat hun kinderen dat vanaf jonge leeftijd zouden doen.

Binnen de politiek correcte beweging doemde ook een puriteinse inslag
op. Doordat deze beweging voortkwam uit de vermeende belangen van
vrouwen -- beschouwd als de meest onderdrukte groep -- ontwikkelde zij
een uitgesproken vijandigheid tegenover de mannelijke seksualiteit,
zowel in open agressieve uitingen als in gedragingen die men voorheen
als onschadelijk beschouwde. Sommige vrouwen waren ervan overtuigd dat
alle mannen van nature verkrachters zijn, waardoor de natuurlijke
afschuw voor verkrachting uitmondde in een algemene veroordeling van het
mannelijk geslacht. Anderen richtten hun aandacht op seksuele
intimidatie, een terechte klacht omdat vele mannen zich extreem grof
seksueel benaderen; in sommige triviale gevallen maakte men er zelfs
belachelijke situaties van. Sommigen beweerden zelfs dat een enkele blik
al als seksuele intimidatie kon worden aangemerkt, zonder dat iemand ook
maar één woord zei of fysiek contact maakte. Hierdoor kwam de nieuwe
moraal soms over als bijzonder censurerend. Witte mensen konden
beschuldigd worden van raciale vooroordelen, niet omdat zij per se
bevooroordeeld waren, maar simpelweg vanwege hun huidskleur. Mannen
kregen de vinger over de neus gelegd omdat zij door hun uitingen lieten
blijken een vrouw aantrekkelijk te vinden -- een houding die men in
voorgaande generaties als een compliment beschouwde in plaats van als
belediging.

De politiek correcte groepen en de fundamentalistische christenen
leveren elkaar bittere kritiek toe, maar in de moderne wereld vertonen
zij opvallend veel overeenkomsten. Beide kampen vinden de autoriteit van
hun eigen morele doctrine universeel, ook al verschillen hun morele
leerstellingen wezenlijk. Men bekritiseert hen beiden vanwege een
overdreven, zelfverzekerde moraliteit die vaak ontbreekt aan diepgang,
historisch perspectief en tolerantie. Sommigen vergelijken hen met het
zeventiende-eeuwse puritanisme, ervaren moralisten zoals Oliver Cromwell
in Engeland -- die bijna naar Nieuw-Engeland vertrok -- of met de
heksenjagers van Salem. Noch de meer rigide vrouwenbeweging, noch de
conservatieve predikanten van de \emph{Bible Belt} kunnen hen verweten
moreel tekort te schieten; integendeel, hun moraliteit blijkt juist
overontwikkeld en inflexibel. Soms lijkt het hart van hun morele
overtuigingen tot steen verworden te zijn. Zo'n verharding schaadt de
gemeenschappelijke moraliteit in de samenleving net zozeer als de `alles
mag'-anarchie waartegen zij zich verzetten.

Het leidt tot een vertekende weergave van morele krachten en culmineert
in zelfgerechtigdheid. Het farizeïsme -- het onwrikbare geloof in de
eigen deugd -- is zo oud als de mensheid en werd door Jezus Christus
krachtig veroordeeld. Daartegenover staat de recente opvatting dat
ethische keuzes louter een kwestie van persoonlijke voorkeur zijn, net
zo individueel als de keuze van kleding. Dit standpunt weerspiegelt het
ontbreken van een gedeelde moraliteit, herdefinieert de klassieke
vrijheidstheorie en verandert `het nastreven van geluk' -- zoals
John~Locke dat oorspronkelijk bedoelde en Jefferson in 1776 verwoorde --
in een hedonistisch streven dat roekeloos omgaat met de gevolgen.

De uitdrukking «het nastreven van geluk» vindt zijn oorsprong in
John~Locke's \emph{Essay on Human Understanding} (1691). Daarin schrijft
hij: «De hoogste volmaaktheid van de intellectuele natuur ligt in een
zorgvuldige zoektocht naar waarachtig en solide geluk, zodat de zorg
voor onszelf -- opdat wij het denkbeeldige niet verwarren met het reële
-- de noodzakelijke basis van onze vrijheid vormt.» Vervolgens merkt hij
op: «Niet iedereen vindt zijn geluk in hetzelfde\ldots{} de geest heeft
een andere smaak dan het gehemelte\ldots{} Mensen mogen voor
verschillende dingen kiezen, maar allen maken de juiste keuze, indien
men hen beschouwt als een gezelschap arme insecten, waarvan sommigen
bijen zijn, die zich verheugen over bloemen en hun zoetheid, en anderen
kevers, die genieten van andere soorten kost.» Toch betoogt hij dat het
verkiezen van ondeugd boven deugd onmiskenbaar een foutieve afweging is.
Hij legt daarbij bijzondere nadruk op religieuze argumenten en stelt dat
de slechten het zwaarst te lijden hebben. Hij is ervan overtuigd dat
wanneer moraliteit op degelijke fundamenten rust, deze de keuzes van
iedereen die erover nadenkt, onvermijdelijk zal bepalen.

De Lockeniaanse doctrine van vrijheid biedt ongetwijfeld meer ruimte
voor individuele voorkeuren dan autoritaire morele systemen die trachten
iedereen gelijk te behandelen en uniform gedrag af te dwingen. Toch
erkent de klassieke vrijheidsleer al snel de noodzaak van collectieve
morele imperatieven, zoals respect voor medemensen -- in het bijzonder
voor hun leven en het vreedzaam bezit van hun eigendommen volgens de
wet. Een afname van de gedeelde moraliteit vormt een bedreiging voor de
vrijheid, enerzijds doordat het een element van anarchie introduceert en
anderzijds doordat het autoritaire krachten in de hand werkt. De
geschiedenis van de publieke moraliteit verloopt in een cyclus van
wanorde en autoritarisme; hedendaagse autoritaire morele stromingen,
zoals het feminisme en het fundamentalisme, zijn ontstaan als reactie op
het hedonisme van de jaren zestig.

We hebben reeds enkele kenmerken geschetst van de wereld van de komende
eeuw. Deze zal worden bepaald door twee hoofdontwikkelingen: de
technologische verschuiving die de Aziatische economieën opent en de
opkomst van wereldwijde elektronische communicatiemiddelen die burgers
steeds minder afhankelijk maken van de lokale overheid. Deze nieuwe
technologie vervangt -- of heeft al vervangen -- veel middelmatige
menselijke vaardigheden, zoals die van de productielijnwerker, de
kantoormedewerker en steeds vaker de middenmanager. Tegelijkertijd
beloont zij zeldzamere competenties en creëert zij een internationale
cognitieve elite van hoogopgeleiden, voor wie de nieuwe
communicatiemiddelen een zo breed mogelijke markt openen voor hun
talenten. Net als de meeste elites beschouwt deze cognitieve elite
zichzelf vaak als superieur, gaat zij arrogant te werk en meent zij haar
eigen normen te mogen stellen. Daardoor vervreemdt zij zich van de
samenleving.

Tijdens de eerste helft van de volgende eeuw vindt een enorme overdracht
van rijkdom plaats van het oude Westen naar het nieuwe Oosten. Politieke
tegenslagen -- en aangezien China nog steeds als politiek achterhaald
wordt gezien -- kunnen deze verschuiving wellicht vertragen, maar zullen
haar vrijwel zeker niet tegenhouden. Ze kunnen het proces niet
terugdraaien.

Deze rijkdomsoverdracht legt ongetwijfeld de grootste druk op de landen
op het noordelijk halfrond die door blanke machten worden gedomineerd,
zoals in Europa en Noord-Amerika. Momenteel telt dit gebied ongeveer 750
miljoen inwoners in de ontwikkelde landen; tot voor kort was Japan het
enige Aziatische, niet-blanke land dat de Euro-Amerikaanse
levensstandaard had bereikt, al waren er etnisch Europese
bevolkingsgroepen in Nieuw-Zeeland, Australië en in de blanke
gemeenschappen van Zuidelijk Afrika. In 1990 behoorden de geavanceerde
industriële landen slechts naar schatting tot 15 procent van een
wereldbevolking van 5 miljard. Damaar lag de verdeling van de
wereldwijde rijkdom op ongeveer 15 procent welvarenden en 85 procent
minderbegaarden -- een verhouding die sterk overeenkomt met de
inkomensverdeling in geavanceerde industriële samenlevingen van een eeuw
geleden. Tegen 2050 wordt verwacht dat de ontwikkelde economieën, binnen
een wereldbevolking die mogelijk tot 7 miljard groeit, ongeveer 3
miljard mensen omvatten -- een situatie die neerkomt op een
rijkdomsverdeling van 40 procent welvarenden en 60 procent armen. Tegen
het einde van de eeuw kunnen de aantallen zelfs omkeren, met 60 procent
welvarenden en 40 procent mensen met een lager inkomen, waarbij armoede
vooral geconcentreerd blijft in Afrika. Hoewel deze verschuiving tussen
naties mondiaal een grotere gelijkheid in rijkdom bevordert, zal de
ongelijkheid binnen landen waarschijnlijk juist toenemen. Wie zijn
talent en kapitaal efficiënt benut, krijgt een aanzienlijk voordeel ten
opzichte van mensen met gemiddelde vaardigheden of weinig middelen.
Bovendien beweegt deze rijkdom zich uiterst mobiel. De armen in de
ontwikkelde landen zullen de rijken niet langer kunnen belasten zoals in
de twintigste eeuw; landen die dat proberen, lopen het risico achterop
te raken in een felle, competitieve race.

Natuurlijk blijft de totale productiviteit van de wereldeconomie
groeien, waarschijnlijk met gemiddeld 3 procent wereldwijd, mits er geen
wereldoorlogen uitbreken. Als dat standhoudt, verdubbelt het
wereldproduct elke vijfentwintig jaar -- waardoor het in 2050 meer dan
vier keer zo hoog ligt als nu en tegen 2100 zestien tot twintig keer zo
groot zal zijn. Zelfs als de wereldbevolking tegen 2100 oploopt tot 8
miljard, resulteert dat per hoofd in een wereld-BBP dat aan het einde
van de eeuw tien keer het huidige niveau bedraagt. Een dergelijke
toename van rijkdom kan zowel de opkomst van nieuwe industriële
samenlevingen als de multimiljoeneninkomens van de cognitieve elite
opvangen, terwijl het tegelijkertijd voor de overige arbeidskrachten in
de ontwikkelde landen een stijgende en fatsoenlijke levensstandaard
waarborgt. De inkomensverschillen zullen daarbij echter radicaal anders
zijn dan in de twintigste eeuw. Op wereldschaal zullen de inkomens in
arme landen veel sneller groeien dan in rijke landen, terwijl nationaal
gezien juist de inkomens van de rijken -- zoals in het Amerika van de
jaren negentig -- veel sneller zullen stijgen dan die van de midden- en
lagere inkomenslagen. In de komende eeuw zullen we getuige zijn van de
opkomst van een wereldsuperklasse, mogelijk bestaande uit 500 miljoen
zeer rijke mensen, waarvan er 100 miljoen rijk genoeg zijn om zich als
onafhankelijke, soevereine individuen te onderscheiden.

Dit proces heeft onvermijdelijk consequenties. Samenlevingen worden veel
minder homogeen, de natiestaat verzwakt of valt zelfs volledig uiteen en
de cognitieve elite zal zichzelf als kosmopolitisch gaan beschouwen.
Mensen die wereldwijd in vergelijkbare functies werken, ontwikkelen
immers een cultuur die veel meer overeenkomt met die van hun collega's
in andere delen van de wereld dan met die van hun medeburgers in
traditionele natiestaten. Een investeringsbankier uit Londen voelt zich
waarschijnlijk meer thuis in Seoel dan in Glasgow, en een ambtenaar uit
Washington voelt zich wellicht prettiger in Bonn dan in de zwarte wijken
van Washington zelf. We zien nu al hoe dit proces de morele waarden
fragmentariseert. De moraal van een individu wordt immers deels bepaald
door zijn opvoeding -- door wat iemand als kind geleerd heeft -- en
deels door zijn levenservaringen. Zowel de scholing als de ervaringen
binnen de cognitieve elite hebben een duidelijke kosmopolitische inslag,
die mensen losmaakt van hun lokale gemeenschappen.

Naarmate we de volgende eeuw ingaan, valt op dat veel vertegenwoordigers
van de groeiende cognitieve elite nauwelijks enige religieuze of morele
vorming binnen het gezin hebben meegekregen. De overheersende
levensbeschouwing in deze kringen is agnostisch humanisme. Bovendien
wordt veelvuldig gezien dat deze families verscheurd worden door
echtscheidingen en hertrouw, met daaropvolgende derde huwelijken. Hoewel
het huwelijksmodel dat in Hollywood heerst niet representatief is voor
de gehele Verenigde Staten, kampt de cognitieve elite in Euro-Amerika
met een hoog percentage echtscheidingen -- gemiddeld wel een derde of
meer. Kinderen van gescheiden ouders krijgen zelden een fundamentele
religieuze vorming mee en raken al vroeg bewust van de uiteenlopende
morele opvattingen die hun ouders, stiefouders en stiefbroers en -zussen
hanteren. Vergelijk je deze initiële morele vorming met die in een
traditioneel Iers of Pools dorp, dan valt op dat de boerenopvoeding een
veel sterkere religieuze basis biedt. Een elite die goddeloos,
wortelloos en welvarend is, zal waarschijnlijk niet gelukkig worden of
geliefd zijn.

Latere levenservaringen versterken ongetwijfeld de tekortkomingen in de
vroege morele vorming van de toekomstige dominante economische groep.
Zij doorlopen een intensieve technische opleiding -- in welke vorm dan
ook -- om zich voor te bereiden op hun nieuwe rol als leiders in het
elektronisch tijdperk. Uit deze opleidingen halen zij echter slechts
enkele morele lessen mee, de lessen die historisch gezien de basis
vormden van ons sociale gedrag. Volgens de maatstaven van Confucius,
Boeddha of Plato (500 v.Chr.), van Sint-Paul (50 n.Chr.) of van Mahomet
(600 n.Chr.) voldoen zij immers nauwelijks; men zou hen als moreel
analfabeten bestempelen. Ze krijgen les in economische efficiëntie, in
het optimaal inzetten van hulpbronnen en in het najagen van geld, maar
niemand leert hen over de deugden van nederigheid, zelfopoffering of
kuisheid. In feite groeien de meesten als heidenen op, met een
waardenstelsel dat meer weg heeft van de late Romeinse Republiek dan van
het christendom. Bovendien zijn deze waarden sterk individualistisch in
plaats van gemeenschappelijk. Zoals we al betoogden, putten
samenlevingen alleen kracht uit morele waarden die breed worden gedeeld.
De ontwikkelde naties bewegen inmiddels richting een situatie waarin
velen zwakke of beperkte morele waarden hanteren, terwijl anderen dit
compenseren door zich fel vast te klampen aan irrationele waarden -- en
slechts enkele waarden vinden breed draagvlak in de samenleving. Zonder
twijfel zullen sommige van de eerder beschreven `concurrerende
territoriale clubs' strikte morele normen gaan hanteren voor wie er in
hun gemeenschap mag wonen.

Historisch gezien veroorzaakten verschillen in rijkdom op zich geen
wezenlijke variaties in religieuze waarden. In dichtbevolkte en stabiele
samenlevingen met sterke tradities kan een rigide hiërarchie -- `de
rijke man in zijn kasteel, de arme man bij zijn poort' -- waarden
bevatten die door de hele samenleving doorklinken. Dat vergt echter een
sterk gemeenschapsgevoel bij zowel de rijken als de armen en robuuste
sociale tradities. Tegenwoordig missen we deze voorwaarden; de
economische en technologische revolutie verzwakt immers zowel het
gemeenschapsgevoel als de tradities. Daardoor lopen het leven van de
meesten en dat van een selecte groep steeds verder uit elkaar.
Innovatoren doorbreken de oude werkwijzen, wat de technologische
revolutie mogelijk maakte; in elk domein komt de radicaal als winnaar
naar voren, terwijl conventionele denkers achterblijven en letterlijk
buiten de boot vallen. Hoewel politici als Bill Clinton, Helmut Kohl en
John Major vaak de politieke koers bepalen, leiden radicale ondernemers
met scherp inzicht in de technologische wereld onze meest succesvolle
bedrijven -- een toonbeeld hiervan is Bill Gates. Het traditionele
denken verliest zijn geloofwaardigheid doordat het niet in staat is de
razendsnel veranderende omstandigheden bij te benen.

Moraliteit houdt het echter anders. Als we Mozes' `wetenschap', ontstaan
rond 1000 v.Chr., nauwkeurig bestuderen, ontdekken we dat deze ons
weinig concreet leert. Het scheppingsverhaal in Genesis draagt
ongetwijfeld een theologische waarheid uit -- namelijk dat God het
universum en de mensheid schiep -- maar beschrijft niet hoe fysieke
structuren daadwerkelijk ontstonden. Daarentegen bevat de moraliteit van
Mozes, zoals verwoord in de Tien Geboden, een schat aan wijsheden.

Zo vormt respect voor ouders en trouw binnen het huwelijk de beste
waarborg voor het behoud van een stabiel gezinsleven, wat op zijn beurt
essentieel is voor de opvoeding van moreel gezonde kinderen. Diefstal
schaadt zowel de dader als de slachtoffers en ontmoedigt arbeid en
sparen. Bovendien steunt een goed functionerende samenleving op de
eerlijkheid van getuigen; moord is verwerpelijk, enzovoort.

Op het gebied van wetenschap heeft de mensheid in drie duizend jaar onze
kennis drastisch getransformeerd, maar moreel gezien lijken we mogelijk
achteruit te gaan. De gemiddelde psychotherapeut biedt zijn patiënt
waarschijnlijk minder gedegen moreel advies over de levenswijze dan een
Jood in Mozes' tijd van zijn leraar ontving. Natuurlijk leeft het
christendom nog voort, maar voor het merendeel van de wereld is het
slechts een schim van zijn vroegere glorie. Slechts weinigen koesteren
het geloof zoals dat in vroegere tijden of in minder verfijnde
gemeenschappen werd beleefd; men zoekt immers geen heiligen op Park
Avenue.

Het afbreken van traditie was onmisbaar voor wetenschappelijke
vooruitgang. Als we vandaag nog zouden geloven dat de zon om de aarde
draait, hadden we nooit satellietcommunicatie ontwikkeld. Wat wij als
wetenschap aanhangen, bestaat immers uit een reeks hypothesen --
onvolmaakte verklaringen die later door sterkere, maar nog steeds
gebrekkige uitspraken worden vervangen. Toch heeft het verwerpen van
traditie desastreuze gevolgen gehad voor de morele orde in de wereld.

Confucius onderwees dat we ons met mate moeten gedragen -- hij noemde de
gulden middenweg `chum yum', zoals zeventiende-eeuwse geleerden die term
hebben vertaald. Bovendien benadrukte hij dat we autoriteit dienen te
respecteren en anderen behandelen op de wijze waarop wij zelf behandeld
willen worden. Die leer is inmiddels 2500 jaar oud en heeft China
gedurende de gehele geschreven geschiedenis diepgaand beïnvloed. Toch
zien veel moderne Chinezen het confucianisme als een achterhaalde
traditie, waarin men geen waarde hecht aan gematigdheid, waarbij kracht
meer telt dan autoriteit en waarin anderen niet worden behandeld zoals
men dat zelf graag zou willen ervaren. Bij het vervallen van traditie
kan een samenleving haar volledige morele referentiekader kwijtraken. En
ondanks haar groeiende macht blijft China moreel achterhaald in
vergelijking met Tibet, waar mensen evenarm en onderdrukt leven als in
de rest van de Tibetaanse gemeenschap.

Een degelijke sociale moraliteit kent een aantal kenmerken. Zij moet
zowel het voortbestaan van de samenleving als dat van het individu
bevorderen, en dat op een dynamische in plaats van op een statische
wijze. Ze stimuleert tolerantie en verdringt zelfgenoegzaamheid.
Bovendien hoort ze een religieus karakter te hebben in plaats van louter
agnostisch te zijn, en doet ze niet alsof ze in staat is om
wetenschappelijke feiten vast te leggen. Ze mag noch anarchistisch noch
autoritair zijn en moet breed gedragen en diepgeworteld zijn. Zo'n
morele basis is van cruciaal belang voor het gezin en voor de opvoeding
van kinderen tot zelfstandige, verantwoordelijke volwassenen; zij vormt
immers de spil van een gezonde samenleving.

Onze ervaringen tonen dat deze morele basis steunt op de logica van
onderlinge afhankelijkheid, afkomstig uit handel en medemenselijkheid.
Tegelijkertijd bedreigen een oppervlakkig scientisme, de vervreemding
tussen een superklasse en een onderklasse en het verdwijnen van de oude,
regiogebonden economische grondslagen deze fundamenten. Wellicht komt er
ooit een tegenreactie op deze trends, want deze ontwikkelingen moeten we
erkennen als uiterst gevaarlijk voor de samenlevingen van de komende
eeuw.

Nu, terwijl volgens Isaiah Berlin `de meest verschrikkelijke eeuw in de
westerse geschiedenis' ten einde lijkt te komen, hoort ook het tijdperk
van gigantische sociale structuren tot het verleden. De laatste dagen
van de twintigste eeuw markeren een periode van inkrimping,
decentralisatie en reorganisatie. We bevinden ons in een tijdperk van
sociale dinosauriërs die vastzitten in de teerpit, en van aaseters die
boven de resten uitstijgen. Vogels zullen de beenderen van die
dinosauriërs rap oprapen. Overheden, bedrijven en vakbonden passen zich
-- zij het onwilkeurig -- aan aan nieuwe, metaconstitutionele
voorwaarden, bepaald door de opkomst van microtechnologie. Deze
technologische doorbraak heeft de grenzen waarbinnen geweld wordt
uitgeoefend drastisch verlegd. Onze wereld is al veel meer veranderd dan
we ons normaal voorstellen, veel meer dan \emph{CNN} en de kranten ons
doen geloven. En de veranderingen sluiten precies aan bij wat studies
over megapolitieke condities voorspellen. Zoals we in \emph{Blood in the
Streets} en later in \emph{The Great Reckoning} betoogden: elke
wijziging in technologie of in andere factoren die de grenzen van geweld
bepalen, leidt onvermijdelijk tot een andere aard van de samenleving.
Alles wat te maken heeft met hoe mensen met elkaar omgaan -- van
moraliteit tot het alledaagse gezond verstand -- zal mee veranderen. Na
een periode van verslapte normen, die het einde van een tijdperk
markeert, komt een heropleving van een strengere moraliteit, met
veeleisender standaarden die passen bij de intensere voorwaarden van een
wereld waar soevereiniteit met elkaar concurreert.

We mogen verschillende kenmerken van de nieuwe moraliteit verwachten.
Enerzijds komt het belang van productiviteit centraal te staan, samen
met het principe dat mensen recht hebben op het behouden van de
beloningen die zij zelf verdienen. Daarnaast speelt efficiëntie bij
investeringen een cruciale rol. De ethiek van het informatietijdperk
stelt efficiëntie voorop en erkent dat middelen pas volledig tot hun
recht komen wanneer ze worden ingezet voor hun meest waardevolle
toepassingen. Kortom, deze nieuwe moraliteit belichaamt in essentie de
ethiek van de markt.

Zoals James Bennett betoogt, berust de moraliteit van het
informatietijdperk eveneens op vertrouwen. De cybereconomie vormt immers
een gemeenschap waarin vertrouwen hoog in het vaandel staat.

Wanneer onbreekbare encryptie een verduisteraar of dief in staat stelt
de opbrengsten van zijn misdrijven veilig buiten bereik te houden,
ontstaat er een krachtige prikkel om verliezen te voorkomen door
aanvankelijk geen transacties met zulke criminelen aan te gaan.

Zoals het voorbeeld van de Quakers al liet zien, telt een reputatie van
integriteit zwaar in de cybereconomie. In de anonimiteit van cyberspace
hoeft deze reputatie niet per se aan een bekende identiteit gekoppeld te
zijn; men kan deze immers op betrouwbare wijze verifiëren via
cryptografische sleutels.

Het vooruitzicht op ernstige problemen als gevolg van een corruptie in
de encryptie of de certificering van versleutelde identiteiten door
gangsters of anderen, biedt al voldoende reden om geen zaken te doen met
personen die onbetrouwbaar overkomen.

Bennett voorziet een `gentleman's club in cyberspace': beschermde zones
waarin deelname strengere veiligheidsmaatregelen vereist, eventueel
ondersteund door biometrische validatie zoals stemprintherkenning. De
beheerders nemen daarbij de verantwoordelijkheid om de identiteit -- en
tot op zekere hoogte de betrouwbaarheid -- van de deelnemers te
waarborgen. Zo ontstaat er een `gentleman's club in cyberspace' (hoewel
vrouwen uiteraard altijd welkom zijn). Binnen deze zones verrichten
mensen transacties met meer zekerheid en vertrouwen dan in het bredere
domein van cyberspace. Hierdoor zou de eenentwintigste eeuw een
terugkeer kunnen zien naar een op Victoriaanse waarden gebaseerde nadruk
op vertrouwen en karakter, in een context die geen enkele Victoriaan had
kunnen voorspellen.

Ook de beveiligde zones in cyberspace kunnen garanties bieden om
risico's te beperken. Ze doen immers denken aan de extraterritoriale
waarborgen die de graven van Champagne verleenden om kooplieden te
beschermen tijdens hun reizen naar en van de Champagne-markten. Andere
rechtsgebieden vergoedden zelfs reizende kooplieden voor eventuele
verliezen die zij opliepen bij het passeren van het grondgebied onder
het gezag van een edelman.

De `Guards of the Fair', ambtenaren die oorspronkelijk door de graven
werden aangesteld, waarborgden de beveiliging en fungeerden als een
`tribunaal van gerechtigheid' voor de marktkooplieden. In de loop der
tijd groeiden zij uit tot zelfstandige entiteiten met een eigen zegel,
die contracten legaliseerden en de nakoming daarvan afdwongen. Zij
kregen zelfs de bevoegdheid om een handelaar die er niet in slaagde zijn
schulden te voldoen of zijn contractuele verplichtingen na te komen, de
toegang tot toekomstige markten te ontzeggen. Deze straf bleek zo streng
dat maar weinigen bereid waren het risico te lopen op het mislopen van
toekomstige winstkansen. Als minder ingrijpende maatregel konden de
bewakers daarnaast de goederen van een wanbetaler in beslag nemen en ten
behoeve van zijn schuldeisers verkopen. \footnote{James Bennett,
  \emph{Cyberspace and the Return of Trust}, \emph{Strategic
  Investment}, oktober 1996.}

Naarmate het aantal alternatieve markten toenam, verloor het uitsluiten
van personen als handhavingsmiddel voor contracten aan belang. Dankzij
de opkomst van nieuwe informatietechnologie kan het uitsluiten van
oplichters en wanpresteerders in hun contractuele verplichtingen echter
opnieuw dienen als een krachtig mechanisme binnen de gefragmenteerde
soevereiniteiten van de volgende fase van de samenleving.
Computernetwerken stellen ons in staat cyberspace te monitoren met
onvervalsbare informatie over kredietwaardigheid en fraude. Doordat de
wereld op dit vlak een uiterst hechte gemeenschap vormt, worden
oplichters en fraudeurs effectief ontmoedigd.

Naast de nadruk op verdiensten en efficiëntie, en een hernieuwd accent
op karakter en vertrouwen, zal de nieuwe moraliteit hoogstwaarschijnlijk
ook een sterke afkeer van geweld benadrukken. Met name ontvoering en
afpersing -- middelen die steeds vaker worden ingezet om mensen af te
persen die anders minder snel het doelwit zouden worden van
criminaliteit -- zullen hierbij centraal staan.

Nog een mogelijke stimulans voor strengere moraliteit zal het einde
inluiden van uitkeringen en inkomensherverdeling. Als de hoop op steun
voor achterblijvers voornamelijk steunt op een beroep op particulieren
en liefdadigheidsinstellingen, wordt het belangrijker dan in de
twintigste eeuw dat de ontvangers van deze hulp moreel welverdiend
overkomen in de ogen van degenen die deze ondersteuning vrijwillig
verlenen.

\begin{quote}
``Subsidies, windfalls en de vooruitzichten op economische kansen nemen
de urgentie weg om te moeten besparen. De mantra's van democratie,
herverdeling en economische ontwikkeling verhogen zowel de verwachtingen
als de vruchtbaarheid, wat de bevolkingsgroei stimuleert en daarmee een
neerwaartse milieu- en economische spiraal versnelt.'' \footnote{Virginia
  Abernethy, \emph{Optimisme en overbevolking}, \emph{Atlantic Monthly},
  december 1994, p.~88.} - VIRGINIA ABERNATHY
\end{quote}

Op bepaalde terreinen staat de nieuwe informatiewereld beter in staat om
morele vraagstukken serieus aan te pakken. De beloftes van
inkomensherverdeling, die de hoop deden oplaaien bij de minder
fortuinlijken in de Verenigde Staten, Canada en West-Europa, hebben
internationaal vaak een averechts effect gehad. Er bestaat overtuigend
bewijs dat buitenlandse hulp en interventiebeloftes -- bedoeld om
hongersnood te voorkomen en de levensstandaard te verhogen --
belangrijke factoren zijn die de bevolkingsgroei stimuleren, zelfs
wanneer deze groei de draagkracht van achterblijvende economieën
overschrijdt. De opzienbarende groei van de wereldbevolking sinds de
Tweede Wereldoorlog, met de vaak verwoestende gevolgen voor bossen,
bodems en watervoorraden, vloeit voort uit wereldwijde interventies die
de negatieve terugkoppelingsmechanismen, waarbij lokale bevolkingen
lange tijd in evenwicht werden gehouden met de beschikbare hulpbronnen,
teniet hebben gedaan.

Degenen die in kleine gemeenschappen met weinig middelen en nauwelijks
groeikansen leefden, waren aanvankelijk zeer tevreden dat de beperkende
praktijken van hun dorpsleven zouden verdwijnen. Zij namen met
enthousiasme de positieve boodschap over, verkondigd door internationale
hulpverleners, vrijwilligers van het \emph{Peace Corps}, lokale
revolutionairen en de strijdende ideologen tijdens de Koude Oorlog, die
iedereen verzekerden dat er betere tijden in het verschiet lagen. Dit
bleek echter precies de verkeerde boodschap te zijn.

Een belangrijk gevolg van culturele herverdeling is dat mensen die in
niet-industriële samenlevingen leefden en trouw bleven aan traditionele
waarden, kunstmatig in concurrentie met elkaar kwamen te staan.
Internationale hulp, reddingsoperaties om hongersnood en epidemieën te
bestrijden en technologische interventies hebben velen voor de gek
gehouden, waardoor ze gingen geloven dat hun vooruitzichten drastisch
waren verbeterd, zonder dat ze hun waarden hoefden aan te passen of hun
gedrag wezenlijk te veranderen.

Internationale inkomensherverdeling heeft niet alleen geleid tot een
onhoudbare bevolkingsgroei wereldwijd, maar heeft ook in belangrijke
mate bijgedragen aan cultureel relativisme en tot wijdverspreide
verwarring over de cruciale rol van cultuur bij het mogelijk maken dat
mensen in hun eigen omgeving floreren. Tegenwoordig zijn de meesten
ervan overtuigd dat culturen meer een kwestie van smaak zijn dan bronnen
van gedragsrichtlijnen, die zowel kunnen misleiden als informeren. We
nemen te snel aan dat alle culturen gelijkwaardig zijn en wachten te
lang met het herkennen van de nadelen van contraproductieve
cultuurvormen. Dit geldt vooral voor de hybride culturen die zich in
deze eeuw in veel delen van de wereld ontwikkelen door subsidies en
overheidsinterventies. Vergelijkbaar met de criminele subculturen in de
binnensteden van Amerika dragen zij onsamenhangende fragmenten uit
vroegere culturele fasen met zich mee en mengen die met waarden die het
gedrag in het informatietijdperk bepalen.

De informatie-revolutie zal daarom niet alleen de geest van genialiteit
bevrijden, maar ook de drang tot vergelding ontketenen. Beide krachten
gaan in het komende millennium strijden als nooit tevoren.

De overgang van een industriële naar een informatiemaatschappij belooft
werkelijk adembenemend te worden. Elke verschuiving van het ene stadium
van economisch leven naar het volgende ging altijd gepaard met een
revolutie. Wij zijn ervan overtuigd dat de informatie-revolutie
waarschijnlijk de meest ingrijpende omwenteling wordt. Zij zal het leven
grondiger herstructureren dan zowel de agrarische als de industriële
revolutie, en de impact zal binnen een fractie van de tijd voelbaar
zijn. Zet uw gordels vast.


\backmatter


\end{document}
