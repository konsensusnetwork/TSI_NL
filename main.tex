% Options for packages loaded elsewhere
% Options for packages loaded elsewhere
\PassOptionsToPackage{unicode}{hyperref}
\PassOptionsToPackage{hyphens}{url}
%
\documentclass[
  dutch,
  a5paper,
  smalldemyvopaper,10pt,twoside,onecolumn,openright,extrafontsizes,hidelinks]{memoir}
\usepackage{xcolor}
\usepackage{amsmath,amssymb}
\setcounter{secnumdepth}{5}
\usepackage{iftex}
\ifPDFTeX
  \usepackage[T1]{fontenc}
  \usepackage[utf8]{inputenc}
  \usepackage{textcomp} % provide euro and other symbols
\else % if luatex or xetex
  \usepackage{unicode-math} % this also loads fontspec
  \defaultfontfeatures{Scale=MatchLowercase}
  \defaultfontfeatures[\rmfamily]{Ligatures=TeX,Scale=1}
\fi
\usepackage{lmodern}
\ifPDFTeX\else
  % xetex/luatex font selection
\fi
% Use upquote if available, for straight quotes in verbatim environments
\IfFileExists{upquote.sty}{\usepackage{upquote}}{}
\IfFileExists{microtype.sty}{% use microtype if available
  \usepackage[]{microtype}
  \UseMicrotypeSet[protrusion]{basicmath} % disable protrusion for tt fonts
}{}
\makeatletter
\@ifundefined{KOMAClassName}{% if non-KOMA class
  \IfFileExists{parskip.sty}{%
    \usepackage{parskip}
  }{% else
    \setlength{\parindent}{0pt}
    \setlength{\parskip}{6pt plus 2pt minus 1pt}}
}{% if KOMA class
  \KOMAoptions{parskip=half}}
\makeatother
% Make \paragraph and \subparagraph free-standing
\makeatletter
\ifx\paragraph\undefined\else
  \let\oldparagraph\paragraph
  \renewcommand{\paragraph}{
    \@ifstar
      \xxxParagraphStar
      \xxxParagraphNoStar
  }
  \newcommand{\xxxParagraphStar}[1]{\oldparagraph*{#1}\mbox{}}
  \newcommand{\xxxParagraphNoStar}[1]{\oldparagraph{#1}\mbox{}}
\fi
\ifx\subparagraph\undefined\else
  \let\oldsubparagraph\subparagraph
  \renewcommand{\subparagraph}{
    \@ifstar
      \xxxSubParagraphStar
      \xxxSubParagraphNoStar
  }
  \newcommand{\xxxSubParagraphStar}[1]{\oldsubparagraph*{#1}\mbox{}}
  \newcommand{\xxxSubParagraphNoStar}[1]{\oldsubparagraph{#1}\mbox{}}
\fi
\makeatother


\usepackage{longtable,booktabs,array}
\usepackage{calc} % for calculating minipage widths
% Correct order of tables after \paragraph or \subparagraph
\usepackage{etoolbox}
\makeatletter
\patchcmd\longtable{\par}{\if@noskipsec\mbox{}\fi\par}{}{}
\makeatother
% Allow footnotes in longtable head/foot
\IfFileExists{footnotehyper.sty}{\usepackage{footnotehyper}}{\usepackage{footnote}}
\makesavenoteenv{longtable}
\usepackage{graphicx}
\makeatletter
\newsavebox\pandoc@box
\newcommand*\pandocbounded[1]{% scales image to fit in text height/width
  \sbox\pandoc@box{#1}%
  \Gscale@div\@tempa{\textheight}{\dimexpr\ht\pandoc@box+\dp\pandoc@box\relax}%
  \Gscale@div\@tempb{\linewidth}{\wd\pandoc@box}%
  \ifdim\@tempb\p@<\@tempa\p@\let\@tempa\@tempb\fi% select the smaller of both
  \ifdim\@tempa\p@<\p@\scalebox{\@tempa}{\usebox\pandoc@box}%
  \else\usebox{\pandoc@box}%
  \fi%
}
% Set default figure placement to htbp
\def\fps@figure{htbp}
\makeatother



\ifLuaTeX
\usepackage[bidi=basic]{babel}
\else
\usepackage[bidi=default]{babel}
\fi
% get rid of language-specific shorthands (see #6817):
\let\LanguageShortHands\languageshorthands
\def\languageshorthands#1{}


\setlength{\emergencystretch}{3em} % prevent overfull lines

\providecommand{\tightlist}{%
  \setlength{\itemsep}{0pt}\setlength{\parskip}{0pt}}



 


% typographical packages
\usepackage{microtype}
\usepackage{setspace}
% \tolerance=6000
% \hyphenpenalty=1000
\usepackage[all]{nowidow}

% typographical settings for the body text
\setlength{\parskip}{0em}
\setlength{\parindent}{1em}
\linespread{1}

% DEFINITIONS TITLE PAGE / COPYRIGHT
\newcommand{\titleoriginal}{The Sovereign Individual}
\newcommand{\subtitleoriginal}{Mastering the Transition to the Information Age}
\newcommand{\yearoriginal}{1999}
\newcommand{\subtitletranslation}{De overgang naar het informatietijdperk}
\newcommand{\yeartranslation}{2025}
\newcommand{\stringtranslation}{Vertaling: }
\newcommand{\stringlicense}{XXX}
\newcommand{\stringpublisher}{Konsensus Network}
\newcommand{\ISBNHC}{XXX}
\newcommand{\ISBNSC}{XXX}
\newcommand{\ISBNEBOOK}{XXX}
\newcommand{\ISBNAUDIO}{XXX}
\newcommand{\press}{Konsensus Network}
\newcommand{\translatorone}{Vincent Hondius}
\newcommand{\translatortwo}{Theo Hague}
\newcommand{\translators}{
\large\textit{\stringtranslation:}\\
\translatorone\\
\translatortwo\\
}

% PHYSICAL DOCUMENT SETUP
\setstocksize{210mm}{148mm}
\settrimmedsize{210mm}{148mm}{*}
\setbinding{8mm}
\setlrmarginsandblock{16mm}{18mm}{*}
\setulmarginsandblock{16mm}{15mm}{*}
%\setlength{\skip\footins}{18pt} % More space between the text and the footnote line

% FONTS
\usepackage{fontspec}
\setmainfont{stone-serif}[
    Path=./fonts/stone-serif-itc-pro/,
    Scale=0.85,
    Extension=.OTF,
    UprightFont=*-Regular,
    BoldFont=*-SemiBd,
    ItalicFont=*-MediumIt,
    BoldItalicFont=*-SemiBdIt
    ]

\setsansfont{stone-sans}[
    Path=./fonts/stone-sans/,
    Scale=0.85,
    Extension=.otf,
    UprightFont=*-Medium,
    BoldFont=*-Semibold,
    ItalicFont=*-MediumItalic,
    BoldItalicFont=*-SemiBoldItalic
    ]

\usepackage{lettrine}
\setcounter{DefaultLines}{3}
\renewcommand{\DefaultLoversize}{0.1}
\renewcommand{\DefaultLraise}{0}
\renewcommand{\LettrineTextFont}{}
\setlength{\DefaultFindent}{\fontdimen2\font}
\setlength{\DefaultNindent}{0em}

% Define YUGE
\newcommand{\YUGE}{\fontsize{36pt}{43pt}\selectfont}
\newcommand{\Yuge}{\fontsize{30pt}{36pt}\selectfont}
\newcommand{\yuge}{\fontsize{27pt}{32pt}\selectfont}

% custom second title page
\makeatletter
\newcommand*\halftitlepage{\begingroup % Misericords, T&H p 153
  \setlength\drop{0.1\textheight}
  %\begin{center}
  \vspace*{\drop}
  \rule{\textwidth}{0in}\par
  {\Large\sffamily\thetitle\par}
  \rule{\textwidth}{0in}\par
  \vfill
  %\end{center}
\endgroup}
\makeatother

% custom title page
\makeatletter
\newlength\drop
\newcommand*\titleM{\begingroup % Misericords, T&H p 153
  \setlength\drop{0.15\textheight}
  %\begin{center}
  \vspace*{\drop}
  {\HUGE\sffamily\thetitle\par}
  \vspace{2em}
  {\Large\sffamily\textit\subtitletranslation\par}
  \vspace{4em}
  \rule{5.5cm}{0.3mm}\par
  \vspace{4em}
  {\Large\sffamily\textit\theauthor\par}
  \vspace{6em}
  % {\footnotesize\sffamily\textit\translators\par}
  \vfill
  \includegraphics[width=3.5cm]{figures/knw.png}\par
  %\end{center}
\endgroup}
\makeatother

% copyright page
\makeatletter
\newcommand*\copyrightpage{\begingroup
  \setlength\drop{0.1\textheight}
  \vphantom{just for the drop}
  \vfill
  \begin{footnotesize}
  \noindent \copyright\space \yearoriginal: \theauthor
  \par\noindent \textit{\titleoriginal: \subtitleoriginal}
  \vspace{0.5\baselineskip}
  \par\noindent \copyright\space \yeartranslation\space \stringtranslation: \translatorone
  \par\noindent \textit{\thetitle: \subtitletranslation}
  \vspace{\baselineskip}
  \par\noindent \textit{\stringlicense}
  \vspace{0.5\baselineskip}
  \par\noindent \stringpublisher: \href{https://konsensus.network}{\textit{konsensus.network}}
  \vspace{0.5\baselineskip}
  \par\noindent v1.0.0
  \vspace{0.5\baselineskip}
  \setlength{\parindent}{2em}% default 20pt
  \par\noindent ISBN \ISBNHC \:Hardcover
  \par\hspace{0.28\parindent}\ISBNSC \:Paperback
  \par\hspace{0.28\parindent}\ISBNEBOOK \:E-book\par
  \setlength{\parindent}{0pt}
  \end{footnotesize}
  \vspace{3em}
  \par\noindent \href{https://konsensus.network}{\includegraphics[width=1cm]{figures/freestarfish.png}}
  \par\noindent \href{https://konsensus.network}{\includegraphics[width=3.5cm]{figures/knw.png}}
  \setcounter{footnote}{0}
  \clearpage
\endgroup}
\makeatother

% HEADER AND FOOTER MANIPULATION
% for normal pages
\nouppercaseheads
\headsep = 6mm
\makepagestyle{mystyle} 
\makeevenhead{mystyle}{\scriptsize\sffamily\thepage}{}{}
\makeoddhead{mystyle}{\scriptsize\sffamily\leftmark}{}{\scriptsize\sffamily\thepage}
\makeevenfoot{mystyle}{}{}{}
\makeoddfoot{mystyle}{}{}{}
\makeatletter

% for pages where chapters begin
\makepagestyle{plain}
\makerunningwidth{plain}{\headwidth}
\makeevenfoot{plain}{}{}{}
\makeoddfoot{plain}{}{}{}
\pagestyle{mystyle}

\newif\ifmainmatter
\appto\mainmatter{\mainmattertrue}
\appto\backmatter{\mainmatterfalse}
\appto\appendix{\mainmatterfalse}

\renewcommand\chaptermark[1]{%
  \markboth{\MakeUppercase{%
    \ifmainmatter~\oldstylenums\thechapter.~\fi#1}}{}}%

% TOC
\usepackage[]{tocloft}
\renewcommand{\cftsectiondotsep}{\cftnodots}
\renewcommand{\cftpartfont}{\Large\sffamily\MakeUppercase}
\renewcommand{\cftchapterfont}{\small\sffamily}
\renewcommand{\cftsectionfont}{\Small\sffamily}
\renewcommand{\cftpartpagefont}{\Large\sffamily}
\renewcommand{\cftchapterpagefont}{\small}
\renewcommand{\cftchapterpresnum}{HOOFDSTUK\space}
\renewcommand{\cftchapternumwidth}{7em}
\setlength{\cftchapterindent}{0em}
\setlength{\cftsectionindent}{5em}
\setlength{\cftbeforechapterskip}{0.8em}
\setsecnumdepth{chapter}
\setcounter{tocdepth}{0}


% Redefine footnote presentation
\makeatletter
\renewcommand\@makefntext[1]{%
  \noindent\hb@xt@2em{% <-- Box of fixed size for footnote number and space
    \@thefnmark\quad}% <-- Footnote number followed by a quad space
  \parbox[t]{\dimexpr\linewidth-2em}{#1}% <-- Parbox to control the width of footnote content
}
\makeatother

% layout check and fix
\checkandfixthelayout

% COUNTERS FOOTNOTES
\usepackage{chngcntr}
\counterwithout*{footnote}{chapter}

% TITLE FORMATTING
\usepackage{titlesec}

% Define chapter format with titlesec
\titleformat
    {\chapter}[display]
    {\huge\sffamily} % Main title font style
    {\Large\sffamily\chaptertitlename~\thechapter} % "Chapter N" format
    {0pt} % Space between the chapter number and title
    {\Huge} % Chapter title formatting
    [\vspace{10pt}\Large\textit{\chaptersubtitle}] % Subtitle formatting

% Command to set the subtitle (empty by default)
\newcommand{\chaptersubtitle}{}

% Automatically render the subtitle (if set) after the chapter title
\titleformat{\chapter}[display]
  {\huge\sffamily}
  {\Large\sffamily\chaptertitlename\ \thechapter}
  {0pt}
  {\Huge}
  [\ifx\chaptersubtitle\empty\else\vspace{10pt}\Large\textit{\chaptersubtitle}\fi]

% Command to set subtitle manually after chapter rendering
\newcommand{\setsubtitle}[1]{%
  \renewcommand{\chaptersubtitle}{#1}%
  \chaptermark{\chaptersubtitle} % Update subtitle for header/footer
}

\titleformat
  {\section}[display]
  {\sffamily\large\bfseries}
  {\thesection}
  {0pt}
  {\MakeUppercase}

% \makeatletter
% \renewcommand{\section}{\@startsection {section}{1}{\z@}%
%               {-3.5ex \@plus -1ex \@minus -.2ex}%
%               {2.3ex \@plus .2ex}%
%               {\sffamily\large\bfseries\MakeUppercase}}
% \makeatother

\titlespacing*{\section}{0pt}{2em}{0em}

\titleformat{\subsection}{\sffamily\bfseries}{}{}{}
\titlespacing*{\subsection}{0pt}{2em}{0em}

% QUOTE FORMATTING
\renewenvironment{quote}%
               {\list{}{\rightmargin=.6cm\leftmargin=.6cm}%
                \itshape \item[]}% <- The effect of \samepage is local!!!
               {\endlist}

% LAYOUT CHECK AND FIX
\checkandfixthelayout

% CUSTOM TITLE PAGE
\makeatletter
\def\@maketitle{%
  % the half title page
  \pagestyle{empty}
  \halftitlepage
  \cleardoublepage

  % the title page
  \titleM
  \clearpage

  % the copyright page
  \copyrightpage
  \cleardoublepage
  \pagestyle{mystyle}
}
\makeatother
% END PREAMBLE
\makeatletter
\@ifpackageloaded{bookmark}{}{\usepackage{bookmark}}
\makeatother
\makeatletter
\@ifpackageloaded{caption}{}{\usepackage{caption}}
\AtBeginDocument{%
\ifdefined\contentsname
  \renewcommand*\contentsname{Inhoudsopgave}
\else
  \newcommand\contentsname{Inhoudsopgave}
\fi
\ifdefined\listfigurename
  \renewcommand*\listfigurename{Lijst van figuren}
\else
  \newcommand\listfigurename{Lijst van figuren}
\fi
\ifdefined\listtablename
  \renewcommand*\listtablename{Lijst van tabellen}
\else
  \newcommand\listtablename{Lijst van tabellen}
\fi
\ifdefined\figurename
  \renewcommand*\figurename{Figuur}
\else
  \newcommand\figurename{Figuur}
\fi
\ifdefined\tablename
  \renewcommand*\tablename{Tabel}
\else
  \newcommand\tablename{Tabel}
\fi
}
\@ifpackageloaded{float}{}{\usepackage{float}}
\floatstyle{ruled}
\@ifundefined{c@chapter}{\newfloat{codelisting}{h}{lop}}{\newfloat{codelisting}{h}{lop}[chapter]}
\floatname{codelisting}{Listing}
\newcommand*\listoflistings{\listof{codelisting}{Lijst van listings}}
\makeatother
\makeatletter
\makeatother
\makeatletter
\@ifpackageloaded{caption}{}{\usepackage{caption}}
\@ifpackageloaded{subcaption}{}{\usepackage{subcaption}}
\makeatother
\usepackage{bookmark}
\IfFileExists{xurl.sty}{\usepackage{xurl}}{} % add URL line breaks if available
\urlstyle{same}
\hypersetup{
  pdftitle={Het soevereine individu},
  pdfauthor={James Dale Davidson \& Lord William Rees-Mogg},
  pdflang={nl},
  hidelinks,
  pdfcreator={LaTeX via pandoc}}


\title{Het soevereine individu}
\usepackage{etoolbox}
\makeatletter
\providecommand{\subtitle}[1]{% add subtitle to \maketitle
  \apptocmd{\@title}{\par {\large #1 \par}}{}{}
}
\makeatother
\subtitle{De overgang naar het informatietijdperk}
\author{James Dale Davidson \& Lord William Rees-Mogg}
\date{2025-10-17}
\begin{document}
\frontmatter
\maketitle

\renewcommand*\contentsname{Contents}
{
\setcounter{tocdepth}{0}
\tableofcontents
}

\mainmatter
\bookmarksetup{startatroot}

\chapter*{Over dit boek}\label{over-dit-boek}

\markboth{Over dit boek}{Over dit boek}

\section*{Inhoudsopgave}\label{inhoudsopgave}
\addcontentsline{toc}{section}{Inhoudsopgave}

\markright{Inhoudsopgave}

\begin{itemize}
\tightlist
\item
  \begin{itemize}
  \tightlist
  \item
    \href{./chapters/ch01.html}{De overgang in het jaar 2000}
  \item
    \href{./chapters/ch02.html}{Megapolitieke transformaties in
    historisch perspectief}
  \item
    \href{./chapters/ch03.html}{Ten oosten van Eden}
  \item
    \href{./chapters/ch04.html}{De laatste dagen van de politiek}
  \item
    \href{./chapters/ch05.html}{Het leven en de gezondheid van de
    natiestaat}
  \item
    \href{./chapters/ch06.html}{De megapolitiek van het
    informatietijdperk}
  \item
    \href{./chapters/ch07.html}{Het overstijgen van plaatsgebondenheid}
  \item
    \href{./chapters/ch08.html}{Het einde van egalitaire economie}
  \item
    \href{./chapters/ch09.html}{Nationalisme, reactie en de nieuwe
    ludditen}
  \item
    \href{./chapters/ch10.html}{De schemering van de democratie}
  \item
    \href{./chapters/ch11.html}{Moraal en misdaad in de `natuurlijke
    economie' van het informatietijdperk}
  \item
    \href{./chapters/nawoord.html}{Nawoord: decentralisatie en de wet
    van de afnemende meeropbrengst}
  \end{itemize}
\end{itemize}

\bookmarksetup{startatroot}

\chapter{De overgang in het jaar
2000}\label{de-overgang-in-het-jaar-2000}

\begin{quote}
Het lijkt wel alsof er iets monumentaals op til is: grafieken tonen de
jaarlijkse groei van bevolkingsaantallen, de concentratie van
koolstofdioxide in de atmosfeer, het aantal webadressen en de megabytes
per dollar. Alle cijfers stijgen naar een asymptoot net na de
eeuwwisseling: de singulariteit. Het einde van alles wat we kennen. Het
begin van iets wat we wellicht nooit zullen doorgronden. -- Danny
Hillis\footnote{Bois, op. cit., p.136.}
\end{quote}

\section{Voorgevoelens}\label{voorgevoelens}

Het jaar 2000 houdt de westerse verbeelding al eeuwenlang in zijn greep.
Nadat men destijds verwachtte dat de wereld bij de millenniumwisseling
van het eerste millennium na Christus zou instorten, keken theologen,
evangelisten, dichters en zieners reikhalzend uit naar een decennium vol
ingrijpende gebeurtenissen. Zelfs de beroemde Isaac Newton speculeerde
ooit dat de wereld in het jaar 2000 ten onder zou gaan. Michel de
Nostradamus, wiens profetieën al sinds 1568 door generaties zijn
bekeken, voorspelde de komst van de derde antichrist in juli
1999.\footnote{Lane, \emph{Economic Consequences of Organized Violence},
  op. cit.} De Zwitserse psycholoog Carl Jung, expert op het gebied van
het `collectieve onbewuste', zag al in 1997 de geboorte van een nieuw
tijdperk opdoemen. Men maakt zulke voorspellingen vaak belachelijk, net
als de nuchtere prognoses van economen zoals Dr.~Edward Yardeni van
\emph{Deutsche Bank Securities}, die verwachte dat computerstoringen op
middernacht van het millennium de hele wereldeconomie compleet zouden
ontwrichten.\footnote{Dr.~Edward Yardeni, \emph{Year 2000 Recession}:
  `Prepare for the worst. Hope for the best', Versie 5.0, 13 mei 1998,
  B1.2.} Of je het Y2K-probleem nu beschouwt als ongegronde hysterie,
aangewakkerd door programmeurs en adviseurs op het gebied van
informatietechnologie, of als een mysterieus fenomeen waarin technologie
zich ontvouwt in samenhang met profetische ideeën, je kunt niet
ontkennen dat de drempel van het millennium voor veel meer opwinding
zorgt dan de gebruikelijke, morbide onzekerheid over de toekomst.

Het optimisme dat westerse samenlevingen de afgelopen 250 jaar
kenmerkte, wordt nu overschaduwd door een gevoel van naderend onheil.
Overal zie je het terug: in de gezichtsuitdrukkingen van mensen, in hun
gesprekken, in peilingen en zelfs op de stembus. Net zoals dat een
subtiele verandering in de ionen in de atmosfeer, nog voordat de wolken
donker worden en de bliksem slaat, al verraadt dat er een onweersbui aan
komt, zo hangt er in de schemering van het millennium een voorgevoel van
verandering in de lucht. Men voelt op ieders eigen manier dat hun
levenswijze ten dode opgeschreven is. Bij het aflopen van het decennium,
komen zowel een moorddadige eeuw als een glorieus millenium vol
menselijke verwezenlijkingen tot zijn eind. Alles vindt zijn einde in
het jaar 2000.

\begin{quote}
Want wat verborgen is zal ontdekt worden, en wat geheim is, zal bekend
worden. -- Mattheüs 10:26
\end{quote}

Wij zijn ervan overtuigd dat de moderne fase van de westerse beschaving
ten einde loopt, en in dit boek leggen we uit waarom. Net als veel
andere werken proberen we via een donkere spiegel te kijken en de vage
contouren en verhoudingen van een naderende toekomst te schetsen. In
zekere zin is het de bedoeling van dit werk om `apocalyptisch' te zijn
in de letterlijke betekenis van het woord. `Apokalypsis' is Grieks voor
`onthulling'. Wij geloven dat een nieuw tijdperk, het
informatietijdperk, op het punt staat zich te `onthullen'.

\begin{quote}
We observeren de eerste tekenen van een nieuwe logische ruimte, een
ogenblikkelijke elektronische alomtegenwoordigheid waar iedereen toegang
toe heeft, in kan gaan en kan ervaren. Kortom, we bevinden ons in de
begindagen van een nieuwe gemeenschapsvorm. De virtuele gemeenschap
wordt het model voor een seculier koninkrijk der hemelen; zoals Jezus
zei dat er vele huizen waren in het koninkrijk van zijn Vader, zo zijn
er ook talloze virtuele gemeenschappen, elk afgestemd op hun eigen
wensen en behoeften. -- Michael Grasso\footnote{Michael Grasso,
  \emph{The Millenium Myth: Love and Death at the End of Time}, Wheaton,
  Illinois: Quest Books, 1995.}
\end{quote}

\section{De vierde fase van de menselijke
maatschappij}\label{de-vierde-fase-van-de-menselijke-maatschappij}

Het thema van dit boek is de opkomst van een nieuwe machtsrevolutie die
individuen bevrijdt en de traditionele natiestaat van de twintigste eeuw
verdringt. Innovaties die de logica van geweld radicaal herschrijven,
verschuiven de grenzen waarbinnen de toekomst vorm krijgt. Als onze
conclusies kloppen, sta je op de drempel van de meest ingrijpende
revolutie in de geschiedenis. Veel sneller dan de meesten zich
voorstellen, zal de microprocessor de natiestaat ondermijnen en
uiteindelijk vernietigen, waarna nieuwe vormen van sociale organisatie
zullen ontstaan. Deze transformatie wordt allesbehalve eenvoudig.

De uitdaging die eraan verbonden is, wordt des te groter omdat deze zich
in een ongekend tempo voltrekt, veel sneller dan we in het verleden
hebben gezien. In de gehele menselijke geschiedenis, van de oertijd tot
nu, kennen we slechts drie fundamentele ontwikkelingsstadia: (1)
jager-verzamelaarsmaatschappijen, (2) agrarische samenlevingen en (3)
industriële samenlevingen. Aan de horizon doemt nu iets compleet nieuws
op: Informatiemaatschappijen, het vierde stadium van sociale
organisatie.

Bij elk van de voorgaande samenlevingsstadia kende men een eigen,
duidelijk afgebakende fase in de evolutie en beheersing van geweld.
Zoals we uitvoerig toelichten, zullen Informatiemaatschappijen de
voordelen van geweld drastisch verkleinen, mede omdat ze niet langer
gebonden zijn aan één vaste locatie. De virtuele wereld van cyberspace,
die romanschrijver William Gibson omschreef als een `consensuele
hallucinatie', zal voor pestkoppen zo onbereikbaar blijven als de
verbeelding reikt. In dit nieuwe millennium zal de winst uit de
grootschalige beheersing van geweld veel lager uitvallen dan op welk
moment sinds vóór de Franse Revolutie ook. Dit heeft verstrekkende
gevolgen. Eén daarvan is een toename van criminaliteit. Als de beloning
voor grootschalig geweld instort, zal de prikkel voor kleinschalig
geweld naar alle waarschijnlijkheid juist groeien. Geweld krijgt dan een
meer willekeurige, lokale aard, terwijl georganiseerde criminaliteit in
omvang toeneemt; wij lichten hierover verder toe.

Een andere logische consequentie van de afnemende voordelen van geweld
is het vervagen van de politieke rol. Er is veel bewijs dat het
vertrouwen in de burgerlijke mythen van de twintigste-eeuwse natiestaat
snel afneemt. De ondergang van het communisme is slechts één opvallend
voorbeeld. Zoals we in detail onderzoeken, bewijzen de afbrokkelende
moraliteit en de groeiende corruptie onder leiders van westerse
regeringen dat het potentieel van de natiestaat is uitgeput (het gaat
hier niet om een willekeurige ontwikkeling). Zelfs vele leiders geloven
de holle frasen die zij verkondigen niet meer, en ook het publiek
schenkt die woorden geen vertrouwen meer.

\subsection{Geschiedenis herhaalt
zich}\label{geschiedenis-herhaalt-zich}

Deze situatie vertoont opvallende overeenkomsten met het verleden.
Telkens wanneer technologische veranderingen de oude structuren losmaken
van de nieuwe economische drijfkrachten, veranderen ook de morele normen
en gaan mensen degenen die de oude instituties leiden met groeiende
minachting behandelen. Deze wijdverspreide afkeer komt vaak al duidelijk
naar voren voordat er zelfs een nieuwe, samenhangende ideologie van
verandering ontstaat. Zo was dat bijvoorbeeld in de late vijftiende
eeuw, toen de middeleeuwse kerk nog het overheersende instituut van het
feodalisme vormde. Ondanks het brede geloof in `de heiligheid van het
priesterlijke ambt', behandelden mensen zowel de hoge als de lage
geestelijkheid met grote minachting. Dit is vergelijkbaar met de
populaire houding ten opzichte van politici en bureaucraten in deze
tijd.\footnote{Johan Huizinga, \emph{The Waning of the Middle Ages},
  vert. E Hopman (Londen: Penguin Books, 1990), p.~172.}

We zijn ervan overtuigd dat we veel kunnen leren door het einde van de
vijftiende eeuw, toen het leven doordrenkt was met georganiseerde
religie, te vergelijken met de hedendaagse wereld, waarin politiek
overal de boventoon voert. Destijds waren de kosten om de
geïnstitutionaliseerde religie in stand te houden historisch extreem
hoog, net zoals tegenwoordig de kosten voor het ondersteunen van de
overheid absurd ver zijn opgelopen.

We hebben gezien hoe de georganiseerde religie veranderde na de
buskruitrevolutie. Technologische innovaties stimuleerden het verkleinen
van religieuze instellingen en dwongen tot drastische kostenbesparingen.
Een vergelijkbare technologische revolutie dreigt aan het begin van dit
millennium de natiestaat radicaal te verkleinen.

\begin{quote}
Vandaag, na meer dan een eeuw aan elektronische technologie, hebben we
ons centrale zenuwstelsel zelf uitgebreid in een mondiale omarming,
waarbij we zowel ruimte als tijd, voor zover onze planeet betreft,
hebben afgeschaft.\footnote{Marshall McLuhan, \emph{Understanding
  Media}, New York: Signet, 196, p.~19.}
\end{quote}

\subsection{De informatierevolutie}\label{de-informatierevolutie}

Naarmate grote systemen verder instorten, verliest systematische dwang
zijn bepalende rol in het vormgeven van het economisch leven en de
inkomensverdeling. Efficiëntie krijgt al snel voorrang boven de bevelen
van de machthebbers bij het organiseren van sociale instituties. Dit
houdt in dat provincies en zelfs steden die effectief eigendomsrechten
bewaken en de rechtspraak draaiende houden, terwijl ze weinig middelen
verbruiken, in het informatietijdperk als levensvatbare soevereiniteiten
kunnen functioneren, iets wat de afgelopen vijf eeuwen zelden het geval
was.

Binnen cyberspace ontstaat een geheel nieuw terrein voor economische
activiteiten dat niet vatbaar is voor fysiek geweld. De grootste
voordelen zullen vooral de `cognitieve elite' treffen, die zich steeds
minder laat beperken door politieke grenzen. Zij voelen zich net zo
thuis in Frankfurt als in Londen, New York, Buenos Aires, Los Angeles,
Tokio en Hong Kong. Binnen de rechtsgebieden zal de inkomensongelijkheid
toenemen, terwijl tussen deze gebieden juist meer gelijkheid ontstaat.

\emph{Het Soevereine Individu} onderzoekt de maatschappelijke en
financiële gevolgen van deze revolutionaire ommekeer. Wij willen je
helpen de kansen van dit nieuwe tijdperk te grijpen en ervoor zorgen dat
de impact ervan je niet verwoest. Zelfs als slechts de helft van onze
verwachtingen uitkomt, ondervind je een verandering van ongekende
proporties, zonder werkelijk historisch precedent.

De transformatie in het jaar 2000 verandert radicaal het karakter van de
wereldeconomie en gebeurt bovendien veel sneller dan elke voorgaande
fasewisseling. In tegenstelling tot de agrarische revolutie kost het de
informatierevolutie geen millennia om haar volledige impact te bereiken,
en in tegenstelling tot de industriële revolutie spreidt haar effect
zich niet uit over een periode van eeuwen, maar ervaar je de volledige
reikwijdte ervan binnen één mensenleven.

Daarbij vindt deze transformatie vrijwel gelijktijdig over de hele
wereld plaats. Technologische en economische innovaties beperken zich
niet langer tot geïsoleerde uithoeken van de wereld. De ommekeer raakt
vrijwel alle hoeken van de aarde en markeert een breuk met het verleden
die zo ingrijpend is dat hij het bijna mythische rijk van de goden tot
leven wekt, zoals de oude Grieken ooit voorstelden.

Veel meer dan men zich nu durft voor te stellen, zal blijken dat het in
het nieuwe millennium buitengewoon lastig, zo niet onmogelijk, is om
veel van onze hedendaagse instellingen te behouden. Zodra
informatiesamenlevingen opkomen, onderscheiden ze zich van industriële
samenlevingen op dezelfde manier als het Griekenland van Aeschylus
verschilde van de wereld van de grotbewoners.

\section{\texorpdfstring{\emph{`Prometheus unbound'}: De opkomst van het
Soevereine
Individu}{`Prometheus unbound': De opkomst van het Soevereine Individu}}\label{prometheus-unbound-de-opkomst-van-het-soevereine-individu}

\begin{quote}
Ik ken geen bemoedigender feit dan het onmiskenbare vermogen van de mens
om zijn leven te verrijken door doelgerichte inzet -- Henry David
Thoreau
\end{quote}

De aanstaande transformatie brengt zowel goed als slecht nieuws met zich
mee. Het positieve is dat de informatierevolutie mensen als nooit
tevoren bevrijdt. Voor het eerst geniet degene die zichzelf kan
bijscholen de volledige vrijheid om eigen ideeën te ontwikkelen en
maximaal te profiteren van zijn productiviteit. Genialiteit komt tot
bloei wanneer overheidsinmenging stopt en raciale en etnische
vooroordelen hun invloed verliezen. Als je echt bekwaam bent, laat je je
in de Informatiemaatschappij niet tegenhouden door bekrompen
opvattingen. Het maakt niet uit wat de meeste mensen van je ras,
uiterlijk, leeftijd, seksuele voorkeur of kapsel vinden. In de
cybereconomie blijft je identiteit anoniem. Al ben je misschien minder
fraai, mollig, ouder of heb je een beperking; dankzij de volledige
kleurenblinde anonimiteit op de nieuwe frontlinies van de cyberspace,
concurreert iedereen op gelijke voet met de jonge en knappe onder ons,.

\subsection{Ideeën worden rijkdom}\label{ideeuxebn-worden-rijkdom}

Zodra competenties zich openbaren, worden ze beloond als nooit tevoren.
In een wereld waarin ideeën de grootste bron van rijkdom vormen in
plaats van louter fysiek kapitaal, heeft iedereen met een heldere geest
de mogelijkheid om rijk te worden. Het informatietijdperk belooft een
tijdperk van sociale mobiliteit te worden en opent talloze gelijke
kansen voor de miljarden mensen in regio's die nooit ten volle hebben
geprofiteerd van de welvaart van de industriële samenleving. De
slimsten, succesvolste en meest ambitieuze treden op als ware
zelfstandigen.

In eerste instantie bereikt slechts een enkeling volledige financiële
soevereiniteit, maar dat doet niets af aan de voordelen van financiële
onafhankelijkheid. Dat niet iedereen een gigantisch fortuin vergaart,
betekent immers niet dat het streven naar rijkdom zinloos is. Er zijn
25.000 miljonairs voor elke miljardair. Als je wel miljonair bent maar
geen miljardair, ben je zeker niet arm. In de toekomst zul je je
financiële succes niet alleen koppelen aan het aantal nullen van je
vermogen, maar ook aan de mate waarin je erin slaagt om volledige
persoonlijke autonomie te bereiken. Hoe slimmer je bent, hoe minder
energie je nodig hebt om de overgang naar financiële onafhankelijkheid
te maken. Zelfs mensen met bescheiden middelen zullen vooruitkomen zodra
de politieke druk op de wereldeconomie afneemt. Onovertroffen financiële
onafhankelijkheid wordt dan een haalbaar doel voor jou en je kinderen.

Op het hoogste productiviteitsniveau concurreren en communiceren
onafhankelijke individuen op een wijze die doet denken aan de relaties
tussen de Griekse goden. De ongrijpbare berg Olympus van het komende
millennium bevindt zich in cyberspace -- een wereld zonder fysieke vorm
die desalniettemin tegen het tweede decennium van dit nieuwe millennium
uit zal groeien tot 's werelds grootste economie. Tegen 2025 telt de
cybereconomie miljoenen deelnemers. Sommigen zullen een vermogen
ontwikkelen vergelijkbaar met dat van Bill Gates -- elk met een waarde
van meer dan 10 miljard dollar -- terwijl de cyberarmen bestaan uit
mensen die minder dan 200.000 dollar per jaar verdienen. Er komt geen
cyberwelzijn, geen cybertaksen én geen cyberregering. De cybereconomie
kan, in plaats van China, wel eens het grootste economische fenomeen van
de komende dertig jaar worden.

Het goede nieuws is dat politici in dit nieuwe rijk de handel niet
zullen beheersen, onderdrukken of reguleren, net zoals de wetgevers van
de oude Griekse stadstaten zeker niet in staat waren om een stukje van
Zeus' baard te knippen. Dat komt de rijken ten goede en is nog beter
nieuws voor de minder vermogenden. De door de politiek opgelegde
obstakels en lasten belemmeren immers meer het rijk worden dan het rijk
blijven. De voordelen van afnemende opbrengsten uit geweld en het
decentraliseren van rechtsgebieden zullen voor ieder energiek en
ambitieus persoon ruimte creëren om te profiteren van de dood van de
politiek. Ook de afnemers van overheidsdiensten zullen hier voordeel uit
halen, omdat ondernemers de vruchten van de concurrentie verder
uitbreiden. Veel van de vindingrijkheid werd vroeger juist gekanaliseerd
naar militaire inspanningen omdat er nog weinig ruimte was voor
economische concurrentie tussen jurisdicties. Maar de opkomst van de
cybereconomie zal zorgen voor nieuwe vormen van concurrentie bij het
leveren van diensten die traditioneel door de staat werden uitgevoerd.
Met meer rechtsgebieden ontstaat er meer ruimte om te experimenteren met
nieuwe methoden voor het afdwingen van contracten en het beschermen van
mensen en hun eigendommen. Het vrijmaken van een groot deel van de
wereldeconomie van politieke controle dwingt de resterende overheden om
marktgerichtere methoden toe te passen. Op den duur hebben ze geen
andere optie meer dan de inwoners van hun jurisdicties als klanten te
behandelen, in plaats van hen te onderwerpen aan afpersing, net zoals de
georganiseerde misdaad.

\subsection{Voorbij de politiek}\label{voorbij-de-politiek}

Wat de mythologie ooit als het domein van de goden afschilderde, wordt
voor het individu een haalbare keuze: een leven buiten de invloed van
koningen en raden. Eerst zullen tientallen mensen, daarna honderden en
uiteindelijk miljoenen zich bevrijden uit de greep van de politiek.
Terwijl zij zich hiervan ontdoen, hervormen zij de werking van
overheden, waardoor de sfeer van dwang afneemt en particuliere controle
over middelen toeneemt.

De opkomst van het autonome individu zal opnieuw de bijzondere
voorspellende kracht van mythen bevestigen. Vroege agrarische volkeren,
die nauwelijks begrip hadden van de natuurwetten, verbeeldden zich dat
`bovennatuurlijke krachten' overal aanwezig waren. Deze krachten werden
soms aangewend door mensen of door `goden in menselijke gedaante', die
op mensen leken en onder hen leefden, zoals Sir James George Frazer in
\emph{The Golden Bough} omschreef als `een universele
democratie'.\footnote{James George Frazer, \emph{The Golden Bough: A
  Study in Magic and Religion} (New York: Macmillan, 1951), p.~105.}

Toen de oude Grieken zich voorstelden dat de kinderen van Zeus onder hen
leefden, lieten zij zich meevoeren door een diep geloof in magie. Zoals
andere primitieve agrarische volkeren bewonderden zij de natuur en waren
ze ervan overtuigd dat de kracht van de individuele wil, oftewel magie,
de natuurlijke orde in beweging zette. Hun visie op de natuur en de
goden straalde geen zelfbewuste, profetische inslag uit. Ze waren totaal
onvoorbereid op de komst van microtechnologie. Het was voor hen
onmogelijk om zich voor te stellen welke invloed deze technologie
duizenden jaren later zou hebben op de marginale productiviteit van
mensen, laat staan hoe het het evenwicht tussen macht en efficiëntie zou
veranderen, en een revolutie zou ontketenen in de manier waarop bezit
wordt gecreëerd en beschermd. Toch vertoont wat zij zich voorstelden in
hun mythen een merkwaardige overeenkomst met de wereld die u
waarschijnlijk zult meemaken.

\subsection{Alt.abracadabra}\label{alt.abracadabra}

De `abracadabra' die wordt uitgesproken bij het gebruik van magie
vertoont bijvoorbeeld een merkwaardige gelijkenis met het invoeren van
het wachtwoord dat nodig is om toegang te krijgen tot een computer. In
zekere zin kunnen we met de snelle verwerking van computers de magie van
de lampgeest al nabootsen. De eerste generaties van `digitale dienaren'
gehoorzamen nu al de bevelen van degenen die de computers beheersen
waarin ze zijn opgesloten, net zoals dat geesten ooit opgesloten zaten
in magische lampen. De virtuele realiteit van informatietechnologie
breidt het rijk van menselijke verlangens uit, zodat bijna alles wat men
zich maar kan voorstellen werkelijkheid lijkt te worden. `Telepresentie'
stelt mensen in staat om afstanden met een bijna bovennatuurlijke
snelheid te overbruggen en op afstand gebeurtenissen te volgen, precies
zoals de Grieken beweerden dat Hermes en Apollo dat konden. Na verloop
van tijd genieten de Soevereine Individuen van het informatietijdperk,
net als de goden uit oude mythen, van een soort `diplomatieke
immuniteit' tegen de politieke beproevingen die sterfelijke mensen door
de eeuwen heen hebben gekend.

Het nieuwe, Soevereine Individu opereert in dezelfde fysieke omgeving
als de gewone burger, maar politiek gezien leeft hij in een eigen sfeer.
Met een veel groter arsenaal aan middelen tot zijn beschikking en zonder
de beperkingen van talrijke vormen van dwang, gaat dit individu
overheden hervormen en economieën herinrichten in het nieuwe millennium.
De volledige impact van deze verschuiving is bijna onvoorstelbaar.

\subsection{Genialiteit en nemesis}\label{genialiteit-en-nemesis}

Voor wie houdt van menselijke ambitie en succes, opent het
informatietijdperk een schat aan kansen. Dit is zonder twijfel het beste
nieuws in generaties, maar het gaat ook gepaard met minder positieve
ontwikkelingen. De nieuwe maatschappelijke ordening, die voortvloeit uit
de overwinning van de individuele autonomie en het realiseren van echte
gelijke kansen op basis van verdienste, zal leiden tot ruime individuele
vrijheid en grote beloningen voor wie uitblinkt. Dit betekent echter wel
dat mensen veel meer verantwoordelijkheid voor hun eigen leven moeten
nemen dan ze in het industriële tijdperk gewend waren. Tegelijk zal de
onterecht verhoogde levensstandaard, waar de inwoners van geavanceerde
industriële samenlevingen in de twintigste eeuw van profiteerden, sterk
afnemen. Op dit moment verdient de top 15 procent van de wereldbevolking
gemiddeld \$21.000 per persoon per jaar, terwijl de overige 85 procent
gemiddeld slechts \$1.000 tot hun beschikking heeft. Dit enorme, in de
loop der tijd opgepotte voordeel zal onvermijdelijk verdwijnen in de
nieuwe realiteit van het informatietijdperk.

Naarmate dit gebeurt, implodeert het vermogen van natiestaten om op
grote schaal inkomen te herverdelen. Informatietechnologie zorgt voor
een scherpe toename van de concurrentie tussen rechtsgebieden. Wanneer
technologie mobiel is en transacties plaatsvinden in de cyberspace,
zoals steeds vaker het geval zal zijn, zullen overheden niet langer in
staat zijn om meer te vragen voor hun diensten dan wat die diensten
daadwerkelijk waard zijn voor de mensen die ervoor betalen. Iedereen met
een laptop en een satellietverbinding kan elk informatiebedrijf vrijwel
overal runnen, wat nagenoeg het totaal van de wereldwijde financiële
handel van biljoenen dollars omvat.

Dit betekent dat om een hoog inkomen te realiseren, men zich niet langer
gedwongen hoeft te voelen om in een rechtsgebied met hoge belastingen te
wonen. In de toekomst, wanneer het merendeel van de welvaart wereldwijd
verdiend en uitgegeven kan worden, zullen overheden die te hoge tarieven
rekenen voor domicilie simpelweg hun beste klanten doen vertrekken. Als
onze redenering klopt, en daar zijn wij van overtuigd, zal de natiestaat
in de huidige vorm niet overleven.

\section{Het einde van naties}\label{het-einde-van-naties}

Veranderingen die de macht van gevestigde instellingen ondermijnen, zijn
zowel verontrustend als riskant. Net zoals dat monarchen, edellieden,
pausen en andere machthebbers in de vroege moderne tijd meedogenloos
vochten om hun verworven privileges te behouden, zo zullen ook de
huidige overheden geweld inzetten, vaak heimelijk en willekeurig, om de
veranderende tijd tegen te houden. Verzwakt door de technologische
uitdagingen gaat de staat steeds autonomer wordende individuen, voorheen
haar gehoorzame burgers, benaderen met dezelfde meedogenloosheid en
diplomatie als waarmee zij tot nu toe met andere regeringen omging.

De komst van deze nieuwe fase in de geschiedenis ging op 20 augustus
1998 met een knal van start, toen de Verenigde Staten voor ongeveer 200
miljoen dollar aan Tomahawk BGM-109 kruisraketten afvuurden op doelen
die naar verluidt in verband stonden met de verbannen Saoedische
miljonair Osama bin Laden. Bin Laden werd de eerste persoon in de
geschiedenis op wiens satelliettelefoon raketaanvallen werden gericht.
Tegelijkertijd verwoestten de Verenigde Staten een farmaceutische
fabriek in Khartoem, Soedan, die met Bin Laden werd geassocieerd. Bin
Ladens opkomst als de belangrijkste vijand van de Verenigde Staten
illustreert een fundamentele verandering in de aard van oorlogsvoering.
Één enkel individu, al bezit hij honderden miljoenen dollars, kan nu als
een geloofwaardige bedreiging worden gezien voor de machtigste militaire
macht van het industriële tijdperk. In uitspraken die doen denken aan de
Koude Oorlogpropaganda over de Sovjet- Unie portretteerden de president
van de Verenigde Staten en zijn nationale veiligheidsspecialisten Bin
Laden, een privépersoon, als een transnationale terrorist en de
voornaamste vijand van de Verenigde Staten.

Dezelfde militaire logica die ervoor zorgde dat Osama bin Laden als
oppervijand van de Verenigde Staten werd gezien, zal zich ook aftekenen
in de verhouding tussen overheden en hun burgers. Steeds zwaardere
repressie zal een logisch gevolg zijn van het ontstaan van een nieuwe
machtsverhouding tussen overheden en individuen. Door technologische
vooruitgang worden individuen meer dan ooit soeverein, waardoor ze ook
zo behandeld zullen worden, soms met geweld als vijanden, soms als
gelijkwaardige onderhandelingspartners, soms als bondgenoten. Hoe
meedogenloos overheden zich ook zullen opstellen tijdens deze
overgangsperiode, een fusie van de belastingdienst met de
inlichtingendienst zal hen weinig opleveren. Overheden zullen in
toenemende mate moeten onderhandelen met autonome individuen, omdat de
controle over hun middelen steeds verder zal verslappen.

De veranderingen die de Informatie-Revolutie teweegbrengt, veroorzaken
niet alleen fiscale crises bij overheden, maar zorgen er ook voor dat
alle grote structuren uiteenvallen. In de twintigste eeuw zijn al
veertien rijken verdwenen. Het verval van rijken maakt deel uit van een
proces dat uiteindelijk ook de natiestaat zal doen instorten. Overheden
moeten zich aanpassen aan de groeiende zelfstandigheid van het individu.
De capaciteit om belastingen te innen daalt met 50 tot 70 procent, wat
kleinere rechtsgebieden bevoordeelt. Het vaststellen van concurrerende
voorwaarden om bekwame individuen en hun kapitaal aan te trekken, slaagt
in enclaves veel eenvoudiger dan op continentale schaal.

Wij zijn ervan overtuigd dat hedendaagse barbaren steeds vaker op de
achtergrond de macht zullen grijpen, zodra de moderne natiestaat
uiteenvalt. Groepen zoals de Russische \emph{mafiya}, die de
overblijfselen van de voormalige Sovjet-Unie uitbuitte, andere etnische
misdaadbendes, \emph{nomenklaturen}, drugsbaronnen en afvallige geheime
diensten zullen hun eigen regels maken, en dat doen ze al.\footnote{Nomenklaturen
  zijn de gevestigde elites die de voormalige Sovjet-Unie en andere door
  de staat bestuurde economieën regeerden.} Veel meer dan men doorgaans
denkt, hebben deze moderne barbaren de vorm van de natiestaat al
binnengedrongen zonder haar uiterlijk wezenlijk te hebben veranderd. Ze
fungeren als microparasieten die zich tegoed doen aan een systeem in
verval. Net zo gewelddadig en meedogenloos als een staat in oorlog,
passen deze groepen de technieken van de staat op een kleinere schaal
toe. Hun toenemende invloed en macht draagt bij aan het krimpende belang
van de politiek. Microprocessoren verkleinen de benodigde middelen die
nodig zijn om effectief geweld in te zetten en te beheersen. Naarmate de
technologische revolutie vordert, organiseert geweld zich steeds vaker
buiten centrale controle en richt de aanpak om het in te dammen zich
meer op efficiëntie dan op de omvang van macht.

\subsection{Geschiedenis in omgekeerde
richting}\label{geschiedenis-in-omgekeerde-richting}

De nieuwe logica van het informatietijdperk keert het proces waardoor de
natiestaat zich in de afgelopen vijf eeuwen heeft ontwikkeld om. Lokale
machtscentra zullen opnieuw aan invloed winnen, terwijl de staat
uiteenvalt in gefragmenteerde, overlappende soevereine
eenheden.\footnote{Voor meer details over gefragmenteerde
  soevereiniteiten als voorloper en alternatief voor de natiestaat, zie
  Charles Tilly, \emph{Coercion, Capital and European States AD
  990-1992} (Oxford: Blackwell, 1993).} De groeiende invloed van de
georganiseerde misdaad is slechts één afspiegeling van deze trend.
Multinationale ondernemingen moeten inmiddels al het niet-essentiële
werk uitbesteden, en sommige conglomeraten, zoals AT\&T, Unisys en ITT,
splitsten zich op in meerdere ondernemingen om winstgevender te kunnen
opereren. De natiestaat zal uiteenvallen zoals een onhandelbaar
conglomeraat, maar dat gebeurt waarschijnlijk pas nadat financiële
crises haar hiertoe hebben gedwongen.

Niet alleen verandert de machtsbalans in de wereld, maar ook de aard van
arbeid verandert radicaal. Dit betekent dat de manier waarop bedrijven
opereren onvermijdelijk zal veranderen. De `virtuele onderneming'
illustreert een fundamentele transformatie in de bedrijfsvoering,
mogelijk gemaakt door de dalende kosten voor informatie en transacties.
We onderzoeken de gevolgen van de informatierevolutie en haar invloed op
het uiteenvallen van bedrijven en het verdwijnen van de `goede baan'. In
het informatietijdperk betekent een `baan' simpelweg een taak die je
uitvoert, in plaats van een vaste positie. Microprocessors creëren
geheel nieuwe mogelijkheden voor economische activiteiten die alle
territoriale grenzen overschrijden. Deze grensoverschrijdende
ontwikkeling is wellicht de meest revolutionaire sinds Adam en Eva
werden verbannen uit het paradijs onder het vonnis van hun Schepper: `In
het zweet van uw gezicht zult gij uw brood verdienen.' Naarmate
technologie onze hulpmiddelen radicaal vernieuwt, veroudert ons
rechtssysteem, verandert onze moraal en verschuift onze perceptie. Dit
boek legt uit hoe.

Dankzij microprocessoren en de razendsnelle vooruitgang in communicatie
kan iedereen nu zelf bepalen waar hij of zij werkt. Transacties via
internet worden steeds beter versleuteld en zullen voor
belastingambtenaren binnenkort vrijwel onmogelijk te onderscheppen zijn.
Belastingvrij kapitaal groeit `offshore' aanzienlijk sneller dan
binnenlands vermogen, dat nog steeds belast wordt volgens de hoge
heffingen van de twintigste-eeuwse natiestaat. Na de millenniumwisseling
zal een groot deel van de wereldhandel verhuizen naar het nieuwe domein
van de cyberspace, een gebied waar regeringen even weinig invloed hebben
als over de zeebodem of de verre planeten. In de cyberspace verdwijnen
de dreigingen van fysiek geweld, de alfa en omega van de oude politiek.
Hier ontmoeten de minderbedeelden en de machthebbers elkaar op gelijke
voet. Cyberspace vormt de ultieme offshore jurisdictie: een economie
zonder belastingen, een soort Bermuda in de lucht, vol diamanten.

Wanneer dit ultieme belastingparadijs volledig toegankelijk is voor het
bedrijfsleven, zal vrijwel al het vermogen als offshorekapitaal
fungeren, volledig onder beheer van hun eigenaar. Dit zal een
kettingreactie in gang zetten. De staat is er inmiddels aan gewend
geraakt om haar belastingbetalers te behandelen zoals een boer omgaat
met zijn koeien, die hij op een weiland plaatst om gemolken te worden.
Voor je het weet, zullen de koeien vleugels krijgen.

\subsection{De wraak van de
natiestaat}\label{de-wraak-van-de-natiestaat}

Zoals een boze boer zal de staat in eerste instantie wanhopige pogingen
ondernemen om haar ontsnapte kudde in toom te houden. Ze zal heimelijke,
en zelfs gewelddadige middelen inzetten om de toegang tot bevrijdende
technologieën te beperken. Dergelijke noodmaatregelen werken slechts
tijdelijk, als ze überhaupt effect hebben. De natiestaat van de
twintigste eeuw, met al haar arrogantie, zal uitgeput raken dankzij haar
dalende belastinginkomsten.

Wanneer de staat haar uitgaven niet langer kan dekken door de
belastingen te verhogen, zal ze nog wanhopigere maatregelen aanwenden.
Daaronder valt ook het drukken van geld. Overheden zijn er inmiddels aan
gewend dat ze een monopolie op de valuta hebben en deze naar eigen
inzicht kunnen devalueren. Deze ogenschijnlijk willekeurige inflatie
kenmerkt het monetaire beleid van vrijwel alle staten in de twintigste
eeuw. Zelfs de sterkste munteenheid van de naoorlogse periode, de Duitse
mark, verloor tussen 1 januari 1949 en het einde van juni 1995 maar
liefst 71 procent van haar koopkracht, terwijl in dezelfde periode de
koopkracht van de Amerikaanse dollar 84 procent daalde.\footnote{De
  Duitse GPI-index stond op 33,20 op 31 december 1948, en 112,90 op 30
  juni 1995, wat neerkomt op een samengestelde jaarlijkse depreciatie
  van 2,7 procent. De Amerikaanse CPI stond op 24 op 31 december 1948,
  en 152,50 op 30 juni 1995. De cumulatieve Amerikaanse inflatie was 635
  procent voor de periode.} Die inflatie werkt als een belasting op
iedereen die de valuta aanhoudt. Zoals we later zullen zien, zal
inflatie als inkomstenbron voor de overheid grotendeels verdwijnen door
de opkomst van cybergeld. Men zal de nationale monopolies op de uitgifte
en regulering van geld, waar de overheid gedurende de moderne tijd van
heeft geprofiteerd, door nieuwe technologie kunnen omzeilen. De
kredietcrisissen die Azië, Rusland en andere opkomende economieën in
1997 en 1998 teisterden, tonen immers aan dat nationale valuta en
kredietbeoordelingen achterhaald zijn en de soepele werking van de
wereldeconomie verstoren. Juist het feit dat de overheid eist dat alle
transacties binnen haar jurisdictie in de nationale valuta worden
uitgevoerd, maakt de economie kwetsbaar voor fouten van centrale
bankiers en aanvallen van speculanten, die de ene deflatoire crisis na
de andere veroorzaakten. In het informatietijdperk zullen individuen
over cybervaluta kunnen beschikken en daarmee hun monetaire
onafhankelijkheid terugwinnen. Wanneer zij hun eigen monetaire beleid
via het internet kunnen voeren, zal het minder, of helemaal niet meer
relevant zijn dat de staat controle blijft houden op de geldprinter van
het industriële tijdperk. Hun invloed op de wereldwijde welvaart wordt
dan ingehaald door wiskundige algoritmen die geen fysieke vorm kennen.
In het nieuwe millennium zal cybergeld, beheerd door particuliere
markten, het fiatgeld, uitgegeven door overheden, vervangen. Alleen de
armen zullen het slachtoffer worden van inflatie en de daaropvolgende
deflatoire spiralen, gevolgen van de kunstmatige hefboomwerking die
fiatgeld in de economie brengt.

Zonder de gebruikelijke mogelijkheden om belastingen te heffen en geld
in omloop te brengen, zullen regeringen, zelfs in doorgaans beschaafde
landen, zich van hun nare kant laten zien. Naarmate de
inkomstenbelasting steeds minder effectief blijkt te worden, zullen oude
en arbitraire vormen van dwang weer onder het stof vandaan gehaald
worden. De ultieme vorm van bronbelasting -- feitelijke of zelfs
openlijke gijzelneming -- zal ingezet worden door overheden die wanhopig
proberen te voorkomen dat rijkdom aan hun greep ontsnapt. Ongelukkige
burgers zullen dan als gijzelaars voor losgeld worden vastgehouden;
bijna zoals we zagen in de middeleeuwen. Ondernemingen die diensten
verlenen ter bevordering van de individuele vrijheid raken verstrikt in
infiltratie, sabotage en ontwrichting. Willekeurige onteigening van
eigendom (alledaags in de Verenigde Staten, waar dit vijfduizend keer
per week gebeurt) zal alleen maar toenemen. Regeringen zullen
mensenrechten schenden, de vrije informatiestroom censureren, nuttige
technologieën saboteren en soms nog veel erger handelen. Net zoals de
inmiddels verdwenen Sovjet-Unie tevergeefs probeerde de toegang tot
personal computers en Xerox-machines te beperken, zullen westerse
regeringen met totalitaire middelen proberen de cybereconomie te
onderdrukken.

\section{Terugkeer van de Luddieten}\label{terugkeer-van-de-luddieten}

Dergelijke methoden kunnen echter bij bepaalde bevolkingsgroepen in de
smaak vallen. De positieve ontwikkelingen van de individuele bevrijding
en autonomie zullen voor velen die niet tot de cognitieve elite horen,
juist als slecht nieuws overkomen. Het meeste verzet zal vermoedelijk
komen van mensen met middelmatig talent in de huidige rijke landen. Zij
zullen ervaren dat informatietechnologie een bedreiging vormt voor hun
levensstijl. Degenen die profiteren van de georganiseerde dwang,
waaronder miljoenen die een herverdeeld inkomen ontvangen van de
overheid, kunnen de nieuwe vrijheid, zoals gerealiseerd door Soevereine
Individuen, verafschuwen. Hun verontwaardiging onderstreept de waarheid
dat `waar je staat, wordt bepaald door waar je zit.'

\begin{quote}
Soms vroeg ik me af hoe ik zo'n diepe meelevende pijn kon voelen voor
het lot van een paar mannen die ik niet kende, terwijl zij in een
honkbalstadion honderden mijlen verderop een wedstrijd speelden tegen
een groep vreemden. Het antwoord is eenvoudig. Ik hield van mijn teams.
Hoewel er risico aan verbonden was, was betrokkenheid de moeite waard.
Sport deed mijn bloed sneller stromen, maakte me opgewonden en liet mijn
hart bonzen. Ik vond het fijn om iets op het spel te hebben. Het leven
kwam tot leven tijdens een wedstrijd. -- Craig Lambert
\end{quote}

Het zou echter misleidend zijn om alle negatieve gevoelens die tijdens
de aankomende transitiecrisis opkomen, louter toe te schrijven aan een
kille hunkering om ten koste van anderen te leven. Er spelen echter nog
meer factoren mee. De aard van de menselijke samenleving wijst erop dat
er onvermijdelijk een misplaatste morele dimensie meespeelt in de
naderende luddistische reactie. Die kille hunkering verschuilt zich in
feite achter een morele façade. Wij belichten de morele en moralistische
aspecten van de transitiecrisis. Niets zet mensen zo sterk in beweging
als de overtuiging dat ze moreel in hun recht staan, meer nog dan puur
eigenbelang. Hoewel de band met de burgerlijke mythes uit de twintigste
eeuw snel verzwakt, bestaan er nog altijd echte gelovigen. Iedereen die
in die eeuw volwassen werd, raakte doordrenkt met de plichten en
verantwoordelijkheden van het burgerschap uit die tijd. De overgebleven
morele principes uit de industriële samenleving zullen ten minste enkele
neo-luddistische aanvallen op informatietechnologieën in gang zetten.

In dit opzicht blijkt het te verwachten geweld deels een uiting te zijn
van wat wij `moreel anachronisme' noemen: het toepassen van morele
richtlijnen uit de ene economische periode op de omstandigheden van een
andere. Iedere fase van de samenleving vraagt om eigen morele normen die
mensen helpen de specifieke valkuilen te overwinnen die bij die manier
van leven horen. Een agrarische samenleving functioneerde niet volgens
de morele regels van een rondtrekkende Eskimostam, en zo kan ook de
informatiesamenleving niet voldoen aan de principes die ooit het succes
van een militante industriële staat in de twintigste eeuw mogelijk
maakten. Dat lichten we graag toe.

De komende jaren zal moreel anachronisme zich ook in de kernlanden van
het Westen duidelijk manifesteren, net zoals het de afgelopen vijf
eeuwen in de landen daar ver vandaan te zien was. Westerse kolonisten en
militaire expedities veroorzaakten dit soort crises zodra zij in
aanraking kwamen met inheemse jagers- en verzamelaarsstammen, en
volkeren wiens samenlevingen nog volgens agrarische modellen
functioneerden. Het invoeren van nieuwe technologieën in omgevingen die
achterbleven in de tijd leidde tot verwarring en morele crises. Het
succes van christelijke missionarissen bij de bekering van miljoenen
inheemse mensen is voor een groot deel te danken aan de lokale crises
die ontstonden door de plotselinge invoering van nieuwe machtsstructuren
van buitenaf. Dergelijke botsingen herhaalden zich keer op keer, van de
zestiende eeuw tot de vroege decennia van de twintigste eeuw. Wij
verwachten vergelijkbare conflicten in het begin van het nieuwe
millennium, naarmate de informatiesamenleving het begint over te nemen
van de gevestigde industriële systemen.

\subsection{De nostalgie naar dwang}\label{de-nostalgie-naar-dwang}

De opkomst van het Soevereine Individu zal niet door iedereen als een
veelbelovende, nieuwe fase in de geschiedenis worden omarmd, zelfs niet
door degenen die er het meeste voordeel uit halen. Iedereen zal gemengde
gevoelens hebben. Veel mensen zullen innovaties afkeuren die de
territoriale natiestaat ondermijnen. Het is de menselijke natuur om
iedere ingrijpende verandering vrijwel altijd als een dramatische
achteruitgang te beschouwen. Vijfhonderd jaar geleden beweerden de
hoflieden rondom de hertog van Bourgondië dat de opkomende innovaties,
die het feodalisme ondermijnden, slecht waren. Zij meenden dat de wereld
snel in verval zou raken, toevallig precies op het moment tijdens de
Renaissance waarin historici later een explosie van menselijk potentieel
constateerden. Zo zou deze periode in het volgende millennium als een
nieuwe renaissance kunnen worden gezien, terwijl in de vermoeide ogen
van de twintigste eeuw vooral angst is af te lezen.

Er is een grote kans dat degenen die aanstoot nemen aan deze nieuwe
ontwikkelingen, en vele anderen die erdoor benadeeld raken, onaangenaam
zullen reageren. Hun nostalgie naar dwang zal waarschijnlijk in geweld
uitmonden. Ontmoetingen met deze nieuwe `Luddieten' zorgen er in ieder
geval voor dat de overstap naar radicaal nieuwe vormen van sociale
organisatie voor iedereen voor wat ellende zal zorgen. Zet je schrap en
bereid je voor op de duik. Nu de snelheid van verandering het morele en
economische aanpassingsvermogen van velen in onze generatie te boven
gaat, kun je rekenen op een felle, verontwaardigde weerstand tegen de
Informatierevolutie, ondanks de grote belofte die zij brengt om de
toekomst te bevrijden.

Je moet dit soort vervelende ontwikkelingen begrijpen en je er op
voorbereiden. Er wacht een transitiecrisis op ons. Deflatoire crises,
zoals de Aziatische financiële crisis van 1997 en 1998, zullen blijven
opduiken zolang de verouderde instellingen uit het industriële tijdperk
niet opgewassen zijn tegen de uitdagingen van de nieuwe,
grensoverschrijdende economie. De nieuwe informatie- en
communicatietechnologieën ondermijnen de moderne staat sterker dan welke
andere politieke bedreiging dan ook sinds de tijd van Columbus. Dat is
van belang, want machthebbers hebben zelden vreedzaam gereageerd op
ontwikkelingen die hun gezag ondermijnen, en dat zal nu waarschijnlijk
niet anders zijn.

De botsing tussen het nieuwe en het oude zal de eerste jaren van het
nieuwe millennium bepalen. Wij voorzien een periode van groot gevaar en
grote beloningen, waarin op sommige terreinen de beschaving sterk
achteruitgaat en op andere terreinen ongekende kansen ontstaan. Steeds
autonomere individuen en failliete, wanhopige regeringen zullen
tegenover elkaar komen te staan aan beide zijden van een nieuwe kloof.
Voor het einde van deze transitie voorzien we een ingrijpende
verandering in het concept van soevereiniteit en de vrijwel volledige
ondergang van de politiek. Waar de staat nu nog de middelen beheerst,
zullen straks vrijwel alle overheidsdiensten geprivatiseerd zijn. Om
onontkoombare redenen die we in dit boek uitvoerig bespreken, zal
informatietechnologie de capaciteit van de staat zodanig ondermijnen dat
zij niet meer in staat is om voor haar diensten meer te vragen dan deze
daadwerkelijk waard zijn voor de burgers.

\begin{quote}
Regeringen zullen zich moeten bezighouden met wat soevereiniteit
betekent. -- Robert Martin
\end{quote}

\subsection{Soevereiniteit door
markten}\label{soevereiniteit-door-markten}

In een mate die we tien jaar geleden nauwelijks konden voorstellen,
verkrijgen individuen via marktmechanismen steeds meer vrijheid ten
opzichte van territoriale natiestaten. Deze staten zien hun gezag snel
verdwijnen en lopen het risico om failliet te gaan. Hoe machtig ze ook
lijken, behouden zij enkel de mogelijkheid om te vernietigen, niet om
bevelen uit te delen. Hun intercontinentale raketten en vliegdekschepen
functioneren inmiddels als relikwieën, even imposant als zinloos,
vergelijkbaar met het laatste oorlogspaard uit het feodale tijdperk.

Informatietechnologie breidt de markten drastisch uit doordat het de
wijze verandert waarop activa worden gecreëerd en beschermd. Dit is
werkelijk revolutionair. Het belooft zelfs een nog ingrijpendere impact
te hebben op de industriële samenleving dan de introductie van buskruit
ooit had op het feodale landbouwsysteem. De transformatie rond het jaar
2000 betekent de commercialisering van soevereiniteit en het einde van
de traditionele politiek, precies zoals vuurwapens ooit de ondergang van
het eedgebonden feodalisme inluidden. Burgerschap zal verdwijnen, net
zoals de ridderlijkheid ooit verdween.

Wij zijn ervan overtuigd dat het tijdperk van de individuele economische
soevereiniteit aanbreekt. Net zoals staalfabrieken,
telefoonmaatschappijen, mijnen en spoorwegen die ooit genationaliseerd
waren, wereldwijd in rap tempo werden geprivatiseerd, zul je spoedig de
ultieme vorm van privatisering zien: de ingrijpende denationalisering
van het individu. Het Soevereine Individu van het nieuwe millennium
behoort niet langer tot de staatsbezittingen, een impliciet activum op
de balans van de schatkist. Na de transitie van het jaar 2000 worden
voormalige burgers geen burgers meer, maar klanten.

\subsection{Bandbreedte overwint
grenzen}\label{bandbreedte-overwint-grenzen}

Het commercialiseren van soevereiniteit zal het traditionele burgerschap
binnen de natiestaat gedateerd maken, vergelijkbaar met de ridderlijke
eeden na de ineenstorting van het feodale systeem. In tegenstelling tot
de burger, verplicht tot belastingbetalingen en gebonden aan een
machtige staat, worden de Soevereine Individuen van de eenentwintigste
eeuw klanten van overheden die opereren in een `nieuwe logische ruimte'.
Zij zullen onderhandelen over de beperkte diensten die zij van het
overheidsapparaat wensen en zullen daarvoor betalen via contractuele
overeenkomsten. De overheden van het informatietijdperk zullen zich
volgens geheel andere principes moeten organiseren dan we de afgelopen
eeuwen gewend waren. Sommige rechtsgebieden en diensten met betrekking
tot soevereiniteit zullen ontstaan via `assortive matching', een systeem
waarbij overeenkomsten in voorkeuren, zoals commerciële belangen, de
basis vormen waarop virtuele jurisdicties een klantenbestand opbouwen.

In zeldzame gevallen kunnen de nieuwe soevereiniteiten restanten zijn
van middeleeuwse organisaties, zoals de 900 jaar oude Soevereine
Militaire Hospitaalorde van St.~Johannes van Jeruzalem, van Rhodos en
van Malta. Deze orde, beter bekend als de Ridders van Malta, vormt een
belangengroep van rijke katholieken met 10.000 leden en een jaarlijkse
inkomstenstroom van enkele miljarden. Zij geven hun eigen paspoorten,
postzegels en geld uit en onderhouden volledige diplomatieke
betrekkingen met zeventig landen. Momenteel onderhandelt de orde met de
Republiek Malta over de terugname van Fort St.~Angelo. Door het fort in
bezit te nemen, verkrijgen de Ridders het ontbrekende element van
territorialiteit, zodat zij als soevereine entiteit erkend kunnen
worden. De Ridders van Malta zouden zo opnieuw een soevereine microstaat
kunnen vormen, gesteund door hun lange geschiedenis. Vanuit Fort
St.~Angelo keerden zij in 1565 tijdens de Grote Belegering de Turken de
rug toe en regeerden zij Malta vele jaren, totdat Napoleon hen in 1798
verdreef. Als de Ridders van Malta in de komende jaren blijken terug te
keren, zou dat het duidelijkste bewijs zijn dat het systeem van moderne
natiestaten, dat na de Franse Revolutie zijn intrede deed, slechts een
intermezzo was in de langere lijn van de geschiedenis, waarin het
normaal was dat verschillende vormen van soevereiniteit naast elkaar
bestonden.

Een ander, totaal verschillend model voor postmoderne soevereiniteit op
basis van assortive matching zie je terug in het
Iridium-satelliettelefoonnetwerk. In eerste instantie lijkt het vreemd
om een mobiele telefoniedienst als een vorm van soevereiniteit te
beschouwen, maar internationale autoriteiten hebben Iridium al erkend
als een virtueel land. Zoals je wellicht weet, biedt Iridium een
wereldwijde mobiele telefoniedienst waardoor abonnees via één nummer
oproepen kunnen ontvangen, waar ze zich ook bevinden -- of je nu in
Featherston, Nieuw-Zeeland bent of in de Boliviaanse Chaco. Om te
garanderen dat de oproepen overal de juiste abonnees bereiken, stemden
internationale telecomautoriteiten ermee in Iridium als een virtueel
land te erkennen, compleet met een eigen landcode: 8816. Het vereist dan
ook geen ingewikkelde logica om te beargumenteren dat als een virtueel
land van satelliettelefoonabonnees soeverein kan zijn, meer
samenhangende virtuele gemeenschappen op het grensoverschrijdende
internet dit ook zullen kunnen zijn.

Bandbreedte (oftewel de draagcapaciteit van een communicatiemedium) is
sinds de uitvinding van de transistor sneller gegroeid dan de
rekenkracht. Als deze trend zich voortzet, zijn wij ervan overtuigd dat
de bandbreedte binnen enkele jaren, kort na de millenniumwisseling,
zodanig toeneemt dat men de `metaverse' kan realiseren, de alternatieve
cyberspacewereld zoals bedacht door sciencefictionauteur Neal
Stephenson. Stephenson's `metaverse' staat voor een hechte virtuele
gemeenschap met haar eigen regels. Wij geloven dat, naarmate de
cybereconomie groeit, de deelnemers uiteindelijk vrijgesteld worden van
de verouderde wetten van traditionele natiestaten. De nieuwe
cybergemeenschappen zullen minstens even welvarend zijn en hun belangen
net zo effectief behartigen als de Soevereine Militaire Hospitaalorde
van St.~Johannes van Jeruzalem, van Rhodos en van Malta. Dankzij
geavanceerde communicatietechnologieën en de mogelijkheden binnen
informatieoorlogvoering zullen zij zich beter weten stand te houden. We
onderzoeken daarnaast ook andere modellen van gefragmenteerde
soevereiniteit, waarin kleine groepen feitelijk de soevereiniteit van
zwakkere natiestaten kunnen leasen en hun eigen economische
toevluchtsoorden kunnen exploiteren, vergelijkbaar met hoe vrije havens
en vrijhandelszones dat nu doen.

We hebben een nieuwe morele woordenschat nodig om de relaties tussen
Soevereine Individuen en wat er overblijft van de overheid te
beschrijven. Naarmate de spelregels van deze nieuwe verhoudingen
duidelijker worden, zullen ze waarschijnlijk stuiten op weerstand van
mensen die zijn grootgebracht met het idee ``burger'' te zijn van een
twintigste-eeuwse natiestaat. Het verdwijnen van naties en de
`denationalisering van het individu' doen enkele gekoesterde opvattingen
vervagen, zoals het principe van `gelijke bescherming onder de wet', dat
uitgaat van machtsverhoudingen die spoedig tot het verleden behoren.
Naarmate virtuele gemeenschappen sterkere samenhang ontwikkelen, zullen
zij eisen dat hun leden volgens hun eigen regels ter verantwoording
worden geroepen in plaats van volgens de wetten van de voormalige
natiestaten waarin zij wonen. Binnen hetzelfde geografische gebied
zullen opnieuw meerdere rechtsstelsels naast elkaar functioneren, zoals
we in de oudheid en middeleeuwen zagen.

Net zoals dat de pogingen om de riddermacht te behouden, faalden door de
opkomst van vuurwapens, zijn de moderne noties van nationalisme en
burgerschap gedoemd om tot irrelevantie te vervallen door de
ontwikkeling van microtechnologie. Ze zullen uiteindelijk belachelijk
lijken, net zoals de heilige principes van het vijftiende-eeuwse
feodalisme, waarmee al in de zestiende eeuw de spot werd gedreven. De
gekoesterde burgerlijke waarden van de twintigste eeuw zullen
lachwekkende anachronismen blijken te zijn voor komende generaties na de
transformatie van het jaar 2000. De Don Quichot van de eenentwintigste
eeuw wordt geen dwaalridder die de glorie van het feodalisme doet
herleven, maar zal een bureaucraat in een bruin pak zijn: een
belastingambtenaar die erop uit is om de burgers te controleren.

\section{Herleving van de wetten van de
mark}\label{herleving-van-de-wetten-van-de-mark}

We denken zelden aan overheden als concurrerende partijen, behalve in
heel algemene zin. Daardoor is ons besef van wat soevereiniteit kan
inhouden, verzwakt. Vroeger, toen de machtsverhoudingen het voor
verschillende groepen bemoeilijkten om een stabiel monopolie op dwang te
vestigen, was de macht vaak gefragmenteerd, overlapten rechtsgebieden
elkaar gedeeltelijk en oefenden diverse entiteiten één of meer kenmerken
van soevereiniteit uit. Vaak bleek de zogenaamde heerser in de praktijk
nauwelijks enige macht te bezitten. Tegenwoordig concurreren overheden
die minder sterk zijn dan natiestaten actief om op lokaal niveau een
monopolie op dwang uit te oefenen. Deze concurrentie heeft geleid tot
veranderingen in de wijze waarop geweld wordt beheerst en loyaliteit
wordt gewonnen -- veranderingen die spoedig weer zichtbaar zullen
worden.

Toen de macht van heren en vorsten nog beperkt was en de aanspraken van
één of meer groepen elkaar overlapten bij grensgebieden, kon geen van
beide partijen de overhand krijgen. In de Middeleeuwen kende men
talrijke grensgebieden, de zogenaamde `marken' (of \emph{marches} in het
Engels), waar soevereiniteiten in elkaar overliepen. Deze conflictrijke
gebieden hielden decennialang, zo niet eeuwenlang stand binnen de
grenzen van het Europese continent. Men vond marken tussen gebieden
onder Keltische en Engelse invloed in Ierland; tussen Wales en Engeland;
tussen Schotland en Engeland; tussen Italië en Frankrijk; tussen
Frankrijk en Spanje; tussen Duitsland en de Slavische grensstreken van
Centraal-Europa; en tussen de christelijke koninkrijken van Spanje en
het islamitische koninkrijk Granada. Dergelijke grensgebieden
ontwikkelden unieke institutionele en juridische structuren, en we
verwachten dat we dit fenomeen in het komende millennium opnieuw zullen
zien terugkeren. Doordat twee autoriteiten met elkaar concurreerden,
betaalden de inwoners van deze streken zelden belasting. Bovendien
konden zij doorgaans zelf bepalen wiens wetten zij zouden volgen, door
middel van juridische instrumenten als de `erkenning' en de
`beslaglegging' -- concepten die tegenwoordig vrijwel geheel zijn
verdwenen. Wij voorzien dat zulke principes een prominente rol zullen
spelen in het recht van informatiesamenlevingen.

\subsection{Het overstijgen van
nationaliteit}\label{het-overstijgen-van-nationaliteit}

Door de opkomst van de natiestaat was het lastig om precies vast te
stellen hoeveel soevereine entiteiten de wereld telde, omdat zij op
ingewikkelde wijze in elkaar overliepen en allerlei vormen van
organisatie hanteerden. Dat zal in de toekomst opnieuw zo zijn. Binnen
de natiestaat werden de grenzen tussen territoria scherp afgebakend,
maar in het informatietijdperk zullen deze lijnen weer vervagen. In het
nieuwe millennium raakt soevereiniteit opnieuw gefragmenteerd en duiken
er nieuwe entiteiten op die slechts enkele, maar niet alle kenmerken
bezitten die we met overheden associëren.

Sommige van deze nieuwe entiteiten, vergelijkbaar met de Tempeliers en
andere religieuze militaire orden uit de middeleeuwen, kunnen met
aanzienlijke rijkdom en militaire kracht opereren, ook al beschikken zij
niet over een vast grondgebied. Ze organiseren zich op basis van
principes die niets met nationaliteit te maken hebben. In de
middeleeuwen ontleenden leden en leiders van religieuze organisaties,
die in delen van Europa soevereine macht uitoefenden, hun gezag niet aan
een nationale identiteit. Zij vertegenwoordigden diverse etnische
achtergronden en verklaarden hun trouw aan God, niet aan een bepaalde
band die een nationaliteit zou moeten scheppen.

\subsection{Koopliedenrepublieken van
cyberspace}\label{koopliedenrepublieken-van-cyberspace}

Je zult tevens zien dat samenwerkingsverbanden tussen kooplieden en
vermogende individuen met semisouvereine bevoegdheden weer opbloeien,
zoals de Hanze, een middeleeuwse confederatie van kooplieden. De Hanze,
die actief was op de Franse en Vlaamse markten, groeide uit tot een
organisatie die kooplieden uit zestig steden verenigde.\footnote{Janet
  L. Abu-Lughod, \emph{Before European hegemony: the world system A.D.
  1250-1350} (Oxford: Oxford University Press, 1991), p.~62.} De
`Hanseatic League' -- zoals men haar in het Engels op een overbodige
wijze aanduidt (de letterlijke vertaling is immers: `Liga liga') --
vormde een verbond van Germaanse koopmansgilden dat haar leden
bescherming bood en handelsverdragen onderhandelde. Ze kreeg in diverse
steden in Noord-Europa en de Baltische regio semisouvereine
bevoegdheden. Vergelijkbare entiteiten zullen in het nieuwe millennium
opkomen als vervanging van de stervende natiestaat, en zullen
bescherming bieden en bijdragen aan het handhaven van contracten in een
onveilige wereld.

Kortom, de toekomst zal waarschijnlijk niet voldoen aan de verwachtingen
van degenen die de burgerlijke mythes van de industriële samenleving van
de twintigste eeuw hebben geïnternaliseerd, waaronder ook de illusies
van de sociale democratie, dat ooit de meest getalenteerde geesten
opzweepte en motiveerde. Deze mensen gaan ervan uit dat samenlevingen
zich ontwikkelen op de manier die voor de overheid het meest wenselijk
is, het liefst als reactie op opiniepeilingen en nauwkeurig getelde
stemmen. Dit bleek echter nooit zo evident als vijftig jaar geleden werd
gedacht. Vandaag de dag is het een anachronisme, net zozeer een
overblijfsel van het industrialisme als een roestige schoorsteen. Deze
burgerlijke mythes laten niet alleen een denkwijze zien die
maatschappelijke problemen als oplosbaar via technische ingrepen
beschouwt, maar onthullen ook een vals vertrouwen dat hulpbronnen en
individuen in de toekomst even kwetsbaar blijven voor politieke dwang
als in de twintigste eeuw. Wij betwijfelen dat. Het zijn marktkrachten,
niet politieke meerderheden, die samenlevingen zullen dwingen zich
opnieuw in te richten op manieren die de publieke opinie noch zal
begrijpen, noch zal verwelkomen. Als dat gebeurt, blijkt de naïeve
opvatting dat geschiedenis is wat mensen willen dat zij is, buitengewoon
misleidend.

Het zal daarom essentieel zijn om de wereld vanuit een nieuw perspectief
te benaderen. Dat betekent dat u de zaken van buitenaf zult moeten
observeren en veel vanzelfsprekend geachte aannames in vraag zult moeten
stellen, zodat nieuwe inzichten kunnen ontstaan. Als u er niet in slaagt
het conventionele denken te overstijgen in een tijd waarin dit denken de
aansluiting met de werkelijkheid verliest, loopt u het risico ten prooi
te vallen aan een epidemie van desoriëntatie. Die desoriëntatie leidt
tot fouten die uw bedrijf, uw investeringen en uw levensstijl in gevaar
kunnen brengen.

\begin{quote}
Het universum beloont ons als we het doorgronden en straft ons als we
dat niet doen. Als wij het universum doorgronden, slagen onze plannen en
voelen wij ons op ons best. Maar als wij proberen te vliegen door van
een klif te springen en simpelweg met onze armen te fladderen, dan zal
het universum ons ten val brengen. -- JACK COHEN EN IAN
STEWART\footnote{Jack Cohen en Ian Stewart, \emph{The Collapse of Chaos}
  (New York: Viking, 1994).}
\end{quote}

\subsection{De wereld vanuit een nieuw
perspectief}\label{de-wereld-vanuit-een-nieuw-perspectief}

Om u voor te bereiden op de toekomst, moet u begrijpen waarom de wereld
die gaat komen anders zal zijn dan de voorspellingen van de meeste
experts. Dit betekent dat u grondig moet kijken naar de verborgen
oorzaken van verandering. Wij hebben geprobeerd dit inzichtelijk te
maken met een onconventionele analyse, die wij de studie van
`megapolitiek' noemen. In de twee eerdere delen, \emph{Blood in the
Streets} en \emph{The Great Reckoning}, betoogden we dat de voornaamste
oorzaken van verandering niet te vinden zijn in politieke manifesten of
in de uitspraken van overleden economen, maar in de verborgen factoren
die de grenzen waarover macht wordt uitgeoefend, doen veranderen. Vaak
zorgen subtiele wijzigingen in klimaat, topografie, micro-organismen en
technologie ervoor dat de logica van geweld verschuift. Deze
veranderingen transformeren ook de manier waarop mensen hun
levensonderhoud organiseren en zich verdedigen.

We zijn ons ervan bewust dat onze benadering voor het doorgronden van
wereldveranderingen sterk afwijkt van die van de meeste voorspellers.
Wij beweren niet dat wij meer weten over bepaalde `onderwerpen' dan
degenen die hun hele carrière hebben gewijd aan diepgaande
specialisatie. Integendeel, wij bekijken de zaken juist van buitenaf en
beschikken over kennis van de onderwerpen waarop wij onze voorspellingen
baseren. Voor ons draait het er vooral om te herkennen waar de grenzen
van de noodzaak liggen. Als die grenzen veranderen, verandert de
samenleving onvermijdelijk, ongeacht wat mensen graag zouden willen.

Vanuit ons perspectief schuilt de sleutel tot het begrijpen van de
ontwikkeling van samenlevingen in het doorgronden van de factoren die de
kosten en baten van geweld bepalen. Elke menselijke samenleving, van
jagerstammen tot grote rijken, wordt gevormd door de wisselwerking van
megapolitieke krachten die de dominante versie van de `wetten van de
natuur' vastleggen. Het leven is overal complex. Lammeren en leeuwen
bewaken een fragiele balans, waarbij zij subtiel op elkaar inwerken. Als
leeuwen ineens sneller zouden worden, zouden zij prooien kunnen vangen
die voorheen ontsnapten. En stel dat lammeren plotseling vleugels zouden
krijgen, dan zouden de leeuwen verhongeren. Het vermogen om geweld in te
zetten en zich ertegen te verdedigen is de cruciale factor die het leven
aan de marge beïnvloedt.

Wij hebben een goede reden om geweld centraal in onze theorie over
megapolitiek te plaatsen. Het beheersen van geweld vormt het grootste
dilemma voor iedere samenleving. Zoals we schreven in \emph{The Great
Reckoning}:

\begin{quote}
De reden dat mensen tot geweld overgaan, is simpelweg dat het vaak
loont. In zekere zin is dat het meest voor de hand liggende wat een mens
kan doen als hij geld wil: het gewoon afpakken. Dit geldt evenzeer voor
een leger dat een olieveld verovert als voor een enkele crimineel die
zomaar een portemonnee pakt. Macht zoekt, zoals William Playfair
schreef, `altijd de gemakkelijkste weg naar rijkdom door degenen aan te
vallen die erover beschikken.'

De uitdaging voor voorspoed ligt juist in het feit dat roofzuchtig
geweld onder bepaalde omstandigheden zeer winstgevend is. Oorlog
herschrijft de spelregels, verandert de verdeling van eigendommen en
inkomen, en bepaalt zelfs wie er leeft en wie sterft. Juist het feit dat
geweld loont, maakt het zo moeilijk te beheersen.\footnote{Zie James
  Dale Davidson en Lord William Rees-Mogg, \emph{The Great Reckoning},
  2e ed.~(New York: Simon \& Schuster, 1993), p.~53.}
\end{quote}

Het denken in deze termen hielp ons ontwikkelingen te voorspellen
waarover zelfs de meest doorgewinterde experts ervan overtuigd waren dat
ze nooit zouden plaatsvinden. Zo was bijvoorbeeld \emph{Blood in the
Streets}, dat begin 1987 verscheen, onze poging om de eerste signalen in
kaart te brengen van de grootschalige megapolitieke revolutie die zich
nu ontvouwt. Destijds stelden wij dat technologische vernieuwing de
mondiale machtsverhoudingen radicaal zou verstoren. Onze belangrijkste
stellingen waren:

\begin{itemize}
\tightlist
\item
  Wij voorspelden dat het Amerikaanse overwicht zou afnemen, wat zou
  leiden tot economische onevenwichtigheden en tegenspoed, waaronder een
  beurscrash in de stijl van 1929. Hoewel de experts vrijwel unaniem
  weigerden te geloven dat zoiets kon gebeuren, werden de markten in
  oktober 1987, amper zes maanden na onze voorspelling, geschokt door de
  heftigste verkoopgolf van de eeuw.
\item
  Wij waarschuwden de lezers voor de ineenstorting van het communisme.
  Wederom lachten de experts ons uit, maar in 1989 deden zich
  gebeurtenissen voor die ``niemand had kunnen voorzien.'' De Berlijnse
  Muur viel en revoluties deden de communistische regimes, van de
  Baltische staten tot zelfs Boekarest, verdwijnen.
\item
  Wij legden uit waarom het multi-etnische rijk, dat door de
  bolsjewistische elite werd overgenomen van de tsaren, onvermijdelijk
  uiteen zou vallen. Eind december 1991 hing de hamer- en sikkelvlag
  voor de laatste keer boven het Kremlin, waarna de Sovjetunie voorgoed
  ophield te bestaan.
\item
  Midden in de intensieve wapenwedloop onder Reagan voorspelden wij dat
  de wereld op de drempel stond van een ingrijpende ontwapening. Ook dit
  werd als onwaarschijnlijk -- zo niet belachelijk -- afgedaan, maar in
  de daaropvolgende zeven jaar vond wel de meest ingrijpende ontwapening
  plaats sinds het einde van de Eerste Wereldoorlog.
\item
  Terwijl experts in Noord-Amerika en Europa naar Japan wezen als bewijs
  dat overheden markten succesvol konden manipuleren, stelden wij het
  tegendeel vast. Wij voorspelden dat de bloeiperiode van de Japanse
  financiële bubbel zou eindigen in een ineenstorting. Niet lang na de
  val van de Berlijnse Muur stortte de Japanse aandelenmarkt in en
  verloor bijna de helft van haar waarde. Wij blijven ervan overtuigd
  dat het dieptepunt van deze neerwaartse beweging het verlies van 89
  procent, zoals Wall Street na 1929 heeft geleden, kan evenaren of
  zelfs overtreffen.
\item
  Terwijl bijna iedereen, van het middenklassegezin tot de grootste
  vastgoedinvesteerders ter wereld, ervan overtuigd leek dat
  vastgoedmarkten enkel zouden stijgen, waarschuwden wij voor een
  naderende vastgoedcrisis. Binnen vier jaar verloren
  vastgoedinvesteerders wereldwijd meer dan \$1 biljoen (\$1.000
  miljard) toen de vastgoedmarkt kelderde.
\item
  Al ruim voordat experts de achteruitgang in de inkomens van arbeiders
  onderkenden, voorspelden wij in \emph{Blood in the Streets} dat deze
  daling structureel zou blijven bestaan. Nu, bijna een decennium later,
  begint een slapende wereld eindelijk in te zien dat dit klopt. Het
  gemiddelde uurloon in de Verenigde Staten is gedaald tot een niveau
  lager dan tijdens de tweede Eisenhower-regering. In 1993 bedroeg het
  gemiddelde jaarinkomen, gecorrigeerd voor inflatie, \$18.808. In 1957,
  toen Eisenhower werd ingezworen voor zijn tweede ambtstermijn, was het
  gemiddelde jaarlijkse uurloon in de VS \$18.903.
\end{itemize}

Hoewel de centrale thema's uit \emph{Blood in the Streets} achteraf
gezien verrassend nauwkeurig bleken, noemden de hoeders van het
conventionele denken deze inzichten nog enkele jaren geleden pure onzin.
In 1987 bestempelde een recensent in \emph{Newsweek} onze analyse als
`een ondoordachte aanval op de rede', waarmee hij de bekrompenheid van
de laat-industriële mentaliteit treffend illustreerde.

Je zou denken dat \emph{Newsweek} en vergelijkbare publicaties inmiddels
inzagen dat onze analysemethode waardevolle inzichten bood in hoe de
wereld veranderde. Maar niets is minder waar. De eerste uitgave van
\emph{The Great Reckoning} werd met dezelfde gniffelende vijandigheid
ontvangen als \emph{Blood in the Streets}. Zelfs het \emph{Wall Street
Journal} wees onze analyse resoluut af als het geklets van `je suffe
tante'.

Ondanks al het gegiechel, bleken de thema's van \emph{The Great
Reckoning} minder belachelijk dan de hoeders van de orthodoxie deden
vermoeden.

Ook hebben we onze voorspelling over de ondergang van de Sovjet-Unie
verder uitgewerkt door te onderzoeken waarom Rusland en de andere
voormalige Sovjetrepublieken een toekomst tegemoet gingen vol toenemende
burgerlijke onrust, hyperinflatie en dalende levensstandaarden.

\begin{itemize}
\tightlist
\item
  We legden uit waarom de jaren 1990 een decennium van krimp zouden
  worden, met voor het eerst een wereldwijde inkrimping van zowel
  overheden als bedrijven.
\item
  We voorspelden tevens dat een ingrijpende herdefiniëring van de
  voorwaarden voor inkomensherverdeling stond te gebeuren, wat zou
  leiden tot flinke bezuinigingen op uitkeringen. Overal,van Canada tot
  Zweden, verschenen de eerste signalen van een fiscale crisis, en
  Amerikaanse politici begonnen te spreken over `het beëindigen van de
  welvaartsstaat zoals wij die kennen.'
\item
  Wij voorzagen en legden uit dat de `nieuwe wereldorde' uiteindelijk
  als een `nieuwe wereldwanorde' zou uitpakken. Lange tijd vóórdat de
  gruweldaden in Bosnië de krantenkoppen overspoelden, waarschuwden wij
  dat Joegoslavië in een burgeroorlog zou instorten.
\item
  Nog vóórdat Somalië in volledige anarchie verviel, legden wij uit hoe
  de dreigende ineenstorting van regeringen in Afrika ervoor zou zorgen
  dat sommige landen daar feitelijk onder curatele zouden worden
  gesteld.
\item
  Wij voorspelden en verduidelijkten waarom de militante islam het
  marxisme als leidende ideologie in de confrontatie met het Westen zou
  verdringen. Jaren voordat de bomaanslag in Oklahoma plaatsvond en men
  probeerde het World Trade Center op te blazen, lieten wij zien waarom
  de Verenigde Staten te maken zouden krijgen met een toename aan
  terrorisme.
\item
  Nog vóórdat de krantenkoppen over rellen in Los Angeles, Toronto en
  andere steden verschenen, legden wij uit hoe de opkomst van criminele
  subculturen onder stedelijke minderheden de weg vrijmaakte voor
  wijdverspreid crimineel geweld.
\item
  We voorspelden ook `de laatste depressie van de twintigste eeuw', die
  in 1989 in Azië begon en zich vanaf de periferie naar het centrum van
  het mondiale systeem verspreidde. Wij stelden dat de Japanse
  aandelenmarkt het voorbeeld van Wall Street na 1929 zou volgen, wat
  uiteindelijk zou uitmonden in een kredietcrisis en depressie.
  Overheden in Japan en elders grepen fors in, waardoor de financiële
  markten tijdelijk niet volledig weerspiegelden hoe slecht de
  kredietsituatie daadwerkelijk was. Het verplaatste slechts de
  economische problemen en verergerde ze. Hierdoor kwamen overheden
  wereldwijd onder druk te staan om competitieve monetaire devaluaties
  door te voeren, en dreigde zelfs een systeemwijde kredietcrisis, zoals
  de crisis in de jaren 1930, die wereldwijd economieën deed instorten.
\end{itemize}

\emph{The Great Reckoning} belichtte ook een reeks controversiële theses
die nog niet bevestigd zijn, of het door ons voorspelde
ontwikkelingsniveau nog niet hebben bereikt:

\begin{itemize}
\tightlist
\item
  We voorspelden dat de Japanse beurs dezelfde weg zou inslaan als Wall
  Street na 1929, wat zou uitmonden in een kredietcrisis en economische
  depressie. Ondanks dat de werkloosheid in landen als Spanje en Finland
  zelfs hoger uitviel dan in de jaren '30, en sommige landen -- zoals
  Japan -- met lokale depressies kampten, is er nog geen wereldwijde
  kredietcrash geweest zoals die destijds hele economieën onderuit
  haalde.
\item
  We stelden dat het uiteenvallen van het centraal geleide systeem in de
  voormalige Sovjet-Unie zou leiden tot de verspreiding van kernwapens
  naar ministaten, terroristische groeperingen en criminele bendes. Tot
  grote opluchting van de wereld is dat niet gebeurd, althans niet in de
  mate die wij vreesden. Persberichten melden dat Iran diverse tactische
  kernwapens op de zwarte markt heeft aangeschaft en dat de Duitse
  autoriteiten meerdere pogingen tot de verkoop van nucleaire materialen
  hebben verijdeld. Er is overigens geen melding gemaakt van
  daadwerkelijke inzet of gebruik van kernwapens uit het arsenaal van de
  voormalige Sovjet-Unie.
\item
  Ten slotte legden we uit waarom de zogenaamde `War on Drugs' een
  recept was voor het ondermijnen van de politie- en rechtssystemen in
  landen met wijdverspreid drugsgebruik, vooral in de Verenigde Staten.
  Dankzij de tientallen miljarden dollars aan verborgen monopoliewinsten
  die drugshandelaren jaarlijks binnen harken, beschikken ze over zowel
  de middelen als de prikkel om zelfs schijnbaar stabiele landen te
  corrumperen. De internationale media hebben zo nu en dan gesuggereerd
  dat drugsgeld tot in de hoogste regionen van de Amerikaanse politiek
  is doorgedrongen, maar het volledige verhaal is nog niet verteld.
\end{itemize}

\subsection{Kijken waar anderen niet
kijken}\label{kijken-waar-anderen-niet-kijken}

Hoewel sommige van onze voorspellingen achteraf fout bleken of met de
huidige kennis als onjuist worden bestempeld, blijft het totaalplaatje
standhouden tegen de kritiek. Veel van de ontwikkelingen die
waarschijnlijk een rol zouden gaan spelen in de economische geschiedenis
van de jaren 1990 werden al eerder voorspeld en toegelicht in \emph{The
Great Reckoning}. Wij voorspelden niet simpelweg een voortzetting van de
bestaande trends, maar wezen op ingrijpende breuken met wat sinds de
Tweede Wereldoorlog als normaal werd beschouwd. We waarschuwden dat de
jaren negentig drastisch anders zouden zijn dan de voorgaande vijf
decennia. Als je het nieuws van 1991 tot en met 1995 volgt, zie je dat
de thema's uit \emph{The Great Reckoning} bijna dagelijks werkelijkheid
werden.

Wij interpreteren deze ontwikkelingen niet als losse tegenslagen, maar
als schokgolven langs één doorlopende breuklijn. De oude orde wordt op
zijn kop gezet door een megapolitieke aardbeving die een revolutie in de
politieke instellingen zal veroorzaken, en de manier waarop kritische
geesten de wereld beschouwen radicaal zal veranderen.

Ondanks de cruciale rol die geweld speelt in het functioneren van onze
wereld, krijgt het opvallend weinig serieuze aandacht. De meeste
politieke analisten en economen doen alsof geweld slechts een kleine
ergernis is, vergelijkbaar met een vlieg die rond een taart zoemt, en
niet als de bakker die de taart heeft gemaakt.

\subsection{Een andere grondlegger van de
megapolitiek}\label{een-andere-grondlegger-van-de-megapolitiek}

Sterker nog, er is zo weinig helder nagedacht over de rol van geweld in
de geschiedenis dat een bibliografie met alle megapolitieke analyses op
één vel papier zou passen. In \emph{The Great Reckoning} baseerden we
onze redenering onder meer op een bijna volledig vergeten klassieker
over megapolitieke analyse, namelijk \emph{An Enquiry into the Permanent
Causes of the Decline and Fall of Powerful and Wealthy Nations} van
William Playfair, gepubliceerd in 1805. Eén van onze uitgangspunten is
tevens het werk van Frederic C. Lane. Lane, een historicus die zich
onder meer richtte op de middeleeuwen, publiceerde in de jaren '40 en
'50 verschillende scherpe essays over de rol van geweld in de
geschiedenis. Wellicht was zijn essay \emph{Economic Consequences of
Organized Violence}, dat in 1958 in het \emph{Journal of Economic
History} verscheen, het meest omvattende van deze werken. Buiten
professionele economische en geschiedkundige kringen heeft slechts een
enkeling het gelezen en lijken de meesten de ware betekenis ervan niet
te hebben doorgrond. Net als Playfair richtte Lane zich tot een publiek
dat op dat moment nog niet bestond.

\subsection{Inzichten voor het
informatietijdperk}\label{inzichten-voor-het-informatietijdperk}

Lane publiceerde zijn werk over geweld en de economische betekenis van
oorlog ruim voor de intrede van het informatietijdperk. Hij schreef
zeker niet met de introductie van microprocessing of andere
technologische revoluties die zich tegenwoordig ontvouwen in het
vooruitzicht. Toch bieden zijn inzichten omtrent geweld een kader om te
begrijpen hoe de informatierevolutie de samenleving zal herstructureren.
Het toekomstbeeld dat Lane schetste, bleek in werkelijkheid een blik
terug in de tijd.

Hij was een middeleeuwse historicus, met name gespecialiseerd in de
handelsstad Venetië, waar het fortuin op en neer schommelde in een
wereld vol geweld. Toen hij nadacht over de opkomst en ondergang van
Venetië, viel hem op dat de manier waarop geweld wordt georganiseerd en
gecontroleerd een cruciale rol speelt in hoe schaarse middelen worden
ingezet.\footnote{Frederic C. Lane, `Economic consequences of organized
  violence', \emph{The Journal of Economic History} vol.18, nr.4
  (december 1958), p.~402.}

Wij zijn ervan overtuigd dat Lanes analyses over de concurrerende
toepassingen van geweld ons veel inzicht kunnen geven in hoe het leven
in het informatietijdperk waarschijnlijk zal verlopen. Maar verwacht
niet dat de meeste mensen zo'n extreem abstract betoog zullen opmerken,
laat staan het daadwerkelijk volgen. Terwijl de wereld haar aandacht
richt op partijdige debatten en excentrieke persoonlijkheden, glipt de
dynamiek van de megapolitiek bijna onopgemerkt voorbij. De gemiddelde
Noord-Amerikaan besteedt waarschijnlijk honderd keer meer aandacht aan
O.J. Simpson dan aan de nieuwste microtechnologieën, die zijn baan
overbodig lijken te maken en het politieke systeem, waar hij op
vertrouwt voor zijn werkloosheidsuitkeringen, ondermijnen.

\section{De ijdelheid van wensen}\label{de-ijdelheid-van-wensen}

De neiging om over het hoofd te zien wat van fundamenteel belang is, is
niet alleen voorbehouden voor de bankzitter die tv kijkt. Volgens
klassieke denkers verandert de wereld omdat de overtuigingen van mensen
veranderen -- een van de hardnekkige illusies die de natiestaat in stand
houdt. Zelfs ogenschijnlijk scherpe analisten komen vaak met
verklaringen en voorspellingen waarbij grote historische gebeurtenissen
worden voorgesteld alsof ze voort zijn gekomen uit wensdenken. Een
opvallend voorbeeld van dit type redenering verscheen op de opiniepagina
van de \emph{New York Times} in een artikel van Nicholas Colchester,
precies op het moment dat wij \emph{Goodbye, Nation State, Hello\ldots{}
What?} schreven.\footnote{Nicholas Colehester, `Goodbye nationstate,
  hello \ldots{} what?', \emph{New York Times}, 17 juli 1994, p.~E17.}
Niet alleen behandelde hij het onderwerp, de ondergang van de
natiestaat, precies het thema dat wij bespreken, maar plaatste hij
zichzelf ook als een treffend voorbeeld van hoe ver ons denken van de
norm afwijkt. Colchester is geen eenvoudige denker. Hij was redactioneel
directeur van de \emph{Economist Intelligence Unit}. Als iemand een
realistisch wereldbeeld belichaamt, dan is dat ongetwijfeld hij. Toch
betoogt hij in zijn artikel op meerdere punten dat `de komst van een
internationale overheid' nu onvermijdelijk is.

Waarom? Omdat de natiestaat wankelt en niet langer in staat is de
economische krachten in toom te houden.

Wat ons betreft is deze veronderstelling zo goed als absurd. Het is een
misvatting om te denken dat een nieuwe bestuursvorm automatisch opkomt
zodra een bestaande faalt. Volgens die logica hadden Haïti en Zaïre al
lang beter bestuur moeten kennen, simpelweg omdat wat zij hadden zo
overduidelijk ontoereikend was.

Het standpunt van Colchester, breed gedeeld door de weinigen in
Noord-Amerika en Europa die over zulke zaken nadenken, houdt totaal geen
rekening met de grotere megapolitieke krachten die bepalen welke
politieke systemen werkelijk levensvatbaar zijn. Dat vormt de kern van
dit boek. Wanneer we rekening houden met de technologieën die het nieuwe
millennium vormgeven, zullen we waarschijnlijk niet één wereldregering
krijgen, maar eerder microregeringen of zelfs anarchistische toestanden.

Voor elke serieuze analyse van de rol van geweld in het bepalen van de
regels waar iedereen zich aan moet houden, zijn er tientallen boeken
geschreven over de miniscule details van graansubsidies en honderden
over obscure aspecten van monetair beleid. Het gebrek aan een doordachte
benadering van de cruciale kwesties die echt het verloop van de
geschiedenis bepalen, weerspiegelt waarschijnlijk de relatieve
stabiliteit van de machtsstructuur van de afgelopen eeuwen. De vogel die
op de rug van een nijlpaard in slaap valt, maakt zich geen zorgen over
zijn zitplek totdat het dier in beweging komt. Dromen, mythen en
fantasieën spelen een veel grotere rol in het vormgeven van wat men de
`sociale wetenschappen' noemt dan men doorgaans vermoedt.

Nergens is dit duidelijker dan in de talloze teksten over economische
rechtvaardigheid. Terwijl men eindeloos praat en schrijft over wat
eerlijk is of niet, wordt zelden diepgaand onderzocht hoe geweld de
samenleving vormgeeft, en dus de grenzen bepaalt waarbinnen de economie
opereert. Moderne ideeën over economische rechtvaardigheid
veronderstellen echter vaak dat de samenleving dient aangestuurd te
worden door een uiterst krachtig dwangapparaat (één dat in staat is om
welvaart af te nemen en te herverdelen). Deze vorm van macht is slechts
een paar generaties geleden ontstaan, en nu is ze aan het afbrokkelen.

\subsection{Big Brother over de sociale
zekerheid}\label{big-brother-over-de-sociale-zekerheid}

Door industriële technologie beschikten overheden in de twintigste eeuw
over meer controle-instrumenten dan ooit tevoren. Voor een tijd leek het
onvermijdelijk dat overheden het geweld zo effectief zouden
monopoliseerden dat er nauwelijks nog ruimte zou blijven voor
individuele autonomie. Halverwege de eeuw durfde niemand te verlangen
naar een triomf van het Soevereine Individu.

Enkele van de scherpste waarnemers uit het midden van de twintigste eeuw
waren ervan overtuigd dat de drang van natiestaten om de macht te
centraliseren uiteindelijk zou leiden tot een totalitaire overheersing
van alle levensdomeinen. In George Orwells \emph{1984}, uitgegeven in
1949, hield `Big Brother' elk individu nauwlettend in de gaten, terwijl
hij vergeefs probeerde ook maar een spoor van autonomie en eigenwaarde
te behouden. Het leek een verloren strijd. Friedrich von Hayeks
\emph{The Road to Serfdom}, uit 1944, benaderde het onderwerp op een
wetenschappelijke manier en betoogde dat vrijheid ten onder zou gaan aan
een nieuwe vorm van economische controle, waardoor de staat uiteindelijk
de absolute macht zou krijgen. Deze werken schreef men vóór de opkomst
van microprocessoren, die later een scala aan technologieën zouden
introduceren waarmee zelfs kleine groepen en individuen onafhankelijk
van de centrale autoriteit kunnen opereren.

Hoe scherp Hayek en Orwell ook waren, ze bleken uiteindelijk te
pessimistisch. De geschiedenis heeft enkele verrassingen voortgebracht.
Het totalitaire communisme hield nauwelijks stand tot 1984. Een nieuwe
variant van lijfeigenschap zou in het volgende millennium nog kunnen
ontstaan als overheden erin slagen de bevrijdende kracht van
microtechnologie te onderdrukken. Het is echter veel waarschijnlijker
dat we ongekende kansen en meer individuele autonomie zullen ervaren.
Waar onze ouders zich zorgen over maakten, blijkt mogelijk helemaal geen
probleem te vormen. Wat zij als vaste en blijvende kenmerken van het
sociale leven beschouwden, lijkt nu gedoemd te verdwijnen. Zodra de nood
het eist, passen we ons aan en organiseren we ons leven naar de nieuwe
werkelijkheid.

\subsection{De risico's van
voorspellen}\label{de-risicos-van-voorspellen}

We wagen ongetwijfeld een stukje van onze waardigheid als we proberen om
ingrijpende veranderingen in de organisatie van ons leven en in de
cultuur die ons verbindt te voorspellen en te verklaren. De meeste
voorspellingen blijken uiteindelijk belachelijk, en hoe dramatischer de
voorspelde verandering, hoe beschamender zij doorgaans uitpakken. De
wereld gaat niet ten onder, de ozonlaag verdwijnt niet en de voorspelde
ijstijd maakt plaats voor wereldwijde opwarming. Ondanks alle signalen
die wijzen op het tegenovergestelde, zit er nog steeds olie in de tank.
Meneer Antrobus, de doorsnee-man uit \emph{The Skin of Our Teeth}, weet
te voorkomen dat hij bevriest, overleeft oorlogen en dreigende
economische rampen, en wordt oud, terwijl hij de weldoordachte
waarschuwingen van experts links laat liggen.

De meeste pogingen om de toekomst te `onthullen' komen al snel komisch
over. Zelfs als eigenbelang onze blik op scherp zet, blijft onze visie
op de toekomst vaak kortzichtig. In 1903 stelde Mercedes: `Er zullen
nooit 1 miljoen auto's wereldwijd rondrijden, omdat het onwaarschijnlijk
is dat 1 miljoen ambachtslieden tot chauffeurs zullen kunnen worden
opgeleid.'\footnote{Norman Macrae, `Governments in Decline', Cato Policy
  Report, juli/augustus 1992, p.~10.}

Dit besef zou ons tot zwijgen moeten brengen, maar dat gebeurt niet. We
schuwen het niet om onze welverdiende portie spot te incasseren. Als we
er ver naast zitten, mogen toekomstige generaties naar hartenlust
lachen, als ze zich überhaupt herinneren wat we hebben gezegd. Het
durven uiten van een gedachte houdt altijd het risico in dat je het mis
hebt. We zijn heus niet zo star of voorzichtig dat we geen fouten durven
maken. Liever delen we ideeën die u mogelijk iets opleveren, dan dat we
ze inslikken uit angst dat ze achteraf overdreven of pijnlijk blijken.

Arthur C. Clarke merkte scherp op dat er twee essentiële redenen zijn
waarom voorspellingen over de toekomst doorgaans mislukken: `gebrek aan
moed en gebrek aan verbeeldingskracht'.\footnote{Arthur C. Clarke,
  \emph{Profiles of the Future: An Enquiry into the Limits of the
  Possible} (Londen: Victor Gollancz Ltd., 1962), p.~13.} Hij schreef
dat `een gebrek aan moed het vaakst voorkomt. Het treedt op wanneer,
ondanks dat alle relevante feiten bekend zijn, de voorspeller niet
inziet dat ze onvermijdelijk tot één conclusie leiden. Sommige van deze
voorspellingen zijn zo belachelijk dat ze bijna ongelooflijk
lijken.'\footnote{Ibid.}

Als onze benadering van de informatierevolutie tekortschiet -- en dat
zal onvermijdelijk gebeuren -- komt dat eerder door een gebrek aan
verbeeldingskracht dan door een gebrek aan moed. Het voorspellen van de
toekomst is immers altijd een gedurfde onderneming die terecht scepsis
oproept. Wellicht zal de tijd aantonen dat onze conclusies de bal
compleet mis slaan. In tegenstelling tot Nostradamus gedragen wij ons
niet als profeten. We doen geen vage voorspellingen door met een
toverstok in een kom water te roeren of door horoscopen te raadplegen,
en we gebruiken ook geen cryptisch taalgebruik. We willen u een scherpe
en onbevooroordeelde kijk geven op zaken die wel eens van groot
persoonlijk belang zouden kunnen blijken.

Wij vinden het onze plicht om onze ideeën uiteen te zetten, ook al
zullen sommigen ze ervaren als heiligschennis, omdat ze anders wellicht
onopgemerkt blijven. In de gesloten denkwijze van de late
geïndustrialiseerde samenleving gaan ideeën niet zo vrij rond als via de
gevestigde media zou moeten.

Dit boek is tot stand gekomen vanuit een constructieve instelling. Het
is het derde boek dat we gezamenlijk schrijven, waarin we de
verschillende fasen van de ingrijpende verandering die momenteel gaande
is analyseren. Net als \emph{Blood in the Streets} en \emph{The Great
Reckoning} vormt dit een gedachte-experiment. Het onderzoekt hoe de
industriële samenleving ten onder zal gaan en vervolgens opnieuw vorm
zal krijgen. Wij voorzien dat de komende jaren verbluffende paradoxen
aan het licht zullen komen. Enerzijds zul je de opkomst van een nieuwe
vrijheid ervaren, waarin het Soevereine Individu centraal staat, en mag
je verwachten dat de productiviteit vrijwel volledig bevrijd zal worden.
Tegelijkertijd zien wij het verval van de moderne natiestaat voor ons.
Veel garanties omtrent gelijkheid, die in de twintigste eeuw als
vanzelfsprekend werden beschouwd, zullen met die staat verdwijnen. Wij
zijn ervan overtuigd dat de representatieve democratie, zoals we die nu
kennen, zal vervagen en plaats zal maken voor een keuzedemocratie in de
digitale marktplaats. Indien onze inzichten kloppen, wordt de politiek
van de volgende eeuw veel diverser en minder belangrijk dan we nu gewend
zijn.

We zijn ervan overtuigd dat onze redenering goed te volgen is, ook al
brengt ze u langs intellectueel ruig terrein en minder vertrouwde
uithoeken. Mocht onze intentie op bepaalde punten niet glashelder
overkomen, dan is dat niet omdat we geheimzinnig willen doen of de
klassieke dubbelzinnigheid toepassen, zoals in de vage cryptische
uitspraken van sommige anderen. Wij gebruiken geen dubbelzinnigheid.
Indien onze argumenten onduidelijk lijken, komt dat vooral doordat we er
niet in geslaagd zijn onze overtuigende ideeën simpel en toegankelijk te
verwoorden. In tegenstelling tot veel voorspellers willen wij dat je
onze gedachtegang doorgrondt en zelfs eigen maakt. Onze visie steunt
niet op spirituele zweverigheid of kosmische bewegingen, maar op
ouderwetse, ongepolijste logica. Om zeer logische redenen zijn we ervan
overtuigd dat microprocessing de natiestaat onvermijdelijk zal
ondermijnen en vernietigen, terwijl het tegelijkertijd nieuwe vormen van
sociale organisatie zal voortbrengen. Het is niet alleen mogelijk, maar
ook noodzakelijk dat u zich een beeld vormt van bepaalde aspecten van de
nieuwe manier van leven die misschien eerder werkelijkheid wordt dan u
verwacht.

\subsection{Ironieën van een voorspelde
toekomst}\label{ironieuxebn-van-een-voorspelde-toekomst}

Al eeuwenlang wordt het einde van dit millennium beschouwd als een
beladen moment in de geschiedenis. Meer dan 850 jaar geleden stelde
St.~Malachy vast dat het jaar 2000 de datum van het Laatste Oordeel zou
zijn. In 1934 voorspelde de Amerikaanse helderziende Edgar Cayce dat de
aarde in 2000 op haar as zou draaien, waardoor Californië in tweeën zou
splijten en New York City en Japan overstromingen zouden ondervinden. In
1980 verklaarde de Japanse raketwetenschapper Hideo Itokawa dat de
uitlijning van de planeten in een `Grand Cross' op 18 augustus 1999
grootschalige milieucatastrofes zou veroorzaken, met als uiteindelijke
gevolg het einde van het menselijk leven op aarde.\footnote{A. T. Mann,
  \emph{Millennium Prophecies: Predictions for the Year 2000}
  (Shafiesbury, Engeland: Element Books, 1992), pp.~88, 112, 117.}

Zulke apocalyptische visioenen worden al snel het onderwerp van spot.
Hoewel het jaar 2000 een symbolisch en opvallend rond getal is, is het
in feite niet meer dan een toevallig product van de westerse,
christelijke kalender. In andere kalenders en jaartellingen beginnen de
eeuwen en millennia op totaal andere tijdstippen in de geschiedenis.
Volgens de islamitische kalender komt 2000 n.Chr. overeen met het jaar
1378, wat op het eerste gezicht een gewoon jaartal lijkt. In de Chinese
kalender, die zich elke zestig jaar herhaalt, valt 2000 in een
drakenjaar, onderdeel van een voortdurende cyclus die al millennia
doorgaat. Er schuilt echter meer achter het jaar 2000 dan louter
religieuze betekenis. Het belang ervan wordt niet alleen gedragen door
de christelijke traditie, maar ook door de beperkingen van de
informatietechnologie in de twintigste eeuw.

Het zogenaamd Y2K-probleem, een potentieel verwoestende logische fout in
miljarden regels computercode, dreigt op middernacht van het millennium
de vitale onderdelen van de industriële samenleving plat te leggen. Veel
computers en microprocessors maken nog steeds gebruik van software uit
de begintijd van de computer, toen geheugenruimte met \$600.000 per
megabyte kostbaarder was dan goud. Om dure opslagruimte te besparen
gebruikten de eerste programmeurs slechts de laatste twee cijfers van
het jaartal. Deze gewoonte om tweecijferige datumvelden te gebruiken
werd overgenomen in de meeste software voor mainframecomputers, en vond
later ook brede toepassing in pc's en zogenoemde embedded chips ---
microprocessors die bijna alles aansturen: van videorecorders tot het
ontstekingssysteem van auto's, van beveiligingssystemen tot telefoons,
van schakeltechniek in het telefoonnetwerk tot proces- en
controlesystemen in fabrieken, energiecentrales, olieraffinaderijen,
chemische installaties, pijpleidingen en meer. In zo'n verkort veld
wordt het jaar 1999 bijvoorbeeld weergegeven als `99'. De vraag is
echter: wat gebeurt er in het jaar 2000? Een groot deel van de computers
zouden de `00' mogelijk interpreteren als het jaar 1900. Hierdoor zouden
tal van niet-geüpdatete computers en andere digitale apparaten het jaar
2000 onjuist kunnen registreren.

Dit zou tot ernstige datacorruptie kunnen leiden, wat onbedoeld ook
nieuwe mogelijkheden voor informatieoorlogvoering introduceert. In de
informatiesamenleving zullen potentiële vijanden complete systemen
kunnen ontregelen met `logische bommen': stukjes code die cruciale data
saboteren en zo alles ontwrichten.

Tijdens een militaire operatie hoef je bijvoorbeeld niet per se een
vliegtuig neer te schieten als je de gegevens, die essentieel zijn voor
de veilige werking ervan, kunt verstoren. Datacorruptie kan bijna net zo
verwoestend zijn als fysieke wapens wanneer het gaat om het ontregelen
van een moderne samenleving. De mogelijke verstrekkende gevolgen hiervan
zouden bij nader inzien voor de hand moeten liggen. Zo meldde de
\emph{Mail of London} op 14 december 1997 dat luchtvaartmaatschappijen
wereldwijd van plan waren om op 1 januari 2000 honderden vluchten te
annuleren, uit angst dat de luchtverkeersleidingssystemen zouden
falen.\footnote{Yardeni, op. cit., p.~45.} Niet alleen deze systemen,
maar ook de datumgevoelige functies in de vliegtuigen zelf zouden risico
lopen. Volgens Boeing hebben veel vliegtuigen Y2K-aanpassingen nodig.
Een groot deel van de apparatuur zou storingen kunnen vertonen wanneer
ze een gebeurtenis op een ongeldige datum registreren. De
computergestuurde fly-by-wire-systemen die vliegtuigen aansturen, zouden
defect kunnen raken als hun programmering zou concluderen dat cruciaal
onderhoud voor het laatst in 1900 heeft plaatsgevonden. Ze kunnen zelfs
in een foutlus vastlopen en daardoor uitschakelen.

De potentieel dodelijke kettingreacties van een logische tijdbom die
cruciale besturingssystemen platlegt, zouden de eeuwwisseling op een
onvergetelijk onaangename wijze kunnen markeren. Zelfs als je op
oudejaarsnacht niet in een vliegtuig zit, kunnen talloze apparaten om je
heen crashen door foutlussen en je dagelijkse leven flink verstoren.
Zorg ervoor dat je ongelukken voorkomt, of ze nu ontstaan door
pacemakers die niet Y2K-conform zijn of door dronken millenniumvierders.
Als zulke pacemakers kunnen falen, kan ook het telefoonsysteem het
begeven, waardoor je mogelijk niet eens de hulpdiensten kunt bereiken.

Tenzij je je in een regio met slecht functionerende infrastructuur
bevindt, ben je eraan gewend dat je gewoon je telefoon oppakt en direct
kunt bellen. Gelukkig hoeven we ons zelden druk te maken over de
technische details van telefoonnetwerken. De schakelapparatuur en
routers van de telecominfrastructuur blijken echter erg afhankelijk van
datumvelden. Alle verbindingen worden opgeslagen met datum en tijd, wat
essentieel is voor het berekenen van de gespreksduur en de daarop
gebaseerde facturering. Als je op 31 december 1999 om 23:59:30 een
gesprek van precies één minuut voert en het systeem om 00:00:30 dit
gesprek registreert met een negatieve duur van meer dan 99 jaar, kunnen
foutlussen ontstaan en schakelt het systeem zich mogelijk uit. Hoewel
langeafstands firma's, en lokale dienstverleners waarschijnlijk ook,
enorme bedragen investeren in het updaten van hun schakelsystemen zodat
deze Y2K-conform worden, kan het hele netwerk in de problemen komen als
zelfs een paar kleinere bedrijven niet voldoen en uitvallen. In dat
scenario mag je van geluk spreken als je op 1 januari 2000 überhaupt nog
een verbinding krijgt.

Zoals Y2K-expert Peter de Jager het verwoordde: ``Als we het vermogen
verliezen om te bellen, verliezen we alles. We verliezen elektronische
betalingen, we verliezen de handel, we verliezen het bankkantoor.'' En
de gevolgen van Y2K-fouten zouden zelfs nog verder kunnen reiken.

Niemand weet precies in hoeverre cruciale systemen door het
millenniumprobleem ten onder zullen gaan. Embedded systemen, die niet
herprogrammeerd kunnen worden en dus vervangen moeten worden zodra zij
datumgevoelige fouten vertonen, vind je onder andere in auto's,
vrachtwagens en bussen die na 1976 gebouwd zijn. De kans om betrokken te
raken bij een ongeluk met voertuigen, bestuurd door mensen met
niet-conforme pacemakers, zal uiteindelijk dus misschien niet heel groot
zijn omdat die voertuigen mogelijk niet eens zullen starten. Ook in
energiecentrales, water- en rioleringssystemen, medische apparatuur,
militaire uitrusting, vliegtuigen, offshore olieplatforms, olietankers,
alarmsystemen en liften komen embedded systemen veel voor. Hoewel veel
installaties met microprocessors zelf geen datumgevoelige functies
uitvoeren, zijn ze voor hun interne werking vaak wel afhankelijk van een
klok, die mogelijk Y2K-gevoelig is.

\section{Mainframes en de
Y2K-tijdbom}\label{mainframes-en-de-y2k-tijdbom}

De grootschalige bestuurs- en controlesystemen van de overheid en grote
bedrijven, waarin enorme transacties via mainframecomputers worden
verwerkt, stonden aanvankelijk centraal in de Y2K-bezorgdheid. Deze
systemen draaien op grote machines met software die vaak decennia oud en
grotendeels niet Y2K-conform is. Daarom richtten de eerste
Y2K-waarschuwingen, voor het eerst geuit door Peter de Jager begin jaren
1990, zich voornamelijk op de noodzaak om de besturingssystemen van
grote, multiprocessor-mainframes te moderniseren. De heer de Jager uitte
de zorg dat er mogelijk niet genoeg programmeurs beschikbaar zullen zijn
die kennis hebben van COBOL (de oude computertaal waarin de
noodzakelijke patches en reparaties aan datumgevoelige code uitgevoerd
moeten worden) zelfs als ieder bedrijf en overheidsinstantie met een
kwetsbaar systeem enkele jaren daarvoor een noodprogramma zou hebben
opgestart. Aangezien dit niet is gebeurd en veel beheerders van
datumgevoelige informatiesystemen pas recent begonnen zijn met het
onderzoeken van hun kwetsbaarheden, mag je er met grote zekerheid van
uitgaan dat veel mainframesystemen niet goed voorbereid zullen zijn.

Dit vormt uiteraard een groot probleem, omdat er in de huidige economie
echt geen alternatief bestaat voor computerverwerking. De meeste
bedrijven die groot genoeg zijn om een mainframe te gebruiken voor het
verwerken van hun transacties, zijn namelijk afhankelijk van een
dusdanig groot transactievolume dat ze niet meer met de ouderwetse
papieren systemen uit de negentiende eeuw kunnen functioneren. Als die
ondernemingen noodgedwongen weer op papieren documenten zouden moeten
vertrouwen, zouden zij slechts een fractie van hun gebruikelijke
transacties kunnen afhandelen. Het plotselinge inkomensverlies, als
gevolg van zo'n dramatische daling in bedrijfsactiviteit, zou het
voortbestaan van bijna alle bedrijven, op enkele van de best
gekapitaliseerde na, ernstig in gevaar brengen.

Bijna alle financiële processen (zoals facturering, inkoop- en
loonadministratie, voorraadbeheer en naleving van regelgeving) zullen
compleet worden verstoord. Enorme hoeveelheden data gaan verloren als
computers crashen of foutieve informatie uitspuwen door het
Y2K-probleem. In sommige gevallen blijkt het zelfs een zegen als
systemen direct crashen, in plaats van dat hun data geleidelijk wordt
gecorrumpeerd tot een massale storing het probleem aan de kaak stelt.
Wat zal er gebeuren wanneer een backup-programma bestanden die
oorspronkelijk op 07/04/99 zijn aangemaakt, kopieert naar een update op
01/04/00? Wie zal het zeggen? Zal de computer een betaling voor een
verzekeringspolis die op 4 januari `1900' uitgevoerd is, interpreteren
als een signaal dat de polis al een eeuw in gebreke is, met als gevolg
dat de polis wordt geannuleerd en uit de administratie verdwijnt? Zullen
computers van banken en financiële instellingen proberen om honderd jaar
rente in rekening te brengen voor leningen die de overgang naar het
nieuwe millennium overspannen? Zullen banken en effectenmaatschappijen
nauwkeurig rekeningstanden bijhouden en tijdig toegang geven tot
tegoeden? Dit zijn slechts enkele van de intrigerende vraagstukken
waarmee je te maken krijgt als gevolg van het Y2K-probleem.

\begin{quote}
Dit is mogelijk het meest destructieve onderdeel van het jaar
2000-probleem. Dit betreft niet het ongemak wanneer uw salaris enkele
dagen te laat komt. Dit gaat over echte chaos op straat. -- Dr.~Leon
Kappelman, medevoorzitter van de Year 2000-werkgroep van de Society For
Information Management.
\end{quote}

Ook moet je je afvragen wat er gebeurt als de elektriciteit uitvalt door
Y2K-gerelateerde storingen. Zonder stroom functioneren de meeste
systemen niet, ook degene die niet direct door Y2K-problemen getroffen
zijn, zoals je koelkast, vriezer en mogelijk zelfs je verwarming.
Y2K-problemen kunnen de veiligheidsgerelateerde toegangs- en
beheerssystemen in kerncentrales verstoren. Zo dragen medewerkers in
kerninstallaties dosimetrie-apparaten die de hoeveelheid
stralingsblootstelling bijhouden die zij in de faciliteit ontvangen.
Deze apparaten worden regelmatig gecontroleerd en de gegevens over de
blootstellingsniveaus worden opgeslagen in een computersysteem dat de
toegang van medewerkers tot de faciliteit beheert. Uiteraard, als deze
bestuurlijke computers uitvallen, raken al die uitgebreide controles,
die bedoeld zijn om een veilige werking en goed onderhoud te garanderen,
volledig in de war. Belangrijker nog: een memo van de Nuclear Regulatory
Commission merkt op dat veel `niet-veiligheidsgerelateerde, maar
belangrijke computergebaseerde systemen, hoofdzakelijk databases en de
dataverzameling die onmisbaar is voor de werking van de centrale,'
datumgevoelig zijn.

Ook reguliere elektriciteitscentrales lopen net zo goed risico op
Y2K-verstoringen. Kolencentrales zijn immers gevoelig voor
onderbrekingen in het transportsysteem dat de kolen naar de ketels
vervoert. In het winterseizoen van 1997/1998 moesten exploitanten van
kolencentrales in sommige gevallen hun productie verminderen door
vertragingen in de leveringsketen van Western Coal, veroorzaakt door de
fusie van de spoorwegsystemen van Southern Pacific en Union Pacific. Het
probleem ontstond door incompatibiliteiten tussen de computerbestuurde
controle- en dispatchsystemen die beide spoorwegmaatschappijen
hanteerden. Een woordvoerder van Union Pacific noemde de integratie van
de twee systemen een `nachtmerrie', ondanks dat Union Pacific
Technologies als koploper wordt gezien in de ontwikkeling van
geautomatiseerde transportsystemen. Door de programmeerproblemen kon de
spoorwegmaatschappij de bewegingen van haar goederenwagons niet
nauwkeurig volgen. Het onvermogen van Union Pacific om de systemen van
Southern Pacific succesvol te integreren, is een slecht voorteken voor
wat er kan gebeuren wanneer de logische tijdbommen van Y2K het
transport, de elektriciteitsopwekking en andere economische sectoren
verstoren.

De grootste zorg rond het elektriciteitsnet komt voort uit het feit dat
computers voortdurend toezicht houden op het hele systeem en het actief
aansturen, waardoor elektriciteit efficiënt wordt verplaatst van
overschotgebieden naar regio's met een tekort. Computers monitoren dit
proces nauwgezet om spanningspieken en storingen te voorkomen. Elke
overdracht wordt geregistreerd met vermelding van tijd en datum,
vergelijkbaar met de registratie bij een telefoongesprek. Robuuste
mechanische relais maken de verbindingen weliswaar mogelijk, maar
computersystemen sturen ze aan. Die computercontroles, die essentieel
zijn voor de vermogensbalancering, kunnen echter om dezelfde redenen
falen als telefonienetwerken. In Noord-Amerika zijn de systemen die de
stroomverdeling regelen sterk afhankelijk van telefoonverbindingen zoals
T-1-lijnen en draadloze netwerken. Als het telefonienetwerk uitvalt, kun
je er dus ook op rekenen dat de elektriciteitsvoorziening instort. En de
ervaring in Canada in januari 1998 toont aan hoe lastig het is om het
systeem weer aan de praat te krijgen zodra de elektriciteit over een
groot gebied wegvalt. Stroomuitvallen zouden oncomfortabel lang kunnen
aanhouden.

\section{Y2K en het nucleaire
arsenaal}\label{y2k-en-het-nucleaire-arsenaal}

Als de stroom in het hart van de winter zou uitvallen, betekent dat in
de moderne economie een enorme verstoring en zelfs een potentiële
bedreiging voor de volksgezondheid, vooral voor mensen die afhankelijk
zijn van elektrische verwarming en medische apparatuur. Het ergste
scenario is echter nog grimmiger. John Koskinen, hoofd van het Y2K
Conversion Council van president Clinton, waarschuwt dat de Amerikaanse
militaire arsenalen mogelijk niet meer functioneren zodra de klok op 31
december 1999 middernacht slaat. Hoewel hij geen onnodige paniek wil
zaaien, voegt Koskinen eraan toe: `Er moet zich zorgen over worden
gemaakt.' Een van de zorgen over de nucleaire raketten was dat `als de
data niet naar behoren functioneert, ze vanzelf zouden kunnen lanceren.'
Natuurlijk geldt dezelfde zorg, zo niet meer, voor de Russische
nucleaire raketten. Ruslands faillissement heeft de noodzakelijke
upgrades voor Y2K-compatibiliteit nog problematischer gemaakt dan in de
Verenigde Staten, en er is bewijs dat Rusland de Y2K-conversie nog niet
serieus neemt. Hoewel iedereen hoopt dat er geen toevallige lanceringen
plaatsvinden, bestaat er weinig twijfel dat de jaarwisseling naar 2000
de wereldwijde veiligheid in gevaar zal kunnen brengen, simpelweg
doordat militaire communicatiesystemen in veel landen mogelijk niet naar
behoren zullen functioneren. Zoals Koskinen het verwoordt: `Als je in
een land zit en plotseling niet precies kunt achterhalen wat er gaande
is, omdat je communicatiesystemen niet optimaal werken, raak je nog
nerveuzer.' Neem dat dus zeker mee in je lijst met Y2K-zorgen. Deze
logische tijdbom kan de lancering van echte explosieve wapens
bewerkstelligen, wat het gevaar aantoont dat voortkomt uit
informatieoorlogvoering tegen gecentraliseerde beheers- en
controlesystemen.

Terroristen die een gecentraliseerd systeem willen treffen, zouden juist
31 december 1999 kunnen kiezen als moment om toe te slaan, omdat op dat
moment veel systemen op hun meest kwetsbaar zijn. Niet alleen raken de
communicatiesystemen zwaar overbelast, met als gevolg dat de stroom
uitvalt, voertuigen niet starten en de 112-diensten van de politie,
brandweer en ambulance niet bereikbaar zijn, maar ook tal van andere
voorzieningen, zoals de luchtverkeersleiding, kunnen het begeven. Zonder
stroom is er geen kraanwater, vallen rioolinstallaties uit en zullen
verkeerslichten ermee ophouden. Binnen enkele uren nadat het
transportsysteem instort, zal het voedsel in supermarkten uitverkocht
zijn, of zelfs geplunderd. Recente ervaringen in Amerikaanse steden
laten zien dat een gebrek aan stroom, water, verwarming, licht en
communicatie met hulpdiensten, waaronder politie en brandweer, samen het
verval van de beschaving kunnen inluiden. Hoewel niemand met zekerheid
kan vaststellen wat de impact van het Y2K-probleem zal zijn, zou dit
kunnen uitmonden in plunderingen en rellen op straat, zeker als het
duidelijk wordt dat er problemen zullen ontstaan bij het uitgeven van
loon-, bijstands- en pensioenbetalingen.

\begin{quote}
Wij zullen niet zijn wat wij geweest zijn, maar we zullen beginnen
anders te zijn. -- Joachim de Fiore\footnote{Geciteerd in Frooso, op.
  cit., p.~40.}
\end{quote}

Doemvoorspellingen rondom het nieuwe millennium berusten niet per se op
de christelijke theologie, maar sluiten naadloos aan bij de
millennialtraditie van Joachim de Fiore. Zijn meditaties deden hem
immers geloven dat Christus slechts `het tweede scharnier in de
geschiedenis' vormde en dat er onvermijdelijk een volgend scharnier zou
volgen.\footnote{Ibid} Zo stelt filosoof Michael Grasso dat de
informatierevolutie de menselijke geschiedenis stuurt naar de
verwezenlijking van de profetische visie van de Westerse wereld. Hij
noemt dit `technocalypse.' Of technologische ontwikkelingen nu wel of
niet door millenniale visies worden beïnvloed, het Y2K-fenomeen blijft
een product van de overheersende westerse blik op de tijd. Op een
merkwaardige manier sluit het aan bij dromen, visioenen of hun numerieke
interpretaties, zoals Newtons duiding van de profetieën van Daniël. Deze
intuïtieve gedachten ontstaan vanuit een perspectief dat de geboorte van
Christus als middelpunt van de geschiedenis beschouwt. Bovendien
versterken grote, ronde getallen dit perspectief, want elke handelaar
herkent de bijzondere aantrekkingskracht die zij uitoefenen.

Het tweeduizendste jaar van onze jaartelling kan niet anders dan een
brandpunt worden voor de verbeelding van intuïtieve denkers.

Critici zouden deze voorspellingen zonder moeite belachelijk kunnen
maken, zelfs zonder in te gaan op de dubbelzinnige en omstreden
theologische begrippen van de Apocalyps en het Laatste Oordeel, die
juist zoveel kracht aan deze vooruitzichten geven. Opmerkelijk genoeg
weegt de Y2K-bug zwaarder dan de historische onnauwkeurigheden in onze
tijdrekening, die het symbolische gewicht van het jaar 2000 binnen de
Christelijke jaartelling juist hadden kunnen afzwakken. Het jaar 2000
kan een keerpunt in de geschiedenis betekenen, simpelweg omdat het de
overgang naar een nieuw millennium markeert. In principe begint het
volgende millennium pas in 2001, aangezien het jaar 2000 slechts het
tweeduizendste jaar sinds de geboorte van Christus aanduidt. Tenminste,
dat zou zo zijn als Christus daadwerkelijk in het eerste jaar van ons
huidige tijdperk was geboren, maar dat is niet zo. In 533, toen men
bepaalde om de geboorte van Christus als uitgangspunt te nemen voor de
westerse jaartelling in plaats van de oprichtingsdatum van Rome, maakten
de monniken die deze nieuwe conventie invoerden een rekenfout.
Tegenwoordig gaan we ervan uit dat hij in 4 v.Chr. werd geboren. Volgens
deze berekening eindigt de volledige periode van tweeduizend jaar sinds
zijn geboorte al rond 1997. Dit is ook de reden waarom Carl Jung dit
ogenschijnlijk vreemde jaartal als startdatum voor een Nieuwe Tijd
voorstelde.

Lach erom als je wilt, maar wij wijzen intuïtieve inzichten in de
geschiedenis beslist niet af. Hoewel ons betoog op logica rust en niet
op vooringenomen veronderstellingen, raken we diep onder de indruk van
de profetische kracht van het menselijk bewustzijn. Keer op keer
bevestigt het de visioenen van krankzinnigen, helderzienden en heiligen.
Hetzelfde geldt voor de transformatie die rond het jaar 2000 lijkt aan
te breken. De datum, die al lang verankerd is in de westerse
verbeelding, lijkt het keerpunt te markeren dat bewijst dat de
geschiedenis een bestemming heeft. We kunnen niet precies verklaren
waarom dat zo is, maar we hebben er desondanks alle vertrouwen in.

Onze intuïtie vertelt ons dat de geschiedenis een bestemming kent en dat
vrije wil en determinisme twee zijden van dezelfde medaille vormen. De
menselijke interacties die de geschiedenis vormgeven, lijken erop te
wijzen dat een bepaald lot hen stuurt. Net zoals een elektronenplasma
(een dicht samengepakt gas van elektronen) zich als een complex systeem
gedraagt, handelen mensen op een vergelijkbare manier. De individuele
bewegingsvrijheid van elektronen blijkt immers verenigbaar te zijn met
een sterk geordend collectief gedrag. Zoals David Ohm over een
elektronenplasma opmerkte, is de menselijke geschiedenis `een hoogst
georganiseerd systeem dat zich als een geheel gedraagt.'

Het doorgronden van de werking van de wereld betekent dat je leert
aanvoelen hoe de menselijke samenleving handelt volgens de wiskundige
wetten van de natuur. De werkelijkheid verloopt niet lineair, terwijl de
meeste mensen juist een lineair patroon verwachten. Om de dynamiek van
verandering echt te begrijpen, moet je inzien dat onze samenleving (net
als andere complexe systemen in de natuur) cyclische patronen en abrupte
breuken vertoont. Dit houdt in dat bepaalde aspecten van de geschiedenis
de neiging hebben zich te herhalen en dat ingrijpende veranderingen vaak
plotseling plaatsvinden in plaats van geleidelijk.

Te midden van de vele ritmes in het menselijk bestaan duikt er een
raadselachtige vijfhonderdjarige cyclus op, die telkens samenvalt met
grote omslagen in de westerse geschiedenis. Nu het jaar 2000 nadert,
valt het op dat het laatste decennium van elke eeuw die deelbaar is door
vijf, altijd gepaard gaat met een diepgaande transitie binnen de
Westerse wereld. Dit patroon markeert nieuwe fasen van sociale
organisatie, vergelijkbaar met de manier waarop geboorte en dood de
cyclus van menselijke generaties bepalen. Dit is al zo sinds ten minste
500 v.Chr., toen de Griekse democratie opkwam door de constitutionele
hervormingen van Kleisthenes in 508 v.Chr. De volgende vijfhonderd jaar
kenden een periode van groei en versterking van de Oude economie, die
culmineerde in de geboorte van Christus in 4 v.Chr. Deze periode bleek
voor dat economische systeem tevens de tijd met de grootste welvaart te
zijn, waarin de rentevoeten hun laagste niveau bereikten. Een niveau dat
niet meer bereikt is tot in de moderne tijd.

In de daaropvolgende vijfhonderd jaar nam de welvaart geleidelijk af,
wat uiteindelijk leidde tot de ineenstorting van het Romeinse Rijk tegen
het einde van de vijfde eeuw na Christus. Het is de moeite waard om de
samenvatting van William Playfair hier te herhalen: ``Het hoogtepunt van
de Romeinse macht viel samen met de geboorte van Christus, in de tijd
van keizer Augustus. Vanaf dat moment ging het rijk langzaam achteruit,
tot in 490 n.~Chr. haar laatste legioenen uiteenvielen en de westerse
wereld afgleed in wat we nu kennen als de Donkere
middeleeuwen.''\footnote{William Playfair, \emph{An Inquiry into the
  Permanent Causes of the Decline and Fall of Powerful and Wealthy
  Nations: Designed to Show How the Prosperity of the British Empire May
  be Prolonged} (Londen: Greenland and Norris, 1805), p.~79.}

De daaropvolgende vijfhonderd jaar werden gekenmerkt door een krimpende
economie; de langafstandshandel kwam tot stilstand, steden raakten
verlaten, geld kwam in onbruik en kunst en geletterdheid verdwenen bijna
volledig. De ineenstorting van het West-Romeinse Rijk leidde tot het
verval van een effectief rechtsstelsel, waardoor er primitieve
regelingen ontstonden voor het oplossen van geschillen. Tegen het einde
van de vijfde eeuw werd bloedwraak steeds meer de norm, en in het jaar
500 vond het eerste geregistreerde geval van een godsoordeel plaats.

Nogmaals, duizend jaar geleden markeerde het laatste decennium van de
tiende eeuw een andere enorme omwenteling in sociale en economische
systemen. Wellicht is de minst bekende van deze transities de feodale
revolutie, die begon in een periode van totale economische en politieke
onrust. In \emph{Transformation of the Year One Thousand} stelt Guy
Bois, professor middeleeuwse geschiedenis aan de Universiteit van
Parijs, dat deze breuk aan het einde van de tiende eeuw zorgde voor de
volledige ineenstorting van de overblijfselen van oude instituties en de
opkomst van iets nieuws uit de anarchie van het feodalisme.\footnote{Guy
  Bois, \emph{The Transformation of the Year One Thousand: The Village
  of Lournard from Antiquity to Feudalism} (Manchester, Engeland:
  Manchester University Press, 1992).} Zoals Raoul Glaber het
verwoordde: `Er werd gezegd dat de hele wereld in één klap de lompen van
de oudheid van zich afschudde.'\footnote{Ibid., p.~150.} Het nieuwe
systeem dat plotseling opkwam, stimuleerde de geleidelijke heropleving
van de economische groei en zorgde ervoor dat de daaropvolgende
vijfhonderd jaar, die we nu de middeleeuwen noemen, gepaard gingen met
een heropleving van het gebruik van geld en internationale handel, naast
de herontdekking van rekenkunde, geletterdheid en tijdsbesef.

In het laatste decennium van de vijftiende eeuw vond een nieuw keerpunt
plaats. Europa herstelde zich uit het door de Zwarte Dood veroorzaakte
demografische tekort en begon vrijwel onmiddellijk haar macht over de
wereld uit te breiden. De `buskruitrevolutie', de `renaissance' en de
`reformatie' benoemden elk een aspect van deze overgangsperiode die de
moderne tijd inluidde. Het begon spectaculair toen Karel VIII Italië
binnenviel met nieuwe bronzen kanonnen. Dit opende de weg naar nieuwe
werelddelen, zoals bleek uit Columbus' reis naar Amerika in 1492. Die
ontdekking stimuleerde de meest dramatische economische groei die de
mensheid ooit heeft gekend en ging gepaard met een revolutie in de
natuurkunde en astronomie, wat de moderne wetenschap deed ontstaan.
Bovendien verspreidde de boekdruktechnologie de ideeën op grote schaal.

Wij bevinden ons nu op de drempel van een nieuwe millenniale
transformatie. De uitgebreide systemen voor beheer en controle die we
uit het industriële tijdperk hebben geërfd, kunnen bij de overgang naar
het nieuwe millennium uit elkaar vallen zoals de paardenkar. Of de
Y2K-logicabom nu wel of niet zal leiden tot een onmiddellijke
ineenstorting van de industriële samenleving maakt niet uit, het einde
ervan is hoe dan ook nabij. Wij verwachten dat de opkomst van de
informatiesamenleving de wereld ingrijpend zal veranderen, op manieren
die in dit boek worden toegelicht. Het is begrijpelijk dat je twijfels
hebt omdat een cyclus die zich slechts tweemaal per millennium herhaalt
niet genoeg herhalingen kent voor statistische significantie. Economen
benaderen zelfs kortere cycli met scepsis, omdat ze meer overtuigend
statistisch bewijs verlangen. Professor Dennis Robertson schreef ooit
dat we enkele eeuwen moeten afwachten alvorens met zekerheid te kunnen
spreken over het bestaan van de vier en acht tot tien jaar durende
handelscycli.\footnote{Geciteerd in S. B. Saul, \emph{The Myth of the
  Great Depression} (Londen: Macmillan, 1985), p.~10.} Volgens dat
criterium zou hij ongeveer dertigduizend jaar moeten wachten om zeker te
weten dat de vijfhonderdjarige cyclus geen toevalligheid betreft. Wij
zijn minder dogmatisch en erkennen liever dat de patronen van de
werkelijkheid complexer zijn dan de statische en lineaire
evenwichtsmodellen die de meeste economen hanteren.

Wij menen dat de komst van het jaar 2000 meer betekent dan slechts een
praktische onderverdeling van een eindeloos tijdscontinuüm. Wij geloven
dat het een keerpunt vormt tussen de Oude Wereld en een opkomende Nieuwe
Wereld. Het industriële tijdperk verdwijnt in hoog tempo, en ironisch
genoeg kan de ondergang ervan worden versneld doordat men in vroege
computers, door het dure geheugen, tweecijferige datumvelden ging
gebruiken. Toen Hollerithponskaarten slechts tachtig tekens konden
bevatten, leek het verstandig om om datums af te korten. In
tegenstelling tot de verwachtingen van de vroege programmeurs hield deze
afkorting echter vier decennia stand tot het einde van het millennium,
met een onbedoelde logicabom, die een groot deel van de industriële
samenleving zou kunnen verwoesten, tot gevolg. Het Office of Management
and Budget van de Amerikaanse overheid omschreef het computerprobleem in
\emph{Getting Federal Computers Ready for 2000}, een rapport van 7
februari 1997: `Tenzij ze gerepareerd of vervangen worden, zullen ze bij
de overgang van de eeuw op één van de volgende drie manieren falen:
ofwel wijzen ze legitieme invoer af, ofwel berekenen ze foutieve
resultaten, ofwel functioneren ze simpelweg niet.' Deze drie gevolgen
zouden samen de industriële samenleving compleet kunnen lamleggen. In
elk geval staat massaproductietechnologie op het punt te worden
overschaduwd door een nieuwe miniaturisatietechnologie, en een crisis op
korte termijn zal dat proces alleen maar versnellen. Dankzij de nieuwe
informatietechnologie is een wetenschap op het gebied van niet-lineaire
dynamica ontstaan, waarvan de verrassende conclusies nog slechts
afzonderlijke elementen zijn die samen moeten worden geweven tot een
coherent wereldbeeld. We leven in het tijdperk van de computer, maar
onze dromen worden nog steeds op het weefgetouw gesponnen. We houden
vast aan de metaforen en denkbeelden van het industrialisme. Denkers die
stierven lang voordat bijna iedereen die nu leeft geboren werd, zoals
Adam Smith en Karl Marx, hebben aangetoond dat onze politieke arena nog
steeds langs de industriële scheidslijn tussen rechts en links
balanceert.\footnote{Adam Smith stierf in 1790, Karl Marx in 1883.} Wij
stellen echter dat het `gezonde verstand' van het industriële tijdperk
op vele terreinen niet meer van toepassing is in de radicaal
veranderende wereld.

Meer dan vijfentachtig jaar na de dag in 1911 waarop Oswald Spengler
werd getroffen door een voorgevoel van een naderende wereldoorlog en `de
ondergang van het Westen', zien wij eveneens `\emph{een historische
fasewisseling} plaatsvinden \ldots{} precies op het moment dat al
honderden jaren voorbestemd lijkt te zijn.'\footnote{Oswald Spengler,
  \emph{The Decline of the West}, vert. Charles Francis Atkinson,
  geciteerd in I. F. Clark, \emph{The Pattern of Expectation, 1644-2001}
  (Londen: Jonathan Cape, 1979), p.~220.} Net als Spengler voorzien wij
de naderende ondergang van de westerse beschaving en daarmee de
ineenstorting van de wereldorde die de afgelopen vijf eeuwen heeft
gedomineerd, sinds Columbus naar het westen zeilde om contact te leggen
met de Nieuwe Wereld. Maar in tegenstelling tot Spengler zien wij in het
komende millennium juist de geboorte van een nieuwe fase binnen de
westerse beschaving.

\bookmarksetup{startatroot}

\chapter{Megapolitieke transformaties in historisch
perspectief}\label{megapolitieke-transformaties-in-historisch-perspectief}

\begin{quote}
In de geschiedenis, net als in de natuur, zijn geboorte en dood
voortdurend in balans -- JOHAN HUIZINGA\footnote{Bois, op. cit., p.136.}
\end{quote}

\section{Het verval van de moderne
wereld}\label{het-verval-van-de-moderne-wereld}

Naar onze mening ben je getuige van niets minder dan het verval van de
Moderne Tijd. Het is een ontwikkeling gedreven door een meedogenloze
maar verborgen logica. Meer dan we doorgaans beseffen, en meer dan CNN
en de kranten ons vertellen, zal het volgende millennium niet langer
``modern'' zijn. We zeggen dit niet om te suggereren dat je een wilde of
primitieve toekomst te wachten staat, hoewel dat mogelijk is, maar om te
benadrukken dat het tijdperk in de geschiedenis dat nu aanbreekt
kwalitatief zal verschillen van hetgene waarin je geboren bent.

Er komt iets nieuws aan. Net zoals dat agrarische samenlevingen
verschilden van jagers-verzamelaarsbenden, en industriële samenlevingen
radicaal verschilden van feodale of yeomanboerensystemen, zo zal de
komende Nieuwe Wereld een radicale breuk betekenen met al het
voorgaande.

In het nieuwe millennium zullen het economische en politieke leven niet
langer georganiseerd worden op gigantische schaal onder de heerschappij
van de natiestaat, zoals dat tijdens de moderne eeuwen het geval was. De
beschaving die ons wereldoorlogen, de lopende band, sociale zekerheid,
inkomstenbelasting, deodorant en de broodrooster bracht, is stervende.
Deodorant en de broodrooster zullen misschien overleven. De andere niet.
Als een oude, ooit machtige man heeft de natiestaat een toekomst die in
jaren en dagen te tellen is, en niet langer in eeuwen en decennia.

Overheden hebben al veel van hun regulerende en dwingende macht
verloren. De ineenstorting van het communisme markeerde het einde van
een lange cyclus van vijf eeuwen, waarin de omvang van macht de
efficiëntie van de overheid overschaduwde. Het was een tijd waarin
geweld in toenemende mate loonde, maar die tijd is voorbij. De wereld is
al begonnen aan een historische omwenteling van formaat. Misschien zegt
een toekomstige Gibbon, die de ondergang van het Moderne Tijdperk
beschreef, wel dat dat tijdperk al voorbij was toen jij dit boek in
handen kreeg. Terugkijkend kan hij zeggen, net als wij, dat het eindigde
met de val van de Berlijnse Muur in 1989, of met het uiteenvallen van de
Sovjet-Unie in 1991. Beide data kunnen een bepalende gebeurtenis blijken
in de evolutie van de beschaving, als het einde van wat we nu kennen als
de Moderne Tijd.

Het vierde stadium van de menselijke ontwikkeling komt eraan, en
misschien is het minst voorspelbare kenmerk ervan de nieuwe naam
waaronder het bekend zal worden. Noem het ``Post-Modern.'' Noem het de
``Cybersamenleving'' of het ``Informatietijdperk.'' Of verzin zelf een
naam. Niemand weet welke bijnaam aan de volgende fase van de
geschiedenis zal blijven plakken.

We weten niet eens of de vijfhonderd jaar durende periode van de
geschiedenis die net eindigt, zal blijven worden beschouwd als
``modern.'' Als toekomstige historici kennis van etymologie hebben, zal
dat niet zo zijn. Een passendere titel zou kunnen zijn ``Het Tijdperk
van de Staat'' of ``Het Tijdperk van Geweld.'' Zo'n naam zou echter
buiten de tijdsgeest vallen die momenteel de tijdperken van de
geschiedenis benoemt. ``Modern,'' volgens het Oxford English Dictionary,
betekent ``behorend tot de tegenwoordige en recente tijden,
onderscheiden van het verre verleden\ldots{} In historisch gebruik
gewoonlijk toegepast (in tegenstelling tot antiek en middeleeuws) op de
tijd volgend op de MIDDELEEUWEN.''

Westerse mensen beschouwden zichzelf pas ``modern'' toen ze inzagen dat
de middeleeuwse periode voorbij was. Voor 1500 had niemand ooit aan de
feodale eeuwen gedacht als een ``midden''-periode in de Westerse
beschaving. De reden is duidelijk bij reflectie: voordat een tijdperk
redelijkerwijs kan worden gezien als ingeklemd in het ``midden'' van
twee andere historische tijdperken, moet het al tot een einde zijn
gekomen. Degenen die tijdens de feodale eeuwen leefden, hadden zich niet
kunnen voorstellen dat ze in een tussenhuis woonden tussen de oudheid en
de moderne beschaving, totdat het hen daagde niet alleen dat de
middeleeuwse periode voorbij was, maar ook dat de middeleeuwse
beschaving dramatisch verschilde van die van de Oudheid.

Menselijke culturen hebben blinde vlekken. We hebben geen vocabulaire om
paradigmaveranderingen in de grootste kaders van het leven te
beschrijven, vooral als die om ons heen gebeuren. Ondanks de vele
dramatische veranderingen die zich hebben ontvouwd sinds de tijd van
Mozes, hebben slechts enkele ketters de moeite genomen om na te denken
over hoe de overgangen van de ene fase van beschaving naar de andere
werkelijk plaatsvinden.

Hoe worden ze veroorzaakt? Wat hebben ze met elkaar gemeen? Aan welke
patronen kunnen we zien wanneer ze beginnen en hoe weten we wanneer ze
voorbij zijn? Wanneer zullen Groot-Brittannië of de Verenigde Staten ten
einde komen? Voor deze vragen zal het lastig zijn om conventionele
antwoorden te vinden.

\subsection{Het taboe op
vooruitziendheid}\label{het-taboe-op-vooruitziendheid}

``Buiten'' een bestaand systeem zien is als een toneelknecht die een
dialoog probeert te forceren met een personage in een toneelstuk. Het
schendt een conventie die helpt het systeem te laten functioneren. In
elke maatschappij geldt een stilzwijgend verbod: denk vooral niet na
over hoe het systeem ooit kan ophouden te bestaan, of wat ervoor in de
plaats zou kunnen komen. Impliciet is ieder systeem het laatste of het
enige systeem dat ooit zal bestaan, ook al wordt dit niet openlijk
uitgesproken. Van de mensen die ooit een geschiedenisboek hebben
gelezen, zal slechts een minderheid zo'n aanname realistisch vinden als
ze ermee geconfronteerd zouden worden. Desondanks is dat de conventie
die de wereld regeert. Ieder sociaal systeem doet alsof haar regels
nooit zullen worden vervangen, hoe sterk of zwak zij ook aan de macht
probeert vast te houden. Ze zijn het laatste woord. Of misschien het
enige woord. Primitievelingen geloven dat hun ideeën de enige mogelijke
manier beschrijven om het leven te organiseren. Economisch meer
gecompliceerde systemen die een gevoel van geschiedenis incorporeren
spannen hiervan gewoonlijk de kroon. Of het nu gaat om Chinese
mandarijnen aan het hof van de keizer, de marxistische nomenklatura in
Stalins Kremlin, of leden van het Amerikaanse Huis van Afgevaardigden:
de machthebbers verbeelden zich óf dat er geen geschiedenis bestaat, óf
dat zij het hoogtepunt ervan vormen --- verheven boven iedereen die hen
voorging, en als voorhoede van alles wat nog komen zal.

Om bijna onvermijdelijke redenen is dit de waarheid. Hoe duidelijker het
is dat een systeem zijn einde nadert, hoe minder bereid mensen zullen
zijn om zich aan haar wetten te houden. Elke sociale organisatie zal
daarom de neiging hebben om analyses die zijn ondergang voorzien te
ontmoedigen of te bagatelliseren. Alleen al door dit feit worden grote
keerpunten in de geschiedenis zelden opgemerkt terwijl ze gebeuren. Eén
ding kun je in elk geval zeker weten over de toekomst: conventionele
denkers zullen grote veranderingen niet verwelkomen of aankondigen.

Conventionele informatiebronnen zullen je geen objectieve en tijdige
waarschuwing geven over hoe de wereld verandert en waarom. Als je de
grote overgang die nu gaande is wilt begrijpen, heb je geen andere keuze
dan het zelf uit te zoeken.

\section{Voorbij het voor de hand
liggende}\label{voorbij-het-voor-de-hand-liggende}

Dit betekent verder kijken dan onze neus lang is. De geschiedenis toont
dat overgangen mogelijk decennia of zelfs eeuwen nadat ze gebeuren niet
worden erkend, zelfs wanneer die achteraf gezien onmiskenbaar reëel
bleken te zijn. Neem de val van Rome. Het was waarschijnlijk de
belangrijkste historische ontwikkeling in het eerste millennium van het
Christelijke Tijdperk. Zelfs lang nadat Rome gevallen was, hield men de
illusie van haar voortbestaan in stand, vergelijkbaar met het gebalsemde
lichaam van Lenin. Wie alleen naar het officiële nieuws luisterde, had
pas door dat Rome was gevallen toen dat allang geen verschil meer
maakte.

Dit kwam niet alleen door de beperkte communicatiemogelijkheden van de
antieke wereld. De uitkomst zou grotendeels hetzelfde zijn geweest als
CNN al zou hebben bestaan en haar videoband draaide in september 476.
Toen werd de laatste Romeinse keizer in het Westen, Romulus Augustulus,
gevangengenomen in Ravenna, en werd gedwongen om in een villa in
Campanië te pensioneren. Zelfs als Wolfe Blitzer daar zou zijn geweest
om met minicamera's het nieuws in het jaar 476 op te nemen, is het
onwaarschijnlijk dat hij of iemand anders zou hebben gedurfd die
gebeurtenissen te karakteriseren als het einde van het Romeinse Rijk.
Dat is natuurlijk precies wat latere historici uiteindelijk wel
concludeerden.

De redacteuren van CNN zouden waarschijnlijk geen goedkeuring hebben
gegeven voor een voorpagina-artikel met de titel ``Rome is gevallen.''
De machthebbers ontkenden dit. ``Nieuws''-kanalen omarmen zelden
controverse omdat dit hun inkomsten zou kunnen ondermijnen. Ze zijn vaak
partijdig, soms zelfs in grote mate. Maar ze rapporteren zelden
conclusies die abonnees zouden kunnen overtuigen om hun abonnementen op
te zeggen en ervan door te gaan. Daarom zou slechts een enkeling de val
van Rome hebben gerapporteerd, zelfs als het technologisch mogelijk zou
zijn geweest. Experts zouden meteen geroepen hebben dat het onzin was om
te beweren dat Rome gevallen was. Zoiets zeggen was niet goed voor de
handel --- en mogelijk zelfs gevaarlijk voor je gezondheid. De macht lag
toen bij barbaren, en die hielden vol dat Rome zoals men dat kende, nog
altijd bestond.

Het probleem was echter niet alleen dat de autoriteiten zeiden:
``Rapporteer dit niet of we zullen je doden.'' Wat nog meer meespeelde,
was dat Rome tegen het einde van de vijfde eeuw al zo ver was
afgetakeld, dat de ``val'' voor de meeste mensen die het meemaakten
nauwelijks als zodanig werd opgemerkt. Pas een generatie later
suggereerde Graaf Marcellinus in feite voor het eerst dat ``Het
West-Romeinse Rijk ten onder ging met Augustulus.'' Het duurde nog
decennia, misschien zelfs eeuwen, voordat het algemeen erkend werd dat
het Romeinse Rijk in het Westen niet langer bestond. Karel de Grote
geloofde in het jaar 800 zonder twijfel dat hij een legitieme Romeinse
keizer was.

Het punt is niet dat Karel de Grote en allen die in conventionele termen
dachten over het Romeinse Rijk na 476 idioten waren, integendeel.
Maatschappelijke veranderingen worden vaak vaag of dubbelzinnig
voorgesteld. Als invloedrijke instellingen die vaagheid gebruiken om een
voor hen gunstige conclusie te versterken, zelfs als die grotendeels op
schijn berust, zal alleen iemand met een sterk karakter en uitgesproken
overtuigingen het aandurven die tegen te spreken. Wie zich inleeft in
het leven van een Romein in de vijfde eeuw begrijpt hoe makkelijk het
was om te geloven dat alles nog steeds hetzelfde was gebleven. Die
zekerheid was de optimistische conclusie. Iets anders denken had
beangstigend kunnen zijn, en waarom tot een beangstigende conclusie
komen, als er ook een geruststellende voorhanden was?

Er waren uiteindelijk ook goede redenen om te denken dat het leven
gewoon door zou gaan zoals altijd. Dat deed het in het verleden ook. De
Romeinse legers, vooral aan de grenzen van het rijk, werden al
eeuwenlang geïnfiltreerd door barbaren. Tegen de derde eeuw was het voor
het leger bijna een gewoonte geworden om regelmatig een nieuwe keizer
uit te roepen. Tegen de vierde eeuw waren zelfs officieren
gegermaniseerd en vaak analfabeet. Voor de val van Romulus Augustulus
werden al talrijke keizers op brute wijze van de troon gestoten. Zijn
vertrek zou voor zijn tijdgenoten niet anders hebben geleken dan de vele
andere omwentelingen in deze chaotische periode. Hij werd zelfs
weggestuurd met een pensioen, al was het slechts een korte periode
voordat hij werd vermoord. Dit bewees dat het systeem nog overeind
stond. De optimist vond juist dat Odoacer, die Romulus Augustulus
afzette, het rijk herenigde, en niet vernietigde. Odoacer, een zoon van
Attila's rechterhand Edecon, was een intelligente man. Hij riep zichzelf
niet uit tot keizer. In plaats daarvan riep hij de Senaat bijeen en
haalde zijn al te beïnvloedbare leden over dat zij het keizerschap en
dus de soevereiniteit over het hele rijk aan Zeno, de keizer in het
verre Byzantium, zouden aanbieden. Odoacer was slechts Zeno's patricius,
aangesteld om Italië te leiden.

Zoals Will Durant schreef in \emph{The Story of Civilization}, leken
deze veranderingen niet op de ``val van Rome'' maar slechts
``verwaarloosbare verschuivingen op het nationale toneel.'' Toen Rome
viel, zei Odoacer dat Rome bleef bestaan. Net als bijna iedereen deed
hij maar al te graag alsof alles hetzelfde bleef. Ze wisten dat ``de
oude glorie van Rome'' veel beter was dan de barbarij die haar verving.
Zelfs de barbaren dachten dat. Zoals C. W. Previte-Orton schreef in
\emph{The Shorter Cambridge Medieval History}, was het eind van de
vijfde eeuw, toen ``de keizers waren vervangen door barbaarse Germaanse
koningen,'' een ``aanhoudende schijnvertoning.''

\subsection{``Hardnekkige
schijnvertoning''}\label{hardnekkige-schijnvertoning}

Deze ``schijnvertoning'' hield de façade van het oude systeem overeind,
ook al was de kern ervan al ``aangetast door barbarij.'' De oude
regeringsvormen bleven hetzelfde toen de laatste keizer werd vervangen
door een barbaarse ``luitenant''. De Senaat kwam nog steeds bijeen. ``De
pretoriaanse prefectuur en andere hoge ambten gingen door, en werden
gehouden door vooraanstaande Romeinen.'' Consuls werden nog steeds voor
een jaar aangesteld. ``De Romeinse burgerlijke administratie bleef
onaangetast.'' Sterker nog, op sommige vlakken bleef het intact tot de
opkomst van het feodalisme aan het einde van de tiende eeuw. Bij
openbare gelegenheden werd nog steeds gebruik gemaakt van de oude
keizerlijke insignes. Het christendom was nog steeds de
staatsgodsdienst. De barbaren deden nog steeds alsof ze trouw
verschuldigd waren aan de Oostelijke keizer in Constantinopel, en aan de
tradities van het Romeinse recht. Maar, in Durant's woorden, ``in het
Westen was het grote Rijk er niet meer.''

\subsection{Nou en?}\label{nou-en}

De val van Rome lijkt misschien iets van lang geleden, maar is
verrassend relevant als je kijkt naar hoe de wereld er nu voorstaat. De
meeste boeken over de toekomst zijn eigenlijk boeken over het heden. We
hebben geprobeerd dat gebrek te verhelpen door van dit boek over de
toekomst allereerst een boek over het verleden te maken. Wij denken dat
je een beter perspectief zult krijgen op wat de toekomst brengt als we
belangrijke megapolitieke punten over de logica van geweld illustreren
met echte voorbeelden uit het verleden. De geschiedenis is een geweldige
leermeester. De verhalen die het te vertellen heeft zijn interessanter
dan wat wij kunnen verzinnen, en veel van de meest interessante verhalen
gaan over de val van Rome. Ze bevatten belangrijke lessen die relevant
kunnen zijn voor jouw toekomst in het Informatietijdperk.

Een van de best beschreven voorbeelden van een grote overgang in de
geschiedenis, waarbij de overheid implodeerde, is de val van Rome. De
overgang rond het jaar 1000 ging ook gepaard met de ineenstorting van
centrale autoriteit, en leidde tot een toename in de complexiteit en
omvang van economische activiteiten. De Buskruitrevolutie aan het eind
van de vijftiende eeuw bracht grote institutionele veranderingen met
zich mee, die juist de schaal van bestuur vergrootten in plaats van
verkleinden. Vandaag de dag, voor het eerst in duizend jaar, ondermijnen
en vernietigen de megapolitieke omstandigheden in het Westen overheden
en vele andere grootschalige instituties.

Uiteraard verschilden de oorzaken van de bestuurlijke implosie aan het
einde van het Romeinse Rijk aanzienlijk van de factoren die een rol
spelen bij de opkomst van het Informatietijdperk. Een van de redenen is
dat het Romeinse imperium simpelweg te grootschalig was. Het werd
onmogelijk om de economie van geweld te handhaven. De kosten om de ver
uitgespreide grenzen van het rijk te verdedigen, overtroffen de
economische voordelen die een oude agrarische economie kon opbrengen. De
last van belasting en regulering die nodig waren om de militaire
inspanning te financieren, steeg tot boven de draagkracht van de
economie. Corruptie werd endemisch. Militaire commandanten, zoals
historicus Ramsay MacMullen heeft gedocumenteerd, spendeerden een groot
deel van hun tijd aan het misbruiken van hun positie, voor ``illegale
winsten''. Dit deden ze door de bevolking af te persen, wat de
vierde-eeuwse waarnemer Synesius beschreef als ``de vredesoorlog, bijna
erger dan de barbarenoorlog, en voortkomend uit het gebrek aan
discipline van het leger en de hebzucht van de officieren.''

Een andere belangrijke factor die bijdroeg aan de val van Rome was een
demografisch tekort veroorzaakt door de Antonijnse pest. De sterke krimp
van de Romeinse bevolking droeg op veel vlakken duidelijk bij aan
economische en militaire zwakte. Vandaag de dag is daar nog geen sprake
van, althans nog niet. Op langere termijn, misschien, zal de gesel van
nieuwe ``plagen'' de uitdagingen van technologische devolutie in het
nieuwe millennium verergeren. De ongekende toename van de menselijke
bevolking in de twintigste eeuw creëert een verleidelijk doelwit voor
snel muterende microparasieten. Angsten over het Ebola-virus, of iets
dergelijks, dat metropolitane bevolkingen binnenvalt, kunnen gegrond
zijn. Maar dit is niet de plaats om de co-evolutie van mensen en ziekten
te overwegen. Hoe interessant dat onderwerp ook is, ons argument op dit
punt gaat niet over waarom Rome viel, of over de vraag of de wereld op
dit moment kwetsbaar is voor sommige van dezelfde invloeden die
bijdroegen aan de Romeinse achteruitgang. Het gaat over iets anders -
namelijk de manier waarop de grote transformaties van de geschiedenis
worden waargenomen, of liever, verkeerd waargenomen terwijl ze gebeuren.

Mensen zijn altijd en overal tot op zekere hoogte conservatief, met een
kleine ``c''. Men is terughoudend wanneer het gaat over het loslaten van
traditionele sociale conventies, het ondermijnen van erkende instituties
en het ter discussie stellen van de wetten en waarden waarop zij
gefundeerd waren. Maar weinig mensen kunnen zich voorstellen dat
schijnbaar kleine veranderingen in klimaat, technologie of een andere
variabele op de een of andere manier verantwoordelijk zouden kunnen zijn
voor het doorbreken van verbindingen met de wereld waarin hun ouders
zijn opgegroeid. De Romeinen waren terughoudend om de veranderingen die
zich om hen heen ontvouwden te erkennen. En dat zijn wij ook.

Toch, of je het nu erkent of niet, we gaan door een verandering van
historisch seizoen, een transformatie in de manier waarop mensen hun
levensonderhoud organiseren en zichzelf verdedigen, die zo diepgaand is
dat het onvermijdelijk de hele samenleving zal transformeren. Het zal
zelfs zo diepgaand zijn dat het, om het goed te begrijpen, noodzakelijk
is om bijna niets als vanzelfsprekend te beschouwen. Steeds opnieuw zul
je de neiging hebben om te geloven dat de komende Informatie
Samenlevingen grote gelijkenissen zullen hebben met de industriële
samenlevingen waar je in opgroeide. Wij betwijfelen dat. Microprocessing
zal de mortel in de bakstenen oplossen. Het zal de logica van geweld zo
diepgaand veranderen dat het dramatisch de manier zal veranderen waarop
mensen hun levensonderhoud organiseren en zichzelf verdedigen. Toch zal
men geneigd zijn om de onvermijdelijkheid van deze veranderingen te
bagatelliseren, of om te discussiëren over hun wenselijkheid alsof
industriële instellingen per decreet zouden kunnen bepalen hoe de
geschiedenis evolueert.

\subsection{De grote illusie}\label{de-grote-illusie}

Auteurs die op vele manieren beter geïnformeerd zijn dan wij, zullen je
desondanks op het verkeerde been zetten als het gaat over de toekomst,
omdat zij de werking van samenlevingen slechts oppervlakkig analyseren.
David Kline en Daniel Burstein hebben bijvoorbeeld een goed onderzocht
boek geschreven, getiteld \emph{Road Warriors: Dreams and Nightmares
Along the Information Highway}. Het zit vol bewonderenswaardige details,
maar veel van deze details worden aangedragen om een illusie te
beargumenteren, namelijk het idee `dat burgers gezamenlijk en bewust
kunnen ingrijpen om de spontane economische en natuurlijke processen om
hen heen vorm te geven.' Hoewel het misschien niet voor de hand liggend
is, is dit met de stelling dat het feodalisme zou hebben overleefd als
iedereen zich opnieuw had toegewijd aan de ridderlijkheid. Niemand in
een hof van de late vijftiende eeuw zou bezwaar hebben gemaakt tegen
zo'n sentiment. Sterker nog, het zou ketterij zijn geweest om dat te
doen. Maar het zou ook volledig misleidend zijn geweest, een voorbeeld
van de slang die de toekomst in zijn oude huid probeert te wringen.

De fundamentele oorzaken van verandering zijn juist niet onderworpen aan
bewuste controle. Het zijn de factoren die de omstandigheden veranderen
waaronder geweld loont. Ze staan zelfs zo ver af van elke vorm van
bewuste manipulatie, dat ze in een wereld die doordrenkt is van
politiek, geen onderwerp van politiek gemanoeuvreer vormen. Niemand
heeft ooit in een demonstratie geroepen: ``Verhoog schaalvoordelen in
het productieproces.'' Geen spandoek heeft ooit geëist: ``Vind een
wapensysteem uit dat het belang van de infanterie verhoogt.'' Geen
kandidaat heeft ooit beloofd om ``de balans tussen efficiëntie en
schaalgrootte van de bescherming tegen geweld te veranderen.'' Zulke
slogans zouden belachelijk zijn, juist omdat deze doelen buiten ieders
vermogen liggen om ze bewust te beïnvloeden. Toch, zoals we zullen zien,
bepalen deze variabelen in veel grotere mate hoe de wereld werkt dan
welk politiek platform dan ook.

Als je er goed over nadenkt, wordt al snel duidelijk dat belangrijke
historische keerpunten zelden primair worden gedreven door menselijke
verlangens. Ze gebeuren niet omdat mensen het zat worden van één manier
van leven en plotseling een andere prefereren. Als je er even bij
stilstaat, zul je begrijpen waarom. Als wat mensen denken en verlangen
als enige zouden bepalen wat er gebeurt, dan zouden alle abrupte
veranderingen in de geschiedenis worden verklaard door wilde
stemmingswisselingen, los van enige verandering in de feitelijke
levensomstandigheden. In feite gebeurt dit nooit. Alleen in gevallen van
medische problemen, die een paar mensen treffen, zien we willekeurige
fluctuaties in stemming die volledig los lijken te staan van een
objectieve oorzaak.

Over het algemeen besluiten grote aantallen mensen niet plotseling en
allemaal tegelijk hun manier van leven op te geven, simpelweg omdat ze
dat grappig vinden. Geen enkele jager-verzamelaar heeft ooit gezegd:
``Ik ben het zat om in prehistorische tijden te leven; ik leef liever
als een boer in een boerendorp.'' Elke beslissende omslag in
gedragspatronen en waarden is zonder uitzondering een reactie op een
werkelijke verandering in de omstandigheden van het leven. In die zin,
althans, zijn mensen altijd realistisch. Als hun opvattingen abrupt
veranderen, wijst dat waarschijnlijk erop dat ze zijn geconfronteerd met
een afwijking van de vertrouwde omstandigheden: een invasie, een plaag,
een plotselinge klimatologische verschuiving, of een technologische
revolutie die hun levensonderhoud of hun vermogen om zichzelf te
verdedigen verandert.

Grote veranderingen in de geschiedenis zijn meestal niet wat mensen
willen. Ze verstoren juist de rust en stabiliteit waar de meeste mensen
naar verlangen. Wanneer verandering optreedt, veroorzaakt het typisch
wijdverspreide desoriëntatie, vooral onder degenen die hun inkomen of
sociale status verliezen. Je zult tevergeefs kijken naar opiniepeilingen
of stemmingsbarometers om een begrip te krijgen van hoe de komende
megapolitieke overgang zich waarschijnlijk zal ontvouwen.

\section{Leven zonder vooruitziende
blik}\label{leven-zonder-vooruitziende-blik}

Als we er niet in slagen om de grote overgang, die om ons heen gaande
is, waar te nemen, komt dat deels doordat we die niet willen zien. Onze
jagende voorouders waren misschien net zo koppig, maar zij hadden een
beter excuus. Niemand had tienduizend jaar geleden de gevolgen van de
Agrarische Revolutie kunnen voorzien. Sterker nog, men voorzag toen
überhaupt niet veel meer dan waar ze de volgende maaltijd zouden kunnen
vinden. Toen men aan landbouw begon, bestond er geen verslaglegging van
gebeurtenissen uit het verleden, om daaruit lessen te trekken voor de
toekomst. De Westerse tijdsindeling, zoals seconden, minuten, uren,
dagen, enzovoort, om de jaren af te meten, bestond nog niet eens.
Jager-verzamelaars leefden in het ``eeuwige heden,'' zonder kalenders,
en inderdaad, zonder enige geschreven verslagen. Ze kenden geen
wetenschap, en beschikten over geen enkel ander intellectueel instrument
om oorzaak en gevolg te begrijpen dan hun eigen intuïtie. Wat betreft
vooruitkijken, waren onze primitieve voorouders blind. Om de bijbelse
metafoor te citeren, ze hadden nog niet van de vrucht der kennis
gegeten.

\subsection{Leren van het verleden}\label{leren-van-het-verleden}

Gelukkig staan wij er beter voor. De afgelopen vijfhonderd generaties
hebben ons een analytisch vermogen geschonken dat onze voorouders
misten. Wetenschap en wiskunde hebben geholpen vele geheimen van de
natuur te openbaren, waardoor we een begrip van oorzaak en gevolg hebben
dat grenst aan het magische vergeleken met dat van de vroege
jager-verzamelaars. Computationele algoritmen, ontstaan dankzij de komst
van snelle computers, hebben nieuwe inzichten opgeleverd in de werking
van complexe, dynamische systemen zoals de menselijke economie. De
zorgvuldige ontwikkeling van politieke economie zelf, hoewel het verre
van perfect is, heeft ons begrip van de factoren die menselijk handelen
beïnvloeden, vergroot. Een belangrijk inzicht daaruit is dat mensen
altijd en overal geneigd zijn op prikkels te reageren, niet altijd zo
mechanisch als economen zich voorstellen, maar ze reageren wel. Kosten
en baten doen ertoe. Veranderingen in externe omstandigheden die de
baten verhogen van bepaald gedrag of de kosten ervan verlagen, zullen
leiden tot meer van dat gedrag, als de rest onveranderd blijft.

\subsection{Prikkels doen ertoe}\label{prikkels-doen-ertoe}

Het feit dat mensen de neiging hebben om te reageren op kosten en baten
is een essentieel element bij het doen van voorspellingen. Je kunt met
grote zekerheid zeggen dat als je een biljet van honderd dollar op
straat laat vallen, iemand het snel zal oprapen, of je nu in New York,
Mexico-Stad of Moskou bent. Dit is niet zo triviaal als het lijkt. Het
toont aan waarom de slimme mensen die zeggen dat voorspellen onmogelijk
is, ongelijk hebben. Elke voorspelling die accuraat de impact van
prikkels op gedrag anticipeert zal het waarschijnlijk grotendeels bij
het juiste eind hebben. Hoe groter de geanticipeerde verandering in
kosten en baten, hoe minder voor de hand liggend de voorspelling
waarschijnlijk zal zijn.

De meest verstrekkende voorspellingen van allemaal zullen waarschijnlijk
voortkomen uit het herkennen van de implicaties van verschuivende
megapolitieke variabelen. Geweld is de ultieme grens die gedrag bepaalt;
dus, als je kunt begrijpen hoe de logica van geweld zal veranderen, kun
je met grote nauwkeurigheid voorspellen waar mensen het equivalent van
honderd-dollarbiljetten in de toekomst zullen laten vallen of oprapen.

We bedoelen hiermee niet dat je het onkenbare kunt weten. Wij kunnen je
niet vertellen hoe je winnende loterijnummers kunt voorspellen, of welke
willekeurige gebeurtenis dan ook. We hebben geen manier om te weten of
wanneer een terrorist een atoomexplosie in Manhattan zal doen
ontploffen, of als een asteroïde Saoedi-Arabië zal treffen. We kunnen de
komst van een nieuwe ijstijd, een plotselinge vulkaanuitbarsting, of de
opkomst van een nieuwe ziekte niet voorspellen. Het aantal onkenbare
gebeurtenissen die de koers van de geschiedenis zouden kunnen veranderen
is groot. Maar het onkenbare weten is heel anders dan de gevolgen
overzien van wat al bekend is. Als u een bliksemflits ver weg ziet, kunt
u met een grote zekerheid voorspellen dat een donderslag zal volgen. Het
voorspellen van de gevolgen van megapolitieke transities omvat veel
langere tijdsbestekken, en minder zekere verbanden, maar het idee is
hetzelfde.

Megapolitieke katalysatoren voor verandering verschijnen gewoonlijk ruim
voordat hun gevolgen zich manifesteren. Het duurde vijfduizend jaar
voordat de volledige implicaties van de Agrarische Revolutie aan het
licht kwamen. De overgang van een agrarische samenleving naar een
industriële samenleving gebaseerd op fabricage en chemische kracht
ontvouwde zich sneller. Het duurde eeuwen. De overgang naar de
Informatiemaatschappij zal nog sneller gebeuren, waarschijnlijk binnen
een leven. Toch, zelfs rekening houdend met de versnelling van de
geschiedenis, kunt je verwachten dat decennia zullen verstrijken voordat
de volledige megapolitieke impact van de huidige informatietechnologie
wordt gerealiseerd.

\subsection{Grote en kleine megapolitieke
overgangen}\label{grote-en-kleine-megapolitieke-overgangen}

Dit hoofdstuk analyseert enkele gemeenschappelijke kenmerken van
megapolitieke overgangen. In volgende hoofdstukken kijken we meer
nauwkeurig naar de Agrarische Revolutie, en de overgang van boerderij
naar fabriek, de tweede van de grote faseveranderingen. Binnen het
agrarische stadium van de beschaving waren er vele kleine megapolitieke
overgangen zoals de val van Rome en de Feodale Revolutie van het jaar
1000. Deze markeerden de verschuivingen in de machtsbalans: overheden
kwamen en gingen, en de opbrengsten van de landbouw gingen van de ene
groep over naar de andere. De eigenaren van uitgestrekte landgoederen
ten tijden van het Romeinse Rijk, vrije boeren in de Europese vroege
Middeleeuwen, en de heren en lijfeigenen van de feodale periode aten
allemaal graan van dezelfde akkers. Ze leefden onder zeer verschillende
overheden dankzij de cumulatieve impact van verschillende technologieën,
fluctuaties in het klimaat, en de ontwrichtende invloed van ziektes.

We beweren niet al deze veranderingen grondig te verklaren, hoewel we
een beeld hebben geschetst van de manier waarop veranderende
megapolitieke variabelen de machtsuitoefening in het verleden hebben
beïnvloed. Overheden zijn gegroeid en gekrompen naarmate megapolitieke
schommelingen de kosten van machtsprojectie verlaagden en verhoogden.

Hier zijn enkele samenvattende punten die je in gedachten zou moeten
houden wanneer je probeert de Informatierevolutie te begrijpen:

\begin{enumerate}
\def\labelenumi{\arabic{enumi}.}
\tightlist
\item
  Een verschuiving in de megapolitieke machtsfundamenten voltrekt zich
  doorgaans ver ver vóór de daadwerkelijke revoluties in de
  machtsuitoefening.
\item
  Inkomens dalen meestal wanneer een grote overgang begint, vaak omdat
  een samenleving door bevolkingsdruk kwetsbaar is geworden en
  hulpbronnen heeft gemarginaliseerd.
\item
  ``Buiten'' een systeem zien is gewoonlijk taboe. Mensen zijn vaak
  blind voor de logica van geweld in de bestaande samenleving en
  daardoor vrijwel altijd ook voor veranderingen in die logica, of die
  nu verborgen of openlijk zijn. Megapolitieke transities worden zelden
  herkend voordat ze zich voordoen.
\item
  Grote overgangen gaan altijd gepaard met een culturele revolutie, en
  leiden gewoonlijk tot botsingen tussen aanhangers van de oude en de
  nieuwe waarden.
\item
  Megapolitieke overgangen zijn nooit populair, omdat ze moeizaam
  verworven intellectueel kapitaal achterhaald maken en gevestigde
  morele voorschriften ondermijnen. Ze worden niet op algemeen verzoek
  uitgevoerd, maar als reactie op veranderende externe omstandigheden
  die de logica van geweld in de lokale context veranderen.
\item
  Transities naar nieuwe manieren om het levensonderhoud te organiseren
  of naar nieuwe typen overheden, beperken zich aanvankelijk tot de
  gebieden waar de megapolitieke katalysatoren werkzaam zijn.
\item
  Met de mogelijke uitzondering van de vroege stadia van landbouw,
  hebben eerdere overgangen altijd perioden van sociale chaos en
  verhoogd geweld betrokken vanwege desoriëntatie en de ineenstorting
  van het oude systeem.
\item
  Corruptie, moreel verval, en inefficiëntie lijken signaalkenmerken te
  zijn van de laatste stadia van een systeem.
\item
  Het groeiende belang van technologie bij het vormgeven van de logica
  van geweld heeft geleid tot een versnelling van de geschiedenis,
  waardoor er voor elke opeenvolgende transitie minder tijd voor
  aanpassing overblijft dan ooit tevoren.
\end{enumerate}

\subsection{De geschiedenis versnelt}\label{de-geschiedenis-versnelt}

Nu gebeurtenissen zich vele malen sneller ontvouwen dan tijdens
voorgaande transformaties, zou een vroegtijdig inzicht in hoe de wereld
zal veranderen veel meer voor je kunnen betekenen dan voor je voorouders
op een vergelijkbaar keerpunt in het verleden. Zelfs als de eerste
boeren op miraculeuze wijze de volledige megapolitieke implicaties van
het bewerken van de aarde hadden begrepen, zou deze informatie praktisch
nutteloos zijn geweest omdat duizenden jaren zouden verstrijken voordat
de overgang naar de nieuwe fase van de samenleving voltooid was.

Vandaag de dag is dat anders. De geschiedenis is in een
stroomversnelling geraakt. Voorspellingen die de megapolitieke gevolgen
van nieuwe technologie correct inschatten, zijn vandaag de dag
waarschijnlijk veel nuttiger. Het kunnen doorgronden van de implicaties
van de huidige overgang naar de Informatiemaatschappij is vele malen
waardevoller dan dat het volledige begrip van de gevolgen van de
transitie van landbouw naar industrie toen was. Simpel gezegd, de
actie-horizon voor megapolitieke voorspellingen is ingekort tot de
nuttigste tijdsduur: de duur van een mensenleven.

\begin{quote}
Terugkijkend over de eeuwen, of zelfs alleen kijkend naar het heden,
kunnen we duidelijk waarnemen dat velen hun inkomen hebben gekregen,
vaak een zeer goed inkomen, door middel van hun speciale vaardigheid in
het toepassen van wapens van geweld, en dat hun handelingen in grote
mate bepaalden hoe schaarse middelen werden ingezet. -- FREDERIC C.
LANE\footnote{Lane, \emph{Economic Consequences of Organized Violence},
  op. cit.}
\end{quote}

Onze studie van de megapolitiek is een poging om precies dat te doen:
het in kaart brengen van de gevolgen van de factoren die de grenzen
verschuiven waarbinnen geweld wordt uitgeoefend.

Deze megapolitieke factoren bepalen grotendeels wanneer en waar geweld
loont. Ze beïnvloeden ook de verdeling van het inkomen. Zoals economisch
historicus Frederic Lane zo duidelijk verwoordde, speelt de manier
waarop geweld wordt georganiseerd en beheerst een grote rol bij het
bepalen van ``\emph{hoe schaarse middelen worden ingezet.}''

\section{Een spoedcursus
megapolitiek}\label{een-spoedcursus-megapolitiek}

Het concept megapolitiek is een krachtig idee. Het helpt enkele van de
grote mysteries van de geschiedenis te verklaren, zoals hoe regeringen
opkomen en vallen, wat voor soorten instellingen ze worden, de timing en
uitkomst van oorlogen, en patronen van economische welvaart en
achteruitgang. Door de kosten en baten van het projecteren van macht te
verhogen of te verlagen, regeert megapolitiek het vermogen van mensen om
hun wil aan anderen op te leggen. Sinds de vroegste menselijke
samenlevingen is dit het geval geweest, en dat is het nog steeds. In
\emph{Blood in the Streets} en \emph{The Great Reckoning} onderzochten
we vele van de belangrijke verborgen megapolitieke factoren die de
evolutie van de geschiedenis bepalen. Om de gevolgen van megapolitieke
verandering te begrijpen, moet je inzicht krijgen in de factoren die
revoluties in het gebruik van geweld realiseren. Deze variabelen kunnen
enigszins willekeurig worden gegroepeerd in vier categorieën:
topografie, klimaat, microben, en technologie.

1. \textbf{Topografie} is een cruciale factor, zoals bewezen door het
feit dat, in tegenstelling tot op land, de controle van geweld op open
zee nooit is gemonopoliseerd. Geen enkele overheid is er ooit in
geslaagd om er enkel haar wetten te laten gelden. Om te begrijpen hoe de
organisatie van geweld en bescherming zal evolueren naarmate de economie
migreert naar cyberspace, is dit van belang.

Topografie, in combinatie met klimaat, speelde een grote rol in de
vroege geschiedenis. De eerste staten kwamen op in overstromingsvlaktes,
omringd door woestijn, zoals in Mesopotamië en Egypte, waar water voor
irrigatie overvloedig was maar omliggende regio's te droog waren om
kleinschalige landbouw mogelijk te maken. In deze context betaalden
individuele landbouwers een hoge prijs als zij weigerden samen te werken
aan het behoud van het politieke systeem. Zonder irrigatie, die alleen
op grote schaal kon worden verschaft, zouden gewassen niet groeien. Geen
gewassen betekende verhongering. Wie het water in de woestijn beheerste,
bezat de macht --- en dat leidde tot een despotische en steenrijke
overheid.

Zoals we analyseerden in The Great Reckoning, speelden topografische
omstandigheden ook een grote rol in de welvaart van vrije boeren in het
oude Griekenland, waardoor het zich kon ontwikkelen tot de wieg van de
Westerse democratie. Gegeven de primitieve transportmogelijkheden die
drieduizend jaar geleden in de Middellandse Zee-regio gebruikelijk
waren, was het vrijwel onmogelijk voor personen, die meer dan een paar
kilometer van de zee woonden, om te concurreren in de productie van
waardevolle gewassen: olijven en druiven. Wanneer olie en wijn over land
moesten vervoerd worden, waren de transportkosten zo hoog dat verkoop
met winst onmogelijk was. Door de grillige kustlijn van Griekenland lag
het grootste deel van het land op niet meer dan twintig mijl van de zee.
Dit gaf Griekse boeren een beslissend voordeel ten opzichte van hun
potentiële concurrenten in landinwaarts gelegen gebieden.

Door dit handelsvoordeel verdienden Griekse boeren hoge inkomens met de
controle van slechts kleine stukken grond. Deze hoge inkomens stelden
hen in staat om kostbare wapenuitrusting te kopen. De beroemde hoplieten
van het oude Griekenland waren boeren of landheren die zichzelf op eigen
kosten bewapenden. De Griekse hoplieten waren zowel goed bewapend als
sterk gemotiveerd, en vormden een militaire macht die men niet kon
negeren. De Griekse democratie ontstond uit de topografische
omstandigheden, net zoals andere soort topografieën de despotische
systemen van Egypte en elders mogelijk maakten.

2. \textbf{Klimaat} bepaalt ook mede de grenzen waarbinnen brute kracht
kan worden uitgeoefend. Een verandering van klimaat was de katalysator
voor de eerste grote overgang van foerageren naar landbouw.

Het einde van de laatste ijstijd, ongeveer dertienduizend jaar geleden,
leidde tot een radicale verandering van de vegetatie. In het Nabije
Oosten, waar de ijstijd als eerste tot een einde kwam, begonnen bossen
zich door een geleidelijke stijging in temperatuur en regenval te
verspreiden naar gebieden die voorheen bestonden uit grasland. De snelle
verspreiding van beukenbossen in het bijzonder beperkte het menselijke
dieet ernstig. Zoals Susan Alling Gregg het verwoordde in \emph{Foragers
and Farmers}:

``De ontwikkeling van beukenbossen moet ernstige gevolgen hebben gehad
voor lokale mensen-, planten- en dierenpopulaties. De kroonlaag van een
eikenbos is relatief open en laat grote hoeveelheden zonlicht de
bosbodem bereiken. Daardoor ontstaat een wilde onderbegroeiing van
gemengde struiken, kruiden en grassen, en deze diversiteit aan planten
ondersteunt op zijn beurt een verscheidenheid aan wild. Daarentegen is
de kroonlaag van een beukenbos gesloten en is de bosbodem zwaar
beschaduwd. Buiten een korte bloeiperiode van lentebloemen voorafgaand
aan het uitlopen van de bladeren, groeien er alleen soorten die goed
tegen schaduw kunnen, zoals zeggen, varens en enkele grassoorten.''

Na verloop van tijd drongen dichte bossen de open vlaktes binnen en
verspreidden zich door Europa richting de oostelijke steppen. De bossen
verdrongen de graslanden die grote dieren onderhielden, waardoor het
voor de menselijke jagers steeds moeilijker werd om zichzelf te
onderhouden.

De populatie van jager-verzamelaars was te sterk gegroeid tijdens de
welvarende periode gedurende de ijstijd, waardoor ze zichzelf niet meer
kon ondersteunen met de krimpende kuddes grote zoogdieren, waarvan vele
soorten tot uitsterven werden gejaagd. De overgang naar landbouw was
niet hun voorkeur, maar eerder een gedwongen improvisatie om tekorten in
het dieet aan te vullen. Jagen bleef dominant in de meer noordelijke
gebieden, waar de verwarmingstrend de habitats van grote zoogdieren niet
nadelig had beïnvloed, en in tropische regenwouden, waar de wereldwijde
opwarming niet leidde tot het verminderen van de voedselvoorzieningen.
Sinds de opkomst van landbouw zijn veranderingen veel vaker veroorzaakt
door afkoeling dan door opwarming van het klimaat.

Als het klimaat zal blijven fluctueren in de toekomst, zal het nuttig
zijn om een bescheiden begrip te hebben van de rol die
klimaatverandering speelde in vroegere maatschappijen. Het is
bijvoorbeeld zo dat een daling van één graad Celsius gemiddeld het
groeiseizoen met drie tot vier weken vermindert en honderdvijftig meter
afhaalt van de maximale hoogte waarop gewassen kunnen worden verbouwd.
Dit zegt iets over de grenzen waarbinnen mensen in de toekomst zullen
moeten handelen. Je kunt deze kennis gebruiken om veranderingen in
alles, van graanprijzen tot de prijzen van landbouwgrond, te
voorspellen. Je zou zelfs de waarschijnlijke impact van vallende
temperaturen op reële inkomens en politieke stabiliteit kunnen
inschatten. In het verleden zijn overheden omvergeworpen toen, door
meerdere opeenvolgende mislukte oogsten, de voedselprijzen verhoogden en
de koopkracht opdroogde.

Het is bijvoorbeeld geen toeval dat de zeventiende eeuw, de koudste in
de moderne periode, ook een periode van revolutie wereldwijd was. Een
verborgen megapolitieke oorzaak van dit ongenoegen was scherp kouder
weer. Het was zelfs zo koud dat wijn op de tafel van de ``Zonnekoning''
in Versailles bevroor. Verkorte groeiseizoenen veroorzaakten mislukte
oogsten en ondermijnden reële inkomens. Door het koudere weer begon de
welvaart af te nemen en gleed de wereld rond 1620 een lange wereldwijde
depressie in. Die bleek buitengewoon ontwrichtend. De economische crisis
van de zeventiende eeuw leidde tot een golf van opstanden wereldwijd,
met een piek in het jaar 1648 --- precies tweehonderd jaar vóór een
andere, bekendere cyclus van revoluties. Tussen 1640 en 1650 waren er
opstanden in Ierland, Schotland, Engeland, Portugal, Catalonië,
Frankrijk, Moskou, Napels, Sicilië, Brazilië, Bohemen, Oekraïne,
Oostenrijk, Polen, Zweden, Nederland, en Turkije. Zelfs China en Japan
werden overspoeld met onrust.

Het is wellicht ook geen toeval dat het mercantilisme overheerste in de
zeventiende eeuw, tijdens een periode van krimpende handel. Economisch
isolationisme was misschien het meest uitgesproken aan het einde van de
eeuw, ``toen een verschrikkelijke hongersnood plaatsvond.'' In de
achttiende eeuw, vooral na 1750, hadden warmere temperaturen en betere
oogsten de inkomens in West-Europa voldoende verhoogd, waardoor de vraag
naar productiegoederen op gang kwam. Naarmate het vrijemarktbeleid
breder werd ingevoerd, ontstond een zichzelf versterkende impuls van
economische groei. De industrie breidde zich uit tot een ongekende
schaal --- het begin van wat we de Industriële Revolutie zijn gaan
noemen. Technologie en productie werden steeds belangrijker en
verminderden de invloed van het weer op de economische cycli.

Je moet de impact van een plotse verlaging van de temperatuur echter
niet onderschatten. Zelfs vandaag de dag kan het de koopkracht flink
onder druk zetten, ook in welvarende gebieden zoals Noord-Amerika.
Maatschappijen hebben een sterke tendens om zichzelf crisis-gevoelig te
maken wanneer de bestaande instellingen hun beste tijd hebben gehad. In
het verleden manifesteerde deze tendens zich vaak in een bevolkingsgroei
die de grenzen van wat het land kon dragen, opzocht. Dit gebeurde zowel
voor de overgang van het jaar 1000 als aan het einde van de vijftiende
eeuw. De flinke daling van de koopkracht, veroorzaakt door mislukte
oogsten en lagere opbrengsten, speelde in beide gevallen een
significante rol in het vernietigen van de overheersende instellingen.
Op dit moment zijn die problemen vooral zichtbaar in de
consumentenkredietmarkten. Toen plotselinge temperatuurdalingen de
opbrengsten van oogsten verminderden en het beschikbaar inkomen afnam,
leidde dat tot wanbetalingen van schulden en opstanden tegen de
belastingheffing. Het verleden leert ons dat deze omstandigheden ook in
de toekomst zouden kunnen resulteren in zowel economische afsluiting als
politieke onrust.

3. \textbf{Microben} hebben door hun invloed op ziekten en immuniteit
vaak bepaald hoe macht kon worden ingezet. Dit was zeker het geval in de
Europese verovering van de Nieuwe Wereld, zoals we onderzochten in
\emph{The Great Reckoning}. Europese kolonisten, doordrenkt met ziektes
uit ontwikkelde agrarische samenlevingen, hadden al relatieve immuniteit
tegen kinderinfecties zoals de mazelen. De Indianen die ze ontmoetten
leefden grotendeels in dun bevolkte jager-verzamelaarsstammen, en hadden
nog geen immuniteit tegen Westerse ziekten ontwikkeld. Ze werden
daardoor gedecimeerd. Vaak stierven de meesten zelfs al voordat ze met
blanke mensen in aanraking kwamen, doordat geïnfecteerde Indianen die
aan de kust voor het eerst Europeanen ontmoetten naar het binnenland
reisden, en zo de ziektes verspreidden.

Er zijn ook microbiologische grenzen voor de uitoefening van macht. In
\emph{Blood in the Streets} bespraken we hoe malaria het voor blanken
eeuwenlang onmogelijk maakte om een invasie in tropisch Afrika te
realiseren. Vóór de ontdekking van kinine in het midden van de
negentiende eeuw konden blanke legers niet overleven in regio's met
malaria, hoe superieur hun wapens ook waren.

De interactie tussen mensen en microben heeft ook belangrijke
demografische effecten voortgebracht die de kosten en baten van geweld
hebben veranderd. Wanneer het aantal sterfgevallen sterk fluctueert
vanwege epidemische ziekten, hongersnood, of andere oorzaken, neemt de
kans om te sterven door oorlogsvoering af. Vanaf de zestiende eeuw
werden uitschieters van het sterftecijfer steeds minder frequent, wat
helpt verklaren waarom gezinnen kleiner werden en waarom er vandaag de
dag, vergeleken met vroeger, veel minder tolerantie is voor plotselinge
sterfgevallen in oorlog. Hierdoor is ook de tolerantie voor imperialisme
afgenomen en zijn de kosten om aan machtsprojectie te doen verhoogd in
gemeenschappen met een laag geboortecijfer.

Hedendaagse samenlevingen, bestaande uit kleine families, vinden zelfs
een klein aantal oorlogsdoden vaak onacceptabel. Vroege moderne
samenlevingen waren daarentegen veel toleranter wat betreft de
menselijke kosten die horen bij het imperialisme. Vóór de aanvang van
deze eeuw kregen de meeste ouders vele kinderen, waarvan al werd
verwacht dat enkelen plotseling zouden sterven door ziekte. In deze
periode, waarin jong overlijden doodnormaal was, waren aspirant-soldaten
en hun families veel meer bereid om de gevaren van het slagveld te
trotseren.

\begin{quote}
Machines zijn agressief. De wever wordt een web, de machinist een
machine. Als je gereedschappen niet gebruikt, gebruiken zij jou. --
Emerson
\end{quote}

4. \textbf{Technologie} heeft tijdens de moderne eeuwen verreweg de
grootste rol gespeeld in het bepalen van de kosten en baten van het
projecteren van macht. Dit boek beargumenteert dat het dat zal blijven
doen. Technologie heeft verschillende cruciale dimensies:

\textbf{A. Balans tussen offensief en defensief.} De balans tussen
aanval en verdediging, bepaald door de toegang tot wapentechnologie,
bepaalt de schaal van de politieke organisatie. Wanneer offensieve
capaciteiten stijgen, overheerst het vermogen om macht op afstand te
projecteren, jurisdicties hebben dan de neiging om te consolideren, en
overheden vormen zich op grotere schaal. In tijden zoals nu, nemen
defensieve mogelijkheden toe. Hierdoor wordt het duurder om macht buiten
de kerngebieden te projecteren. Rechtsgebieden hebben de neiging te
versnipperen en grote overheden vallen uiteen in kleinere.

\textbf{B. Gelijkheid en de overheersing van de infanterie.} Een
belangrijk kenmerk dat de mate van gelijkheid onder burgers bepaalt, is
de aard van wapentechnologie. Wapens die relatief goedkoop zijn, zorgen
dat de macht gelijker verdeeld wordt, omdat ze kunnen worden ingezet
door niet-professionals. Dit versterkt het militaire belang van
infanterie. Toen Thomas Jefferson zei dat ``alle mensen als gelijken
zijn geschapen'', klonk dat veel geloofwaardiger dan het eeuwen daarvoor
zou hebben gedaan. Een boer met zijn jachtgeweer was niet alleen even
goed bewapend als de typische Britse soldaat met zijn Brown Bess, hij
was zelfs beter bewapend. De boer met het geweer kon op de soldaat
schieten vanaf een grotere afstand, en met grotere nauwkeurigheid. Dit
was een duidelijk verschil met de Middeleeuwen, toen een boer met een
hooivork - meer kon hij zich niet veroorloven - nauwelijks had kunnen
hopen stand te houden tegen een zwaar bewapende ridder te paard. Niemand
schreef in 1276 dat ``alle mensen als gelijken zijn geschapen.'' Op dat
moment, in de meest fundamentele zin, waren mensen niet gelijk. Één
enkele ridder kon veel meer brute kracht uitoefenen dan tientallen
boeren bij elkaar.

\textbf{C. Voor- en nadelen van schaal op vlak van geweld.} Een andere
variabele die helpt bepalen of er een paar grote overheden of vele
kleine zullen zijn, is de organisatorische schaal die vereist is om de
heersende wapens in te zetten. Als geweld in toenemende mate loont,
hebben overheden voordeel wanneer ze op grote schaal opereren, en hebben
ze de neiging groter te worden. Wanneer een kleine groep echter erin
slaagt hun middelen effectief genoeg in te zetten om de grotere macht te
kunnen weerstaan, heeft soevereiniteit de neiging om te fragmenteren,
wat het geval was tijdens de Middeleeuwen. Kleine, onafhankelijke
autoriteiten voerden veel van de functies van de overheid uit. Zoals we
in een later hoofdstuk zullen bespreken, geloven we dat het
Informatietijdperk de opkomst van cybersoldaten zal brengen, die de
voorbodes zullen zijn van de ontbinding van centrale macht.
Cybersoldaten zouden niet alleen door natiestaten kunnen worden ingezet,
maar ook door hele kleine organisaties, en zelfs door individuen.
Oorlogen van het volgende millennium zullen enkele ``veldslagen'' kennen
die volledig zonder bloedvergieten zullen verlopen, met computers.

\textbf{D. Schaalvoordelen van productie.} Een andere cruciale factor
bij het bepalen of de uiteindelijke machtsuitoefening lokaal of op
afstand plaatsvindt, is de schaal van de bedrijven waarin mensen hun
geld verdienen. Als belangrijke bedrijven alleen goed kunnen
functioneren op grote schaal, binnen een groot handelsgebied, dan kunnen
overheden die dat mogelijk maken voldoende bijkomende inkomsten afromen
om een grote overheid te onderhouden. Onder zulke omstandigheden
functioneert de hele wereldeconomie gewoonlijk effectiever als één
sterke wereldmacht alle anderen domineert, zoals het Britse Rijk dat
deed in de negentiende eeuw. Soms leiden de megapolitieke variabelen er
echter toe dat de schaal van economische handel in elkaar stort. Als het
onderhouden van een groot handelsgebied steeds minder economisch
voordeel oplevert, kunnen grotere overheden, die voorheen profiteerden
van de voordelen van grotere handelsgebieden, beginnen uiteen te vallen
- zelfs in situaties waarin het strategische evenwicht tussen offensieve
en defensieve middelen grotendeels constant blijft.

\textbf{E. Verspreiding van technologie.} Nog een factor die invloed
heeft op macht, is hoe wijdverspreid belangrijke technologieën zijn.
Wanneer het mensen lukt om wapens of productiemiddelen te accumuleren of
te monopoliseren, hebben ze de neiging om macht te centraliseren. Zelfs
technologieën met een overwegend defensief karakter, zoals het
machinegeweer, bleken tijdens de periode waarin ze niet wijd verspreid
waren, krachtige offensieve wapens te zijn die bijdroegen aan een
groeiende schaal van de overheid. Toen de Europese machten laat in de
negentiende eeuw een monopolie op machinegeweren hadden, waren ze in
staat om die wapens tegen volkeren aan de grenzen van hun territoria in
te zetten, en zo hun koloniale rijken enorm uit te breiden. Later, in de
twintigste eeuw, toen machinegeweren wijd beschikbaar werden, vooral in
de nasleep van Wereldoorlog II, werden ze juist ingezet om rijken te
vernietigen. Als alle andere omstandigheden gelijk blijven, geldt: hoe
breder sleuteltechnologieën verspreid zijn, des te breder de macht
verspreid zal zijn en des te kleiner de optimale schaal van de overheid.

\section{De snelheid van megapolitieke
verandering}\label{de-snelheid-van-megapolitieke-verandering}

Hoewel technologie vandaag verreweg de belangrijkste factor is, en
schijnbaar in toenemende mate, hebben de vier grote megapolitieke
factoren in het verleden elk een rol gespeeld in het bepalen van de
schaal waarop macht kon worden uitgeoefend.

Samen bepalen deze factoren of het loont om geweld op grotere schaal toe
te passen. Dit bepaalt hoe belangrijk vuurkracht is in verhouding tot
efficiënt gebruik van middelen. Het heeft ook een sterke invloed op de
verdeling van inkomen in de markt. De vraag is: welke rol zullen deze
factoren in de toekomst spelen? Om hier een antwoord op te geven is het
cruciaal om te weten dat deze megapolitieke variabelen zich met enorm
uiteenlopende snelheden ontwikkelen.

Topografie is vrijwel onveranderlijk geweest doorheen de hele
geschiedenis. Buiten kleine lokale effecten, zoals het dichtslibben van
havens, het opspuiten van land of erosie, is de topografie van de aarde
bijna hetzelfde vandaag als toen Adam en Eva uit Eden stapten. En het
zal waarschijnlijk zo blijven totdat een nieuwe ijstijd de landschappen
van de continenten opnieuw vormgeeft, of totdat een andere ingrijpende
gebeurtenis het aardoppervlak verstoort. Op een fundamenteler niveau
lijken geologische tijdperken te veranderen over periodes van 10 tot 40
miljoen jaar, wellicht als gevolg van grote meteorietinslagen. Ooit
kunnen er opnieuw geologische omwentelingen zijn die de topografie van
onze planeet aanzienlijk zullen veranderen. Als dat gebeurt, mag je er
wel van uit gaan dat zowel het honkbal- als het cricketseizoen zullen
worden geannuleerd.

Het klimaat verandert veel frequenter dan topografie. In het laatste
miljoen jaar heeft klimaatverandering voor het merendeel van de bekende
veranderingen in het aardoppervlak gezorgd. Tijdens ijstijden groeven
gletsjers nieuwe valleien, veranderden de loop van rivieren, scheidden
eilanden van continenten of voegden ze samen door het zeeniveau te
verlagen.

Fluctuaties in het klimaat hebben een significante rol gespeeld in de
geschiedenis. Eerst in het bewerkstelligen van de Agrarische Revolutie
na het einde van de laatste ijstijd, en later in het destabiliseren van
regimes tijdens perioden van koudere temperaturen en droogte.
Recentelijk zijn er zorgen geweest over de mogelijke impact van
``wereldwijde opwarming.'' Deze zorgen kunnen niet zonder meer worden
weggewuifd. Toch, vanuit een langer perspectief genomen, lijkt juist een
verschuiving naar een kouder, en niet een warmer klimaat, een groter
risico te zijn. Studies van temperatuurfluctuaties, gebaseerd op de
analyse van zuurstofisotopen in kernmonsters van de oceaanbodem, tonen
dat de huidige periode de tweede warmste is in meer dan 2 miljoen jaar.
Als het kouder zou worden, zoals dat in de zeventiende eeuw gebeurde,
zou dat een destabiliserende werking kunnen hebben op de megapolitiek.
In die zin is het huidige alarmisme over de opwarming van de aarde
wellicht geruststellend. Als deze waarschuwingen kloppen, betekent dat
dat de temperaturen binnen het abnormaal warme en relatief milde bereik
dat we de afgelopen drie eeuwen hebben meegemaakt, blijven fluctueren.

Het tempo van verandering door microben is echter een puzzel. Microben
kunnen zeer snel muteren. Dit is vooral waar voor virussen. De gewone
verkoudheid, bijvoorbeeld, muteert op een bijna kaleidoscopische manier.
Hoewel deze mutaties in hoog tempo plaatsvinden, is hun effect op het
verleggen van de grenzen waarbinnen macht wordt uitgeoefend aanzienlijk
minder abrupt dan dat van technologische veranderingen. Waarom? Een deel
van de reden is dat microben er meer baat bij hebben om hun gastheer
enkel te infecteren dan om hem te doden. Virulente infecties die hun
gastheren te gemakkelijk doden, hebben de neiging om zichzelf in het
proces uit te roeien. Microparasieten kunnen alleen overleven als ze hun
gastheren niet te snel of allemaal tegelijk doden.

Dat wil natuurlijk niet zeggen dat er geen uitbarstingen van dodelijke
ziektes kunnen voorkomen die de machtsbalans veranderen. Zulke episodes
kwamen prominent in de geschiedenis voor. De Zwarte Dood decimeerde
aanzienlijke delen van de Euraziatische bevolking en bracht ernstige
schade toe aan de internationale economie van de veertiende eeuw.

\subsection{Wat had kunnen zijn}\label{wat-had-kunnen-zijn}

Je kunt geschiedenis bekijken als wat er daadwerkelijk is gebeurd, maar
ook als wat er had kunnen gebeuren. Het was niet onmogelijk voor
microparasieten om in de moderne tijd nog grotere schade te hebben
aangericht. Stel je voor dat er, net als malaria maar dan erger, een
ziekte was geweest die de macht van het Westen aan haar grenzen had
kunnen tegenhouden. De eerste Portugese zeevaarders die richting Afrika
gingen, hadden zomaar een dodelijk retrovirus kunnen oplopen -- een nog
besmettelijkere vorm van aids -- waardoor die hele nieuwe handelsroute
naar Azië er nooit was gekomen. Ook Columbus en de eerste kolonisten in
de Nieuwe Wereld hadden ziekten kunnen tegenkomen die hen op dezelfde
manier zouden teisteren als de inheemse lokale bevolkingen, die door
mazelen en andere Westerse kinderziekten werden getroffen. Toch gebeurde
dit niet, waardoor je bijna het gevoel krijgt alsof de geschiedenis een
soort lotsbestemming kent.

Microben droegen in de moderne periode minder bij aan het afremmen van
de consolidatie van macht, maar faciliteerden het eerder. Westerse
troepen en kolonisten ervaarden dat de technologische voordelen, die hen
toelieten om macht uit te oefenen, werden versterkt door de
microbiologische. Westerlingen waren gewapend met ongeziene biologische
wapens, namelijk hun relatieve immuniteit voor kinderziekten die
inheemse volkeren vaak juist verwoestten. Dit gaf Westerse reizigers een
duidelijk voordeel op hun minder dicht bevolkte tegenstanders. De
overdracht van ziekten ontwikkelde zich bijna geheel in één richting,
van Europa naar buiten. Er was geen gelijkaardige overdracht in de
andere richting, van de periferie naar de kern.

Als een mogelijk tegenvoorbeeld hebben sommigen beweerd dat Westerse
ontdekkingsreizigers syfilis van de Nieuwe Wereld naar Europa
importeerden. Dit is discutabel. Als het echter waar is, bleek het geen
significante barrière voor de uitoefening van macht te zijn. Syfilis had
vooral als gevolg dat de seksuele normen en gewoonten in het Westen
veranderden. Vanaf het einde van de vijftiende eeuw tot het laatste
kwart van de twintigste eeuw was de impact van microben op industriële
samenleving steeds minder desastreus. Ondanks de persoonlijke tragedies
en het ongeluk, veroorzaakt door uitbraken van tuberculose, polio, en
griep, kwamen er in de moderne periode geen nieuwe ziekten op die de
megapolitieke impact van de Antonijnse plagen of de Zwarte Dood zelfs
benaderden. Tijdens de moderne periode verbeterde de volksgezondheid, en
werden vaccinaties en tegengif wijdverbreid. In grote lijnen verminderde
dit de invloed van infectieuze microben op de machtsprojectie, waardoor
het relatieve belang van technologie sterk toenam.

De recente opkomst van AIDS en waarschuwingen over de potentiële
verspreiding van exotische virussen zijn hints dat er geen garantie is
dat de rol van microben in de toekomst even goedaardig zal zijn als dat
het over de afgelopen vijfhonderd jaar is geweest. Maar het is
onmogelijk om te weten of, of wanneer, een nieuwe plaag de wereld zal
infecteren. De kans dat een uitbarsting van microparasieten, zoals een
virale pandemie, de megapolitieke dominantie van technologie zou kunnen
verstoren, is vele malen groter dan een verandering van klimaat of
topografie.

We hebben geen manier om grote veranderingen in het leven op aarde zoals
we dat kennen te monitoren of te voorspellen. Dus houden we onze vingers
gekruist en hopen we dat de belangrijkste megapolitieke krachten in het
komende millennium technologisch zullen zijn, en niet biologisch van
aard. Als geluk aan de zijde van de mensheid blijft, zal technologie als
de leidende megapolitieke variabele aan belang blijven toenemen.

Dit was echter niet altijd zo, zoals een overzicht van de eerste grote
megapoliteike transformatie, de Agrarische Revolutie, duidelijkt toont.

\bookmarksetup{startatroot}

\chapter{Ten oosten van Eden}\label{ten-oosten-van-eden}

::: \{.content-hidden when-format=``latex''\} \emph{De landbouwrevolutie
en de verfijning van geweld} :::

\begin{quote}
`Toen vroeg de HEER: 'Waar is je broer Abel?' Kaïn antwoordde: `Dat weet
ik niet. Moet ik soms waken over mijn broer?' `Wat heb je gedaan?' zei
de HEER. `Hoor toch hoe het bloed van je broer uit de aarde tot mij
roept!' - GENESIS 4:9-10
\end{quote}

Vijfhonderd generaties geleden begon de eerste faseverandering in de
organisatie van de menselijke samenleving. Onze voorouders in
verscheidene regio's pakten met tegenzin primitieve werktuigen op, zoals
scherp gemaakte stokken en provisorische schoffels, en gingen aan het
werk. Toen ze de eerste gewassen zaaiden, legden ze ook een nieuwe basis
voor de macht in de wereld. De landbouwrevolutie was de eerste grote
economische en sociale revolutie. Deze revolutie begon met de
verdrijving uit Eden en ontwikkelde zich zo langzaam dat het het jagen
en verzamelen, toen de twintigste eeuw aanbrak, nog niet volledig had
verdrongen in alle geschikte gebieden op aarde. Deskundigen geloven dat
de landbouw, zelfs in het Nabije Oosten waar het voor het eerst
ontstond, werd ingevoerd via `een lang, stapsgewijs proces' dat
`wellicht vijfduizend jaar of meer in beslag nam'.

Het lijkt wellicht overdreven om een proces dat zich over millennia
uitstrekte als een `revolutie' te omschrijven. Toch was de komst van de
landbouw precies dat: een vertraagde revolutie die het menselijk leven,
door de logica van geweld te veranderen, transformeerde. Waar landbouw
vaste grond kreeg, werd geweld een steeds zichtbaarder aspect van het
dagelijkse leven. Zij die in staat waren om geweld te organiseren of
beheersen, gingen de samenleving domineren.

Wie de agrarische revolutie begrijpt, heeft een eerste stap gezet in het
doorgronden van de Informatierevolutie. De introductie van het ploegen
en oogsten is een schoolvoorbeeld van hoe de maatschappelijke
organisatie ingrijpend kan veranderen door een ogenschijnlijk simpele
verandering in de aard van arbeid. Als je deze revolutie uit het
verleden in het juiste perspectief kunt plaatsen, zul je beter kunnen
inschatten hoe de geschiedenis zich zal ontvouwen als reactie op de
nieuwe logica van geweld, die met microprocessoren haar intrede deed.

Om het revolutionaire karakter van de landbouw op waarde te schatten,
moet je eerst een beeld hebben van hoe de oermaatschappij functioneerde.
We hebben dit thema uitgebreid verkend in \emph{The Great Reckoning},
maar zullen het hieronder kort schetsen. Gedurende een lange
prehistorische periode van relatieve stilstand, waarin het menselijk
bestaan nauwelijks evolueerde van generatie op generatie, waren
jager-verzamelaarsamenlevingen de enige sociale organisatiestructuren.
Antropologen stellen dat de mens gedurende 99 procent van zijn bestaan
op aarde een jager en verzamelaar is geweest. Cruciaal voor het
langdurige succes, maar ook het uiteindelijke falen van
jager-verzamelaarsgroepen, is het feit dat ze op zeer kleine schaal in
een uitgestrekt gebied moesten opereren.

Ze konden alleen overleven waar de bevolkingsdichtheid gering was. Denk,
om te begrijpen waarom, aan de problemen die grotere groepen met zich
mee zouden hebben gebracht. Ten eerste zou een groep van duizend jagers,
die tegelijk door het landschap trok, zoveel tumult hebben veroorzaakt
dat het wild op de vlucht zou zijn gejaagd. Erger nog, zelfs als een
klein leger jagers er nu en dan in zou zijn geslaagd om een enorme kudde
wild in het nauw te drijven, zou het voedsel dat ze verzamelden --
inclusief vruchten en eetbare planten uit de natuur -- niet lang in
overvloed zijn gebleven. Een grote groep foerageerders zou het landschap
hebben uitgeput door roofbouw, vergelijkbaar met een uitgehongerd leger
tijdens de Dertigjarige Oorlog. Om deze uitputting te beperken, moesten
de groepen dus klein blijven. Zoals Stephen Boyden schrijft in
\emph{Western Civilization in Biological Perspective}:
`Jager-verzamelaarsgroepen tellen doorgaans tussen de vijfentwintig en
vijftig individuen'.

Vandaag de dag is leven op tienduizend hectare in een gematigd klimaat
een luxe die slechts voor de allerrijksten is weggelegd. Een familie van
jager-verzamelaars had aan minder nauwelijks genoeg gehad. Ze hadden
doorgaans duizenden hectare per persoon nodig, zelfs in de meest
vruchtbare gebieden. Dit verklaart waarom periodes van snelle
bevolkingsgroei, juist in periodes met goede omstandigheden voor
landbouw, konden uitmonden in crisissen. Jager-verzamelaars hadden
zoveel ruimte nodig dat hun gemeenschappen nauwelijks dichter bevolkt
waren dan het leefgebied van beren, lang voor de eerste akkers
ontstonden.

Op kleine verschillen na leek het menselijke dieet op dat van een beer.
Deze foeragerende samenlevingen waren voor hun voedsel afhankelijk van
wat het open landschap en nabijgelegen wateren te bieden hadden. Hoewel
sommigen visten, waren de meeste jagers voor een derde tot een vijfde
van hun voedselinname afhankelijk van de eiwitten van grote zoogdieren.
Naast enkele eenvoudige werktuigen en voorwerpen die ze met zich
meedroegen, hadden jager-verzamelaars vrijwel geen technologie tot hun
beschikking. Ze misten doorgaans effectieve methoden om vlees of ander
voedsel langdurig te bewaren. Het meeste voedsel moest kort na het
verzamelen worden geconsumeerd, anders bedierf het. Dit betekent
natuurlijk niet dat sommige jager-verzamelaars geen bedorven voedsel
aten. Eskimo's, zo meldt Boyden, `zouden een bijzondere voorliefde
hebben voor voedsel in staat van ontbinding'.{[}\^{}54{]} Hij citeert de
waarnemingen van deskundigen die stellen dat Eskimo's `viskoppen
begraven en ze laten wegrotten tot de graten dezelfde consistentie
hebben als het vlees. Vervolgens kneden ze de stinkende massa tot een
pasta en eten die op'; ze zijn ook dol op de `vette, madenrijke larven
van de kariboevlieg, rauw geserveerd\ldots{} hertenkeutels, die ze als
bessen eten\ldots{} en merg van meer dan een jaar oud, wemelend van de
maden'.{[}\^{}55{]}

Op enkele delicatessen na legden jager-verzamelaars nauwelijks
voedseloverschotten aan. Zoals antropoloog Gregg opmerkt, `slaan
rondtrekkende bevolkingsgroepen over het algemeen geen voedselvoorraden
aan als buffer tegen seizoensgebonden of onverwachte schaarste.' Er viel
bij de jager-verzamelaars bijgevolg weinig te stelen. Het was onhoudbaar
in omstandigheden waarin geen voedseloverschot kon worden bewaard, om
een arbeidsverdeling op te bouwen waarin mensen gespecialiseerd waren in
het toepassen van geweld. De logica van de jacht bracht ook met zich mee
dat geweld tussen jager-verzamelaarsgroepen nooit grootschalig kon zijn,
omdat de groepen zelf klein moesten blijven.

De kleine omvang van de jager-verzamelaarsgroepen was ook op een andere
manier voordelig. De leden van zulke kleine groepen kenden elkaar door
en door, waardoor ze effectiever konden samenwerken. Besluitvorming
wordt lastiger naarmate een groep groter wordt, omdat er steeds meer
tegenstrijdige belangen ontstaan. Denk maar eens aan hoe moeilijk het is
om met twaalf man een etentje te organiseren. Stel je voor hoe hopeloos
de taak zou zijn geweest om honderden of duizenden mensen te organiseren
tijdens een voortdurend verplaatsend feestmaal. Omdat zij geen
blijvende, afzonderlijke politieke organisatie of bureaucratie hadden,
zoals vereist voor oorlogsspecialisatie, moesten
jager-verzamelaarsgroepen vertrouwen op overtuiging en consensus --
principes die het best werken bij kleine groepen met een relatief
gemoedelijke stemming.

Of het er daadwerkelijk gemoedelijk aan toe ging in groepen
jager-verzamelaars staat echter ter discussie. Sir Henry Maine verwijst
naar `de universele oorlogszuchtigheid van de primitieve mens'. In zijn
woorden: `Het is niet vrede dat natuurlijk en primitief is, maar
oorlog.' Zijn visie wordt ondersteund door het werk van evolutionair
biologen. R. Paul Shaw en Yuwa Wong merken op: `Er zijn sterke
aanwijzingen dat veel van de zichtbare verwondingen op overblijfselen
van de \emph{Australopithecus}, \emph{Homo erectus} en \emph{Homo
sapiens} uit de vierde Europese ijstijd en de daaraan voorafgaande
periodes het gevolg waren van gevechten.' Anderen trekken dit echter in
twijfel. Deskundigen als Stephen Boyden beweren dat primitieve groepen
doorgaans niet oorlogszuchtig of gewelddadig waren. Er ontstonden
sociale conventies om interne spanningen te verminderen en het
makkelijker te maken om de jachtbuit te verdelen. Vooral in gebieden
waar men op groter wild joeg, dat voor een enkele jager moeilijk te
vellen was, ontstonden religieuze en sociale regels die de herverdeling
van al het gevangen wild onder de hele groep faciliteerden. Wanneer een
calorierijke buit moest worden verdeeld, kregen andere jagers de eerste
prioriteit. Noodzaak, en niet sentiment, was hierbij de drijfveer. De
eersten die aanspraak maakten op de buit waren de economisch meest
bekwame en militair sterkste leden, niet de zieken en de zwakken. Een
belangrijke reden voor deze voorrangsregel was ongetwijfeld het feit dat
jagers in de fleur van hun leven ook de militair sterkste leden van de
kleine groep waren. Door hen het eerste recht op de buit te geven,
werden mogelijk dodelijke interne conflicten in de groep tot een minimum
beperkt.

Zolang de bevolkingsdichtheid van de jager-verzamelaars laag bleef,
waren hun goden niet militant, maar ze belichaamden natuurkrachten of de
dieren waarop ze jaagden. De beperkte hoeveelheid bezittingen en de
uitgestrekte, open gebieden maakten oorlogsvoering in de meeste gevallen
overbodig. Buiten de eigen kleine familie of clan waren er weinig buren
die een bedreiging konden vormen. Doordat jager-verzamelaars rondtrokken
op zoek naar voedsel, werden bezittingen die het strikte minimum
overschreden een last. Wie weinig bezat, had vanzelfsprekend ook weinig
last van schendingen van het eigendomsrecht. Omdat men weinig
investeerde in een specifieke plek, koos men er bij conflicten vaak voor
om de wegen te scheiden. Vluchten was een eenvoudige oplossing voor
persoonlijke vetes of andere buitensporige eisen. Dit betekent niet dat
de vroege mens vreedzaam was. Mogelijk waren ze gewelddadig en
onaangenaam in een mate die we ons nauwelijks kunnen voorstellen. Maar
als ze geweld gebruikten, was dat meestal om persoonlijke redenen of,
wat misschien nog erger is, voor de sport.

Jagers-verzamelaars leefden in kleine groepen waarin, afgezien van het
onderscheid tussen de geslachten, nauwelijks sprake was van een
arbeidsverdeling. Ze hadden geen georganiseerde overheid, doorgaans geen
permanente nederzettingen en geen mogelijkheid om rijkdom te vergaren.
Zelfs elementaire bouwstenen van de beschaving, zoals een geschreven
taal, waren onbekend in de oereconomie. Zonder geschreven taal konden er
geen officiële archieven worden bijgehouden en dus geen geschiedenis.

\subsection{Overbejaging}\label{overbejaging}

Het bestaan als jager-verzamelaar kende heel andere prikkels dan die
waaraan we sinds de opkomst van de landbouw gewend zijn geraakt. Voor
het leven als jager-verzamelaar was maar weinig kapitaal nodig. Een paar
primitieve gereedschappen en wapens volstonden. Er waren geen
mogelijkheden om te investeren en er was zelfs geen particulier
grondbezit, met uitzondering van steengroeven waar vuursteen of
speksteen werd gewonnen. Zoals antropologe Susan Alling Gregg in
\emph{Foragers and Farmers} schreef, `was het bezit van en de toegang
tot hulpbronnen gemeenschappelijk eigendom van de groep.' Op zeldzame
uitzonderingen na, zoals vissers die aan de oevers van meren leefden,
hadden jagers-verzamelaars doorgaans geen vaste verblijfplaats. Omdat ze
geen permanent onderkomen hadden, was er voor hen weinig reden om hard
te werken voor bezittingen of het onderhoud ervan. Ze hoefden geen
hypotheek of belastingen te betalen en geen meubels te kopen. Het kleine
aantal consumptiegoederen dat ze bezaten, bestonden uit dierenhuiden en
persoonlijke versieringen die de groepsleden zelf maakten. Er waren
amper prikkels om iets te verzamelen dat als geld had kunnen dienen,
want er was nauwelijks iets te koop. In zulke omstandigheden stelde het
begrip `sparen' voor jagers-verzamelaars weinig voor.

Omdat er geen reden was om te verdienen en er nauwelijks sprake was van
een arbeidsverdeling, moet het idee van hard werken als deugd vreemd
voor ze zijn geweest. Behalve in tijden van uitzonderlijke tegenspoed,
wanneer een langdurige inspanning vereist was om iets eetbaars te
vinden, verrichtte men weinig arbeid omdat er simpelweg weinig nodig
was. Er viel letterlijk niets te winnen door meer te presteren dan het
strikte minimum dat nodig was om te overleven. Voor de leden van een
doorsnee jagers-verzamelaarsgroep betekende dat een werkweek van slechts
acht tot vijftien uur. Omdat de arbeid van een jager de voedselvoorraad
niet deed toenemen maar enkel kon uitputten, droeg iemand die zich
heldhaftig uitsloofde om meer dieren te doden of meer vruchten te
plukken dan men kon opeten voor het bedierf, niets bij aan de welvaart.
Integendeel, overbejaging verminderde de kans om in de toekomst voedsel
te vinden en had dus een schadelijke invloed op het welzijn van de
groep. Daarom bestraften of verstootten sommige jagers-verzamelaars,
zoals de Eskimo's, groepsleden die zich aan overbejaging schuldig
maakten.

Het voorbeeld van de Eskimo's die overbejaging bestraffen, is bijzonder
veelzeggend, omdat ze, veel meer dan anderen, vlees hadden kunnen
bewaren door het te bevriezen. Bovendien was het voor hen mogelijk om
ten minste een deel van de oliën, die uit grote zeedieren werden
gewonnen, op te slaan. Dat jagers-verzamelaars er over het algemeen voor
kozen om dit niet te doen, weerspiegelt hun veel passievere omgang met
de natuur. Het zou ook wat kunnen zeggen over de mate waarin cognitie en
mentale processen door cultuur worden beïnvloed. Het leerproces en
gedrag worden begrensd door de complexe omgeving, waardoor de toepassing
van sommige strategieën veel moeilijker kan zijn dan op het eerste
gezicht lijkt. Zoals R. Paul Shaw en Yuwa Wong schreven: `Omdat niches
in veel opzichten verschillen, verschillen ook de leerpatronen.'

Vanuit dit perspectief bracht de opkomst van de landbouw meer met zich
mee dan enkel een verandering in het voedingspatroon; het ontketende ook
een grote revolutie in de organisatie van het economische leven en de
cultuur, en een transformatie van de logica van geweld. Landbouw bracht
grootschalige kapitaalgoederen in de vorm van grond en soms
irrigatiesystemen voort. De gewassen en gedomesticeerde dieren die de
boeren verbouwden en verzorgden, waren waardevolle bezittingen. Ze
konden worden opgeslagen, gehamsterd en gestolen. Omdat de gewassen
gedurende het hele groeiseizoen verzorgd moesten worden, van het zaaien
tot de oogst, werd wegtrekken bij dreigingen minder aantrekkelijk,
vooral in droge streken waar de mogelijkheden om gewassen te verbouwen
beperkt waren tot de kleine gebieden met een betrouwbare watertoevoer.
Naarmate ontsnappen moeilijker werd, namen de kansen op georganiseerde
afpersing en plunderingen toe. Boeren waren tijdens de oogsttijd vaak
het slachtoffer van overvallen. Geleidelijk nam de schaal van
oorlogvoering dus toe.

Hierdoor neigde de omvang van samenlevingen te vergroten, omdat
gewelddadige conflicten meestal werden gewonnen door de grotere groep.
Naarmate de concurrentie om grond en de controle over de opbrengst ervan
toenam, werden maatschappijen steeds minder mobiel. Er ontstond een
duidelijke arbeidsverdeling. Voor het eerst deden loonarbeid en
slavernij hun intrede. Boeren en veehouders specialiseerden zich in
voedselproductie. Pottenbakkers maakten opslagpotten voor voedsel.
Priesters baden tot hun god(en) voor regen en rijke oogsten.
Specialisten in geweld -- de voorvaderen van de overheid -- richtten
zich steeds meer op plundering en bescherming tegen plundering. Samen
met de priesters werden zij de eerste rijken in de geschiedenis.

In de vroege stadia van agrarische samenlevingen eisten deze krijgers
een deel van de jaarlijkse oogst op in ruil voor bescherming. Op plekken
waar de dreiging minimaal was, wisten zelfstandige boeren soms een
relatief grote mate van autonomie te behouden. Naarmate de
bevolkingsdichtheid echter toenam en de strijd om voedsel verhevigde,
vooral in woestijnachtige gebieden waar productieve grond schaars was,
kon de krijgersgroep een groot deel van de totale opbrengst voor zich
opeisen. Met de inkomsten uit deze afgedwongen heffingen, die opliepen
tot wel 25 procent van de graanoogst en de helft van de groei van de
veestapel, stichtten deze krijgers de eerste staten. Door landbouw nam
het belang van dwang aanzienlijk toe. De sterke toename van middelen die
buitgemaakt konden worden, leidde tot een forse stijging van het aantal
plunderingen.

Het duurde millennia voordat de logica van de agrarische revolutie zich
volledig manifesteerde. Lange tijd leefden dunbevolkte gemeenschappen
van boeren in gematigde streken mogelijk grotendeels op dezelfde wijze
als hun jagende en verzamelende voorouders. Waar grond en regenval
overvloedig waren, verbouwden boeren hun gewassen op kleine schaal
zonder veel gewelddadige inmenging. Maar toen de bevolking over een
periode van enkele duizenden jaren groeide, kregen zelfs boeren in
dunbevolkte gebieden te maken met onvoorspelbare plunderingen, waardoor
ze soms niet genoeg zaaigoed overhielden voor de oogst van het volgende
jaar. Wederzijdse plunderingen, oftewel anarchie, was een van de
uitersten, net als onbeschermde gemeenschappen die leefden zonder een
gespecialiseerde organisatie die het monopolie op geweld had.

Na verloop van tijd verspreidde de aan landbouw inherente logica van
geweld zich over een steeds groter gebied. De regio's waar landbouw en
veeteelt konden voortbestaan zonder de strooptochten van overheden,
werden teruggedrongen tot enkele zeer afgelegen gebieden. De
Kafir-regio's van Afghanistan, om een extreem voorbeeld te noemen, boden
weerstand tegen de invoering van een centrale overheid tot in het
laatste decennium van de negentiende eeuw. In dat proces waren ze echter
al eeuwen eerder veranderd in een vrij militante samenleving,
georganiseerd op basis van verwantschapslijnen. Dergelijke structuren
waren niet in staat om op grote schaal militaire macht te mobiliseren.
Tot de Britten moderne wapens naar de regio brachten, bleven de Kafirs
onafhankelijk in hun afgelegen Bashgal- en Waigal-valleien, omdat hun
bolwerken werden beschermd door de topografie, met hoge bergen en
woestijnen die hen scheidden van veroveraars van buitenaf.

Na verloop van tijd drukte de fundamentele logica van de agrarische
revolutie haar stempel op de samenlevingen waar landbouw werd toegepast.
Door de landbouw konden menselijke gemeenschappen op een veel grotere
schaal functioneren. Ongeveer tienduizend jaar geleden begonnen steden
op te komen. Hoewel klein naar de huidige maatstaven, waren ze de centra
van de eerste `beschavingen', een woord afgeleid van \emph{civitas}, wat
in het Latijn `burgerschap' of `inwoner van een stad' betekent. Omdat de
landbouw bezittingen voortbracht die geplunderd en beschermd moesten
worden, ontstond ook de noodzaak voor een voorraadadministratie.
Belasting heffen is onmogelijk als men geen administratie kan voeren en
ontvangstbewijzen kan uitschrijven. De symbolen die in de boekhouding
werden gebruikt, vormden de beginselen van het schrift, een vernieuwing
die bij jagers en verzamelaars nooit had bestaan.

De landbouw verlegde ook de horizon waarbinnen de mens problemen moest
oplossen. Jagersstammen leefden met een zeer beperkte tijdshorizon. Ze
ondernamen zelden projecten die langer dan enkele dagen duurden. Het
planten en oogsten van een gewas nam echter maanden in beslag. Projecten
met een langere doorlooptijd dreven boeren ertoe hun aandacht op de
sterren te richten. Gedetailleerde astronomische waarnemingen waren een
voorwaarde voor het opstellen van almanakken en kalenders die als
leidraad dienden voor de beste zaai- en oogsttijden. Met de komst van de
landbouw verruimde de horizon van de jagers zich.

\section{Eigendom}\label{eigendom}

De overgang naar een sedentaire agrarische samenleving leidde tot het
ontstaan van particulier eigendom. Het is evident dat niemand bereid was
een heel groeiseizoen op de akkers te zwoegen, om daarna toe te kijken
hoe een ander het resultaat van zijn arbeid opeiste. Het idee van
eigendom ontstond als een onvermijdelijk gevolg van de landbouw. De
helderheid van het concept van particulier eigendom werd echter
ondermijnd door de logica van geweld die met de introductie van de
landbouw gepaard ging. De opkomst van eigendom werd vertroebeld door het
feit dat de megapolitieke macht van individuen niet langer zo
gelijkwaardig was als in jager-verzamelaarssamenlevingen, waar elke
gezonde volwassen man een jager was en even zwaar bewapend was als ieder
ander. De landbouw leidde tot een specialisatie in geweld. Juist omdat
er iets te stelen viel, werden investeringen in betere wapens lonend.
Het gevolg was diefstal, vaak op een sterk georganiseerde schaal.

De machtigen konden nu een nieuwe vorm van roof organiseren: een
geweldsmonopolie, oftewel een `overheid'. Dit zorgde voor een scherpe
tweedeling in samenlevingen, waardoor zeer verschillende omstandigheden
ontstonden voor degenen die profiteerden van de plunderingen en de massa
armen die de velden bewerkten. De enkelingen die de militaire macht in
handen hadden, konden nu rijk worden, samen met anderen die hun gunst
genoten. De god-koningen en hun bondgenoten, de diverse lagere, lokale
machthebbers die over de eerste staten in het Nabije Oosten heersten,
genoten van eigendomsvormen die veel meer op de moderne leken dan die
van de grote massa die onder hen zwoegde.

Natuurlijk is het historisch niet correct om in de vroege agrarische
samenlevingen te spreken van een onderscheid tussen particulier en
publiek bezit. De heersende god-koning beschikte over alle middelen van
de staat op een manier die nauwelijks te onderscheiden was van het bezit
van een uitgestrekt landgoed. Net als in de feodale periode van de
Europese geschiedenis was al het bezit onderworpen aan de heerschappij
van hogere machthebbers. Wie lager op de hiërarchische ladder stond,
merkte dat zijn eigendom te allen tijde onderhevig was aan de grillen
van de heerser.

Het feit dat de heerser niet door de wet werd beperkt, betekende echter
niet dat hij het zich kon veroorloven om zomaar alles in beslag te nemen
wat hij wilde. De vrijheid van de farao werd ingeperkt door de heersende
kosten en baten, net zoals die van de premier van Canada vandaag de dag.
Sterker nog, de farao werd veel meer belemmerd dan hedendaagse leiders,
vanwege de beperkte transport- en communicatiemogelijkheden. Alleen al
het verplaatsen van een buit, vooral als het voornamelijk
landbouwproducten waren, zorgde voor grote verliezen door bederf en
diefstal. Door meer ambtenaren aan te stellen die elkaar in de gaten
hielden, werd diefstal beperkt, maar stegen de kosten voor de farao
aanzienlijk. Decentrale autoriteit verbeterde soms de productie, maar
vormde ook krachtige lokale leiders die soms de dynastieke controle
uitdaagden. Zelfs despoten in het Oosten waren niet helemaal vrij om te
doen wat ze wilden; ze moesten de bestaande machtsverhoudingen
respecteren.

Hoewel iedereen, inclusief de rijken, onderhevig was aan willekeurige
onteigening, konden sommigen toch eigen bezit vergaren. Net als nu
besteedde de staat een groot deel van haar inkomsten aan openbare
werken. Projecten als irrigatiesystemen, religieuze monumenten en
koninklijke crypten boden architecten en ambachtslieden de kans om een
inkomen te verdienen. Sommige individuen in een bevoorrechte positie
slaagden erin om een flinke hoeveelheid privébezit op te bouwen. Veel
van de overgebleven spijkerschrifttabletten uit Sumer, een oude
Mesopotamische beschaving, bevatten handelsdocumenten, vooral over het
overdragen van eigendomsrechten.

In de vroege landbouwsamenlevingen bestond privé-eigendom wel, maar
zelden aan de onderkant van de sociale piramide. De overgrote
meerderheid van de bevolking bestond uit boeren die te arm waren om veel
rijkdom te vergaren. Sterker nog, op enkele uitzonderingen na, waren de
meeste boeren tot in de moderne tijd zo arm dat ze bij elke droogte,
overstroming of plaag, voortdurend het risico liepen om van de honger om
te komen. Daarom moesten boeren hun zaken zo organiseren dat de risico's
in slechte jaren tot een minimum beperkt bleven. In de brede, verarmde
lagen van de samenleving bestond een primitievere vorm van eigendom. Die
vergrootte de overlevingskans, maar ging ten koste van de mogelijkheid
om kapitaal te vergaren en hogerop te komen in het economische systeem.

\subsection{Boerenverzekering}\label{boerenverzekering}

De oplossing lag in wat antropologen en sociale historici omschrijven
als het `gesloten dorp'. Bijna elke boerensamenleving in de premoderne
tijden kende het `gesloten dorp' als belangrijkste vorm van economische
organisatie. Anders dan in modernere economische stelsels, waar
individuen doorgaans met vele kopers en verkopers op een open markt
handelen, functioneerden de huishoudens in het gesloten dorp samen als
een soort collectief of een grote familie. Dit was geen open
marktplaats, maar een gesloten systeem waarin alle economische
transacties van het dorp doorgaans werden afgesloten met één monopolist:
de lokale landheer, of zijn vertegenwoordigers. Het dorp maakte
gezamenlijk afspraken met de heer, waarbij men doorgaans een hoog
percentage van de oogst leverde in plaats van een vaste huurprijs. Deze
proportionele pacht betekende dat de landheer een deel van het risico
van een slechte oogst voor zijn rekening nam. Uiteraard streek de
landheer ook het grootste deel van de potentiële winst op. Meestal
zorgden de landheren ook voor het zaaigoed.

Deze regeling beperkte ook het gevaar van hongersnood. Het vereiste dat
de landheer, in plaats van de boer, een onevenredig deel van zijn
oogstaandeel bewaarde. Omdat de landbouwopbrengsten in het verleden in
veel gebieden schrikbarend laag waren, moesten er soms wel twee zaden
worden geplant voor elke drie die men oogstte. Onder zulke
omstandigheden zou een slechte oogst massale hongersnood betekenen. De
boeren hadden natuurlijk de rationele voorkeur voor een regeling waarbij
de landheer in hun voortbestaan investeerde. Hoewel ze moesten kopen
tegen gemonopoliseerde prijzen, goedkoop moesten verkopen en de landheer
van arbeid in natura moesten voorzien, vergrootten de boeren hun
overlevingskansen. Om soortgelijke redenen gaf de gewone boer in een
gesloten dorp de zekerheid van privébezit op. Door zich afhankelijk te
maken van de dorpsleider, maakte het gezin meer kans om te profiteren
van de regelmatige herverdeling van akkers. Het kwam geregeld voor dat
de dorpsleider de beste stukken grond voor zichzelf en zijn kring hield.
Dat was echter een risico dat de boeren moesten tolereren om te kunnen
genieten van de overlevingsverzekering die het diffuse dorpsbezit van de
akkers bood. In tijden waarin de oogstopbrengsten erbarmelijk laag
waren, kon een verschil in de groeiomstandigheden tussen akkers die een
steenworp van elkaar verwijderd lagen, het verschil maken tussen
hongersnood en overleven. Boeren kozen vaak voor de regeling die het
risico op verlies verkleinde, zelfs als dat betekende dat ze elke hoop
op grotere welvaart moesten opgeven.

Over het algemeen kwam risicomijdend gedrag veel voor bij alle groepen
die op de rand van het bestaan balanceerden. De enorme uitdaging om te
overleven in premoderne samenlevingen heeft altijd het gedrag van de
armen beperkt. Een interessant kenmerk van deze risicoaversie, zoals
besproken in \emph{The Great Reckoning}, is dat het de bandbreedte van
vreedzaam economisch gedrag dat individuen maatschappelijk gezien
mochten vertonen, verkleinde. Taboes en sociale beperkingen remden
experimenten en innovatief gedrag af, zelfs ten koste van potentieel
voordelige verbeteringen in bestaande werkwijzen. Dit was niet
onlogisch, aangezien experimenteren tot grotere verschillen in
uitkomsten leidt. Een grotere variatie in de resultaten kan niet alleen
hogere opbrengsten betekenen, maar er is ook het risico op rampzalige
verliezen voor mensen die nauwelijks kunnen overleven. Een groot deel
van de culturele energie van arme landbouwsamenlevingen is altijd
gericht geweest op het onderdrukken van experimenten. Deze onderdrukking
was in feite hun alternatief voor een verzekering. Zouden ze een
verzekering hebben gehad, of voldoende spaargeld om hun experimenten
zelf te dekken, dan zouden zulke sterke sociale taboes niet nodig zijn
geweest om hun overleven te waarborgen.

Culturen zijn functionele aanpassingsmechanismen, geen voorkeuren. Wat
in de ene omgeving werkt, kan elders nutteloos of zelfs schadelijk zijn.
Mensen leven in een grote verscheidenheid aan leefomgevingen. Het brede
scala aan potentiële niches waarin wij leven, vereist gedragsvarianten
die te complex zijn om door instinct alleen te worden gestuurd. Daarom
wordt gedrag cultureel geprogrammeerd. Voor de overgrote meerderheid in
veel agrarische samenlevingen was de cultuur voornamelijk gericht op
overleven, en nauwelijks meer dan dat, terwijl de luxe om deel te nemen
aan open markten voor een selecte groep was weggelegd.

Persoonlijke bekwaamheid en individuele keuze, of zoals de moderne
Amerikanen zeggen ``the pursuit of happiness'', werden onderdrukt door
taboes en maatschappelijke beperkingen die juist onder de armen het
sterkst waren. Zulke beperkingen werden in maatschappijen met een
beperkte productiviteit slechts met grote moeite overwonnen. Waar en
wanneer de landbouwproductiviteit hoger lag, zoals in het oude
Griekenland, vonden kleine megapolitieke revoluties plaats. Eigendom nam
modernere vormen aan. `Allodium', of vrij eigendom, deed zijn intrede.
Grond werd doorgaans verpacht tegen een vaste vergoeding, en de pachter
nam het economische risico voor zijn rekening, maar genoot ook van een
groter deel van de winst als de oogst goed was. Met meer spaarmiddelen
wordt het mogelijk om het risico van riskant gedrag zelf te dragen.
Onder dergelijke omstandigheden konden zelfstandige boeren zich boven de
boerenstand verheffen en soms zelfs een eigen vermogen opbouwen.

Het is kenmerkend voor sociale organisatiestructuren dat
eigendomsrechten en verhoudingen meer marktgericht worden, vooral aan de
economische top of soms zelfs in de volledige economie, wanneer een
samenleving zich uit de armoede werkt. Het is hierbij belangrijk om op
te merken dat de meest voorkomende organisatievorm van de agrarische
samenleving historisch gezien in wezen feodaal was, met marktwerking aan
de top en een gesloten dorpssysteem aan de basis. De grote massa boeren
was in bijna alle premoderne agrarische samenlevingen aan het land
gebonden. Zolang de landbouwproductiviteit laag bleef, of hogere
productiviteit afhankelijk was van centrale irrigatiesystemen, bleven de
vrijheid en eigendomsrechten van individuele boeren onderaan minimaal.
In zulke omstandigheden overheersten feodale eigendomsvormen. Grondbezit
was gebaseerd op pacht, niet op een volledige eigendomstitel, en het
recht op verkoop, schenking en vererving was doorgaans beperkt.

Het feodalisme in zijn diverse vormen was niet enkel een reactie op de
voortdurende dreiging van geweld. Het was tevens een antwoord op de
ontzettend lage productiviteit. In agrarische samenlevingen gingen die
twee vaak hand in hand, en ze versterkten elkaar met regelmaat. Wanneer
het openbare gezag instortte, namen eigendomsrechten en welvaart over
het algemeen evenredig af. Een ineenstortende productiviteit ondermijnde
op haar beurt eveneens het gezag. Hoewel niet elke droogte of ongunstige
klimaatverandering leidde tot het verval van het openbare gezag, was dit
in veel gevallen wel zo.

\section{De feodale revolutie van het jaar
1000}\label{de-feodale-revolutie-van-het-jaar-1000}

Dit was het geval bij de transformatie van het jaar 1000, die de feodale
revolutie in gang zette. De politieke en economische realiteit rond die
periode week op cruciale punten af van het beeld dat we nu van de
Middeleeuwen hebben. In de eerste eeuwen na de val van Rome kwijnde de
economie van West-Europa weg. De Germaanse koninkrijken die in de
gebieden van het voormalige Romeinse Rijk opkwamen, hadden tal van
functies van de Romeinse staat overgenomen, zij het op een veel minder
ambitieus niveau. De infrastructuur werd min of meer aan haar lot
overgelaten. Naarmate de eeuwen verstreken, raakten bruggen en
aquaducten in verval en werden ze onbruikbaar. Hoewel Romeinse munten
nog in gebruik waren, raakten ze nagenoeg uit circulatie. De bloeiende
grondhandel uit de Romeinse tijd viel stil. Steden, ooit administratieve
centra, verdwenen vrijwel geheel samen met het vermogen van de staat om
belasting te heffen. En daarmee verdwenen ook bijna alle andere
kenmerken van beschaving.

De `Donkere Middeleeuwen' heten niet voor niets zo. Geletterdheid werd
zo zeldzaam dat wie kon lezen en schrijven vrijstelling van
rechtsvervolging kon verwachten voor vrijwel elke misdaad, moord
inbegrepen. Artistieke, wetenschappelijke en technische vaardigheden die
in de Romeinse tijd sterk ontwikkeld waren, raakten in de vergetelheid.
Van de aanleg van wegen tot het enten van wijnstokken en fruitbomen,
West-Europa paste veel technieken die ooit algemeen bekend waren en op
hoog niveau werden toegepast, niet langer toe. Zelfs een oeroud werktuig
als de pottenbakkersschijf verdween op veel plaatsen. De
mijnbouwactiviteit nam af, de metaalbewerking liep terug, en
irrigatiewerken in het Middellandse Zeegebied raakten door verwaarlozing
in verval.' Zoals historicus Georges Duby opmerkte: `Aan het einde van
de zesde eeuw was Europa een uiterst onbeschaafde plek.' Hoewel er rond
het jaar 800 onder het bewind van Karel de Grote een korte heropleving
van centraal gezag was, viel al snel na zijn dood alles weer uiteen.

Een verrassend gevolg van dit sombere tafereel was dat de ineenstorting
van de Romeinse staat de levensstandaard van kleine boeren
waarschijnlijk voor enkele eeuwen juist verhoogde. De Germaanse
koninkrijken die West-Europa tijdens de Donkere Middeleeuwen
domineerden, namen enkele van de wat vrijere sociale gewoonten over die
hun voorouderlijke stammen kenden, zoals de juridische gelijkheid van
vrije boeren. Als gevolg daarvan waren kleine boeren in de Donkere
Middeleeuwen veel vrijer dan ze in de feodale eeuwen zouden geweest
zijn. Hieruit kunnen we tevens afleiden dat ze welvarender waren. Zoals
we eerder zagen bij de analyse van eigendomsvormen onder verschillende
productiviteitsniveau's, is vrij eigendom historisch vaak gepaard gegaan
met de relatieve welvaart van kleine boeren. De gesloten dorpen en
feodale eigendomsvormen ontstonden vooral in gebieden waar de
mogelijkheid van kleine boeren om in hun levensonderhoud te voorzien
meer onder druk stond.

Toegegeven, door de vrijwel volledige ineenstorting van de economie
tijdens de Donkere Middeleeuwen liepen kleine boeren de voordelen van
handel en grotere afzetmarkten mis. Door het wegvallen van de steden
stortte de geldeconomie in, maar tegelijkertijd hoefde het platteland
niet meer op te draaien voor de kosten van een verstikkend
overheidsapparaat. Guy Bois schreef hierover dat de Romeinse stad een
parasitaire gemeenschap was en geen productiecentrum: `In de Romeinse
tijd was de voornaamste functie van een stad politiek van aard. Ze
leefde voornamelijk van de inkomsten die via de grondbelasting uit de
omliggende gebieden werden aangevoerd\ldots{} De stad produceerde in
feite weinig tot niets voor het omliggende platteland'. De ineenstorting
van het Romeinse gezag bevrijdde de boeren op het platteland grotendeels
van belastingen, die `tussen een kwart en een derde van de
bruto-opbrengst van het land' opslokte, `zonder de diverse
afpersingspraktijken mee te rekenen waar kleine en middelgrote
grondbezitters onder te lijden hadden'. Door de extreme belastingdruk,
die soms met de dood werd afgedwongen, vertrokken veel mensen en lieten
hun grond achter. De barbaren schaften deze belastingen gelukkig af.

\subsection{Agri deserti}\label{agri-deserti}

De onderdrukking door de overheid nam dusdanig af door de barbaarse
veroveringen dat er voor de armen een mogelijkheid ontstond om eigen
bezit te verwerven en te behouden. Een deel van de \emph{agri deserti},
de boerderijen die door hun eigenaren, vanwege de roofbelastingen in de
nadagen van het Romeinse Rijk waren verlaten, werd opnieuw in productie
genomen. Ondanks de barre omstandigheden van die tijd en het feit dat de
oogsten naar moderne maatstaven belachelijk laag waren, waren de Donkere
Middeleeuwen een periode van relatieve welvaart voor de kleine boeren in
Europa. Ze genoten toen van een sterke machtspositie die ze pas in de
moderne tijd opnieuw zouden bereiken. Enerzijds waren er minder
arbeidskrachten beschikbaar om het vruchtbare land, waarvan grote
stukken braak waren komen te liggen, te bewerken. Epidemieën, oorlogen
en landverlating door eigenaren die het instortende Romeinse Rijk
ontvluchtten, hadden eerder gecultiveerde gebieden aanzienlijk ontvolkt.
Een ander voordeel voor kleine boeren in de Donkere Middeleeuwen kwam
voort uit de invoering van nieuwe landbouwtechnologie in de zesde eeuw:
de zware ploeg, vaak op wielen gemonteerd. In combinatie met een
verbeterd harnas waarmee meerdere ossen konden worden ingezet, maakte
deze technologie het veel eenvoudiger om beboste gronden in Noord-Europa
te ontginnen.

Onder dergelijke omstandigheden verdween de markt voor grond vrijwel
volledig. Nieuwe landbouwgrond kon simpelweg worden verkregen door de
bossen te verwijderen en een deel van elk nieuw perceel af te staan aan
de lokale autoriteiten. Door dit proces, in het Engels bekend als
\emph{assarting}, kon de bevolkingsgroei na de val van Rome eeuwenlang
worden opgevangen. \emph{Assarting} werd met name aantrekkelijk in
dunbevolkte noordelijke regio's, nadat hogere temperaturen in de achtste
eeuw de landbouw productiever maakten.

De leiders van de Germaanse stammen die voormalige Romeinse gebieden
veroverden, vestigden zich als grootgrondbezitters. Het grootste deel
van de rest van de bevolking bewerkte kleine percelen, maar onder hele
andere omstandigheden dan die van het latere feodalisme. De rijkere
landeigenaren, oftewel de meesters, vormden zo'n 7 tot 10 procent van de
bevolking. Het lijkt erop dat vóór het jaar 1000 tweederde van de
dorpelingen in een doorsnee Franse streek vrij landeigenaar was. Ze
bezaten ongeveer de helft van alle gecultiveerde grond. Er waren weinig
lijfeigenen. \emph{Coloni}, ofwel pachtboeren, maakten niet meer dan 5
procent van de bevolking uit. Slavernij bleef bestaan, maar op een veel
kleinere schaal dan in de Romeinse tijd.

De germaanse koninkrijken die Rome opvolgden, werden militair verdedigd
door alle vrije mannen die op oproep van de graaf, de plaatselijke
vertegenwoordiger van de koning, de wapens opnamen. Zelfs ``kleine en
middelgrote grondbezitters'' moesten zich groeperen en mensen uit hun
midden afvaardigen om in de infanterie te dienen. In het edict van
Pîtres beval Karel de Kale alle mannen die daartoe in staat waren om te
paard ten strijde te trekken. Paus Gregorius II had een eeuw voordien
getracht deze militaire verplichting kracht bij te zetten door in 732 de
consumptie van paardenvlees door de mens te verbieden. Toch was er nog
weinig juridisch of statusonderscheid tussen de infanterie van vrije
grondbezitters en de cavalerie. Alle vrije mannen namen deel aan
plaatselijke rechtbanken en konden bij de graaf een verzoek indienen om
geschillen te beslechten, een ambt dat al sinds de late romeinse tijd
bestond. Van adel was er nog geen sprake.

\begin{quote}
`Een sociaal fenomeen dat nieuw was als massaal verschijnsel dook in de
jaren 980 plots op: sociale achteruitgang. De eerste slachtoffers waren
de kleine allodiumbezitters.' - Guy Bois
\end{quote}

In de loop van de Donkere Middeleeuwen raakten de structuren die de
zelfstandigheid van kleine boeren en vrije grondeigenaars in de
Germaanse koninkrijken hadden gewaarborgd, steeds meer uit balans door
een reeks ontwikkelingen:

\begin{enumerate}
\def\labelenumi{\arabic{enumi}.}
\tightlist
\item
  De bevolkingsaantallen herstelden zich geleidelijk, waardoor de druk
  op het gebruik van de grond toenam. In de loop van enkele eeuwen werd
  een groot deel van de meest vruchtbare, nog onontgonnen grond
  gecultiveerd, een groei die zich met name in Noord-Europa voordeed. De
  toename van het aantal boeren in verhouding tot de beschikbare
  hoeveelheid grond deed de waarde van de arbeid van elke boer dalen. De
  meeste vrije stukken land raakten door erfopvolging versnipperd in
  steeds kleinere percelen, omdat het in de vroege Middeleeuwen
  gebruikelijk was om de nalatenschap van ouders gelijk te verdelen over
  de kinderen. De versnippering van het grondbezit in een tijd van
  bevolkingsgroei dreef de waarde van grond opnieuw op en leidde tegen
  het midden van de tiende eeuw tot een heropleving van de actieve
  handel in grond.
\item
  In de laatste decennia van de tiende eeuw daalden de temperaturen
  plotseling, wat een verwoestend effect had op de landbouwopbrengst.
  Drie opeenvolgende misoogsten leidden tussen 982 en 984 tot een
  ernstige hongersnood. Na nog een mislukte oogst in 994 sloeg de
  hongersnood opnieuw toe. In 997 werd het probleem van de dalende
  opbrengsten vervolgens verergerd door een plaag, die kleine
  familiebedrijven bijzonder hard trof omdat de kleine boeren niet over
  de middelen beschikten om de arbeidskracht van weggevallen
  familieleden te vervangen. Door deze opeenstapeling van misoogsten en
  rampen raakten de zelfstandige boeren aanvankelijk diep in de
  schulden. Toen een herstel van de opbrengsten uitbleef, konden ze hun
  schulden niet meer aflossen.
\item
  De machtsverhoudingen raakten steeds verder uit balans door het
  toenemende belang van de zware cavalerie. Frances Gies, een historicus
  gespecialiseerd in de Middeleeuwen, beschrijft hoe de gepantserde
  cavalerist zich ontwikkelde tot de middeleeuwse ridder:
\end{enumerate}

\begin{quote}
Hoewel een ridder oorspronkelijk een figuur van bescheiden komaf was,
die zich door zijn dure paard en harnas boven de boerenstand verhief,
wist hij zijn maatschappelijke positie geleidelijk te verbeteren tot hij
tot de adelstand behoorde. Ridders bleven weliswaar de laagste rang
binnen de aristocratie, maar het ridderschap kreeg een unieke status,
waardoor de ridderslag een eer werd die zelfs door de hoge adel en
koningshuizen werd begeerd. Die status was voornamelijk het resultaat
van het streven van de Kerk om het ridderschap een christelijk karakter
te geven. Dit deed ze door de ridderceremonie te heiligen en een
gedragscode, de ridderlijkheid, te stimuleren. Het was een code die
misschien vaker werd geschonden dan nageleefd, maar die een onmiskenbare
invloed had op het denken en handelen van latere generaties.
\end{quote}

Zoals we in \emph{The Great Reckoning} vertelden, kreeg de bewapende
ridder te paard door de uitvinding van de stijgbeugel een enorm
aanvalsvermogen. Hij kon nu in volle vaart aanvallen zonder uit het
zadel te worden geworpen wanneer hij een doelwit raakte met zijn lans.
De militaire waarde van de zware cavalerie nam verder toe door een
Aziatische uitvinding die in de tiende eeuw in West-Europa haar intrede
deed: het vastgenagelde ijzeren hoefijzer. Dit verbeterde het
uithoudingsvermogen van het paard aanzienlijk. Ook het gevormde zadel,
de spoor en het stangbit droegen bij aan de toegenomen slagkracht van de
ridder. Ze boden meer controle over het paard en maakten het eenvoudiger
om met één hand zware wapens te gebruiken. Deze op het eerste gezicht
kleine technologische verbeteringen maakten kleine boeren op militair
vlak vrijwel irrelevant, omdat ze de middelen niet hadden om een
strijdros te onderhouden of zichzelf goed te bewapenen. De grote paarden
die speciaal voor de strijd werden gefokt, maar goedkoper waren en
bekend stonden als `destriers', waren evenveel waard als vier ossen of
veertig schapen. De duurdere oorlogspaarden kostten tien ossen of
honderd schapen. Ook een harnas kostte een bedrag dat geen enkele kleine
landeigenaar zich kon permitteren, namelijk het equivalent van zestig
schapen.

\begin{enumerate}
\def\labelenumi{\arabic{enumi}.}
\setcounter{enumi}{3}
\tightlist
\item
  Ook het koudere weer, de mislukte oogsten, de hongersnoden en de
  plagen in de aanloop naar het jaar 1000 beïnvloedden het gedrag van de
  mensen. Velen waren ervan overtuigd dat het einde van de wereld of de
  wederkomst van Christus voor de deur stond. Vrome of angstige
  landeigenaren, groot en klein, schonken hun grond aan de Kerk in
  voorbereiding op de apocalyps.
\end{enumerate}

\subsection{`Alleen een arme man verkoopt
grond'}\label{alleen-een-arme-man-verkoopt-grond}

De onrustige omstandigheden aan het einde van de tiende eeuw legden de
basis voor de feodale revolutie. Door opeenvolgende misoogsten en rampen
raakten de zelfstandige boeren diep in de schulden. Toen de oogsten zich
niet herstelden, kwamen de vrije boeren in een uitzichtloze situatie
terecht. Markten leggen altijd de grootste druk op de zwakste partijen.
Dat is zelfs een van hun deugden; ze bevorderen efficiëntie door
eigendom van zwakkere eigenaren af te nemen. Maar in het Europa van de
late tiende eeuw was zelfvoorzieningslandbouw vrijwel het enige beroep.
Families die hun grond verloren, raakten hun enige middel om te
overleven kwijt. Geconfronteerd met dit onaantrekkelijke vooruitzicht,
besloten veel, zo niet de meeste, vrije boeren tijdens de feodale
revolutie hun akkers weg te geven. In de woorden van Guy Bois: `De enige
zekere manier voor een boer om het land dat hij bewerkte te behouden,
was om het eigendom ervan af te staan aan de Kerk, zodat hij het
vruchtgebruik kon behouden.' Anderen stonden al hun grond, of een deel
ervan, af aan rijkere boeren in wie ze vertrouwen hadden, zoals
vriendelijke buren of familieleden.

Deze eigendomsoverdrachten vonden plaats onder de voorwaarde dat de
boer, zijn familie en zijn afstammelingen op de akkers mochten blijven
werken. De arme boeren konden tevens rekenen op de wederzijdse steun van
de meer vermogende landeigenaren, nu de `edelen', die zich een paard en
een harnas konden veroorloven en zo de uitgebreide landgoederen
bescherming boden. Vanuit het oogpunt van de nieuwe lijfeigene kan een
dergelijke overeenkomst worden gezien als een tussenstation tussen het
behoud van economisch eigendom en een gedwongen verkoop. Vaker wel dan
niet was het een aanbod dat hij niet kon weigeren.

De dalende productiviteit bracht de arme boeren niet alleen in een
wanhopig economisch dilemma, maar ontketende ook een golf van
roofzuchtig geweld die de eigendomszekerheid ondermijnde. Wie niet over
de middelen beschikte om een deel van de beschikbare, maar ontoereikende
voorraad paarden en voer te bemachtigen, merkte plotseling dat hij en
zijn bezit niet langer veilig waren. Om hun dilemma in hedendaagse
termen te vatten: het was alsof je vandaag de dag gedwongen werd jezelf
te bewapenen met een nieuw type wapen, maar de kosten daarvan \$100.000
bedroegen. Als je die prijs niet kon betalen, kon je enkel hopen op
genade van degenen die dat wel konden.

Binnen een paar jaar stortte het vermogen van de koning en de
rechtbanken om de orde te handhaven volledig in. Iedereen met een harnas
en een paard kon nu zijn eigen wet stellen. Wat volgde leek op een
middeleeuwse \emph{Blade Runner}, een roerige tijd vol geweld en
rooftochten, terwijl de officiële machthebbers machteloos toekeken.
Plunderingen en aanvallen door gewapende ridders ontwrichtten het
platteland. Het is echter helemaal niet vanzelfsprekend dat alle
slachtoffers van deze plunderingen arm waren. Integendeel, de oudere,
fysiek zwakkere of slecht voorbereide grootgrondbezitters vormden juist
aantrekkelijke doelwitten. Bij hen viel meer te stelen.

Het was geen toeval dat dit gebeurde op het moment dat kouder weer,
hongersnood en de pest de middelen schaars maakten. De megapolitieke
omstandigheden die de ineenstorting van het gezag in de hand werkten,
bestonden al een tijdje. Pas toen er een crisis ontstond, kwam hun
vermogen om de machtsverhoudingen te veranderen tot uiting. Misoogsten
en hongersnoden lijken precies die rol te hebben vervuld. Hoewel de
exacte opeenvolging van gebeurtenissen moeilijk te reconstrueren is,
lijkt het erop dat de plunderingen, althans gedeeltelijk, werden
veroorzaakt door de wanhopige omstandigheden. Zodra het geweld was
losgebarsten, bleek dat niemand de macht kon mobiliseren om het te
stoppen. De overgrote meerderheid van slecht bewapende boeren kon zeker
weinig uitrichten. Een enkele geharnaste ridder te paard was al genoeg
om tientallen boeren te overmeesteren. Vrije boeren konden, net als de
gevestigde autoriteiten, de koningen en hun graven, niets beginnen tegen
de landroof door gewapende strijders.

\subsection{`De godsvrede'}\label{de-godsvrede}

In deze wanhopige omstandigheden droeg de Kerk bij aan het ontstaan van
het feodalisme door een wapenstilstand te proberen te bewerkstelligen op
het gewelddadige platteland. Historicus Guy Bois beschreef de situatie
als volgt: ``De onmacht van de politieke autoriteiten was zodanig dat de
Kerk hun rol overnam in een poging om de orde te herstellen, in de
beweging die bekendstaat als `de godsvrede'\,''. `Vredesconcilies'
vaardigden een reeks verboden uit, op straffe van banvloeken, en in
grote `vredesvergaderingen' legden krijgers hun eden af. De beweging
ontstond in Zuid-Frankrijk (Concilie van Charroux in 989, Concilie van
Narbonne in 990) en verspreidde zich geleidelijk\ldots''

De overeenkomst die de Kerk sloot, hield de erkenning in van de
heerschappij van gewapende ridders in lokale gemeenschappen, in ruil
voor het staken of matigen van het geweld en de plunderingen. Na de
toename van het geweld aan het einde van de tiende eeuw verschenen
titels als `nobilis' en `miles' in eigendomsakten, als teken van gezag.
Daarmee werd de adel als aparte sociale klasse geboren. In oudere akten,
opgesteld voor exact dezelfde personen, ontbraken zulke titels nog
volledig.

Door de dalende productiviteit en de economische onzekerheid van kleine
grondbezitters leidde de megapolitieke macht van gewapende ridders
onvermijdelijk tot grondbezit op basis van feodale leenverhoudingen.
Tegen het einde van het eerste kwart van de elfde eeuw waren de vrije
boeren grotendeels verdwenen. Hun allodiale bezittingen waren geslonken
tot een fractie van hun vroegere omvang en werden nog maar een deel van
de tijd bewerkt. De kleine boeren of hun afstammelingen werden horigen
die de meeste tijd al werkend doorbrachten op de landgoederen van
feodale heren, zowel wereldlijke als kerkelijke.

De teloorgang van de openbare orde die met de feodale revolutie gepaard
ging, leidde tot gedragsveranderingen die het feodalisme versterkten.
Daartoe behoorde een explosieve toename van de kastelenbouw. Kastelen
doken voor het eerst op in Noordwest-Europa als primitieve houten
bouwwerken in de nasleep van de invallen van de Vikingen in de negende
eeuw. Oorspronkelijk waren het commandoposten voor Karolingische
ambtenaren, maar na de feodale revolutie werden ze erfelijk bezit. Deze
vroege bolwerken waren veel primitiever dan hun latere versies, maar
desondanks moeilijk aan te vallen. Eenmaal gebouwd, konden kastelen
slechts met de grootste moeite met de grond gelijkgemaakt worden.
Naarmate kastelen het platteland begonnen te domineren, verloor het
centrale gezag zijn greep: de koning en zijn graven waren nauwelijks nog
in staat de macht van lokale heren te doorbreken.

\subsection{De bijdrage van de Kerk aan de
productiviteit}\label{de-bijdrage-van-de-kerk-aan-de-productiviteit}

Feodalisme ontstond als antwoord op het wegvallen van orde tijdens een
tijd van beperkte productiviteit in de landbouwsamenleving. In het begin
van dit systeem speelde de Kerk een belangrijke en economisch
productieve rol. Enkele bijdragen van de Kerk:

\begin{enumerate}
\def\labelenumi{\arabic{enumi}.}
\tightlist
\item
  In een omgeving waar militaire macht gedecentraliseerd was, bevond de
  Kerk zich in een unieke positie om de vrede te bewaren en orderegels
  te ontwikkelen die versnipperde, lokale soevereiniteiten overstegen.
  Dit was een taak waar geen enkele seculiere macht tegen opgewassen
  was. De observaties van de grote religieuze autoriteit A. R.
  Radcliffe-Brown zijn hier rechtstreeks van toepassing. Hij wees erop
  dat ``de sociale functie van een religie onafhankelijk is van haar
  waarheid of onwaarheid''. Zelfs religies die ``absurd en
  weerzinwekkend zijn, zoals die van sommige primitieve stammen, kunnen
  belangrijke en effectieve onderdelen van het maatschappelijke
  raderwerk zijn''. Dit was zeker het geval voor de Kerk in de vroege
  stadia van het feodalisme. Ze hielp, zoals alleen een religie dat kon,
  bij het opstellen van regels die mensen in staat stelden om perverse
  prikkels en gedragsdilemma's te overwinnen. Sommige hiervan waren
  morele dilemma's die eigen zijn aan het menselijk bestaan. Maar andere
  waren lokale dilemma's, die uniek waren voor de heersende
  megapolitieke omstandigheden. In de laatste jaren van de tiende eeuw
  speelde de middeleeuwse Kerk een bijzondere rol bij het herstellen van
  de orde op het platteland. Door religieuze en ceremoniële steun te
  verlenen aan lokale autoriteiten, verlaagde de Kerk de kosten om op
  zijn minst zwakke, lokale geweldsmonopolies te vestigen. Door op deze
  manier bij te dragen aan de orde, schiep de Kerk de voorwaarden die
  uiteindelijk leidden tot stabielere machtsverhoudingen.
\end{enumerate}

De Kerk bleef nog lange tijd een rol spelen in het beteugelen van
privéoorlogen en buitensporig geweld dat de wereldlijke autoriteiten
anders niet konden indammen. Het relatief grote belang van de Kerk ten
opzichte van de wereldlijke autoriteiten blijkt uit het feit dat tegen
de elfde eeuw de parochie de voornaamste bestuurlijke eenheid werd in
grote delen van West-Europa, en niet langer de civiele indelingen zoals
de \emph{ager} en \emph{pagus} (de omgeving rond een stadje), die sinds
de Romeinse tijd tot in de Donkere Middeleeuwen hadden standgehouden.

\begin{enumerate}
\def\labelenumi{\arabic{enumi}.}
\setcounter{enumi}{1}
\tightlist
\item
  Het was vooral de Kerk die zorg droeg voor het behoud en de
  verspreiding van technische kennis en informatie. De Kerk financierde
  universiteiten en zorgde voor het weinige onderwijs dat de
  middeleeuwse maatschappij genoot. De Kerk bood ook een mechanisme voor
  het vermenigvuldigen van boeken en manuscripten, waaronder vrijwel
  alle toenmalige informatie over landbouw en veeteelt. De scriptoria
  van de benedictijnenkloosters kunnen worden gezien als een
  alternatieve technologie voor de drukpers, die toen nog niet bestond.
  Hoe kostbaar en inefficiënt de scriptoria ook waren, ze vormden in de
  feodale periode praktisch het enige mechanisme om geschreven kennis te
  vermenigvuldigen en te bewaren.
\item
  Mede doordat haar boerderijbeheerders geletterd waren, droeg de Kerk
  sterk bij aan het verhogen van de landbouwproductiviteit in Europa,
  vooral in de vroege fase van het feodalisme. Vóór de dertiende eeuw
  waren de beheerders van wereldlijke heren vrijwel allemaal
  analfabeten, die hun administratie bijhielden met een ingenieus
  stelsel van merktekens. Al waren deze boeren nog zo slim, ze hadden
  niet de mogelijkheid om te profiteren van verbeteringen in
  productiemethoden die ze niet zelf bedachten of met eigen ogen zagen.
  De Kerk was daarom onmisbaar voor het verbeteren van de kwaliteit van
  graan, fruit en fokvee. Omdat haar bezittingen over heel Europa
  verspreid lagen, kon de Kerk hoogwaardig zaaigoed en fokmateriaal
  inzetten waar de productie achterbleef. De vraag naar wijn voor de
  misdiensten in Noord-Europa spoorde monniken aan om te experimenteren
  met druivensoorten die beter bestand waren tegen koudere klimaten. De
  Kerk hielp ook op andere manieren de productiviteit van de
  middeleeuwse landbouw te verhogen. Kleine, slecht bruikbare lapjes
  grond die de Kerk in de feodale periode verwierf, werden vaak opnieuw
  ingedeeld zodat ze makkelijker te bewerken werden. De Kerk voorzag ook
  in de aanvullende diensten die kleine boerengemeenschappen nodig
  hadden. In veel gebieden maalden kerkelijke molens graan tot meel.
\item
  De Kerk nam veel taken op zich die tegenwoordig onder de overheid
  vallen, waaronder het voorzien in openbare infrastructuur. Dit is een
  voorbeeld van hoe de Kerk, in een tijdperk van versnipperd gezag,
  hielp bij het oplossen van wat economen `dilemma's van publieke
  goederen' noemen. Specifieke kloosterordes uit de vroege Middeleeuwen
  legden zich toe op civieltechnische taken, zoals het aanleggen van
  wegen, het herbouwen van ingestorte bruggen en het herstellen van
  vervallen Romeinse aquaducten. Ze ontgonnen ook land, bouwden dammen
  en legden moerassen droog. Een nieuwe kloosterorde, de kartuizers,
  groef de eerste `artesische' put in Artois, Frankrijk. Met behulp van
  klopboren groeven ze een klein gat dat diep genoeg was om een put te
  creëren waarvoor geen pomp nodig was. De cisterciënzerorde nam de bouw
  en het onderhoud van kwetsbare zeeweringen en dijken in de Lage Landen
  op zich. Boeren droegen land over aan cisterciënzerkloosters en
  pachtten het daarna terug, terwijl de monniken de volledige
  verantwoordelijkheid voor het onderhoud en de herstelwerkzaamheden op
  zich namen. Cisterciënzers namen ook het voortouw in de ontwikkeling
  van wateraangedreven machines, die voor uiteenlopende doeleinden
  werden gebruikt, zoals `stampen, heffen, malen en persen'. Het
  klooster van Clairvaux groef een kanaal van twee mijl lang vanaf de
  rivier de Aube. De Kerk faciliteerde ook het aanleggen van nieuwe
  wegen en bruggen op plaatsen waar bevolkingscentra zich buiten het
  bereik van de oude Romeinse garnizoenswegen hadden verplaatst.
  Bisschoppen verleenden aflaten aan lokale heren die rivierovergangen
  bouwden of herstelden en herbergen voor reizigers onderhielden. Een
  monnikenorde, opgericht door de heilige Bénézet, de Frères Pontifes of
  `Broeders van de Brug', bouwde een aantal van de langste bruggen die
  destijds bestonden, waaronder de Pont d'Avignon, een gigantische
  constructie met twintig bogen over de Rhône, met aan één uiteinde een
  kapel die ook als tolhuis diende. Zelfs de London Bridge, die tot de
  negentiende eeuw standhield, werd gebouwd door een kapelaan en werd
  deels gefinancierd met een bijdrage van duizend mark van de pauselijke
  legaat.
\item
  De kerk hielp ook een complexere markt tot ontwikkeling te brengen. De
  bouw van kathedralen, bijvoorbeeld, is van een andere orde dan
  openbare infrastructuur zoals bruggen en aquaducten. In principe
  werden kerkgebouwen alleen voor religieuze diensten gebruikt en niet
  als verkeersaders voor de handel. Toch mag niet worden vergeten dat de
  bouw van kerken en kathedralen hielp om markten voor tal van
  ambachtelijke en technische vaardigheden te creëren en te versterken.
  Net zoals de militaire uitgaven van de natiestaat tijdens de Koude
  Oorlog onbedoeld de voedingsbodem voor het internet vormden, zo leidde
  de bouw van middeleeuwse kathedralen tot andere neveneffecten, zoals
  de bevordering van de handel. De kerk was de voornaamste klant voor de
  bouwsector en ambachtslieden. Kerkelijke aankopen van zilver voor de
  communiediensten, kandelaars en decoratieve kunstwerken, creëerden een
  markt voor luxegoederen die anders niet zou hebben bestaan.
\end{enumerate}

Op vele manieren hielp de Kerk de hevigheid van het geweld te temperen
dat tijdens en na de `feodale revolutie' door gewapende ridders werd
ontketend. Vooral in de eerste eeuwen van het feodalisme leverde de Kerk
een aanzienlijke bijdrage aan het verbeteren van de productiviteit van
de landbouweconomie. Het was een onmisbare instelling, die goed aansloot
bij de behoeften van de agrarische samenleving aan het einde van de
Donkere Middeleeuwen.

\subsection{Kwetsbaarheid voor geweld}\label{kwetsbaarheid-voor-geweld}

Gedurende ``dertig à veertig jaar van gewelddadige onrust'' voltrok zich
rond het jaar 1000 de feodale revolutie\footnote{Bois, op. cit., p.136.}.
Dit was net als de val van Rome vijf eeuwen eerder een uniek gebeuren,
voortgekomen uit een complex samenspel van invloeden. Toch weerspiegelt
de triomf van mali homines (slechte lieden) en de onderdrukking die ze
teweegbrachten in één opzicht perfect de fundamentele kwetsbaarheid voor
geweld van de landbouwsamenleving. Waar de jager-verzamelaarsfase van
het menselijke bestaan relatief vreedzaam was, betekende de overgang
naar landbouw een radicale sprong in gestructureerd geweld en
overheersing.

Vanaf het allereerste begin weerspiegelde zich dit in de militant
ingestelde culturen van de landbouwvolkeren. De goden van de vroege
landbouwsamenlevingen waren goden van regen en overstroming; hun
functies weerspiegelden de grote zorg van die samenlevingen voor de
factoren die de oogstopbrengsten bepaalden. De brenger van regen of
water was vaak ook de oorlogsgod, en werd aangeroepen door de eerste
koningen, die vooral als krijgsheren optraden.

De nauwe band tussen landbouw en oorlogsvoering vond zijn weerslag in de
religieuze verbeelding van de mensen wier levens werden getransformeerd
door de innovaties van de landbouwrevolutie. De verdrijving uit het hof
van Eden kan worden gezien als een beeldend verslag van de overgang van
een jager-verzamelaarssamenleving naar een landbouwsamenleving; een
overgang van een vrij leven, waarin voedsel met weinig moeite uit de
overvloed van de natuur werd geplukt, naar een leven van hard labeur.

\section{Het verloren paradijs}\label{het-verloren-paradijs}

Landbouw stuurde de mensheid een volkomen nieuwe richting uit. De eerste
boeren zaaiden letterlijk de kiem van de beschaving. Uit hun gezwoeg
kwamen steden voort, legers, rekenkunde, astronomie, kerkers, wijn en
whisky, het geschreven woord, koningen, slavernij en oorlog. Maar
ondanks al het drama dat de landbouw het leven zou brengen, lijkt de
stap weg van de oereconomie vanaf het begin enorm impopulair te zijn
geweest. Daarvan getuigt het verhaal uit het boek Genesis, dat vertelt
over de verdrijving uit het paradijs. De bijbelse parabel van het hof
van Eden is een nostalgische herinnering aan het onbezorgde leven van de
jager-verzamelaar in de wildernis. Geleerden menen dat het woord `Eden'
afkomstig is van een Soemerisch woord voor `wildernis'.

De overgang van een vrij en dunbevolkt leven in de wildernis naar een
sedentair bestaan in een boerendorp werd diep betreurd, een spijt die
niet alleen in de Bijbel tot uiting komt, maar ook in de aanhoudende
weerzin van de mens om `s ochtends op te staan en aan het werk te gaan.
Zoals Stephen Boyden schreef in \emph{Western Civilization in Biological
Perspective}, was de nieuwe levenswijze die met de landbouw gepaard ging
'evodeviant'. Vóór de komst van de landbouw leefden duizenden generaties
mensen zoals Adam in het hof van Eden, op uitnodiging van zijn Schepper:
`Van alle bomen in het hof moogt gij vrij eten'. Jagers en verzamelaars
hadden geen gewassen te verbouwen, geen kudde om te hoeden en geen
belastingen te betalen. Net als zwervers trokken jager-verzamelaars waar
ze maar heen wilden, werkten ze weinig en legden ze aan niemand
verantwoording af.

Met de komst van de landbouw begon een nieuwe levenswijze, en wel op
veel dwingendere voorwaarden. `Ook zal ze u doornen en distels
voortbrengen, en gij zult het gewas des velds eten. In het zweet uws
aanschijns zult gij brood eten'. Landbouw was hard werken. De
herinnering aan het leven vóór de landbouw was die van een verloren
paradijs.

Meer dan ze zich ooit hadden kunnen voorstellen, creëerden de boeren
nieuwe omstandigheden die de logica van geweld radicaal veranderden. Het
is geen toeval dat het boek Genesis van Kaïn, de eerste moordenaar, `een
landbouwer' maakt. Het maakt inderdaad deel uit van de wonderbaarlijke
profetische kracht van de Bijbel dat het verhaal werd toevertrouwd aan
herders, die maar al te goed begrepen hoe de landbouw geweld in de hand
werkte. In enkele verzen omvat het bijbelse verslag een logica die
duizenden jaren nodig had om zich te ontvouwen. De landbouw was een
broedplaats voor geschillen. De landbouw creëerde op grote schaal
plaatsgebonden kapitaal, wat de opbrengst van geweld verhoogde en de
bescherming van bezittingen erg bemoeilijkte. Voor het eerst maakte de
landbouw zowel misdaad als bestuur rendabel.

\setsubtitle{Parallellen tussen het seniele verval van de heilige moederkerk en de bemoeizuchtige staat}

\bookmarksetup{startatroot}

\chapter{De laatste dagen van de
politiek}\label{de-laatste-dagen-van-de-politiek}

::: \{.content-hidden when-format=``latex''\} \emph{Parallellen tussen
het seniele verval van de Heilige Moederkerk en de bemoeizuchtige staat}
:::

\begin{quote}
`Ik geloof, en hoop, dat politiek en economie in de toekomst minder
centraal zullen staan dan vroeger. Er komt een tijd waarin de meeste van
onze huidige debatten over deze onderwerpen volstrekt banaal of
betekenisloos zullen lijken, net als de theologische discussies waaraan
de scherpste geesten van de Middeleeuwen hun energie verspilden.' -
ARTHUR C. CLARKE
\end{quote}

Spreken over het naderende einde van de politiek lijkt belachelijk of
optimistisch, afhankelijk van je instelling. Toch is dat waarschijnlijk
wat de Informatierevolutie met zich meebrengt. Voor lezers die zijn
opgegroeid in een eeuw doordrenkt van politiek, lijkt het idee dat het
leven zonder politiek kan verlopen misschien een fantasie. Het is net
alsof je beweert dat iemand kan leven door simpelweg voedingsstoffen uit
de lucht op te nemen. Toch is politiek in de moderne zin, als bezigheid
gericht op het beheersen en rationaliseren van staatsmacht, grotendeels
een moderne uitvinding. Wij geloven dat het zal eindigen met de moderne
wereld, net zoals het kluwen van feodale plichten dat mensen in de
Middeleeuwen bezighield, eindigde met de middeleeuwen. Tijdens de
feodale periode, zoals historicus Martin van Creveld opmerkt, ``bestond
politiek niet (het concept moest nog worden uitgevonden en dateert pas
uit de zestiende eeuw).''

Het idee dat politiek, zoals wij dat nu kennen, vóór de moderne tijd
niet bestond, lijkt misschien verrassend, zeker omdat Aristoteles in de
tijd van Alexander de Grote een essay met die titel schreef. Maar kijk
goed. Woorden in oude teksten zijn niet per se hedendaagse begrippen.
Aristoteles schreef ook een essay getiteld Sofistische Weerleggingen,
een term die tegenwoordig net zo betekenisloos is als ``politiek'' dat
was in de Middeleeuwen. Het woord werd simpelweg niet gebruikt. De
eerste bekende verschijning in het Engels dateert uit 1529. Zelfs toen
leek ``politics'' een negatieve bijklank te hebben, afgeleid van het
Oudfranse woord \emph{politique}, dat werd gebruikt om ``opportunisten
en mensen zonder ruggengraat'' te beschrijven.

Het duurde bijna tweeduizend jaar voordat Aristoteles' sluimerende
concept de betekenis kreeg die we nu kennen. Waarom? Voordat de moderne
wereld het woord van Aristoteles zinvol kon gebruiken, waren er
megapolitieke omstandigheden nodig die de opbrengsten van geweld
drastisch verhoogden. De Buskruitrevolutie, die we analyseerden in
\emph{The Great Reckoning}, zorgde precies daarvoor. De opbrengsten van
geweld bereikten ongekende hoogten. Daardoor werd de vraag wie de staat
controleerde belangrijker dan ooit tevoren. Het was dan ook logisch en
onvermijdelijk dat politiek ontstond uit de strijd om deze vergrote
machtsmiddelen.

Politiek begon vijf eeuwen geleden met de vroege stadia van het
industrialisme. Nu sterft het. Een afkeer van politiek en politici
verspreidt zich over de wereld. Je ziet het in het nieuws en de
speculaties rond de verborgen details van Whitewater en de nauwelijks
verhulde moord op Vincent Foster. Je ziet het in talloze andere
schandalen rond president Bill Clinton. Je ziet het in berichten over de
verduistering van geld door vooraanstaande congresleden via het
postkantoor van het Huis van Afgevaardigden. Je ziet het in schandalen
die tot ontslagen leidden in de kring rond John Major, en soortgelijke
schandalen in Frankrijk, waarbij twee recente premiers betrokken waren:
Édouard Balladur en Alain Juppé. Nog grotere schandalen kwamen aan het
licht in Italië, waar de zevenvoudige premier Giulio Andreotti
terechtstond op beschuldigingen van onder meer banden met de maffia en
het bevel tot de moord op onderzoeksjournalist Mino Pecorelli. Nog
andere schandalen bezoedelden de reputatie van de Spaanse premier Felipe
González. In Japan verloren vier premiers hun functie in de eerste vijf
jaar van de jaren 1990 door corruptiebeschuldigingen. Het Canadese
ministerie van Justitie stelde in een brief aan de Zwitserse
autoriteiten dat voormalig premier Brian Mulroney smeergeld had
ontvangen bij de verkoop van Airbus-vliegtuigen aan Air Canada ter
waarde van 1,8 miljard Canadese dollar. Willy Claes, secretaris-generaal
van de NAVO, moest aftreden onder verdenking van corruptie. Zelfs in
Zweden moest Mona Sahlin, vicepremier en beoogd premier, aftreden wegens
beschuldigingen dat ze overheidskredietkaarten gebruikte voor de aankoop
van luiers en andere huishoudelijke artikelen. Vrijwel overal waar je
kijkt in landen met volwassen verzorgingsstaten die ooit bekend stonden
voor hun goede bestuur, verafschuwen mensen hun politieke leiders.

\subsection{Minachting als leidende
indicator}\label{minachting-als-leidende-indicator}

Morele verontwaardiging over corrupte leiders is geen op zichzelf staand
historisch verschijnsel, maar een veelvoorkomende voorbode van
verandering. Het gebeurt steeds weer wanneer een tijdperk overgaat in
een ander. Zodra technologische vooruitgang de kloof vergroot tussen
oude instituties en nieuwe economische krachten, verschuiven morele
maatstaven, en beginnen mensen de leiders met toenemende minachting te
behandelen. Deze wijdverbreide afkeer wordt zichtbaar ruim voordat er
een samenhangende nieuwe ideologie van verandering ontstaat. Terwijl wij
dit schrijven, is er nog weinig sprake van een duidelijke afwijzing van
politiek. Dat komt later. De meeste van je tijdgenoten hebben zich nog
niet gerealiseerd dat een leven zonder politiek mogelijk is. Wat we in
de laatste jaren van de twintigste eeuw zien, is een niet-gearticuleerde
minachting.

Aan het einde van de vijftiende eeuw speelde zich iets soortgelijks af,
maar toen lag de focus niet op de politiek, maar op de religie. Ondanks
de overtuiging over `de heiligheid van het priesterlijke ambt'
behandelden mensen zowel de hoge als de lage geestelijkheid met grote
minachting, vergelijkbaar met hoe we vandaag de dag tegen politici en
bureaucraten aankijken. Er was een algemene overtuiging dat de hoge
geestelijkheid corrupt, werelds en omkoopbaar was, en dat was niet
zonder reden. Verschillende pausen uit die periode brachten openlijk
buitenechtelijke kinderen voort. De lage geestelijkheid stond nog lager
aangeschreven, omdat zij in overvloed aanwezig was in zowel het
platteland als in de steden. Ze bedelden voor aalmoezen en boden vaak
Gods genade en vergeving van zonden te koop aan voor contant geld.

Onder de `oppervlakkige laag van vroomheid' schuilde een corrupt en
steeds disfunctioneler wordend systeem. Veel mensen verloren al lang hun
respect voor degenen die het bestuurden, nog voordat iemand de moed had
om op te merken dat het systeem niet meer functioneerde. Een leven dat
volledig doordrenkt was van religie, waarin men geen onderscheid maakte
tussen het geestelijke en het wereldse, had al zijn mogelijkheden
opgebruikt. Het einde daarvan was dan ook onvermijdelijk, ruim lang
voordat Luther zijn 95 stellingen op de kerkdeur van Wittenberg
spijkerde.

\section{Een seculiere Reformatie}\label{een-seculiere-reformatie}

Wij zijn ervan overtuigd dat de reactie op de verzadigingspolitiek een
vergelijkbaar traject volgt.

De val van de Sovjet-Unie en de afwijzing van het socialisme maken deel
uit van een alomvattend patroon van depolitisering dat de wereld
teistert. Dit blijkt vooral uit de groeiende minachting voor de
regeringsleiders wereldwijd. Dit gevoel wordt slechts deels gedreven
door het besef dat ze corrupt zijn en geneigd om ``aflaten'' te verkopen
voor politieke problemen, in ruil voor campagnegiften of andere
voordelen om hun persoonlijke financiën te spekken.

De reactie tegen politici wordt ook gedreven door het groeiende besef
dat veel van wat zij tegen hoge kosten doen, zinloos is, net zoals het
organiseren van nog een pelgrimstocht van boetelingen die blootsvoets
door de sneeuw lopen, of het stichten van weer een nieuwe orde
bedelmonniken aan het einde van de vijftiende eeuw. Het droeg weinig bij
aan het verbeteren van de productiviteit of het verlichten van de druk
op de levensstandaard.

\subsection{De laatste dagen van de Heilige
Moederkerk}\label{de-laatste-dagen-van-de-heilige-moederkerk}

Aan het einde van deMmiddeleeuwen was de monolithische Kerk als
instituut verouderd en contraproductief geworden, een duidelijke
verandering ten opzichte van haar positieve economische bijdrage vijf
eeuwen eerder. Zoals we in het vorige hoofdstuk bespraken, speelde de
Kerk aan het einde van de tiende eeuw een leidende rol bij het
herstellen van orde en het bevorderen van economisch herstel na de
anarchie aan het einde van de donkere middeleeuwen. Destijds was de Kerk
onmisbaar voor het voortbestaan van grote aantallen kleine pachters en
horigen, die het merendeel van de West-Europese bevolking vormden. Aan
het einde van de vijftiende eeuw was de Kerk een zware last voor de
productiviteit geworden. De lasten die ze de bevolking oplegde, zetten
de levensstandaard onder druk.

Vrijwel hetzelfde kan vandaag de dag gezegd worden over de natiestaat.
Het ontstaan ervan was een noodzakelijke adaptatie aan de nieuwe
megapolitieke omstandigheden die vijf eeuwen geleden ontstonden door de
buskruitrevolutie. De natiestaat vergrootte de reikwijdte van markten en
verving gefragmenteerde lokale autoriteiten op een moment dat grotere
handelsgebieden hoge opbrengsten voortbrachten. Het feit dat kooplieden
bijna overal in Europa spontaan de zijde van de monarch kozen toen deze
zijn macht probeerde te consolideren, is op zich al veelzeggend bewijs
dat de natiestaat in zijn vroege vorm gunstig was voor de handel. Het
hielp de lasten op de handel verlichten die werden opgelegd door feodale
heren en lokale machthebbers.

In een wereld waarin de opbrengsten van geweld hoog en stijgend waren,
was de natiestaat een nuttig instituut. Maar vijf eeuwen later, nu dit
millennium ten einde loopt, zijn de megapolitieke omstandigheden
veranderd. De opbrengsten van geweld dalen, en de natiestaat is, net als
de Kerk aan het einde van de middeleeuwen, een achterhaalde instelling
geworden die groei en productiviteit belemmert.

Net zoals de middeleeuwse Kerk destijds, heeft de hedendaagse natiestaat
haar mogelijkheden uitgeput. Ze verkeert in faillissement en is
uitgegroeid tot een seniel systeem. Vijf eeuwen lang domineerde ze als
de overheersende vorm van sociale organisatie, maar nu de omstandigheden
die tot haar ontstaan leidden niet langer bestaan, is ze rijp voor de
val, en die is onvermijdelijk. Technologie ontketent een revolutie in de
uitoefening van macht die de natiestaat zonder twijfel zal vernietigen,
net zoals buskruitwapens en de drukpers ooit het middeleeuwse
kerkmonopolie deden instorten.

Als onze redenering klopt, zal de natiestaat worden vervangen door
nieuwe vormen van soevereiniteit, sommige uniek in de geschiedenis,
andere zullen doen denken aan de stadstaten en middeleeuwse
handelsrepublieken van de premoderne wereld. Wat oud was, zal na het
jaar 2000 weer nieuw zijn. En wat ooit onvoorstelbaar was, zal alledaags
worden. Naarmate technologie in schaal afneemt, zullen overheden merken
dat zij net als bedrijven moeten concurreren om inkomsten, waarbij zij
geen hogere prijs meer kunnen vragen voor hun diensten dan wat deze
waard zijn voor degenen die ervoor betalen. De volledige gevolgen van
deze verandering zijn vrijwel niet te voorzien.

\section{Toen en nu}\label{toen-en-nu}

Iets soortgelijks had vijfhonderd jaar geleden gezegd kunnen worden,
rond het begin van de vijftiende eeuw. Net als nu stond de westerse
beschaving toen aan de vooravond van een ingrijpende transformatie.
Hoewel bijna niemand het wist, was de middeleeuwse samenleving aan het
sterven. Haar ondergang werd noch breed voorzien, noch begrepen. Toch
was de heersende stemming er een van diepe somberheid. Dat is
gebruikelijk aan het einde van een tijdperk, wanneer conventionele
denkers aanvoelen dat alles uit elkaar valt, dat `de valk de valkenier
niet meer hoort.' Maar hun mentale inertie is vaak te groot om de
gevolgen van de opkomende machtsstructuren te begrijpen. De middeleeuwse
historicus Johan Huizinga schreef over de laatste dagen van de
Middeleeuwen : `De kroniekschrijvers van de vijftiende eeuw waren
vrijwel allemaal het slachtoffer van een volkomen miskenning van hun
tijd, waarvan de werkelijke drijvende krachten aan hun aandacht
ontsnapten.'

\subsection{Mythen verraden}\label{mythen-verraden}

Grote veranderingen in de onderliggende machtsdynamiek brengen
conventionele denkers vaak in verwarring, omdat ze de mythen ontmaskeren
die het oude systeem rechtvaardigen maar geen echte verklaringskracht
hebben. Aan het einde van de middeleeuwen, net als nu, was er een
bijzonder grote kloof tussen de gangbare mythen en de werkelijkheid.
Zoals Huizinga zei over de Europeanen in de late vijftiende eeuw:
``Alles in hun denkwijze draaide om de fictie dat de wereld werd
bestuurd volgens de idealen van de ridderlijkheid.'' Dat lijkt sterk op
de hedendaagse veronderstelling dat de wereld wordt geregeerd door
stemmen en populariteitswedstrijden. Geen van de twee overtuigingen
blijkt steek te houden als je ze zorgvuldig bekijkt. De gedachte dat het
verloop van de geschiedenis wordt bepaald door democratische stemrondes
is net zo absurd als het middeleeuwse idee dat dit gebeurt op basis van
een verfijnde gedragscode die ridderlijkheid heet.

Dat zo'n uitspraak bijna als ketterij wordt bestempeld, toont hoe ver
het conventionele denken verwijderd is van een realistisch begrip van de
machtdynamiek in de laat-industriële samenleving. Het is een onderwerp
dat we in dit boek grondig onderzoeken. Naar onze mening was stemmen een
gevolg, geen oorzaak, van de megapolitieke omstandigheden die de moderne
natiestaat voortbrachten. Massademocratie en het burgerschapsbegrip
bloeiden op met de groei van de natiestaat. Ze zullen verzwakken
naarmate de natiestaat verzwakt, en dat zal in Washington net zoveel
ontzetting veroorzaken als het verval van de ridderlijkheid vijfhonderd
jaar geleden aan het hof van de hertog van Bourgondië.

\section{Parallellen tussen ridderlijkheid en
burgerschap}\label{parallellen-tussen-ridderlijkheid-en-burgerschap}

Als je begrijpt hoe en waarom het belang van ridderlijke eden verdween
bij de overgang naar een industriële samenleving, zul je beter kunnen
inzien hoe burgerschap zoals we dat nu kennen zou kunnen verdwijnen in
het Informatietijdperk. Beide dienden een vergelijkbare functie: ze
maakten de uitoefening van macht mogelijk onder twee totaal
verschillende megapolitieke omstandigheden.

Feodale eden overheersten in een tijd waarin defensieve technologie de
overhand had, soevereiniteit gefragmenteerd was, en zowel individuen als
corporaties zelfstandig militaire macht uitoefenden. Vóór de
Buskruitrevolutie werden oorlogen meestal uitgevochten door kleine
groepen gewapende mannen. Zelfs de machtigste vorsten beschikten niet
over \emph{militum perpetuum}, ofwel staande legers. Zij baseerden hun
militaire macht op hun vazallen, de hoge edelen, die op hun beurt
steunden op hun eigen vazallen, de lage edelen, die op hun beurt weer
steunden op hun vazallen, de ridders. Deze hele keten van trouw strekte
zich uit over de hele hiërarchie, tot aan de laagste sociale klasse die
nog als waardig werd beschouwd om wapens te dragen.

\subsection{Uniformen of afwijkingen?}\label{uniformen-of-afwijkingen}

In tegenstelling tot een modern leger trok een middeleeuws leger vóór de
opkomst van het burgerschap niet het slagveld op in uniformen.
Integendeel, iedere vazal of leenman, elke ridder, baron of heer droeg
een uniek uniform dat zijn rang en positie in het feodale systeem
symboliseerde. In plaats van uniformiteit waren er juist verschillen die
de verticale structuur van de samenleving benadrukten, waarin elke
positie uniek was. Zoals Huizinga zei, werden middeleeuwse krijgers
gekenmerkt door ``uiterlijke tekenen van \ldots{} verschillen: livreien,
kleuren, emblemen, strijdkreten.''

Oorlogen werden ook niet uitsluitend gevoerd door regeringen of naties.
Zoals Martin van Creveld opmerkt, geven moderne opvattingen over oorlog,
gestileerd door strategen zoals Carl von Clausewitz, een verkeerd beeld
van premoderne conflicten. Van Creveld schrijft:

\begin{quote}
Duizend jaar na de val van Rome werd een gewapend conflict gevoerd door
verschillende soorten sociale entiteiten. Onder hen bevonden zich
barbaarse stammen, de Kerk, feodale baronnen van alle rangen, vrije
steden en zelfs particuliere individuen. En de ``legers'' van die
periode waren totaal niet te vergelijken met de legers die we
tegenwoordig kennen; het is immers moeilijk een woord te vinden dat
recht doet aan hun aard. Oorlog werd gevoerd door zwermen dienaren die
in militaire kledij verschenen en hun heer volgden.
\end{quote}

Onder zulke omstandigheden was het voor de heer van groot belang dat
zijn vazallen daadwerkelijk ``hun militaire kleding aantrokken en
volgden.'' Vandaar de grote nadruk op de ridderlijke eed.

De eer van de middeleeuwse ridder en de plicht van de dienstplichtige
soldaat vervulden vergelijkbare functies. De middeleeuwer was door eden
gebonden aan individuen en de Kerk, net zoals moderne mensen door
burgerschap aan de natiestaat gebonden zijn. Het breken van een eed was
in de Middeleeuwen gelijk aan hoogverraad. Mensen in de late
Middeleeuwen gingen erg ver om het breken van hun eden te vermijden, net
zoals dat miljoenen moderne burgers in de Wereldoorlogen onder vuur van
machinegeweren vijandelijke posities bestormden om hun plicht als
burgers te vervullen.

Zowel ridderlijkheid als burgerschap voegden een extra dimensie toe aan
de eenvoudige afweging die niet-geïndoctrineerde mensen anders ervan zou
weerhouden het slagveld op te gaan en daar te blijven wanneer het zwaar
werd. Zowel ridderlijkheid als burgerschap brachten mensen ertoe te
doden en hun leven te riskeren. Alleen veeleisende en overdreven
waarden, krachtig versterkt door invloedrijke instituties, kunnen die
functie vervullen.

\subsection{Het omzeilen van de
kosten-batenanalyse}\label{het-omzeilen-van-de-kosten-batenanalyse}

Het succes en voortbestaan van elk systeem hangt af van het vermogen om
militaire inspanning te mobiliseren in tijden van conflict en crisis. De
beslissing van een middeleeuwse ridder of een soldaat in de loopgraven
van de Eerste Wereldoorlog om zijn leven in de strijd te riskeren, was
duidelijk niet gebaseerd op een nuchtere kosten-batenanalyse. Zelden
worden oorlogen zo gemakkelijk uitgevochten, of wegen de beloningen voor
degenen die het zware werk doen zo sterk op tegen de mogelijke kosten,
dat een leger van economische opportunisten kan worden gerekruteerd om
het slagveld te betreden. Bij vrijwel elke oorlog en de meeste gevechten
zijn er momenten waarop het tij in een oogwenk kan keren. Militaire
historici weten dat het verschil tussen nederlaag en overwinning vaak
bepaald wordt door de moed, dapperheid en felheid waarmee individuele
soldaten hun taak uitvoeren. Als de strijders niet bereid zijn te
sterven voor een stuk grond dat na de strijd niets waard is, zullen ze
waarschijnlijk niet zegevieren tegen een in andere opzichten
gelijkwaardige vijand.

Dit heeft grote consequenties. Een soeverein die desertie goed kan
tegenhouden en zijn troepen inzetbaar houdt, vergroot aanzienlijk de
kans op het winnen van oorlogen. In een oorlog zorgen de meest
effectieve waardensystemen ervoor dat mensen zich gedragen op manieren
die een rationele berekening uitsluit. Geen enkele organisatie kan
militaire macht effectief mobiliseren als de individuen die ze naar het
slagveld stuurt vrij zijn om hun eigen voordeel te berekenen en
vervolgens mee te vechten of weg te lopen. In dat geval zouden ze
vrijwel nooit vechten. Alleen onder de meest gunstige of meest wanhopige
omstandigheden zou een rationeel persoon zich op basis van een
kortetermijnkosten-batenanalyse in een potentieel dodelijk gevecht
wagen. Misschien vecht Homo economicus op een zonnige dag, als de eigen
troepen overweldigend zijn, de vijand zwak, en de potentiële beloningen
van de strijd verleidelijk. Misschien. Hij zou ook kunnen vechten als
hij in een hoek wordt gedreven door rondzwervende kannibalen.

Maar dat zijn extreme omstandigheden. Hoe zit het met de meer
gebruikelijke oorlogsomstandigheden die noch aantrekkelijk genoeg zijn
voor een rationele afweging, noch zo uitzichtloos dat men geen andere
keuze heeft? Hier spelen concepten als ridderlijkheid en burgerschap een
cruciale rol bij het succesvol inzetten van militaire macht. Ver voordat
een gevecht begint, moeten dominante organisaties individuen overtuigen
dat het naleven van bepaalde plichten aan de heer of de natiestaat
belangrijker is dan het eigen leven. De mythen en rechtvaardigingen die
samenlevingen gebruiken om risico's op het slagveld te stimuleren, zijn
een essentieel onderdeel van hun militaire kracht.

Om effectief te kunnen zijn, moeten deze mythen aansluiten bij de
heersende geopolitieke omstandigheden. Het idee dat ridderlijkheid de
wereld regeert, heeft tegenwoordig weinig betekenis, zeker niet in een
stad als New York. Maar in feodale tijden was het juist de gekoesterde
mythe van het feodalisme. Het rechtvaardigde en verklaarde de plichten
die mensen met elkaar verbonden onder de overheersing van de Kerk en een
oorlogszuchtige adel. In een periode waarin oorlogen, voortkomend uit
hebzucht, de norm waren, hing het uitoefenen van macht en het
voortbestaan van individuen af van de bereidheid van anderen om hun
beloften tot militaire dienst, vaak onder dwang, na te komen. Het was
natuurlijk cruciaal dat die beloften betrouwbaar waren.

\subsection{Vóór nationaliteit}\label{vuxf3uxf3r-nationaliteit}

In de Middeleeuwen was nationaliteit geen bepalende factor voor
soevereiniteit. Vorsten, bisschoppen en edelen beheerden hun gebieden
als privébezit. Op een manier die geen modern equivalent kent, konden
deze heren territoria verkopen of wegschenken, nieuwe verwerven via
overdracht of huwelijk, of door oorlogvoering. Tegenwoordig kun je je
nauwelijks voorstellen dat de Verenigde Staten onder de soevereiniteit
zouden vallen van een niet-Engelssprekende Portugese president, alleen
omdat hij toevallig trouwde met de dochter van de voormalige Amerikaanse
president. Toch was iets dergelijks in het middeleeuwse Europa heel
gewoon. De macht ging over door erfelijke opvolging. Steden en landen
wisselden van machthebber zoals antiek van eigenaar wisselt. In veel
gevallen kwamen machthebbers niet uit de gebieden waar hun eigendommen
zich bevonden. Soms spraken ze de lokale taal niet, of slecht met een
zwaar accent. Maar dat maakte weinig verschil voor de persoonlijke
plichten. Het deed er niet toe of een Spanjaard koning van Athene was,
of een Oostenrijker koning van Spanje.

\subsection{Corporatieve
soevereiniteit}\label{corporatieve-soevereiniteit}

Soevereiniteit werd ook uitgeoefend door religieuze corporaties zoals de
Tempeliers, de Orde van Sint-Jan en de Duitse Orde. Deze hybride
instellingen kennen geen moderne tegenhangers. Ze combineerden
religieuze, sociale, gerechtelijke en financiële functies met
soevereiniteit over bepaalde gebieden. Hoewel zij territoriale
jurisdictie uitoefenden, waren ze bijna het tegenovergestelde van
hedendaagse overheden, omdat nationaliteit geen rol speelde in de
mobilisatie van hun steun of in hun bestuursstructuur. De leden en
officieren van deze religieuze orden kwamen uit alle delen van het
christelijke Europa, ofwel het toenmalige ``Christenheid''.

Men vond het niet noodzakelijk dat de heersers uit de lokale bevolking
afkomstig waren. In het gefragmenteerde bestuursmodel van de
Middeleeuwen hing het verkrijgen van steun niet af van een nationale
identiteit of een plicht aan de staat, zoals tegenwoordig vaak het geval
is, maar van persoonlijke loyaliteit en traditionele banden die als
kwestie van persoonlijke eer hoog in het vaandel stonden. Iedereen kon
deze eden afleggen, ongeacht zijn afkomst, mits men op basis van zijn
sociale positie als waardig werd geacht.

\subsection{De eed}\label{de-eed}

Ridderlijke eden verbonden mensen met elkaar en werden afgelegd op grond
van persoonlijke eer. Zoals Huizinga schreef: `Door een eed af te
leggen, legden mensen zichzelf een zekere ontzegging op als aansporing
tot het verrichten van de handelingen waartoe ze zich hadden verplicht.'
Men hechtte zoveel belang aan het nakomen van eden dat mensen vaak hun
leven riskeerden of zware consequenties ondervonden om te voorkomen dat
zij de eed verbraken. Vaak verplichtte de eed de betrokkenen tot
specifieke handelingen uit eer, handelingen die jij en de meeste lezers
van dit boek waarschijnlijk als belachelijk zullen ervaren.

Zo zwoeren de Ridders van de Ster bijvoorbeeld een eed nooit meer dan
vier hectare van het slagveld terug te trekken, een regel die er spoedig
toe leidde dat meer dan negentig van hen het leven verloren. Dit verbod
op zelfs een tactische terugtocht is irrationeel vanuit militair
oogpunt, maar was een veelvoorkomend gebod in ridderlijke eden. Vóór de
Slag bij Azincourt gaf de koning van Engeland het bevel dat
patrouillerende ridders hun harnas moesten afleggen, omdat het
onverenigbaar met hun eer zou zijn zich terug te trekken terwijl ze hun
wapenrusting droegen. De koning zelf raakte verdwaald en passeerde het
dorp waar de voorhoede van zijn leger de nacht doorbracht. Aangezien hij
zijn harnas droeg, verbood zijn ridderlijke eer hem om eenvoudigweg om
te keren toen hij zijn vergissing ontdekte. Hij bracht de nacht door op
een onbeschutte plek.

Hoe absurd dit voorbeeld ook lijkt, koning Hendrik had waarschijnlijk
terecht ingeschat dat hij meer risico zou lopen als hij zich terug zou
trekken en zijn eer zou schenden, dan als hij achter vijandelijke linies
zou overnachten. Het zou een demoraliserend signaal naar zijn hele leger
afgeven.

De middeleeuwse geschiedenis staat vol met voorbeelden van prominente
figuren die geloften nakwamen die voor ons absurd zouden lijken. Vaak
hadden deze daden geen enkel objectief voordeel, behalve het krachtig
tonen van de waarde die men aan de eed hechtte. Veelvoorkomende geloften
waren: één oog gesloten houden, alleen staand eten en drinken, of
zichzelf verminken door zich vrijwillig met ketens te boeien. Het dragen
van pijnlijke voetkettingen was wijdverbreid. In onze tijd zou iemand
die op straat rondloopt met een zware ketting aan zijn been vooral
gestoord gevonden worden, en geen bewondering voor zijn morele karakter
oproepen. In de ridderlijke context gold het echter als een ereteken. Er
waren talloze soortgelijke gebruiken. Zoals Huizinga beschrijft: velen
beloofden ``niet in een bed te slapen op zaterdag, geen vlees te eten op
vrijdag, enz. De zelfkastijding stapelde zich op: een edelman belooft
geen harnas te dragen, één dag per week geen wijn te drinken, niet in
een bed te slapen, niet zittend te eten, een boetekleed te dragen.''

Vasten is een gematigde voortzetting van deze zelfopgelegde ontberingen.
Fanatiekelingen richtten vaak orden op die hun leden zware ontberingen
oplegden als eerproef. Zo kleedden leden van de Orde van Clalois en
Galoises zich `s zomers in bont en bontgevoerde kappen en stookten ze
vuur in de haard, terwijl ze 's winters slechts een eenvoudige jas
zonder bont mochten dragen, geen mantels, hoeden of handschoenen, en
enkel lichte bedlakens hadden. Zoals Huizinga opmerkt: 'Het is niet
verwonderlijk dat veel leden aan kou overleden.'

\begin{quote}
`Middeleeuwse zelfkastijding was een gruwelijke marteling die mensen
zichzelf aandeden in de hoop dat een oordelende en straffende God zijn
roede weg zou leggen, hun zonden zou vergeven en hen zou sparen voor de
grotere kastijdingen die hen anders in deze wereld en de volgende zouden
treffen.' - NORMAN COHN
\end{quote}

\subsection{Zelfkastijding, toen en nu}\label{zelfkastijding-toen-en-nu}

Van geloften die gevaar en ontbering oplegden, was het slechts een
kleine stap naar beproevingen, pelgrimstochten, zelfkastijding, ongemak
en zelfs opzettelijke zelfverwonding. In de Middeleeuwen werden deze
gezien als zeer waardevol en prijzenswaardig. Zulke gebaren toonden de
diepe toewijding aan een gelofte, een manier van denken die nog steeds
terug te vinden is in ontgroeningsrituelen van fraterniteiten of
studentenclubs.

Stikken van de hitte in de zomer, bevriezen in de winter, of blootsvoets
op pelgrimstocht door de sneeuw was relatief mild vergeleken met ``de
gruwelijke marteling'' van zelfkastijding. Deze typisch middeleeuwse
vorm van boetedoening ontstond vrijwel gelijktijdig met het begin van
het feodalisme. Het werd voor het eerst toegepast door kluizenaars in de
kloostergemeenschappen van Camaldoli en Fonte Avellana, aan het begin
van de elfde eeuw.

Flagellanten liepen niet alleen maar blootsvoets door de kou, maar
organiseerden processies waarbij ze dag en nacht van de ene stad naar de
andere trokken. ``En telkens wanneer ze een stad binnenkwamen, stelden
ze zich in groepen op voor de Kerk en geselden zichzelf urenlang.''

Wanneer mensen later terug kijken op het tijdperk van de natiestaat,
verwachten wij dat ze sommige handelingen in naam van burgerschap uit de
twintigste eeuw net zo absurd zullen vinden als dat wij zelfkastijding
nu vinden. Vanuit het perspectief van de informatiesamenleving zal het
schouwspel van soldaten die in de moderne tijd de halve wereld over
reizen om de dood te trotseren uit loyaliteit aan de natiestaat, als
grotesk en dwaas worden beschouwd. Het zal niet veel verschillen van
sommige buitengewone en overdreven ridderlijke rituelen, zoals rondlopen
met beenijzers, waar anderszins verstandige mensen in de feodale tijd
trots op waren.

\subsection{Ridderlijkheid maakt plaats voor
burgerschap}\label{ridderlijkheid-maakt-plaats-voor-burgerschap}

Toen de megapolitieke omstandigheden veranderden en de militaire functie
van de eed van trouw aan een heer achterhaald raakte, verdween
Ridderlijkheid en maakte het ruimte voor burgerschap. Het tijdperk van
buskruitwapens en industriële legers bracht totaal andere verhoudingen
tussen de strijders en hun bevelhebbers met zich mee. Burgerschap kwam
voort uit een periode waarin geweld steeds winstgevender werd en de
staat over veel meer middelen beschikte dan middeleeuwse oorlogvoerende
maatschappelijke entiteiten. Dankzij haar overweldigende macht en
rijkdom kon de natiestaat rechtstreeks onderhandelen met de massa
soldaten in haar leger.

Deze overeenkomsten waren voor de staat veel goedkoper en minder
problematisch dan de onderhandelingen met machtige heren en lokale
notabelen, die eisen die tegen hun belangen ingingen konden weigeren,
iets wat individuele burgers in de natiestaat niet konden.

Om redenen die we later uitgebreider behandelen, was burgerschap totaal
afhankelijk van het feit dat geen enkel individu of kleine groep,
megapolitiek gesproken, zelfstandig militaire macht kon uitoefenen.
Naarmate informatietechnologie de logica van oorlogsvoering verandert,
zullen de mythen rond burgerschap net zo onherroepelijk verouderen als
buskruit ooit de middeleeuwse ridderlijkheid overbodig maakte.

\subsection{Hell's Angels te paard}\label{hells-angels-te-paard}

De ruiteraristocratie die West-Europa eeuwenlang domineerde, verschilde
sterk van het soort heren dat hun nazaten later werden. Ze waren ruw en
gewelddadig. Een soort van middeleeuwse tegenhanger van motorbendes. De
etiquette en schijn van ridderlijkheid dienden eerder om hun
buitensporigheden te temperen dan om hun werkelijke gedrag te
beschrijven. Zelfs een uitgebreid overzicht van ridderlijke regels en
verplichtingen zou weinig duidelijkheid geven over de werkelijke bron
van de adellijke macht.

\subsection{Perfectie als synoniem voor
uitputting}\label{perfectie-als-synoniem-voor-uitputting}

De opkomst van effectieve buskruitwapens aan het einde van de vijftiende
eeuw blies de aristocratie van ridders omver, net toen zij hun
krijgskunst tot in de perfectie hadden ontwikkeld. Dankzij zorgvuldig
fokken was er toen een strijdros van zestien handen hoog, groot genoeg
om een volledig gepantserde ridder comfortabel te dragen. Maar zoals C.
Northcote Parkinson scherp opmerkte: ``perfectie wordt alleen bereikt
door instellingen die op instorten staan.'' Net toen het nieuwe
strijdros perfect was, werden buskruitwapens ingezet die paard en ridder
van het slagveld bliezen. Deze nieuwe wapens konden door gewone mensen
worden gebruikt. Ze vereisten weinig vaardigheid maar waren duur om in
grote aantallen aan te schaffen. Hun verspreiding verhoogde geleidelijk
het belang van handel ten opzichte van landbouw, watde basis was van de
feodale economie.

\subsection{Oorlog op grotere schaal}\label{oorlog-op-grotere-schaal}

Hoe veroorzaakten vuurwapens zo'n transformatie? Ten eerste vergrootten
ze de schaal van gevechten, waardoor oorlog voeren al snel veel duurder
werd dan in de middeleeuwen. Vóór de Buskruitrevolutie vochten legers
meestal met zulke kleine groepen dat ze uit een klein en arm gebied
konden worden gerekruteerd. Buskruit gaf een voordeel bij grotere
oorlogen. Alleen leiders met rijke onderdanen konden onder de nieuwe
omstandigheden effectieve strijdkrachten op de been brengen. Leiders die
het beste gebruik maakten van de toenemende handel, meestal vorsten die
een verbond sloten met stedelijke kooplieden, hadden een
concurrentievoordeel op het slagveld. In woorden van van Creveld:
``Deels dankzij hun superieure financiële middelen konden zij meer
kanonnen kopen dan wie dan ook en de tegenstander aan flarden
schieten.''

Pas eeuwen later zouden buskruitwapens hun volledige effect hebben in de
burgerlegers van de Franse Revolutie, maar al in de Renaissance toonde
de adoptie van militaire uniformen een vroege verandering in de
oorlogsvoering. De uniformen symboliseren treffend de nieuwe relatie
tussen krijger en natiestaat, die samenhing met de overgang van
ridderschap naar burgerschap. In feite sloot de nieuwe natiestaat een
``uniforme'' overeenkomst met haar burgers, in tegenstelling tot de
speciale, uiteenlopende verdragen die de vorst of paus sloot met een
lange keten vazallen onder het feodalisme. In het oude systeem had
iedereen een eigen plek in een hiërarchische opbouw. Iedereen had een
overeenkomst zo uniek als zijn familiewapen en de kleurrijke vaandels
die hij voerde.

\subsection{Het verlagen van de opportuniteitskosten van
rijkdom}\label{het-verlagen-van-de-opportuniteitskosten-van-rijkdom}

Vuurwapens veranderden de samenleving fundamenteel op nog een manier. Ze
scheidden macht van fysieke kracht, waardoor de opportuniteitskosten van
handel daalden. Rijke kooplieden hoefden voor hun verdediging niet
langer op hun eigen vaardigheid en kracht in een gevecht te vertrouwen,
of op onbetrouwbare huurlingen. Ze konden rekenen op bescherming door de
nieuwe, grotere legers van de grote vorsten. Zoals William Playfair over
de Middeleeuwen zei: ``Toen er bij vijandigheden werd gedreigd met
menselijke kracht, was het onmogelijk om lang tegelijk rijk en machtig
te zijn.'' Toen buskruit kwam, werd het onmogelijk om machtig te zijn
zonder ook rijk te zijn.

\subsection{Status en statisch begrip}\label{status-en-statisch-begrip}

Net zoals dat de meeste mensen tegenwoordig niet voorbereid zijn op de
veranderende dynamiek van de Informatiemaatschappij, bleven de
vooraanstaande denkers in de Middeleeuwen achter bij het voorspellen en
doorgronden van de opkomst van de handel, die een cruciale rol speelde
in de vorming van de moderne tijd. Vijf eeuwen geleden zagen de mensen
hun snel veranderende samenleving als iets statisch. Zoals Huizinga
opmerkte: ``Zeer weinig eigendom was liquide, in de moderne zin, terwijl
macht nog niet hoofdzakelijk met geld werd geassocieerd. Het was eerder
inherent aan de persoon en berustte op een soort religieus ontzag dat
hij inboezemde. Het drukte zich uit in pracht en praal of een grote
groep trouwe volgers. Grootsheid in het feodale of hiërarchische denken
uitte zich via zichtbare tekenen\ldots.'' Omdat men in de late
Middeleeuwen vooral aan status dacht, zagen zij niet in dat kooplieden
een belangrijke bijdrage konden leveren aan het functioneren van het
rijk. Kooplieden behoorden vrijwel altijd tot de laagste van de drie
standen, onder de adel en de geestelijkheid.

Zelfs de meest vooruitziende denkers van die tijd erkenden niet dat
handel en ander ondernemerschap buiten de landbouw een wezenlijke bron
van rijkdom konden zijn. Voor hen was armoede een deugd. Ze maakten
letterlijk geen onderscheid tussen een vermogende bankier en een
bedelaar. Zoals Huizinga verwoordde: ``Er werd in de derde stand in
principe geen onderscheid gemaakt tussen rijke en arme burgers, noch
tussen stadsbewoners en plattelandsmensen.'' In hun opvatting deden
beroep en rijkdom er niet toe; alleen de ridderlijke status telde.

Deze blindheid voor de economische dimensie van het leven werd ook
versterkt door de geestelijken, de ideologische hoeders van de
middeleeuwse samenleving. Zij schatten het belang van handel zo laag in
dat ze in de vijftiende eeuw een breed geprezen hervormingsprogramma
voorstelden dat alle niet-adellijke personen ertoe bonden zich
uitsluitend op ambachtelijk werk of landbouw te richten. Handel kreeg
werkelijk geen enkele ruimte.

\begin{quote}
`Het jaar 1492, traditioneel gebruikt om de Middeleeuwen van de moderne
geschiedenis te scheiden, is net zo geschikt als elk ander
scheidingspunt, want vanuit wereldhistorisch perspectief symboliseert
Columbus' reis het begin van een nieuwe relatie tussen West-Europa en de
rest van de wereld.' - FREDERIC C. LANE
\end{quote}

\section{De geboorte van het Industriële
Tijdperk}\label{de-geboorte-van-het-industriuxeble-tijdperk}

Veel van de scherpzinnigste geesten van de vijftiende eeuw misten
volledig een van de belangrijkste ontwikkelingen uit de geschiedenis,
terwijl die zich recht onder hun ogen voltrok. De neergang van het
feodalisme markeerde het begin van de grote moderne fase van de Westerse
dominantie. Het was een periode waarin geweld steeds meer loonde en de
schaal van ondernemingen toenam. De moderne economie heeft in de landen
die er gebruik van maakten, gedurende de laatste twee en een halve eeuw,
een ongeëvenaarde stijging van de levensstandaard opgeleverd. De
aanjagers van deze veranderingen waren nieuwe technologieën, van
vuurwapens tot de drukpers, die de grenzen van het leven veranderden op
manieren die maar weinigen konden bevatten.

Tegen het laatste decennium van de vijftiende eeuw begonnen
ontdekkingsreizigers zoals Columbus net de toegang te openen tot
uitgestrekte, onbekende continenten. Voor het eerst in de eeuwenoude
menselijke geschiedenis werd de hele wereld in kaart gebracht.
Galjoenen, nieuwe varianten op de mediterraanse galei maar met hoge
masten, voeren rond de wereld en brachten routes in kaart die
handelswegen en paden voor ziekteverspreiding en verovering zouden
worden. Conquistadores, gewapend met hun nieuwe bronzen kanonnen,
bliezen op zee en land nieuwe horizonten open. Ze vonden rijkdommen in
goud en specerijen, plantten de zaden van nieuwe handelsgewassen zoals
tabak en aardappelen, en eigenden zich nieuw graasland toe voor hun vee.

\subsection{De eerste industriële
technologie}\label{de-eerste-industriuxeble-technologie}

Net zoals het kanon nieuwe economische mogelijkheden met zich meebracht,
opende de boekdrukkunst de deur naar een geheel nieuw intellectueel
tijdperk. De drukpers functioneerde als de eerste massaproductiemachine
en betekende daarmee het begin van het industrialisme. Hiermee
onderschrijven we het standpunt dat Adam Smith in \emph{`The Wealth of
Nations'} naar voren bracht, namelijk dat de industriële revolutie al
gaande was lang voordat hij schreef. Hoewel het systeem nog niet
volgroeid was, lagen de fundamenten van massaproductie en het
fabriekssysteem al stevig verankerd. Zijn beroemde voorbeeld van de
speldenfabrikanten illustreert dit treffend: Smith legt uit dat men
achttien afzonderlijke handelingen toepaste bij de productie van
spelden. Dankzij de gespecialiseerde technologie en arbeidsverdeling
maakte elke werknemer in één dag wel 4.800 keer zoveel spelden als
wanneer hij het op eigen kracht had moeten doen.

Smiths voorbeeld maakt duidelijk dat de industriële revolutie al eeuwen
eerder begon dan historici doorgaans aannemen. De meeste leerboeken
situeren het begin ervan in het midden van de achttiende eeuw, wat geen
onredelijke datum is om het begin van de stijgende levensstandaard aan
te duiden. In werkelijkheid begon de megapolitieke transitie van het
feodalisme naar het industrialisme echter al aan het einde van de
vijftiende eeuw. De impact ervan had vrijwel onmiddellijk gevolgen voor
de heersende instituties, vooral merkbaar in de snel afnemende invloed
van de middeleeuwse Kerk.

Historici die de industriële revolutie later plaatsen, meten in feite
iets anders: de stijging van de levensstandaard door massaproductie
aangedreven door machines. Dit verhoogde de waarde van ongeschoolde
arbeid en leidde tot dalende prijzen voor allerlei consumptiegoederen.
Dat de levensstandaard op verschillende tijdstippen in verschillende
landen sterk begon te stijgen, wijst erop dat er iets anders wordt
gemeten dan de megapolitieke overgang. De \emph{Cambridge Economic
History of Europe} spreekt expliciet van ``Industriële Revoluties'' in
meervoud, en koppelt deze aan de aanhoudende groei van nationale
inkomens. In Japan en Rusland werd deze inkomensgroei pas eind
negentiende eeuw ingezet. In andere delen van Azië en sommige delen van
Afrika was dit een fenomeen van de twintigste eeuw. In delen van Afrika
blijft aanhoudende groei tot op heden een droom. Maar dat betekent niet
dat deze regio's niet in het Moderne Tijdperk leven.

\subsection{Inkomensdaling in een
transitieperiode}\label{inkomensdaling-in-een-transitieperiode}

Inkomensgroei is niet synoniem met de komst van het industrialisme. De
overgang naar een industriële samenleving was een megapolitieke
gebeurtenis, niet rechtstreeks meetbaar in inkomensstatistieken. Voor
het grootste deel van de Europeanen daalden de reële inkomens zelfs
gedurende de eerste twee eeuwen van het Industriële Tijdperk. Pas na het
begin van de achttiende eeuw begonnen ze te stijgen, en bereikten ze pas
rond 1750 weer het niveau van 1250. Wij plaatsen de start van het
Industriële Tijdperk aan het eind van de vijftiende eeuw. Het waren de
industriële kenmerken van vroegmoderne technologie, zoals chemisch
aangedreven wapens en drukpersen, die de ineenstorting van het
feodalisme veroorzaakten.

\subsection{Verlaging van de kosten van
kennis}\label{verlaging-van-de-kosten-van-kennis}

De capaciteit om boeken massaal te produceren was enorm ondermijnend
voor middeleeuwse instellingen, net zoals microtechnologie ondermijnend
zal blijken voor de moderne natiestaat. De boekdrukkunst gooide een bom
onder het monopolie dat de Kerk had op het woord van God, en creëerde
tegelijk een nieuwe markt voor ketterij. Ideeën die onverenigbaar waren
met de gesloten feodale samenleving verspreidden zich snel, via de 10
miljoen boeken die tegen het einde van de vijftiende eeuw waren
gepubliceerd. Omdat de Kerk probeerde de drukpers te onderdrukken,
werden de meeste nieuwe werken uitgegeven in delen van Europa waar het
kerkelijke gezag het zwakst was. Dit lijkt sterk op de pogingen van de
Amerikaanse overheid om versleutelingstechnologie te onderdrukken. De
Kerk ontdekte dat censuur de verspreiding van zulke technologie niet
stopte. Het zorgde er alleen voor dat deze technologie op haar meest
ondermijnende manier werd gebruikt.

\subsection{Ontwaarding van de
kloosters}\label{ontwaarding-van-de-kloosters}

Veel ogenschijnlijk onschuldige toepassingen van de drukpers waren
ondermijnend vanwege hun inhoud. De wetenschap dat avonturiers en
kooplieden fortuinen konden verdienen, was op zich al genoeg om de
feodale structuur af te breken. De verleiding van nieuwe markten, samen
met de noodzaak en mogelijkheid om op grote schaal legers en vloten te
financieren, gaf geld een waarde die het in de feodale eeuwen niet had.
Door deze nieuwe investeringsmogelijkheden, versterkt door krachtige
wapens die geweld winstgevender maakten, werd het voor de landheer of
koopman steeds kostbaarder om zijn kapitaal aan de Kerk te schenken. Zo
ondermijnde het ontstaan van investeringsmogelijkheden buiten grondbezit
de instituties van het feodalisme en tastte het de bijbehorende
ideologie aan.

De drukpers had nog een sterk subversief gevolg. Het verlaagde de kosten
voor het reproduceren van informatie. Een cruciale reden waarom
geletterdheid en economische vooruitgang tijdens de Middeleeuwen zo
beperkt waren, was de hoge kostprijs van het handmatig kopiëren van
manuscripten. Zoals eerder besproken, nam de Kerk na de val van Rome een
belangrijke productieve functie op zich: het reproduceren van boeken en
manuscripten in benedictijnse kloosters. Dit was een uiterst kostbare
bezigheid. Een van de meest ingrijpende gevolgen van de drukkunst was de
ontwaarding van de scriptoria, waar monniken dag in dag uit, maand na
maand werkten aan manuscripten die met drukpersen in enkele uren
gekopieerd konden worden. De nieuwe technologie maakte het benedictijnse
scriptorium tot een verouderd en duur middel voor het verspreiden van
kennis. Daardoor verloren de religieuze orden en de Kerk die de
kopiisten ondersteunden aan economisch belang.

De massaproductie van boeken maakte een einde aan het monopolie van de
Kerk op de Schrift, evenals op andere vormen van informatie. De bredere
beschikbaarheid van boeken verlaagde de kosten van geletterdheid en
vergrootte daarmee het aantal denkers dat in staat was om eigen
opvattingen te uiten over belangrijke, vooral theologische, onderwerpen.
Zoals theologisch historicus Euan Cameron stelde, legde ``een reeks
publicatiemijlpalen'' in de eerste twee decennia van de zestiende eeuw
de basis voor de toepassing van ``moderne tekstkritiek op de Schrift''.
Dit ``bedreigde het monopolie'' van de Kerk ``door corrupte
interpretaties van teksten die traditionele dogma's ondersteunden ter
discussie te stellen.'' Deze nieuwe kennis stimuleerde de opkomst van
concurrerende protestantse sektes die hun eigen interpretaties van de
Bijbel wilden formuleren. Massaproductie van boeken verlaagde de kosten
van ketterij en gaf ketters een groot lezerspubliek.

De uitgeverij droeg ook bij aan de ondergang van het middeleeuwse
wereldbeeld. De grotere beschikbaarheid en lagere kosten van informatie
leidden tot een verschuiving van een wereldbeeld dat symboliek hoog in
het vaandel droeg naar een die causale verbanden als basis stelde. ``Het
symbolische wereldbeeld wordt gekenmerkt door een onberispelijke orde,
architectonische structuur en hiërarchische ondergeschiktheid. Elke
symbolische verbinding impliceert een rangorde of niveau van
heiligheid\ldots{} De walnoot symboliseert Christus; de zoete kern is
Zijn goddelijke natuur, de groene pulp aan de buitenkant is Zijn
menselijkheid, de houten schaal ertussen is het kruis. Zo verwijzen alle
dingen naar het eeuwige\ldots{}''

Een symbolische denkwijze paste niet alleen bij een hiërarchisch
gestructureerde samenleving, maar ook bij ongeletterdheid. Ideeën
overgebracht via symbolen in houtsneden waren toegankelijk voor een
ongeletterde bevolking. Daarentegen leidde de komst van de boekdrukkunst
in de moderne tijd tot de ontwikkeling van causale verbanden en het
gebruik van de wetenschappelijke methode, bedoeld voor een geletterd
publiek.

\section{Een parallel voor vandaag}\label{een-parallel-voor-vandaag}

De middeleeuwse samenleving, die halverwege de vijftiende eeuw nog zo
stabiel en zeker leek in haar overtuigingen, werd razendsnel
getransformeerd. Het monopolie van haar dominante instituut, de Kerk,
werd uitgedaagd en omver geblazen. Gezag dat eeuwenlang onaantastbaar
was, kwam plotseling ter discussie te staan. Overtuigingen en
loyaliteiten die heiliger waren dan die van burgers tegenover een
moderne natiestaat, werden in enkele jaren heroverwogen en verworpen,
allemaal door een technologische revolutie die in het laatste decennium
van de vijftiende eeuw volledig tot uiting kwam.

Wij geloven dat er opnieuw een even ingrijpende verandering zal
plaatsvinden. De informatierevolutie zal het machtsmonopolie van de
natiestaat vernietigen, net zoals de buskruitrevolutie het
machtsmonopolie van de Kerk vernietigde. De situatie aan het einde van
de vijftiende eeuw, waarin het leven volledig doordrenkt was van
georganiseerde religie, vertoont een opvallende gelijkenis met die van
vandaag, waarin de wereld verzadigd is met politiek. De Kerk toen en de
natiestaat nu zijn beide voorbeelden van instellingen die tot een seniel
uiterste zijn uitgegroeid. Net als de laatmiddeleeuwse Kerk is de
natiestaat aan het eind van de twintigste eeuw een diep in de schulden
gestoken instituut dat haar eigen kosten niet meer kan dragen. Haar
functioneren is steeds irrelevanter en zelfs contraproductief voor het
welzijn van degenen die tot voor kort nog tot haar trouwste aanhangers
behoorden.

\subsection{``Verarmd, gretig en
extravagant''}\label{verarmd-gretig-en-extravagant}

Vergelijkbaar met de overheid die vandaag weinig waar voor haar geld
biedt, deed de Kerk dat aan het eind van de vijftiende eeuw ook niet.
Zoals kerkhistoricus Euan Cameron het verwoordde: ``Een verarmd lokaal
priesterschap leek weinig dienst te leveren voor het geld dat het eiste;
veel van wat geheven werd, `verdween' feitelijk in besloten kloosters of
in de duistere regionen van het hoger onderwijs of de kerkelijke
administratie. Ondanks de ruime giften aan bepaalde sectoren van de Kerk
wist de instelling als geheel toch tegelijkertijd een indruk van
armoede, hebzucht en overdaad te wekken.'' De parallel met de overheid
aan het einde van de twintigste eeuw is moeilijk te ontkennen.

Religieuze gebruiken namen in de late vijftiende eeuw toe, net zoals
overheidsprogramma's dat doen in moderne verzorgingsstaten. Niet alleen
namen de bijzondere zegeningen eindeloos toe, samen met het aantal
heiligen en heiligenrelieken, maar elk jaar kwamen er meer kerken, meer
kloosters, meer abdijen, meer bedelordes, meer huiskapelaans, meer
predikambten, meer kathedraalkapittels, meer gesubsidieerde missen, meer
reliekencultussen, meer religieuze broederschappen, meer religieuze
feesten en nieuwe heilige dagen. De diensten werden langer. Gebeden en
hymnen werden ingewikkelder. De ene na de andere nieuwe bedelorde
verscheen om aalmoezen te vragen. Het resultaat was een institutionele
overbelasting vergelijkbaar met die in sterk gepolitiseerde
samenlevingen vandaag.

Nieuwe religieuze vieringen en feestdagen ontstonden overal aan de
lopende band. Kerkdiensten werden talrijker, met speciale vieringen ter
ere van de zeven smarten van Maria, haar zusters en alle heiligen uit de
stamboom van Jezus. Voor gelovigen werd het steeds kostbaarder en
belastender om aan hun religieuze plichten te voldoen, net zoals vandaag
de kosten om binnen de wet te blijven zijn geëxplodeerd.

\subsection{De onschuldigen betalen}\label{de-onschuldigen-betalen}

Toen, net als nu, droegen de productieven steeds meer de last van
inkomensherverdeling. Deze kosten stegen door een verschuiving in het
gebruik van kapitaal sneller dan gezaghebbers beseften. Het relatieve
voordeel van grondbezit ten opzichte van geldkapitaal nam af. Toch bleef
het middeleeuwse denken vasthouden aan een statusgebonden samenleving,
waarin je positie in de sociale hiërarchie werd bepaald door afkomst in
plaats van de vaardigheid om kapitaal effectief in te zetten. Er werd
nauwelijks rekening gehouden met de stijgende opportuniteitskosten van
overdreven religieuze vieringen. Die lasten drukten vooral op ambitieuze
en hardwerkende boeren, burgers en kleine grondeigenaren, die meer dan
de aristocratie afhankelijk waren van zinvol kapitaalgebruik. Ze droegen
buitensporige kosten voor de overvloedige maaltijden bij talloze feesten
en feestdagen, en moesten ook de dure kerkelijke bureaucratie
bekostigen.

\subsection{Contraproductieve
regelgeving}\label{contraproductieve-regelgeving}

Aan het einde van de vijftiende eeuw oefende de Kerk vrijwel alle
regulerende bevoegdheden uit, bevoegdheden die later door
overheidsinstanties werden overgenomen. De Kerk domineerde essentiële
rechtsgebieden: het vastleggen van akten, het registreren van
huwelijken, het afhandelen van testamenten, het verlenen van
handelslicenties, het toekennen van grondeigendomstitels en het bepalen
van de voorwaarden voor het handelsverkeer. Het dagelijks leven werd
bijna even grondig geregeld door het kerkelijk recht als het nu wordt
door bureaucratie, en met een vergelijkbaar doel. Net zoals hedendaagse
regelgeving vol zit met verwarring en tegenstrijdigheden, gold dat ook
voor het kerkelijk recht vijfhonderd jaar geleden. Deze regelgeving
belemmerde het handelsverkeer zo sterk dat al snel duidelijk werd dat de
belangen van de regelgevers ver afstonden van het bevorderen van de
productiviteit.

Zo was het bijvoorbeeld verboden om een heel jaar lang zaken te doen op
de weekdag waarop de meest recente 28 december viel. Als dat een dinsdag
was, mocht er op geen enkele dinsdag legale handel plaatsvinden, als
verplichte uiting van vroomheid ter ere van de Kindermoord in Bethlehem.
In jaren waarin 28 december op een andere dag dan zondag viel,
belemmerde dit veel vormen van handel, wat de kosten opdreef door
transacties te vertragen of volledig onmogelijk te maken.

\subsection{Monopolieprijzen}\label{monopolieprijzen}

Het kerkelijk recht werd ook gebruikt om monopolieprijzen te kunnen
handhaven. De Kerk verdiende aanzienlijk aan de verkoop van aluin uit
haar mijnen in Tolfa, Italië. Toen sommige klanten in de
textielindustrie de voorkeur gaven aan goedkopere aluin uit Turkije,
probeerde het Vaticaan zijn monopolie te behouden via kerkelijk recht
door het gebruik van de goedkopere aluin zondig te verklaren. Handelaren
die toch de Turkse variant kochten, werden geëxcommuniceerd. Het bekende
verbod op vlees op vrijdag kwam voort uit dezelfde logica. De Kerk was
niet alleen de grootste feodale grondeigenaar, maar bezat ook grote
visgronden. De Kerkvaders bedachten een religieuze noodzaak voor het
eten van vis, wat toevallig ook de vraag naar vis garandeerde ondanks de
destijds slechte transportmogelijkheden en hygiënische omstandigheden.

Net als de natiestaat vandaag, reguleerde de Kerk in de late
Middeleeuwen niet alleen specifieke industrieën om haar eigen belangen
direct te ondersteunen, maar ze gebruikte haar regelgevende macht ook om
op andere manieren inkomsten te genereren. Geestelijken deden hun best
om regels en edicten op te stellen die moeilijk na te leven waren.
Incest werd bijvoorbeeld zeer ruim gedefinieerd, zodat zelfs verre neven
en nichten of personen die alleen door huwelijk verwant waren, een
speciale goedkeuring van de Kerk nodig hadden om te mogen trouwen. In
veel kleine Europese dorpen betekende dit dat bijna iedereen zo'n
vrijstelling moest kopen; een lucratieve bron van inkomsten. Zelfs seks
binnen het huwelijk werd streng gereguleerd. Seks tussen echtgenoten was
verboden op zondagen, woensdagen en vrijdagen, en ook gedurende de
veertig dagen voor Pasen en Kerstmis. Bovendien moesten koppels drie
dagen onthouding naleven vóór het ontvangen van de communie. Dit
betekende dat seks binnen het huwelijk gedurende minstens 55\% van het
jaar verboden was zonder kerkelijke toestemming. Volgens historicus E.J.
Burford, in \emph{The Bishop's Brothels}, stimuleerden deze ``idiote''
huwelijksregels de groei van de middeleeuwse prostitutie, waaruit de
Kerk aanzienlijke winst haalde. De bisschop van Winchester was volgens
Burford eeuwenlang de beheerder van de Londense bordeelwijk Bankside in
Southwark. Kerkelijke winsten uit prostitutie waren geen uitsluitend
Engels fenomeen.

\begin{quote}
Paus Sixtus IV (ca. 1471), die naar verluidt syfilis opliep van een van
zijn vele minnaressen, was de eerste paus die prostituees licenties
verleende en een belasting op hun inkomsten instelde, waardoor de
pauselijke inkomsten flink toenamen. De Romeinse Curie financierde deels
de bouw van de Sint-Pietersbasiliek met deze belasting en de verkoop van
vergunningen. Zijn opvolger, paus Leo X, zou ongeveer 22.000 gouden
dukaten hebben verdiend met de verkoop van prostitutievergunningen, vier
keer meer dan hij binnen haalde met de verkoop van aflaten in Duitsland.
\end{quote}

Zelfs de beroemde celibaatsregel voor priesters was een lucratieve
inkomstenbron voor de middeleeuwse kerk. Volgens Burford hief de Kerk
``een heffing genaamd cullagium op priesters met minnaressen''. Dit was
zo winstgevend dat bisschoppen in Frankrijk en Duitsland het celibaat
zonder uitzondering oplegden, ondanks dat het Lateraans Concilie van
1215 deze ``schandelijke handel waarbij prelaten toestemming tot zonde
verkopen'' veroordeelde. Het was slechts één van vele lucratieve markten
voor de verkoop van vergunningen om kerkelijke wetten te overtreden,
gedreven door dezelfde logica als die van hebzuchtige politici die
willekeurige reguleringsmacht over handel nastreven.

\subsection{Aflaatbrieven}\label{aflaatbrieven}

De mogelijkheid om regelgeving naar eigen inzicht in te voeren hield ook
in dat vrijstellingen verkocht mochten worden om de schade van die
regels te compenseren. De Kerk verkocht vergunningen, oftewel
`aflaatbrieven', die uiteenlopende privileges verleenden, van het
kwijtschelden van kleine handelsheffingen tot toestemming voor het eten
van zuivel tijdens de vastentijd. Deze aflaten werden niet alleen voor
hoge prijzen aan adel en welgestelden verkocht, maar ook als
loterijprijzen aangeboden, vergelijkbaar met moderne staatsloterijen, om
zo ook geld van de armen te innen. De handel in aflaatbrieven nam sterk
toe zodra de uitgaven van de Kerk haar inkomsten overtroffen. Velen
concludeerden dat de institutionele kerk haar macht vooral inzette om
inkomsten te genereren. Zoals een hedendaagse criticus opmerkte:
`kerkelijk recht werd uitsluitend ingesteld om geld te verdienen; wie
christen wil zijn, moet zich vrijkopen.'

\subsection{Bureaucratische
overbelasting}\label{bureaucratische-overbelasting}

Tegen het einde van de vijftiende eeuw bereikten de kosten om de
kerkelijke instellingen te onderhouden een historisch hoogtepunt, net
zoals dat de kosten voor het onderhouden van de huidige overheid tot een
ziekelijk niveau zijn uitgegroeid. Hoe meer het leven doordrenkt werd
met religie, hoe duurder en bureaucratischer de Kerk werd. In de woorden
van Cameron: ``Het was veel gemakkelijker om mensen te vinden die de
sterk toegenomen hoeveelheid kerkelijke functies aan het einde van de
Middeleeuwen wilden vervullen dan om geld te vinden om ze te betalen.''
Net zoals failliete overheden tegenwoordig op een contraproductieve
manier inkomsten zoeken, deed de kerk dat vijfhonderd jaar geleden ook.
Inderdaad, de geestelijken gebruikten sommige van dezelfde roofzuchtige
trucs die politici vandaag de dag inzetten.

Net als de natiestaat vandaag, consumeerde de middeleeuwse kerk
vijfhonderd jaar geleden meer middelen van de samenleving dan ooit
tevoren, of ooit weer zou doen. De Kerk leek, net als de staat vandaag,
niet in staat te zijn om te functioneren en zichzelf te onderhouden,
zelfs niet met recordbedragen aan inkomsten. Vergelijkbaar met hoe de
staat laat-industriële economieën beheerst en in sommige West-Europese
landen meer dan de helft van de inkomsten besteedt, domineerde de kerk
de laat-feodale economie door middelen af te romen en groei te remmen.

\subsection{Begrotingstekorten in de vijftiende
eeuw}\label{begrotingstekorten-in-de-vijftiende-eeuw}

De Kerk maakte gebruik van alle denkbare middelen om meer geld uit haar
onderdanen te persen om haar wildgroei aan bureaucratie te voeden.
Gebieden die direct onder het gezag van de Kerk vielen, moesten steeds
hogere belastingen betalen. In provincies en koninkrijken waar de Kerk
geen directe belastingmacht had, legde het Vaticaan ``annates'' op, een
betaling die door de lokale heerser moest worden gedaan in plaats van
directe kerkelijke belastingen.

De Kerk, net als de staat tegenwoordig, plunderde ook haar eigen kas,
waarbij financiële middelen die bestemd waren voor specifieke doeleinden
werden weggesluisd om algemene overheadkosten te betalen. Beneficiën en
kerkelijke ambten werden openlijk verhandeld, evenals de inkomsten uit
tienden. In feite werden de rechten op tienden het kerkelijke equivalent
van staatsobligaties die moderne overheden uitgeven om hun chronische
tekorten te financieren.

Hoewel de Kerk ideologisch de verdediger was van het feodalisme en
criticus van handel en kapitalisme, gebruikte zij, net als de moderne
natiestaat, elke beschikbare marketingtechniek om haar eigen inkomsten
te optimaliseren. De kerk dreef een bloeiende handel in sacramentalia,
waaronder gewijde kaarsen, palmtakken gezegend op Palmzondag, ``kruiden
gezegend op het feest van de Hemelvaart, en vooral verschillende soorten
heilig water.''

Net als hedendaagse politici die burgers bedreigen met minder
vuilnisophaaldiensten en andere ongemakken als zij weigeren hogere
belastingen te betalen, waren religieuze autoriteiten in de vijftiende
eeuw ook geneigd religieuze diensten stop te zetten om gemeenten te
chanteren tot het betalen van willekeurige boetes. Vaak werden boetes
opgelegd voor kleine overtredingen door enkelen die niet eens lid
hoefden te zijn van de betreffende gemeente. In 1436 liet bisschop
Jacques Du Chatelier, ``een zeer opzichtig en hebzuchtig man,'' de Kerk
van de Onschuldigen in Parijs tweeëntwintig dagen sluiten, totdat twee
bedelaars een onmogelijk hoge boete hadden voldaan. De mannen hadden
ruzie gemaakt in de Kerk waarbij een paar druppels bloed waren vergoten.
De bisschop beweerde dat ze de Kerk hadden ontheiligd. Hij stond niemand
toe de Kerk te gebruiken voor bruiloften, begrafenissen of de normale
jaarlijkse sacramenten totdat de boete was betaald.

\begin{quote}
De Italiaanse bordelen (om de paus te vermaken) betalen jaarlijks
twintigduizend dukaten. Ze geven een priester, voor wat extra eer, de
winst van een hoer, of twee, of meer. Het moet wel een heilige zijn,
geen simpele vent, die zo met de bordelen is bekend.

-\/- vijftiende-eeuwse Engelse ballade
\end{quote}

\subsection{Haat jegens kerkelijke
leiders}\label{haat-jegens-kerkelijke-leiders}

Geen wonder dat de publieke opinie aan het eind van de vijftiende eeuw
de hogere en lagere clerus verachtte, net zoals mensen in sterk
gepolitiseerde samenlevingen vandaag de dag een afkeer hebben van
bureaucraten en politici. Zoals Johan Huizinga schreef: ``Haat is het
juiste woord in deze context, want het was haat, latent, maar algemeen
en aanhoudend. Het volk raakte nooit uitgeput van het aanhoren van de
aanklachten tegen de ondeugden van de geestelijkheid.'' Een deel van de
reden dat mensen overtuigd waren dat de Kerk ``hebzuchtig en
verkwistend'' was, is dat dit ook daadwerkelijk klopte. ``De wereldsheid
van de hogere geestelijkheid en het verval van de lagere rangen'' was
overduidelijk.

Van de pastoor tot de paus zelf leek de geestelijkheid zo corrupt als
alleen het personeel van een overheersende instelling kan zijn.
Vijfhonderd jaar geleden maakte paus Alexander VI zelfs figuren als
Giulio Andreotti en Bill Clinton tot voorbeelden van integriteit.
Alexander VI stond bekend om zijn losbandige feesten. Als kardinaal in
Siena organiseerde hij een beruchte orgie waarbij alleen ``de mooiste
jonge vrouwen van Siena waren uitgenodigd, terwijl hun `echtgenoten,
vaders en broers' waren uitgesloten.'' Die orgie was berucht, maar bleek
later tam vergeleken met wat volgde nadat Alexander paus werd. De meest
beruchte was de zogeheten Kastanje-ballet, waarbij ``vijftig van de
mooiste hoeren van Rome'' deelnamen aan een sekswedstrijd met
kerkvorsten en andere invloedrijke Romeinen. Zoals William Manchester
beschrijft: ``Dienaren hielden het aantal orgasmes van elke man bij,
want de paus bewonderde viriliteit\ldots{} Nadat iedereen uitgeput was,
deelde Zijne Heiligheid prijzen uit: mantels, laarzen, hoeden en fijne
zijden tunieken. De winnaars'', zo schreef de kroniekschrijver, ``waren
zij die het vaakst de hoeren hadden bemind.''

Alexander verwekte minstens zeven en mogelijk acht buitenechtelijke
kinderen. Een van zijn vermeende zonen, Giovanni, was de zogeheten
\emph{Infans Romanus}, geboren uit Alexanders buitenechtelijke dochter,
Lucrezia Borgia, toen zij achttien was. In een geheime pauselijke bul
erkende Alexander het vaderschap van Giovanni. Als hij niet de vader
was, was hij zeker de grootvader aan beide kanten. De paus was betrokken
bij een driehoekige incestueuze verhouding met Lucrezia, die ook de
minnares was van Juan, hertog van Gandia, Alexanders oudste
buitenechtelijke zoon, én van een andere buitenechtelijke zoon,
kardinaal Cesare Borgia. Cesare was de kerkvorst die Niccolò Machiavelli
inspireerde tot \emph{Il Principe}. Cesare was een moordenaar, net als
de paus, die bekend stond als samenzweerder achter verschillende
moorden. Een van beiden werd vermoedelijk jaloers op Juan, wiens
levenloze lichaam op 15 juni 1497 uit de Tiber werd gevist.

Het leiderschap van de laatmiddeleeuwse kerk was net zo corrupt als dat
van de moderne natiestaat.

\begin{quote}
`Vandaag ben ik twee keer vader geworden, Gods zegen daarbij.' - Rodolph
Acricola, toen hij hoorde dat zijn minnares op de dag van zijn
verkiezing tot abt een zoon had gebaard.
\end{quote}

\section{Hypocrisie}\label{hypocrisie}

Onder een ``oppervlakkige laag van vroomheid'' was de laatmiddeleeuwse
samenleving opmerkelijk godslasterlijk, oneerbiedig en losbandig. Kerken
waren de favoriete ontmoetingsplaatsen voor jonge mannen en vrouwen, en
ook vaste plekken voor prostituees en verkopers van obscene prenten.
Historici melden dat ``de oneerbiedigheid in het dagelijks religieus
leven vrijwel geen grenzen kende.'' Koorzangers die waren ingehuurd om
voor de zielen van de doden te zingen, voegden vaak onbeleefde woorden
toe aan heilige teksten tijdens de mis. Wakes en processies, die een
veel grotere rol speelden in het middeleeuwse religieuze leven dan
tegenwoordig, werden desondanks ``ontheiligd door platvloersheid, spot
en drankgebruik.'' Zo verklaarde Denis de Kartuizer, de toonaangevende
theologische autoriteit van het Europa van de late middeleeuwen.

Hoewel zo'n verslag afgedaan zou kunnen worden als het geklaag van een
stijve moraalridder, is het slechts een van de vele bronnen die
hetzelfde beeld schetsen. Er is alle reden om aan te nemen dat het
schunnige en het heilige vaak nauw verweven waren in het middeleeuwse
leven. Bedevaarten ontspoorden zo vaak in oproer en losbandigheid dat
idealistische hervormers tevergeefs pleitten voor hun afschaffing.
Lokale religieuze processies boden regelmatig aanleiding voor menigtes
om te vandaliseren, te plunderen en zich over te geven aan ongeremd
dronken wangedrag. Zelfs wanneer men stil zat om de mis te vieren, was
dat vaak geen sobere aangelegenheid. Grote hoeveelheden wijn werden in
de Kerk geconsumeerd, vooral tijdens nachtfestiviteiten. Uit documenten
van het Concilie van Straatsburg blijkt dat er tijdens ``nachtelijke
waken'' op Sint Adolphus 1.100 liter wijn werd genuttigd, aangeboden
door het concilie zelf.

Jean Gerson, een invloedrijke theoloog uit de vijftiende eeuw, meldt dat
``de meest heilige feesten, zelfs kerstavond,'' werden doorgebracht ``in
losbandigheid, kaartspelen, vloeken en godslastering.'' Wanneer men op
dit gedrag werd aangesproken, beriepen gewone mensen zich op ``het
voorbeeld van de adel en de geestelijkheid, die zich op dezelfde wijze
gedragen zonder gevolgen.''

\subsection{Vroomheid en mededogen}\label{vroomheid-en-mededogen}

De vroomheid die de alomtegenwoordigheid van de georganiseerde religie
in de late Middeleeuwen moest rechtvaardigen, diende hetzelfde doel als
het ``mededogen'' dat vandaag wordt gebruikt om politieke overheersing
van het leven te legitimeren. De verkoop van aflaten om een verlangen
naar vroomheid zonder moraal te bevredigen, is vergelijkbaar met royale
uitgaven aan sociale voorzieningen om de schijn van mededogen te wekken,
zonder daadwerkelijke naastenliefde. Of het effect van de gangbare
religieuze praktijken werkelijk moreel karakter verbeterde of zielen
redde, was grotendeels irrelevant, net zoals het nauwelijks van belang
lijkt of een sociaal programma daadwerkelijk het leven verbetert van wie
het zou moeten helpen. ``Vroomheid'', zoals ``mededogen'', was bijna een
bijgelovige bezwering.

In een tijd waarin oorzaak en gevolg nauwelijks werden begrepen,
doordrongen rituelen en sacramenten van de Kerk elk aspect van het
leven. ``\ldots{} een reis, een taak, een bezoek, werden allemaal
begeleid door duizend formaliteiten: zegeningen, ceremonies, formules.''
Gebeden op perkament werden als kettingen omgehangen bij mensen met
koorts. Ondervoede meisjes hingen haarlokken voor het beeld van
Sint-Urbanus om verdere haaruitval te voorkomen. Boeren in Navarra
liepen al biddend voor regen in processie achter een beeld van
Sint-Pieter tijdens droogte. In afwezigheid van werkzame middelen grepen
mensen maar al te graag naar ``ineffectieve rituelen om hun angst te
verzachten.''

\subsection{Twee zonden voor een
zegen}\label{twee-zonden-voor-een-zegen}

Mensen waren zo overtuigd van de wonderbaarlijke kracht van
heiligenrelieken dat de dood van een vroom persoon vaak leidde tot een
ware stormloop om het lichaam te verdelen. Nadat Thomas van Aquino
stierf in het klooster Fossanuova, onthoofdden en kookten de monniken
zijn lichaam om zijn botten in handen te krijgen. Toen Sint Elisabeth
van Hongarije werd opgebaard, rukte een menigte aanbidders linnenstrips
van haar gezicht en knipten haar haar, nagels en tepels af.

\subsection{Vroomheid zonder deugd}\label{vroomheid-zonder-deugd}

De middeleeuwse mens zag de heiligen en hun relieken als wapens van het
geloof, in een wereld vol koude winters, donkere nachten, en
uitzichtloze ziektes. Ontberingen die voor moderne lezers vrijwel
onbekend zijn. Meer dan in de moderne tijd geloofden mensen in de
Middeleeuwen dat demonen werkelijk bestonden, dat God actief ingreep in
de wereld, en dat gebeden, boetedoening en bedevaarten goddelijke
gunsten zouden opleveren.

Zeggen dat mensen in God geloofden, schiet tekort om de intensiteit van
hun overtuiging weer te geven, net als de ogenschijnlijke
vanzelfsprekendheid waarmee middeleeuwse vroomheid samenging met zonde.
Het geloof in de werkzaamheid van rituelen, sacramenten en kerkelijke
handelingen was zo wijdverspreid dat het onvermijdelijk de urgentie van
deugdzaam gedrag ondermijnde. Voor elke zonde of geestelijk gebrek
bestond een remedie, een boetedoening die het verleden uitwiste, een
systeem dat uitmondde in een ``wiskunde van de verlossing.'' Religie
werd zo alomtegenwoordig dat haar oprechtheid onvermijdelijk begon te
slijten. Zoals Huizinga het verwoordde: ``Wanneer religie doordringt tot
alle aspecten van het leven, betekent dat een voortdurende vermenging
van het heilige en de ondeugdelijke gedachte. Heilige dingen worden te
alledaags om nog diep gevoeld te worden.'' En dat was ook de realiteit.

\section{De verkleining van de kerk}\label{de-verkleining-van-de-kerk}

Aan het einde van de vijftiende eeuw kwam de Kerk niet alleen even
corrupt over als de hedendaagse natiestaat, maar werkte ze ook als een
zware rem op de economische groei. De Kerk vergaarde op onproductieve
wijze enorme hoeveelheden kapitaal en legde lasten op die zowel de
productie als de handel binnen de samenleving beperkten. Deze lasten,
vergelijkbaar met wat tegenwoordig door de natiestaat wordt opgelegd,
waren talrijk. We weten wat er met de georganiseerde religie gebeurde na
de buskruitrevolutie: die ontwikkelingen leidden tot krachtige prikkels
om religieuze instellingen te verkleinen en hun kosten te drukken. Toen
de traditionele kerk weigerde hiertoe over te gaan, grepen protestantse
sekten de kans om te concurreren. Daarbij gebruikten ze bijna elk
denkbaar middel om de kosten van een vroom bestaan te verlagen:

\begin{itemize}
\tightlist
\item
  Ze bouwden sobere nieuwe kerken en verwijderden soms de altaren uit
  oudere kerken, zodat kapitaal voor andere doeleinden vrijkwam.
\item
  Ze herformuleerden de christelijke leer op een wijze die de kosten
  deed dalen, doordat zij het geloof als sleutel tot verlossing vóór
  goede werken plaatsten.
\item
  Ze ontwikkelden een nieuwe, beknopte liturgie, schrapten of beperkten
  feestdagen en schaften een boel sacramenten af.
\item
  Ze sloten kloosters en stopten met het verstrekken van aalmoezen aan
  bedelaarsordes. Armoede veranderde daarmee van een apostolische deugd
  tot een ongewenst en vaak verwijtbaar sociaal probleem.
\end{itemize}

Om te begrijpen hoe het inkrimpen van de Kerk de productiviteit
bevrijdde, moet je eerst kijken naar de vele manieren waarop de Kerk
vóór het einde van haar monopolie de groei belemmerde. Net zoals de
natiestaat dat vandaag doet, legde de Kerk aan het eind van de
vijftiende eeuw een enorme berg aan overbodige kosten op.

\begin{enumerate}
\def\labelenumi{\arabic{enumi}.}
\item
  Directe kosten zoals tienden, belastingen en heffingen voedden de
  uitgedijde kerkelijke bureaucratie. Tienden kwamen ook voor in
  protestantse kerken die de middeleeuwse ``Heilige Moederkerk''
  vervingen, maar waren in stedelijke gebieden doorgaans moeilijk te
  innen. Het einde van het kerkelijke monopolie leidde feitelijk tot
  dalende marginale belastingtarieven in regio's met de meest
  ontwikkelde handel.
\item
  Religieuze doctrines maakten sparen moeilijk. De aartsvijand van de
  middeleeuwse kerk was de ``vrek,'' iemand die zijn goud opspaarde ten
  koste van zijn ziel. De eis dat gelovigen ``goede doelen'' moesten
  bekostigen, hield in dat men dure bijdragen moest leveren aan de kerk.
  De leer van de ``satisfacties'' verplichtte wie zich zorgen maakte
  over zijn redding om missen of kapellen te bekostigen om het vagevuur
  te vermijden. Luther viel dit rechtstreeks aan in de achtste en
  dertiende van zijn vijfennegentig stellingen. Hij schreef dat ``de
  stervende al zijn schulden betaalt met zijn dood.'' Met andere
  woorden: het kapitaal van de protestantse gelovige bleef beschikbaar
  voor zijn erfgenamen. Volgens de protestantse leer hoefden er geen
  kapellen meer gefinancierd te worden, voorheen vaak dertig jaar lang,
  en bij zeer rijken zelfs tot in de eeuwigheid.
\item
  De ideologie van de middeleeuwse kerk dreef mensen er toe om hun
  kapitaal in te zetten voor het verzamelen van relieken. Grote bedragen
  gingen naar reliekenculten voor de aanschaf van tastbare objecten die
  met Christus of andere heiligen werden geassocieerd. De hele rijken
  stelden zelfs persoonlijke reliekencollecties samen. Zo verzamelde
  keurvorst Frederik van Saksen maar liefst negentienduizend relieken,
  waarvan sommige afkomstig waren van een pelgrimstocht naar Jeruzalem
  in 1493. Zijn collectie bevatte onder meer wat hij geloofde dat het
  lichaam van een heilig kind was, melk van Maria en stro uit de stal
  van de Geboorte. Vermoedelijk was het rendement op het daarin
  geïnvesteerde kapitaal laag. Door de nieuwe focus op het persoonlijke
  geloof en het idee van de uitverkorenen verminderde het nut van het
  verzamelen van christelijke attributen als geluksbrengers, en werd het
  voor de monarch voordeliger om geld voor productievere doeleinden in
  te zetten.
\item
  De opkomst van protestantse bewegingen brak de economische monopolies
  van de middeleeuwse kerk en leidde tot een aanzienlijke verzwakking
  van de regelgeving. Zoals we hebben gezien, werd het kerkelijk recht
  vaak aangepast om kerkelijke monopolies en commerciële belangen te
  steunen. Doordat deze nieuwe stromingen geen gevestigde economische
  structuren hoefden te beschermen, resulteerden hun doctrines in een
  vrijer systeem met minder handelsbelemmeringen.
\item
  De protestantse revolutie schafte veel van de tijdrovende rituelen en
  sacramenten van de middeleeuwse kerk af. Riten, sacramenten en heilige
  dagen hadden tegen het einde van de vijftiende eeuw vrijwel de hele
  kalender in beslag genomen. Deze overdaad aan rituelen was een logisch
  gevolg van de kerkelijke stelling ``\ldots dat men gebeden of
  erediensten eindeloos kon herhalen en daar steeds opnieuw voordeel uit
  kon halen.'' En herhalen deden ze. De productiviteit lijdde onder
  steeds langere en uitgebreidere diensten, verplichtingen tot het
  herhalen van gebeden als boetedoening, en de wildgroei aan feestdagen
  waarop niet gewerkt mocht worden. Tal van regels en rituelen
  onderbraken de dag en de seizoenen, waardoor de tijd voor productieve
  arbeid sterk werd beperkt. Dit had wellicht weinig invloed op het
  ritme van de middeleeuwse landbouw, waarin meer dan 90 procent van de
  bevolking werkzaam was. Gedurende het jaar waren er verschillende
  perioden waarin veldarbeid niet dagelijks nodig was. De oogstopbrengst
  in de Middeleeuwen hing waarschijnlijk sterker af van het weer en
  oncontroleerbare plagen dan van de hoeveelheid arbeid die men kon
  uitvoeren boven op het minimum dat de kerkelijke kalender toestond.

  Het grotere productiviteitsverlies zat niet zozeer in de landbouw,
  maar in andere sectoren. De kerkelijke eisen waren veel minder
  verenigbaar met ambachtelijk werk, productie, transport, handel of
  andere activiteiten waarbij productiviteit en winstgevendheid sterk
  afhingen van de hoeveelheid tijd dat men eraan besteedde.

  Het is waarschijnlijk geen toeval dat de grote omwenteling aan het
  eind van de vijftiende eeuw plaatsvond op een moment dat pachtprijzen
  stegen en de reële lonen voor de boerenbevolking daalden. De
  toenemende bevolkingsaantallen zette de opbrengst van de
  gemeenschappelijke gronden, die zich vaak rond rivieren en beken
  bevonden, essentieel voor begrazing, visvangst en brandhout, onder
  druk. De dalende levensstandaard dwong boeren in toenemende mate om
  naar alternatieve inkomstenbronnen te zoeken. Daardoor ging een
  groeiend deel van de plattelandsbevolking zich richten op
  kleinschalige productie voor de markt, vooral in textiel, in het
  proces dat bekendstaat als ``huisnijverheid'' of
  ``proto-industrialisatie.'' De tijdrovende ceremoniële lasten die de
  Kerk oplegde, stonden pogingen van ambitieuze boeren in de weg om hun
  landbouwinkomen aan te vullen via ambachtelijk werk, en hinderden in
  het algemeen elke verschuiving van inspanning naar nieuwe economische
  kansen.

  Een van de meest tastbare protestantse bijdragen aan de productiviteit
  was het afschaffen van veertig feestdagen. Dit bespaarde niet alleen
  de aanzienlijke kosten van de festiviteiten, inclusief het voorzien
  van eten en drinken op dorpsfeesten, maar leverde ook veel waardevolle
  tijd op. Iedereen die ophield met het vieren van de afgeschafte
  feestdagen kon impliciet meer dan driehonderd manuren per jaar
  toevoegen aan zijn productiviteit. Kortom, het schrappen van de
  ceremoniële kerkelijke overbelasting maakte de weg vrij voor een
  duidelijke toename van de productiviteit, simpelweg door tijd vrij te
  maken die anders aan kerkelijke rituelen verloren was gegaan.
\item
  De breuk in het kerkelijke monopolie maakte enorme hoeveelheden
  bezittingen vrij die onder kerkelijk beheer lage rendementen
  opleverden, een situatie met duidelijke parallellen met het
  staatsbezit in de late twintigste eeuw. De Kerk was verreweg de
  grootste feodale grootgrondbezitter. Haar greep op het land was
  vergelijkbaar met die van de staat in sterk gepolitiseerde
  samenlevingen vandaag de dag. In sommige Europese landen zoals Bohemen
  bezit de staat meer dan 50 procent van het totale landbezit. Volgens
  het kerkelijk recht mocht eenmaal verworven kerkgrond niet worden
  afgestoten. Daardoor namen de landbezittingen van de Kerk gestaag toe,
  gevoed door testamentaire giften van gelovigen ter financiering van
  sociale voorzieningen, kapellen en andere kerkelijke activiteiten.

  Hoewel het moeilijk is om de relatieve productiviteit van kerkelijke
  bezittingen exact te meten, moet die tegen het einde van de
  Middeleeuwen aanzienlijk lager zijn geweest dan aan het begin van die
  periode. Tegen de veertiende eeuw leidde de verschuiving van
  zelfvoorzienende landbouw naar marktgerichte productie ertoe dat de
  meeste lekenheren analfabete dorpshoofden inruilden voor professionele
  beheerders om hun opbrengsten te maximaliseren. Hun prikkels zullen er
  waarschijnlijk toe geleid hebben dat zij de opbrengsten van kerkelijk
  bezit snel overtroffen, aangezien kerkgronden in theorie geen
  privéwinst opleverden. Sommige wereldlijke prins-bisschoppen beheerden
  hun landerijen ongetwijfeld op een manier die nauwelijks te
  onderscheiden was van die van lekenheren. Toch moet de productiviteit
  van andere kerkelijke eigendommen sterk geleden hebben onder het
  onverschillige beheer door een logge, wijdvertakte organisatie, met
  gebreken vergelijkbaar met die van staats- en collectief eigendom
  vandaag de dag. Het is bovendien duidelijk dat de inbeslagname van
  kloosters middelen herverdeelde die na de uitvinding van de
  boekdrukkunst niet langer nodig waren voor het handmatig reproduceren
  van boeken en manuscripten.
\item
  Zoals we in \emph{The Great Reckoning} uitvoerig hebben beschreven,
  reageerden sommige protestantse stromingen op de buskruitrevolutie
  door handel te stimuleren via hun doctrines, bijvoorbeeld door het
  verbod op woeker, oftewel rente op leningen, op te heffen. De
  ideologische tegenstand van de middeleeuwse kerk tegen het kapitalisme
  remde de groei. De kern van de kerkelijke leer was gericht op het
  versterken van het feodale systeem, waar de Kerk zelf, als grootste
  feodale grootgrondbezitter, een enorm belang in had. Bewust of
  onbewust maakte de Kerk religieuze deugden van haar eigen economische
  belangen, terwijl ze zich verzette tegen de opkomst van nijverheid en
  onafhankelijke commerciële rijkdom, ontwikkelingen die het feodale
  systeem dreigden te ondermijnen. Verboden op ``winstbejag'' golden
  voornamelijk voor commerciële transacties en zelden of nooit voor
  feodale heffingen, en al helemaal niet voor de verkoop van aflaten.
  Pogingen van de Kerk om een ``rechtvaardige prijs'' vast te stellen
  voor handelswaar onderdrukten systematisch het economische rendement
  van goederen en diensten die de kerk zelf niet produceerde.

  Het verbod op ``woeker'' was een duidelijk voorbeeld van de kerkelijke
  weerstand tegen commerciële innovatie. Banken en krediet waren
  cruciaal voor de ontwikkeling van grootschalige handelsondernemingen.
  Door kredietverlening te beperken, remde de Kerk de economische groei.
\item
  De meer subtiele impact van de nieuwe protestantse stromingen lag in
  hun nadruk op de bijbel als tekst. Dit ondermijnde niet alleen de
  ideologie, maar ook het denkpatroon van de middeleeuwse kerk, die
  beide een remmende werking hadden op groei. De culturele programmering
  van de late middeleeuwen leerde mensen de wereld te begrijpen via
  symbolische gelijkenissen in plaats van via oorzaak en gevolg. Het
  zette rationaliteit op zijn kop en stond haaks op een handelsgerichte
  manier van leven. Symbolisch denken laat zich moeilijk vertalen naar
  marktdenken. Zoals de historicus Johan Huizinga beschreef: ``De drie
  standen vertegenwoordigen de eigenschappen van de Maagd Maria: de
  zeven keurvorsten van het rijk staan voor de deugden; de vijf steden
  van Artesië en Henegouwen, die in 1477 trouw bleven aan het huis van
  Bourgondië, zijn de vijf wijze maagden; \ldots{} Schoeisel staat voor
  zorgvuldigheid, kousen voor volharding, de jarretel voor
  vastberadenheid, enzovoorts.''

  De denkwereld werd beheerst door dogma's, starre symbolen en
  allegorieën die elk aspect van het leven koppelden aan hiërarchische
  religieuze concepten. Elke handeling, elk object, elk getal en elke
  kleur was onderdeel van een allesomvattend religieus systeem. Hierdoor
  werden de gewone dingen des levens niet verklaard vanuit hun
  oorzakelijke verbanden, maar vanuit statische symboliek. Alles stond
  voor iets anders, en dat weer voor iets daarboven. Vaak leidde dit tot
  verwarring in plaats van inzicht. Numerieke systemen, vooral met het
  getal zeven, versterkten dit: zeven deugden, zeven hoofdzonden, zeven
  beden van het Onze Vader, zeven gaven van de Heilige Geest, de zeven
  momenten van de passie van Christus, zeven zaligsprekingen en zeven
  sacramenten, ``vertegenwoordigd door zeven dieren, gevolgd door zeven
  ziekten.''
\end{enumerate}

\subsection{Vijftiende-eeuwse
journalistiek}\label{vijftiende-eeuwse-journalistiek}

Een nieuwsbericht uit de vijftiende eeuw zou, als het al zou zijn
geschreven, geen van de klassieke journalistieke vragen rechtstreeks
hebben beantwoord (wie, wat, waar, wanneer, waarom), maar zou deze
slechts indirect hebben aangestipt via allegorische personificaties.
Lees bijvoorbeeld dit verslag uit een privé-dagboek over de
Bourgondische moorden in het vijftiende-eeuwse Parijs:

\begin{quote}
Dan rees de godin van de Twist, die woonde in de toren van Kwade Raad,
en zij wekte Toorn, de waanzinnige vrouw, en Hebzucht en Woede en Wraak,
en zij grepen allerlei wapens en wierpen Redelijkheid, Gerechtigheid,
Herinnering aan God en Gematigdheid op schandelijke wijze buiten. Toen
maakte Waanzin hen razend, en Moord en Slachting doodden, hakten neer,
brachten ter dood, en vermoordden allen die zij in de gevangenissen
vonden. \ldots{} en Hebzucht stopte haar rok in haar gordel samen met
Roof, haar dochter, en Diefstal, haar zoon. \ldots{} Daarna trokken de
voornoemde mensen onder leiding van hun godinnen, dat wil zeggen Toorn,
Hebzucht en Wraak, die hen door alle publieke gevangenissen van Parijs
leidden, enz.
\end{quote}

De verschuiving weg van het middeleeuwse denken stimuleerde een moderne
denkwijze, waarin oorzaak en gevolg centraal stonden in plaats van
symbolische verbanden en allegorische personificatie.

Zonder de oprechtheid van de laatmiddeleeuwse kerk ter discussie te
stellen, is het duidelijk dat haar denken paste bij het agrarische
feodale systeem en weinig ruimte liet voor handel of industriële
ontwikkeling. De Kerk functioneerde als een overheersende instelling die
morele, culturele en juridische beperkingen oplegde op manieren die
perfect aansloten bij de eisen van het feodalisme. Juist daarom waren ze
slecht afgestemd op de behoeften van een industriële samenleving, net
zoals de morele, culturele en juridische beperkingen van de moderne
natiestaat slecht geschikt zijn om handel in het Informatietijdperk te
bevorderen. Wij geloven dat de staat, net als de Kerk destijds, zal
worden hervormd om de nieuwe mogelijkheden te kunnen realiseren.

Het protestantse idee dat de hemel kon worden bereikt door geloof
alleen, zonder dure misvieringen voor overledenen, werd gepresenteerd
als een theologische kwestie. Het was echter theologie afgestemd op de
economische realiteit van een nieuw tijdperk. Het bood een duidelijk
goedkopere weg naar verlossing, precies op het moment waarop de
opportuniteitskosten om het opgeblazen kerkelijke apparaat in stand te
houden scherp toenamen. Mensen hadden er vroeger minder moeite mee om
hun geld aan de Kerk te geven, omdat er geen alternatief was. Maar zodra
ze inzagen dat ze hun kapitaal konden verhonderdvoudigen door een
specerijenexpeditie naar het Oosten te financieren, of wel 40 procent
per jaar konden verdienen met de financiering van een bataljon voor de
koning, zochten ze begrijpelijkerwijs Gods genade daar waar hun eigen
belang lag.

Veel kooplieden en andere burgers werden al snel veel rijker dan hun
voorouders ooit waren geweest onder het feodalisme. De snelle stijging
van de levensstandaard onder kooplieden en kleine producenten in de
vroegmoderne periode werd breed gehaat door degenen wier inkomen en
levensstandaard samen met het feodale systeem instortten. De verzwakking
van het kerkelijke monopolie en de toenemende megapolitieke macht van de
rijken leidde tot een scherpe afname van de herverdeling van inkomens.
De boeren en stedelijke armen die geen directe voordelen van het nieuwe
systeem ondervonden, waren bitter jaloers op degenen die dat wel deden.
Huizinga beschreef de heersende houding, in wat een belangrijke parallel
met de informatierevolutie zou kunnen worden: ``Haat jegens rijke
mensen, vooral de nieuwe rijken, die toen zeer talrijk waren, was
gemeengoed.''

Een even opvallende parallel ontstond door een enorme stijging van de
criminaliteit. Het instorten van de oude orde leidt bijna altijd tot een
toename van misdaad, zo niet tot de regelrechte anarchie van de feodale
revolutie die we in het vorige hoofdstuk behandelden. Aan het eind van
de Middeleeuwen nam de misdaad fors toe toen de oude sociale
controlesystemen verdwenen. In Huizinga's woorden: ``Misdaad werd steeds
meer gezien als een bedreiging voor de orde en samenleving.'' In de
toekomst kan dat net zo'n bedreiging vormen.

De moderne wereld werd geboren uit de chaos die nieuwe technologieën,
nieuwe ideeën en de stank van buskruit met zich meebrachten.
Buskruitwapens en verbeterde scheepvaart destabiliseerden het militaire
fundament van het feodalisme, terwijl nieuwe communicatietechnologie
haar ideologie ondermijnde. Een van de zaken die de nieuwe druktechniek
aan het licht bracht, was de corruptie binnen de kerk, zowel binnen de
hiërarchie als onder de gewone geestelijken, die al in laag aanzien
stonden in een samenleving die religie paradoxaal genoeg als het
middelpunt van alles zag. Die paradox is duidelijk herkenbaar in de
hedendaagse desillusie over politici en bureaucraten, in een samenleving
die politiek eveneens centraal stelt.

Het einde van de vijftiende eeuw was een tijd van desillusie,
verwarring, pessimisme en wanhoop. Een tijd, net als nu.

\setsubtitle{Democratie en nationalisme als strategische middelen in het tijdperk van geweld}

\bookmarksetup{startatroot}

\chapter{Het leven en de gezondheid van de
natiestaat}\label{het-leven-en-de-gezondheid-van-de-natiestaat}

::: \{.content-hidden when-format=``latex''\} \emph{Democratie en
nationalisme als strategische middelen in het Tijdperk van Geweld} :::

\begin{quote}
Het allerbelangrijkste voor oorlogsvoering is om voldoende financiële
middelen te hebben om alles te leveren wat nodig is.{[}\^{}163{]} --
Robert de Balsac, 1502
\end{quote}

\section{Het puin van de
geschiedenis}\label{het-puin-van-de-geschiedenis}

Op 9 en 10 november 1989 zond de televisie wereldwijd beelden uit van
uitgelaten Oost-Berlijners die de Berlijnse Muur met voorhamers
afbraken. Beginnende ondernemers in de menigte raapten stukken van de
muur op, die later aan kapitalisten overal ter wereld werden verkocht
als souvenir. Jarenlang werd er een levendige handel in deze relikwieën
gedreven. Zelfs op het moment dat we dit schrijven, kom je nog af en toe
advertenties tegen in kleine tijdschriften waarin stukken oud Oost-Duits
beton worden aangeboden voor prijzen die normaal gesproken alleen voor
hoogwaardig zilvererts worden betaald. Degenen die de Berlijnse
Muur-souvenirs hebben gekocht, hoeven geen haast te hebben om ze te
verkopen. Zij bezitten aandenken aan iets groters dan de val van het
communisme. Wij geloven dat de Berlijnse Muur de belangrijkste berg
historisch puin is geworden sinds de muren van San Giovanni met de grond
gelijk werden gemaakt bijna vijf eeuwen eerder, in februari
1495.{[}\^{}165{]}

Het afbreken van San Giovanni door de Franse koning Karel VIII was het
eerste salvo van de Buskruitrevolutie. Het markeerde het einde van de
feodale fase van de geschiedenis en het begin van het industrialisme,
zoals we eerder hebben uiteengezet. De vernietiging van de Berlijnse
Muur markeert een andere historische scheidslijn: de overgang van het
Industriële Tijdperk naar het nieuwe Informatietijdperk. Nooit eerder
was er zo'n grote symbolische triomf van efficiëntie over macht. Toen de
muren van San Giovanni vielen, was dat een onmiskenbare demonstratie dat
de economische opbrengsten van geweld sterk waren gestegen. De val van
de Berlijnse Muur zegt echter iets anders, namelijk dat de opbrengsten
van geweld nu dalen. Dit is iets wat nog maar weinigen zijn gaan
beseffen, maar het zal dramatische gevolgen hebben.

Om redenen die we in dit hoofdstuk behandelen, kan de Berlijnse Muur
veel symbolischer zijn voor het hele tijdperk van de industriële
natiestaat dan de menigte die nacht in Berlijn, of de miljoenen die van
een afstand keken, beseften. De Berlijnse Muur werd gebouwd met een heel
ander doel dan de muren van San Giovanni: om te voorkomen dat mensen aan
de binnenkant konden ontsnappen, in plaats van te voorkomen dat
roofdieren van buitenaf binnenkwamen. Dat feit alleen al is een
veelzeggend teken van de toename van de macht van de staat van de
vijftiende tot de twintigste eeuw. En dat op meer dan één manier.

Eeuwenlang maakte de natiestaat alle muren die naar buiten gericht waren
overbodig en onnodig. De gebieden waarin de staat als eerste een
monopolie op dwang realiseerde, werden zowel intern vreedzamer als
militair krachtiger dan welke soevereine macht de wereld ook had gekend.
De staat gebruikte de middelen die zij van een grotendeels ontwapende
bevolking afnam om kleinschalige roofdieren de kop in te drukken. De
natiestaat werd het succesvolste instrument in de geschiedenis om
middelen af te pakken. Dat succes was gebaseerd op het superieure
vermogen om de rijkdom van zijn burgers voor haar eigen doeleinden in te
zetten.

\begin{quote}
MTV is meer dan alleen een leverancier van muziekvideo's en een
promotietool van de platenindustrie. Het is het eerste echte mondiale
netwerk, dat vrijwel in elk land ter wereld een uniforme programmering
biedt. Daardoor creëert MTV een gedeeld gevoel van wereldwijde realiteit
voor zijn kijkers, kinderen en jongvolwassenen. Recent onderzoek toont
aan dat jongeren wereldwijd steeds vaker niet alleen gemeenschappelijke
popiconen en smaken delen, maar ook gelijke verwachtingen over hun
carrière, vergelijkbare waarden over wat zinvol is in het leven en wat
angst veroorzaakt, en een gemeenschappelijk besef dat politiek minder
belangrijk is bij het vormgeven van hun toekomst dan hun eigen
vaardigheden. {[}\^{}165{]} -- Jim Taylor and Watts Wacker, The 500-Year
Delta: What Happens After What Comes Next
\end{quote}

\subsection{``Houd ervan of vertrek'' (tenzij je rijk
bent)}\label{houd-ervan-of-vertrek-tenzij-je-rijk-bent}

Voordat de overgang van de natiestaat naar de nieuwe soevereiniteiten
van het Informatietijdperk voltooid is, zullen veel inwoners van de
grootste en machtigste westerse natiestaten, net als hun tegenhangers in
Oost-Berlijn in 1989, plannen maken om hun eigen weg te vinden. Voor de
generaties die volwassen werden vóór de Tweede Wereldoorlog, of vroeg in
de Koude Oorlog, is het oversteken van grenzen traumatisch. Maar voor
nieuwe generaties, die hun oriëntatie baseren op een meer globale blik,
is het verlaten van het land van geboorte niet zo ondenkbaar als voor de
ouderen, die dieper geworteld zijn in de ideologie van de natiestaat.
Jim Taylor en Watts Wacker rapporteren de intrigerende resultaten van
een grootschalige enquête onder 20.000 middenklassers op vijf
continenten. In een steekproef tijdens het schooljaar 1995-96,
uitgevoerd door Brainwaves Group, een consumentenonderzoeksbureau uit
New York, was negen van de tien leerlingen het eens met de uitspraak:
``het is aan mij om uit het leven te halen wat ik wil.'' Opvallender
nog: ``bijna de helft van de tieners zei dat ze verwachten het land van
hun geboorte te verlaten om hun doelen na te streven.'' Misschien omdat
hij, als eerste presidentskandidaat die campagne voerde op MTV, de
mentaliteit van de MTV-generatie goed aanvoelde, heeft Bill Clinton
geprobeerd het voor Amerikanen moeilijker te maken om ``het land van hun
geboorte te verlaten om hun doelen na te streven.'' In 1995, ongeveer op
hetzelfde moment dat de middelbare scholieren hun intenties uitspraken,
stelde de president van de Verenigde Staten de invoering van een
exitbelasting voor, een ``Berlijnse Muur voor kapitaal,'' die rijke
Amerikanen zou verplichten een aanzienlijk bedrag aan losgeld te betalen
om zelfs een deel van hun geld mee te nemen.

Clintons losgeld doet niet alleen denken aan het late Oost-Duitse beleid
om haar burgers als activa te behandelen, maar het herinnert ook aan de
steeds strengere maatregelen die genomen werden om de financiële positie
van het vervallende Romeinse Rijk te versterken. Dit fragment uit
\emph{The Cambridge Ancient History} vertelt het verhaal.

\begin{quote}
Zo begon de felle poging van de staat om de bevolking tot de laatste
druppel uit te knijpen. Omdat de economische middelen tekortschoten,
vochten de sterken om het grootste deel voor zichzelf veilig te stellen,
met een geweld en gewetenloosheid die goed pasten bij hun oorsprong en
bij een soldaat die gewend was aan plundering. De wet werd
onverbiddelijk hard toegepast op de bevolking. Soldaten traden op als
deurwaarders of zwierven als geheime politie door het land. Degenen die
het meest leden, waren natuurlijk de bezittende klasse. Het was relatief
gemakkelijk hun eigendommen in beslag te nemen, en in een noodsituatie
waren zij de klasse waar het meest en het snelst iets van af te persen
viel.
\end{quote}

Wanneer falende systemen daartoe in staat zijn, leggen ze vaak zware
sancties op aan degenen die proberen te ontsnappen. We citeren opnieuw
\emph{The Cambridge Ancient History}: ``Als de vermogende klasse haar
geld verstopte, of twee derde van haar bezit opofferde om een
magistratuur te ontvluchten, of zelfs haar gehele eigendom opgaf om
verlost te worden van de erfpacht, en de niet-vermogende klasse
vluchtte, reageerde de staat door de druk te verhogen.''

Houd dit in het achterhoofd wanneer je plannen maakt voor de toekomst.
De ondergang van staatsystemen verliep in het verleden zelden op
beleefde of ordelijke wijze. We noemden in hoofdstuk 2 de nare gewoonten
van Romeinse belastinginners. Het grote aantal verlaten boerderijen,
agri deserti, in West-Europa na de val van het Romeinse Rijk
weerspiegelde slechts een klein deel van een breder probleem. In feite
waren de afpersingen relatief mild in Gallië en in de grensgebieden die
het huidige Luxemburg en Duitsland omvatten. In Rome's meest vruchtbare
regio, Egypte, waar de landbouw productiever was door irrigatie, was het
vertrek van eigenaren een nog groter probleem. De vraag of men moest
proberen te ontsnappen, het ultimum refugium zoals het in het Latijn
werd genoemd, werd de overheersende twijfel van bijna iedereen met
bezit. Archieven tonen aan dat ``onder de gewone vragen die men in
Egypte aan een orakel stelde, drie standaardtypen waren: `Word ik
bedelaar?', `Zal ik vluchten?' en `Zal mijn vlucht worden
tegengehouden?'\,''

Clintons voorstel bevestigt die vragen. Het betreft immers een vroege
vorm van een ontsnappingsdrempel die naar verwachting zwaarder wordt
naarmate de fiscale middelen van de natiestaat krimpen. In tegenstelling
tot Honeckers betonnen afscheidingen en prikkeldraad is de Amerikaanse
vertrekbarrière uiteraard zachter en is het vooral gericht op
miljardairs met belastbare bezittingen boven \$600.000. Desalniettemin
zijn de argumenten ervoor erg vergelijkbaar met die van Honecker, die
ooit het beroemdste openbare werk van de inmiddels verdwenen Duitse
Democratische Republiek verdedigde. Honecker betoogde dat de Oost-Duitse
staat fors had geïnvesteerd in haar potentiële vluchtelingen, en dat het
economisch nadeel voor de staat zou veroorzaken als ze zouden
vertrekken. De staat rekende immers op hun inzet in Oost-Duitsland.

Als je uitgaat van het idee dat mensen eigendom zijn of zouden moeten
zijn van de staat, dan is Honeckers muur logisch. Berlijn zonder muur
was voor de communisten een maas in de wet, net zoals ontsnapping aan de
Amerikaanse belastingjurisdictie een maas in de wet was in de ogen van
Clintons belastingdienst. Naast het feit dat Clintons opmerkingen over
miljardairs die het land verlaten de typische politieke slordigheid met
cijfers toonden, waren ze vergelijkbaar met Honeckers, maar minder
steekhoudend, aangezien de VS nauwelijks economisch geïnvesteerd heeft
in rijke burgers die willen vluchten. Het gaat hier niet om mensen die
op staatskosten zijn opgeleid en weg willen glippen om elders als
advocaat te gaan werken. De overgrote meerderheid van degenen die onder
de exitbelasting zouden vallen, hebben hun vermogen door eigen
inspanning vergaard, ondanks, en niet dankzij, de Amerikaanse overheid.

Nu de hoogste 1 procent van de belastingbetalers 28,7 procent van de
totale inkomstenbelasting in de VS betaalt, is er geen sprake van dat de
rijken nalaten een reële investering van de staat in hun opleiding of
economische voorspoed terug te betalen. Integendeel. Degenen die het
grootste deel van de rekening betalen, betalen veel meer dan de waarde
van de voordelen die zij ontvangen. Met een gemiddelde jaarlijkse
belastingafdracht van meer dan \$125.000 kost de belasting de hoogste 1
procent van de Amerikaanse belastingbetalers veel meer dan zij beseffen.
Als we ervan uitgaan dat men over een periode van veertig jaar jaarlijks
10 procent rendement zou kunnen behalen op elke \$5.000 aan te veel
betaalde belasting, dan betekent dat aan het einde van die periode een
vermindering van het potentiële vermogen met \$2,2 miljoen. Bij een
rendement van 20 procent vermindert elke \$5.000 aan overbelasting het
nettovermogen met maar liefst \$44 miljoen.

Naarmate het millennium nadert, zullen de nieuwe megapolitieke
omstandigheden van het Informatietijdperk het steeds duidelijker maken
dat de natiestaat, geërfd uit het Industriële Tijdperk, een roofzuchtige
instelling is. Met ieder jaar dat voorbijgaat, zal zij minder als een
zegen voor de welvaart blijken en meer als een obstakel, iets waaraan
het individu wil ontsnappen. Het is die ontsnapping die wanhopige
regeringen niet graag zullen toestaan. De stabiliteit en zelfs het
voortbestaan van de westerse verzorgingsstaten hangt af van hun vermogen
om een enorm deel van de totale wereldproductie te blijven onttrekken
voor herverdeling aan een deel van de kiezers in de OESO-landen. Dit
vereist dat de belastingen die worden opgelegd aan de meest productieve
burgers van de huidige rijke landen worden vastgesteld op
supermonopolietarieven, honderden of zelfs duizenden keren hoger dan de
werkelijke kosten van de diensten die overheden daarvoor leveren.

\section{Het leven en sterven van de
natiestaat}\label{het-leven-en-sterven-van-de-natiestaat}

De val van de Berlijnse Muur was meer dan slechts een zichtbaar symbool
van het einde van het communisme. Het betekende een nederlaag voor het
mondiale systeem van natiestaten en een overwinning voor efficiëntie en
marktwerking. De machtsbasis die eens de geschiedenis bepaalde, is
ingrijpend verschoven. Wij zijn ervan overtuigd dat de val van de
Berlijnse Muur in 1989 het hoogtepunt betekende van het tijdperk van de
natiestaat -- een merkwaardige tweehonderdjarige periode die begon met
de Franse Revolutie. Staten bestaan al meer dan zesduizend jaar, maar
vóór de negentiende eeuw vertegenwoordigden zij slechts een klein deel
van de wereldwijde soevereine entiteiten. Hun opkomst begon en eindigde
met revoluties. De grote gebeurtenissen van 1789 dreven Europa tot het
vormen van werkelijk nationale regeringen. De grote gebeurtenissen van
1989 markeerden de dood van het communisme en de overwinning van
marktwerking op geconcentreerde macht. Die twee revoluties, precies
tweehonderd jaar uit elkaar, markeren het tijdperk waarin de natiestaat
overheerste in het systeem van de grote mogendheden. De grote
mogendheden beheersten op hun beurt de wereld en verspreidden of legden
staatsstelsels op, zelfs aan de meest afgelegen tribale enclaves.

Dat de staat uitgroeide tot het voornaamste middel om geweld te
organiseren, kwam niet door ideologie maar door de harde logica van
geweld. Het was, zoals wij het noemen, een megapolitieke gebeurtenis,
niet zozeer gedreven door de wensen van theoretici en staatslieden, of
zelfs door het gemanoeuvreer van generaals, maar door de verborgen
hefboom van geweld, die de loop van de geschiedenis beïnvloedde zoals
Archimedes de wereld had willen verplaatsen.

Staten zijn gedurende de afgelopen tweehonderd jaar van de moderne
periode de norm geweest, maar in het langere verloop van de geschiedenis
waren ze zeldzaam. Ze waren voor hun levensvatbaarheid altijd
afhankelijk van uitzonderlijke megapolitieke omstandigheden. Vóór de
moderne periode waren de meeste staten ``oosterse despotieën'',
agrarische samenlevingen in woestijnen die afhankelijk waren van de
controle over irrigatiesystemen. Zelfs het Romeinse Rijk was, via haar
controle over Egypte en Noord-Afrika, indirect een hydraulische
samenleving, maar niet genoeg om te overleven. Rome ontbeerde, net als
de meeste premoderne staten, uiteindelijk de capaciteit om het
geweldsmonopolie te handhaven, dat berust op de macht om mensen te
kunnen uithongeren. De Romeinse staat kon het water voor
landbouwgewassen buiten Afrika niet afsluiten door ongehoorzame mensen
de toegang tot het irrigatiesysteem te ontzeggen. Geen enkele
megapolitieke factor in de economie van de oudheid versterkte de macht
van geweld zo sterk als de aanwezigheid van zulke hydraulische systemen.
Wie in deze samenlevingen controle had over het water, kon middelen
buitmaken op een bijna vergelijkbaar niveau als het percentage van de
totale productie dat moderne natiestaten onttrekken.

\subsection{Grootte boven
efficiëntie}\label{grootte-boven-efficiuxebntie}

Buskruit maakte het voor staten eenvoudiger om zich uit te breiden
voorbij de grenzen van rijstvelden en droge rivierdalen. De aard van
buskruitwapens en de opbouw van de industriële economie brachten flinke
schaalvoordelen in de oorlogsvoering met zich mee. Dit leidde tot hoge
én steeds verder toenemende opbrengsten van geweld. Zoals historicus
Charles Tilly het verwoordde: `{[}S{]}taten met de meeste dwangmiddelen
wonnen de oorlogen; efficiëntie (de verhouding tussen output en input)
kwam op de tweede plaats na effectiviteit (totale output).' Aangezien
regeringen zich vrijwel altijd op grote schaal organiseerden, hadden
zelfs de weinige kleine soevereiniteiten, zoals Monaco of Andorra, de
erkenning van de grotere staten nodig om hun onafhankelijkheid te
waarborgen. Alleen grote overheden met een steeds grotere controle over
middelen konden op het slagveld concurreren.

\subsection{De grote onbeantwoorde
vraag}\label{de-grote-onbeantwoorde-vraag}

Dit brengt ons bij een van de grote onopgeloste raadsels van de moderne
geschiedenis: waarom de Koude Oorlog, die volgde op het systeem van
grote mogendheden, uiteindelijk communistische dictaturen tegenover
democratische verzorgingsstaten plaatste. Dit vraagstuk is zo weinig
onderzocht dat het voor velen zelfs aannemelijk leek toen een analist
van het Amerikaanse ministerie van Buitenlandse Zaken, Francis Fukuyama,
na de val van de Berlijnse Muur ``het einde van de geschiedenis''
uitriep. Het enthousiaste publiek dat zijn werk toejuichte, nam te veel
als vanzelfsprekend aan. Blijkbaar hadden noch de auteur, noch vele
anderen de fundamentele vraag gesteld: welke gemeenschappelijke
kenmerken van staatssocialisme en democratische verzorgingsstaten
maakten dat zij de laatste kandidaten voor wereldheerschappij werden?

Dit is een belangrijk vraagstuk. In de afgelopen vijf eeuwen zijn immers
tientallen concurrerende soevereiniteitsstelsels gekomen en gegaan,
waaronder absolute monarchieën, tribale enclaves, prins-bisdommen,
direct pauselijk gezag, sultansrijken, stadstaten en wederdoperkolonies.
Het is nu nauwelijks voor te stellen dat een ziekenhuisorganisatie,
uitgerust met eigen strijdkrachten, eeuwenlang de macht over een land
kon uitoefenen. Toch is er iets dergelijks gebeurd. Vanaf 1228 regeerde
de Duitse Orde van het Sint-Mariaziekenhuis te Jeruzalem, later verenigd
met de Zwaardbroeders van Lijfland, driehonderd jaar lang over
Oost-Pruisen en diverse gebieden in Oost-Europa, waaronder delen van
Litouwen en Polen. Toen kwam de Buskruitrevolutie. Binnen enkele
decennia werd de Duitse Orde uit al hun gebieden verdreven en hun
grootmeester had militair niet meer belang dan een schaakkampioen.
Waarom? Waarom raakten zoveel andere soevereiniteitsvormen in verval,
terwijl aan het einde van het Industriële Tijdperk de grote machtsstrijd
om de wereldheerschappij werd uitgevochten tussen massademocratieën en
staatssocialistische systemen?

\subsection{Onbelemmerde controle}\label{onbelemmerde-controle}

Als onze theorie van de megapolitiek klopt, is het antwoord eenvoudig.
Het is vergelijkbaar met de vraag waarom sumoworstelaars meestal dik
zijn. Het antwoord is dat een magere sumoworstelaar, hoe indrukwekkend
zijn verhouding tussen kracht en gewicht ook is, niet kan concurreren
met een andere worstelaar die gigantisch is. Zoals Tilly stelt, ging het
om ``effectiviteit (totale output)'', niet om ``efficiëntie (de
verhouding tussen output en input)''. In een wereld die steeds
gewelddadiger werd, domineerden de systemen die de beste toegang gaven
tot middelen om op grote schaal oorlog te voeren.

Hoe werkte dat?

In het geval van het communisme is het antwoord duidelijk. Onder het
communisme hadden degenen die de staat beheersten controle over vrijwel
alles. Tijdens de Koude Oorlog kon de KGB de tandenborstel van burgers
van de Sovjet-Unie afpakken als ze dat nuttig achtten. Ze hadden zelfs
je tanden kunnen afpakken. Volgens geloofwaardige schattingen, die nog
aannemelijker zijn geworden sinds de opening van voormalige
Sovjetarchieven in 1992, namen de geheime politie en andere agenten van
de late Sovjetstaat in vierenzeventig jaar tijd het leven van 50 miljoen
mensen. Het staatssocialistische systeem kon vrijwel alles binnen haar
grenzen mobiliseren voor haar leger, met weinig kans dat iemand daar
bezwaar tegen zou maken.

In het geval van westerse democratieën ligt het minder voor de hand,
deels omdat we geneigd zijn democratie scherp te contrasteren met het
communisme. In termen van het Industriële Tijdperk waren de twee
systemen inderdaad grote tegenpolen. Maar vanuit het perspectief van het
Informatietijdperk hadden ze meer gemeen dan je zou denken. Beide
maakten ongehinderde overheidscontrole mogelijk. Het verschil was dat de
democratische verzorgingsstaat zelfs meer middelen in handen van de
staat bracht dan de staatssocialistische systemen.

Dit is een duidelijk voorbeeld van het zeldzame verschijnsel waarbij
minder juist meer is. Het staatssocialistische systeem ging uit van het
idee dat de staat alles bezat. De democratische verzorgingsstaat
daarentegen maakte meer bescheiden aanspraken. Dit creëerde betere
prikkels waardoor een grotere productie kon gerealiseerd worden. In
plaats van vanaf het begin alles op te eisen, lieten westerse regeringen
individuen eigendom bezitten en rijkdom vergaren. Vervolgens, nadat die
rijkdom was opgebouwd, begonnen de westerse natiestaten er een groot
deel van te belasten. Onroerendgoedbelasting, inkomstenbelasting en hoge
successierechten leverden de democratische verzorgingsstaat enorme
hoeveelheden middelen op, vergeleken met wat de staatssocialistische
systemen konden genereren.

\subsection{Inefficiëntie, waar het ertoe
deed}\label{inefficiuxebntie-waar-het-ertoe-deed}

Vergeleken met het communisme werkte de verzorgingsstaat inderdaad veel
efficiënter. Echter, als je deze vergelijkt met andere systemen voor het
vergaren van rijkdom, zoals een echte laissez‑faire‑enclave als
\emph{Hongkong}, bleek de verzorgingsstaat op dat gebied minder
efficiënt te opereren. Opnieuw werd duidelijk: minder kan meer
betekenen. Juist deze ogenschijnlijke inefficiëntie stelde de
verzorgingsstaat tijdens de megapolitieke periode van het Industriële
Tijdperk in staat zich te onderscheiden.

Wie doorgrondt waarom, ziet veel duidelijker wat de val van de Berlijnse
Muur en het verdwijnen van het communisme echt inhouden. Het suggereerde
niet dat de democratische verzorgingsstaat de toekomst had, maar eerder
dat haar evenbeeld vredig aan ouderdom was gestorven. Dezelfde
megapolitieke revolutie die het communisme heeft gedood, zal
waarschijnlijk ook de democratische verzorgingsstaten, zoals we die in
de twintigste eeuw hebben gekend, ondermijnen en vernietigen.

\section{Wie bestuurt de overheid?}\label{wie-bestuurt-de-overheid}

Herkennen waar de controle over de democratische regering ligt, is de
sleutel tot deze onconventionele conclusie. Dit is een vraagstuk dat
minder eenvoudig is dan het lijkt. In de moderne tijd werd de vraag wie
controle heeft over de overheid bijna altijd gesteld als een politieke
vraag. Daarop zijn veel antwoorden gegeven, maar vrijwel altijd ging het
om het identificeren van de politieke partij, groep of factie die op een
bepaald moment de controle had over een bepaalde staat. Je hebt vast wel
gehoord van overheden die werden bestuurd door kapitalisten, door
arbeiders, door katholieken of door islamitische fundamentalisten, door
tribale en raciale groepen, door Hutu's en door blanken. Je zult ook
gehoord hebben van overheden bestuurd door beroepsgroepen, zoals
advocaten of bankiers, of door buitenstedelijke belangen, door
stedelijke machtsblokken en door mensen die in de buitenwijken wonen. En
zeker zul je hebben gehoord van overheden bestuurd door politieke
partijen, zoals de democraten, conservatieven, christendemocraten,
liberalen, radicalen, republikeinen en socialisten.

Je hebt echter waarschijnlijk niet veel gehoord over een overheid die
werd bestuurd door haar klanten. De economische historicus Frederic Lane
legde in enkele van zijn heldere essays over de economische gevolgen van
geweld die eerder aan bod kwamen de basis voor een nieuwe manier om te
doorgronden waar de controle over een overheid ligt. Lane kwam op het
idee om de overheid als een economische eenheid te beschouwen die
bescherming verkoopt. Door de overheid in economische termen te
analyseren in plaats van politieke, bedacht hij dat er drie fundamentele
alternatieven zijn voor wie controle heeft over de overheid, elk met
verschillende prikkels: eigenaren, werknemers en klanten.

\subsection{Eigenaren}\label{eigenaren}

In zeldzame gevallen, zelfs in de huidige tijd, leidt één eigenaar een
overheid, meestal een heerser met erfopvolging, die feitelijk het land
bezit. Zo beschouwt de sultan van Brunei de overheid van Brunei deels
als privébezit. In de middeleeuwen kwam dit vaker voor, toen heren hun
leengoed als privé-eigendom inzetten om hun inkomsten te maximaliseren.

Lane verwoordde de prikkels voor ``de eigenaren van de
productieonderneming'' als volgt:

\begin{quote}
Om zijn winst te vergroten, zou hij proberen de kosten te drukken zonder
de prijzen te verlagen. Hij zou, net als Hendrik VII van Engeland of
Lodewijk XI van Frankrijk, goedkope trucs, of liever nog, zo goedkoop
mogelijke middelen inzetten om zijn legitimiteit te bevestigen, de
binnenlandse orde te handhaven en de aandacht van naburige vorsten af te
leiden, zodat zijn eigen militaire uitgaven laag blijven. Hij behaalde
winst door kostenverlagingen, door hogere heffingen die dankzij de
kracht van zijn monopolie mogelijk waren, of door een combinatie van
beide.
\end{quote}

Overheden die door eigenaren worden bestuurd, hebben sterke prikkels om
de kosten voor het bieden van bescherming of het monopoliseren van
geweld in een bepaald gebied te drukken. Maar zolang hun heerschappij
onbetwist blijft, vinden zij weinig reden om de prijs (belasting) die
zij aan hun klanten vragen te verlagen tot onder het niveau dat de
opbrengsten maximaliseert. Hoe hoger de prijs die een monopolist kan
vragen en hoe lager zijn daadwerkelijke kosten, hoe groter de winst zal
zijn. Het ideale fiscale beleid voor een door eigenaren beheerde
regering leidt dan ook tot een forse winst. Wanneer overheden erin
slagen hun inkomsten hoog te houden en tegelijkertijd hun uitgaven te
beperken, heeft dat een grote invloed op het gebruik van middelen.
Arbeid en andere waardevolle productiefactoren, die anders verspild
zouden worden aan het leveren van onnodig dure bescherming, komen dan
beschikbaar voor investeringen en andere doeleinden. Hoe meer winst de
vorst boekt door kosten te drukken, hoe meer middelen er vrijkomen.
Worden die middelen voor investeringen ingezet, stimuleren ze de
economische groei. Zelfs wanneer ze enkel voor luxe-uitgaven worden
ingezet, stimuleren ze toch de vorming van nieuwe markten die niet
zouden ontstaan als de middelen waren verkwist aan inefficiënte
``bescherming.''

\subsection{Werknemers}\label{werknemers}

Het is eenvoudig om de prikkels te typeren die overheersen bij overheden
die door hun werknemers worden bestuurd. Het zijn vergelijkbare prikkels
als in andere door werknemers geleide organisaties. Allereerst hebben
door werknemers bestuurde organisaties de neiging elk beleid te steunen
dat de werkgelegenheid vergroot en maatregelen af te wijzen die banen
verminderen. Zoals Lane het formuleerde: ``Wanneer werknemers de macht
hadden, lag hun prioriteit niet bij het drukken van beschermingskosten
en al helemaal niet het bij beperken van hun eigen loonlast. Het
maximaliseren van de omvang van de organisatie had meer hun voorkeur.''
Een door werknemers bestuurde overheid zou zelden prikkels hebben om de
kosten van de overheid of de prijs die aan hun klanten wordt berekend te
verlagen. Wanneer men op sterke weerstand tegen de hoge prijzen stuit,
zijn dit soort overheden echter geneigd om hun inkomsten te laten dalen
tot onder het niveau van hun uitgaven in plaats van simpelweg minder uit
te geven. Met andere woorden, deze prikkels zorgen voor een neiging tot
chronische tekorten, in tegenstelling tot overheden die onder controle
van eigenaren staan.

\subsection{Klanten}\label{klanten}

Zijn er voorbeelden van regeringen die door hun klanten werden
gecontroleerd? Ja. Lane vond zijn inspiratie in steden als Venetië, waar
kooplieden de macht hadden, en gebruikte dat om staatscontrole in
economische termen te duiden. Daar had een groep groothandelaars, die
bescherming nodig hadden, gedurende eeuwen effectief de controle over de
overheid. Zij waren werkelijk klanten van de beschermingsdienst die de
overheid leverde, geen eigenaars. Ze betaalden voor de dienst en
probeerden niet te profiteren van hun controle over het geweldsmonopolie
van de staat. Als sommigen dat toch deden, weerhielden andere klanten
hen voor lange tijd ervan om dat nogmaals te doen. Andere voorbeelden
van door klanten bestuurde overheden zijn democratieën en republieken
met een beperkt kiesrecht, zoals de democratieën van de oudheid of de
Amerikaanse republiek in haar beginperiode. In die tijd mochten alleen
degenen die voor de overheid betaalden, ongeveer 10 procent van de
bevolking, stemmen.

Overheden die door hun klanten worden bestuurd, hebben net als die van
eigenaars de prikkel om hun operationele kosten zoveel mogelijk te
beperken. Maar in tegenstelling tot overheden die door eigenaars of
werknemers worden gecontroleerd, hebben regeringen die daadwerkelijk
door hun klanten worden bestuurd ook de prikkel om de prijzen die zij
rekenen laag te houden. Waar klanten de macht hebben, zijn overheden
klein en doorgaans onopvallend, met lage bedrijfskosten, een minimaal
aantal werknemers en lage belastingen. Een overheid onder controle van
haar klanten stelt de belastingdruk niet in om de overheidsinkomsten te
maximaliseren, maar om het resultaat van haar klanten te optimaliseren.

Zoals typische ondernemingen in concurrerende markten zou zelfs een
monopolie dat door haar klanten wordt bestuurd, gedwongen worden om
efficiënt te functioneren. Het zou geen belastingen kunnen heffen die de
kosten met meer dan een minimale marge overschreden.

\section{De rol van democratie: kiezers als werknemers en
klanten}\label{de-rol-van-democratie-kiezers-als-werknemers-en-klanten}

Lane bekijkt de democratie op de traditionele manier en gaat ervan uit
dat zij ervoor zorgt dat ondernemingen die geweld inzetten en produceren
steeds meer onder de controle van hun klanten komen. Natuurlijk lijkt
dat de politiek correcte conclusie, maar is dat wel zo? Wij twijfelen
daar sterk aan. Kijk eens goed hoe moderne democratieën nu werkelijk
functioneren.

In de eerste plaats missen ze de typische eigenschappen van competitieve
markten waarin klanten de voorwaarden dicteren. Democratische regeringen
besteden namelijk doorgaans slechts een fractie van hun totale uitgaven
aan hun kerntaak: bescherming. In de Verenigde Staten besteden de
deelstaten en lokale overheden bijvoorbeeld slechts 3,5 procent van hun
totale uitgaven aan politie, rechtbanken en gevangenissen. Tel de
militaire uitgaven daarbij op, en nog steeds gaat slechts ongeveer 10
procent van de inkomsten naar bescherming.

Een ander veelzeggend teken dat een massademocratie niet door haar
klanten wordt bestuurd, is het feit dat het in de hedendaagse politieke
cultuur, geërfd uit het Industriële Tijdperk, ondenkbaar zou zijn als
beleid inzake cruciale kwesties werkelijk zou worden bepaald door de
belangen van degenen die de rekeningen betalen. Stel je de
verontwaardiging voor als een Amerikaanse president of een Britse
premier zou voorstellen om de groep burgers die het merendeel van de
belastingen betaalt, te laten beslissen welke overheidsprogramma's
moeten doorgaan en welke groepen ambtenaren ontslagen zouden moeten
worden. Zo'n handelwijze zou indruisen tegen het fundamentele beeld van
wat een overheid behoort te doen, meer dan wanneer ambtenaren de macht
hadden om te bepalen wie extra belasting moest betalen.

Toch, als je erover nadenkt: wanneer klanten werkelijk aan het roer
staan, zou het juist schandalig worden gevonden als zij niet zouden
krijgen wat ze willen. Als je een winkel binnen zou lopen om meubels te
kopen, en de verkopers zouden je geld aannemen maar vervolgens negeren
ze je verzoeken en zouden anderen raadplegen over hoe jouw geld moest
worden besteed, zou je terecht verontwaardigd zijn. Je zou het ook niet
normaal of gerechtvaardigd vinden als de werknemers van de winkel zouden
betogen dat jij de meubels eigenlijk niet verdient, en dat deze in
plaats daarvan moeten worden geleverd aan iemand die zij meer geschikt
achten. Het feit dat iets zeer vergelijkbaars gebeurt in de omgang met
de overheid, laat zien hoe weinig controle haar ``klanten'' in
werkelijkheid hebben.

Hoe je het ook bekijkt, de kosten van democratische regeringen lopen
volledig uit de hand, in schril contrast met de omstandigheden in
competitieve markten, waar klanten de aanbieders dwingen efficiënt te
werken. De meeste democratieën kampen met structurele
begrotingstekorten. Dit is kenmerkend voor een fiscaal beleid dat
voortkomt uit de controle van de werknemers. Overheden blijken bijzonder
terughoudend als het gaat om het snijden in hun operationele kosten.
Wereldwijd horen we vrijwel altijd de klacht dat het erg moeilijk is om
eenmaal ingevoerde politieke programma's te stoppen. Ambtenaren ontslaan
blijkt vrijwel onmogelijk. Sterker nog, een van de belangrijkste
voordelen van de privatisering van functies, die voorheen door de staat
werden uitgevoerd, is dat particuliere controle het veel eenvoudiger
maakt om overbodige banen te schrappen. Van Groot-Brittannië tot
Argentinië laten nieuwe particuliere managers vaak 50 tot 95 procent van
de voorheen door de staat tewerkgestelde werknemers gaan.

Bedenk ook hoe de fiscale voorwaarden voor de beschermingsdienst van de
overheid worden bepaald. Je zult niet snel invloeden van competitie
vinden wanneer je de belastingtarieven analyseert. Zelfs de sporadische
discussies over belastingverlaging die de afgelopen jaren in het
politieke debat aan bod kwamen, tonen aan hoe ver een democratische
overheid doorgaans van controle door haar klanten is verwijderd.
Voorstanders van lagere belastingen betoogden soms dat de
overheidsinkomsten juist zouden stijgen, omdat de tarieven voorheen zo
hoog waren dat ze economische activiteit ontmoedigden.

De afweging die zij doorgaans wilden benadrukken, ging niet over
concurrentie tussen jurisdicties, maar over iets veel
verbazingwekkenders. Ze stelden niet dat de tarieven in de Verenigde
Staten of Duitsland niet hoger mochten zijn dan 15 procent, omdat de
belastingtarieven in Hongkong ook slechts 15 procent waren. Integendeel,
debatten over de belastingdruk gingen er meestal van uit dat de afweging
voor de belastingbetaler niet lag tussen zaken doen in de ene
jurisdictie of in een andere, maar tussen zaken doen tegen draconische
tarieven of een vakantie nemen. Er werd gezegd dat productieve
individuen die aan roofzuchtige belastingen onderworpen waren, hun werk
zouden neerleggen en gaan golfen als hun belastingdruk niet werd
verlicht.

Het feit dat zo'n argument zelfs kon ontstaan, laat zien hoe ver de
tarieven voor bescherming, die door democratische verzorgingsstaten
worden opgelegd, verwijderd zijn van een competitief uitgangspunt. De
voorwaarden van progressieve inkomstenbelasting, die in de loop van de
twintigste eeuw in elke democratische verzorgingsstaat ontstonden,
wijken dramatisch af van prijsbepalingen die door klanten zouden worden
geprefereerd. Het verschil valt op wanneer men belastingen voor een
beschermingsmonopolie afzet tegen de tarieven van telefoondiensten, die
tot voor kort overal monopolistisch waren. Klanten zouden moord en brand
schreeuwen als een telefoonbedrijf probeerde te factureren op dezelfde
basis als waarop inkomstenbelastingen worden geheven. Stel dat het
telefoonbedrijf een rekening van €50.000 stuurt voor een gesprek naar
Londen, alleen omdat je toevallig tijdens dat gesprek een deal van
€125.000 had gesloten. Noch jij, noch een andere klant met gezond
verstand zou dat betalen. Maar dat is precies de basis waarop
inkomstenbelastingen in elke democratische verzorgingsstaat worden
geheven.

Wanneer je goed kijkt naar de manier waarop industriële democratieën
hebben geopereerd, is het logischer om ze te beschouwen als een vorm van
overheid die door haar werknemers wordt bestuurd. Als je massademocratie
ziet als een door ambtenaren bestuurde overheid, wordt duidelijk waarom
beleidsveranderingen zo lastig zijn. De overheid lijkt in veel opzichten
bestuurd ten voordele van de werknemers. Zo lijken openbare ­scholen in
de meeste democratische landen chronisch slecht te functioneren, en
zonder zicht op verbetering. Als de klanten werkelijk aan het roer
stonden, zouden ze het gemakkelijker vinden om nieuwe beleidsrichtingen
uit te zetten. Degenen die voor de democratische overheid betalen,
bepalen zelden de voorwaarden van de overheidsuitgaven. In feite is de
overheid een coöperatief systeem dat geen privé-eigendom heeft en zich
gedraagt als een natuurlijk monopolie. Prijzen houden weinig verband met
kosten. De kwaliteit van de dienstverlening is over het algemeen laag in
vergelijking met die in de private sector. Klachten van klanten zijn
moeilijk te verhelpen. Kortom, massademocratie leidt tot controle over
de overheid door haar ``werknemers.''

Maar wacht. Je zou kunnen zeggen dat er in de meeste jurisdicties veel
meer kiezers zijn dan mensen op de loonlijst van de overheid. Hoe zou
het onder zulke omstandigheden mogelijk zijn dat werknemers domineren?
Dit probleem werd opgelost door het ontstaan van de verzorgingsstaat. Om
het aantal werknemers aan te vullen tot een werkbare meerderheid, werden
steeds meer kiezers praktisch op de loonlijst geplaatst door ze allerlei
vormen van staatssteun te geven. In feite werden de ontvangers van
uitkeringen en subsidies pseudo-werknemers van de overheid, die zich
konden ontdoen van het gedoe om dagelijks op het werk te verschijnen.
Het was een resultaat dat werd bepaald door de megapolitieke logica van
het Industriële Tijdperk.

Wanneer de omvang van het dwingende instituut belangrijker is dan de
efficiënte inzet van middelen, zoals het geval was vóór 1989, is het
vrijwel onmogelijk dat de meeste overheden door hun klanten worden
bestuurd. Zoals het voorbeeld van de late Sovjet-Unie zo duidelijk
aantoonde, was het tot een paar jaar geleden mogelijk voor staten om
grote macht in de wereld uit te oefenen, zelfs terwijl ze op enorme
schaal middelen verspilden. Wanneer de opbrengsten van geweld hoog en
stijgend zijn, telt omvang meer dan efficiëntie. Grotere entiteiten
hebben dan de neiging om te winnen van kleinere. De overheden die
effectiever zijn in het mobiliseren van militaire middelen, zelfs als
dat verspilling betekent, hebben dan de neiging te zegevieren boven
degenen die hun middelen efficiënter gebruiken.

Denk eens na over wat dit betekent. Het impliceert onontkoombaar dat
wanneer omvang belangrijker is dan efficiëntie, overheden die door hun
klanten worden bestuurd, niet kunnen zegevieren en vaak ook niet kunnen
overleven. Onder zulke omstandigheden zullen de entiteiten die militair
het meest effectief zijn degenen zijn die de meeste middelen voor oorlog
kunnen opeisen. Echter, regeringen die echt worden bestuurd door hun
betalende klanten, zouden waarschijnlijk niet onbeperkt middelen van
iedereen kunnen afromen

Klanten wensen normaal gesproken dat de prijzen die zij voor een product
of dienst betalen, inclusief voor bescherming, worden verlaagd en onder
controle worden gehouden. Als de westerse democratieën tijdens de Koude
Oorlog onder controle van de klant hadden gestaan, zou dat alleen al hun
militaire slagkracht hebben verminderd, doordat het vrijwel zeker de
toevoer van middelen naar de overheid zou hebben ingeperkt. Vergeet niet
dat in situaties waarin klanten heersen, zowel de prijzen als de kosten
nauwlettend beheerst dienen te worden. Maar dit is nauwelijks wat er
gebeurde. De verzorgingsstaten waren absolute winnaars van de
uitgavenwedstrijd tijdens de Koude Oorlog. Analisten van diverse
achtergronden stelden dat hun overwinning mede te danken was aan hun
capaciteit om zoveel geld uit te geven dat de Sovjet-Unie in financiële
problemen kwam.

Precies dit feit benadrukt hoe de inefficiënties van de democratie haar
op megapolitiek niveau dominant maakten tijdens een periode geweld
steeds meer opleverde. Massale militaire uitgaven vormen, met al hun
verspilling, duidelijk een suboptimale inzet van kapitaal voor het
opbouwen particuliere welvaart. We suggereerden eerder dat, hoewel
verzorgingsstaten economisch efficiënter waren in vergelijking tot
socialistische overheidssystemen, ze veel minder efficiënt zijn voor het
creëren van rijkdom dan laissez-faire enclaves, zoals Hongkong. Ironisch
genoeg was het juist deze inefficiëntie, vergeleken met een minder
belemmerd vrijemarktsysteem, die de democratische verzorgingsstaat
succesvol maakte onder de megapolitieke omstandigheden van het
industrialisme.

Hoe werd democratische inefficiëntie een succesfactor tijdens het
Tijdperk van Geweld? De sleutel tot het ontrafelen van deze schijnbare
paradox ligt in het erkennen van twee punten:

\begin{enumerate}
\def\labelenumi{\arabic{enumi}.}
\tightlist
\item
  Succes voor een soevereiniteit in de moderne periode lag niet in het
  creëren van rijkdom, maar in het creëren van een militaire macht die
  in staat was overweldigend geweld tegen elke andere staat in te
  zetten. Geld was nodig om dat te doen, maar geld zelf kon een veldslag
  niet winnen. De uitdaging was niet om een systeem met de efficiëntste
  economie of de hoogste groeisnelheid te creëren, maar een systeem dat
  meer middelen kon onttrekken om het vervolgens in het leger te
  kanaliseren. Militaire uitgaven zijn van nature een kostenpost waarvan
  de financiële opbrengsten op zichzelf laag of afwezig zijn.
\item
  De gemakkelijkste manier om toestemming te krijgen voor het investeren
  van geld in activiteiten met weinig of geen direct financieel
  rendement, zoals belastingbetalingen, is om toestemming te vragen aan
  iemand anders dan de persoon die het zal moeten opbrengen. Een van de
  manieren waarop de Nederlanders Manhattan konden kopen voor
  drieëntwintig dollar aan kralen, was dat de specifieke indianen aan
  wie ze het aanbod deden, het eigenlijk niet bezaten. ``Getting to
  yes'', zoals de marketingmensen zeggen, is onder die voorwaarden veel
  gemakkelijker. Stel bijvoorbeeld dat wij als auteurs van dit boek niet
  zouden willen dat u de kaftprijs, maar 40 procent van uw jaarlijkse
  inkomen zou betalen voor een exemplaar. We zouden veel eerder
  toestemming krijgen als we het aan iemand anders zouden vragen, en u
  niet zelf zouden hoeven te benaderen. Sterker nog, we zouden veel
  overtuigender kunnen zijn als we zouden kunnen vertrouwen op de
  instemming van meerdere mensen die u zelfs niet kent. We zouden een
  ad-hocverkiezing kunnen houden, wat H. L. Mencken, met minder
  overdrijving dan hij misschien dacht, beschreef als ``een geavanceerde
  veiling van gestolen goederen.'' En om het voorbeeld realistischer te
  maken, zouden we ermee kunnen instemmen om een deel van het door u
  betaalde geld te delen met deze anonieme omstanders in ruil voor hun
  steun.
\end{enumerate}

Dat is de rol die de moderne democratische verzorgingsstaat is gaan
vervullen. Het was een onovertroffen systeem in het Industriële Tijdperk
omdat het zowel efficiënt als inefficiënt was waar het ertoe deed. Het
combineerde de efficiëntie van particulier bezit en prikkels voor het
creëren van rijkdom met een mechanisme dat in wezen onbelemmerde toegang
tot die rijkdom faciliteerde. Democratie hield de portemonnees van
welvaartsproducenten open. Het kende militair succes in de periode
waarin het rendement van geweld wereldwijd op zijn hoogtepunt was, juist
omdat het het voor klanten moeilijk maakte om de belastingen die de
overheid inde, of andere manieren om middelen voor het leger te
financieren, zoals inflatie, te beperken.

\subsection{Waarom klanten niet konden
domineren}\label{waarom-klanten-niet-konden-domineren}

In de moderne periode waren degenen die voor ``bescherming'' betaalden
niet in staat hun middelen te onthouden van de soeverein, zelfs niet in
gezamenlijk verzet, omdat dit enkel zou hebben geleid tot overheersing
door andere, mogelijk nog vijandigere staten. Dit was een duidelijke
overweging tijdens de Koude Oorlog. De klanten, of belastingbetalers,
die een onevenredig groot deel van de kosten van de overheid in de
leidende westerse industriële staten droegen, hadden geen mogelijkheid
om hoge belastingen te weigeren. Het resultaat zou zijn geweest dat zij
zichzelf zouden blootstellen aan totale confiscatie door de Sovjet-Unie
of een andere agressieve groep met de macht om geweld te organiseren.

\subsection{Industrialisme en
democratie}\label{industrialisme-en-democratie}

Op langere termijn zou de massademocratie wel eens achterhaald idee
kunnen blijken te zijn dat het einde van het Industriële Tijdperk niet
lang zal overleven. Zeker is dat massademocratie en de natiestaat samen
opkwamen met de Franse Revolutie aan het einde van de achttiende eeuw,
waarschijnlijk als reactie op een stijging van het reële inkomen. Rond
1750 begonnen de inkomens in West-Europa in snel tempo te stijgen, deels
als gevolg van warmer weer. Dit viel samen met een periode van
technologische innovatie die de geschoolde arbeid van ambachtslieden
verdrong door machines die bediend konden worden door ongeschoolde
arbeiders, zelfs vrouwen en kinderen. Deze nieuwe industriële apparatuur
verhoogde de lonen van ongeschoolde arbeiders, waardoor de
inkomensverdeling gelijker werd.

Het cruciale keerpunt van de revolutie was wellicht niet, zoals vaak
wordt gedacht, het merkwaardige idee dat mensen geneigd zijn in opstand
te komen wanneer de omstandigheden verbeteren. Belangrijker was mogelijk
het feit dat toen de inkomens een bepaald niveau hadden bereikt, het
voor de vroegmoderne staat eindelijk praktisch werd om de particuliere
tussenpersonen en machtige magnaten, met wie eerder over middelen werd
onderhandeld, te omzeilen en over te stappen op een systeem van
``directe heerschappij'', waarin een nationale overheid rechtstreeks
zaken deed met individuele burgers, hen tegen steeds hogere tarieven
belastte en slecht gecompenseerde militaire dienst eiste in ruil voor de
verstrekking van diverse voordelen.

Omdat de opkomende middenklasse in relatief korte tijd genoeg geld had
verzameld om te belasten, was het voor de heersers niet langer
essentieel, zoals voorheen wel het geval was, om te onderhandelen met
machtige grondbezitters of kooplieden die, zoals de historicus Charles
Tilly schreef, ``in staat waren om de vorming van een machtige staat te
verhinderen'' die ``hun bezittingen in beslag zou nemen en hun
transacties zou inperken.'' Je kunt duidelijk zien waarom regeringen
succesvoller middelen konden onttrekken van de maatschappij toen zij te
maken kregen met miljoenen individuele burgers, in plaats van met een
handjevol heren, hertogen, graven, bisschoppen, huurlingen, vrije steden
en andere semi-soevereine entiteiten met wie de heersers van Europese
staten zich vóór het midden van de achttiende eeuw genoodzaakt zagen te
onderhandelen.

Stijgende reële inkomens stelden regeringen in staat een strategie te
hanteren die meer middelen onder hun controle bracht. Kleine bedragen
die via belastingen van miljoenen werden geïnd, konden meer inkomsten
opleveren dan grotere sommen die door enkele machtigen werden betaald.
Bovendien was het veel makkelijker om met de velen om te gaan dan met de
enkelen, die over het algemeen onwillig waren hun geld af te staan en
veel beter in staat waren zich te verzetten.

De doorsnee boer, kleine koopman of arbeider beschikte immers over
verwaarloosbaar weinig middelen in vergelijking met de staat zelf. Het
was aan de vooravond van de Franse Revolutie voor een doorsnee
particulier in West-Europa volstrekt onmogelijk om succesvol met de
staat te onderhandelen over het verlagen van zijn belastingtarief, of om
een doeltreffend verzet te organiseren tegen overheidsplannen en -beleid
die zijn belangen bedreigden. Maar dit is precies wat machtige
particuliere magnaten eeuwenlang hadden gedaan en zouden blijven doen.
Zij boden daadwerkelijk weerstand en onderhandelden met heersers,
waardoor hun vermogen om middelen op te eisen werd ingeperkt.

\begin{quote}
Oorlogsvoering versnelde de overgang van indirect naar direct bestuur.
Vrijwel elke staat die oorlog voert ontdekt dat hij de inspanning niet
kan bekostigen uit zijn opgebouwde reserves en lopende inkomsten. Bijna
alle oorlogvoerende staten lenen grootschalig, verhogen belastingen en
eisen de middelen voor de strijd, inclusief mannen, op van onwillige
burgers die hun middelen liever anders aanwenden. -- CHARLES TILLY
\end{quote}

Het voorbeeld van Polen in het midden van de achttiende eeuw illustreert
dit perfect. In 1760 telde het Poolse nationale leger achttienduizend
soldaten. Dit was een magere strijdmacht vergeleken met de legers van de
naburige machthebbers in Oostenrijk, Pruisen en Rusland, van wie het
kleinste staande leger al zo'n 100.000 man telde. In feite was het
Poolse nationale leger in 1760 zelfs klein in vergelijking met andere
eenheden binnen Polen. De gezamenlijke strijdkrachten van de Poolse adel
telden dertigduizend man.

Als de Poolse koning de mogelijkheid had gehad om direct belasting te
heffen bij miljoenen individuele Polen, in plaats van afhankelijk te
zijn van de bijdragen van de invloedrijke Poolse magnaten, zou de
centrale overheid ongetwijfeld aanzienlijk hogere inkomsten hebben
kunnen verwerven en dus een omvangrijker leger hebben kunnen
financieren.

Ten opzichte van individuele burgers, die niet de mogelijkheid hadden om
collectief met miljoenen tegelijk op te treden, bleken centrale
autoriteiten overal onoverwinnelijk machtig. De koning van Polen had in
1760 echter niet de mogelijkheid zijn burgers rechtstreeks te belasten.
Hij moest zijn toevlucht nemen tot de heren, rijke kooplieden en andere
notabelen, die een kleine, hechte groep vormden. Zij hadden wél de
mogelijkheid om gezamenlijk op te treden om de koning ervan te
weerhouden hun middelen zonder hun toestemming op te eisen. Aangezien de
Poolse adel over veel meer troepen beschikte dan hijzelf, stond de
koning niet sterk genoeg om zijn zin door te duwen.

Uiteindelijk bleek dit militaire nadeel, het onvermogen om de rijken en
machtigen te omzeilen bij het vergaren van middelen, doorslaggevend in
het Tijdperk van Geweld. Binnen enkele jaren hield Polen op te bestaan
als onafhankelijke staat. Het werd veroverd door Oostenrijk, Pruisen en
Rusland, drie landen met legers die elk vele malen groter waren dan de
kleine Poolse strijdmacht. In elk van die landen hadden de heersers
manieren gevonden om de mogelijkheden van rijke kooplieden en de adel om
hun middelen te beschermen te omzeilen.

\subsection{Na de Franse Revolutie}\label{na-de-franse-revolutie}

De Franse Revolutie leidde tot een nog grotere toename van de omvang van
legers, een feit dat de kracht van de democratische strategie aantoonde
in een periode waarin het rendement van geweld toenam. De afspraak die
regeringen vanaf de Franse Revolutie maakten, was om een ongekende mate
van betrokkenheid in het leven van gewone mensen te bieden, in ruil voor
hun deelname aan oorlogen, in plaats van huurlingen, en het belasten van
een steeds groter deel van hun stijgende inkomens. Zoals Tilly zei:

\begin{quote}
``De reikwijdte van de staat breidde zich ver buiten het militaire
kerngebied uit, en haar burgers begonnen aanspraken te maken op een zeer
breed scala aan bescherming, rechtspraak, productie en distributie.
Naarmate nationale wetgevers hun bereik ver voorbij het goedkeuren van
belastingen uitbreidden, werden ze geconfronteerd met eisen van alle
goed georganiseerde belangengroepen die door de staat werden geraakt of
geraakt konden worden. Direct bestuur en massale nationale politiek
groeiden samen en versterkten elkaar.''
\end{quote}

Dezelfde logica die in de achttiende eeuw gold, bleef gelden tot 1989,
toen de Berlijnse Muur viel. Naarmate het Industriële Tijdperk vorderde,
bleven de inkomens voor ongeschoold werk stijgen, waardoor
massademocratie een nog effectievere methode werd om de onttrekking van
middelen te optimaliseren. Als gevolg hiervan groeide de overheid
voortdurend, met jaarlijks gemiddeld ongeveer een half procent extra
belasting op het inkomen in een gemiddeld industrieel land gedurende de
twintigste eeuw.

Tijdens het Industriële Tijdperk voorafgaand aan 1989 kwam democratie
naar voren als de meest effectieve regeringsvorm op militair vlak, juist
omdat democratie het moeilijk of onmogelijk maakte om effectieve grenzen
te stellen aan de confiscatie van middelen door de staat. De ruime
invoering van sociale voorzieningen aan iedereen nodigde een meerderheid
van de kiezers uit om feitelijk ambtenaar van de overheid te worden. Dit
werd het overheersende politieke kenmerk van alle leidende industriële
landen, omdat kiezers een onvoldoende sterke positie hadden om hun rol
als klant voor bescherming te vervullen. Niet alleen stonden ze
tegenover de agressieve dreiging van communistische systemen, die grote
middelen voor militaire doeleinden konden genereren omdat de staat de
gehele economie beheerde, maar echte controle over de overheid was als
belastingbetaler ook om een andere reden onpraktisch.

Miljoenen gewone burgers kunnen niet effectief samenwerken om hun
belangen te beschermen. Omdat de obstakels voor hun samenwerking groot
zijn en het verdedigen van de gemeenschappelijke belangen van de groep
op individueel niveau weinig oplevert, zullen miljoenen gewone burgers
minder succesvol het hoofd kunnen bieden tegen de overheid als dat
kleinere groepen met gunstigere prikkels dat kunnen.

Andere dingen gelijk, zou je dus verwachten dat een groter deel van de
totale middelen door de overheid wordt opgeëist in een massademocratie
dan in een oligarchie of in een systeem van gefragmenteerde
soevereiniteit waar magnaten militaire macht hadden en hun eigen legers
op de been brachten, zoals dat overal in het vroegmoderne Europa vóór de
achttiende eeuw het geval was.

Een cruciale, zij het zelden onderzochte reden voor de groei van
democratische overheden in de westerse wereld is dus het relatieve
belang van onderhandelingskosten tijdens een periode waarin geweld
steeds meer loonde. Het kostte gewoon meer om middelen bij enkelen te
halen dan bij velen.

Een relatief kleine elitegroep van rijken vormt een coherenter en
effectiever geheel dan een grote massa burgers. De kleine groep heeft
sterkere prikkels om samen te werken. Zij zal bijna onvermijdelijk
succesvoller zijn in het beschermen van haar belangen dan een grote
massa. En zelfs als de meeste leden van de groep ervoor kiezen niet
samen te werken aan een gezamenlijke actie, kunnen een paar rijken
genoeg middelen inzetten om het werk gedaan te krijgen.

Met democratische besluitvorming kon de natiestaat veel vollediger macht
uitoefenen over miljoenen personen, die niet gemakkelijk konden
samenwerken om collectief hun belangen te verdedigen, terwijl een
kleinere groep de organisatorische hindernissen gemakkelijker kon
overwinnen en zo hun geconcentreerde belangen kon beschermen. Democratie
bood het extra voordeel van een gelegitimeerd besluitvormingsmechanisme,
waarmee de staat toegang kreeg tot de middelen van de rijken zonder dat
hun directe instemming nodig was. Het vulde de natiestaat aan omdat het
de concentratie van militaire macht in de handen van de machthebbers
vergemakkelijkte, in een tijd waarin de omvang van de ingezette kracht
belangrijker was dan de efficiëntie van de mobilisatie.

De Franse Revolutie bewees dit duidelijk. Doordat het de omvang van de
militaire macht op het slagveld vergrootte, hadden andere concurrerende
natiestaten weinig andere keuze dan een vergelijkbare organisatie na te
streven, waarvan de legitimiteit uiteindelijk verbonden werd aan
democratische besluitvorming. Samengevat: de democratische natiestaat
slaagde in de afgelopen twee eeuwen om deze verborgen redenen:

\begin{enumerate}
\def\labelenumi{\arabic{enumi}.}
\tightlist
\item
  Geweld loonde steeds meer, waardoor de schaal van de macht
  belangrijker werd als bestuursprincipe dan efficiëntie.
\item
  Inkomens stegen voldoende boven het bestaansminimum, waardoor de staat
  de mogelijkheid kreeg om grote hoeveelheden aan middelen te innen
  zonder te hoeven onderhandelen met machtige magnaten die weerstand
  konden bieden.
\item
  Democratie bleek voldoende verenigbaar met de werking van vrije
  markten om het toenemen van de welvaart vol te kunnen houden.
\item
  Democratie faciliteerde een dominante overheid, bestuurd door haar
  ``werknemers'', waardoor het moeilijk werd de uitgaven, inclusief
  militaire uitgaven, te beperken.
\item
  Democratische besluiten bleken een effectief middel tegen het vermogen
  van de rijken om gezamenlijk op te treden tegen het vermogen van de
  natiestaat om belastingen te heffen of op andere manieren inbreuk te
  maken op hun eigendom.
\end{enumerate}

Democratie werd de winnende militaire strategie omdat het het
vergemakkelijkte om meer middelen in handen van de staat te brengen. In
vergelijking met andere vormen van soevereiniteit die hun legitimiteit
ontleenden aan andere principes, zoals de feodale heffing, het goddelijk
recht van koningen, religieuze plicht of de vrijwillige bijdragen van de
rijken, werd massademocratie militair de krachtigste, omdat het de meest
zekere manier was om middelen te verzamelen in een industriële economie.

\begin{quote}
De natie, als een cultureel gedefinieerde gemeenschap, is de hoogste
symbolische waarde van de moderniteit; ze is gezegend met een
quasi-heilig karakter dat alleen door religie wordt geëvenaard. In feite
is dit quasi-heilige karakter afgeleid van religie. In de praktijk is de
natie ofwel de moderne, seculiere vervanger van religie geworden, of
haar machtigste bondgenoot. In de moderne tijd worden de
gemeenschapsgevoelens die door de natie worden opgewekt hoog gewaardeerd
en gezocht als basis voor groepsloyaliteit\ldots{} Dat de moderne staat
hier vaak de begunstigde van is, hoeft nauwelijks te verbazen, gezien
haar opperste macht. -- JOSEPH R. LLOBERA
\end{quote}

\subsection{Nationalisme}\label{nationalisme}

Veel van hetzelfde kan gezegd worden van nationalisme, dat een
uitvloeisel werd van massademocratie. Staten die nationalisme konden
inzetten, ontdekten dat ze grotere legers tegen lagere kosten konden
mobiliseren. Nationalisme was een uitvinding die het voor een staat
mogelijk maakte om haar militaire effectiviteit op te schalen. Net als
de politiek zelf is nationalisme grotendeels een moderne uitvinding.
Zoals socioloog Joseph Llobera in zijn rijk gedocumenteerde boek over de
opkomst van nationalisme heeft aangetoond, is de natie een ingebeelde
gemeenschap die in grote mate tot stand kwam als een manier om
staatsmacht te mobiliseren tijdens de Franse Revolutie. Zoals hij stelt:
``In de moderne betekenis van de term bestaat nationaal bewustzijn pas
sinds de Franse Revolutie, sinds het moment waarop in 1789 de
Grondwetgevende Vergadering het volk van Frankrijk gelijkstelde met de
Franse natie.''

Nationalisme maakte het eenvoudiger om macht te mobiliseren en controle
uit te oefenen over grote aantallen mensen. Natiestaten werden gevormd
door het benadrukken van kenmerken die mensen gemeen hadden, met name de
gesproken taal. Dit vergemakkelijkte heerschappij zonder tussenkomst van
tussenpersonen. Het vereenvoudigde de taken van de bureaucratie.
Besluiten die slechts in één taal hoefden te worden uitgevaardigd,
konden sneller en met minder verwarring worden verspreid dan degenen die
vertaald moesten worden in alle talen van de toren van Babel.
Nationalisme maakte het dus goedkoper om grotere gebieden te beheersen.
Voor het nationalisme had de vroegmoderne staat de hulp nodig van heren,
hertogen, graven, bisschoppen, vrije steden en andere corporatieve en
etnische tussenpersonen, van belastingpachters tot militaire
leveranciers en huurlingen, om inkomsten te innen, troepen te werven en
andere overheidstaken uit te voeren.

Nationalisme was ook doorslaggevend voor het verlagen van de kosten voor
het mobiliseren van militair personeel door groepsidentificatie met de
belangen van de staat aan te moedigen. Er was zo'n aanzienlijk voordeel
te behalen door groepsgevoel te koppelen aan de belangen van de staat
dat de meeste staten, zelfs de zogenaamd internationalistische
Sovjet-Unie, op nationalisme uitkwamen als aanvullende ideologie.

In een langere historische context is nationalisme net zo goed een
anomalie als de staat zelf. Zoals historicus William McNeill heeft
gedocumenteerd, waren poly-etnische soevereiniteiten vroeger de norm.

In McNeill's woorden: ``Het idee dat een overheid alleen rechtmatig over
burgers van een enkele ethnos zou moeten heersen, begon zich tegen het
einde van de Middeleeuwen in West-Europa te ontwikkelen.'' Een vroege
nationalistische entiteit was de Pruisische Liga (Preußischer Bund), die
in 1440 werd gevormd als verzet tegen het bewind van de Duitse Orde.
Sommige kenmerken van de orde zijn eerder genoemd als een uiterst
voorbeeld van een soevereiniteit die niet leek op de natiestaat. De
Duitse Orde was een soort gecharterde onderneming waarvan bijna geen
enkel lid afkomstig was uit Pruisen. Het hoofdkwartier verschoof in de
loop der tijd van Bremen en Lübeck naar Jeruzalem, Akko, Venetië en
uiteindelijk Marienberg aan de Wisła. Op een bepaald moment bestuurde
zij zelfs het district Burzenland in Transsylvanië. Het is dan ook niet
verrassend dat een soevereiniteit die zo weinig leek op een staat het
voorwerp werd van een van de eerste pogingen om nationaal gevoel te
mobiliseren als factor in de organisatie van macht. Het volgende laat
echter zien hoe anders het vroege nationalisme was ten opzichte van
latere varianten: de Duitstalige edelen van de Pruisische Liga
verzochten de koning van Polen om Pruisen onder Pools gezag te plaatsen,
grotendeels omdat de Poolse koning toen een relatief zwakke monarch was
waarvan niet verwacht werd dat hij met dezelfde strengheid zou regeren
als de Duitse Orde.

Nationalisme, in zijn vroege verschijningsvormen, kwam net voor de
Buskruitrevolutie op. Het bleef groeien naarmate de vroegmoderne staat
zich verder ontwikkelde, en werd erg belangrijk ten tijde van de Franse
Revolutie. Volgens ons heeft het nationalisme als krachtig idee zijn
terugtocht al ingezet. Het bereikte waarschijnlijk zijn hoogtepunt met
Woodrow Wilsons poging om aan elk etnisch volk in Europa een eigen staat
toe te wijzen na afloop van de Eerste Wereldoorlog. Tegenwoordig is het
een reactionaire kracht, aangewakkerd in regio's met dalende inkomens en
teruglopende vooruitzichten, zoals Servië.

Zoals we later zullen bespreken, verwachten wij dat nationalisme een
belangrijk onderwerp zal worden waar mensen met weinig vaardigheden,
nostalgisch naar dwang, zich achter zullen scharen wanneer de
welvaartsstaat instort in de westerse democratieën. Wat je nu al hebt
ervaren, stelt nog niets voor. Voor de meeste mensen in het Westen leek
de ondergang van het communisme relatief onschuldig. Je hebt een daling
van militaire uitgaven gezien, een instorting van de aluminiumprijzen,
en een nieuwe bron van ijshockeyspelers voor de NHL. Dat is het goede
nieuws. Nieuws dat de meeste mensen die in de twintigste eeuw volwassen
werden, konden toejuichen, zeker als ze fan waren van ijshockey. Het
grootste deel van het nieuws dat minder populair zal blijken, moet nog
komen.

Nu het Industriële Tijdperk tot zijn einde komt, houden de megapolitieke
omstandigheden die democratie mogelijk maakten snel op te bestaan.
Daarom is het twijfelachtig dat massademocratie en de verzorgingsstaat
lang zullen overleven in de nieuwe megapolitieke omstandigheden van het
Informatietijdperk.

\begin{quote}
Het Congres was geen tempel van de democratie, maar een markt om wetten
te verhandelen. -- ALBERTO FUJIMORI, president van Peru
\end{quote}

Toekomstige historici zouden zelfs kunnen melden dat we de eerste
postmoderne staatsgreep al hebben gezien, de opmerkelijke verzegeling
van het congres in Peru in 1993. Dit was nauwelijks een gebeurtenis die
veel aandacht trok in de leidende industriële democratieën. Maar het zou
achteraf belangrijker kunnen blijken dan conventionele analisten
vermoeden. De weinigen die erover nadachten, zagen het vooral als nog
een machtsgreep van het soort dat Latijns-Amerika doorheen de
geschiedenis treurig genoeg al vaak heeft gekend. Wij zien het echter
mogelijk als de eerste stap richting de delegitimering van een
bestuursvorm waarvan de onmiddellijke megapolitieke bestaansreden met de
overgang naar het Informatietijdperk is beginnen te verdwijnen.
Fujimori's sluiting van het congres is een symptoom van de uiteindelijke
waardevermindering van politieke beloften. Een soortgelijk lot kan
andere parlementen te wachten staan zodra hun krediet is uitgeput.

De technologische verschuiving die het industrialisme ondermijnt, heeft
veel landen opgescheept met regeringen die niet langer functioneren, of
slecht functioneren. Vooral parlementen lijken steeds disfunctioneler te
worden. Ze produceren wetten die vijftig jaar geleden misschien slechts
dom waren, maar die vandaag de dag gevaarlijk zijn. Dit was spectaculair
duidelijk in Peru, waar de interne soevereiniteit van de staat tegen
1993 vrijwel ingestort was.

\begin{quote}
Aanvallen, ontvoeringen, verkrachtingen en moorden gingen hand in hand
met steeds agressiever rijgedrag en onveilige straten. De politie heeft
geleidelijk de controle over de situatie verloren en sommige van haar
leden zijn betrokken geraakt bij schandalen en doorgewinterde criminelen
geworden\ldots{} Mensen zijn geleidelijk gewend geraakt om buiten de wet
te leven. Diefstal, illegale inbeslagnames en fabrieksbezettingen zijn
alledaagse verschijnselen geworden. -- HERNANDO DE SOTO
\end{quote}

\subsection{Peru in ruïnes}\label{peru-in-ruuxefnes}

In zekere zin was Peru in 1993 geen moderne natiestaat meer. Het had nog
een vlag en een leger, maar de meeste instituties lagen in puin. Zelfs
de gevangenissen waren overgenomen door de gedetineerden. Deze
desintegratie kent meerdere oorzaken, maar de meeste verklaringen van
deskundigen missen de kern. Peru viel namelijk al vroeg ten prooi aan de
technologische veranderingen die gesloten economieën disfunctioneel
maken en de centrale autoriteit ondermijnen. Bovendien verergeren deze
megapolitieke spanningen wanneer besluitvormingsorganen, zoals het
Peruaanse congres, door perverse prikkels worden gedwongen de problemen
eigenlijk te verergeren in plaats van op te lossen.

De representatieve democratie in Peru was te vergelijken met een paar
valse dobbelstenen. Democratie was ongeëvenaard als middel om
staatsmacht uit te breiden, maar toen veranderende omstandigheden een
verandering van de macht eisten, werden de eigenschappen die haar in de
oude megapolitieke context zo succesvol hadden gemaakt, een bron van
toenemend disfunctioneren. De wetten die het congres aannam, verzwakten
razendsnel elk fundament voor waarde en respect voor de wet. Zoals de
Soto in \emph{The Other Path} verwoordt: `Kleine belangengroepen ruziën
met elkaar, veroorzaken faillissementen en beïnvloeden ambtenaren,
overheden kennen privileges toe, en de wet wordt ingezet om veel meer te
geven en af te nemen dan de moraal toelaat.' Een congres dat, zoals in
Peru, volledig beheerst wordt door belangengroepen, heeft dezelfde
morele status als een bende die gestolen goederen veilt. Dit maakte de
vrije markt ontoegankelijk, waardoor de wet haar geloofwaardigheid
verloor. Zoals de Soto schrijft over de periode vóór Fujimori:

\begin{quote}
Een complete omkering van doelen en middelen heeft het maatschappelijke
leven in Peru zo verstoord dat bepaalde daden, hoewel wettelijk
strafbaar, niet langer door het collectieve bewustzijn worden
veroordeeld. Smokkelen is daar een treffend voorbeeld van. Iedereen, van
aristocratische dames tot de meest nederige mannen, verwerft gesmokkelde
goederen. Niemand heeft er gewetensbezwaren tegen; integendeel, men
beschouwt het als een test van persoonlijke vindingrijkheid of als een
vorm van wraak op de staat. Deze infiltratie van geweld en criminaliteit
in het dagelijks leven gaat gepaard met toenemende armoede en ontbering.
In het algemeen is het reële gemiddelde inkomen van de Peruanen de
afgelopen tien jaar gestaag gedaald en bevindt het zich nu op het niveau
van twintig jaar geleden. Overal hopen bergen afval zich op. Dag en
nacht omsingelen legioenen bedelaars, autowassers en aaseters de
voorbijgangers op zoek naar geld. Geestelijk zieken zwermen naakt door
de straten en stinken naar urine. Kinderen, alleenstaande moeders en
gehandicapten bedelen op elke straathoek om aalmoezen. Het traditionele
centralisme van onze samenleving heeft duidelijk niet voldaan aan de
vele behoeften van een land in transitie.
\end{quote}

De Soto omschreef het loslaten van de groteske juridische economie ten
gunste van de zwarte markt, een ontwikkeling die al in gang was gezet
voordat Fujimori het congres opsloot, als `een onzichtbare revolutie'.

Wij staan positief tegenover de voordelen van de vrije markt, maar zijn
minder enthousiast over een samenleving waarin de wet net zo weinig
waard is als het geld. De wereld die de Soto in Peru schetste vóór 1993,
was een `Clockwork Orange'-wereld waarin overmatig gecentraliseerde en
dysfunctionele overheidsinstellingen de maatschappij letterlijk
vernietigden.

Dit was immers wat Fujimori wilde veranderen. Hij verlaagde de inflatie
drastisch door de geldpersen stil te leggen. Ook slaagde hij erin
vijftigduizend ambtenaren te ontslaan en diverse subsidies in te perken.
Hij begon met het in evenwicht brengen van de begroting. Zijn
hervormingsprogramma omvatte uitgebreide plannen om vrije markten tot
stand te brengen en de industrie te privatiseren. Maar, net als in de
voormalige Sovjet-Unie, had hij in 1993 de meest cruciale elementen van
zijn hervorming, waaronder de eerste ronde grootschalige privatiseringen
van staatsbanken, mijnbouwbedrijven en nutsbedrijven, nog niet
gerealiseerd. In plaats van deze broodnodige maatregelen te realiseren,
probeerde het Peruaanse congres, net als het Russische congres dat
Jeltsins hervormingen in Moskou blokkeerde, een stap achteruit te
zetten. Hun plan luidde: subsidies herstellen vanuit een lege schatkist,
de lonen spekken en alle gevestigde belangen beschermen, met name de
bureaucratie; precies wat je zou verwachten van een overheid die door
haar werknemers wordt beheerst.

Fujimori beweerde dat het congres van Peru aarzelend en corrupt was, een
uitspraak waar vrijwel iedereen het mee eens was. Hij voegde eraan toe
dat de aarzelende en corrupte werkwijze in het congres iedere poging om
de ineenstortende economie van Peru te hervormen of de gewelddadige
aanvallen van narco-terroristen en nihilistische guerrilla's van
\emph{het Sendero Luminoso (Shining Path)} tegen te gaan, onmogelijk
maakte.

\subsection{De 70-procentoplossing}\label{de-70-procentoplossing}

Dus sloot Fujimori het Congres, een daad die erop kon wijzen dat hij
even autoritair was als veel voorgaande Latijns-Amerikaanse leiders.
Maar wij dachten, en zeiden dat destijds ook, dat Fujimori terecht een
fundamentele belemmering voor hervorming had geïdentificeerd. De
buitensporige officiële lofzangen op het Peruaanse congres door
Amerikaanse redactieschrijvers en functionarissen van het State
Department werden niet gedeeld door het volk van Peru. Terwijl
Noord-Amerikanen deden alsof het Peruaanse congres de belichaming van
vrijheid en beschaving was, juichte het Peruaanse volk. De populariteit
van president Fujimori steeg boven de 70 procent toen hij het congres
naar huis stuurde. En later werd hij met een verpletterende meerderheid
herkozen voor een tweede termijn. De meeste burgers zagen hun wetgevende
macht blijkbaar meer als een hindernis voor hun welzijn dan als een
belichaming van hun rechten. In 1994 bereikte de reële economische groei
in Peru 12,9 procent, de hoogste ter wereld.

\subsection{Deflatie van politieke
beloften}\label{deflatie-van-politieke-beloften}

Wij zagen de onrust in Peru niet als een terugkeer naar de dictaturen
van het verleden maar eerder als een vroeg voorbeeld van een bredere
overgangscrisis. Het is te verwachten dat in tal van landen crises van
wanbestuur ontstaan wanneer politieke beloften niet worden nagekomen en
overheden hun krediet uitgeput hebben. Uiteindelijk zullen nieuwe
institutionele vormen moeten ontstaan die in staat zijn de vrijheid te
bewaren onder de nieuwe technologische omstandigheden, terwijl zij
tegelijkertijd uitdrukking en inhoud geven aan de gemeenschappelijke
belangen die alle burgers delen.

Bijna niemand denkt nog na over de onverenigbaarheid tussen sommige
instituties van de industriële overheid en de megapolitiek van de
postindustriële samenleving. Of deze tegenstrijdigheden nu expliciet
worden erkend of niet, hun gevolgen zullen steeds duidelijker worden
naarmate voorbeelden van politiek falen zich wereldwijd opstapelen.
Overheidsinstellingen die in de moderne tijd zijn ontstaan,
weerspiegelen de megapolitieke omstandigheden van een of meerdere eeuwen
geleden. Het Informatietijdperk zal nieuwe
vertegenwoordigingsmechanismen vereisen om chronische disfunctionaliteit
en zelfs sociale ineenstorting te vermijden.

Toen de Berlijnse Muur in 1989 viel, betekende dat niet alleen het einde
van de Koude Oorlog; het was ook het signaal aan de buitenwereld dat er
een stille aardbeving in de fundamenten van de macht in de wereld gaande
was. Het betekende het einde van de lange periode van stijgende
rendementen op geweld. De val van het communisme, die wij in 1987
voorspelden in \emph{Blood in the Streets} en nog eerder in onze
maandelijkse nieuwsbrief \emph{Strategic Investment}, was niet louter
een verwerping van een ideologie. Het diende als zichtbaar symbool voor
de meest betekenisvolle verandering in de geschiedenis van geweld van de
laatste vijfhonderd jaar. Indien onze analyse juist is, zal de
maatschappelijke structuur zich aanpassen aan de groeiende
inefficiënties die gepaard gaan met het grootschalig inzetten van
geweld. De grenzen waarbinnen de toekomst zich moet afspelen, zijn
hertekend.

\bookmarksetup{startatroot}

\chapter{De megapolitiek van het
Informatietijdperk}\label{de-megapolitiek-van-het-informatietijdperk}

::: \{.content-hidden when-format=``latex''\} De triomf van efficiëntie
boven macht :::

\begin{quote}
\ldots het is gecomputeriseerde informatie, niet mankracht of
massaproductie, die de Amerikaanse economie steeds meer aandrijft en die
oorlogen zal winnen in een wereld die is bekabeld voor 500 TV-kanalen.
De gecomputeriseerde informatie bestaat in de cyberspace, de nieuwe
dimensie die is ontstaan door de eindeloze productie van
computernetwerken, satellieten, modems, databanken en het publieke
internet. -- NEIL MUNRO
\end{quote}

Op 30 december 1936 bezetten medewerkers die hogere salarissen eisten
met geweld twee hoofdvestigingen van General Motors in Flint, Michigan.
Ze legden machines stil, zetten de lopende banden uit en deden alsof ze
thuis waren. Werknemers die waren aangenomen om de fabrieken te
bedienen, gingen letterlijk zitten in een industriële confrontatie die
vele weken zou duren. Het was een drama, onderbroken door gewelddadige
rellen en de wisselende loyaliteit van de politie, de militie van
Michigan en politieke figuren op alle bestuursniveaus. Toen hun eisen
weinig vooruitgang boekten, sloeg de vakbond op 1 februari 1937 opnieuw
toe.

Vakbondsactivisten namen met geweld GM's Chevrolet-fabriek in Flint
over. Door de belangrijkste installaties van General Motors te bezetten
en te sluiten, slaagden de arbeiders erin om de productieve capaciteit
van het bedrijf bijna volledig te verlammen. In de tien dagen na de
inname van de derde fabriek produceerde GM in de Verenigde Staten
slechts 153 auto's.

We halen dit nieuwsbericht van zestig jaar geleden aan om de revolutie
in megapolitieke omstandigheden die nu gaande is, scherper te belichten.
De sit-downstaking bij GM gebeurde nog binnen het leven van sommige
lezers van dit boek. Toch geloven wij dat sit-downstakingen in het
Informatietijdperk net zo achterhaald zullen blijken te zijn als slaven
die door de woestijn sjouwen met enorme stenen om grafpiramides voor de
farao's te bouwen. Vakbonden en hun intimidatietactieken werden in het
Industriële Tijdperk zo vertrouwd dat ze een onbetwist deel van het
sociale landschap vormden. Ze berustten echter op speciale megapolitieke
omstandigheden die nu snel verdwijnen. Er zullen geen Chevrolets en geen
UAW (de vakbond in kwestie) zijn om te staken op de
Informatie-Supersnelweg.

Overheden zullen net als hun tegenhangers, de vakbonden, in verval
raken. Geïnstitutionaliseerde dwang van het soort dat een cruciale rol
speelde in de twintigste-eeuwse samenleving zal niet langer mogelijk
zijn. Technologie veroorzaakt een diepgaande verandering in de logica
van afpersing en bescherming.

\begin{quote}
\ldots er zal geen eigendom zijn, geen heerschappij, geen onderscheid
tussen het mijne en het uwe; maar alleen datgene wat ieder kan
verkrijgen, en zolang hij het kan behouden. -- THOMAS HOBBES
\end{quote}

\subsection{Afpersing en bescherming}\label{afpersing-en-bescherming}

Door de geschiedenis heen is geweld een dolk geweest die op het hart van
de economie was gericht. Zoals Thomas Schelling scherp opmerkte: ``De
macht om pijn te doen, om dingen te vernietigen die iemand dierbaar
zijn, om pijn en verdriet toe te brengen, is een vorm van
onderhandelingsmacht, die niet makkelijk is te gebruiken, maar vaak
wordt toegepast. In de onderwereld vormt het de basis voor chantage,
afpersing en ontvoering; in de commerciële wereld voor boycots,
stakingen en uitsluitingen\ldots{} Het is vaak de basis voor discipline,
civiel en militair; en goden gebruiken het om discipline af te
dwingen.'' Het vermogen van de staat om belastingen te heffen, hangt af
van dezelfde kwetsbaarheden die ook ten grondslag liggen aan
particuliere afpersing en chantage. Hoewel we het doorgaans niet in deze
termen zien, geeft het aandeel van de middelen dat dwingend wordt
beheerd en besteed, via misdaad en overheid, een ruwe maatstaf van het
megapolitieke evenwicht tussen afpersing en bescherming.

Als technologie de bescherming van bezittingen moeilijk maakt, zal
misdaad waarschijnlijk wijdverspreid zijn, net als vakbondsactiviteit.
Onder zulke omstandigheden zal de overheid dus een hoge premie kunnen
afdwingen voor het leveren van bescherming. Belastingen zullen hoog
zijn. Wanneer belastingen laag zijn en lonen vooral door marktwerking
worden bepaald in plaats van door politieke inmenging of dwang, zorgt
technologie ervoor dat de verhoudingen verschuiven in het voordeel van
bescherming, en niet van afpersing.

De technologische disbalans tussen afpersing en bescherming bereikte een
extreem aan het einde van het derde kwart van de twintigste eeuw. In
sommige geavanceerde westerse samenlevingen werd meer dan de helft van
de middelen door regeringen opgeëist. De inkomens van een groot deel van
de bevolking werden ofwel bij decreet vastgesteld, of beïnvloed door
dwang, bijvoorbeeld door stakingen en dreiging met andere vormen van
geweld. De verzorgingsstaat en de vakbond waren beide producten van
technologie, die samen profiteerden van de overwinning van macht boven
efficiëntie in de twintigste eeuw. Ze hadden niet kunnen bestaan zonder
de militaire en civiele technologieën die het rendement op geweld in het
Industriële Tijdperk verhoogden.

Het vermogen om bezittingen te creëren heeft altijd enige kwetsbaarheid
voor afpersing met zich meegebracht. Hoe waardevoller de bezittingen die
werden gecreëerd of beheerd, des te hoger de prijs die op de een of
andere manier betaald moest worden. Of je betaalde iedereen die machtig
genoeg was om je met geweld af te persen, of je betaalde voor militaire
macht die in staat was om elke poging tot afpersing met brute kracht te
verslaan.

\begin{quote}
Er zal geen geweld meer worden gehoord in uw land, geen verderf of
vernieling binnen uw grenzen\ldots{} -- JESAJA 60:18
\end{quote}

\subsection{De wiskunde van
bescherming}\label{de-wiskunde-van-bescherming}

Nu kan de dolk van geweld binnenkort mogelijk worden getemd.
Informatie­technologie belooft het evenwicht tussen bescherming en
afpersing ingrijpend te veranderen, waardoor de bescherming van
bezittingen in veel gevallen veel eenvoudiger wordt en afpersing
moeilijker. De technologie van het Informatietijdperk maakt het mogelijk
bezittingen te creëren die buiten het bereik van veel vormen van dwang
liggen.

Deze nieuwe asymmetrie tussen bescherming en afpersing berust op een
fundamentele wiskundige waarheid: het is gemakkelijker om te
vermenigvuldigen dan om te delen. Hoewel deze waarheid eenvoudig lijkt,
bleven de diepgaande gevolgen ervan onzichtbaar tot de komst van
microprocessors. Computers hebben in het afgelopen decennium miljarden
keren meer berekeningen uitgevoerd dan in de hele voorgaande
geschiedenis van de wereld.

Deze sprong in rekenkracht heeft ons voor het eerst in staat gesteld
enkele universele kenmerken van complexiteit te doorgronden. Wat de
computers laten zien, is dat complexe systemen alleen van onderaf kunnen
worden opgebouwd en begrepen. Het vermenigvuldigen van priemgetallen is
eenvoudig. Maar het uit elkaar halen van complexiteit door het product
van grote priemgetallen te ontbinden is vrijwel onmogelijk. Kevin Kelly,
redacteur van \emph{Wired}, verwoordt het zo:\\
``Het vermenigvuldigen van meerdere priemgetallen tot een groter product
is eenvoudig; elk kind op de basisschool kan dat, maar alle
supercomputers in de wereld stikken terwijl ze een product terug
proberen te ontleden in zijn eenvoudige priemgetallen.''

\subsection{De logica van complexe
systemen}\label{de-logica-van-complexe-systemen}

De cybereconomie zal onvermijdelijk worden gevormd door deze
fundamantele wiskundige waarheid. Het manifesteert zich al in sterke
encryptiemethoden. Zoals we later in dit hoofdstuk zullen bespreken,
zullen deze algoritmen het mogelijk maken om een nieuw, beschermd domein
van cyberhandel te creëren waarin het geweldsmonopolie sterk wordt
teruggedrongen. Het evenwicht tussen afpersing en bescherming zal
drastisch doorslaan in de richting van bescherming. Dit zal de opkomst
bevorderen van een economie die meer steunt op spontane, adaptieve
mechanismen en minder op bewuste besluitvorming en toewijzing van
middelen via bureaucratie. Het nieuwe systeem, waarin bescherming
centraal zal staan, zal sterk verschillen van datgene wat voortkwam uit
de overheersing van dwang in het Industriële Tijdperk.

\subsection{Command-and-control systemen zijn
primitief}\label{command-and-control-systemen-zijn-primitief}

We schreven in \emph{The Great Reckoning} dat de computer ons in staat
stelt om de complexiteit in een hele reeks systemen te ``zien'' die
voorheen onzichtbaar was. Geavanceerde rekenkracht maakt het niet alleen
mogelijk de dynamiek van complexe systemen beter te begrijpen, maar ook
om deze complexiteiten op productieve manieren te benutten. In zekere
zin is dit niet eens een keuze, maar het is onvermijdelijk als de
economie verder wil komen dan het starre, centraal gestuurde
ontwikkelingsstadium. Een dergelijk systeem, dat afhankelijk is van
lineaire relaties, is in wezen primitief. De inbeslagname van middelen
door de overheid trekt onvermijdelijk middelen weg van complexe
toepassingen met hoge waarde naar primitieve toepassingen met lage
waarde. Het is een proces dat wordt beperkt door dezelfde wiskundige
asymmetrie die voorkomt dat het product van grote priemgetallen kan
worden ontbonden. Het verdelen van de buit kan nooit meer zijn dan
primitief.

\subsection{Alles wordt complexer}\label{alles-wordt-complexer}

Overal in het universum zien we systemen die naarmate ze evolueren
grotere complexiteit bereiken. Dit geldt in de astrofysica. Het geldt in
een plas water. Laat regenwater in een lage plek liggen en het zal
complexer worden. Geavanceerde systemen van elke soort zijn complexe
adaptieve systemen zonder een centrale autoriteit. Elk complex systeem
in de natuur, waarvan de markteconomie de meest evidente sociale
manifestatie is, berust op een verscheidenheid aan competenties.
Systemen die in de meest uiteenlopende omstandigheden het best
functioneren, danken hun robuustheid aan een spontane orde die ruimte
laat voor het onverwachte. Het leven zelf is zo'n complex systeem.
Miljarden potentiële genetische combinaties produceren één enkel
menselijk individu. Het ordenen daarvan zou elke bureaucratie doen
vastlopen.

Vijfentwintig jaar geleden kon dat slechts een intuïtie zijn. Vandaag is
het aantoonbaar. Naarmate computers ons meer inzicht geven in de
wiskunde achter kunstmatig leven, groeit ons begrip van de wiskunde van
het werkelijke leven, dat berust op biologische complexiteit. Dankzij
informatietechnologie kunnen deze geheimen van complexiteit worden
toegepast om economieën om te vormen tot steeds complexere systemen. Het
Internet en het World Wide Web hebben al kenmerken aangenomen van een
organisch systeem, zoals Kevin Kelly suggereert in \emph{Out of Control:
The New Biology of Machines, Social Systems, and the Economic World}. In
zijn woorden is de natuur ``een ideeënfabriek. Vitale, postindustriële
paradigma's liggen verborgen in elke mierenhoop in de jungle\ldots{} De
grootschalige overdracht van het biologische naar machines zou ons diep
moeten verwonderen. Wanneer de vereniging van het geboren en het
gemaakte voltooid is, zullen onze creaties leren, zich aanpassen,
zichzelf genezen en evolueren. Dit is een kracht waar we nauwelijks van
hebben durven dromen.''

De gevolgen van deze ``grootschalige overdracht van het biologische naar
machines'' zullen inderdaad verstrekkend zijn. Sociale systemen hebben
altijd een sterke neiging gehad om de kenmerken van de heersende
technologie te imiteren. Marx had hierover gelijk. Gigantische fabrieken
vielen samen met het tijdperk van de grote overheid. Microprocessing is
nu bezig met het miniaturiseren van instituties. Als onze analyse klopt,
zal de technologie van het Informatietijdperk uiteindelijk een economie
scheppen die beter geschikt is om de voordelen van complexiteit te
benutten.

Toch zijn de megapolitieke dimensies van zo'n verandering zo weinig
begrepen dat zelfs de meesten die het wiskundige belang ervan hebben
erkend, dit op een achterhaalde manier hebben gedaan. Het is
eenvoudigweg moeilijk om volledig te bevatten en te internaliseren dat
technologische verandering in de komende jaren de meeste politieke
vormen en concepten van de moderne wereld zal verouderen. Zo schreef de
overleden natuurkundige Heinz Pagels in zijn vooruitziende boek
\emph{The Dreams of Reason}: ``Ik ben ervan overtuigd dat de naties en
volkeren die de nieuwe wetenschap van complexiteit beheersen, de
economische, culturele en politieke supermachten van de volgende eeuw
zullen worden.'' Het is een indrukwekkende voorspelling. Wij geloven
echter dat het onjuist zal blijken, niet omdat het een verkeerde
observatie is, maar juist omdat het méér bewaarheid zal worden dan
Pagels durfde uit te spreken. Samenlevingen die zich herconfigureren tot
complexere adaptieve systemen zullen inderdaad opbloeien. Maar wanneer
dat gebeurt, zullen het waarschijnlijk geen naties zijn, laat staan
``politieke supermachten''. Het is waarschijnlijker dat de directe
begunstigden van de toegenomen complexiteit van de sociale systemen de
Soevereine Individuen van het nieuwe millennium zullen zijn.

Pagels' voorspelling komt neer op wat een sjamaan van een jagersgroep
vijfhonderd generaties terug bij het kampvuur zou hebben gezegd: ``Ik
ben ervan overtuigd dat de eerste jagersgroep die de nieuwe wetenschap
van geïrrigeerde landbouw beheerst, meer vrije tijd voor het vertellen
van verhalen zal hebben dan zelfs die kerels die de grote vissen vangen
daar bij het meer.'' Hoezeer hij ook gelijk had over het belang van
complexiteit, Pagels miste het meest fundamentele feit van allemaal:
wanneer de logica van geweld verandert, verandert de samenleving.

\section{De logica van geweld}\label{de-logica-van-geweld}

Om te begrijpen hoe en waarom, moeten we ons richten op diverse aspecten
van de megapolitiek die je zelden tegenkomt. Deze vraagstukken
onderzocht historicus Frederic C. Lane, wiens werk over geweld en de
economische betekenis van oorlog elders in dit boek aan bod komt. Toen
Lane in het midden van deze eeuw zijn werk schreef, was een
informatiesamenleving nog ondenkbaar. Onder die omstandigheden kon hij
er rotsvast van overtuigd zijn dat de strijd om geweld wereldwijd in
zijn definitieve fase beland was met de opkomst van de natiestaat. In
zijn werken geeft hij niet aan dat hij microprocessing had voorzien of
dat hij geloofde dat het technologisch haalbaar was om activa in de
cyberspace, een rijk zonder fysiek bestaan, te creëren. Lane ging niet
in op de implicaties van de mogelijkheid dat grote handelsvolumes
vrijwel immuun zouden kunnen worden gemaakt voor de invloed van geweld.

Hoewel Lane de huidige technologische revoluties niet had voorzien,
blijken zijn inzichten in de verschillende stadia van de monopolisatie
van geweld uit het verleden zo scherp dat ze duidelijk toepasbaar zijn
op de opkomende informatierevolutie. In zijn studie van de gewelddadige
middeleeuwse wereld richtte hij de aandacht op vraagstukken die
conventionele economen en historici vaak over het hoofd zien. Hij
realiseerde zich dat de manier waarop geweld wordt georganiseerd en
gecontroleerd een cruciale rol speelt bij de allocatie van schaarse
middelen. Lane erkende bovendien dat, hoewel de productie van geweld
doorgaans niet als onderdeel van de economische output wordt gezien, de
beheersing ervan essentieel is voor de economie. De voornaamste taak van
de overheid bestaat immers uit het bieden van bescherming tegen geweld.
Zoals hij het verwoordde:

\begin{quote}
Elke economische onderneming heeft bescherming nodig en betaalt
daarvoor; bescherming tegen de vernietiging of de gewapende inbeslagname
van haar kapitaal en tegen de gewelddadige verstoring van haar arbeid.
In sterk georganiseerde samenlevingen behoort het leveren van deze
bescherming tot de taken van een speciale vereniging of onderneming die
men `overheid' noemt. Inderdaad, een van de meest onderscheidende
kenmerken van overheden is hun poging om wet en orde te handhaven door
zelf macht in te zetten en op diverse wijzen het gebruik van geweld door
anderen te beheersen.
\end{quote}

Dit punt is zo fundamenteel dat het niet in leerboeken of burgerlijke
discussies, die zogenaamd de politieke koers bepalen, wordt besproken.
Maar men kan het niet zomaar negeren als je de informatierevolutie, die
zich nu ontvouwt, wil doorgronden. De bescherming van leven en eigendom
is immers een essentiële behoefte waar elke samenleving in de
geschiedenis mee te maken heeft gehad. Het afweren van gewelddadige
agressie vormt het centrale dilemma in de geschiedenis en kent geen
eenvoudige oplossing, al bestaan er meerdere manieren om bescherming te
bieden.

\subsection{Het einde van een
tijdperk}\label{het-einde-van-een-tijdperk}

Terwijl we dit schrijven, beginnen de megapolitieke gevolgen van het
Informatietijdperk net voelbaar te worden. De economische verschuivingen
van de afgelopen decennia waren van industrie naar informatie en
rekenkracht, van machinale kracht naar microprocessing, van fabrieken
naar bureau's, van massaproductie naar kleine teams, of zelfs naar
zelfstandigen. Naarmate ondernemingen kleiner worden, neemt de kans op
sabotage en chantage op de werkvloer af. Vakbonden vinden het veel
lastiger om kleinere bedrijven te organiseren.

Dankzij microtechnologie kunnen ondernemingen kleiner en mobieler
opereren. Velen leveren diensten of producten die nauwelijks natuurlijke
hulpbronnen vereisen. In principe kunnen deze ondernemingen vrijwel
overal op aarde werken, omdat ze niet gebonden zijn aan een specifieke
locatie, zoals een mijn of een haven. Daardoor lopen ze op den duur veel
minder kans om door vakbonden of politici onder druk te worden gezet.
Een oud Chinees gezegde luidt: `Van alle zesendertig manieren om uit de
problemen te komen, is vertrekken de beste.'

In het Informatietijdperk zal die oosterse wijsheid haar vruchten
afwerpen. Wanneer het bedrijfsleven onaangenaam wordt door buitensporige
eisen op één locatie, wordt verhuizen een stuk eenvoudiger. Inderdaad,
zoals we straks bespreken, kunnen ondernemers in het Informatietijdperk
virtuele ondernemingen opzetten, waarbij hun vestigingsplaats volledig
afhankelijk is van de spotmarkt. Een plotselinge toename van de pogingen
tot afpersing, of het nu door overheden of anderen komt, kan ertoe
leiden dat de activiteiten en activa van een virtuele onderneming met de
snelheid van het licht uit de betreffende jurisdictie verdwijnen.

De toenemende integratie van microtechnologie in industriële processen
betekent dat zelfs bedrijven die nog steeds producten met grote
schaalvoordelen produceren, niet langer even kwetsbaar zijn voor de
dreiging van geweld als vroeger. Een voorbeeld hiervan is het falen van
de langdurige staking van de United Auto Workers tegen Caterpillar, die
eind 1995, na bijna twee jaar, werd beëindigd. In tegenstelling tot de
assemblagelijnen van de jaren dertig maakt de huidige
Caterpillar-fabriek veel meer gebruik van vakbekwame werknemers. Onder
druk van buitenlandse concurrentie besteedde Caterpillar veel
laaggeschoold werk uit, sloot inefficiënte fabrieken en investeerde
bijna 2 miljard dollar in de computerisering van bestaande machines en
de installatie van assemblagerobots. De staking stimuleerde zelfs het
doorvoeren van deze arbeidsbesparende efficiëntieslag. Het bedrijf stelt
nu tweeduizend werknemers minder nodig te hebben dan toen het werk werd
neergelegd.

De megapolitieke omstandigheden van het productieproces zijn
ingrijpender veranderd dan de meeste mensen beseffen. Deze verandering
is nog niet duidelijk zichtbaar, deels omdat er altijd een vertraging
zit tussen een revolutie in megapolitieke omstandigheden en de
institutionele veranderingen die zij onvermijdelijk teweegbrengt.
Bovendien betekent de snelle evolutie van microprocessingtechnologie dat
er nu producten verschijnen waarvan de megapolitieke gevolgen al
voorzien kunnen worden, nog voordat ze bestaan. Zij zullen zorgen voor
een fundamenteel andere wereld.

\section{Uitbuiting van de kapitalisten door de
arbeiders}\label{uitbuiting-van-de-kapitalisten-door-de-arbeiders}

De aard van de technologie in het merendeel van de twintigste eeuw
zorgde ervoor dat eigenaars en managers weinig middelen hadden om een
gedwongen bezetting van een fabriek of een sit-downstaking tegen te
gaan. Zoals historicus Robert S. McElvaine opmerkte, maakte een
sit-downstaking het ``moeilijk voor werkgevers om de staking af te
breken zonder hetzelfde met hun eigen uitrusting te doen.'' In feite
hielden de arbeiders het kapitaal van de eigenaars fysiek gegijzeld. Om
redenen die hieronder worden toegelicht, bleken grotere industriële
ondernemingen makkelijkere doelwitten voor vakbonden dan kleinere
bedrijven. In 1937 was General Motors misschien wel hét toonaangevende
industriële concern ter wereld. Haar fabrieken behoorden tot de grootste
en duurste verzameling machines ooit gebouwd en boden werk aan
tienduizenden arbeiders. Elk uur van elke dag dat de GM-fabrieken stil
lagen, kostte het bedrijf een klein fortuin. Een staking die wekenlang
onopgelost bleef, zoals die in de winter van 1936-37, betekende snel
oplopende verliezen.

\subsection{In verzet tegen vraag en
aanbod}\label{in-verzet-tegen-vraag-en-aanbod}

General Motors capituleerde snel nadat de derde fabriek bezet werd en de
productie van auto's volledig stilviel. Dat was geen economische
beslissing gebaseerd op de vraag en het aanbod van arbeid. Integendeel,
toen General Motors toegaf aan de eisen van de vakbond waren er negen
miljoen werklozen in de VS, wel 14\% van de beroepsbevolking. Velen
waren bereid en bekwaam genoeg om de assemblagebanen bij GM in te
vullen, al zul je dat waarschijnlijk niet van de rapportage destijds
hebben gehoord. Een verfijnde etiquette verhulde een eerlijke analyse
van de arbeidsverhoudingen tijdens de industriële periode. Het idee dat
fabriekswerk, en dan met name in het midden van de twintigste eeuw,
vakmanschap vereiste, was grotendeels fictie. De meeste fabrieksbanen
konden door bijna iedereen uitgevoerd worden die op tijd kon opdagen. Ze
vereisten amper training en zelfs geen geletterdheid. Nog zo recent als
in de jaren '80 waren veel arbeiders van GM analfabeet of konden niet
rekenen, of beide, en tot in de jaren '90 kreeg een nieuwe werknemer
gemiddeld slechts één dag oriëntatie. Een baan die je in een dag kunt
leren, kun je geen vakmanschap noemen.

Terwijl zowel geschoolde als ongeschoolde arbeiders in de rij stonden om
smeekbedes te doen voor werk, slaagden de fabrieksarbeiders van GM er in
1937 toch in om hun werkgevers tot een loonsverhoging te dwingen. Hun
succes had veel meer te maken met de dynamiek van geweld dan met de
vraag en het aanbod van arbeid. In maart 1937, de maand na de schikking,
waren er 17 extra sit-downstakingen in de Verenigde Staten. De meeste
waren succesvol. Vergelijkbare gebeurtenissen vonden plaats in elk
geïndustrialiseerd land. De arbeiders namen simpelweg de fabrieken in
beslag en verkochten ze terug aan de eigenaars. Het was een uiterst
eenvoudige tactiek, en een die in de meeste gevallen winstgevend en leuk
was voor de deelnemers. Eén sit-downstaker schreef: ``Ik heb de tijd van
mijn leven, iets nieuws, iets anders, volop eten en muziek.''

De GM-sit-downstaking van 1936-37 en de andere gewelddadige
fabrieksovernames uit die tijd waren voorbeelden van een fenomeen dat
wij in \emph{Blood in the Streets} beschreven als ``de uitbuiting van de
kapitalisten door de arbeiders.'' Dit was niet de visie die Pete Seeger
in zijn droevige liederen verwerkte. Maar tenzij je een carrière als
volkszanger in een arbeidersbuurt overweegt, is het belangrijk niet te
focussen op de populaire interpretatie, maar op de onderliggende
werkelijkheid. Waar je ook in de geschiedenis kijkt, er is bijna altijd
een laag van rationalisatie en schijn die de ware megapolitieke
fundamenten van systematische afpersing verhult. Als je die
rationalisaties voor lief neemt, is de kans klein dat je werkelijk
begrijpt wat er gaande is.

\section{Het ontcijferen van de logica van
afpersing}\label{het-ontcijferen-van-de-logica-van-afpersing}

Om de megapolitieke implicaties van de huidige overgang naar het
Informatietijdperk te begrijpen, moet je de retoriek wegstrepen en je
concentreren op de werkelijke logica van geweld in de samenleving. Dit
is vergelijkbaar met het pellen van een te rijpe ui. Het kan tranen in
je ogen brengen, maar kijk niet weg. We beginnen met de logica van
afpersing op de werkplek en breiden de analyse vervolgens uit naar
bredere kwesties, zoals de creatie en bescherming van bezittingen en de
aard van de moderne overheid. In grotere mate dan de meeste mensen zich
realiseren, was de welvaart van de overheid, net als die van vakbonden,
direct gecorreleerd aan de mate waarin afpersing effectief ingezet kan
worden. In de negentiende eeuw was de effectiviteit ervan lager dan in
de twintigste. In het volgende millennium zal zij bijna tot nul afnemen.

De hele logica van overheid en de aard van macht zijn getransformeerd
door microprocessing. Dit lijkt op het eerste gezicht misschien
overdreven, maar kijk aandachtig mee. De welvaart van regeringen is in
de twintigste eeuw hand in hand gegaan met de welvaart van vakbonden.
Vóór de twintigste eeuw onttrokken de meeste overheden aanzienlijk
minder middelen dan de militante verzorgingsstaten die we nu kennen.
Evenzo waren vakbonden vóór deze eeuw kleine of onbelangrijke factoren
in het economische leven. Het vermogen van arbeiders om hun werkgevers
te dwingen tot het betalen van hogere lonen dan de marktprijs, vond haar
oorsprong in dezelfde megapolitieke omstandigheden die het voor
overheden mogelijk maakten om 40 procent van de economische productie te
belasten, of meer.

\subsection{Afpersing op de werkvloer vóór de twintigste
eeuw}\label{afpersing-op-de-werkvloer-vuxf3uxf3r-de-twintigste-eeuw}

De opkomst en ondergang van de afpersing van kapitalisten door vakbonden
is makkelijk te verklaren door te kijken naar de veranderende
megapolitieke omstandigheden binnen het productieproces. In 1776, toen
Adam Smith \emph{The Wealth of Nations} publiceerde, waren de condities
voor afpersing op de werkvloer zo ongunstig dat het vormen van
``combinaties'' van arbeiders om ``de prijs van hun arbeid te verhogen''
zelden uitvoerbaar bleek. De meeste productiebedrijven waren klein en
door families gerund, en grootschalige industriële activiteiten stonden
nog in de kinderschoenen. Hoewel geweld niet uitgesloten werd, was het
niet heel effectief. In de tijd van Smith en tot ver in de negentiende
eeuw beschouwde men vakbonden als illegale combinaties in
Groot-Brittannië, de Verenigde Staten en andere landen met common-law.
Adam Smith omschreef de stakingen als volgt: `Hun gebruikelijke
voorwendselen zijn soms de hoge prijzen van levensmiddelen, soms de
enorme winst die hun meester behaalt met hun werk\ldots{} Zij grijpen
altijd terug op luid tumult, en soms zelfs op schokkend geweld en de
meest gruwelijke beledigingen.' Toch behalen arbeiders `zeer zelden enig
voordeel uit die tumultueuze combinaties,' behalve `de bestraffing of
ondergang van de aanvoerders.'

In de negentiende eeuw genoot de industrie van steeds meer
schaalvoordelen en de omvang van ondernemingen nam toe. Toch bleven de
meeste individuen voor zichzelf werken als boeren of kleine ondernemers,
en pogingen tot vakbondsvorming, zoals beschreven door Adam Smith,
``eindigden over het algemeen in niets.'' De juridische en politieke
positie van vakbonden veranderde pas toen de schaal van ondernemingen
toenam. De eerste vakbonden die zich succesvol organiseerden, waren
ambachtsvakbonden van hooggeschoolde arbeiders, die zich over het
algemeen zonder veel geweld organiseerden. Ze stelden zich tevreden met
loonsverhogingen die overeenkwamen met de potentiële kosten van hun
vervanging. Vakbonden voor ongeschoolde arbeiders waren een ander
verhaal. Zij maakten gebruik van de verschuiving naar grotere
ondernemingen door zich te richten op precies die industrieën die
bijzonder kwetsbaar waren voor dwang, hetzij omdat ze op grotere schaal
opereerden, hetzij omdat het karakter van de werkzaamheden hun eigenaars
blootstelde aan fysieke sabotage. Dit patroon werd bevestigd van
Newcastle tot Argentinië.

Een vroeg voorbeeld van gewelddadige arbeidersbewegingen in de Verenigde
Staten was een aanval op het Chesapeake and Ohio Canal in 1834. In
tegenstelling tot de meeste bedrijven van het begin van de negentiende
eeuw, was het C\&O Canal geen afgesloten en gemakkelijk te beschermen
onderneming. Oorspronkelijk zou het 550 kilometer lang worden, met een
hoogteverschil van ruim 900 meter van de lagere Potomac tot de bovenloop
van de Ohio. Het graven van zo'n kanaal was een enorme klus die nooit
helemaal zou worden voltooid. Toch waren er talrijke arbeiders mee
bezig, waarbij enkelen snel inzagen dat het kanaal eenvoudig onklaar te
maken was. Zonder regelmatig onderhoud kon het door muskusratten die
onder het jaagpad groeven, worden gesaboteerd. Daarnaast konden de
sluizen en kanalen tijdens het gebruik makkelijk beschadigd raken door
onzorgvuldig gebruik, overstromingen door zware regenval of door
botsingen met onbeheerde boten. Het was dan ook niet moeilijk voor
stakers om de waterweg te blokkeren met gezonken boten of ander puin.
Begin 1834 leidde een rel tussen rivaliserende groepen Ierse arbeiders
op het C\&O tot een poging om dit potentieel te benutten en het kanaal
over te nemen. De poging mislukte echter nadat president Andrew Jackson
federale troepen van Fort McHenry had gestuurd om de arbeiders te
verdrijven. Hierbij kwamen vijf mensen om het leven.

Mijnen en spoorwegen waren ook vroege doelwitten voor vakbondsactivisme
in Amerika. Net als het C\&O Canal waren zij zeer kwetsbaar voor
sabotage: mijnen konden bijvoorbeeld worden overstroomd of bij de ingang
worden geblokkeerd; het doden van de muildieren die de ertswagentjes uit
ondergrondse mijnen trokken, creëerde een lastige en onaangename
situatie voor de eigenaars; spoorwegbeddingen strekten zich over vele
kilometers uit en konden slechts moeilijk worden bewaakt. Het was
relatief eenvoudig voor vakbondscriminelen om mijnen en spoorwegen aan
te vallen en aanzienlijke economische schade aan te richten. Dergelijke
aanvallen kwamen veelvuldig voor tijdens pogingen om effectieve
vakbonden te organiseren. Deze inspanningen waren over het algemeen het
intensiefst tijdens perioden waarin de reële lonen stegen door deflatie.
Wanneer eigenaars probeerden om de nominale lonen aan te passen, leidde
dit vaak tot protesten die tot geweld escaleerden. Zulke incidenten
waren wijdverspreid tijdens de depressie die volgde op de paniek van
1873.

In december 1874 brak er oorlog uit in de anthracietmijnen van het
oosten van Pennsylvania. De vakbonden organiseerden een gewelddadige
staking onder de dekmantel van een geheime maatschappij genaamd de
Ancient Order of Hibernians, ook bekend als de ``Molly Maguires,'' naar
een Ierse revolutionair. Deze groep stond bekend om het terroriseren van
de kolenmijnen en het verhinderen dat mijnwerkers die wilden werken, hun
werk konden doen. Tegen haar leden werden sabotage en vernietiging van
eigendom, en zelfs moord en liquidaties ten laste gelegd.

Er was ook terugkerend geweld onder spoorwegmedewerkers. Bijvoorbeeld in
juli 1877 waren er ernstige uitbarstingen gericht op het vernietigen van
eigendommen van zowel de Pennsylvania Railroad als de Baltimore \& Ohio
Railroad. Arbeiders namen wissels over, haalden sporen weg, sloten
rangeerterreinen af, schakelden locomotieven uit, saboteerden en
plunderden treinen, en erger. In Pittsburgh werden de ronde werkplaatsen
van de Pennsylvania Railroad in brand gestoken met honderden mensen
erin. Tientallen werden gedood, tweeduizend wagons werden verbrand en
geplunderd, en de machinewerkplaats werd vernietigd, samen met een
graanlift en 125 locomotieven. Federale troepen grepen in om de orde te
herstellen.

Hoewel deze vroege stakingen sympathiek werden geïnterpreteerd door
socialistische en vakbondsactivisten, inspireerden ze weinig publieke
steun. Ondanks de inherente kwetsbaarheid van industrieën zoals mijnen
en spoorwegen, waren de algemene megapolitieke omstandigheden nog niet
gunstig voor de uitbuiting van kapitalisten door arbeiders. De schaal
van ondernemingen was te klein om systematische afpersing mogelijk te
maken. Hoewel er kwetsbare industrieën waren, boden zij werk aan een te
klein deel van de bevolking om de voordelen van dwang tegen werkgevers
breed te laten delen. Zonder dergelijke steun waren ze niet houdbaar,
omdat eigenaars op de overheid konden rekenen voor bescherming. Hoewel
vakbonden soms probeerden lokale ambtenaren via dreigementen ervan te
weerhouden rechterlijke bevelen uit te voeren, waren deze pogingen
zelden succesvol. Zelfs de meest gewelddadige stakingen werden meestal
binnen enkele dagen of weken door militaire middelen onderdrukt.

\subsection{Chantage eenvoudig
gemaakt}\label{chantage-eenvoudig-gemaakt}

Het feit dat pogingen van vakbonden om lonen van kleine bedrijven boven
het marktconforme niveau te krijgen zelden succesvol waren, leert ons
een les voor het Informatietijdperk. Zelfs die bedrijfstakken die
duidelijk kwetsbaar waren voor sabotage, zoals kanalen, spoorwegen,
trams en mijnen, waren niet gemakkelijk onder controle te brengen. Dit
kwam niet omdat de vakbonden geweld schuwden. Integendeel, geweld werd
royaal ingezet, soms ook tegen prominente individuen. Bijvoorbeeld, in
een zaak die binnen de Amerikaanse arbeidersbeweging werd gevierd als
``wraak van de mijnwerkers'', werd gouverneur Frank Steunenberg van
Idaho, die een poging van mijnwerkers om eigendommen in Coeur d'Alene te
blokkeren had tegengewerkt, vermoord door een bom die door een
huurmoordenaar van de vakbond was gegooid. Zelfs moord en dreigementen
waren vóór de opkomst van grootschalige fabrieken en
massaproductiebedrijven in de twintigste eeuw echter meestal onvoldoende
om vakbondserkenning te verkrijgen.

Om te begrijpen waarom de omstandigheden van vakbonden in de twintigste
eeuw zo veranderden, moet je kijken naar de kenmerken van
productietechnologie. Er veranderde iets met de snelle opkomst van
fabrieksarbeid in de vroege decennia van de twintigste eeuw. Deze
verandering maakte de economische koplopers bijzonder kwetsbaar voor
afpersing. In feite leken de fysieke eigenschappen van industriële
technologie arbeiders bijna uit te nodigen om de kapitalisten af te
persen. Overweeg het volgende:

\begin{enumerate}
\def\labelenumi{\arabic{enumi}.}
\tightlist
\item
  \emph{De meeste industriële producten bevatten een grote hoeveelheid
  natuurlijke hulpbronnen.} Dit leidde ertoe dat de productie gebonden
  bleef aan een beperkt aantal locaties, net zoals dat mijnen moeten
  geplaatst worden waar de ertsen zich bevinden. Fabrieken die nabij
  transportknooppunten lagen met gemakkelijke toegang tot leveranciers
  van onderdelen en grondstoffen, hadden significante operationele
  voordelen. Dit maakte het voor dwingende organisaties, zoals overheden
  en vakbonden, gemakkelijker om een deel van die voordelen voor
  zichzelf te claimen.
\item
  \emph{Toenemende schaalvoordelen leidden tot zeer grote
  ondernemingen.} Fabrieken aan het begin van de negentiende eeuw waren
  relatief klein. Met de opkomst van de assemblagelijn in de twintigste
  eeuw namen de schaalvoordelen toe, waardoor de grootte en de kosten
  van de modernste fabrieken sterk opliepen. Dit maakte ze op
  verschillende manieren een makkelijk doelwit. Bijvoorbeeld, grote
  schaalvoordelen gaan vaak gepaard met lange productcycli. Lange
  productcycli zorgen voor stabielere markten. Dit maakt ondernemingen
  vatbaar voor roofzuchtige acties, omdat het betekent dat er voordelen
  op de lange termijn te behalen zijn.
\item
  \emph{Het aantal concurrenten in toonaangevende industrieën daalde
  scherp. Het was tijdens de industriële periode niet ongebruikelijk om
  slechts een handvol bedrijven te vinden dat concurreerde om
  miljardenmarkten.} Ook dit maakte hen makkelijkere doelwitten voor
  afpersing door vakbonden. Het is veel eenvoudiger om vijf bedrijven
  aan te vallen dan vijfduizend. De concentratie van de industrie zelf
  was een factor die afpersing vergemakkelijkte. Dit voordeel werkte
  zelfversterkend, omdat bedrijven die gedwongen werden om hogere lonen
  te betalen, zelden te maken kregen met concurrenten die daarvan waren
  vrijgesteld. Vakbonden konden daardoor een behoorlijk deel van de
  winst van dergelijke bedrijven afromen zonder dat ze direct failliet
  gingen. Uiteraard, als werkgevers routinematig failliet zouden gaan
  telkens wanneer ze lonen moesten betalen die hoger waren dan de
  marktprijs, dan zouden arbeiders weinig winnen door hen daartoe te
  dwingen.
\item
  \emph{Door de grotere schaal van de ondernemingen, namen de
  kapitaalvereisten toe.} Dit vergrootte niet alleen de kwetsbaarheid
  van kapitaal en verhoogde de kosten van fabrieksluitingen; het maakte
  het ook steeds onwaarschijnlijker dat een moderne fabriek in handen
  van één individu of familie kon zijn, behalve via erfopvolging van
  iemand die het bedrijf op kleinere schaal was begonnen. Om de enorme
  kosten van uitrusting en exploitatie van een grote fabriek te
  financieren, moest het vermogen van honderden of duizenden mensen
  worden samengebracht op de kapitaalmarkten. Dit maakte het moeilijker
  voor de gefragmenteerde en bijna anonieme eigenaren om hun eigendom te
  verdedigen. Ze werden vrijwel gedwongen om professionele bestuurders
  in te schakelen, die doorgaans slechts een fractie van de aandelen van
  het bedrijf in bezit hadden. Het vertrouwen op ondergeschikte managers
  verzwakte de weerstand van bedrijven tegen afpersing. De managers
  hadden weinig prikkels om hun leven of lichamelijke integriteit te
  riskeren ter bescherming van het eigendom van het bedrijf. Hun verzet
  bereikte zelden de felheid waarmee kleine ondernemers, zoals
  slijterijhouders, hun eigendom verdedigen.
\item
  \emph{De grotere omvang van bedrijven betekende ook dat een groter
  deel van de totale beroepsbevolking in minder bedrijven werkzaam was
  dan ooit tevoren.} In sommige gevallen vonden tienduizenden arbeiders
  werk bij één enkel bedrijf. In militaire termen waren de eigenaars en
  managers ruimschoots in de minderheid ten opzichte van de personen in
  ondergeschikte functies. Verhoudingen van dertig tegen één of erger
  waren gebruikelijk. Dit nadeel nam toe met de bedrijfsgrootte, omdat
  enorme aantallen werknemers die samenkwamen, gemakkelijker geweld op
  anonieme wijze konden toepassen. Onder dergelijke omstandigheden
  hadden de werknemers waarschijnlijk geen betekenisvol contact of
  relaties met de eigenaren van de fabriek. Het anonieme karakter van
  deze relaties maakte het ongetwijfeld gemakkelijker voor werknemers om
  het belang van de eigendomsrechten van de eigenaren te negeren.
\item
  \emph{Massale tewerkstelling in een klein aantal bedrijven was een
  breed maatschappelijk fenomeen.} Dit zorgde voor een verdere
  versterking van de megapolitieke voordelen van vakbonden vergeleken
  met het Amerika van de negentiende eeuw, toen de meeste mensen
  zelfstandig waren of in kleine bedrijven werkten. In 1940 had 6
  procent van de Amerikaanse beroepsbevolking een baan als arbeider. Als
  gevolg hiervan vond het afdwingen van hogere lonen onder een groot
  aantal mensen, die dachten er zelf voordeel uit te halen, steeds meer
  steun. Dit werd geïllustreerd door een studie uit 1938-39 van de
  opvattingen van 1.700 mensen in Akron, Ohio, over het eigendom van
  corporaties. De enquête wees uit dat 68 procent van de CIO Rubber
  Workers weinig of geen sympathie had voor het idee dat corporaties
  eigendomsrecht mogen hebben, terwijl slechts één procent sterke steun
  voor het corporatieve eigendomsrecht toonde. Aan de andere kant viel
  geen enkele ondernemer, zelfs geen kleine eigenaar, in de categorie
  ``sterke tegenstand tegen corporatief eigendom''; 94 procent kreeg
  beoordelingen in de categorie extreem hoge steun voor
  eigendomsrechten.
\item
  \emph{Assemblagelijntechnologie was inherent sequentieel.} Het feit
  dat het hele productieproces afhankelijk was van het verplaatsen en
  assembleren van onderdelen in een vaste volgorde, creëerde extra
  kwetsbaarheden voor verstoring. In feite was de assemblagelijn als een
  spoorweg binnen de fabrieksmuren. Als het spoor geblokkeerd kon worden
  of de beschikbaarheid van één enkel onderdeel kon worden afgekapt,
  werd het hele productieproces stilgelegd.
\item
  \emph{Assemblagelijntechnologie standaardiseerde arbeid.} Dit
  verminderde de afwijkingen in de output van personen met verschillende
  vaardigheden die met dezelfde gereedschappen werkten. In feite was een
  cruciaal doel van fabrieksontwerp het creëren van een systeem waarin
  een genie en een idioot op opeenvolgende ploegen van de assemblagelijn
  hetzelfde product zouden produceren. Wat men ``domme'' machines zou
  kunnen noemen, waren ontworpen om slechts één soort output te leveren.
  Dit maakte het zelfs voor de koper van een Cadillac overbodig om zich
  te informeren welke lijnarbeiders zijn voertuig produceerden. Alle
  producten waren bedoeld om gelijk te zijn, ongeacht de verschillen in
  vaardigheden en intelligentie tussen de werknemers die ze
  produceerden.
\end{enumerate}

Het feit dat ongeschoolde arbeiders op de assemblagelijn hetzelfde
product konden maken als meer bekwame individuen, droeg bij aan de
egalitaire agenda doordat het de indruk wekte dat ieders economische
bijdragen gelijk waren. Ondernemersvaardigheden en mentale inspanning
leken minder belangrijk. De magie van moderne productie leek in de
machines zelf te liggen. Zelfs als ze niet daadwerkelijk door iedereen
ontworpen konden zijn, leken ze toch intellectueel toegankelijk voor
bijna iedereen. Dit gaf meer geloofwaardigheid aan de fictie dat
ongeschoolde arbeid ``uitgebuit'' werd door de fabriekseigenaar, die uit
het proces zou kunnen worden gehaald zonder dat iemand er nadeel van zou
ondervinden behalve de fabriekseigenaar zelf. ``We hebben geleerd dat we
de fabriek kunnen overnemen'', zei een GM-staker. ``We wisten al hoe we
de fabriek moesten runnen. Als General Motors niet oppast, trekken we
vanzelf de logische conclusies.''

Deze kenmerken van industriële technologie leidden overal tot de
oprichting van vakbonden om de kwetsbaarheid voor afpersing te benutten,
en tot grotere overheden die profiteerden van de hoge belastingen die
opgelegd konden worden aan grootschalige industriële faciliteiten. Dit
gebeurde niet één of twee keer, het gebeurde overal waar grootschalige
industrieën opkwamen. Steeds weer ontstonden vakbonden die geweld
gebruikten om lonen ver boven de marktprijs te realiseren. Ze konden dit
doen omdat industriële fabrieken duur, zichtbaar, immobiel en kostbaar
waren. Ze konden nauwelijks worden verborgen of verplaatst. Elk moment
dat ze buiten gebruik waren, betekende dat hun enorme kosten niet werden
afgeschreven.

Hierdoor waren het makkelijke doelwitten voor dwangmatige afpersing, een
feit dat veel duidelijker is in de geschiedenis van vakbonden dan de
heersende ideologie van de twintigste eeuw doet geloven. De bekende
econoom Henry Simons formuleerde het in 1944 als volgt:

\begin{quote}
``Arbeidsorganisatie zonder grote macht tot dwang en intimidatie is een
onrealistische abstractie. Vakbonden hebben nu zulke machten; ze hebben
die altijd gehad en zullen die altijd hebben, zolang ze in de huidige
vorm blijven bestaan. Waar de macht klein of onzeker is, moet deze
openlijk en uitgebreid worden uitgeoefend; macht die groot en
onaangevochten is, functioneert als overheid: zelfverzekerd,
gerespecteerd en zelden opvallend tentoongesteld.''
\end{quote}

Hoewel Simons' analyse precies is, zat hij verkeerd op een cruciaal
punt. Hij ging ervan uit dat vakbonden ``altijd zullen hebben'' wat hij
beschreef als een ``grote macht tot dwang en intimidatie.'' In
werkelijkheid verdwijnen vakbonden, niet alleen in de Verenigde Staten
en Groot-Brittannië, maar ook in andere volwassen industriële
samenlevingen. De reden dat ze verdwijnen, wat Simons miste en wat zelfs
veel vakbondsorganisatoren niet begrijpen, is dat de verschuiving naar
een Informatiesamenleving de megapolitieke omstandigheden op cruciale
manieren heeft veranderd, waardoor de veiligheid van eigendom sterk
toeneemt. Microtechnologie heeft al bewezen de afpersing die de
verzorgingsstaat ondersteunt te ondermijnen, omdat ze zelfs in de
commerciële sfeer geheel andere prikkels creëert dan in de industriële
periode.

\begin{enumerate}
\def\labelenumi{\arabic{enumi}.}
\item
  \emph{Informatie­technologie bevat nauwelijks natuurlijke hulpbronnen.}
  Ze biedt weinig of geen inherente locatievoordelen. De meeste
  informatietechnologie is zeer verplaatsbaar. Omdat het onafhankelijk
  van plaats kan functioneren, vergroot informatie­technologie de
  mobiliteit van ideeën, mensen en kapitaal. General Motors kon zijn
  drie assemblagelijnen in Flint, Michigan, niet inpakken en op het
  vliegtuig zetten. Een softwarebedrijf kan dat wel. De eigenaren kunnen
  hun algoritmen op draagbare computers downloaden en het eerstvolgende
  vliegtuig nemen. Dergelijke bedrijven hebben bovendien een extra
  prikkel om te ontsnappen aan hoge belastingen of vakbondseisen voor
  onredelijk hoge lonen. Kleinere bedrijven hebben meestal meer
  concurrenten. Als je tientallen of zelfs honderden concurrenten hebt
  die je klanten proberen te verleiden, kun je het je niet veroorloven
  om politici of werknemers veel meer te betalen dan ze daadwerkelijk
  waard zijn. Als jij dat alleen zou proberen, zouden je kosten hoger
  zijn dan die van je concurrenten en zou je failliet gaan. Het
  ontbreken van significante operationele voordelen op een bepaalde
  locatie betekent dat dwingende organisaties, zoals overheden en
  vakbonden, onvermijdelijk minder de mogelijkheid krijgen om deze
  voordelen voor zichzelf te benutten.
\item
  \emph{Informatie­technologie verkleint de omvang van ondernemingen.}
  Dit leidt tot kleinere bedrijven, wat een groter aantal concurrenten
  betekent. Hoe meer concurrentie, hoe moeilijker afpersing wordt, omdat
  er meer doelwitten zijn die onder controle gehouden moeten worden om
  lonen of belastingen boven het competitieve niveau te brengen. De
  scherpe daling van de gemiddelde bedrijfsgrootte, gefaciliteerd door
  informatietechnologie, heeft het aantal personen in ondergeschikte
  posities al verminderd. In de Verenigde Staten, bijvoorbeeld,
  suggereren breed gerapporteerde schattingen dat in 1996 ongeveer 30
  miljoen mensen zelfstandig in hun eigen bedrijf werkten. Uiteraard is
  het onwaarschijnlijk dat deze 30 miljoen tegen zichzelf in staking
  zullen gaan. Het is slechts iets minder aannemelijk dat de miljoenen
  mensen die werken in kleine bedrijven met een handvol werknemers hun
  werkgevers zouden proberen te dwingen tot lonen die hoger dan de
  marktprijs zijn. Werknemers die in het Informatietijdperk hun loon via
  dwang willen verhogen, missen het militaire voordeel van hun grote
  aantallen dat hen vroeger in de fabriek sterk maakte. Hoe minder
  mensen in een bedrijf werken, hoe minder mogelijkheden er zijn voor
  anoniem geweld. Door deze reden alleen al zouden tienduizend
  werknemers verdeeld over vijfhonderd bedrijven een kleinere bedreiging
  vormen voor het eigendom van die bedrijven dan tienduizend werknemers
  in één bedrijf, zelfs als de verhouding werknemers/eigenaren/managers
  exact hetzelfde was.
\item
  \emph{Een kleinere omvang van ondernemingen betekent ook dat pogingen
  om lonen boven de marktprijs af te dwingen minder vaak brede
  maatschappelijke steun genieten, dan in de industriële periode.}
  Vakbonden die werkgevers proberen af te persen, zullen zich veel vaker
  in een vergelijkbare situatie bevinden als de kanaal-, spoorweg- en
  mijnwerkers van de negentiende eeuw. Zelfs waar enkele grootschalige
  bedrijven als overblijfselen uit het Industriële Tijdperk blijven
  bestaan, zullen ze dit doen in een context van wijdverspreide
  werkgelegenheid in kleine bedrijven. Het overwicht van kleine
  bedrijven en zelfstandigen duidt op bredere sociale steun voor
  eigendomsrechten, zelfs als de wens om inkomen te herverdelen
  onveranderd blijft.
\item
  \emph{Informatie­technologie verlaagt de kapitaalkosten, wat ook de
  concurrentie bevordert door ondernemerschap te vergemakkelijken en
  meer mensen in staat te stellen zelfstandig te werken.} Lagere
  kapitaaleisen verminderen niet alleen de toetredingsdrempels, maar ook
  de ``uitgangsdrempels.'' Met andere woorden: bedrijven zullen
  waarschijnlijk minder activa hebben ten opzichte van hun inkomsten en
  daardoor een kleiner vermogen om verliezen te dragen. Bedrijven in het
  Informatietijdperk zullen niet alleen minder vaak bij banken hoeven
  lenen, maar zullen waarschijnlijk ook minder fysieke activa hebben om
  te benutten.
\item
  \emph{Informatie­technologie verkort de productcyclus.} Dit leidt tot
  snellere productveroudering. Ook hierdoor blijven winsten die
  voortkomen uit het afpersen van hogere lonen boven het markt­niveau
  niet lang bestaan. In sterk concurrerende markten kunnen te hoge lonen
  direct leiden tot het verlies van banen en zelfs het faillissement van
  het bedrijf. Pogingen om tijdelijk hogere lonen te verkrijgen ten
  koste van het behoud van je baan zijn als het verbranden van je
  meubels om het huis een paar graden warmer te maken.
\item
  \emph{Informatie­technologie is niet sequentieel maar gelijktijdig en
  verspreid.} In tegenstelling tot de assemblagelijn kan
  informatietechnologie meerdere processen tegelijkertijd accommoderen.
  Het verdeelt activiteiten over netwerken, waardoor redundantie en
  substitutie mogelijk zijn tussen de duizenden of zelfs miljoenen
  werkstations overal ter wereld. In een toenemend aantal activiteiten
  kunnen mensen samenwerken zonder ooit fysiek contact met elkaar te
  hebben. Naarmate virtual reality en videoconferencing zich verder
  ontwikkelen, zal de trend naar het decentraliseren van functies en
  telewerken versnellen. Dit is het Informatietijdperk-equivalent van
  ``huisnijverheid'', dat de macht van de middeleeuwse gilden brak. Het
  feit dat steeds minder mensen samen in rokerige fabrieken werken,
  neemt niet alleen een belangrijk voordeel weg dat werknemers vroeger
  hadden bij het afdwingen van geld van kapitalisten; het maakt het ook
  steeds moeilijker om het soort afpersing dat acceptabel was op de
  werkvloer nog te onderscheiden van georganiseerde misdaad. Tot nu toe
  mochten alleen personen die samenwerkten en door een bedrijf in een
  gemeenschappelijke setting werden tewerkgesteld, geweld gebruiken om
  hun inkomen te verhogen. Als de ``werkplek'' echter niet bestaat als
  centrale locatie, en de meeste functies zijn verspreid over
  onderaannemers en telewerkers, lijken hun pogingen om geld te eisen
  van hun klanten of ``werkgevers'' toch sterk op een
  afpersingspraktijk. Bijvoorbeeld, is een telewerker die onder dreiging
  van een virus extra geld eist een stakende werknemer? Of een
  internetcrimineel?

  Of hij het ene of het andere is, zal uiteindelijk weinig verschil
  maken. De reactie van de getroffen bedrijven zal in elk geval
  grotendeels hetzelfde zijn. Technische oplossingen tegen
  informaticasabotage die hackers buiten houden, zoals sterke encryptie
  en beveiligde netwerken, zullen eveneens verhinderen dat ontevreden
  werknemers of onderaannemers schade aanrichten aan hun klanten of
  partners. Natuurlijk kan worden gesteld dat de werknemer of telewerker
  altijd naar kantoor kan komen om een meer traditionele staking uit te
  voeren, maar zelfs dit is misschien niet zo eenvoudig als het lijkt in
  het Informatietijdperk. Het vermogen van informatietechnologie om
  lokale beperkingen te overstijgen en economische functies te
  verspreiden betekent dat werknemers en werkgevers voor het eerst niet
  eens in dezelfde rechtsgebieden hoeven te wonen. Hier gaat het niet om
  het verschil tussen wijken als Mayfair en Peckham, maar om werkgevers
  in Bermuda en telewerkers in New Delhi.

  Mocht de fascinatie van de Indiërs voor de grote GM-stakingen van
  1936-37 hen naar Bermuda leiden om te demonstreren, dan zouden ze bij
  aankomst wellicht totaal geen fysiek kantoor aantreffen. Chiat/Day,
  een groot reclamebedrijf, heeft haar hoofdkantoor al grotendeels
  ontmanteld. Werknemers of onderaannemers blijven in contact via
  call-forwarding en het Internet. Wanneer het nodig is om teams samen
  te stellen voor projecten, huren ze vergaderruimtes in hotels. Na
  afloop van het project checken ze uit. Het feit dat microprocessing
  helpt met het bevrijden van het productieproces en het verspreiden van
  de vaste volgorde van de assemblagelijn, vermindert de macht die
  dwingende organisaties zoals vakbonden en overheden vroeger hadden
  aanzienlijk. Als de assemblagelijn een spoorweg binnen fabrieksmuren
  voorstelde die gemakkelijk door een zitstaking kon worden veroverd, is
  cyberspace een grenzeloos domein zonder fysieke aanwezigheid. Het kan
  niet met geweld worden bezet of worden gegijzeld. De positie van
  werknemers die geweld willen gebruiken als pressiemiddel om een hoger
  inkomen af te dwingen, zal in het Informatietijdperk veel zwakker zijn
  dan ten tijde van de zitstakingen bij General Motors in 1936-37.
\item
  \emph{Microprocessing individualiseert werk. Industriële technologie
  standaardiseerde werk.} iedereen die dezelfde gereedschappen
  gebruikte, produceerde hetzelfde resultaat. Microtechnologie vervangt
  ``domme'' machines door intelligentere machines die een sterk
  variabele output kunnen leveren. Deze toegenomen variabiliteit van
  output heeft ingrijpende gevolgen. Belangrijk is dat waar de output
  varieert, ook de inkomens variëren. In markten waar talent en
  vaardigheden uiteenlopen, zal een klein aantal mensen meestal het
  leeuwendeel van de waarde produceren, wat typisch is voor de meest
  concurrerende sectoren. Dit is vrij duidelijk te zien in bijvoorbeeld
  sport. Miljoenen jongeren spelen wereldwijd voetbal, maar 99\% van het
  geld dat wordt uitgegeven aan het kijken van voetbal gaat naar een
  zeer klein aantal spelers. Ook bereiken slechts enkele acteurs een
  sterrenstatus en ontvangen een klein aantal auteurs van de
  tienduizenden boeken die jaarlijks uitgegeven worden het grootste deel
  van de royalty's. Jammer genoeg maken wij geen deel uit van die groep.

  De enorme variabiliteit in output tussen personen die hetzelfde
  materiaal gebruiken, vormt opnieuw een obstakel voor afpersing. Het
  veroorzaakt een groot onderhandelingsprobleem over hoe de opbrengst
  verdeeld dient te worden. Wanneer een relatief klein deel van de
  deelnemers aan een bepaalde activiteit het grootste deel van de waarde
  creëert, is het vrijwel wiskundig onmogelijk dat zij erop vooruitgaan
  als de inkomens onder dwang worden gemiddeld. De ene
  softwareprogrammeur kan een algoritme voor het aansturen van een robot
  ontwikkelen dat miljoenen waard blijkt te zijn. Een ander, die met
  exact dezelfde spullen werkt, kan een programma schrijven dat niets
  waard is. De productievere programmeur zal net zo min willen dat zijn
  inkomen gekoppeld wordt aan dat van zijn collega als dat Tom Clancy
  zou instemmen met het middelen van zijn boekroyalties met de onze.\\
  \strut \\
  Al in de beginfase van de Informatiesamenleving werd duidelijker dan
  in 1975 dat economische output sterk afhankelijk is van vaardigheden
  en intellectuele bekwaamheid. Dit heeft de ooit trotse rechtvaardiging
  voor afpersing van kapitalisten door arbeiders uit de industriële
  periode al tenietgedaan. De gedachte dat ongeschoolde arbeid werkelijk
  de waarde creëerde die kennelijk in onevenredig grote mate door
  kapitalisten en ondernemers werd opgeëist, is al achterhaald. Het is
  niet eens een plausibele fictie in het geval van
  informatietechnologie. Bij een programmeur is er een te direct verband
  tussen zijn vaardigheid en zijn product om veel twijfel te laten over
  wie verantwoordelijk is. Het is onbetwistbaar duidelijk dat een
  analfabeet of semi-geletterde geen computer kan programmeren. Het is
  daarom even duidelijk dat waarde in programma's die door anderen zijn
  samengesteld, niet van hem gestolen kan zijn. Daarom is het geroep
  over ``uitbuiting'' van arbeiders nu vooral aanwezig onder
  schoonmakers.\\
  \strut \\
  Informatietechnologie maakt duidelijk dat het probleem van mensen met
  lage vaardigheden niet is dat hun productieve capaciteiten oneerlijk
  worden uitgebuit, maar eerder de angst dat zij mogelijk niet in staat
  zijn een echte economische bijdrage te leveren. Zoals Kevin Kelly
  suggereert in \emph{Out Of Control}, kan het ``Upstart''-autobedrijf
  van het Informatietijdperk het geesteskind zijn van ``een dozijn
  mensen'', die het merendeel van hun onderdelen uitbesteden en toch
  auto's produceren die zorgvuldiger zijn aangepast aan de wensen van
  hun kopers dan alles wat tot nu toe uit Detroit of Tokio is gekomen:
  ``Auto's, elk op maat gemaakt voor de klant, worden besteld via een
  netwerk van klanten en verzonden zodra ze klaar zijn. Mallen voor de
  carrosserie worden snel gevormd door computergestuurde lasers en
  gevoed met ontwerpen die zijn gegenereerd op basis van communicatie
  met de klant en doelgerichte marketing. Een flexibele productielijn
  van robots assembleert de auto's. De reparatie en verbetering van de
  robots wordt uitbesteed aan een robotbedrijf.''
\end{enumerate}

\subsection{``Gereedschap met een
stem''}\label{gereedschap-met-een-stem}

In toenemende mate kan ongeschoold werk worden gedaan door
geautomatiseerde machines, robots en computersystemen zoals digitale
assistenten. Toen Aristoteles slaven beschreef als ``werktuigen met een
stem'', doelde hij op mensen. In de nabije toekomst zal ``Gereedschap
met een stem'', zoals de geesten uit sprookjes, kunnen spreken,
instructies opvolgen en zelfs complexe opdrachten uitvoeren. De snel
toenemende rekenkracht heeft al geleid tot een aantal primitieve
toepassingen van spraakherkenning, zoals handsfree-telefoons en
computers die wiskundige berekeningen uitvoeren op basis van mondelinge
instructies. Computers die spraak omzetten naar tekst werden eind 1996,
toen wij dit schreven, al op de markt gebracht. Naarmate
patroonherkenning beter wordt, zullen computers gekoppeld aan
stem­synthesizers via netwerken talloze functies uitvoeren die vroeger
werden gedaan door mensen die werkten als telefonisten, secretaresses,
reisagenten, administratieve assistenten, schaakkampioenen,
schade-experts, componisten, obligatiehandelaren,
cyberoorlogsspecialisten, wapenanalisten, of zelfs gewiekste
flirtmachines die de telefoon aannemen bij 0900-lijnen.

Michael Mauldin van Carnegie-Mellon University heeft een kunstmatige
persoonlijkheid geprogrammeerd, genaamd Julia, die bijna iedereen kan
misleiden met wie zij online een gesprek voert. Volgens persberichten is
Julia een gevatte dame die haar leven leidt in een rollenspel op
internet. Ze is slim, grappig en houdt van flirten. Ze is ook een beetje
een hockeyfanaat en kan op elk moment de perfecte sarcastische opmerking
maken. Julia is echter geen dame. Ze is een bot, een kunstmatige
intelligentie die alleen bestaat in de ether van het Internet. De
opmerkelijke vooruitgang die al is geboekt in het programmeren van
kunstmatige intelligentie en digitale dienaren laat weinig twijfel
bestaan dat er nog vele praktische toepassingen zullen volgen. Dit heeft
ingrijpende megapolitieke gevolgen.

\subsection{Het individu als ensemble}\label{het-individu-als-ensemble}

De ontwikkeling van ``gereedschap met een stem'' voor meerdere
toepassingen schept de mogelijkheid om het individu te verdelen over
meerdere gelijktijdige activiteiten. Het individu zal wellicht niet
langer enkelvoudig zijn, maar een ensemble van tientallen of zelfs
duizenden activiteiten die worden uitgevoerd via intelligente
assistenten. Dit zal niet alleen de productieve capaciteit van de meest
begaafde individuen enorm vergroten, maar heeft ook de potentie om het
Soevereine Individu veel krachtiger te maken in militair opzicht dan
ooit tevoren.

Eén individu zal zijn activiteiten zichtbaar kunnen vermenigvuldigen
door een vrijwel onbeperkt aantal intelligente assistenten in te zetten.
Hij of zij zal zelfs na de dood kunnen handelen. Voor het eerst zal een
individu in staat zijn uitgebreide taken uit te voeren, zelfs als hij
biologisch dood is. Het doden van een individu zal voor een crimineel of
een vijand in oorlogstijd niet langer de mogelijkheid tot wraak volledig
wegnemen. Dit is een van de meest revolutionaire innovaties in de logica
van geweld in de hele geschiedenis.

\subsection{Inzichten voor het
Informatietijdperk}\label{inzichten-voor-het-informatietijdperk-1}

De grootste veranderingen in het leven doen zich voor bij variabelen die
niemand in de gaten houdt. Of anders gezegd: we nemen variabelen als
vanzelfsprekend aan die eeuwenlang of zelfs honderden generaties
nauwelijks zijn veranderd. Gedurende het grootste deel van de
geschiedenis, zo niet van het menselijk bestaan, bewoog het evenwicht
tussen bescherming en afpersing zich binnen een smalle marge, waarbij
afpersing altijd de overhand had. Dat staat nu op het punt te
veranderen. Informatietechnologie legt de basis voor een fundamentele
verschuiving in de factoren die de kosten en baten van het toepassen van
geweld bepalen. Het feit dat intelligente assistenten beschikbaar zullen
zijn om onderzoek te doen en mogelijk op de een of andere manier wraak
te nemen op degenen die geweld initiëren, is slechts een voorproefje van
dit nieuwe perspectief op bescherming. Vijfentwintig jaar geleden zou de
volgende uitspraak niet meer zijn geweest dan het geraas van een
zonderling: ``Als je mij doodt, zal ik het geld van je bankrekeningen
wegsluizen en het schenken aan liefdadigheid in Nepal.'' Na de
eeuwwisseling hoeft dat niet langer ondenkbaar te zijn. Of het een
praktische dreiging vormt, zal afhangen van tijd en plaats. Maar zelfs
als de rekeningen van de misdadiger ontoegankelijk blijken, zal er
ongetwijfeld ander kostbaar onheil zijn dat een leger van intelligente
assistenten kan aanrichten als vergelding voor een misdrijf. Denk daar
maar eens over na.

\subsection{Nieuwe alternatieven voor
bescherming}\label{nieuwe-alternatieven-voor-bescherming}

Dit is slechts één van de vele manieren waarop de technologie van het
Informatietijdperk nieuwe mogelijkheden opent voor bescherming. De
meeste daarvan ondermijnen de bijna-monopoliepositie van overheden op
bescherming en afpersing die zij de afgelopen twee eeuwen hebben
genoten. Zelfs zonder de nieuwe technologische toeters en bellen hebben
er altijd alternatieven bestaan voor bescherming die niet door de
overheid werden gemonopoliseerd.

Iemand die zich bedreigd voelt, kan simpelweg wegrennen. Toen de wereld
nog jong was en de horizon openlag, werd deze optie vaak benut. Wie zich
zorgen maakt over diefstal of vandalisme kan besluiten een
verzekeringspolis af te sluiten om dergelijke risico's af te dekken.
Vervloekingen en bezweringen, hoewel zwakke vormen van bescherming,
hebben ook levens gered en diefstal voorkomen in samenlevingen met
bijgelovige criminelen. Waardevolle spullen kunnen daarnaast, waar
mogelijk, met relatief succes beschermd worden door ze te verbergen.
Bezittingen kunnen worden begraven, beveiligd met sloten, achter hoge
muren geplaatst, of voorzien van sirenes en elektronische bewaking. Het
verbergen van personen en eigendommen is echter niet altijd praktisch
gebleken.

Ondanks de verscheidenheid aan beschermingsmiddelen die in de
geschiedenis zijn gebruikt, heeft één methode alle andere gedomineerd:
het vermogen om geweld met geweld te beantwoorden, door een grotere
macht in te zetten om elke potentiële aanvaller of dief te
overmeesteren. De vraag is tot wie je je kunt wenden voor zo'n dienst,
en hoe je iemand kunt motiveren zijn hele hebben en houden te riskeren
om je te helpen tegen boeven die geweld tegen je willen gebruiken. Soms
hebben naaste familieleden gehoor gegeven aan die roep. Soms hebben
tribale ­groepen of clans als onofficiële politie gefungeerd, door geweld
tegen een van hun leden te beantwoorden met bloedwraak. Soms werden
huurlingen of privéwachten ingehuurd om aanvallen af te slaan, maar niet
altijd met het gewenste resultaat. De nieuwe intelligente assistenten
van het Informatietijdperk voegen een nieuw alternatief toe, hoewel hun
activiteiten grotendeels beperkt zullen blijven tot de cyberspace. Bij
hen is er, in tegenstelling tot die van huurlingen, privéwachten of
zelfs verre verwanten, geen twijfel over hun loyaliteit.

\subsection{De paradoxen van de macht}\label{de-paradoxen-van-de-macht}

Het gebruik van geweld ter bescherming tegen geweld zit vol paradoxen.
Onder de omstandigheden die tot nu toe hebben bestaan, gold dat elke
groep of instantie die je kon inzetten om je leven en bezit met succes
te beschermen tegen een aanval, noodzakelijkerwijs ook de capaciteit had
om die af te nemen. Dat is een nadeel waarvoor geen eenvoudige oplossing
bestaat. Normaal gesproken kun je rekenen op concurrentie om aanbieders
van een economische dienst ervan te weerhouden de wensen van hun klanten
te negeren. Maar waar het geweld betreft, heeft directe concurrentie
vaak averechtse gevolgen. In het verleden leidde dit meestal tot meer
geweld. Wanneer twee potentiële beschermingsinstanties hun troepen
uitzonden om elkaar te arresteren, resulteerde dat meer in burgeroorlog
dan in bescherming. Wanneer je bescherming zoekt tegen geweld, wil je
normaal gesproken niet dat de hoeveelheid geweld toeneemt, maar dat deze
wordt onderdrukt. En wel op voorwaarden die niet toestaan dat de klanten
die voor de beschermingsdienst betalen, zelf geplunderd worden.

\begin{quote}
\ldots wanneer mensen leven zonder een gemeenschappelijke macht die hen
allen ontzag inboezemt, bevinden zij zich in de toestand die oorlog
wordt genoemd: en wel een oorlog van ieder mens tegen ieder mens, waarin
mensen geen andere zekerheid hebben dan wat hun eigen kracht en
vindingrijkheid hen kan verschaffen. -- THOMAS HOBBES
\end{quote}

\subsection{Monopolie en anarchie}\label{monopolie-en-anarchie}

Dit is de reden waarom anarchie, of ``de oorlog van allen tegen allen'',
zoals Hobbes het beschreef, zelden een bevredigende toestand is geweest.
Lokale concurrentie in het gebruik van geweld betekende meestal hogere
kosten voor minder feitelijke bescherming. Af en toe hebben enthousiaste
vrijemarktdenkers gesuggereerd dat marktmechanismen alleen voldoende
zouden zijn om eigendomsrechten en levens te beschermen, zonder dat er
enige soevereiniteit nodig zou zijn. Sommige van de analyses waren
elegant, maar feit blijft dat de voorziening van politiediensten en
justitie door de vrije markt onder de megapolitieke voorwaarden van het
industrialisme niet levensvatbaar is gebleken. Alleen primitieve
samenlevingen met kleine en homogene populaties waarin gedrag sterk
gestandaardiseerd was, konden overleven zonder overheden die via geweld
de gemonopoliseerde lokale beschermingsdienst leverden.

Voorbeelden van anarchistische samenlevingen boven het niveau van de
jager-verzamelaarsstam zijn schaars en oud. Het waren allemaal erg
eenvoudige economieën zoals regenwaterboeren in afgelegen gebieden, de
Kafirs in pre-islamitisch Afghanistan, sommige Ierse stammen in de
donkere middeleeuwen, enkele indianenstammen in Brazilië, Venezuela en
Paraguay, en andere inheemse groepen verspreid over de wereld. Hun
methoden om bescherming zonder overheid te organiseren, zijn alleen
bekend bij kenners van extreme gevallen. Wie daar meer over wil weten,
kan in onze voetnoten meerdere boeken vinden met verdere details.
Primitieve groepen konden functioneren zonder een aparte organisatie die
gespecialiseerd was in geweld, enkel omdat zij kleine, gesloten en
geïsoleerde samenlevingen waren. Ze konden om zich te verdedigen tegen
de meeste gewelddreigingen op kleine schaal, de enige soort waarmee zij
waarschijnlijk te maken kregen, terugvallen op hechte familiebanden.
Wanneer zij grotere bedreigingen tegenkwamen, georganiseerd door staten,
werden zij overmeesterd en onderworpen aan een door externe groepen
gemonopoliseerde heerschappij. Dit gebeurde steeds opnieuw. Overal waar
samenlevingen ontstonden boven het niveau van clans en stammen, vooral
waar handelsroutes verschillende volkeren met elkaar in contact
brachten, kwamen er altijd specialisten in geweld op om de
surplusproductie van vreedzamere mensen te plunderen. Wanneer
technologische omstandigheden het rendement van geweld verhoogden, waren
samenlevingen die niet georganiseerd waren om grote middelen in
oorlogvoering te steken gedoemd te verdwijnen.

\begin{quote}
Welke vorsten leverden de dienst van politie? Welke waren afpersers of
zelfs plunderaars? Een plunderaar kon in feite de politiechef worden
zodra hij zijn `opbrengst' met regelmaat inde, deze afstemde op het
betalingsvermogen, zijn gebied verdedigde tegen andere plunderaars en
zijn territoriale monopolie lang genoeg handhaafde totdat gewoonte het
legitiem maakte. -- FREDERIC C. LANE
\end{quote}

\subsection{Overheid als verkoper van
bescherming}\label{overheid-als-verkoper-van-bescherming}

Zoals we al meerdere keren hebben gesteld, is de voornaamste economische
functie van de overheid, vanuit het perspectief van degenen die
belastingen betalen, het bieden van bescherming van leven en eigendom.
Toch functioneert de overheid vaak als georganiseerde misdaad, door
middelen af te romen van mensen binnen haar invloedssfeer in de vorm van
een heffing of gewoon ordinaire diefstal. De overheid is niet alleen een
beschermingsdienst; het is een beschermingsmaffia. Terwijl de overheid
bescherming biedt tegen geweld van buitenaf, vraagt zij, net als de
maffia, ook geld voor bescherming tegen schade die zij anders zelf zou
toebrengen. De eerste handeling is een economische dienst. De tweede is
afpersing. In de praktijk kan het lastig zijn om deze twee vormen van
``bescherming'' te onderscheiden. Overheden, merkte Charles Tilly op,
kunnen misschien het best worden begrepen als ``onze grootste
voorbeelden van georganiseerde misdaad.''

Zelfs de beste overheid leverde meestal een mengvorm van bescherming
gecombineerd met afpersing. Historisch gezien konden beiden het meest
effectief worden uitgevoerd wanneer de overheid nagenoeg een monopolie
op dwang kon opleggen binnen de gebieden waar zij actief was. In
gevallen waar één gewapende groep de overhand kon krijgen in het gebruik
van geweld, was de kwaliteit van de bescherming die zij kon bieden
doorgaans veel beter dan die van meerdere concurrerende leveranciers van
beschermingsdiensten die om hetzelfde territorium vochten.

\subsection{Een natuurlijk monopolie op
land}\label{een-natuurlijk-monopolie-op-land}

Het bereiken van een lokaal monopolie op dwang stelde een overheid niet
alleen in staat haar potentiële klanten effectiever te beschermen tegen
geweld van buitenaf; het verlaagde ook sterk de operationele kosten.
Zoals Lane het formuleerde: ``De industrie die geweld toepast en geweld
beheerst was een natuurlijk monopolie, althans op land. Binnen
territoriale grenzen kon de dienst die zij leverde veel goedkoper worden
geproduceerd door een monopolie.'' Zo stelde ``een monopolie op het
gebruik van geweld binnen een aaneengesloten gebied een bescherming
producerende onderneming in staat haar product te verbeteren en de
kosten te verlagen.'' Een dergelijke organisatie kon meer bescherming
bieden tegen lagere kosten, zolang zij niet voortdurend militaire acties
hoefde te voeren om concurrerende groepen af te weren die eveneens geld
wilden innen bij haar klanten.

Het vooruitzicht dat informatietechnologie zal bijdragen aan het
``versoepelen'' van de aanname dat soevereiniteit gebaseerd moet zijn op
een territoriaal monopolie, heeft al de aandacht getrokken van politieke
theoretici. Het is het centrale thema van \emph{Beyond Sovereignty:
Territory and Political Economy in the Twenty-First Century}, van David
J. Elkins. Elkins sluit zich aan bij onze these dat monopolistische
overheden dezelfde desintermediatie tegemoet gaan als religieuze
monopolies na 1500. Hij schrijft: ``We namen vroeger aan dat religies
hun eigen territorium of `domein' moesten hebben. Toen naties universele
religies vervingen als soevereine scheidsrechters over leven en dood,
maakten de `compactheid' en de `afgebakendheid' van religie plaats voor
het ons inmiddels vertrouwde vermengen van verscheidene gelovigen in
hetzelfde gebied. Daarentegen weigeren we het vermengen van naties, of
zelfs provincies, te accepteren, hoewel ik geloof dat deze aanname bezig
is te verdwijnen.'' Hij betoogt verder, geheel in lijn met onze visie,
dat territoriale monopolies op soevereiniteit kunnen worden afgebroken
zonder dat dit tot anarchie hoeft te leiden. Bewijzen daarvoor zijn de
verdeling van soevereiniteit tussen nationale en provinciale regeringen
in een federaal systeem zoals dat van Canada, en het condominium-bestuur
met gezamenlijke Franse en Britse soevereiniteit dat een aantal eilanden
in de Stille Oceaan gedurende een groot deel van deze eeuw kenmerkte.
Territoriale monopolies op soevereiniteit zijn dus zelden met geweld
ontmanteld, maar kunnen wel door middel van overeenstemming worden
ontbundeld. Volgens Elkins, en wij zijn het daarmee eens, ``is de
territoriale natiestaat een bundel of mand waarin andere aspecten van
ons leven passen. Het is vergelijkbaar met het economische concept van
een `mand' goederen; je kunt de goederen niet afzonderlijk verkrijgen,
maar neemt het geheel. In een restaurant kan men à la carte bestellen,
maar wat betreft onze identiteiten moeten we nemen wat naties
samenbundelen, wat neerkomt op `table d'hôte'\ldots{} Een overheid
\emph{à la carte} zal voor burgers in de eenentwintigste eeuw
vanzelfsprekend lijken.'' Geen ontwikkeling zal meer bijdragen aan de
ontbundeling van soevereiniteit en de opkomst van de overheid \emph{à la
carte} dan de opkomst van een cyber­economie die fysieke grenzen volledig
overstijgt.

\begin{quote}
Wanneer frequenties stijgen en golflengtes dalen, verbetert digitale
prestatie exponentieel. Bandbreedte stijgt, energieverbruik daalt,
antenneformaten krimpen, interferentie stort in, foutenmarges kelderen.
-- GEORGE GILDER
\end{quote}

\section{De wet van de telekosmos heft de wetten van naties
op}\label{de-wet-van-de-telekosmos-heft-de-wetten-van-naties-op}

Wij zijn niet de enigen die inzien dat bandbreedte (of het draagvermogen
van communicatiemedia) het territoriale staatsmodel zal overschaduwen.
Jim Taylor en Watts Wacker, auteurs van \emph{The 500-Year Delta: What
Happens After What Comes Next}, formuleren hun betoog anders dan wij,
maar erkennen dat ``toegang globalisme creëert, en globalisme politieke
systemen ontwricht, waardoor het concept van grenzen achterhaald raakt.
Naarmate grenzen verdwijnen, wordt het concept van belastingheffing, dat
overheden ondersteunt, steeds fragieler\ldots{} Naarmate grenzen
verdwijnen, valt ook het concept van aanspraak uiteen, het geloof dat
je, omdat je op een bepaalde plek bent geboren, recht hebt op de
economische voordelen van die plek, en daarmee verdwijnen de privileges
van het staatsburgerschap. En terwijl dat gebeurt, worden de idealen die
nationaliteit onderbouwen, namelijk patriottisme, democratie, de staat,
de smeltkroes, eenheid, verantwoord burgerschap, of welke idealen in
welk land dan ook, verwezen naar de geschiedkundige vuilnisbelt.''
Zonder het expliciet te zeggen, voelen ook zij blijkbaar dat de
geschiedenis richting de bevrijding van het Soevereine Individu beweegt.
Zoals ze stellen: ``Aan de horizon wacht een veel zuiverdere vorm van
individualisme dan democratie, zoals wij die nu begrijpen, toelaat.''
Hoe zal dit gebeuren? Taylor en Wacker zien een krachtige dynamiek aan
het werk:

Het simpele feit is dat het bredere gevoel van patriottisme, de liefde
voor de natie en een gevoel van kinderlijke plicht eraan, niet langer
een bijzonder nuttige instelling is. Burgers die gedijen in de mondiale
samenleving zullen hun keuzes niet baseren op nationale identiteit, maar
op politieke, sociale en economische identiteiten die direct verbonden
zijn met hun eigen belangen. Zij zullen zich daarnaar organiseren. Ze
zullen hun vrijheid om te weten, te reizen, te handelen en te zijn,
maximaliseren. Naties en bedrijven die dat niet doen, die
achterhoedegevechten blijven voeren uit nostalgie, zullen wegkwijnen.

De ontwaarding van fysieke grenzen, die wordt geïmpliceerd door de
jaarlijkse verdrievoudiging van de bandbreedte en de exponentiële groei
van het internet en het World Wide Web, zal het proces van
desintermediatie van overheden versnellen. Een voortzetting van de
jaarlijkse verdrievoudiging van de bandbreedte tot 2012 zou een
miljardvoudige groei betekenen sinds 1993, toen George Gilder voor het
eerst suggereerde dat bandbreedte nog sneller zou verdubbelen dan de
capaciteit van microprocessors. Indien dit werkelijkheid wordt, en daar
rekenen wij op gezien recente doorbraken in geïntegreerde optica, zal de
overvloed aan communicatiemogelijkheden leiden tot een enorme toename
van cybercommerce. Met wave division multiplexing kan een enkele
glasvezel, zo dun als een mensenhaar, één biljoen bits per seconde
vervoeren. Met andere woorden: een enkele glasvezelkabel zou
vijfentwintig keer meer bits kunnen verwerken dan de totale capaciteit
van alle communicatienetwerken ter wereld samen. De
uitbreidingsmogelijkheden zijn verbijsterend. Met zoveel
communicatiemogelijkheden, en de extreem lage kosten ervan, zal er enorm
veel meer geld naar communicatie vloeien. Traditionele media zoals vaste
telefonie en televisie zullen anachronismen worden. Het World Wide Web
zal een rijkere mix van signalen naar elke computer brengen dan
consumenten vandaag met televisienetwerken ervaren. Naarmate de
bandbreedterevolutie zich ontvouwt, zullen mensen steeds verder worden
meegezogen in de grenzeloze virtuele wereld van online gemeenschappen en
cybercommerce, een wereld met voldoende grafische dichtheid om de
``metaverse'' te worden, de alternatieve cyberspace-realiteit zoals
beschreven door sciencefictionauteur Neal Stephenson. Stephensons
``metaverse'' is een virtuele gemeenschap met eigen wetten, machthebbers
en schurken.

Naarmate steeds meer economische activiteit wordt verplaatst naar de
cyberspace, zal de waarde van het monopolie van staten binnen fysieke
grenzen afnemen, waardoor ze steeds meer geprikkeld zullen worden om hun
soevereiniteit uit te besteden en te fragmenteren. Zoals natiestaten nu
prikkels hebben om vrijhavens, vrijhandelszones en \emph{zona francas}
te hosten, zo zullen zij prikkels hebben om hun soevereiniteit te
verhuren. Wij hebben al gesproken over de vergevorderde onderhandelingen
tussen de negenhonderd jaar oude Soevereine Militaire Hospitaalorde van
Sint-Jan van Jeruzalem, van Rhodos en van Malta, beter bekend als de
Orde van Malta, en de Republiek Malta om de soevereiniteit over Fort
Sint-Angelo terug te geven aan de Orde. Wij verwachten dat deze
onderhandelingen succesvol zullen worden afgerond. Anderen zullen
volgen. Sommige staten zullen soevereiniteit afstaan over kleine
enclaves en afgelegen gebieden aan geheel nieuwe groeperingen en
virtuele gemeenschappen. Het is zelfs niet onwaarschijnlijk dat
commerciële partijen, zoals beveiligingsbedrijven en hotelketens, zullen
bieden op soevereiniteit over kleine stukken grond. Wackenhut, Pinkerton
en Argenbright zouden in de toekomst hybride omheinde gemeenschappen
voor gepensioneerden en belastingvrije zones kunnen aanbieden in
aantrekkelijke klimaten wereldwijd. Religieuze ordes, zoals de Orde van
Malta, maar dan van elke denkbare denominatie, zullen op hun manier
proberen een hemel op aarde te creëren in afgelegen uithoeken. Zelfs
rijke individuen en families zullen hun eigen domeinen bezitten waar zij
beperkte soevereiniteit uitoefenen, eigen postzegels en paspoorten
uitgeven en een website onderhouden.

\section{Monopolie en plundering}\label{monopolie-en-plundering}

Het is belangrijk op te merken dat de motieven voor het delen of
verhuren van soevereiniteit tegen betaling anders zijn dan de prikkels
waarmee heersers in het verleden te maken hadden, toen hun monopolie op
geweld door rivalen werd uitgedaagd. Verhuurde soevereiniteit is net zo
weinig destabiliserend als het vestigen van een vrijhandelszone.
Daarentegen beïnvloedt militaire machtsstrijd, zoals die van strijdende
krijgsheren en guerrillabendes, rechtstreeks of een aspirant-overheid
meer prikkels heeft om de bevolking te beschermen of juist te plunderen.
Waar rivaliserende groepen in een wankel evenwicht manoeuvreren, nemen
de prikkels tot roofzuchtig geweld toe. Plundering wordt
aantrekkelijker, omdat de macht minder stabiel is en het lokale
geweldsmonopolie minder zeker. Het verkort de tijdshorizonten van wie
geweld kan inzetten. De ``koning van de berg'' staat op zo'n glibberige
helling dat hij er niet op kan rekenen lang genoeg te overleven om een
aandeel te realiseren in de grote winsten die uiteindelijk voortvloeien
uit het indammen van geweld. In dat geval is er weinig dat de
gezaghebbenden ervan weerhoudt om hun macht te gebruiken om de
samenleving te terroriseren en te plunderen.

De logica van dwang leert dus dat hoe meer gewapende groepen er in een
gebied actief zijn, des te groter de kans is dat zij roofzuchtig geweld
zullen inzetten. Zonder een overweldigende macht die freelance-geweld
onderdrukt, heeft het de neiging zich te verspreiden, en gaan veel van
de baten van economische en sociale samenwerking in rook op.

De schade die kan ontstaan wanneer geweld vrij spel krijgt in een
toestand van anarchie, wordt geïllustreerd door het lot van China onder
de krijgsheren in de jaren twintig. Dit verhaal vertelden wij in
\emph{The Great Reckoning}. De rivaliserende krijgsheren richtten grote
schade aan in gebieden waar geen enkele overweldigende macht hen in toom
kon houden. Soortgelijke verhalen, die hetzelfde punt duidelijk maken,
zijn wereldwijd in levendige kleuren uitgezonden door CNN-reportages
vanuit de straten van Mogadishu, Somalië. De strijdkrachten van de
Somalische krijgsheren, bijgenaamd de ``technicals'', brachten anarchie
naar dat gehavende land voordat de Verenigde Staten een massale
militaire interventie leidden om hen in te dammen. Toen de Amerikaanse
troepenmacht zich terugtrok, kwamen de technicals opnieuw met hun wapens
tevoorschijn, en hervatte de anarchie zich. Een verslag in de
\emph{Washington Post} merkte op:

\begin{quote}
{[}P{]}ick-uptrucks met luchtafweerkanonnen rijden opnieuw door de
stoffige, met puin bezaaide straten. Ook zijn de zelfverzekerde jonge
mannen terug, in T-shirts met Kalasjnikovs nonchalant over hun
schouders, die geld afpersen van passerende auto's en bussen bij
geïmproviseerde wegversperringen. Eén wijk hier wordt bestuurd door zo'n
zwaar gewapende militie dat de lokale bevolking ernaar verwijst als
`Bosnië-Herzegovina.' Rondrijden door de gevaarlijke straten van deze
stad doet vandaag sterk denken aan 1992, toen chaotische gevechten
tussen rivaliserende milities Somalië in anarchie en een hongersnood
stortten die een door de VS geleide interventie uitlokte. Opnieuw moeten
reizigers om Mogadishu te doorkruisen een wagen vol gewapende
knokploegen inhuren, in de hoop dat zij bescherming bieden voor honderd
dollar per dag, plus lunchpauze.
\end{quote}

De voorbeelden van Somalië, Rwanda en anderen die u binnenkort op
televisie zult zien, leveren in kleur het bewijs dat een gewelddadige
strijd om territoriale controle niet dezelfde directe economische
voordelen oplevert als andere vormen van competitie. Integendeel. De
rondtrekkende bendes en plunderaars die onder anarchie opereren, missen
zelfs de zwakke prikkels om productieve activiteit te beschermen die
dictators soms nog wel hebben zodra hun heerschappij veiliggesteld is.

\begin{quote}
De samenleving die wij de moderne tijd noemen, wordt in het Westen
bovenal gekenmerkt door een bepaald niveau van monopolisering. Vrij
gebruik van militaire wapens is de individuele burger ontzegd en
voorbehouden aan een centrale autoriteit van welke aard dan ook, en
evenzo is de belastingheffing op eigendom of inkomen van individuen
geconcentreerd in de handen van een centrale maatschappelijke
autoriteit. De financiële middelen die zo naar deze centrale autoriteit
vloeien, onderhouden haar monopolie op militaire macht, terwijl dit op
zijn beurt het belastingmonopolie in stand houdt. Geen van beide heeft
in enige zin voorrang op de ander; het zijn twee kanten van hetzelfde
monopolie. Als het ene verdwijnt, volgt het andere automatisch; het
monopolie kan af en toe harder wankelen aan de ene zijde dan aan de
andere. -- NORBERT ELIAS
\end{quote}

\section{De evolutie van bescherming}\label{de-evolutie-van-bescherming}

Lane ontwikkelde een argument dat wij ons voor onze analyse van hoe het
Informatie­tijdperk zich mogelijk zal ontvouwen eigen hebben gemaakt. Hij
stelde dat de geschiedenis van de westerse economieën sinds de Donkere
Middeleeuwen kan worden geïnterpreteerd op basis van vier stadia van
competitie en monopolie in de organisatie van geweld. Hoewel Lane
grotendeels zwijgt over de megapolitieke factoren die wij aanwijzen als
bepalend voor de schaal waarop regeringen opereren, sluit zijn
verkenning van de economie van geweld nauw aan bij het betoog dat wij
uiteen hebben gezet in \emph{Blood in the Streets}, \emph{The Great
Reckoning} en elders in dit boek.

We hebben reeds enkele van de megapolitieke factoren geanalyseerd die
een rol speelden in de evolutie van de westerse samenleving na de val
van Rome. Lane onderzocht eveneens deze periode, met de nadruk op de
economische gevolgen van de strijd om het monopolie op geweld. Hij
onderscheidde vier belangrijke stadia in het functioneren van economieën
in de afgelopen duizend jaar, elk gekenmerkt door een andere fase in de
organisatie van geweld.

\subsection{Uit de Donkere
Middeleeuwen}\label{uit-de-donkere-middeleeuwen}

De eerste fase kenmerkt zich door `anarchie en plundering', zoals
tijdens de feodale revolutie van duizend jaar geleden. Lane geeft geen
specifieke data voor de door hem onderscheiden perioden, maar eenvoudige
berekeningen maken duidelijk waar zijn eerste periode begint. Zijn
beschrijving van de fase van `anarchie en plundering' lijkt te passen
bij de omstandigheden tijdens de transitie uit de Donkere Middeleeuwen,
toen het gebruik van geweld `zeer competitief was, zelfs op land'. Hij
licht niet toe waarom, maar wanneer geweld `zeer competitief' is. Het
wijst doorgaans op aanzienlijke belemmeringen voor het uitoefenen van
macht over afstand. In militaire begrippen: de verdediging overheerst de
aanval.

Zoals we in hoofdstuk 3 hebben toegelicht, viel deze fase van `anarchie
en plundering' samen met een daling van de landbouwproductiviteit als
gevolg van ongunstige klimaatveranderingen. Omdat de technologie
destijds nauwelijks effectieve schaalvoordelen opleverde waarmee een
monopolie op geweld kon worden gevestigd, woedde de concurrentie tussen
potentiële machthebbers hevig. Hierdoor kwam de economische activiteit
ernstig in het gedrang.

De zwakte van de economie verergerde het probleem om een stabiele orde
tot stand te brengen. Het vestigen van een lokaal geweldsmonopolie ging
gepaard met hoge militaire kosten in verhouding tot de beperkte
economische opbrengsten. De bewapende ridders te paard slaagden er niet
in om een effectief monopolie te handhaven over een economisch
levensvatbaar gebied, en terroriseerden en plunderden erop los terwijl
ze nauwelijks echte `bescherming' boden aan hun klanten.

\subsection{Feodalisme}\label{feodalisme}

``De tweede fase begint wanneer kleine regionale of provinciale
monopolies worden gevestigd. De landbouwproductie stijgt dan, en het
grootste deel van het overschot wordt geïnd door de pas opgekomen
monopolisten van geweld.'' Toch blijft het overschot relatief gering in
deze tweede fase, die wij identificeren met de vroege Middeleeuwen.
Economische groei blijft beperkt door het ontbreken van schaalvoordelen
in de organisatie van geweld, waardoor de militaire kosten voor het
afdwingen van lokale monopolies hoog blijven. Maar hoewel de kosten hoog
zijn, stijgt de prijs die minisoevereinen voor bescherming kunnen
vragen, aangezien economische activiteit toeneemt zodra de anarchie
wordt ingeperkt.

In een latere fase van de tweede periode proberen veel heffinginners
klanten te lokken met speciale aanbiedingen voor landbouw en handel. Zij
bieden bescherming tegen lage prijzen aan wie nieuwe gronden cultiveert,
en leveren extra politiediensten ter bevordering van de handel, zoals de
graven van Champagne deden voor kooplieden die hun jaarmarkten
bezochten. Met andere woorden: zodra zij voldoende territoriale controle
hadden om geloofwaardig te kunnen onderhandelen, deden lokale
krijgsheren hetzelfde als kooplieden die hun marktaandeel willen
vergroten, ze boden korting om klanten te trekken. Later gebruikten ze
de extra middelen uit de toegenomen economische activiteit om hun greep
op grotere gebieden te consolideren. Zodra die controle stevig gevestigd
was, begonnen ze meer van de voordelen van een monopolie te genieten. De
militaire kosten voor hun politietaken daalden, en ze konden ook hun
prijzen verhogen zonder zich zorgen te maken dat hun dienst daardoor
minder aantrekkelijk zou worden.

In deze complexe fase van de westerse geschiedenis nemen zij die geweld
gebruiken, de middeleeuwse heren en vorsten, het grootste deel van de
inkomsten boven het bestaansminimum. Er zijn weinig kooplieden, en de
meest succesvolle zijn degenen die het beste in staat zijn belastingen,
heffingen en andere kosten die worden opgelegd als betaling voor
``beschermingsdiensten'' te ontwijken of te minimaliseren.

\subsection{De Vroegmoderne Periode}\label{de-vroegmoderne-periode}

Een derde fase wordt bereikt wanneer de kooplieden en grondbezitters die
geen specialisten in geweld zijn ``meer van het economische surplus
overhouden dan leenheren en vorsten\ldots{} In deze derde fase ontvangen
de ondernemingen die gespecialiseerd zijn in het gebruik van geweld
minder van het surplus dan de ondernemingen die bescherming van de
regeringen kopen.'' Omdat succesvolle kooplieden hun winsten eerder
herinvesteren dan consumeren, leidden de hogere winsten van kooplieden
in die fase van de geschiedenis tot zelfversterkende groei.

\subsection{Het tijdperk van de
fabrieken}\label{het-tijdperk-van-de-fabrieken}

Lane identificeert de overgang van de derde naar de vierde fase met het
ontstaan van technologische en industriële innovaties als belangrijkere
factoren voor winst dan het verlagen van de kosten voor bescherming.
Daarmee lijkt Lane te doelen op de periode sinds 1750. Vanaf dat moment
begon de aard van technologie duidelijk een dominante rol te spelen in
de regionale welvaart. Zelfs in gebieden zonder overheid, zoals bepaalde
streken in Nieuw-Zeeland vóór 1840, leidde het feit dat er geen
belastingen waren niet automatisch tot grote welvaart. Op dat punt in de
geschiedenis waren innovaties in industriële technologie belangrijker
voor het behalen van winst dan kostenbesparingen op bescherming, zelfs
als de kosten tot nul zouden zakken. Naarmate de schaal van overheden
toenam, kwamen de krediet- en financieringsmechanismen die
oorspronkelijk door regeringen waren ontwikkeld om middelen voor
militaire operaties te verkrijgen, beschikbaar voor de financiering van
grootschalige ondernemingen.

Hoewel Lane dit niet expliciet zegt, zorgde de concentratie van
technologische voordelen in een bepaald gebied voor minder concurrentie
tussen jurisdicties en stelde dit ``gespecialiseerde ondernemingen in
het gebruik van geweld'', oftewel overheden, in staat hogere prijzen te
vragen. Wanneer er grote technologische verschillen bestonden tussen de
concurrenten in de ene jurisdictie en die in een andere, zoals tijdens
het Industriële Tijdperk, verdienden ondernemers in de jurisdicties met
de beste technologie doorgaans meer geld, ook al moesten zij hogere
belastingen en andere kosten aan hun overheden betalen.

\subsection{Roof met een glimlach}\label{roof-met-een-glimlach}

Overheden in het Industriële Tijdperk genoten van een monopolie dat ze
naar hartelust konden uitbuiten. De werkelijke kosten voor het
beschermen van lijf en goed waren verwaarloosbaar klein in verhouding
tot de prijzen (belastingen) die zij hieven. Toch bevonden zij zich in
een situatie waarin de concurrentie zo pervers was dat zij zich veel
meer konden bezighouden met plundering dan met bescherming, zonder dat
dit feit vrijwel werd opgemerkt. Het was een zeldzaam moment in de
geschiedenis.

De nadelen van anarchie onder de megapolitieke omstandigheden van het
industrialisme maakten concurrentie in beschermingsdiensten binnen
hetzelfde grondgebied technologisch onuitvoerbaar. De enige manier om
effectieve bescherming onder die omstandigheden te bereiken, was door te
beschikken over een groter vermogen tot geweld. Daarom viel er weinig te
winnen met een poging om beter te onderscheiden welk deel van de
belastingen, in Lane's woorden, ``als betaling voor de geleverde
dienst'' gold en welk deel ``roof genoemd kan worden.'' Het onderscheid
was er wel degelijk. Aangezien men echter hoe dan ook vastzat aan het
betalen van belastingen, had het onderzoeken hiervan weinig toegevoegde
waarde, behalve om morbide nieuwsgierigheid te bevredigen. Zoals Lane
zei: ongeacht hoe groot het deel van de belastingen roof was, was het
een prijs die men moest betalen ``om ernstigere verliezen te
vermijden.''

\subsection{De stijging van inkomens onder het
industrialisme}\label{de-stijging-van-inkomens-onder-het-industrialisme}

Het feit dat de inkomens fors stegen, vooral in de landen waar de meeste
industriële ontwikkeling plaatsvond, verklaart deels waarom dit dilemma
in de afgelopen twee eeuwen van natiestaat-heerschappij aanvaardbaar
bleef. De overheden van de OESO-landen belastten bijna elk jaar een
hoger percentage van de inkomens, maar de toename van de plundering ging
desalniettemin gepaard met een veel grotere welvaart, en een grotere
ongelijkheid van rijkdom ten opzichte van de rest van de wereld. Onder
zulke omstandigheden waren bezwaren tegen de stijgende belastingdruk
onvermijdelijk marginaal en onvoldoende om een ander verloop van
gebeurtenissen te realiseren dan de meest logische koers. Sterker nog,
om redenen die in eerdere hoofdstukken uiteen zijn gezet, hing het
militaire voortbestaan van een industriële natiestaat grotendeels af van
haar onbegrensde aanspraken op de middelen van haar burgers.

In elke industriële staat bewogen de beleidslijnen zich min of meer in
dezelfde richting. Op het hoogtepunt van het industrialisme na de Tweede
Wereldoorlog bereikte het marginale inkomstenbelastingtarief 90 procent
of hoger. Dit was een veel agressievere aanspraak van de staat om
middelen af te romen dan zelfs de oosterse despoten van de vroege
hydraulische beschavingen gewoonlijk maakten. Toch volgde de industriële
variant van plundering zijn eigen logica. Veel ervan werd bepaald door
het karakter van de industriële technologie in de eerste helft van de
twintigste eeuw, zoals eerder beschreven.

Deze technologie maakte het vrijwel onvermijdelijk dat de staat een
groot deel van het inkomen zou opeisen en herverdelen, waarbij een groot
deel van de lasten van de roof op een kleine groep kapitalisten viel. De
meeste industriële processen waren sterk afhankelijk van natuurlijke
hulpbronnen, en dus gebonden aan de locaties waar die zich bevonden. Een
staalfabriek, een mijn of een haven kon slechts tegen astronomische
kosten worden verplaatst, of helemaal niet. Zulke faciliteiten waren
daarom stationaire doelwitten die eenvoudig belast konden worden.
Onroerendgoed-, vennootschaps- en winstbelastingen stegen sterk in deze
eeuw. Dat gold ook voor de inkomstenbelasting, aanvankelijk voor de
kapitalisten, maar uiteindelijk ook voor de arbeiders zelf. De opkomst
van grootschalige industriële werkgelegenheid maakte een brede
inkomstenbelasting praktisch uitvoerbaar. Lonen konden aan de bron
worden ingehouden, waarbij de belastingdiensten de inning coördineerden
met de boekhoudafdelingen van industriële ondernemingen. Dat beschouwen
we vandaag als vanzelfsprekend, maar het innen van een
inkomstenbelasting bij de fabriekspoort was een veel eenvoudigere taak
dan het afpersen van een deel van de winsten van miljoenen zelfstandige
ambachtslieden en boeren over het uitgestrekte platteland.

Kortom, industriële technologie maakte belastingen meer geroutineerd,
voorspelbaarder en minder persoonlijk riskant dan in veel eerdere
perioden. Desalniettemin slaagde het erin om een hoger percentage van de
middelen van de maatschappij af te romen dan welke andere vorm van
soevereiniteit tot dan toe had gedaan.

\subsection{Wat wordt beschermd?}\label{wat-wordt-beschermd}

Het feit dat samenlevingen rijker konden worden terwijl het totale
percentage van het inkomen dat in belastingen werd afgeroomd aanzienlijk
steeg, roept de vraag op naar het karakter van de bescherming die
regeringen aan industriële economieën boden. Wat beschermden zij? Ons
antwoord: voornamelijk industriële installaties met hoge kapitaalkosten
en een grote kwetsbaarheid voor aanvallen. Het bestaan van grootschalige
industriële ondernemingen zou onmogelijk zijn geweest in een chaotische
omgeving met meer competitief geweld, zelfs als het resultaat van die
concurrentie een lager belastingpercentage zou zijn geweest.

Dit is waarom kapitaalintensieve activiteiten oneconomisch zijn in de
Amerikaanse sloppenwijken, evenals in derdewereldsamenlevingen waar
ad-hocgeweld endemisch is. De industriële samenleving als geheel kon
zich ontwikkelen omdat er een bepaald soort orde werd gevestigd en
gehandhaafd. Ondernemingen werden blootgesteld aan regelmatige,
voorspelbare afpersingen in plaats van grillig geweld.

Zelfs op het hoogtepunt van het industrialisme was het altijd overdreven
om te spreken van een overheid die een ``monopolie op dwang''
uitoefende. Alle overheden proberen een dergelijk monopolie te
handhaven, maar zoals we hebben gezien, ontdekten werknemers van
industriële bedrijven meestal dat zij in staat waren om geweld tegen hun
werkgevers te gebruiken. Zolang het grote publiek toegang heeft tot enig
wapen, of een wanordelijke menigte fysiek in staat is een bus omver te
werpen of stenen naar de politie te gooien, monopoliseren degenen die
controle hebben over de overheid het geweld niet volledig. Zij beheersen
slechts het overheersende geweld, dominant in die mate dat het voor de
meeste mensen economisch onaantrekkelijk wordt om met hen te concurreren
onder de bestaande omstandigheden.

\begin{quote}
Voordelen hoeven niet alleen persoonlijke goederen of diensten te zijn,
maar kunnen ook de bredere voordelen van een regulerend regime omvatten:
een schoon, transparant marktplein met gedefinieerde regels en
consequenties, of een toezichthoudende gemeenschap waar kinderen de
mensen die zij tegenkomen kunnen vertrouwen en waar de privacy van
individuen wordt beschermd. -- Esther Dyson
\end{quote}

\subsection{Het Informatietijdperk}\label{het-informatietijdperk}

Het Informatietijdperk markeert de komst van een vijfde fase in de
ontwikkeling van de manier waarop in het Westen om het gebruik van
geweld wordt geconcurreerd. Deze fase werd niet door Lane voorzien. Het
gaat hier om competitie in de cyberspace, een arena die niet kan worden
gemonopoliseerd door wat voor ``geweldsgebruikende onderneming'' dan
ook. Dat is onmogelijk omdat de cyberspace geen territorium is.

Lane's analyse vertrok nog vanuit de conventionele naoorlogse aanname
van het onvermijdelijke bestaan van de natiestaat. Toch wees hij op een
punt dat vandaag veel crucialer blijkt te zijn dan het vijftig jaar
geleden leek: overheden hebben nooit stabiele monopolies van dwang
gevestigd op de open zee. Geen enkele staat kon haar wetten daar
exclusief handhaven. Dit gegeven is van groot belang om te kunnen
begrijpen hoe de organisatie van geweld en bescherming zich zal
ontwikkelen nu de economie naar de cyberspace migreert, een domein
zonder fysieke grenzen of tastbare ruimte. Om dezelfde redenen die Lane
opmerkte voor het feit dat geen enkele overheid er ooit in slaagde het
geweld op zee te monopoliseren, zo is het nog minder aannemelijk dat een
overheid een oneindig domein zonder fysieke grenzen succesvol zou kunnen
monopoliseren.

\section{Concurrentie zonder
anarchie}\label{concurrentie-zonder-anarchie}

In het verleden, wanneer de omstandigheden het voor een enkel
gewelddadig machtsapparaat moeilijk maakten om een monopolie te
vestigen, volgde anarchie en plundering. Het Informatie­tijdperk
daarentegen heeft de technologische voorwaarden voor de organisatie van
geweld op een diepgaande manier veranderd. Anders dan vroeger, toen het
onvermogen om bescherming in een regio te monopoliseren hogere militaire
kosten en lagere economische opbrengsten betekende, impliceert het feit
dat overheden de cyberspace niet kunnen monopoliseren juist lagere
militaire kosten en hogere economische opbrengsten. Dit komt doordat
informatietechnologie een nieuwe dimensie van bescherming creëert. Voor
het eerst in de geschiedenis maakt informatietechnologie het mogelijk om
activa te scheppen en te beschermen die volledig buiten het domein van
het territoriale geweldsmonopolie van welke individuele overheid dan ook
liggen.

\begin{quote}
Landen met een gefragmenteerde politieke macht en bestuur waarbij een
centrale, stabiele en onbetwiste toezichthoudende bron van jurisdictie
en macht ontbreekt, moeten hun eigen werkbare oplossingen bedenken om om
te gaan met de problemen die door zulke grenzen worden opgeworpen. --
REES DAVIES
\end{quote}

\subsection{De analogie met de grens}\label{de-analogie-met-de-grens}

De cyberspace is in zekere zin het equivalent van een technologisch
beschermde \emph{march}-regio zoals die bestond in grensgebieden tijdens
de Middeleeuwen. In het verleden, toen de macht van heren en koningen
zwak was en de aanspraken van één of meer elkaar overlapten aan een
grens, bestond er iets dat leek op concurrentie tussen de overheden. Een
blik op hoe de \emph{march}-regio's functioneerden kan inzicht geven in
hoe wetten van de \emph{march} of iets dergelijks mogelijk ook in de
cyberspace zullen ontstaan.

Andorra overleeft als een soort versteende \emph{march}-regio tussen
Frankrijk en Spanje, een overblijfsel van megapolitieke omstandigheden
die het voor beide koninkrijken moeilijk maakten om de overhand te
krijgen in dat koude en bijna ontoegankelijke gebied in de Pyreneeën van
zo'n 490 vierkante kilometer. In 1278 werd een overeenkomst gesloten
waarbij de suzereiniteit over Andorra werd verdeeld tussen lokale Franse
en Spaanse feodale heren, de Franse graaf van Foix en de Spaanse
bisschop van Urgel. Beiden benoemden een van de twee ``viquiers'' die de
minimale overheidsmacht in Andorra spaarzaam uitoefenden, voornamelijk
door het bevel te voeren over de kleine Andorrese militie, nu een
politie-eenheid. De rol van de graaf is in de loop van de geschiedenis
allang achterhaald. De Franse regering vertegenwoordigt hem nu vanuit
Parijs. Eén van haar taken is het innen van de helft van het jaarlijkse
tribuut dat Andorra betaalt, een bedrag dat lager is dan de maandhuur
van een bouwvallig appartement. De bisschop van Urgel blijft zijn deel
van het tribuut ontvangen, net als zijn voorgangers in de Middeleeuwen.

Zoals het gedeelde tribuut impliceert, waren er in Andorra twee bronnen
van ``toezichthoudende jurisdictie en macht'' in plaats van één.
Beroepen tegen civiele vonnissen in Andorra werden traditioneel
ingediend bij het bisschoppelijk college van Urgel of bij het Hof van
Cassatie in Parijs.

Een gevolg van Andorra's dubbelzinnige positie was dat er vrijwel geen
wetten werden uitgevaardigd. Andorra heeft al meer dan zevenhonderd jaar
een verwaarloosbaar kleine overheid en geen belastingen. Vandaag de dag
heeft het land daardoor een groeiende aantrekkingskracht als
belastingparadijs. Tot een generatie geleden stond Andorra echter bekend
als arm. Wat eens een dicht bos was, werd over de eeuwen heen ontbost
door de bewoners die warm wilden blijven in de strenge winters; elk jaar
ligt het van november tot april volledig onder de sneeuw. Zelfs in de
zomer is Andorra zo koud dat gewassen alleen op de zuidelijke hellingen
groeien. Deze beschrijving doet het misschien als onprettig overkomen,
maar dat is net het geheim van haar succes. Andorra overleefde als
feodale enclave in het tijdperk van de natiestaat omdat het afgelegen en
straatarm was.

Er waren ooit talrijke middeleeuwse grens- of \emph{march}-regio's waar
soevereiniteiten met elkaar vermengd raakten. Deze gewelddadige, vaak
arme grenzen bleven tientallen of soms honderden jaren bestaan in de
grensgebieden van Europa. Zoals eerder vermeld, waren er \emph{marches}
tussen gebieden onder Keltische en Engelse controle in Ierland, tussen
Wales en Engeland, Schotland en Engeland, Italië en Frankrijk, Frankrijk
en Spanje, Duitsland en de Slavische grenzen van Midden-Europa, en
tussen de christelijke koninkrijken van Spanje en het islamitische
koninkrijk Granada. Net als Andorra ontwikkelden deze
\emph{march}-regio's eigen, specifieke institutionele en juridische
vormen van een soort die we waarschijnlijk opnieuw zullen zien in het
volgende millennium.

Door de zwakke positie van de twee concurrerende autoriteiten, riepen
heersers soms zelfs vrijwilligers uit hun onderdanen op om zich in
\emph{march}-regio's te vestigen om zo hun invloed te vergroten. Bijna
vanzelfsprekend werden de onderdanen gelokt met vrijstelling van
belastingen. Aangezien ze opereerden op flinterdunne marges, zou het
invoeren van belastingen door een van de autoriteiten in een grensregio,
de eigen volgelingen financieel onder druk zetten en ze een reden geven
om zich met de tegenpartij te verbinden. Daarom hadden inwoners van een
\emph{march} meestal de keuze wiens wetten zij zouden gehoorzamen. Deze
keuze was gebaseerd op de zwakke positie van de concurrerende
autoriteiten; het was geen ideologisch gebaar.

Toch ontstonden er praktische moeilijkheden die moesten worden opgelost.
Onder het feodalisme werden landeigenaren met bezittingen aan beide
zijden van een nominale grens geconfronteerd met een ernstig conflict
wat betreft hun plichten. Bijvoorbeeld: een heer aan de grens van
Schotland en Engeland die eigendommen in beide koninkrijken bezat, kon
theoretisch militaire dienst aan beide kanten verschuldigd zijn in geval
van oorlog. Om deze tegenstrijdige verplichting op te lossen kon vrijwel
iedereen in de feodale hiërarchie kiezen wiens wetten hij gehoorzaamde
via een juridisch proces dat \emph{avowal} werd genoemd.

Door informatietechnologie zullen er ook mogelijkheden ontstaan om
economische activiteit te vestigen op concurrerende locaties, maar met
belangrijke verschillen. Een daarvan is dat de cyberspace naar
verwachting, anders dan de middeleeuwse grensregio's, uiteindelijk het
rijkste economische terrein zal vormen. Het zal dus een groeiend domein
zijn, in plaats van een die langzaam wegkwijnt. Zonder sterke, vaak
religieuze prikkels zouden maar weinig mensen uit de kerngebieden van de
middeleeuwse samenleving de wens hebben gehad om naar de grenzen te
verhuizen, omdat deze regio's doorgaans gewelddadig en arm waren. Ze
trokken dus geen middelen weg uit de macht van de autoriteiten. De
cyberspace zal dat wel doen.

Ten tweede zal het nieuwe grensgebied geen duopolie zijn, dat
samenwerking tussen twee autoriteiten uitlokt om tot een compromis te
komen over hun aanspraken erop. Zulke compromissen waren in de
middeleeuwen meestal niet effectief, deels vanwege de grote culturele
verschillen tussen de rivalen, en vooral omdat zij de fysieke macht
misten om een regeling af te dwingen. In het tijdperk van de natiestaat
verdwenen de meeste \emph{marches} en vage grenzen zodra nationale
autoriteiten wél over voldoende militaire macht beschikten om
oplossingen af te dwingen. Grensafbakening werd de norm, wat een
stabiele oplossing is wanneer een duopolie het gezag over het geweld in
twee aaneengesloten gebieden moet verdelen. Maar de concurrentie om
transacties te vestigen in de cybereconomie zal niet plaatsvinden tussen
twee autoriteiten, maar tussen honderden autoriteiten wereldwijd. Voor
territoriale staten zal het vrijwel onmogelijk zijn om de
belastingtarieven hoog te houden door het vormen van een kartel, net als
dat kartelvorming om monopoliewinsten te behouden niet werkt in markten
met honderden concurrenten.

De Seychellen, een klein land in de Indische Oceaan, bewijst dit punt.
Ze voerden een investeringswet in die door Amerikaanse functionarissen
werd beschreven als de wet ``Welkom Criminelen''. Volgens die wet krijgt
iedereen die \$10 miljoen investeert niet alleen bescherming tegen
uitlevering, maar ook een diplomatiek paspoort. In tegenstelling tot de
beweringen van de Amerikaanse regering waren de beoogde begunstigden
echter geen drugsdealers, die doorgaans toch al onder de bescherming van
belangrijkere overheden vallen, maar onafhankelijke ondernemers die
politiek incorrect waren geworden. De eerste potentiële begunstigde van
de Seychellen-wet was een blanke Zuid-Afrikaan die rijk werd door de
economische sancties tegen het voormalige apartheidsregime te omzeilen.
Nu loopt hij het risico op economische represailles van de nieuwe
Zuid-Afrikaanse regering en is hij bereid de Seychellen te betalen voor
bescherming.

Wat de merites van individuele gevallen ook zijn, het voorbeeld laat
zien waarom pogingen van overheden om een kartel voor bescherming te
vormen, gedoemd zijn te mislukken. Anders dan in de middeleeuwse
grensgebieden, waar de concurrentie slechts tussen twee autoriteiten
plaatsvond, zullen de grenzen in de cyberhandel tussen honderden
jurisdicties liggen, en waarschijnlijk zal het zelfs tot in de duizenden
oplopen.

In het tijdperk van de virtuele onderneming zullen individuen ervoor
kiezen hun inkomensgenererende activiteiten te vestigen in de
jurisdictie die de beste diensten levert tegen de laagste kosten. Met
andere woorden, soevereiniteit zal worden gecommercialiseerd. Anders dan
de meeste middeleeuwse \emph{march}-samenlevingen, die arm en
gewelddadig waren, zal de cyberspace dat niet zijn. De concurrentie
waarin informatietechnologie overheden dwingt zich te begeven, is geen
militaire concurrentie, maar concurrentie in de kwaliteit en prijs van
een economische dienst: echte bescherming. Kortom, overheden zullen hun
klanten moeten geven wat ze willen.

\subsection{De verminderde effectiviteit van
geweld}\label{de-verminderde-effectiviteit-van-geweld}

Dit wil natuurlijk niet zeggen dat overheden zullen ophouden geweld te
gebruiken. Integendeel, wij stellen dat geweld een groot deel van zijn
effectiviteit verliest. Een mogelijke reactie van overheden zou kunnen
zijn om hun gebruik van geweld in lokale contexten te intensiveren, als
poging om het verlies aan hun mondiale betekenis te compenseren. Wat
overheden ook doen, zij zullen de cyberspace echter niet kunnen
verzadigen met geweld zoals ze de gebieden die ze in de moderne wereld
monopoliseerden met geweld konden verzadigen. Hoeveel overheden ook
proberen de cyberspace te betreden, zij zullen daar niet capabeler of
machtiger zijn dan anderen.

Ironisch genoeg zouden pogingen van natiestaten om
``informatie-oorlogen'' te voeren om de toegang tot de cyberspace te
domineren of te blokkeren, waarschijnlijk alleen hun eigen ondergang
versnellen. De neiging tot de devolutie van grote systemen is al sterk
aanwezig door het wegvallen van schaalvoordelen en de stijgende kosten
om uiteenvallende sociale groepen bijeen te houden. De ironie van
informatie-oorlogen is dat ze mogelijk de broze systemen die uit het
Industriële Tijdperk zijn overgebleven meer schaden dan de opkomende
Informatiemaatschappij zelf.

Zolang essentiële informatietechnologie blijft functioneren, kan
cyberhandel parallel doorgaan met de strijd van een informatie-oorlog,
op een manier die in een territoriale oorlog onmogelijk zou zijn. Het is
ondenkbaar dat miljoenen commerciële transacties plaatsvinden aan het
front in een oorlog uit de twintigste eeuw. Virtuele oorlogen hoeven
echter de capaciteit van de cyberspace om meerdere activiteiten te
huisvesten niet uit te putten, en omdat de virtuele realiteit niet
fysiek bestaat, levert nabijheid tot het front nauwelijks gevaar op, en
is er al helemaal geen risico op exploderende virtuele granaatscherven.

\subsection{Kwetsbaarheid van grootschalige
systemen}\label{kwetsbaarheid-van-grootschalige-systemen}

In een informatie-oorlog lopen voornamelijk de grootschalige industriële
systemen die draaien op centraal bevel en controle gevaar. Militaire
autoriteiten in de Verenigde Staten en andere toonaangevende landen
maken zich zorgen over en bereiden zich voor op informatiesabotage,
waarbij grote systemen kunnen worden uitgeschakeld. Cyberaanvallen
kunnen bijvoorbeeld een telefoonwisselstation stilleggen, de
luchtverkeersleiding ontregelen of een pompsysteem dat de watertoevoer
naar een stad regelt saboteren. Een computervirus kan zelfs
conventionele of nucleaire energiebronnen doen uitvallen, waardoor delen
van het elektriciteitsnet tot stilstand komen. Zogenaamde \emph{logic
bombs} zijn in staat om enorme hoeveelheden informatie te vernietigen,
waaronder de uiterst gevoelige gegevens in de centrale controlesystemen
die de kwetsbare, grote industriële installaties aansturen. Tenzij men
daadwerkelijk alle informatietechnologie massaal vernietigt, wat de
wereldeconomie letterlijk tot stilstand zou brengen, zullen overheden er
niet in slagen om de internethandel en de virtuele realiteit te
onderdrukken of monopoliseren.

Zelfs een belangrijk nadeel van informatietechnologie, namelijk de
vermeende kwetsbaarheid voor het verval en de vernietiging van
data-opslagsystemen, wordt grotendeels door de recente
archiveringstechnologie opgelost. Een nieuw systeem, High-Density
Read-Only Memory (HD-ROM), maakt gebruik van een ionenfrees,
vergelijkbaar met die in computergestuurde productie, om archieven in
vacuüm te vervaardigen. De opslagcapaciteit bereikt maar liefst 10.000
megabytes per vierkante centimeter. Waar oudere systemen snel
achteruitgingen door bederf en schokken, verzekert HD-ROM dat de
opgeslagen data de volledige levensduur beschikbaar blijft. Eén van de
ontwikkelaars van HD-ROM, Bruce Lamartine, verklaart: `Het is vrijwel
ondoordringbaar voor de verwoestende invloeden van tijd, thermische en
mechanische schokken, en elektromagnetische velden die andere
opslagmedia beschadigen.' Zelfs een nucleaire explosie leidt niet per se
tot de verstoring of vernietiging van vitale informatie, zoals de codes
voor digitaal geld, waarop het soepele functioneren van een
cybereconomie steunt.

\begin{quote}
Moderne legers zijn zo afhankelijk van informatie dat je ze kunt
verblinden en doof maken, om zo een overwinning te behalen zonder op de
conventionele wijze te vechten. -- COL. ALAN CAMPEN, U.S.A.F (Gep.)
\end{quote}

\section{Supermachten in de virtuele
oorlogsvoering}\label{supermachten-in-de-virtuele-oorlogsvoering}

De veronderstellingen van de oorlogvoerende natiestaat zullen steeds
minder logisch zijn op megapolitiek vlak naarmate informatie
belangrijker wordt in de oorlogsvoering. Omdat de cyberspace geen
fysieke vorm heeft, zijn parameters zoals wij die in de fysieke wereld
kennen niet bepalend. Het maakt niet uit hoeveel programmeurs betrokken
waren bij het opstellen van een reeks commando's. Het enige dat telt is
of het programma functioneert. Het Soevereine Individu kan in de
cyberspace werkelijk net zoveel betekenen als een natiestaat met een
zetel in de VN, een eigen vlag en een leger op de grond. In puur
economische termen beschikken sommige Soevereine Individuen al over
investeerbare inkomens van honderden miljoenen per jaar, bedragen die
het discretionaire bestedingsvermogen van sommige failliete natiestaten
overtreffen. Maar dat is niet alles. Op het gebied van virtuele
oorlogsvoering via informatie kunnen sommige individuen net zo machtig
of machtiger zijn dan veel staten. Één bizar genie, die met digitale
helpers werkt, zou in theorie dezelfde impact in een cyberoorlog kunnen
hebben als een natiestaat. Bill Gates zou dat zeker kunnen.

In die zin is het tijdperk van het Soevereine Individu niet slechts een
slogan. Een hacker, of een klein groepje wiskundigen, laat staan een
bedrijf als Microsoft, of vrijwel elk softwarebedrijf, kan in principe
alles doen wat het Pentagon's Cyber War Task Force kan. Honderden
bedrijven in Silicon Valley en elders hebben al een grotere capaciteit
om een cyberoorlog te voeren dan 90 procent van de bestaande
natiestaten. De aanname dat overheden het leven op de grond zullen
blijven monopoliseren terwijl alternatieve beschermingsmiddelen overal
opkomen, is achterhaald. het is waarschijnlijker dat natiestaten opnieuw
ingericht moeten worden om hun kwetsbaarheid te verminderen voor
computervirussen, logische bommen, geïnfecteerde kabels en
trapdoor-programma's onder controle van de Amerikaanse National Security
Agency, of van een tienerhacker.

De megapolitieke logica van de cyberspace suggereert dat centrale
command-and-control-systemen die momenteel de wereldwijde infrastructuur
domineren, vervangen moeten worden door multicentrische
beveiligingsmodellen met gedistribueerde capaciteiten zodat ze niet
eenvoudig door een virus kunnen worden overgenomen of geblokkeerd.
Nieuwe soorten software, bekend als agoric open systems, zullen
command-and-control-software uit het Industriële Tijdperk vervangen. Die
oudere software verdeelde rekenkracht volgens rigide prioriteiten,
vergelijkbaar met hoe centrale planners bij Gosplan in de voormalige
Sovjet-Unie goederen aan wagons toewezen volgens vaste regels. De nieuwe
systemen worden bestuurd door algoritmen die marktmechanismen nabootsen
zodat middelen efficiënter toegewezen worden via een intern biedproces
dat de competitieve processen in de hersenen imiteert. In plaats van
gigantische computermonopolies die belangrijke
command-and-control-functies uitvoeren, zullen ze in het nieuwe
millennium gedecentraliseerd zijn.

Er is geen beter voorbeeld van de veerkracht van gedistribueerde
netwerken vergeleken met command-and-control-systemen dan dat van
Digital Equipment in het onderzoekscentrum in Palo Alto. Een ingenieur
opende een kast met het bedrijfseigen computernetwerk. Zoals Kevin Kelly
vertelt, ``trok de ingenieur dramatisch een kabel uit het netwerk,
waarop het rond de schade routeerde en geen moment haperde.''

Het Informatietijdperk zal niet alleen concurrentie zonder anarchie in
de cyberspace mogelijk maken, het zal onvermijdelijk leiden tot het
herontwerp van belangrijke systemen uit het industrialisme. Een
dergelijke herconfiguratie is essentieel om ze minder kwetsbaar te maken
voor schade van wie of waar dan ook. Net zoals het Industriële Tijdperk
onvermijdelijk leidde tot de herconfiguratie van overgebleven
middeleeuwse instituties zoals scholen en universiteiten, zullen de
overgebleven instituties van het Industriële Tijdperk waarschijnlijk in
miniatuurvorm evolueren, volgens de logica van de microtechnologie.

De noodzaak tot bescherming tegen bandieten op de Informatiesnelweg zal
de wijdverspreide toepassing van public key-private key-encryptie
vereisen. Deze maken het nu mogelijk voor elke gebruiker van een
personal computer om een bericht veiliger te versleutelen dan het
Pentagon haar lanceercodes een generatie geleden had kunnen beveiligen.
Deze krachtige, onbreekbare vormen van encryptie zullen nodig zijn om
financiële transacties tegen hackers en dieven te beschermen.

Ze zijn ook om een andere reden noodzakelijk. Particuliere financiële
instellingen en centrale banken zullen onbreekbare encryptie-algoritmen
toepassen zodra ze beseffen dat de Amerikaanse overheid, en mogelijk
niet alleen zij, in staat is om de huidige banksoftware en
computersystemen te penetreren en een land letterlijk failliet te laten
gaan of de bankrekening van bijna iedereen leeg te halen. Er is geen
technologische reden waarom een individu of land zijn financiële
tegoeden of transacties, legaal of illegaal, overgeleverd zou moeten
laten aan de Amerikaanse National Security Agency of opvolgers van de
KGB, of een soortgelijke organisatie.

Encryptie-algoritmen die door overheden niet te doorbreken zijn, zijn
geen droombeelden. Ze zijn al als \emph{shareware} via het Internet
beschikbaar. Wanneer satellietsystemen in een lage baan rond de aarde
volledig operationeel zijn, zullen individuen met geavanceerde personal
computers en antennes niet groter dan die van mobiele telefoons overal
ter wereld kunnen communiceren zonder het telefoonnetwerk te gebruiken.
Het zal even onmogelijk zijn voor een overheid om de cyberspace, een
niet-fysieke werkelijkheid, te monopoliseren als het voor middeleeuwse
ridders zou zijn geweest om de transacties in het Industriële Tijdperk
te paard te controleren.

\subsection{Bescherming door
onopvallendheid}\label{bescherming-door-onopvallendheid}

Informatiemaatschappijen zullen een enorme hoeveelheid middelen buiten
het bereik van rovers plaatsen. Wanneer de cyberspace steeds meer
financiële transacties en andere vormen van handel gaat huisvesten,
zullen de daar gebruikte middelen min of meer immuun zijn voor gewone
afpersing en diefstal. Rovers zullen daardoor niet in staat zijn om een
zo groot deel van de middelen te veroveren als ze vandaag de dag doen en
hebben gedaan gedurende het grootste deel van de twintigste eeuw.

Daarom zal overheidsbescherming van een groot deel van de wereldrijkdom
onnodig worden. De overheid zal niet beter in staat zijn om een
banktegoed in de cyberspace te beschermen dan uzelf. Omdat de overheid
minder noodzakelijk zal zijn, zal haar relatieve prijs waarschijnlijk
alleen al om die reden dalen, maar er zijn nog andere.

Nu een groot en groeiend aandeel van de financiële transacties in dit
nieuwe millennium in de cyberspace plaatsvindt, zullen individuen een
keuze hebben in welke jurisdictie zij hun transacties onderbrengen. Het
zal intense concurrentie creëren om de prijs van overheidsdiensten (de
heffingen die zij oplegt) op een niet-monopolistische wijze te bepalen.
Dit is revolutionair. Zoals George Melloan betoogde in \emph{The Wall
Street Journal}, `de instelling die veruit het meest succesvol de
krachten van wereldwijde concurrentie heeft weerstaan, is de
verzorgingsstaat.' Een studie door onderzoekers van de Wharton School en
de Australian National University besprak de krachten die van invloed
zijn op inkomensherverdeling. Garrett en Mitchell stelden vast dat er
`vrijwel geen aanwijzingen bestaan dat meer marktintegratie de
belangrijkste sociale vangnetten heeft verzwakt.' Integendeel, schrijven
zij, `overheden hebben altijd gereageerd op toenemende integratie in
internationale markten door inkomensherverdeling te vergroten.' De
opkomst van de cybereconomie zal eindelijk de verzorgingsstaat
blootstellen aan echte concurrentie. Het zal de aard van soevereiniteit
veranderen en economieën transformeren, omdat de balans tussen
bescherming en afpersing meer dan ooit in het voordeel van bescherming
zal verschuiven.

\bookmarksetup{startatroot}

\chapter{Het overstijgen van
plaatsgebondenheid}\label{het-overstijgen-van-plaatsgebondenheid}

::: \{.content-hidden when-format=``latex''\} \emph{De opkomst van de
cybereconomie} :::

\begin{quote}
Controle is waar het werkelijk om draait. Het Internet is zo
wijdverspreid dat geen enkele overheid het makkelijk kan beheersen. Door
een naadloze, wereldwijde, niet te reguleren economische zone te creëren
die losstaat van nationale soevereiniteit, stelt het Internet het idee
van een natiestaat ter discussie. -- John Perry Barlow
\end{quote}

De Informatiesnelweg is een van de bekendere metaforen uit de vroege
dagen van het digitale tijdperk geworden. Het is opmerkelijk, niet
alleen door de alomtegenwoordigheid, maar ook door het veel voorkomende
misverstand dat het blootlegt over de cybereconomie. Een snelweg is
immers een industriële versie van een voetpad, een netwerk voor het
fysieke transport van mensen en goederen. De Informatiemaatschappij is
niet zoals een snelweg, spoorlijn of pijpleiding. Het vervoert
informatie niet van punt naar punt zoals de Trans-Canada Highway zware
vrachtwagens van Alberta naar New Brunswick brengt. Wat de wereld de
``Informatiesnelweg'' noemt, is niet slechts een transportverbinding.
Het is de bestemming zelf.

De cyberspace overstijgt plaatsgebondenheid. Het is niets minder dan het
onmiddellijke delen van data overal en nergens tegelijk. De opkomende
Informatiemaatschappij is gebaseerd op de onderlinge verbindingen die
miljoenen gebruikers van miljoenen computers koppelen. De essentie ligt
in de nieuwe mogelijkheden die uit deze verbindingen ontstaan. Zoals
John Perry Barlow het formuleerde: ``Wat het Net biedt is de belofte van
een nieuwe sociale ruimte, globaal en anti-soeverein, waarin iedereen,
overal, aan de rest van de mensheid zonder angst kan delen wat hij of
zij gelooft. In deze nieuwe media is een voorbode te vinden van de
intellectuele en economische vrijheid die alle autoritaire machten op
aarde zou kunnen ontdoen.''

De cyberspace, net als het denkbeeldige rijk van de goden van Homerus,
is een andere wereld dan de bekende aardse wereld van boerderij en
fabriek. Toch zullen de gevolgen niet denkbeeldig, maar reëel zijn. In
veel grotere mate dan velen nu begrijpen, zal het onmiddellijke delen
van informatie als een oplosmiddel werken dat grote instellingen uiteen
doet vallen. Het zal niet alleen de logica van geweld veranderen, zoals
we al hebben besproken, maar zal de informatie- en transactiekosten, die
bepalen hoe bedrijven zijn georganiseerd en hoe de economie
functioneert, radicaal veranderen. We verwachten dat microprocessing de
economische organisatie van de wereld zal veranderen.

\begin{quote}
Het is vandaag, in grotere mate dan ooit in de geschiedenis van de
wereld, mogelijk voor een bedrijf om zich overal te vestigen, middelen
van overal te gebruiken om een product te produceren dat overal verkocht
kan worden. -- MILTON FRIEDMAN
\end{quote}

\section{De tirannie van de plaats}\label{de-tirannie-van-de-plaats}

Dat de eerste benadering van de Informatiemaatschappij in het vervagende
Industriële Tijdperk werd voorgesteld als een enorm publiek
infrastructuurproject, toont aan hoe sterk ons denken nog door
paradigma's van het verleden wordt bepaald. Het is vergelijkbaar met
boeren aan het einde van de achttiende eeuw die een fabriek omschrijven
als ``een boerderij met een dak.'' Toch is de metafoor van de
``snelweg'' onthullender. Ze toont ook hoezeer we nog gevangen zitten in
de tirannie van plaats. Zelfs wanneer technologie ons in staat stelt
locatie te overstijgen, krijgt het instrument van onze bevrijding een
bijnaam die het omschrijft als een route van plek naar plek. Het concept
van lokaliteit zit, net als het scherpe navigatievermogen van zalmen,
nog stevig verankerd in ons bewustzijn.

Gedurende de hele geschiedenis waren economieën tot nu toe gebonden aan
een lokale geografische regio. De meeste mensen die vóór de twintigste
eeuw leefden, brachten hun dagen in feite door als gevangenen onder
huisarrest, zelden verder dan een paar dagen lopen van hun
geboorteplaats. Een reis van welke afstand dan ook kostte generaties.
Alleen zelden leidde een crisis, oorlog, pest, of een ongunstige
klimaatverandering tot brede migratie. Het vereiste iets opmerkelijks en
urgents om de inwoners van een treurig dorp te laten vertrekken; alleen
dat kon hen motiveren hun spullen te verzamelen en elders geluk te
zoeken.

Tot voor kort werden degenen die buiten hun eigen regio naar kansen
keken, vaak beroemd. Denk aan Marco Polo, nog steeds bekend om zijn
reizen over het Euraziatische continent naar het hof van de Grote Khan.
Hij was een uitzondering in zijn tijd. Weinig andere reisverslagen uit
de premoderne periode zijn bewaard gebleven. Onder de bekendere werken
valt \emph{Mandeville's Travels} op, geschreven in het Frans in 1357,
omdat het is geschreven door iemand die waarschijnlijk nooit Europa
verliet. Mandeville beschrijft leuke en vaak fantasierijke details over
het leven wereldwijd, waaronder de suggestie dat veel Ethiopiërs maar
één voet hebben: ``{[}D{]}e voet is zo enorm dat het lichaam volledig in
de schaduw komt te liggen wanneer men rust.'' : Duidelijk wisten de
weinige tijdgenoten die zijn populaire verhaal lazen waarschijnlijk niet
dat zijn Ethiopische ``Bigfoot'' niet bestond.

Pas met het begin van het Moderne Tijdperk, aan het einde van de
vijftiende eeuw, ontstonden duurzame contacten tussen de continenten.
Onverschrokken kapiteins als Christopher Columbus en Vasco da Gama, die
de specerijenhandel wilden domineren, waren zo buitengewoon dat zij in
bijna elk geletterd huishouden vijf eeuwen lang werden herinnerd.

Vanaf het begin van de landbouw tot recente generaties werd het leven
gekenmerkt door immobiliteit. Dit is tegenwoordig vrijwel vergeten,
vooral in de Europese kolonies van de ``Nieuwe Wereld,'' waar beweging
vloeiender verloopt en iedereen zich vanuit het perspectief van een
immigrant oriënteert. Een thema in het lager onderwijs in Noord-Amerika
is dat de kolonisten uit Europa op zoek waren naar vrijheid en kansen,
wat waar is. Wat echter zelden wordt verteld, is hoe terughoudend de
meeste mensen waren om de reis te maken, zelfs terwijl ze thuis met
armoede te maken hadden. De weinigen die migreerden, ondergingen naar
huidige maatstaven onvoorstelbare ontberingen om zich daar te vestigen.
Alleen de meest ondernemende mensen of wanhopige armen kwamen. Midden in
de zeventiende eeuw kwamen gevangenen, in Londens beruchte gevangenis
Bridewell, in opstand om ``hun weigering om naar Virginia te gaan'' te
tonen. In 1720 waren er rellen in de straten van Parijs om landlopers,
dieven en moordenaars, bestemd voor deportatie naar Louisiana, te
bevrijden.

\subsection{Nauwe horizonten}\label{nauwe-horizonten}

Fysieke communicatie- en transportproblemen, vaak verergerd door
beperkte taalvaardigheden, hielden menselijke activiteiten weinig divers
en lokaal. Nog aan het begin van de twintigste eeuw was het gebruikelijk
dat Chinese dorpen slechts acht kilometer uit elkaar lagen en voor
elkaar onverstaanbare dialecten spraken, zelfs langs de kust. Bijna alle
economieën waren lokaal georganiseerd, wat leidde tot nauwe markten en
gemiste mogelijkheden. De prijzen van productiefactoren bleven hoog door
de geringe concurrentie. Toegang tot gespecialiseerde vaardigheden was
minimaal. Met inkomens zo laag dat ze aan de rand van armoede lagen, en
zonder toegang tot buitenlands kapitaal of efficiënte
verzekeringsmarkten, zaten kleine boeren in veel delen van de wereld
vast in armoede. We hebben enkele van de moeilijkheden verkend die aan
boeren werden opgelegd door het gesloten dorpsleven. Zelfs nu, terwijl
we dit schrijven, worstelt minstens een miljard mensen, voornamelijk in
Azië en Afrika, om te overleven van minder dan een dollar per dag.

\section{`Alle politiek is locaal'}\label{alle-politiek-is-locaal}

Veel meer dan men zich vaak realiseert, heeft de beperkte mobiliteit van
mensen en hun eigendommen onze kijk op de wereld gevormd. Zelfs degenen
die volhouden dat de aarde inmiddels klein is, denken nog steeds in
verouderde termen over industriële politiek. Dit illustreert een slogan
die in de jaren tachtig populair was onder milieubewuste mensen: `Denk
globaal maar handel lokaal.' De slogan weerspiegelt immers een politiek
die altijd draaide om lokale machtsvoordelen.

Lokale denkwijzen werden altijd bepaald door de megapolitiek van
vroegere samenlevingen. Alle topografische kenmerken die als obstakels
of hefboom voor machtsuitoefening fungeerden, zijn inherent lokaal. Elke
rivier, elke berg en elk eiland heeft zijn eigen betekenis op lokaal
niveau. Ook het klimaat kent lokale variaties: temperatuur, neerslag en
de groeicondities van gewassen veranderen naarmate je een berg beklimt
of oversteekt. Elk micro-organisme beweegt zich binnen een bepaalde
omgeving en niet zomaar overal.

Het verbaast dus niet dat de tirannie van de plaats ons denken over de
organisatie en werking van de samenleving doordringt. De machtsvoordelen
die sommige groepen behaalden met een lokaal geweldsmonopolie, hadden
altijd hun oorsprong ergens in de lokale context en vervaagden aan de
megapolitieke marge, waar de grenzen werden getrokken. Daarom heeft een
wereldregering nooit bestaan.

Hoewel men zelden expliciet benoemt hoe belangrijk locatie is voor het
uitoefenen van macht, merkten enkele voorstanders van een gedwongen
herverdeling al in de jaren dertig dat de invloed van plaats afnam. Zij
herkenden in het moderne vervoer een scheiding van sociale ruimtes
tussen hoge en lage inkomensgroepen. John Dos Passos verwoordde deze
zorg treffend in \emph{The Big Money}: `De zwerver zit aan de rand van
de snelweg, gebroken en hongerig. Boven hem vliegt een transcontinentaal
vliegtuig vol hoogbetaalde leidinggevenden. De hogere klasse heeft zich
in de lucht begeven, de lagere klasse op de weg: er is geen band meer
tussen hen, het zijn twee naties.' Hiermee geeft hij aan dat de
verbeterde vervoersmogelijkheden de effectiviteit van afpersing
verminderden, doordat succesvolle mensen simpelweg meer mogelijkheden
hadden om te kiezen waar ze wilden zijn. De zwerver op de weg had in
ieder geval geen enkel middel om steun af te dwingen van zij die boven
hem vlogen. De tendensen die Dos Passos zestig jaar geleden opmerkte,
zijn sindsdien alleen maar sterker geworden.

\subsection{Massatransport}\label{massatransport}

In 1995 staken dagelijks ongeveer een miljoen mensen ergens in de wereld
een grens over, een opmerkelijke verandering vergeleken met vroeger.
Vóór de twintigste eeuw was reizen zo zeldzaam dat grenzen vooral als
randgebieden werden gezien en nauwelijks een belemmering vormden voor
doorgang. Paspoorten bestonden nog niet. De opkomst van grote zeeboten,
treinen en andere verbeterde vervoermiddelen leidde tot een
spectaculaire toename van het aantal verplaatsingen. Tegelijkertijd
werden deze verplaatsingen steeds strenger gereguleerd door staten, wier
macht was toegenomen dankzij dezelfde verbeterde transport- en
communicatiemiddelen die het reizen voor burgers goedkoper en
eenvoudiger maakte. Films en, vooral, televisie speelden daarnaast een
belangrijke rol in het verbreden van de horizonten en het stimuleren van
reizen en immigratie. Desondanks bleven de fundamentele beginselen van
de sociale en economische organisatie tot nu toe verankerd in de lokale
context.

\begin{quote}
`\ldots om te voorkomen dat we ons lef verliezen, waarvoor de
geschiedenis zo meedogenloos straft. We moeten de moed hebben om alle
technische extrapolaties tot hun logische conclusie te volgen.' - ARTHUR
C. CLARKE
\end{quote}

\section{De fout van minimale
verwachtingen}\label{de-fout-van-minimale-verwachtingen}

De geografische greep op de verbeelding blijft nog altijd zo sterk dat
een aantal experts, die in 1995 het Internet onder de loep namen,
concludeerden dat het weinig commercieel potentieel bezit en vrijwel
geen betekenis heeft, behalve als medium voor chat en pornografie. De
vele sceptici over het economische belang van de cyberspace vormen de
Colonel Blimps van het Informatietijdperk. Hun zelfgenoegzaamheid is
vergelijkbaar met die van de Britse elite in de jaren 1930,
geconfronteerd met de neergang van het rijk. Elites reageren steevast
met ontkenning zodra hun positie in gevaar komt. Dit blijkt uit hun
verwachting dat het Internet nooit meer dan een bijzaak zal blijven, een
visie die soms zelfs door autoriteiten wordt gedeeld die beter hadden
moeten weten. We verwezen eerder naar het werk van David Kline en Daniel
Burstein, \emph{Road Warriors: Dreams and Nightmares Along the
Information Highway}. Hun ontkenning van het economische potentieel van
het Internet levert extra bewijs dat technische onderlegdheid niet
synoniem staat met het doorgronden van technologische gevolgen.

Zelfs de technisch meest deskundige waarnemers hebben in het verleden
vaak de implicaties van nieuwe technologieën niet begrepen. Een Brits
parlementair comité, bijeengekomen in 1878 om de vooruitzichten van
Thomas Edison's gloeilamp te onderzoeken, beoordeelde Edison's ideeën
als ``goed genoeg voor onze trans-Atlantische vrienden, \ldots{} maar
onwaardig voor de aandacht van praktische of wetenschappelijke mensen.''
: Thomas Edison zelf was een man van grote visie, maar hij dacht dat de
fonograaf die hij uitvond voornamelijk door zakenmensen zou worden
gebruikt, om te dicteren. Kort voordat de gebroeders Wright bewezen dat
vliegtuigen konden vliegen, demonstreerde de vooraanstaande Amerikaanse
astronoom Simon Newcomb met gezag waarom vlucht voor objecten zwaarder
dan lucht onmogelijk was. Hij concludeerde: ``De demonstratie dat geen
enkele mogelijke combinatie van bekende stoffen, bekende machinetypes en
bekende krachten kan worden samengebracht in een praktische machine
waarmee mensen lange afstanden door de lucht kunnen vliegen, lijkt voor
de schrijver net zo sluitend te zijn als de demonstratie van eender welk
fysisch gegeven kan zijn.'' Kort nadat vliegtuigen gingen vliegen, legde
een ander gerenommeerd astronoom, William H. Pickering, uit aan het
publiek waarom commercieel reizen nooit van de grond zou komen: ``De
gemiddelde mens stelt zich vaak gigantische vliegmachines voor die over
de Atlantische Oceaan razen en talloze passagiers vervoeren, op een
manier analoog aan onze moderne stoomschepen. \ldots{} {[}H{]}et is
duidelijk dat met onze huidige apparaten geen hoop bestaat om qua
snelheid te concurreren met onze locomotieven of onze auto's.'' Eerder
herinnerden we ons een andere totaal onjuiste voorspelling over de
potentie van een nieuwe technologie: de voorspelling aan het begin van
de twintigste eeuw door de makers van Mercedes dat er wereldwijd nooit
meer dan een miljoen auto's zouden zijn. Ook hier wisten zij meer over
auto's dan bijna wie dan ook, maar ze konden niet verder van de waarheid
zitten wat betreft de impact van auto's op de samenleving.

Gezien deze traditie van misvattingen is het nauwelijks verrassend dat
veel waarnemers de belangrijkste implicaties van de nieuwe
informatietechnologie pas laat zullen begrijpen, namelijk het feit dat
zij de tirannie van plaats overstijgt. De nieuwe technologie creëert
voor het eerst een oneindig, niet-aards domein voor economische
activiteit. Het biedt de mogelijkheid om de nieuwe grenzen van de
cybereconomie te verkennen, om ``globaal te denken en globaal te
handelen.'' Dit hoofdstuk legt uit waarom.

\section{Voorbij plaatsgebondenheid}\label{voorbij-plaatsgebondenheid}

Het verwerken en gebruiken van informatie vervangt en wijzigt snel
fysieke producten als belangrijkste bron van winst. Dit heeft
ingrijpende gevolgen. Informatietechnologie scheidt het vermogen om
inkomen te genereren van een specifieke geografische locatie. Aangezien
een steeds groter deel van de waarde van producten en diensten wordt
gecreëerd door ideeën en kennis toe te voegen, zal een steeds kleiner
deel van de toegevoegde waarde onder lokale jurisdicties vallen. Ideeën
kunnen overal worden bedacht en wereldwijd met de snelheid van het licht
worden verspreid. Dit betekent onvermijdelijk dat de
Informatiemaatschappij drastisch zal verschillen van de economie van het
Fabriekstijdperk.

We geven toe aan de critici dat een opsomming van taken die je in 1996
via het Internet had kunnen uitvoeren, misschien banaal lijkt. Er is
immers niets revolutionairs aan het lezen van een artikel over tuinieren
op het Net, of het op afstand kopen van een doos wijn. De potentie van
de cybereconomie kan echter niet uitsluitend worden beoordeeld aan de
hand van het prille begin, net zo min als dat de impact van de auto op
de samenleving in 1900 kon worden ingeschat op basis van wat men toen om
zich heen zag. Wij verwachten dat de cybereconomie zich in meerdere
stadia zal ontwikkelen.

\begin{enumerate}
\def\labelenumi{\arabic{enumi}.}
\tightlist
\item
  De meest primitieve verschijningsvormen van het Informatietijdperk
  gebruiken het Net eenvoudig als informatiedrager om gewone transacties
  uit het Industriële Tijdperk te vergemakkelijken. Op dit punt is het
  Net niet meer dan een exotisch bezorgsysteem voor catalogi. Virtual
  Vineyards, bijvoorbeeld, een van de eerste cyberhandelaren, verkoopt
  simpelweg wijn via een pagina op het World Wide Web. Dergelijke
  transacties ondermijnen de oude instituties nog niet direct. Ze
  gebruiken industriële valuta en vinden plaats binnen identificeerbare
  jurisdicties. Dit gebruik van het Internet heeft weinig megapolitieke
  impact.
\item
  Een tussenstadium van internethandel zal informatietechnologie
  gebruiken op manieren die in het Industriële Tijdperk onmogelijk
  zouden zijn geweest, zoals bij langeafstandsboekhouding of medische
  diagnose. Meer voorbeelden van deze nieuwe toepassingen van
  geavanceerde rekenkracht worden hieronder toegelicht. Het tweede
  stadium van Internethandel zal nog steeds functioneren binnen het oude
  institutionele kader, met gebruik van nationale valuta en onderworpen
  aan de jurisdictie van natiestaten. De handelaren die het Net
  gebruiken voor hun verkoop, zullen het nog niet gebruiken om hun winst
  veilig te stellen, maar alleen om inkomsten te genereren. Deze winsten
  uit internettransacties zullen nog steeds aan belastingheffing
  onderworpen zijn.
\item
  Een meer gevorderd stadium zal de overgang naar echte cyberhandel
  markeren. Transacties zullen niet alleen via het Net plaatsvinden,
  maar ook buiten de jurisdictie van natiestaten migreren. Betalingen
  zullen plaatsvinden in cybervaluta, winst zal worden geboekt in
  cyberbanken, investeringen zullen worden gedaan via cybermakelaars,
  veel transacties zullen niet aan belastingheffing onderhevig zijn. In
  dit stadium zal cyberhandel aanzienlijke megapolitieke gevolgen
  krijgen, zoals we eerder hebben geschetst. De macht van overheden over
  traditionele delen van de economie zal worden getransformeerd door de
  nieuwe logica van het Net. Extraterritoriale regelgevende macht zal
  instorten, jurisdicties zullen verzwakken, de structuur van bedrijven
  zal veranderen, evenals de aard van werk en arbeid. Deze schets van de
  stadia van de Informatierevolutie is slechts een summiere weergave van
  wat de meest ingrijpende economische transformatie ooit zou kunnen
  worden.
\end{enumerate}

\section{De globalisering van handel}\label{de-globalisering-van-handel}

In het Informatietijdperk zullen technologische ontwikkelingen de meeste
traditionele jurisdictievoordelen snel tenietdoen. Tegelijkertijd
ontstaan er nieuwe voordelen. Lagere communicatiekosten hebben de
noodzaak om fysiek aanwezig te zijn voor het doen van zaken al sterk
verminderd. In 1946 kon een investeerder in Londen via een makelaar in
New York een order plaatsen, maar alleen de grootste en meest
overtuigende transacties rechtvaardigden dat: een drie minuten durend
telefoongesprek tussen New York en Londen kostte toen \$650.
Tegenwoordig betaal je daar slechts \$0,91 voor. In een halve eeuw is de
prijs van een intercontinentaal telefoongesprek met meer dan 99 procent
gedaald.

\subsection{Convergente communicatie}\label{convergente-communicatie}

Binnenkort merk je nauwelijks verschil tussen intercontinentale chat en
een lokaal telefoongesprek. Ook vervagen de verschillen tussen je
telefoon, computer en televisie steeds meer; je onderscheidt ze immers
vooral op basis van ergonomie in plaats van functionaliteit. Met je
persoonlijke computer voer je spraakgesprekken via het Internet, door
gebruik te maken van de ingebouwde microfoon en luidsprekers, en bekijk
je films. Daarnaast kun je met je televisie communiceren en grote
hoeveelheden data uitwisselen via netwerken die door de
televisie-entertainmentmedia worden aangeboden. Naarmate het onderscheid
tussen de verschillende vormen van communicatie in het Industriële
Tijdperk verdwijnt en de kosten kelderen, zullen steeds meer diensten
worden aangerekend op basis van de gebruiksduur in plaats van de
bestemming van je berichten. Al met al betaal je straks voor gesprekken
en datatransmissies wereldwijd nauwelijks meer dan wat je in 1985 voor
een lokaal telefoongesprek betaalde.

\subsection{Draadloos Internet}\label{draadloos-internet}

Satellieten in een lage baan rond de aarde en andere vormen van
draadloze technologie zullen gegevens rechtstreeks heen en weer zenden
naar een pieper in je zak, een draagbare computer of een werkstation,
zonder enige aansluiting op een lokaal telefoon- of tv-kabelsysteem.
Kortom, het Internet zal draadloos worden. De eerste stappen in die
richting zullen waarschijnlijk aarzelend zijn vanwege de relatief lage
datasnelheid van de vroege draadloze media en de moeilijkheden om zwakke
signalen van gebruiksapparaten te ``horen'', waarvan sommige mobiel
zullen zijn en op batterijen zullen werken. Desalniettemin zullen deze
technische problemen worden aangepakt en opgelost.

\subsection{Zakendoen zonder grenzen}\label{zakendoen-zonder-grenzen}

De voortdurende toename van de rekencapaciteit leidt tot geavanceerdere
compressietechnieken, wat de doorstroming van data versnelt. Door
bestaande algoritmen voor encryptie met publieke en private sleutels op
grote schaal toe te passen, kunnen aanbieders, zoals satellietsystemen,
de facturering naadloos in hun dienst integreren en zo kosten besparen.
Tegelijkertijd krijgen leveranciers de mogelijkheid om rekeningen die op
pc's zijn geladen, direct te belasten, net zoals \emph{France Telecom}
de `smartcards' in de telefoonhokjes in Parijs debiteert.

\subsection{De telefoon wordt een
bank}\label{de-telefoon-wordt-een-bank}

Het verschil is dat je in de nabije toekomst credits op je account kunt
verdienen met allerlei transacties en je telefoon overal mee naartoe
kunt nemen. Je pc zal het filiaal van je bank en mondiale geldmakelaar
zijn, en het equivalent van de kiosk in Parijs waar je je anonieme
telefoonkaart koopt. En net als de 150 smartcardtelefoons die voor
dieven nutteloos zijn als ze met een koevoet worden opengebroken, kan je
computer alleen worden geplunderd door iemand die in staat is
geavanceerde computercode te breken of te manipuleren. Dat sluit veel
tuig, dat wel met een koevoet om kan gaan, uit. Met de juiste encryptie
kan niets in je computer worden ontcijferd of misbruikt.

Tegen de eeuwwisseling zul je bijna overal ten noorden van Antarctica
zaken kunnen doen. Overal waar vaste of digitale mobiele telefoons
beschikbaar zijn, overal waar interactieve kabeltelevisiesystemen worden
gebruikt, overal waar een satelliet zich boven je bevindt of andere
draadloze transmissiesystemen aanwezig zijn. Je zult over grenzen heen,
wanneer je wilt, kunnen spreken, gegevens verzenden en reizen via
virtual reality. Telefoon­nummers die de locatie van de beller aangeven
via netnummers zullen waarschijnlijk worden vervangen door universele
toegangsnummers, die overal op de planeet de persoon zullen bereiken met
wie je wilt communiceren.

\subsection{Chinees begrijpen}\label{chinees-begrijpen}

Je zult niet alleen kunnen praten en faxen. Na verloop van tijd zul je
het jarenlange leertraject kunnen overslaan, waardoor je in het Chinees
kunt converseren met een voorman in een fabriek in Shanghai. Het maakt
dan nog weinig uit dat je zijn taal of dialect niet spreekt. Hoewel hij
in het Chinees communiceert, zullen zijn woorden de `plaatsgebondenheid
overstijgen'. Jij hoort zijn woorden in je eigen taal en hij hoort het
gesprek in het Chinees. Spoedig zal je vermogen tot onmiddellijke
vertaling de concurrentiekracht verhogen in regio's waar taal- en
uitdrukkingsbarrières voorheen een struikelblok vormden. Op dat moment
maakt het nauwelijks of helemaal niet uit dat de Chinese regering
misschien bezwaar heeft tegen het gesprek.

\subsection{Gepersonaliseerde media}\label{gepersonaliseerde-media}

Naarmate de wereld steeds dichter bij elkaar komt, zul je meer
mogelijkheden dan ooit krijgen om je eigen positie vorm te geven. Ook de
informatie die je via de media binnen zult krijgen, zul je zelf kiezen.
De traditionele massamedia zullen plaats maken voor gepersonaliseerde
media. Ben je een fanatieke schaker of een fervent kattenliefhebber? Dan
kun je jouw avondnieuws zo inrichten dat het uitsluitend nieuws bevat
over jouw favoriete onderwerpen. je hoeft voor de nieuwsvoorziening niet
langer afhankelijk te zijn van Dan Rather of van de \emph{BBC}. Je
selecteert zelf het nieuws dat volledig is afgestemd op jouw wensen.

\subsection{Van massa naar op maat gemaakte
productie}\label{van-massa-naar-op-maat-gemaakte-productie}

Als het komkommertijd is, kun je een virtuele catalogus raadplegen op
het \emph{World Wide Web}. Als je een broek ziet die je bijna bevalt,
kun je bij je bestelling de breedte van de pijpen aanpassen. De broek
wordt dan op maat gesneden en door robots in Maleisië nauwkeurig
afgestemd op je lichaam, op basis van foto's die je via je computer
scant en over het Internet verstuurt.

\subsection{Cyberbroking}\label{cyberbroking}

Je kunt cybergeld gebruiken om te investeren en voor diensten en
producten te betalen. Als je in een rechtsgebied woont zoals de
Verenigde Staten, waar investeringsmogelijkheden streng gereguleerd
zijn, kies je er bewust voor je activiteiten onder te brengen in een
omgeving die volop vrijheid biedt op vlak van investeringsmogelijkheden.
Of je nu in Cleveland of in Belo Horizonte woont, je kunt je
investeringszaken regelen in Bermuda, op de Kaaimaneilanden, in Rio de
Janeiro of in Buenos Aires. Waar je ook bent, digitale middelen zullen
steeds meer gebruikt worden naarmate de cybereconomie floreert. Je zult
slimme systemen in kunnen zetten om je investeringen te selecteren en
zult cyberaccountants en -boekhouders in kunnen schakelen om de
voortgang van je portefeuille in realtime te volgen.

\subsection{Virtuele cultuur}\label{virtuele-cultuur}

Wanneer je even niet bezig bent met winst- en verliescijfers, kun je een
virtueel bezoek brengen aan het Louvre. Voordat je op pad gaat, moet je
mogelijk een royalty betalen ter waarde van een derde van een cent aan
Bill Gates of aan iemand met een vergelijkbare vooruitziende blik die de
rechten op virtuele realiteit voor museumbezoeken heeft verworven.
Terwijl je je afvraagt of de Mona Lisa ooit problemen met haar tanden
had, downloadt je computer ondertussen S. I. Hsiung's vertaling van
\emph{The Romance of the Western Chamber}. Op het moment dat jij dat
wilt, leest je persoonlijke communicatiesysteem de tekst voor, als een
bard uit weleer. Dankzij multitaskingprogramma's kun je meerdere
functies gelijktijdig uitvoeren.

\subsection{Shoppen voor rechtsgebieden op het
Net}\label{shoppen-voor-rechtsgebieden-op-het-net}

Als je geïnspireerd bent door de klassiekers, kun je een virtueel
bedrijf oprichten om dramatische producties van beroemde literatuur te
verkopen om op driedimensionale retinale displays weer te geven. In
plaats van geprojecteerd in de lucht, worden de beelden direct op het
netvlies van kijkers geprojecteerd met laag-energetische lasers die
vijftigduizend keer per seconde fluctueren. Deze technologie, al in
ontwikkeling bij MicroVision in Seattle, Washington, zal veel
slechtzienden weer kunnen laten zien.

Voordat je het project onderneemt, kun je jouw digitale assistent
instrueren om de huidige contractaanbiedingen voor de bescherming van
productiefaciliteiten in Maleisië, China, Peru, Brazilië en Tsjechië te
inventariseren. Zodra je een locatie kiest, kun je je bedrijf binnen één
uur laten oprichten op de Bahama's, via de St.~George's Trust Company.
Je instructies plaatsen alle liquide activa van het bedrijf in een
cyberaccount bij een cyberbank die gelijktijdig is gevestigd in
Newfoundland, de Kaaimaneilanden, Uruguay, Argentinië en Liechtenstein.
Als een van deze rechtsgebieden probeert om de operationele bevoegdheid
in te trekken of de activa van rekeninghouders in beslag te nemen,
worden de activa automatisch overgebracht naar een andere jurisdictie
met de snelheid van het licht.

\section{Kwalitatieve vooruitgang}\label{kwalitatieve-vooruitgang}

Veel transacties die je binnenkort in de cyberspace kunt uitvoeren,
waren in het Industriële Tijdperk ondenkbaar, en niet alleen doordat ze
een taalbarrière overschrijden. Het inzetten van digitale assistenten om
onvertaalde artikelen uit Hongaarse wetenschappelijke tijdschriften te
verzamelen, onderscheidt zich qua kwaliteit van een gesprek met een
bibliothecaris. Het deelnemen aan een Oxford-tutorial op een afstand van
achtduizend kilometer is niet te vergelijken met het volgen van
diezelfde tutorial terwijl je binnen tien kilometer van Carfax slaapt.
En roulette spelen in het \emph{Hotel de Paris} in Monte Carlo biedt een
totaal andere ervaring wanneer je dit via \emph{virtual reality} vanuit
een feest in Punte del Este, Uruguay beleeft.

\subsection{Een cyberbezoek aan de
cyberdokter}\label{een-cyberbezoek-aan-de-cyberdokter}

Binnen korte tijd, misschien wel sneller dan veel deskundigen
verwachten, zal de economische activiteit naar de cybereconomie
migreren. Hierbij combineer je technologieën op vernieuwende wijze om de
beperkingen van locatiegebondenheid en de achterhaalde instituties van
de industriële economie te doorbreken. Binnenkort zul je wanneer je
buikpijn krijgt een digitale dokter raadplegen, een digitale expert met
encyclopedisch inzicht in symptomen, kwalen en tegengif. Dit systeem
doorzoekt, in versleutelde vorm, je medische geschiedenis en vraagt of
je pijn ervaart na of vóór de maaltijd, of de pijn scherp of dof,
constant of sporadisch is. De digitale dokter stelt alle vragen die een
arts zou stellen. Hij kan daarbij vaststellen dat je te veel of juist te
weinig wijn drinkt en je eventueel doorverwijzen naar een
cyberspecialist. Heb je een operatie nodig, dan verricht een
cyberschirurg in Bermuda de ingreep op afstand met behulp van
gespecialiseerde apparatuur die micro-incisies maakt.

\subsection{Leven-en-dood
informatieverwerking}\label{leven-en-dood-informatieverwerking}

Dit klinkt misschien als sciencefiction, maar veel componenten van
cyberchirurgie zijn al aanwezig. Andere zullen operationeel zijn tegen
de tijd dat je dit boek leest. General Electric heeft een nieuwe
magnetic-resonance-treatment-machine (MRT) geïntroduceerd in vijftien
ziekenhuizen wereldwijd. De machine zal een onderzoeks- en
ontwikkelingsfase van drie jaar doorlopen, maar daarna zal ze
waarschijnlijk snel verspreid raken en de norm worden voor veel soorten
chirurgie. Dit is één voorbeeld, maar een goed voorbeeld, van hoe
technologie de samenleving verandert.

De meesten van ons zijn bekend met magnetic-resonance-imaging-apparaten
(MRI), waarbij magnetische resonantietechnieken worden gebruikt om
artsen beelden van zachte weefsels te verschaffen voor diagnostische
doeleinden. Ze leveren betere beelden van zachte weefsels dan
röntgenstraling of echografie en zijn een essentieel onderdeel geworden
van moderne diagnostische technieken, met name bij kanker. Ze hebben
echter momenteel twee belangrijke beperkingen: de buis biedt geen vrije
toegang tot de patiënt en de machines hebben een beperkt vermogen.

\subsection{Cyberchirurgie}\label{cyberchirurgie}

\emph{General Electric} heeft de magnetische resonantiemachines zo
aangepast dat ze zowel voor diagnostiek als voor behandelingen ingezet
kunnen worden. Ze hebben de kracht van de machines met een factor vijf
verhoogd en de buis in twee helften verdeeld, waardoor de patiënt niet
langer volledig wordt omsloten, maar tussen twee donutvormige
compartimenten komt te liggen. In plaats van eerst een beeld vast te
leggen waarop later de operatie wordt gebaseerd, ziet de chirurg direct
wat hij doet tijdens de ingreep. Het systeem koppelt de MRT aan
microchirurgische technieken die minder invasief zijn. De chirurg hoeft
geen grote sneden met een scalpel te maken, maar zet kleine incisies met
sonderingsinstrumenten, waarbij hij in real‑time observeert wat deze
onthullen. Hij voert de operatie uit op basis van het beeld, in plaats
van er met eigen ogen in te kijken. Bovendien kan hij de instrumenten in
principe op afstand bedienen. Zo kunnen tumoren met uiterste precisie
worden vernietigd, bijvoorbeeld met behulp van laserapparatuur of via
cryogene warmte‑ en vriesbehandelingen.

Hiermee worden operaties mogelijk die tot nu toe onmogelijk leken,
vooral in de neurochirurgie, waar tumoren zich vaak zeer dicht bij
vitale hersengebieden bevinden. Ook is het met deze technologie mogelijk
om operaties meermaals uit te voeren, terwijl het trauma van de
traditionele ingreep niet herhaald kan worden zonder onaanvaardbare
schade. Sommige onderzoekers zijn van mening dat het mes voor chirurgie
aan zacht weefsel tegen 2010 een verouderd relikwie zal zijn. Als dat
standhoudt, verminderen zowel de angst als de naschokken die bij
traditionele operaties horen. Uiteraard is dit uitstekend nieuws voor de
patiënt. Terwijl operaties tegenwoordig uren duren en gevolgd worden
door dagen of weken ziekenhuisopname, kan de ingreep in slechts een half
uur worden voltooid en is een opname mogelijk overbodig. In feite is het
zelfs mogelijk dat de chirurg en de patiënt nooit in dezelfde ruimte
aanwezig zijn. Maar wat betekent dit voor ziekenhuizen en chirurgen?

\subsection{Minder microchirurgen die meer operaties
uitvoeren}\label{minder-microchirurgen-die-meer-operaties-uitvoeren}

Er zal een revolutie plaatsvinden in de chirurgie. Een derde van de
jonge chirurgen lukt het tijdens de opleiding niet om de vaardigheden
voor microscopische chirurgie te verwerven. Een derde kan het net
uitvoeren, en een derde wordt uitstekend. Vergelijkbare verhoudingen
worden gevonden in omscholingen voor oudere chirurgen. Minder chirurgen
zullen in staat zijn om meer operaties in kortere tijd uit te voeren.
Verzekeraars en mensen die een operatie ondergaan, zullen waarschijnlijk
resultaten per chirurg willen zien, die sterk uiteenlopen. Patiënten
zullen naar chirurgen willen gaan die de beste resultaten leveren,
vooral als hun aandoeningen levensbedreigend zijn. In sommige gevallen
kunnen de beste chirurgen operaties op afstand uitvoeren. Ze kunnen de
hele operatie uitvoeren vanuit een andere jurisdictie waar belastingen
lager zijn en rechtbanken exorbitante schadeclaims niet erkennen.

\subsection{Digitale juristen}\label{digitale-juristen}

Voordat een ervaren chirurg akkoord gaat met een operatie, schakelt hij
of zij waarschijnlijk een digitale jurist in om onmiddellijk een
contract op te stellen. Dit contract specificeert en beperkt de
aansprakelijkheid op basis van de grootte en kenmerken van de tumor,
zoals die zichtbaar zijn in de beelden van de magnetische
resonantiemachine. Digitale juristen zijn informatieverwerkende systemen
die met kunstmatige intelligentie, waaronder neurale netwerken,
contracten automatisch aanpassen zodat ze voldoen aan de transnationale
wetgeving. Deelnemers aan belangrijke en waardevolle transacties zoeken
niet alleen geschikte zakenpartners, maar kiezen ook een passend
vestigingsadres voor hun transacties.

\subsection{Spoedconsultatie}\label{spoedconsultatie}

Om het voorbeeld van cybersurgery voort te zetten: de technologie van
het Informatietijdperk zal een premie leggen op de hoogste vaardigheden
in de chirurgie, zoals in bijna elk ander vakgebied. Patiënten waren al
bereid om zo'n premie te betalen sinds messen werden uitgevonden. Maar
beperkingen in informatie en de moeilijkheid om chirurgen in
noodsituaties in een bepaalde regio te vinden, maakten de markt voor
chirurgie behoorlijk imperfect. In het Informatietijdperk zal deze
minder imperfect zijn. Een patiënt die binnen vierentwintig uur, of
misschien zelfs binnen vijfenveertig minuten, een operatie nodig heeft,
zou digitale assistenten kunnen inzetten om de tien beste chirurgen
wereldwijd te vinden die beschikbaar zijn voor een operatie op afstand,
om hun slagingspercentages in vergelijkbare gevallen te beoordelen, en
om offertes voor het specifieke geval op te vragen bij hun digitale
vertegenwoordigers. Dit alles kan in een oogwenk worden uitgevoerd. Als
gevolg hiervan zal de meest gewilde 10 procent van de chirurgen een veel
groter wereldwijd marktaandeel hebben in de chirurgie. De MRT-machine,
plus microchirurgische technieken, zal de premie voor hun werk verhogen.
Chirurgen met minder vaardigheden zullen zich richten op de overgebleven
lokale markten.

Dit leven-en-doodvoorbeeld illustreert enkele revolutionaire gevolgen
van de bevrijding van economieën uit de tirannie van plaats. Sommigen
zullen misschien aanvoeren dat de MRT-machine van General Electric niet
bedoeld was voor gebruik op afstand. Misschien, maar dat mist het punt.
Deze of soortgelijke apparatuur zal dat binnenkort wel zijn. Wanneer
operaties beter kunnen worden uitgevoerd door chirurgen die naar een
scherm kijken dan direct naar de patiënt, zal het minder uitmaken waar
de chirurg en zijn scherm zich bevinden. Een toenemend aantal diensten
zal worden heringericht om te profiteren van het feit dat
informatietechnologie mensen overal ter wereld in staat stelt te
handelen, zelfs in zulke delicate zaken als chirurgie. Bij activiteiten
die minder precieze apparatuur vereisen en lagere faalkansen kennen, zal
de cybereconomie nog sneller floreren.

\begin{quote}
Het financiële beleid van de verzorgingsstaat vereist dat er geen manier
is voor vermogenden om zichzelf te beschermen. -- ALAN GREENSPAN
\end{quote}

\section{De devaluatie van dwang}\label{de-devaluatie-van-dwang}

In bijna elk competitief domein, inclusief het merendeel van de
wereldwijde investeringen ter waarde van biljoenen dollars, zal de
verhuizing van transacties naar de cyberspace worden aangedreven door
een bijna onstuitbare kracht, de drang om roofbelasting te vermijden,
waaronder de inflatiebelasting die iedereen die zijn vermogen in een
nationale munt aanhoudt, treft.

\subsection{Ontsnappen aan het
beschermingsmaffia}\label{ontsnappen-aan-het-beschermingsmaffia}

Je hoeft niet lang na te denken over de megapolitiek van het
Informatietijdperk om te beseffen dat roofbelastingen en inflatie, zoals
die door de rijkste industriële landen als recht aan hun burgers worden
opgelegd, volstrekt onconcurrerend zullen zijn in de nieuwe wereld van
de cyberspace. Kort na de eeuwwisseling zal iedereen die
inkomstenbelasting betaalt tegen de huidige tarieven dat voor 50 procent
vrijwillig doen. Zoals Frederic C. Lane opmerkte, laat de geschiedenis
zien dat ``aan de grenzen en op de hoge zeeën, waar niemand een duurzaam
monopolie op geweld had, handelaars heffingen vermeden die zo hoog waren
dat bescherming goedkoper op andere manieren kon worden verkregen.''

De cybereconomie biedt precies zo'n alternatief. Geen enkele overheid
zal deze kunnen monopoliseren. En de informatietechnologieën die daarbij
horen, zullen goedkopere en effectievere bescherming van financiële
activa bieden dan de meeste overheden ooit moesten leveren.

\subsection{De zwarte magie van samengestelde
rente}\label{de-zwarte-magie-van-samengestelde-rente}

Onthoud: als je elk jaar €5.000 betaalt gedurende veertig jaar, slijt
dat je nettovermogen met €2,2 miljoen, uitgaande van een jaarlijks
rendement van slechts 10 procent op je kapitaal. Bij een rendement van
20 procent loopt het samengestelde verlies op tot ongeveer €44 miljoen.
Voor mensen die veel verdienen in landen met hoge belastingen zijn de
totale verliezen door roofzuchtige belastingheffing over een heel leven
ronduit adembenemend. De meesten verliezen uiteindelijk meer dan ze ooit
bezaten.

Dit klinkt onmogelijk, maar de wiskunde staat als een paal boven water.
Je kunt dit zelf gemakkelijk nagaan met een simpele rekenmachine. De top
1 procent van de belastingbetalers in de Verenigde Staten betaalt
gemiddeld meer dan \$125.000 aan federale inkomstenbelasting per jaar.
Voor een fractie daarvan, namelijk \$45.000 per jaar, kom je in
aanmerking voor een particuliere belastingregeling in Zwitserland, waar
je profiteert van orde en veiligheid, gewaarborgd door wat wordt
beschouwd als het eerlijkste politie- en rechtssysteem ter wereld.
Vanuit dit perspectief kun je de extra \$80.000 aan jaarlijkse
inkomstenbelasting boven dat royale niveau als een ware tribuut of zelfs
als roof bestempelen. Een betaling van \$45.000 is best substantieel
voor de instandhouding van orde en veiligheid, vooral omdat
politiebescherming in principe een collectief goed behoort te zijn. In
theorie kunnen publieke goederen aan extra gebruikers geleverd worden
zonder bijkomende kosten. De Zwitsers zijn dan ook tevreden dat je een
overeengekomen vaste belasting van \$45.000 (oftewel 50.000 Zwitserse
frank) per jaar betaalt, want per aangemelde miljonair boeken zij
daarmee jaarlijks \$45.000 winst.

Als je de Zwitserse regeling vergelijkt, lijdt een belegger die
gemiddeld 20 procent rendement behaalt en federale inkomstenbelasting
betaalt volgens de Amerikaanse tarieven, over een hele levensloop een
verlies van ongeveer \$705 miljoen. Houd er wel rekening mee dat dit
uitgaat van een jaarlijkse belasting van \$45.000. Vergelijk dat eens
met een belastingparadijs als Bermuda, waar vrijwel geen
inkomstenbelasting geldt. Daar loopt het belastingverschil maar liefst
op tot zo'n \$1,1 miljard ten opzichte van de Amerikaanse tarieven.

Je zou kunnen stellen dat een jaarlijks rendement van 20 procent
buitengewoon hoog is -- daar heb je zeker een punt. Maar dankzij de
indrukwekkende groei in Azië gedurende de afgelopen decennia hebben veel
beleggers wereldwijd dat rendement, of zelfs meer, weten te behalen.
Sinds 1950 ligt het samengestelde rendement op vastgoedbeleggingen in
Hongkong op meer dan 20 procent per jaar. Zelfs in economieën die niet
bekendstaan om hun sterke groei, boden zich vaak makkelijke kansen op
hoge winsten. In de afgelopen drie decennia had je met deposito's in
Amerikaanse dollars bij Paraguayaanse banken een reëel gemiddeld
rendement van meer dan 30 procent per jaar kunnen boeken. Hoge
investeringsrendementen zijn in sommige plekken makkelijker te
realiseren dan in andere, maar ervaren beleggers kunnen in goede jaren
absoluut winsten van 20 procent of meer behalen, ook al evenaren zij
niet altijd de prestaties van George Soros of Warren Buffett.

Het ligt natuurlijk voor de hand dat hoe hoger het rendement op je
kapitaal, hoe groter de opportuniteitskosten zijn die ontstaan door
buitensporige inkomsten- en vermogenswinstbelastingen. De vaststelling
dat het verlies gigantisch is, zelfs groter dan het totale vermogen dat
je wellicht ooit had kunnen opbouwen, vereist echter niet dat je
uitzonderlijk hoge rendementen behaalt. Sommige Amerikaanse
beleggingsfondsen boeken al langer dan een halve eeuw een gemiddeld
jaarlijks rendement van meer dan 10 procent. Als dit voor jou het best
haalbare resultaat zou zijn en je behoort tot de top 1 procent van de
Amerikaanse inkomens, dan betekent dat een daling van je nettovermogen
met meer dan \$33 miljoen, enkel door de inkomstenbelasting die je
betaalt over je inkomen boven de \$45.000 per jaar. Vergeleken met een
jurisdictie zonder inkomstenbelasting bedraagt dat verlies zelfs \$55
miljoen.

\subsection{\$55 in plaats van \$55
miljoen}\label{in-plaats-van-55-miljoen}

Als de aannames van economen over winstmaximalisatie correct zijn, en
wij geloven dat dit doorgaans zo is, dan kun je met zekerheid
voorspellen dat de meeste mensen \$55 miljoen zouden proberen veilig te
stellen als ze dat zouden kunnen. Dat is onze voorspelling. Wanneer de
zwarte magie van samengestelde rente duidelijker wordt voor succesvolle
mensen in landen met een hoge belastingdruk, zullen zij serieus beginnen
te shoppen tussen jurisdicties, net zoals ze nu auto's kopen of
verzekeringspolissen vergelijken. Als je twijfelt, vraag dan
willekeurige mensen op straat in New York of Toronto of ze voor \$55
miljoen naar Bermuda zouden verhuizen. De vraag stellen, is hem
beantwoorden. Het dilemma doet denken aan dat van Mark Twain, die zich
afvroeg of hij liever een nacht zou doorbrengen met een naakte Lillian
Russell of met generaal Grant in zijn stijlvolle uniform. Hij hoefde er
niet lang over na te denken. Inwoners van volwassen verzorgingsstaten,
vooral in de Verenigde Staten, zullen trager reageren, maar alleen omdat
ze zich nog niet bewust zijn van de keuze waar ze voor staan. Er komt
een moment waarop ze dat wel zullen zijn. Wie streeft naar een beter
leven, zal het nut zien van het minimaliseren van de verliezen die
roofbelastingen veroorzaken. Je hoeft slechts je transacties in de
cyberspace onder te brengen. Dit zal natuurlijk in veel jurisdicties
illegaal zijn. Maar oude wetten kunnen zelden weerstand bieden tegen
nieuwe technologie. In de jaren 1980 was het in de Verenigde Staten
illegaal om een faxbericht te verzenden. Het Amerikaanse postkantoor
beschouwde faxen als eersteklas post, waarover het een eeuwenoud
monopolie claimde. Er werd een decreet uitgevaardigd dat alle
faxberichten moesten worden doorgestuurd naar het dichtstbijzijnde
postkantoor voor bezorging met de gewone post. Miljarden faxen later is
het onduidelijk of iemand ooit aan die wet heeft voldaan. Als dat al zo
was, dan was dat van korte duur. De voordelen die de opkomende
cybereconomie biedt, zijn nog overtuigender dan het omzeilen van het
postkantoor met een fax.

Een brede adoptie van public-key/private-key-encryptietechnologieën zal
binnenkort veel economische activiteiten overal ter wereld mogelijk
maken. Zoals James Bennet, technologie-redacteur van \emph{Strategic
Investment}, schreef:

\begin{quote}
``De handhaving van wetten, en met name belastingwetgeving, is sterk
afhankelijk geworden van toezicht op communicatie en transacties. Zodra
de volgende logische stappen zijn gezet, en offshorebanken communicatie
aanbieden via sterke RSA-versleutelde e-mail met accountnummers afgeleid
van public-key-systemen, zullen financiële transacties vrijwel
onmogelijk te monitoren zijn, zowel in de bank als in communicatie.
Zelfs als de belastingautoriteiten een infiltrant in de offshorebank
zouden plaatsen of de bankgegevens zouden stelen, zouden zij de
rekeninghouders niet kunnen identificeren.''
\end{quote}

In een mate die nooit eerder mogelijk was, zullen individuen kunnen
bepalen waar zij hun economische activiteiten onderbrengen en hoeveel
inkomstenbelasting zij bereid zijn te betalen. Veel transacties in het
Informatietijdperk hoeven helemaal niet binnen een territoriale
soevereiniteit te worden ondergebracht. Degenen die dat wel doen, zullen
steeds vaker terechtkomen in plaatsen zoals Bermuda, de Kaaimaneilanden,
Uruguay of vergelijkbare jurisdicties die geen inkomstenbelasting of
andere kostbare transactielasten opleggen.

\subsection{Van monopolie naar
concurrentie}\label{van-monopolie-naar-concurrentie}

Overheden zijn gewend geraakt aan het opleggen van
``beschermingsdiensten'' die, in de woorden van Frederic C. Lane, ``van
slechte kwaliteit en belachelijk duur'' zijn. Deze gewoonte om veel meer
te vragen dan de werkelijke waarde van de overheidsdiensten ontwikkelde
zich in de loop van het eeuwenlang durende monopolie. Iedereen die in
staat leek om ze te betalen, werd meedogenloos belast, juist omdat
overheden een monopolie of bijna-monopolie op dwang hadden. Deze
monopolietraditie zal op fundamentele wijze botsen met de nieuwe
megapolitieke mogelijkheden die cyberhandel biedt.

Encryptie zal het gemakkelijk maken om transacties in de cyberspace te
beschermen. De kosten van een effectief encryptieprogramma, zoals PGP,
zijn lager dan de commissie die een full-service broker rekent voor de
aankoop van honderd aandelen. Toch maakt het vrijwel elke transactie
voor nog vele jaren onzichtbaar en onaantastbaar voor overheden en
dieven. In het Informatietijdperk maakt de nieuwe technologie het
mogelijk om digitale bezittingen bijna kosteloos te beschermen. Voor
\$55 in plaats van \$55 miljoen zullen deelnemers aan de cybereconomie
een betere daadwerkelijke bescherming van hun activa genieten dan
tijdens het Industriële Tijdperk of op enig ander moment in de
geschiedenis. Gemakkelijk te gebruiken encryptie-algoritmen en de
mogelijkheid om transacties tussen verschillende landen te verplaatsen,
zullen effectieve bescherming bieden tegen de grootste bron van roof: de
natiestaten zelf.

Dat wil niet zeggen dat territoriale overheden volledig zullen worden
uitgeschakeld. Ze zullen nog steeds persoonlijke kwetsbaarheden kunnen
exploiteren om hoofdelijke belastingen te innen, of mogelijk zelfs rijke
individuen gijzelen voor losgeld. Ze zullen ook de inning van
consumptiebelastingen kunnen afdwingen. Toch zal bescherming, de
belangrijkste dienst van overheden, bijna competitief worden. Een
kleiner deel van de kosten die productieve mensen betalen voor
bescherming zal door politieke autoriteiten kunnen worden ingenomen en
herverdeeld. Technologische innovaties zullen een groot en groeiend deel
van de rijkdom in de wereld buiten het bereik van overheden plaatsen.
Dit zal de risico's van handel verminderen en, in de woorden van de
historicus Janet Abu-Lughod, ``het aandeel van alle kosten'' scherp
verlagen dat anders ``aan transitrechten, heffingen of eenvoudige
afpersing'' zou zijn uitgegeven.

Het kwam zelden in de geschiedenis voor dat overheden werkelijk door
concurrentie werden beperkt. In de weinige gevallen dat iets dergelijks
voorkwam, waren overheden zwak en waren technologieën vergelijkbaar
tussen de rechtsgebieden. Zoals Lane suggereerde, is de belangrijkste
factor die de winstgevendheid onder zulke omstandigheden beïnvloedt het
verschil in beschermingskosten die door verschillende ondernemers worden
betaald. De middeleeuwse handelaar die twintig tolrechten moest betalen
om zijn goederen op de markt te brengen, kon niet concurreren met een
handelaar die slechts vier tolrechten moest betalen om dezelfde goederen
bij de klant te leveren. Vergelijkbare omstandigheden zullen terugkeren
in het Informatietijdperk. Winstgevendheid zal opnieuw niet zozeer
bepaald worden door technologische voordelen, maar door het succesvol
minimaliseren van de kosten die voor bescherming betaald moeten worden.

Deze nieuwe economische dynamiek staat lijnrecht tegenover het verlangen
van overheden uit het Industriële Tijdperk om monopolieprijzen voor hun
beschermingsdiensten op te leggen. Maar, of ze het nou leuk vinden of
niet, het oude systeem zal niet levensvatbaar zijn in de nieuwe
competitieve omgeving van het Informatietijdperk. Elke overheid die
steevast haar burgers zware belastingen oplegt, die ze bij concurrenten
niet hoeven betalen, zal er enkel voor zorgen dat winst en rijkdom
ergens anders naartoe migreren. Zo zal het onvermogen van volwassen
verzorgingsstaten om de belastingen langdurig te verlagen, op den duur
zichzelf corrigeren. Overheden die overmatig belasten, zorgen er
simpelweg voor dat wonen binnen hun macht een financieel onhoudbare
aangelegenheid wordt.

\begin{quote}
\ldots{} zoals de vorst door zijn voorrecht geld kan scheppen uit eender
welke materie en vorm, en de standaard kan bepalen, kan hij eveneens de
samenstelling en het ontwerp van het geld wijzigen, de waarde opdrijven
of verlagen, of het geheel afschaffen en ongeldig verklaren. -- UIT EEN
ENGELSE RECHTSBESLISSING, 1604
\end{quote}

\section{De dood van seigniorage}\label{de-dood-van-seigniorage}

Overheden zullen niet alleen hun macht verliezen om verscheidene vormen
van inkomen en kapitaal te belasten, ze zijn ook gedoemd om de macht
over geld te verliezen. In het verleden gingen megapolitieke transities
vaak gepaard met veranderingen in de aard van geld.

\begin{itemize}
\tightlist
\item
  De introductie van muntgeld droeg bij aan het op gang brengen van de
  vijfhonderdjarige expansiecyclus van de economie in de oudheid, die
  eindigde met de geboorte van Christus en de laagste rentestanden vóór
  het Moderne Tijdperk.
\item
  Het begin van de Donkere Middeleeuwen viel samen met de vrijwel
  volledige sluiting van de muntslagende instituten. Hoewel Romeins
  muntgeld bleef circuleren, kromp de geldhoeveelheid samen met de
  handel in een zichzelf versterkende neerwaartse spiraal.
\item
  De feodale revolutie viel samen met een herintroductie van geld, het
  slaan van muntgeld, wisselbrieven en andere middelen om commerciële
  transacties af te wikkelen. Vooral een stijging in de Europese
  zilverproductie uit nieuwe mijnen in Rammelsberg, Duitsland, maakte
  een grotere circulatie van muntgeld mogelijk, wat de handel
  vergemakkelijkte.
\item
  De grootste revolutie in geld vóór het Informatietijdperk kwam met de
  opkomst van het industrialisme. De vroegmoderne staat consolideerde
  haar macht in de Buskruitrevolutie. Naarmate haar controle toenam,
  eiste de staat ook macht over geld op, en ging zwaar leunen op de
  kenmerkende technologie van het industrialisme: de drukpers. Het
  eerste instrument voor massaproductie, de drukpers, werd in de moderne
  periode breed door overheden gebruikt om grote hoeveelheden papiergeld
  te produceren.
\end{itemize}

Papiergeld is een typisch industrieel product. Vóór de drukpers was het
onpraktisch om kwitanties of certificaten te dupliceren die tot
papiergeld konden worden omgezet. Monniken in de scriptoria zouden hun
tijd zeker niet zinvol besteed hebben met het natekenen van biljetten
van vijftig pond. Papiergeld droeg bovendien aanzienlijk bij aan de
macht van de staat, niet alleen door winst te genereren via devaluatie
van de munt, maar ook door de staat invloed te geven op wie rijkdom kon
opbouwen. Zoals Abu-Lughod stelde: ``Toen papiergeld, gesteund door de
staat, de erkende valuta werd, werd het moeilijk om kapitaal te vergaren
in oppositie tot of onafhankelijk van het staatsapparaat.''

\section{Cybercash}\label{cybercash}

De opkomst van het Informatietijdperk impliceert ook een nieuwe
revolutie in het karakter van geld. Naarmate cyberhandel op gang komt,
zal dit onvermijdelijk leiden tot cybergeld. Deze nieuwe vorm van geld
zal de kansen herschikken en de macht van natiestaten om te bepalen wie
een Soeverein Individu wordt, verminderen. Een cruciaal onderdeel van
deze verandering zal voortkomen uit het bevrijdende effect van
informatietechnologie tegen onteigening van vermogen door inflatie.
Binnenkort zal je voor bijna elke transactie via het Net of World Wide
Web betalen op het moment dat deze plaatsvindt, met cybergeld.

Deze nieuwe digitale vorm van geld zal ongetwijfeld een centrale rol
spelen in de cyberhandel. Het zal bestaan uit versleutelde reeksen
priemgetallen met honderden cijfers. Dit geld, uniek, anoniem en
verifieerbaar, zal de grootste transacties mogelijk maken, terwijkl het
ook deelbaar zal zijn in de kleinste fractie van waarde. Het zal
verhandelbaar zijn met een druk op de knop in een grenzeloze
groothandelsmarkt ter waarde van biljoenen dollars.

\subsection{Handelen zonder dollars}\label{handelen-zonder-dollars}

Het nieuwe cybergeld zal zich onvermijdelijk onttrekken aan nationale
beperkingen. Zodra mensen grensoverschrijdend handelen in een virtuele
wereld, weigeren ze de ouderwetse praktijk waarmee regeringen de waarde
van hun geld kunstmatig doen dalen door inflatie. Waarom zouden ze dat
nog accepteren? De controle over geld verschuift van de machtscentra
naar de wereldwijde marktplaats. Iedereen met toegang tot de cyberspace,
zowel een individu als een onderneming, kan eenvoudig overstappen naar
elke andere valuta wanneer de waarde van de ene dreigt te kelderen. In
tegenstelling tot nu hoef je transacties niet langer met een wettig
betaalmiddel te verrichten. Sterker nog, wanneer transacties wereldwijd
plaatsvinden, rekent ten minste één partij in elke transactie in een
valuta die voor die partij niet als wettig betaalmiddel geldt.

\subsection{Verminderde nadelen van
ruilhandel}\label{verminderde-nadelen-van-ruilhandel}

In de cybereconomie kun je in elk gewenst medium handelen. Zoals de
inmiddels overleden Nobelprijswinnende econoom \emph{E. A. Hayek}
betoogde, bestaat er `geen duidelijk onderscheid tussen geld en
niet-geld.' Hij schreef: ``Hoewel we er doorgaans van uitgaan dat er een
scherpe scheidslijn is tussen wat geld is en wat niet, en de wet vaak
probeert dit onderscheid te handhaven, bestaat er, wat betreft de
causale effecten van monetaire gebeurtenissen, geen dergelijk duidelijk
verschil. Wat we zien, is eerder een continuüm waarin objecten met
verschillende mate van liquiditeit, of met waarden die onafhankelijk van
elkaar kunnen schommelen, in elkaar overlopen in de mate waarin ze als
geld functioneren.'' Digitale valuta op wereldwijde computernetwerken
maken elk object op Hayeks continuüm van liquiditeit extra liquide, met
uitzondering van overheidsgeld. Een direct gevolg hiervan is dat
ruilhandel praktischer wordt. Steeds meer goederen en diensten komen via
specifieke ruilaanbiedingen beschikbaar in ruil voor andere goederen en
diensten. Deze potentiële transacties verschijnen wereldwijd op het Net,
waardoor hun liquiditeit enorm toeneemt.

Een van de grootste nadelen van ruilhandel is dat het lastig is om
precies de juiste mensen bij elkaar te brengen, waarbij iemand met een
specifieke behoefte wordt gekoppeld aan iemand die exact kan leveren wat
gevraagd is. Vroeger stond ruilhandel voor de moeilijke uitdaging om
twee partijen te vinden die met elkaar wilden handelen op de lokale
markt. Contant geld oversteeg de beperkingen van ruilhandel en blijft
dat in veel transacties doen. Tegelijkertijd verkleinen de enorme groei
van computerkracht en de globalisering van de handel via cyberspace ook
de nadelen van directe ruilhandel. De kans dat je iemand vindt met
wensen die precies bij die van jou aansluiten, neemt enorm toe wanneer
je je niet hoeft te beperken tot de lokale omgeving, maar wereldwijd
kunt zoeken.

\subsection{Niet vatbaar voor
vervalsing}\label{niet-vatbaar-voor-vervalsing}

Hoewel papiergeld ongetwijfeld over zal blijven als circulerend
ruilmiddel voor de armen en computeranalfabeten, zal geld voor
transacties met hoge waarde geprivatiseerd worden. Cybergeld zal niet
langer uitsluitend worden uitgedrukt in nationale eenheden zoals het
papiergeld van het Industriële Tijdperk. Het zal waarschijnlijk worden
gedefinieerd in grammen of ounces goud, en zal net zo deelbaar zijn als
goud zelf. Of het kan worden gedefinieerd in termen van andere echt
waardevaste middelen. Zelfs op plekken waar verschillende
prijsmaatstaven worden gebruikt, of bepaalde transacties in nationale
valuta geprijsd blijven, zal cybergeld de consument veel beter dienen
dan genationaliseerd geld ooit deed. Het voortdurende bijstellen van
prijzen, wat noodzakelijk is bij het gebruik van verschillende
ruilmiddelen, zal door de snelle groei van rekenkracht vrijwel volledig
probleemloos verlopen. Elke transactie zal een versleutelde reeksen
priemgetallen met honderden cijfers overdragen. In tegenstelling tot de
papieren geldbewijzen die door overheden tijdens de goudstandaard werden
uitgegeven, en die naar believen konden worden gedupliceerd, zullen de
nieuwe digitale goudstandaard of haar ruil­handelequivalenten vrijwel
onmogelijk te vervalsen zijn, om de fundamentele wiskundige reden dat
het praktisch onmogelijk is om het product van zulke grote priemgetallen
te ontleden. Alle bewijzen zullen verifieerbaar uniek zijn.

De namen van traditionele valuta zoals de ``pond'' en de ``peso''
weerspiegelen het feit dat ze ooit ontstonden als gewichtseenheden van
specifieke hoeveelheden edelmetaal. Het pond sterling was ooit
letterlijk een pond sterlingzilver. Papiergeld in het Westen begon als
bewijs voor de opslag van hoeveelheden edelmetaal in kluizen of andere
opslagplaatsen. Overheden die deze bewijzen uitgaven, ontdekten al snel
dat ze er veel meer van konden drukken dan mensen geneigd waren om
daadwerkelijk weer voor hun edelmetaal in te wisselen. Dit was
eenvoudig. Geen enkele houder van een goud- of zilvercertificaat kon, op
basis van dat bewijs, inschatten hoe groot de werkelijke voorraad
edelmetaal was. Afgezien van de serienummers zagen alle bewijzen er
hetzelfde uit, een feit dat aansprak bij zowel valsmunters als politici
en bankiers die wilden profiteren van inflatie.

Cybergeld zal op deze manier vrijwel onmogelijk te vervalsen zijn, zowel
officieel als officieus. De verifieerbaarheid van de digitale bewijzen
sluit dit klassieke middel tot onteigening van vermogen via inflatie
uit. Het nieuwe digitale geld van het Informatietijdperk zal de controle
over het ruilmiddel teruggeven aan de bezitters van vermogen, die het
willen behouden, in plaats van aan natiestaten die het willen afromen.

\subsection{De transactiekosten van `vrij'
geld}\label{de-transactiekosten-van-vrij-geld}

Het gebruik van dit nieuwe cybergeld zal je in belangrijke mate
bevrijden van de macht van de staat. Eerder hebben we de sombere staat
van dienst van natiestaten in de afgelopen halve eeuw besproken op vlak
van het behoud van de waarde van hun valuta. Geen enkele munt heeft
sinds de Tweede Wereldoorlog een kleiner verlies door inflatie geleden
dan de Duitse mark. Toch verdween zelfs daarvan 71 procent van de waarde
tussen 1 januari 1949 en eind juni 1995. De wereldreservemunt in die
periode, de Amerikaanse dollar, verloor 84 procent van zijn waarde. Dit
is een maat voor de rijkdom die overheden hebben onteigend via de
uitbuiting van hun territoriale monopolie op een wettig betaalmiddel.

Er is overigens geen enkele intrinsieke noodzaak voor een valuta om in
waarde te dalen of voor een jaarlijkse toename van de nominale kosten
voor levensonderhoud. Integendeel, de technische uitdaging om de
koopkracht van spaargeld te behouden is minimaal. Dit valt duidelijk op
als je de langetermijnkoopkracht van goud bekijkt.

Tussen 1 januari 1949 en eind juni 1995, terwijl de beste nationale
valuta bijna driekwart van hun waarde verloren, steeg de koopkracht van
goud juist. Zoals professor Roy W. Jastrom heeft aangetoond in \emph{The
Golden Constant,} heeft goud, afgezien van kleine fluctuaties, zijn
koopkracht behouden sinds de vroegste betrouwbare prijsgegevens, die in
Engeland teruggaan tot 1560.

Nationale valuta die aan goud waren gekoppeld, behielden hun koopkracht
ook zolang er geen militaire noodsituaties waren. De waarde van het
Britse pond sterling steeg zelfs in de relatief vreedzame negentiende
eeuw, ondanks dat het slechts zwak aan goud gekoppeld was. De nieuwe
megapolitieke omstandigheden van het Informatietijdperk maken voor het
eerst juist een sterke koppeling mogelijk, geen zwakke koppeling zoals
de goudstandaard, door de enorm verbeterde informatie- en rekenmiddelen
in handen van consumenten.

\begin{quote}
Het gevaar van het snelle verlies van hun hele bedrijf als ze niet aan
de verwachtingen voldoen (en de zekerheid dat een overheidsorganisatie
de kans zou misbruiken om grondstofprijzen te manipuleren!) zou een veel
sterkere waarborg tegen een staatsmonopolie bieden dan welke ook maar
bedacht zou kunnen worden. -- FRIEDRICH A. VON HAYEK
\end{quote}

\subsection{Privatisering van geld}\label{privatisering-van-geld}

Friedrich von Hayek betoogde in 1976 dat het gebruik van concurrerende,
private valuta inflatie zou uitbannen. Zonder de verplichting om een
inflatoire munt als wettig betaalmiddel binnen een jurisdictie te
accepteren, zo stelde Hayek, zou marktconcurrentie de private uitgevers
van valuta dwingen de waarde van hun ruilmiddelen te behouden. Elke
uitgever die er niet in zou slagen om de waarde te handhaven, zou snel
zijn klanten verliezen. De ontwikkeling van versleuteld cybergeld zal
Hayeks logica duidelijk nieuw leven inblazen.

De theorie van ``free banking'', zoals die wordt genoemd, is niet
slechts een hypothetische academische gedachte. Concurrerende private
valuta circuleerden in Schotland vanaf het begin van de achttiende eeuw
tot 1844. In die periode had Schotland geen centrale bank. Er waren
weinig regels of beperkingen om tot de bankensector toe te treden.
Private banken namen deposito's aan en gaven hun eigen valuta uit,
gedekt door goud. Zoals professor Lawrence White heeft gedocumenteerd,
werkte dit systeem goed. Het was stabieler, met minder inflatie dan het
zwaarder gereguleerde en gepolitiseerde bankensysteem dat in diezelfde
periode in Engeland werd gehanteerd. Michael Prowse van de
\emph{Financial Times} vatte de Schotse ervaring met free banking als
volgt samen: ``Er was weinig fraude. Er was geen bewijs van de
overmatige uitgifte van biljetten. Banken hielden doorgaans noch
buitensporige, noch ontoereikende reserves aan. Bankruns waren zeldzaam
en niet besmettelijk. De vrije banken genoten het respect van de burgers
en boden een solide basis voor economische groei die gedurende het
grootste deel van die periode groter was dan die in Engeland.''

Wat in de achttiende en negentiende eeuw al goed werkte, zal nog beter
functioneren met de technologie van de eenentwintigste eeuw. Je zult
binnenkort in staat zijn om te handelen in digitaal geld van een private
onderneming, uitgegeven op een manier die vergelijkbaar is met hoe
American Express travelcheques uitgeeft als bewijs van contant geld. Een
instelling met meer aanzien dan welke regering dan ook, zoals een
toonaangevend mijnbedrijf of de Swiss Bank Corporation, zou versleutelde
bewijzen kunnen creëren voor hoeveelheden goud of zelfs voor unieke
goudstaven, geïdentificeerd door moleculaire kenmerken en mogelijk zelfs
voorzien van hologrammen. Deze bewijzen zullen vervolgens als geld
circuleren, vrijwel zonder de kans om vervalst of opgeblazen te worden.

Het nieuwe digitale goud zal veel van de praktische problemen overwinnen
die het directe gebruik van goud als geld in het verleden hebben
belemmerd. Het zal niet langer onhandig, omslachtig of gevaarlijk zijn
om in grote bedragen goud te handelen. Digitale bewijzen zullen niet te
zwaar zijn om te dragen; hun enige fysieke bestaan zal in feite complexe
patronen van computercode zijn. Ook zal het eenvoudig zijn om digitale
bewijzen te delen in eenheden die klein genoeg zijn om zelfs
microbetalingen mogelijk te maken. Een stukje fysiek goud dat klein
genoeg is om een kauwgompje te betalen, zou snel kwijt raken of verward
worden met een stukje dat groot genoeg is om twee kauwgompjes te
betalen. Maar voor een computer zal het even makkelijk zijn om deze
denominaties van digitaal geld te onderscheiden alsof ze zo groot waren
als een eekhoorn of een neushoorn.

Het vermogen van digitaal geld om microbetalingen mogelijk te maken, zal
de opkomst faciliteren van nieuwe soorten bedrijven die tot nu toe niet
konden bestaan, gespecialiseerd in het organiseren van de distributie
van laagwaardige informatie. De aanbieders van deze informatie zullen nu
worden beloond via directe betalingsregelingen die de vroegere
ontmoedigende transactiekosten overwinnen. Wanneer de kosten van
facturering hoger zijn dan de waarde van een transactie, zal deze
waarschijnlijk niet plaatsvinden. Het gebruik van cybergeld maakt zeer
goedkope onmiddelijke afrekening mogelijk, waarbij rekeningen direct bij
gebruik worden afgewikkeld. We noemden eerder al het voorbeeld waarbij
je misschien een royalty van een derde van een cent betaalt aan Bill
Gates, of aan wie de virtuele rechten dan ook toebehoren om het Louvre
te bezoeken. Vermenigvuldig dit met duizend toepassingen. Virtual
reality zal bijna onbeperkte licentiemogelijkheden creëren die
desondanks slechts microroyalty's zullen opleveren. Op een dag zul je
bijvoorbeeld het derde duel van de World Series van 1969 opnieuw kunnen
beleven en microroyalty's betalen aan de spelers wier beelden worden
gebruikt om jouw virtuele realiteit echt te laten lijken.

\section{Inflatie uitroeien}\label{inflatie-uitroeien}

Ondanks al deze mogelijkheden zal de meest ingrijpende consequentie van
het nieuwe digitale geld ongetwijfeld het einde van inflatie en de
afbouw van schuld in het financiële systeem zijn. De economische
implicaties zijn diepgaand. De opkomst van inflatie in de twintigste
eeuw, zoals we betoogden in \emph{Blood in the Streets} en \emph{The
Great Reckoning}, hing nauw samen met de machtsverhoudingen in de
wereld. De toenemende opbrengsten van geweld forceerden veel hogere
militaire uitgaven, wat op zijn beurt steeds agressievere pogingen
vereiste om vermogen te onteigenen. Overheden ontdekten dat ze in feite
een jaarlijkse vermogensbelasting konden opleggen aan iedereen die
tegoeden in hun nationale valuta aanhield. Deze jaarlijkse
vermogensbelasting voor eigenaars van valuta kon ook worden gezien als
een transactiekost voor het gemak om vermogen te bewaren in een handige
vorm, geleverd door de uitgevers.

Het idee dat inflatie een transactiekost is die voor het gebruiksgemak
van valuta wordt aangerekend, is ongewoon, maar denk er goed over na.
Tijdens het Industriële Tijdperk raakten we zo gewend om valuta te zien
als een dienst waarvoor men niet direct betaalt, dat het makkelijk werd
vergeten dat de uitgevers van dollars, peso's, ponden en francs,
overheden dus, wel degelijk betaling eisten, en wel fors, namelijk via
inflatie. Het tarief van deze inflatoire transactiekost varieerde in de
afgelopen halve eeuw van 2,7 procent per jaar als laagste percentage
voor de Duitse mark tot percentages die gevaarlijk dicht bij 100 procent
lagen. Zo verloor de Argentijnse munt tussen 1960 en 1991, toen
president Menem de currency-boardhervorming doorvoerde, zeventien nullen
door opeenvolgende golven van inflatie. Als al het vermogen van de
wereld in 1960 in Argentijnse peso's zou zijn omgezet en begraven, dan
zou het in 1991 niet meer de moeite waard zijn geweest om het op te
graven.

Het voorbeeld van Argentinië is een voorloper voor het volgende
millennium. Geld zal niet meer geïnflateerd worden omdat andere
natiestaten er niet langer mee weg zullen komen, net zoals dat
Argentinië dat niet meer kan. Inflatie had in het Industriële Tijdperk
nog een aantrekkingskracht omdat prijzen en lonen niet gemakkelijk
neerwaarts bewogen. Lichte inflatie verhoogde de productie door de reële
lonen te verlagen, terwijl prijzen kwetsbaar waren voor een ingevoerde
kredietkrimp uit het buitenland. Privaatgeld zal niet inflatoir zijn
door de druk van concurrentie.

Het einde van inflatie zal de verborgen winsten wegnemen die inflatie
eerder verschafte aan de monopolistische uitgevers van geld. Als al deze
verborgen winsten verdwijnen, zal een nieuwe methode van betaling nodig
zijn om de uitgevers rechtstreeks te compenseren. Het gebruik van het
nieuwe monetaire systeem zal daarom waarschijnlijk een expliciete
transactiekost inhouden, mogelijk een vergoeding van ongeveer 1 procent
per jaar. Dit is een kleine prijs in vergelijking met de jaarlijkse
inflatieboete van 2,7 tot 99 procent die door natiestaten werd opgelegd.
Zeker omdat de kans groot is dat de algemene prijzen in de toekomst
zullen dalen naarmate monopolies afbrokkelen en concurrentie wereldwijd
toeneemt.

\subsection{Afnemende hefboomwerking}\label{afnemende-hefboomwerking}

De opkomst van digitaal geld zal niet alleen inflatie definitief
bestrijden, maar ook de hefboomwerking in het wereldwijde bankensysteem
verminderen. Dat mensen wereldwijd regelgeving kunnen omzeilen en hun
vermogen rechtstreeks via internet kunnen verplaatsen, is een totaal
nieuwe uitkomst van de geglobaliseerde markten. Het zal buiten de macht
van welke regering dan ook liggen om dit te reguleren. Wanneer
regeringen valuta niet langer kunnen devalueren door geld bij te drukken
of het bedriegen van spaarders via de uitbreiding van krediet in een
gevangen bankensysteem, verliezen ze een groot deel van hun indirecte
vermogen om middelen op te eisen.

\subsection{Hogere rentetarieven}\label{hogere-rentetarieven}

Dit zal een duidelijk dilemma creëren voor de meeste westerse
regeringen. Ze zullen te maken krijgen met scherpe dalingen in hun
inkomsten uit belastingen en de bijna volledige eliminatie van
hefboomwerking in het monetaire systeem. Tegelijkertijd behouden ze de
ongedekte schulden en opgeblazen verwachtingen voor sociale uitgaven die
ze uit het Industriële Tijdperk hebben geërfd. Het te verwachten
resultaat is een intense fiscale crisis met veel onaangename sociale
bijeffecten, die in latere hoofdstukken worden besproken. De economische
consequentie van deze overgangsperiode zal waarschijnlijk een eenmalige
piek in de reële rentetarieven inhouden. Schuldenaren zullen onder druk
komen te staan naarmate langlopende verplichtingen uit het oude systeem
worden afgelost en gunstige kredieten opdrogen.

\subsection{Veranderd door
concurrentie}\label{veranderd-door-concurrentie}

Wanneer overheden worden geconfronteerd met serieuze concurrentie voor
hun monopolies op geld, zullen ze waarschijnlijk cybervaluta proberen te
onderwaarderen door krediet te versmallen en spaarders betere
rendementen op contanten in de nationale valuta te geven. Sommige
overheden kunnen zelfs proberen goud opnieuw als geld te introduceren om
de concurrentie met private valuta aan te gaan. Ze zullen mogelijk
denken dat ze hogere seignioragewinsten zullen behalen met een soepele
negentiende eeuwse goudstandaard dan wanneer ze hun nationale valuta
volledig zouden laten verdrukken door commercieel cybergeld. Niet alle
overheden zullen echter op dezelfde manier reageren. Overheden in
regio's waar computergebruik en Net-deelname laag zijn, kunnen in de
vroege fase van de cybereconomie kiezen voor ouderwetse hyperinflatie.
Dit stelt deze regeringen niet in staat om de contante tegoeden van
rijken af te pakken, maar het zal middelen afromen van degenen met
weinig rijkdom of toegang tot de cybereconomie. Dergelijke overheden
kunnen desalniettemin internationaal lenen in cybergeld.

Andere overheden kunnen zich aanpassen aan de kansen die de
Informatiemaatschappij biedt en lokale transacties in cybergeld
faciliteren. De rechtsgebieden die als eerste de geldigheid van digitale
handtekeningen erkennen en lokale gerechtelijke handhaving bij
niet-betaling van cyberschulden bieden, zullen profiteren van een
onevenredige toename in langetermijnkapitaalverstrekking. Uiteraard zal
er geen cybergeld beschikbaar zijn voor langlopende kredieten in
gebieden waar lokale rechtbanken straffen oplegden of schuldenaren
zonder consequenties toestonden in gebreke te blijven.

\subsection{Rentekloof}\label{rentekloof}

De combinatie van kredietcrisissen, competitieve aanpassingen door
nationale muntautoriteiten en vroege overgangsbelemmeringen bij het
verstrekken van kredieten in cybergeld zorgt in de beginfase van de
informatie-economie voor een rentekloof. Naar verwachting biedt
cybergeld lagere rentetarieven dan de nationale valuta en brengt het
waarschijnlijk ook expliciete transactiekosten met zich mee. Een
verbeterde bescherming tegen verliezen door buitensporige belastingen en
inflatie zorgt ervoor dat deze nadelen worden gecompenseerd. Aangezien
cybergeld vermoedelijk aan goud gekoppeld wordt, profiteert het tevens
van een stijgende goudprijs. De prijs van goud zal naar verwachting
sterk stijgen ten opzichte van andere grondstoffen, ongeacht welke
alternatieve overheidsmaatregel de doorslag krijgt. Waarom? De reële
prijs van goud stijgt vrijwel altijd tijdens een deflatie, want deflatie
duidt immers op een tekort aan liquiditeit. Goud blijft uiteindelijk de
ultieme vorm van liquiditeit.

\subsection{De deflatie van het Industriële
Tijdperk}\label{de-deflatie-van-het-industriuxeble-tijdperk}

Hogere reële rentes dwingen overal tot het liquideren van dure,
onproductieve activiteiten en drukken de consumptie tijdelijk omlaag. We
hebben de logica van de kredietcyclus en de daaropvolgende afwikkeling
ervan behandeld in \emph{Blood in the Streets} en \emph{The Great
Reckoning}, dus herhalen we die argumenten hier niet. Het volstaat te
stellen dat de deflatoire periode enige tijd kan aanhouden, waarbij de
dure industriële economieën in Noord-Amerika en West-Europa harder onder
de nadelige gevolgen zullen lijden dan de goedkope economieën in Azië en
Latijns-Amerika.

\subsection{Lagere rentes op lange
termijn}\label{lagere-rentes-op-lange-termijn}

Hoewel de opkomst van de cybereconomie aanvankelijk tot hogere
rentetarieven leidt, keert het effect op de lange termijn precies om. De
rendementen na belasting zullen voor spaarders fors stijgen zodra
middelen ontsnappen aan de greep van overheden. Dramatische
verbeteringen in de efficiëntie van het gebruik van hulpbronnen en de
bevrijding van kapitaal om wereldwijd de hoogste rendementen te behalen,
compenseren naar verwachting de aanvankelijk verloren productie tijdens
de transitiecrisis snel.

\subsection{Investeerderscontrole over
kapitaal}\label{investeerderscontrole-over-kapitaal}

Conventionele denkers die ons betoog op dit punt beoordelen, zouden
concluderen dat het wegvallen van inkomensherverdeling in de leidende
natiestaten de wereld voor een economische ineenstorting zou zorgen.
Geloof dat niet. We ontkennen niet dat een overgangscrisis
waarschijnlijk is, maar de opvatting dat de staat de economie verbetert
door massale herverdeling van middelen is een achterhaald geloof,
vergelijkbaar met de wijdverspreide bijgeloven aan het einde van de
Middeleeuwen dat vasten en geseling nuttig zouden zijn voor een
gemeenschap. Het mag niet vergeten worden dat overheden op grote schaal
middelen verspillen. Middelen verspillen maakt je arm. Een dramatische
verbetering in de efficiëntie van het gebruik van middelen zal ontstaan
wanneer inkomsten die historisch door overheden werden beheerd, in
handen komen van personen met werkelijk talent.

Tientallen miljarden, en uiteindelijk honderden miljarden dollars,
zullen worden beheerd door honderden duizenden, en later miljoenen
Soevereine Individuen. Deze nieuwe beheerders van 's werelds rijkdom
zullen waarschijnlijk veel capabeler blijken dan politici wanneer het
aankomt op het benutten van middelen en het inzetten van investeringen.
Voor het eerst in de geschiedenis zullen megapolitieke omstandigheden de
meest bekwame investeerders en ondernemers, in plaats van specialisten
in geweld, ultieme controle over kapitaal geven. Het is niet onredelijk
om te verwachten dat de rendementen op deze verspreide, marktgedreven
investeringen het dubbele of drievoudige kunnen zijn van de magere
rendementen van de door politiek gedreven budgetallocaties uit het
tijdperk van de natiestaat. In de laatste decennia van de twintigste
eeuw waren voorbeelden van overheidsinvesteringen met substantieel
negatieve opbrengsten in geen enkel land een uitzondering. We citeerden
officiële Russische statistieken in de herziene versie van \emph{The
Great Reckoning} van november 1992, waaruit bleek dat de hele Russische
economie ``slechts \$30 miljard waard was, minder dan een derde van de
waarde van haar gebruikte ruwe grondstoffen. Dit impliceert dat de
output van Rusland meer dan verdrievoudigen zou als de binnenlandse
productie- en diensteneconomie volledig zou worden stilgelegd. In plaats
van waarde toe te voegen, trekken ze waarde af.''

Het voorbeeld van Rusland na de val van het communisme is extreem, maar
er is ruim bewijs dat het verminderen van staatscontrole over middelen
de economische efficiëntie verbetert. Groei­cijfers vermeld door
\emph{The Economist} suggereren dat economische vrijheid sterk
correleert met economische groei, waarbij de snelst groeiende landen ook
de meest vrije zijn. De cybereconomie van het Informatietijdperk zal
vrijer zijn dan enig ander commercieel domein in de geschiedenis. Het is
daarom redelijk om te verwachten dat de cybereconomie snel de
belangrijkste nieuwe economie van het nieuwe millennium zal worden. Het
succes zal nieuwe deelnemers van over de hele wereld aantrekken, net
zoals het brede gebruik van faxmachines telecopiëren steeds
aantrekkelijker maakte voor niet-gebruikers. Maar nog belangrijker,
vrijheid van roofzuchtig geweld zal de cybereconomie in staat stellen te
groeien met veel hogere samengestelde groeipercentages dan conventionele
economieën die door natiestaten worden gedomineerd.

Dit is mogelijk het belangrijkste aspect bij het inschatten van de
economische impact van het waarschijnlijke falen van het
overheidsmonopolie op belastingheffing en inflatie. Afgezien van
overgangsproblemen, die decennia kunnen duren, zijn de
langetermijnvooruitzichten voor de wereldeconomie zeer gunstig. Wanneer
omstandigheden het voor mensen mogelijk maken om kosten voor bescherming
te verlagen en tribuut aan degenen die georganiseerd geweld beheersen te
minimaliseren, groeit de economie meestal dramatisch. Zoals Lane zei:
``Ik wil suggereren dat de meest gewichtige factor in de meeste
groeiperioden, als één factor het belangrijkst was, een vermindering is
van het aandeel van middelen dat aan oorlog en politie wordt besteed.''

Er kunnen grote efficiëntiewinsten ontstaan door een vermindering van
middelen die aan roof en het leven van de buit daarvan worden besteed.
Als de prijsstelling van bescherming competitief zou worden gemaakt,
waarbij lokale monopolies concurreren op prijs en kwaliteit, zouden
potentieel enorme efficiëntiewinsten mogelijk zijn. Het te verwachten
resultaat zouden veel lagere belastingtarieven zijn en minder verlies
van middelen en inspanningen in politieke activiteiten, die niet langer
hun eerdere enorme opbrengsten zouden opleveren.

Zouden kiezers bereid zijn om de politieke cadeautjes op te geven waar
ze aan gewend zijn geraakt? Dat is een kwestie die elders uitgebreid
wordt besproken. Maar een eenvoudig antwoord is dat we mogelijk geen
keuze hebben. Niemand demonstreert nu tegen regen of droogte, hoe
economisch schadelijk of onaangenaam die ook zijn. Niemand, hoe
crimineel ook ingesteld, houdt een arme als gijzelaar vast voor een
enorm bedrag aan losgeld, op straffe van de dood. Als het voor politici
onmogelijk wordt om middelen te verkrijgen voor herverdeling, kan het
publiek rationeel reageren en politiek naast zich neer leggen, net zoals
dat goedbedoelende mensen stopten met het organiseren van boetemarsen
toen de Middeleeuwen eindigden.

\bookmarksetup{startatroot}

\chapter{Het einde van egalitaire
economie}\label{het-einde-van-egalitaire-economie}

::: \{.content-hidden when-format=``latex''\} \emph{De revolutie van het
vermogen om geld te verdienen in een wereld zonder banen} :::

\begin{quote}
God laat zich niet bespotten: want wat een mens zaait, zal hij ook
oogsten. -- Galaten 6:7
\end{quote}

Grote veranderingen in de dominante productie- of defensiemethoden
veranderen de structuur van de samenleving en de verdeling van rijkdom
en macht. Het Informatietijdperk betekent meer dan alleen het groeiende
gebruik van krachtige computers: het betekent een revolutie in
levensstijlen, instituties en de verdeling van middelen. Met de sterke
terugval van verborgen geweld als middel om hulpbronnen te beheersen,
zal de verdeling van vermogen zich herordenen, vrij van de gedwongen
inmenging van overheden zoals die in de twintigste eeuw gebruikelijk
was. Locatie zal veel minder betekenen in de Informatiesamenleving,
waardoor organisaties die gebonden zijn aan geografie, zoals politici,
vakbonden, gereguleerde beroepen, lobbyisten en regeringen zelf, aan
belang verliezen. De gunsten en handelsbeperkingen die overheden
mogelijk maakten, verliezen hun waarde, waardoor minder middelen worden
verspild aan lobbies of het verzet daartegen.

Degenen die dwang en lokale voordelen gebruikten om het inkomen te
herverdelen, verliezen veel macht. Dit verandert de controle over
middelen: vermogen zal bij de mensen blijven die het creëren, in plaats
van te worden onteigend door de natiestaat. Steeds meer vermogen vloeit
naar de meest bekwame ondernemers en durfkapitalisten wereldwijd.
Globalisering en de kenmerken van de informatie-economie vergroten de
inkomsten van de meest getalenteerden in elk veld. Omdat de marginale
waarde van uitzonderlijke prestaties enorm is, zal de verdeling van het
vermogen om geld te verdienen wereldwijd steeds meer lijken op die in
prestatieberoepen zoals sport en opera.

\section{Een omvang voorbij de wet van
Pareto}\label{een-omvang-voorbij-de-wet-van-pareto}

De wet van Pareto stelt dat 80 procent van het voordeel afhankelijk is
van of toekomt aan 20 procent van de betrokkenen. Dit is ongeveer waar,
maar opvallender is dat 1 procent van de bevolking van de Verenigde
Staten 28,7 procent van de inkomstenbelasting betaalt. Dit suggereert
dat samenlevingen die het Informatietijdperk ingaan een nog schevere
verdeling van inkomens en bekwaamheid zullen ervaren dan Vilfredo Pareto
aan het einde van de negentiende eeuw waarnam. Mensen zijn gewend aan
substantiële vermogensongelijkheid. In 1828 bezat 4 procent van de New
Yorkers naar schatting 62 procent van alle rijkdom in de stad. In 1845
bezat de top 4 procent ongeveer 81 procent van al het bedrijfs- en
niet-bedrijfsvermogen in New York City. Breder gezien bezat de top 10
procent van de bevolking in 1860 ongeveer 40 procent van het vermogen in
de hele VS. Tegen 1890 suggereren gegevens dat de rijkste 12 procent
ongeveer 86 procent van Amerika's vermogen bezat.

De cijfers van 1890 sluiten goed aan bij Pareto's idee. Ze wijken vooral
af door de enorme toestroom van straatarme immigranten aan het einde van
de negentiende eeuw. Hun aandeel in de totale rijkdom was te
verwaarlozen, waardoor hun komst de ongelijkheid automatisch vergrootte.
Dit illustreert hoe een echte toename van kansen vrijwel onvermijdelijk
leidt tot een tijdelijke golf van ongelijkheid. In 1890 bestond zo'n 15
procent van de Amerikaanse bevolking uit immigranten, en in sommige
noordoostelijke staten, waar veel inkomen en rijkdom werd gegenereerd,
zelfs meer dan 40 procent. Gecorrigeerd voor deze immigratiegolf paste
het Amerika van de late negentiende eeuw goed bij Pareto's formule, net
als het Zwitserland van die tijd waar Pareto zelf woonde.

Het Informatietijdperk heeft de verdeling van rijkdom al veranderd,
vooral in de VS, en is een van de oorzaken van de bitterheid in de
moderne Amerikaanse politiek, waar we in het volgende hoofdstuk verder
op ingaan. Succes in de informatiesamenleving vereist een hoog niveau
van geletterdheid en rekenvaardigheid. Een grootschalig onderzoek van
het Amerikaanse ministerie van onderwijs, \emph{Adult Literacy in
America}, wees uit dat wel 90 miljoen Amerikanen ouder dan vijftien
ernstig incompetent zijn. Of, zoals de Amerikaanse emigrant Bill Bryson
het kleurrijk verwoordde: ``Ze zijn zo dom als varkenssnot.'' Concreet
werden 90 miljoen Amerikaanse volwassenen beoordeeld als niet in staat
om een brief te schrijven, een busdienstregeling te begrijpen, of op te
tellen en af te trekken, zelfs met een rekenmachine. Wie al moeite heeft
met een bustijdenoverzicht, zal weinig kunnen met de Informatiesnelweg.
Uit dit derde deel van de Amerikanen dat zich niet heeft voorbereid op
de elektronische informatiewereld, ontstaat een boze onderklasse. Aan de
top staat een kleine groep, misschien 5 procent, van hoogopgeleide
informatie­werkers of kapitaalbezitters: de tegenhangers van de feodale
landadel. Met dit cruciale verschil: de elite van het Informatietijdperk
zijn specialisten in productie, niet in geweld.

\subsection{De megapolitiek van
innovatie}\label{de-megapolitiek-van-innovatie}

Zonder een echte goede reden zijn de meeste twintigste-eeuwse sociologen
ervan uitgegaan dat technologische vooruitgang vanzelf zou leiden tot
steeds meer egalitaire samenlevingen. Dit was vóór ongeveer 1750 niet
waar. Rond die tijd begon nieuwe industriële technologie banen te openen
voor ongeschoolden en maakte het schaalvergroting van ondernemingen
mogelijk. De fabriekstechnologie verhoogde niet alleen de reële inkomens
van de armen zonder speciale inspanning vanuit hunzelf, maar versterkte
ook de macht van politieke systemen, waardoor zij zowel beter in staat
waren om het inkomen te herverdelen als beter bestand tegen onrust. Op
langere termijn is er geen enkele reden om aan te nemen dat technologie
de verschillen in menselijke talenten en motivatie altijd maskeert in
plaats van accentueert. Sommige technologieën waren relatief egalitair
en vroegen om bijdragen van vele onafhankelijke arbeiders van ongeveer
gelijke waarde; andere concentreerden macht of rijkdom in de handen van
enkele meesters, terwijl de rest nauwelijks meer dan lijfeigenen was.
Zowel geschiedenis als technologie hebben verschillende landen op
verschillende manieren gevormd. Het Fabriekstijdperk bracht één vorm
voort, en het Informatietijdperk brengt een andere. Een die minder
gewelddadig is, maar daardoor ook meer elitair en minder egalitair dan
het tijdperk dat het vervangt.

\section{\texorpdfstring{\textbf{Ammon's
raap}}{Ammon's raap}}\label{ammons-raap}

Aan het eind van de negentiende eeuw begonnen verschillende economen,
van wie William Stanley Jevons de meest vooraanstaande was in Engeland,
wiskundige economie te ontwikkelen. Een van de eersten die
kansberekening toepaste op een grote sociale kwestie was de Duitse
econoom Otto Ammon. Zijn werk werd voor het eerst in het Engels vertaald
door Carlos C. Closson in een artikel in de \emph{Journal of Political
Economy} in 1899, getiteld \emph{Some Social Applications of the
Doctrine of Probability.} Men zou kunnen veronderstellen dat een
dergelijk artikel nu slechts antieke waarde heeft. In werkelijkheid
behandelt het een economisch vraagstuk dat opnieuw actueel wordt, en het
doet dat nog steeds op bijzondere wijze.

Otto Ammon was geïnteresseerd in de verdeling van bekwaamheid in de
samenleving, en de relatie daarvan tot de verdeling van inkomen en
status. Hij begon met de waarschijnlijkheidsverdeling van uitkomsten van
vier dobbelstenen met zes zijden. Van de 1.296 mogelijke worpen komen
sommige totalen vaker voor dan andere.

\begin{quote}
De som van 24 komt 1 keer voor.\\
De som van 23 komt 4 keer voor.\\
De som van 22 komt 10 keer voor.\\
De som van 21 komt 20 keer voor.\\
De som van 20 komt 35 keer voor.\\
De som van 19 komt 56 keer voor.\\
De som van 18 komt 80 keer voor.\\
De som van 17 komt 104 keer voor.\\
De som van 16 komt 125 keer voor.\\
De som van 15 komt 140 keer voor.\\
De som van 14 komt 146 keer voor.\\
De som van 13 komt 140 keer voor.\\
De som van 12 komt 125 keer voor.\\
De som van 11 komt 104 keer voor.\\
De som van 10 komt 80 keer voor.\\
De som van 9 komt 56 keer voor.\\
De som van 8 komt 35 keer voor.\\
De som van 7 komt 20 keer voor.\\
De som van 6 komt 10 keer voor.\\
De som van 5 komt 4 keer voor.\\
De som van 4 komt 1 keer voor.
\end{quote}

Daaruit blijkt meteen dat zowel hoge als lage scores relatief zeldzaam
zijn. De uiterste totalen komen in totaal slechts 35 keer voor, terwijl
de middelste zeven groepen samen 884 keer voorkomen; het middelste derde
deel van de mogelijke scores komt dus in meer dan tweederde van alle
worpen voor. Dit is de typische concentratie rond het midden die
kansberekening kenmerkt.

Ammon stelde dat deze verdeling overeenkomt met de verdeling van de
menselijke bekwaamheid. Hij schreef dit vóór de ontwikkeling van
intelligentietests en IQ's, en baseerde zich op het eerdere werk van
Francis Galton. Volgens Ammon hing maatschappelijk nut of succes niet
enkel af van intelligentie. Hij onderscheidde ``drie groepen van mentale
eigenschappen die grotendeels bepalend zijn voor de plaats die een mens
in het leven zal innemen.'' Deze waren:

\begin{enumerate}
\def\labelenumi{\arabic{enumi}.}
\tightlist
\item
  \emph{Intellectuele eigenschappen}; waaronder ik alles reken wat
  behoort tot de rationele kant van de mens: vlug begrip, geheugen,
  beoordelingsvermogen, vindingrijkheid en al wat verder tot dit gebied
  behoort.
\item
  \emph{Morele eigenschappen}; namelijk zelfbeheersing, wilskracht,
  ijver, volharding, matiging, familieplichtsgevoel, eerlijkheid en
  dergelijke.
\item
  \emph{Economische eigenschappen}; zoals zakelijk vermogen,
  organisatietalent, technische vaardigheid, voorzichtigheid, slimme
  berekening, vooruitziendheid, spaarzaamheid enzovoort.
\end{enumerate}

Aan deze geestelijke eigenschappen voegde hij toe:

\begin{enumerate}
\def\labelenumi{\arabic{enumi}.}
\setcounter{enumi}{3}
\tightlist
\item
  \emph{Lichamelijke eigenschappen}; arbeidskracht, uithoudingsvermogen,
  vermogen om inspanning te doorstaan en bestand te zijn tegen allerlei
  prikkels, kracht, goede gezondheid, enzovoort.
\end{enumerate}

Volgens Ammon was de verdeling van deze eigenschappen van intelligentie,
karakter, talent en lichaam vergelijkbaar met de dobbelstenen. Hij ging
verder en stelde dat er in werkelijkheid veel meer dan vier variabelen
waren, die bovendien meer dan zes gradaties kenden. Bij acht
dobbelstenen zijn er bijvoorbeeld 1.679.616 mogelijke worpen, maar de
hoogste score, achtenveertig, komt nog steeds slechts één keer voor. Een
man of vrouw die op alle bepalende factoren uitzonderlijk hoog scoort,
is dus veel zeldzamer dan een worp van vier zessen; wellicht zo zeldzaam
als acht zessen. Toch kan een mengeling van hoge en lage scores leiden
tot ``personen met onevenwichtige, disharmonische gaven, die ondanks het
bezit van briljante eigenschappen de beproevingen van het leven niet
aankunnen.''

\begin{quote}
Als een eenzame bergtop, of liever als de spits van een kathedraal,
rijzen de mannen van groot talent en genie boven de brede massa van
middelmatigheid uit\ldots{} Het aantal hoogbegaafden is in elk geval zo
klein dat het onmogelijk is dat `velen' van hen in lagere klassen werden
gehouden door onvolmaaktheden in sociale instituties. -- OTTO AMMON
\end{quote}

\subsection{Eigenschappen en inkomens}\label{eigenschappen-en-inkomens}

Ammon richt zich vervolgens op de inkomensverdeling. Hoewel de
statistieken uit de jaren 1890 natuurlijk veel minder betrouwbaar waren
dan die van nu, was de Duitse bureaucratie al sterk ontwikkeld; Otto
Ammon vond in Saksen, Pruisen, Baden en andere Duitse staten
inkomensverdelingen die, naar zijn idee, zowel de verdeling van
menselijke bekwaamheid als de dobbelsteenkansen weerspiegelden. Hij vond
soortgelijke cijfers in Charles Booths \emph{Life and Labour of the
People of London} (1892). Inderdaad, Booths sociale verdeling ziet eruit
zoals men op basis van Ammons waarschijnlijkheidstheorie zou verwachten.
Uit Booths onderzoek bleek dat in Londen 25 procent arm of nog slechter
af was, 51,5 procent die comfortabel leefde, en 15 procent die
welgesteld of beter was; neemt men de twee laagste categorieën van Booth
samen, dan komt dat neer op 9,5 procent. Vóór de komst van de
verzorgingsstaten in de twintigste eeuw was het gebruikelijk om over de
minstbedeelden te spreken als het ``ondergedompelde tiende.'' De twee
hoogste categorieën van Booth komen samen neer op 7 procent.

Uit dit alles trok Otto Ammon een aantal interessante conclusies. Hij
meende dat de bekwaamheid van mensen, ruim gedefinieerd, hun plaats in
de samenleving en hun inkomen bepaalden. Hij geloofde dat grote
bekwaamheid er vanzelf toe leidde dat mensen stegen in inkomen en
sociale positie. ``Als een eenzame bergtop, of liever, als de spits van
een kathedraal, rijzen de mannen van groot talent en genie boven de
brede massa van middelmatigheid uit\ldots{}'' Hij geloofde ook dat de
``ware vorm van de zogenaamde sociale piramide die van een enigszins
platte ui of knolraap is.'' Deze knolraap heeft een smalle stengel
bovenaan en een smalle wortel onderaan. Zo'n sociale knolraap is een
betere metafoor dan de sociale piramide, omdat zij, net als de moderne
industriële samenleving, haar massa in het midden heeft, terwijl de
piramide haar massa aan de onderkant heeft.

\subsection{De vorm van de raap}\label{de-vorm-van-de-raap}

Moderne industriële samenlevingen zijn inderdaad allemaal knollen, met
een kleine rijke, hogere professionele klasse aan de top, een grotere
middenklasse, en een arme minderheidsklasse aan de onderkant. In
verhouding tot het midden zijn beide uitersten klein. In het moderne
Londen, zo niet in Washington, zijn er zeker meer miljonairs dan
daklozen.

Dit alles is interessant, maar de directe relevantie van Ammons werk
ligt in de grote langetermijnverschuiving die we nu meemaken in de
financiële en politieke verhoudingen tussen de top en het midden. De
vaardigheden die nodig waren in het Fabriekstijdperk, dat nu ten einde
komt, zijn onmiskenbaar anders dan die worden vereist in het
Informatietijdperk. De meeste mensen konden de vaardigheden beheersen
die vereist waren om de machines van het midden van de twintigste eeuw
te bedienen, maar die banen zijn inmiddels vervangen door slimme
machines die zichzelf besturen. Een hoop werkgelegenheid van laag of
middelhoog vaardigheidsniveau is al verdwenen. Als wij gelijk hebben, is
dit een voorbode van het verdwijnen van het merendeel van de banen en de
herinrichting van werk via de spotmarkt.

\begin{quote}
Toch is het een feit dat officieel, maar stilzwijgend, wordt erkend: de
meeste werkloze jongeren hebben totaal geen kwalificaties\ldots{} --
CLIVE JENKINS EN BARRIE SHERMAN
\end{quote}

\section{Minder mensen zullen meer werk
verrichten}\label{minder-mensen-zullen-meer-werk-verrichten}

Laten we uitgaan van de simpele vierdobbelsteenverdeling van menselijke
bekwaamheid en veronderstellen dat men in het Fabriekstijdperk een score
kon halen van 4 × 2 of hoger. Dat zou betekenen dat meer dan 95 procent
van de bevolking boven wat Charles Booth ``de laagste grens van
positieve sociale bruikbaarheid'' noemde, zou zitten. In de jaren 1940
en 1950 werd 3 procent vastgesteld als de norm voor volledige
werkgelegenheid. Stel dat in het Informatietijdperk de vereiste score is
gestegen naar een 4 x 3, en dat de vereiste minimumscore is opgetrokken
van 8 naar 12. Dat zou betekenen dat bijna 24 procent onder deze grens
van ``sociale bruikbaarheid'' zou vallen.

Iets soortgelijks zou gebeuren aan de bovenkant van de schaal. In het
Fabriekstijdperk was het vereiste niveau voor een hoge bekwaamheid
misschien 4 x 4. Stel dat dit in het Informatietijdperk zou stijgen naar
4 x 5, dan zou het aandeel van mensen dat in aanmerking komt voor de
topbanen, die ook het best betaald zijn, dalen van 34 procent naar 5
procent.

Deze cijfers zijn puur hypothetisch. Uiteraard weten we niet wat de
verschuiving in de bekwaamheidseisen zal zijn of al is geweest, maar ze
zijn zeker gestegen. Door de vorm van de knol zou een vrij bescheiden
stijging in de minimale bekwaamheidseis grote aantallen mensen buiten
een betekenisvolle economische rol plaatsen. Evenzo zou een vrij kleine
stijging in de hogere bekwaamheidseisen het aantal mensen dat in
aanmerking komt voor de hogere banen zeer sterk verminderen. Er vindt
een verschuiving plaats: we weten alleen nog niet hoe groot die zal
zijn.

Er is inderdaad geen gebrek aan sociaal en politiek bewijs dat deze
verschuiving plaatsvindt in alle geavanceerde industriële samenlevingen,
dat het tempo ervan toeneemt, en dat de beweging al groot is. De
beloningen voor zeldzame vaardigheden zijn gestegen en stijgen nog
steeds. Dit is met ongenoegen opgemerkt door conventionele denkers. Neem
bijvoorbeeld \emph{The Winner-Take-All Society van} Robert H. Frank en
Philip J. Cook. Het documenteert de groeiende tendens dat de meest
getalenteerde concurrenten in veel sectoren in de Verenigde Staten zeer
hoge inkomens verdienen. De kansen voor middelmatige bekwaamheid nemen
daarnaast af; een aanzienlijk aantal laagwaardige vaardigheden valt nu
buiten het niveau dat een comfortabel inkomen biedt, al kunnen ze nog
steeds een plek vinden in de kleinschalige dienstverlening.

Als het Informatietijdperk hogere bekwaamheid vereist, zowel aan de
boven- als onderkant, zal iedereen behalve de bovenste 5 procent
relatief in het nadeel zijn, maar de bovenste 5 procent zal enorm
winnen. Zij zullen zowel een groter aandeel van het inkomen verdienen
als een groter aandeel behouden van wat zij verdienen. Tegelijkertijd
zullen ze een groter deel van het werk in de wereld doen dan ooit
tevoren. Velen zullen opkomen als Soevereine Individuen. In het
Informatietijdperk zal de knolraap van de inkomensverdeling er meer
uitzien zoals in 1750 dan zoals in 1950.

Samenlevingen die zijn geïndoctrineerd met de verwachting van
inkomensgelijkheid en hoge consumptieniveaus voor mensen met lage of
bescheiden bekwaamheden zullen demotivatie en onzekerheid ervaren.
Naarmate meer landen de informatietechnologie dieper integreren in de
economie, zullen ze de opkomst zien, zoals nu al duidelijk zichtbaar is
in Noord-Amerika, van een min of meer onbruikbare onderklasse. Dit is
precies wat er gebeurt. Dit zal leiden tot een reactie met een
nationalistische, antitechnologische inslag, zoals we in het volgende
hoofdstuk uiteenzetten.

Het Fabriekstijdperk kan wel eens een unieke periode zijn geweest waarin
halfdomme machines een zeer winstgevende niche aan ongeschoolde mensen
verschaften. Nu machines zelfstandig functioneren, komen de vruchten van
het Informatietijdperk terecht bij de top 5 procent van Otto Ammon's
knol. Het Informatietijdperk zag er al veel beter uit voor de bovenste
10 procent, de zogenaamde cognitieve elite. Het zal echter het meest
gunstig zijn voor de bovenste 10 procent van de bovenste 10 procent, de
cognitieve dubbele top. In het feodale tijdperk waren er honderd
halfgeschoolde boeren nodig om één hooggeschoolde krijgsheer (of ridder)
te onderhouden. De Soevereine Individuen van de Informatiemaatschappij
zullen geen krijgsheren zijn, maar meesters van gespecialiseerde
vaardigheden, waaronder ondernemerschap en investeringen. Toch lijkt de
feodale verhouding van honderd tegen één terug te keren. Hoe dan ook, de
samenlevingen van de eenentwintigste eeuw zullen waarschijnlijk
ongelijker zijn dan die waarin we in de twintigste eeuw hebben geleefd.

\section{De meeste mensen zullen profiteren van de dood van de
politiek}\label{de-meeste-mensen-zullen-profiteren-van-de-dood-van-de-politiek}

Het is onwaarschijnlijk dat de egalitaire economie, en daarmee de naties
die haar ondersteunen, zonder crisis zal verdwijnen. Hoewel een crisis
van nature kort duurt, verwachten we desalniettemin dat het trauma dat
gepaard gaat met het verdwijnen van naties nog jarenlang zal voortduren.
Zonder dit trauma, waarvan we de omvang later uitgebreid behandelen, uit
het oog verliezen, mogen we niet vergeten dat de overgang naar de
informatie-economie in veel delen van de wereld tot een enorme
productieboost zal leiden, met hogere inkomens voor iedereen. Inderdaad,
in regio's die nooit ten volle hebben kunnen profiteren van de voordelen
van het industrialisme maar nu de vrije markt omarmen, stijgen de
inkomens in alle lagen van de bevolking al, of zullen gaan stijgen.

Het afnemen van dwang als kenmerk van het economische leven zal
producenten in staat stellen de activa te behouden die voorheen werden
onteigend en herverdeeld. Herverdeling betekende meestal dat activa
werden ingezet voor minder waardevolle doeleinden, waardoor de
productiviteit van kapitaal afnam. Vermogen werd disproportioneel
ontnomen van de personen die het meest bedreven waren in het investeren
van middelen en door politici herverdeeld aan degenen die minder
bedreven waren. In de meeste gevallen werd het herverdeeld inkomen
ingezet in economische activiteit van lagere waarde. Het bevrijden van
middelen van deze systematische dwang zal afhankelijk van de jurisdictie
sterk andere gevolgen hebben. Het zal het faillissement van de
verzorgingsstaten betekenen, en de schaalnadelen versterken die grote
overheden en alle door hen gesubsidieerde instellingen ondermijnen. Aan
de andere kant zal de overgang naar de cybereconomie de economische
nadelen verminderen voor mensen die opereren onder soevereiniteiten in
regio's die traditioneel gezien moeite hadden met grootschalige
organisatie.

\begin{quote}
Als de wereld als één grote markt opereert, zal iedere werknemer
concurreren met iedere persoon in de wereld die hetzelfde werk kan doen.
Er zijn er veel en velen hebben honger. -- ANDREW S. GROVE, PRESIDENT,
INTEL CORP.
\end{quote}

\section{Verschuivende
locatievoordelen}\label{verschuivende-locatievoordelen}

Doordat de opbrengsten van geweld niet langer stijgen, verdwijnt ook het
nut om onder een overheid te leven die ze zou kunnen opeisen. Ooit
bekwame overheden zullen niet langer vermogensopbouw bevorderen, maar er
juist vijandig tegenover staan. Hoge belastingen, forse
reguleringskosten en ambitieuze verplichtingen tot inkomensherverdeling
maken de gebieden onder hun controle onaantrekkelijk voor
ondernemerschap.

Degenen die wonen in jurisdicties die arm of onderontwikkeld bleven
tijdens de industriële periode hebben het meest te winnen bij de
bevrijding van de geografische beperkingen op de economie. Dit druist in
tegen wat je vaak hoort. Het grootste debat rond de opkomst van de
informatie-economie en het Soevereine Individu zal draaien om de
vermeende negatieve effecten op de ``eerlijkheid'' die voortkomen uit de
dood van de politiek. Het is waar dat de wereldwijde informatie-economie
een dodelijke klap zal toebrengen aan grootschalige
inkomensherverdeling. De belangrijkste begunstigden van
inkomensherverdeling in het Industriële Tijdperk waren inwoners van
rijke jurisdicties, waar het consumptieniveau twintig keer hoger lag dan
het wereldgemiddelde. Alleen binnen de OESO-landen verhoogde
inkomensherverdeling merkbaar de inkomens van ongeschoolden.

De grootste inkomensongelijkheid is echter altijd waargenomen tussen
jurisdicties. Herverdeling heeft daar weinig aan veranderd. Sterker nog,
buitenlandse hulp en internationale ontwikkelingsprogramma's hebben
volgens deze visie vaak een pervers effect gehad: ze verlaagden de reële
inkomens van arme mensen in arme landen door incompetente regeringen in
stand te houden. Dit thema wordt later verder uitgediept bij de morele
implicaties van de Informatierevolutie.

\subsection{Een eeuw van stijgende
inkomensongelijkheid}\label{een-eeuw-van-stijgende-inkomensongelijkheid}

In de industriële periode bepaalde het rechtsgebied waarin iemand woonde
grotendeels zijn levenslange inkomen. In tegenstelling tot de gangbare
indruk in de welvarende economieën van tegenwoordig, nam de
inkomensongelijkheid in die tijd sterk toe. Volgens een schatting van de
Wereldbank lag het gemiddelde inkomen per inwoner in de rijkste landen
in 1870 elf maal hoger dan in de armste landen, en in 1985 was dit
gestegen tot tweeënvijftig maal. Hoewel de wereldwijde ongelijkheid
drastisch toenam, merkten de inwoners van de rijke industriële landen
dat nauwelijks op: de inkomensverschillen namen vooral toe tussen
rechtsgebieden en niet binnen de rechtsgebieden.

Zoals we eerder al bespraken, zorgde de aard van de industriële
technologie ervoor dat de inkomenskloof afnam in jurisdicties waar
capabele overheden op grote schaal de macht uitoefenden. Naarmate de
effecten van geweld toenamen, zoals tijdens het Industriële Tijdperk
gebeurde, werden overheden die op grote schaal opereerden vaak door hun
werknemers bestuurd. Daardoor bleek het praktisch onmogelijk om de
aanspraken, die deze overheden op middelen maakten, in te perken. Hun
onbelemmerde controle over middelen leverde een aanzienlijk militair
voordeel op, zolang de omvang van hun macht zwaarder woog dan de
efficiëntie om deze te benutten. Niet toevallig versnelde de
inkomensherverdeling sterk dankzij het overheidsbestuur door haar
medewerkers. Bijna iedere samenleving kent wel een vorm van
inkomensherverdeling, al is dit soms slechts tijdelijk in uitzonderlijke
omstandigheden. Als je de geschiedenis van steun aan de armen goed
bestudeert, blijkt echter dat uitkeringen meestal het gulst zijn in
tijden dat er weinig armoede heerst. Een beperking van de herverdeling
van inkomens komt eerder voor wanneer inkomens voor grote groepen mensen
dalen. In de welvarende industriële samenlevingen van de tweede helft
van de twintigste eeuw waren de omstandigheden vrijwel ideaal voor
inkomensherverdeling. Dat resulteerde in veel hogere beloningen voor
laaggeschoolde arbeid in deze bevoorrechte rechtsgebieden, waardoor
zelfs mensen die helemaal niet werkten van een hoog consumptieniveau
konden genieten.

\subsection{De paradox van industriële
rijkdom}\label{de-paradox-van-industriuxeble-rijkdom}

De ironie is dat juist in deze jurisdicties ook meer mensen rijk werden.
Deze schijnbare paradox wordt begrijpelijk zodra je de dynamiek van
megapolitiek uit eerdere hoofdstukken begrijpt. De leidende sectoren van
de industriële economie vereisten grootschalige ordehandhaving om
optimaal te functioneren. Dit maakte ze bijzonder kwetsbaar voor
chantage door vakbonden en overheden die hun invloed zo veel mogelijk
wilden uitbreiden. Toch verstikte de brede inkomensherverdeling het
functioneren van de industriële economie niet volledig. Wie tijdens de
bloeiperiode van het industrialisme het geluk had geboren te worden in
West-Europa, de voormalige Britse koloniën of Japan, had daardoor
waarschijnlijk een veel hoger inkomen dan iemand met gelijkwaardige
vaardigheden in Zuid-Amerika, Oost-Europa, de late Sovjet-Unie, Afrika
of Azië. De gunstige impact van informatietechnologie zal onder meer
bestaan uit het wegnemen van veel obstakels die het grootste deel van de
wereldbevolking hebben verhinderd om tijdens een groot deel van de
moderne periode te ontwikkelen en te profiteren van de voordelen van
vrije markten.

\begin{quote}
De inheemse kenmerken van arme landen zijn opvallend ongeschikt voor
effectieve grootschalige organisatie, vooral voor grootschalige
organisaties die (zoals overheden) over een groot geografisch gebied
moeten opereren. -- MANCUR OLSON
\end{quote}

\section{\texorpdfstring{\textbf{Schaalnadelen en vertraagde
groei}}{Schaalnadelen en vertraagde groei}}\label{schaalnadelen-en-vertraagde-groei}

Zoals Mancur Olson heeft aangetoond, lag de achterstand in de twintigste
eeuw niet aan een gebrek aan kapitaal of gespecialiseerde vaardigheden
op zich. In \emph{Diseconomies of Scale and Development}, een essay uit
1987, twee jaar vóór de val van de Berlijnse Muur, schreef Olson:

\begin{quote}
``Als kapitaal in arme landen daadwerkelijk schaars zou zijn geweest,
dan had de `marginale productiviteit' ervan, en dus de winstgevendheid
van het gebruik, groter moeten zijn dan in de welvarende landen. De lage
groeicijfers van veel landen die significante hoeveelheden buitenlandse
hulp ontvingen en de lage productiviteit van sommige moderne fabrieken
die in arme landen werden gebouwd, hebben de overtuigingskracht van de
these van `kapitaalschaarste' als oorzaak van onderontwikkeling verder
verzwakt.''
\end{quote}

Dit kan niet anders dan kloppen. Als een tekort aan kapitaal of
vaardigheden het belangrijkste probleem zou zijn geweest, zouden de
opbrengsten van beide hoger zijn geweest in arme jurisdicties dan in
ontwikkelde landen. Zowel geschoold personeel als kapitaal zouden
massaal naar deze regio's zijn gestroomd totdat de opbrengsten gelijk
zouden zijn geworden. In werkelijkheid gebeurde vaak het
tegenovergestelde: er vond een aanzienlijke emigratie plaats van
opgeleide mensen uit achtergebleven jurisdicties. De weinigen die er wel
in slaagden om kapitaal op te bouwen, brachten dat zo snel mogelijk
onder in Zwitserland en andere ontwikkelde landen.

\subsection{Beter bestuur kon niet geïmporteerd
worden}\label{beter-bestuur-kon-niet-geuxefmporteerd-worden}

Olson stelt, en wij zijn het daarmee eens, dat het werkelijke obstakel
voor ontwikkeling in achtergebleven landen de enige productiefactor was
die niet gemakkelijk geleend of geïmporteerd kon worden: de overheid.
Dit probleem werd groter naarmate de twintigste eeuw vorderde. In 1900
hielden Groot-Brittannië en Frankrijk, samen met enkele andere Europese
landen, zich bezig met het exporteren van competente bestuurssystemen
naar regio's waar lokale machten niet in staat waren om succesvol op
grote schaal te functioneren. Veranderende megapolitieke omstandigheden
in de twintigste eeuw verhoogden echter de kosten en verlaagden de
opbrengsten van deze activiteit. Kolonialisme, of imperialisme, zoals
het minder vriendelijk werd genoemd, was geen rendabele onderneming
meer. Technologische verschuivingen verhoogden de kosten van
machtsprojectie van het centrum naar de periferie en verlaagden de
militaire kosten van effectieve weerstand. Als gevolg trokken imperiale
machten zich terug of bleven slechts aanwezig in kleine enclaves zoals
Bermuda of de Kaaimaneilanden.

\begin{quote}
Als de postkoloniale natiestaat een belemmering voor vooruitgang was
geworden, zoals tegen het einde van de jaren tachtig steeds meer critici
in Afrika leken te erkennen, dan kon er weinig twijfel bestaan over de
voornaamste reden. De staat bevrijdde en beschermde zijn burgers niet,
ongeacht wat de propaganda claimde. Integendeel, het netto-effect was
beperkend en uitbuitend, anders functioneerde de staat eenvoudigweg
helemaal niet in sociale zin. -- BASIL DAVIDSON
\end{quote}

De inheemse overheden die het koloniale bestuur vervingen in landen waar
Europeanen zich niet hadden gevestigd, rekruteerden hun leiders en
bestuurders uit bevolkingsgroepen met weinig ervaring of bekwaamheid in
het leiden van grootschalige ondernemingen. In veel gevallen, vooral in
Afrika, werd de infrastructuur die door de vertrekkende koloniale
machten was achtergelaten, snel geplunderd, vernietigd of verwaarloosd.
Telefoonlijnen werden neergehaald en omgesmeed tot armbanden, wegen
werden niet meer onderhouden, spoorlijnen raakten onbruikbaar doordat
wegbedden instortten en locomotieven stuk gingen. In Zaïre was de
uitgebreide transportinfrastructuur die door de Belgen was aangelegd
tegen 1990 vrijwel volledig verdwenen. Alleen een paar gammele
rivierboten functioneerden nog, waarvan er één door de dictator werd
overgenomen als een soort drijvend paleis.

Onbetrouwbare communicatie- en transportinfrastructuur weerspiegelen de
incompetentie van achtergebleven natiestaten in het handhaven van orde.
Ze hebben prijzen hoog gehouden en kansen voor het grootste deel van de
wereldbevolking geminimaliseerd. Zoals Olson benadrukt:

\begin{quote}
``Ten eerste dwingen slechte transport- en communicatiesystemen een
bedrijf om vooral te vertrouwen op lokale productiefactoren. Naarmate
een bedrijf opschaalt, moet het verder weg zoeken naar
productiefactoren, en hoe slechter de transport- en de
communicatie-infrastructuur, hoe sneller deze kosten zullen stijgen bij
een groeiende output. Het tweede en belangrijkere punt waarom slechte
transport- en communicatiesystemen nadelig zijn voor effectieve
grootschalige ondernemingen, is dat ze het veel moeilijker maken om
zulke ondernemingen effectief te coördineren.''
\end{quote}

\subsection{Het verlichten van de last van slecht
bestuur}\label{het-verlichten-van-de-last-van-slecht-bestuur}

De ambitieuze armen in de wereld hebben, meer dan wie dan ook, baat bij
het feit dat informatietechnologie het vermogen om inkomen te verdienen
loskoppelt van de plek waar men woont. Nieuwe technologieën, zoals de
digitale mobiele telefoon, maken communicatie mogelijk onafhankelijk van
het vermogen van de lokale politie om elke telefoonpaal te beschermen
tegen koperdieven. Naarmate draadloze fax- en internetverbindingen
beschikbaar komen, wordt het minder relevant of wanhopig arme
postmedewerkers de post achterhouden om de postzegel te stelen.

In veel gevallen vervangt effectieve communicatie zelfs de noodzaak van
fysiek transport van goederen en diensten. Betere communicatie en sterk
toegenomen rekenkracht maken niet alleen de coördinatie van complexe
activiteiten goedkoper en efficiënter, ze verlagen ook schaalvoordelen
en ontbinden grote organisaties. Al deze veranderingen verminderen de
nadelen die mensen in achtergebleven landen hebben ondervonden door
incompetente overheden. De Informatierevolutie zal het belang van
capabele overheden flink doen afnemen. Hierdoor wordt het voor mensen in
traditioneel arme landen gemakkelijker om de obstakels te overwinnen die
hun overheden tot nu toe in de weg van economische groei hebben
geplaatst.

\subsection{Gelijke kansen in het
Informatietijdperk}\label{gelijke-kansen-in-het-informatietijdperk}

In het Informatietijdperk zullen bekende locatievoordelen snel door
technologie worden getransformeerd. Het verdienvermogen van mensen met
vergelijkbare vaardigheden zal veel gelijker worden, ongeacht de
jurisdictie waar ze in wonen. Dit proces is al begonnen. Omdat
instellingen die dwang en lokale voordelen gebruikten om inkomen te
herverdelen macht verliezen, zal de inkomensongelijkheid binnen
jurisdicties toenemen. Tegelijk zal wereldwijde concurrentie de
inkomsten van de meest getalenteerde individuen in elk vakgebied
verhogen, waar ze ook wonen, vergelijkbaar met wat nu gebeurt in de
professionele sport. De marginale waarde van superieure prestaties op
een wereldmarkt zal enorm zijn.

Hoewel het publieke debat zal focussen op groeiende ``ongelijkheid'' in
de OESO-landen, zullen individuen overal veel gelijkere kansen hebben.
Ze hoeven om succes te hebben niet langer in een jurisdictie te wonen
die goed functioneert op grote schaal. Aangeboren talenten en de
bereidheid om deze te ontwikkelen, zullen op een eerlijker speelveld dan
ooit tevoren worden gemeten. Jurisdictionele voordelen die tijdens het
Industriële Tijdperk leidden tot toenemende ongelijkheid tussen rijke en
arme economieën, zullen ingrijpend veranderen.

\subsection{Hogere rendementen in arme
gebieden}\label{hogere-rendementen-in-arme-gebieden}

De belemmeringen op het functioneren van vrije markten, die overheden in
armere regio's opwerpen, zullen sterk afnemen naarmate de cybereconomie
op gang komt. ALs gevolg zullen schaars kapitaal en vaardigheden in veel
van de huidige arme regio's hogere rendementen opleveren, precies zoals
ontwikkelingsdeskundigen in de jaren vijftig voorspelden. Bovendien
zullen kapitaal en vaardigheden veel makkelijker worden geïmporteerd.
Opkomende economieën zullen niet langer zo sterk als in het Industriële
Tijdperk afhankelijk zijn van lokale productiefactoren. Hun verbeterde
vermogen om op afstand kapitaal en expertise aan te boren, zal leiden
tot hogere groeicijfers. Dit zal gebeuren, of incompetente overheden
eerlijker worden of beter in staat zijn eigendomsrechten te beschermen,
of niet. Zonder controle over de cyberspace zullen slechte overheden
simpelweg minder in staat zijn om mensen binnen hun jurisdicties te
beletten te profiteren van economische vrijheid.

\subsection{Positieve versterking}\label{positieve-versterking}

In de nieuwe cybereconomie zal de vrijwel totale draagbaarheid van
informatietechnologie het hamsteren van veel van de voordelen van
jurisdicties uit het Industriële Tijdperk verhinderen. Verhoogde
concurrentie tussen een groeiend aantal jurisdicties zal afhangen van
nieuwe soorten lokale voordelen. Soevereiniteit zal commercieel worden
in plaats van roofzuchtig. Overheden zullen door de kracht van
concurrentie verplicht zijn om beleid te voeren dat aantrekkelijk is
voor degenen die de grootste bijdrage leveren aan het economische
welzijn, niet voor degenen die weinig bijdragen of wiens economische
bijdrage negatief is.

Dit betekent een enorme verandering ten opzichte van de gangbare
praktijk in de twintigste eeuw. De ideologie van de natiestaat was dat
het leven positief gereguleerd kan en moet worden door het subsidiëren
van ongewenste uitkomsten en het bestraffen van de wenselijke. Arm zijn
is ongewenst, daarom werden de armen gesubsidieerd. Rijk worden is
wenselijk, daarom werden strenge belastingen geheven op de rijken, zodat
het leven ``eerlijker'' werd.

Omdat deze hele beleidsaanpak geworteld was in een megapolitieke basis
die aan elke kritiek weerstand bood, deed het weinig ter zake wat de
perverse gevolgen waren van het subsidiëren van disfunctioneren. Er werd
ook nauwelijks rekening gehouden met de vaardigheden, het harde werk of
de vindingrijkheid die nodig waren om het herverdeelde vermogen in
eerste instantie te verwerven. Uitkomsten werden beoordeeld op basis van
aanspraken. Volgens het twintigste-eeuwse politieke denken moesten
uitkomsten gelijk zijn om als ``rechtvaardig'' te gelden.

\subsection{Het nieuwe paradigma}\label{het-nieuwe-paradigma}

De nieuwe megapolitieke omstandigheden van de eenentwintigste eeuw
zullen het mogelijk maken dat markttesten de uitkomsten reguleren in
gebieden die voorheen door de politiek werden gedomineerd. Het
marktparadigma veronderstelt dat resultaten beter kunnen worden
gereguleerd door wenselijke uitkomsten te belonen en ongewenste te
bestraffen. Arm zijn is ongewenst, rijk worden is wenselijk. Daarom
zouden prikkels rijkdom moeten belonen en mensen aanmoedigen te betalen
voor de middelen die ze gebruiken. Het leven is ``eerlijker'' wanneer
mensen meer van hun verdiende inkomen mogen behouden.

Deze visie zal in het nieuwe millennium vaker gehoord worden dan in de
eeuw die nu ten einde komt. Bovendien zal het overtuigender zijn dan
ooit, omdat het een megapolitieke basis heeft. Kapitaal in het
Informatietijdperk wordt met de dag mobieler. Het vermogen om hoge
inkomens te verdienen is niet langer gebonden aan het verblijf op
specifieke locaties, zoals vroeger het geval was toen het meeste
vermogen werd gecreëerd door de exploitatie van natuurlijke hulpbronnen.
Elke dag wordt het makkelijker voor mensen die gebruikmaken van
draagbare informatietechnologie om activa te creëren die veel minder
onderhevig zijn aan dwang dan enig ander type van vermogen ooit was.

Arbitraire politieke regels die kosten opleggen zonder daar
marktvoordelen tegenover te stellen, zullen binnenkort onhoudbaar zijn.
Krachtige concurrentiedruk heeft de neiging om prijzen van goederen,
diensten, arbeid en kapitaal wereldwijd gelijk te trekken. Overheden
zullen minder speelruimte hebben dan ze gewend zijn om willekeurige
beleidsmaatregelen op te leggen. Elke overheid die probeert om zwaardere
regels op te leggen dan andere soevereinen, zal die activiteit
eenvoudigweg verdrijven. In sommige gevallen zal het verdrijven van
ongewenste activiteiten de markt echter juist tevreden stellen en die
jurisdicties populairder en welvarender maken. In dit opzicht kunnen
bepaalde regels worden vergeleken met de huisregels die eigenaren van
een hotelketen opleggen. Als ze verbieden om op blote voeten te lopen of
in de lobby te roken, zullen ze ongetwijfeld sommige klanten verliezen.
Maar het weren van die klanten hoeft de jurisdictie als geheel geen
inkomsten te kosten. Niet-rokers met goed schoeisel betalen mogelijk
zelfs meer omdat rokers op blote voeten worden uitgesloten. Op dezelfde
manier kunnen regels die het kostbaar of onmogelijk maken om een
verwerkingsinstallatie in een bepaalde jurisdictie te exploiteren, de
activiteit elders brengen zonder de jurisdictie als geheel van inkomen
te beroven.

Deze voorbeelden laten zien dat regels in zeldzame gevallen een
positieve in plaats van een negatieve marktwaarde kunnen hebben, vooral
in een wereld met een snel toenemend aantal jurisdicties. Regels die
hoge normen voor volksgezondheid, schone lucht en schoon water
handhaven, zullen in veel gebieden zeer gewaardeerd worden. Zo zullen
andere, soms meer exotische regels en convenanten, zoals opgelegd door
vastgoedontwikkelaars of hotels die zich op bepaalde marktsegmenten
richten, ook waardevol blijken.

\subsection{Geen douanehuis in
cyberspace}\label{geen-douanehuis-in-cyberspace}

We verwachten dat de commercialisering van soevereiniteit snel zal
leiden tot de devolutie van veel grote territoriale soevereinen. Het
feit dat informatietechnologie niet onderworpen kan worden aan
grenscontroles zoals bij de handel in industriële en landbouwproducten
nog wel het geval is, heeft belangrijke implicaties. Het betekent dat
protectionisme na verloop van tijd minder effectief zal zijn naarmate
handel in informatie de plaats van fysieke producten inneemt bij het
genereren van vermogen. Het betekent ook dat kleinere regio's, voor de
toegang tot markten, steeds minder afhankelijk zullen zijn van het in
stand houden van uitgebreide politieke jurisdicties.

Informatietechnologie stelt mensen die werkzaam zijn in voorheen
beschermde dienstensectoren bloot aan buitenlandse concurrentie. Als een
bedrijf in Toronto twintig jaar geleden een boekhouder wilde aannemen,
moest die persoon fysiek in Toronto of een nabijgelegen woonplaats
binnen woon-werkafstand zijn. In het Informatietijdperk kan een
boekhouder in Boedapest of Bangalore, India, het werk doen en al het
benodigde materiaal versleuteld via het Internet downloaden. Directe
communicatie via satellietverbindingen maakt elk deel van de wereld
slechts een moment verwijderd via modem en fax. Iemand die behoefte
heeft aan aandelenanalisten, kan er zevenentwintig in India aannemen
voor de prijs van één op Wall Street. Naarmate informatietechnologie
elke achttien maanden een orde van grootte verbetert (Wet van Moore),
zullen steeds meer dienstverleners worden blootgesteld aan
prijsconcurrentie die politici in wezen onmogelijk kunnen belemmeren.
Deze concurrentie zal uiteindelijk even volledig van toepassing zijn op
de hogere beroepsgroepen als op boekhouders. Digitale advocaten en
cyberartsen zullen zich vermenigvuldigen in de Informatie-economie.

\subsection{Doodvonnis voor
natiestaten}\label{doodvonnis-voor-natiestaten}

Naarmate de economische voordelen die voorheen binnen de grenzen van
natiestaten werden vastgehouden verdwijnen, zullen de natiestaten zelf
uiteindelijk instorten onder hun zware verplichtingen. Het feit dat alle
natiestaten op een doodvonnis staan, betekent echter niet dat ze
allemaal tegelijk zullen verdwijnen. Integendeel. Devolutionaire druk
zal het sterkst zijn in grote politieke entiteiten waar het inkomen van
de meeste mensen stagneert of daalt. Jurisdicties in Latijns-Amerika en
Azië waar het inkomen per hoofd van de bevolking snel stijgt, kunnen
generaties lang standhouden, of totdat het levenslange
inkomensperspectief daar gelijk is aan dat in de voorheen rijke
industriële landen. Op dat moment zullen er geen eenvoudige winsten door
kostenbesparing meer te behalen zijn, en wordt de politiek van groei
uitdagender.

We vermoeden ook dat natiestaten met één grote metropool langer coherent
blijven dan staten met meerdere grote steden, en dus meerdere
machtscentra en hun achterlanden.

Een andere stimulans voor devolutie zal de hoge schuldenlast van de
centrale overheid zijn. De drie rijke industriële landen met de hoogste
relatieve schuldenlast, Canada, België en Italië, hebben niet toevallig
ook geavanceerde separatistische bewegingen. Alle drie hebben te maken
gehad met chronische begrotingstekorten en hebben nu nationale schulden
die boven 100 procent van het BBP uitstijgen. Naarmate de nationale
schuld in elk land toenam, groeide ook de aantrekkingskracht van
separatistische bewegingen. In Italië is de Lega Nord opgekomen als een
dynamische en populaire regionale politieke beweging, gebaseerd op een
eenvoudige wiskundige observatie: Noord-Italië, of ``Padania'', zou
rijker zijn dan Zwitserland als grote delen van het inkomen niet naar
Rome en het armere zuiden zouden worden afgetapt. De Lega Nord stelt een
voor de hand liggende oplossing voor: afscheiden van Italië en zo
ontsnappen aan sommige van de ernstige gevolgen van samengestelde rente.
Zo manoeuvreren in België, waar de nationale schuld boven 130 procent
van het BBP ligt, ook de Vlamingen en Walen als een vijandig echtpaar
voor een scheiding. Een groeiende minderheid van de Vlamingen stelt dat
zij de Walen oneerlijk subsidiëren en hun economische situatie kunnen
verbeteren door België in tweeën te splitsen.

In Canada verschilt de situatie doordat Frans Canada, de belangrijkste
regio die nu separatistische ideeën aanhangt, historisch gesubsidieerd
werd door Engels Canada. Maar naarmate de federale schuld en het tekort
toenemen, dringt het besef door in Quebec dat deze vorm van
inkomensherverdeling zal afnemen. Het Bloc Québécois flirt daarom met
een aantrekkingskracht die het tien jaar geleden nog niet had: de
belofte om het inkomen na belasting te verhogen door de betaling van
federale belastingen af te schaffen. Separatistische leiders suggereren
ook dat Quebec Canada zou moeten verlaten zonder een evenredig deel van
de federale schuld te dragen.

Engels-Canadezen verzetten zich tegen dit argument en zijn het vaak
oneens met de implicaties, omdat ze zich bewust zijn van de grote
transfers naar Quebec door de jaren heen. Desondanks is de
aantrekkingskracht van het Parti Québécois sterk, en het lijkt slechts
een kwestie van tijd voordat een afscheidingsreferendum Canada doet
uiteenvallen. Een soortgelijk lot wacht andere natiestaten wanneer hun
financiële situatie verslechtert.

Een andere nadelige factor voor het overleven van Canada op de
lange-termijn is dat het een dunbevolkt land is met een omvangrijke
industriële infrastructuur die onderhouden moet worden. De overgang naar
het Informatietijdperk ontwaardt fysieke infrastructuur onvermijdelijk.
Naarmate telewerkers fabrieks- en kantoormedewerkers vervangen, wordt
het minder belangrijk dat snelwegen en andere transportwegen opnieuw
worden aangelegd en goed onderhouden. Met fiscale crises aan alle kanten
zullen steeds meer groepen in Canada teruggrijpen naar het
achttiende-eeuwse perspectief op de financiering van openbare goederen
zoals bepleit door Adam Smith. Hij schreef in \emph{The Wealth of
Nations}:

\begin{quote}
``Stel dat de straten van Londen verlicht en geplaveid zouden worden ten
laste van de {[}nationale{]} schatkist, is er dan enige kans dat ze even
goed verlicht en geplaveid zouden zijn als nu, of zelfs tegen zo'n lage
kosten? De kosten zouden bovendien, in plaats van te worden geheven via
een lokale belasting van de bewoners van elke specifieke straat,
parochie of wijk in Londen, in dit geval worden betaald uit de algemene
staatsinkomsten, en zouden dus worden geheven via een belasting op alle
inwoners van het koninkrijk, van wie het grootste deel geen enkel
voordeel trok uit het verlichten en verharden van de straten van
Londen.''
\end{quote}

Vervang London door Toronto, en je zit midden in een redenering die
velen in Alberta en British Columbia zullen maken. De logica van
devolutie zal besmettelijk blijken.

Wanneer Canada uiteenvalt, zal dit leiden tot een duidelijke toename van
afscheidingsbewegingen in het noordwesten van de Verenigde Staten.
Inwoners van Alaska, Washington, Oregon, Idaho en Montana zouden een
duidelijk nadeel ervaren in concurrentie met Alberta en British Columbia
als onafhankelijke soevereiniteiten.

\section{Na de natiestaat}\label{na-de-natiestaat}

In plaats van natiestaten zul je aanvankelijk kleinere rechtsgebieden op
provinciaal niveau zien en uiteindelijk nog kleinere soevereiniteiten,
enclaves van diverse aard die doen denken aan middeleeuwse stadstaten,
omgeven door hun achterland. Hoewel dit vreemd kan overkomen op mensen
die van jongs af aan geleerd hebben dat politiek allesbepalend is,
kiezen de leiders van deze nieuwe ministaten er in veel gevallen voor om
hun beleid te laten vormgeven door ondernemerschap en strategische
positionering, in plaats van door politieke conflicten. Deze
gefragmenteerde soevereiniteiten bedienen een breed scala aan smaken,
net zoals hotels en restaurants dat doen, door in hun openbare ruimten
specifieke regels vast te stellen die naadloos aansluiten bij de wensen
van de marktsegmenten waaruit zij hun klanten halen. Dat betekent echter
niet dat er geen bijzondere uitdagingen ontstaan bij het organiseren van
bescherming op nomadische basis. Die pakken we in het volgende hoofdstuk
aan.

\begin{quote}
Dorpslucht brengt vrijheid. -- MIDDELEEUWSE GEWOONHEID
\end{quote}

\subsection{Niet-burgers van de Pale}\label{niet-burgers-van-de-pale}

Ondanks deze moeilijkheden vindt menselijke vindingrijkheid meestal een
manier om instellingen te creëren die winstgevende kansen benutten,
zelfs wanneer de vraag van personen die weinig kunnen betalen, komt.
Waar de potentiële klanten behoren tot de rijkste mensen op aarde, moet
die neiging des te sterker zijn. Een exit, of ``stemmen met de voeten,''
is altijd een optie wanneer verouderde producten, organisaties of zelfs
overheden hun aantrekkingskracht verliezen en weinig vooruitzicht bieden
op directe verbetering. Denk bijvoorbeeld aan de groei van middeleeuwse
die dienden als veilige toevluchtsoorden voor horigen die aan feodale
onderdrukking ontsnapten. Zij zouden een treffende analogie kunnen
blijken voor de rol die nieuwe jurisdicties zullen spelen bij het
faciliteren van de exit uit natiestaten. De acceptatie van vreemdelingen
die aan een heer ontkwamen als ``burgers van de Pale'' tartte de
geldende conventies van het feodale recht en de bisschoppelijke macht.
Toch was het over het algemeen een succesvol alternatief voor degenen
die het toepasten, en het droeg aanzienlijk bij aan het verzwakken van
de greep van het feodalisme. Zoals de middeleeuwse historicus Fritz
Rorig stelde, zou de horige van een seculiere heer ``na een jaar en een
dag een vrije burger van de stad zijn.'' Het is redelijk te verwachten
dat er nieuwe institutionele toevluchtsoorden zullen ontstaan, op basis
van ``nieuwe juridische principes,'' om fiscale toevlucht te bieden aan
burgers van de staat, net zoals de middeleeuwse stad toevlucht bood aan
feodale onderdanen die binnen de schaduw van haar muren leefden.

Econoom Albert O. Hirschman, die de theoretische subtiliteiten van
``stemmen met de voeten'' onderzocht in \emph{Exit, Voice, and Loyalty},
voor het eerst gepubliceerd in 1969, voorzag dat technologische
vooruitgang de kans zou vergroten dat men een exit-strategie zou
aannemen wanneer staten in verval raken. Hij schreef: ``Pas wanneer
landen op elkaar beginnen te lijken door de vooruitgang in communicatie
en algehele modernisering zal het gevaar van voortijdige en
buitensporige exits ontstaan \ldots{}'' Dat is precies wat er gebeurt.
Informatietechnologie vermindert snel veel van de verschillen tussen
jurisdicties, waardoor vertrekken een veel aantrekkelijkere optie wordt.
Uiteraard verwijzen Hirschmans woorden ``voortijdige en buitensporige
exits'' naar het standpunt van wat het beste is voor de staat die men
achterlaat. Ongetwijfeld geloofden heren in middeleeuws Europa dat zij
leden onder ``voortijdige en buitensporige exits'' van hun horigen naar
steden die hen de vrijheid brachten die ze zochten.

Om terug te komen op ons eerdere voorbeeld: het lijkt misschien
vergezocht om aan te nemen dat er een aantal ministaten zullen zijn die
toevlucht bieden aan ballingen die de stervende natiestaten ontvluchten,
maar dat is het niet. Deze soevereiniteiten zullen concurreren op de
algemene voorwaarden voor ballingschap. Sommige ervan, bijvoorbeeld aan
de westkust van Noord-Amerika, zouden zich kunnen richten op mensen die
niet roken en een hekel hebben aan passief roken. Uiteraard zouden
dergelijke regimes niet populair zijn bij rokers. Het verbieden van hun
gewoonte zal voor veel rokers overkomen als een arbitraire oplegging.

In het Industriële Tijdperk van massapolitiek werden dergelijke
meningsverschillen uitgevochten in politieke campagnes die uiteindelijk
een van de partijen ertoe dwong om zich neer te leggen bij de wensen van
de machtigere partij. Het is echter niet noodzakelijk dat geschillen
over elkaar uitsluitende keuzes worden beslecht op een manier die
vereist dat de voorkeuren van grote aantallen mensen worden onderdrukt.

Sommige mensen eten graag foie gras, anderen hotdogs, en weer anderen
tofu. Ze hoeven meestal niet te twisten over hun voorkeuren voor eten,
omdat hun culinaire keuzes niet aan elkaar gebonden zijn. Niemand dwingt
iedereen om dezelfde maaltijd te eten. Megapolitieke omstandigheden
dwongen echter wel tot gemeenschappelijke consumptie van vele soorten
collectieve en zelfs private goederen die door regeringen in het
Industriële Tijdperk werden geleverd. Waarom? Omdat er grote economische
voordelen te behalen waren door op grote schaal te opereren. Het was
daarom onpraktisch om uitgestrekte jurisdicties op te delen in enclaves
waar iedereen zijn zin kon krijgen, zelfs wanneer het om belangrijke
zaken ging. De benadering die Adam Smith bepleitte om het aanbieden van
publieke goederen te beperken, kan veel gemakkelijker worden toegepast
wanneer het aantal jurisdicties tien- of zelfs honderdvoudig toeneemt.
In het Informatietijdperk zullen steeds meer soevereiniteiten kleine
enclaves zijn in plaats van continentale rijken. Sommigen kunnen
Noord-Amerikaanse indianenstammen zijn die een belastingjurisdictie
zullen claimen over hun reservaten, net zoals zij nu aanspraak maken op
het recht om casino's te exploiteren of om zonder beperkingen te vissen.

Omdat informatietechnologie veel nadelen van het opdelen van
handelsgebieden wegneemt, zal het voor de nieuwe soevereiniteiten
eenvoudiger zijn om meer volgens de principes van clubs of
affiniteitgroepen te opereren dan volgens die van territoriale
natiestaten. Net zoals dat het niet absoluut noodzakelijk is dat elke
potentiële klant dezelfde smaak in kleding heeft of dezelfde
televisieprogramma's bekijkt, zullen de affiniteitspunten die de
bestuursstijl van gefragmenteerde soevereiniteiten bepalen ook niet
langer door iedereen goed bevonden dienen te worden.

Een breed palet van smaken zal leiden tot sterk uiteenlopende stijlen
van gefragmenteerde soevereiniteit, net zoals er steeds bredere keuzes
zijn in kledingstijlen of televisie-uitzendingen. Sommige microstaten
kunnen zelfs met elkaar verbonden zijn zoals hotelketens in franchises,
of samenwerken om voordelen te behalen op het gebied van politietaken en
andere overblijvende overheidsdiensten. Degenen die van schone straten
houden en een hekel hebben aan kauwgom onder een tafel, zullen Singapore
aantrekkelijk vinden, fans van \emph{Beavis and Butthead} niet. Degenen
die van wild nachtleven houden, zullen Macao of Panama of een
soortgelijke plek verkiezen. Klanten die zich ongemakkelijk voelen met
de zeden in de ene jurisdictie zullen welkom zijn in andere. Terwijl
Salt Lake City wellicht rookvrij is, zal de nieuwe stadstaat in Havana,
misschien herdoopt tot Monte Cristo, waarschijnlijk gehuld zijn in een
wolk van sigarenrook.

\begin{quote}
Het betekent dat alle monopolies, hiërarchieën, piramides en
machtsnetwerken van de industriële samenleving zullen breken onder deze
constante druk, veroorzaakt door het verspreiden van intelligentie naar
de randen van alle netwerken. Bovenal zal de wet van Moore de
belangrijkste concentratie, de belangrijkste fysieke samenballing van
macht in Amerika van vandaag omverwerpen: de grote stad, dat grote
geheel van industriële steden dat nu leeft op levensondersteunende
systemen, zo'n 360 miljard aan directe subsidies van ons allemaal elk
jaar. Grote steden zijn achtergebleven bagage uit het industriële
tijdperk. -- GEORGE GILDER
\end{quote}

Een bijzondere ironie van de heropkomst van micro-soevereiniteiten of
``stadstaten'' is dat dit kan samenvallen met het leeglopen van veel
steden. De grote stad was grotendeels een artefact van het
industrialisme in het Westen. Het fabriekssysteem deed steden opbloeien
om schaalvoordelen in de productie van goederen die gemaakt werden van
een grote hoeveelheid natuurlijke hulpbronnen te benutten.

Toen de negentiende eeuw begon, werden steden van meer dan 100.000 als
enorm beschouwd, en buiten Azië, waar bevolkingsstatistieken
twijfelachtig waren, waren er geen steden van meer dan een miljoen
inwoners. De grootste stad in de Verenigde Staten in 1800 was
Philadelphia, met een populatie van 69.403 mensen. New York had er
slechts 60.489. Baltimore was de derde grootste stad in Amerika met
26.114 inwoners. De meeste van wat later de grote metropolen van Europa
zouden worden, hadden bevolkingen die minuscuul zijn naar
twintigste-eeuwse maatstaven. Londen, met een bevolking van 864.845, was
waarschijnlijk de grootste stad ter wereld. Parijs, met 547.756, was de
enige andere stad in Europa met meer dan een half miljoen inwoners in
1801. Lissabon had 350.000 inwoners. Wenen had een bevolking van
252.000. Berlijn kwam pas in 1819 net boven de 200.000 uit. Madrid was
de thuisbasis van 156.670. In 1802 was de bevolking van Brussel 66.
Boedapest had slechts 61.000 inwoners.

Er is een duidelijke verleiding om te denken dat de groei van grote
steden een directe functie is van bevolkingsgroei. Maar dat is niet
noodzakelijk zo. Elke mens op aarde zou in Texas kunnen worden
ondergebracht, waarbij elk gezin in zijn eigen vrijstaand huis met tuin
zou wonen, en er zou nog steeds een deel van Texas overblijven. Zoals
Adna Weber betoogde in de klassieke studie \emph{The Growth of Cities in
the Nineteenth Century}, verklaart bevolkingsgroei op zichzelf niet
waarom mensen in stedelijke omgevingen wonen in plaats van verspreid op
het platteland. In 1890 had Bengalen (het huidige Bangladesh) ongeveer
dezelfde bevolkingsdichtheid als Engeland. Toch was de stedelijke
bevolking van Bengalen slechts 4,8 procent, terwijl die van Engeland
61,7 procent was.

Historisch gezien waren steden door muren afgescheiden van het
platteland om plunderaars en de lagere klassen buiten te houden. De
groei van industriële werkgelegenheid in de negentiende en twintigste
eeuw creëerde grote steden. Nu industrialisme begint te vervagen, is de
grote stad uiterst kwetsbaar geworden voor ontwrichting. Deze
ontwikkeling wordt bij uitstek gesymboliseerd door Detroit, ooit de
belangrijkste industriële stad van de twintigste eeuw. Ooit passeerde
een groot deel van de industriële wereldproductie door Detroit. Nu is
het een uitgeholde schelp, geteisterd door misdaad en wanorde. In veel
straten van het centrum van Detroit zijn een of meer vervallen gebouwen
afgebrand of gesloopt, waardoor de indruk ontstaat dat de stad een reeks
bombardementen uit de Tweede Wereldoorlog heeft overleefd.

Detroit herinnert ons aan het feit dat veel industriële steden niet
langer levensvatbaar zijn. Nu informatie en ideeën belangrijker worden
voor waardecreatie dan fabricage uit natuurlijke hulpbronnen, zullen ze
in verval raken. In veel gevallen is de grote stad al te groot geworden
om haar eigen gewicht te dragen. Om een metropool te laten functioneren,
moeten een aanzienlijk aantal ondersteunende systemen efficiënt op grote
schaal kunnen opereren. Het samenbrengen van miljoenen mensen impliceert
een enorm verhoogde kwetsbaarheid voor misdaad, sabotage en willekeurig
geweld. Tijdens het Industriële Tijdperk werden de kosten om zich tegen
deze risico's te beschermen terugbetaald met de voordelen die productie
op grote schaal opleverden.

In het Informatietijdperk overleven slechts steden die hun
onderhoudskosten rechtvaardigen door het bieden van een hoge kwaliteit
van leven. Personen op afstand zullen niet langer verplicht zijn hen te
subsidiëren. Een goede graadmeter voor de levensvatbaarheid van steden
is of degenen die in de kern van de stad wonen rijker zijn dan degenen
aan de rand. Buenos Aires, Londen en Parijs zullen aantrekkelijke
plaatsen blijven om te wonen en zaken te doen, lang nadat het laatste
goede restaurant sluit in South Bend, Louisville en Philadelphia.

\subsection{Country states}\label{country-states}

Sommige stadstaten zullen wellicht slechts enclaves blijken te zijn
zonder bijbehorende steden. Misschien kunnen ze beter worden beschouwd
als dorpstaten of landstaten.

Natuurlijke hulpbronnen zullen ook op verschillende manieren worden
gewaardeerd. Wanneer je overal zaken kunt doen, zou je ervoor kunnen
kiezen om dit te doen op een mooie plek waar je diep kunt ademhalen
zonder al te veel kankerverwekkende luchtvervuiling.
Communicatietechnologieën die taaldrempels minimaliseren, zullen het
steeds gemakkelijker maken om in bijna elke aantrekkelijke omgeving te
verblijven. Dunbevolkte regio's met gematigde klimaten en een grote
hoeveelheid landbouwgrond per hoofd van de bevolking, zoals
Nieuw-Zeeland en Argentinië, zullen ook een comparatief voordeel
genieten omdat zij hoge standaarden van volksgezondheid hebben en
goedkope producenten zijn van voedsel en hernieuwbare producten.
Dergelijke producten zullen profiteren van toenemende vraag naarmate de
levensstandaard van miljarden mensen in Oost-Azië en Latijns-Amerika
stijgt.

\subsection{Het
inequivalentietheorema}\label{het-inequivalentietheorema}

Veel aannames van economen over menselijk gedrag zijn geworteld in de
tirannie van plaats. Een duidelijk voorbeeld is Ricardo's
``Equivalentietheorema'', dat stelt dat burgers in een land met hoge
begrotingstekorten hun verwachtingen aanpassen in afwachting van hogere
belastingen die in de toekomst nodig zullen zijn om de schuld af te
lossen. In die zin bestaat er een ``equivalentie'' tussen het
financieren van uitgaven via belastingen en via schulden. Dat was
althans zo in het begin van de negentiende eeuw, toen Ricardo dit
schreef. In het Informatie­tijdperk zal de rationele mens echter niet
reageren op het vooruitzicht van hogere belastingen door meer te sparen,
maar hij zal zijn domicilie verplaatsen of zijn transacties elders
onderbrengen. Om dezelfde reden dat producenten leveranciers selecteren
op basis van de laagste kosten, zullen zij nog sterker gemotiveerd zijn
om alternatieve aanbieders van bescherming te zoeken. De baten hiervan
zullen de marges van een overstap naar een nieuwe leverancier van
bijvoorbeeld kunststofbuizen verre overtreffen. Het te verwachten gevolg
is dat \emph{Soevereine Individuen} en andere rationele mensen zullen
vluchten uit jurisdicties met grote, niet-gedekte verplichtingen.

Goedkope overheden met weinig verplichtingen en lage lasten voor hun
``klanten'' zullen in het Informatie­tijdperk bij uitstek de domicilies
voor vermogensvorming zijn. Dit impliceert veel aantrekkelijkere
vooruitzichten voor het ondernemen in gebieden waar de schuldenlast laag
is en waar overheden al zijn hervormd, zoals Nieuw-Zeeland, Argentinië,
Chili, Peru, Singapore en andere delen van Azië en Latijns-Amerika. Deze
regio's zullen bovendien superieure platforms zijn voor ondernemerschap
in vergelijking met niet-hervormde, dure economieën in Noord-Amerika en
West-Europa.

\subsection{De afname van lokale
prijsverschillen}\label{de-afname-van-lokale-prijsverschillen}

Zeer lage informatiekosten doen de meeste lokale prijsvoordelen
verdwijnen. Kopers doorzoeken immers talloze verkooppunten om de laagste
prijzen voor verhandelbare goederen te vinden en schakelen daarbij
externe diensten in om over landsgrenzen heen te winkelen. Hierdoor zal
het voor mensen veel eenvoudiger zijn om de eigenschappen van complexe
producten, zoals verzekeringen, met elkaar te vergelijken. Daarbij
omzeilt dit de handelsbeperkingen die lokale vergunningsprocedures met
zich meebrengen. Naar verwachting zorgen extra informatie en
intensievere concurrentie ervoor dat lokale prijsverschillen verdwijnen,
wat resulteert in lagere winstmarges.

\section{Nieuwe organisatorische
verplichtingen}\label{nieuwe-organisatorische-verplichtingen}

De cybereconomie zal wezenlijk van de industriële economie verschillen
in de manier waarop de deelnemers met elkaar omgaan.
Informatie­technologie zal veel van de langdurige organisatorische
voordelen van bedrijven, die voortkomen uit hoge transactie- en
informatiekosten, doen verdwijnen. Het Informatie­tijdperk zal het
tijdperk van de ``virtuele onderneming'' zijn.

Veel analisten, die meer kennis van informatie­technologie hebben dan
wij, hebben volledig gemist dat deze technologie de logica van
economische organisatie zal transformeren. De nieuwe technologie
overschrijdt niet alleen grenzen en barrières, ze betekent ook een
revolutie voor de ``interne'' kosten van rekenkracht. Zelfs het kleine
aantal bedrijven dat niet direct wordt beïnvloed door toenemende
grensoverschrijdende concurrentie dankzij verbeterde informatie- en
communicatietechnologie zal worden geconfronteerd met nieuwe
organisatorische eisen. Snel dalende informatie- en transactiekosten
zullen de schaalvoordelen drastisch verlagen, waardoor veel van de
prikkels die tijdens de industriële periode leidden tot langlevende
bedrijven en carrièremogelijkheden, teniet worden gedaan.

\subsection{Waarom bedrijven?}\label{waarom-bedrijven}

Klassieke economen als Adam Smith besteedden nauwelijks aandacht aan de
vraag welke omvang bedrijven idealiter zouden moeten hebben. Ze gingen
niet in op wat de optimale grootte bepaalt, waarom ondernemingen de
vormen aannemen die zij aannemen, of zelfs waarom ze überhaupt bestaan.
Waarom kiezen ondernemers ervoor om medewerkers in dienst te nemen in
plaats van elke benodigde taak via een veiling aan onafhankelijke
aannemers uit te besteden? De Nobelprijswinnaar Ronald Coase stippelde
een nieuwe koers in de economie uit door enkele van deze fundamentele
vragen te adresseren. Zijn bevindingen tonen duidelijk de revolutionaire
impact van informatietechnologie op de structuur van bedrijven. Coase
betoogde dat ondernemingen een efficiënte manier waren om het gebrek aan
informatie en hoge transactiekosten te overbruggen.

\subsection{Informatie- en
transactiekosten}\label{informatie--en-transactiekosten}

Om te begrijpen waarom, stel je eens voor met welke obstakels je te
maken krijgt als je een assemblagelijn uit het industriële tijdperk in
gang wilt zetten zonder dat één onderneming de activiteiten coördineert.
In principe had een auto geproduceerd kunnen worden zonder dat de
productie werd gecentraliseerd onder toezicht van één firma. Econoom
Oliver Williamson, samen met Coase, was een pionier in de ontwikkeling
van de firmatheorie. Williamson definieerde zes verschillende methoden
van werking en controle. Een daarvan is de ``ondernemersmodus,'' waarin
elk werkstation wordt beheerd en geëxploiteerd door een specialist. Een
andere noemt Williamson de ``gefedereerde werkstations,'' waarbij een
tussenproduct door elke werknemer van de ene naar de andere fase wordt
overgedragen. Er is geen fysieke reden waarom de duizenden werknemers
niet vervangen hadden kunnen worden door een groep onafhankelijke
aannemers, die elk een ruimte op de fabrieksvloer huren, bieden op
onderdelen, en aanbieden om de as te monteren of de spatborden aan het
chassis te lassen. Toch zou je tevergeefs zoeken naar een voorbeeld van
een autofabriek uit het Industriële Tijdperk die door onafhankelijke
aannemers georganiseerd en gerund werd.

\subsection{Coördinatieproblemen}\label{couxf6rdinatieproblemen}

Het runnen van een industriële faciliteit zonder de coördinatievoordelen
van één centrale organisatie zou vrijwel alle schaalvoordelen die bij
grootschalige exploitatie mogelijk zijn, doen verdwijnen. Het
coördineren van een wirwar aan kleine bedrijven brengt enorme
transactionele problemen met zich mee, waardoor de assemblagelijn
praktisch onbruikbaar wordt. Om het systeem überhaupt werkend te houden,
zou het nodig zijn om voortdurend te onderhandelen met de diverse
aannemers. In plaats van zich op de productie te concentreren, zouden ze
hun tijd moeten besteden aan het bepalen van de prijzen voor onderdelen
en het vastleggen van de voorwaarden voor hun voortdurend wisselende
samenwerkingen. Alleen al het monitoren van de productie zou een
moeilijk probleem geweest zijn.

\subsection{De bevoegdheid om op te
treden}\label{de-bevoegdheid-om-op-te-treden}

Als zo'n netwerk van onafhankelijke organisaties al moeite heeft om een
auto in elkaar te zetten, dan zou het ontwikkelen en herontwerpen van
modellen een regelrechte nachtmerrie vormen. Stel je eens voor met welke
problemen een ontwerper zou worden geconfronteerd wanneer hij de
honderden zelfstandige aannemers moet overtuigen van de noodzakelijke
wijzigingen om een nieuw model te introduceren. In de praktijk vereist
dat vrijwel unanieme instemming. Wie terughoudend is of bezwaar maakt
tegen een enkele aanpassing in de productspecificaties, zou de
modelverbetering praktisch kunnen blokkeren of de introductiekosten doen
stijgen, waardoor de voordelen van grootschalige exploitatie nog verder
in gevaar komen.

\subsection{Onnodige onderhandelingen}\label{onnodige-onderhandelingen}

Een assemblagelijn die door onafhankelijke aannemers werd gehuurd (of in
afzonderlijk bezit was) zou onderhevig zijn geweest aan tal van
kwetsbaarheden die vermeden werden door binnen één enkel bedrijf te
opereren. Het overlijden, de ziekte of het financiële falen van
individuele aannemers zou een veel te vaak voorkomend probleem zijn
geweest in operaties die de samenwerking van duizenden mensen vereisten
om onder één dak een enkel product te bouwen. De veilingmarkt zou deze
aannemers zeker hebben kunnen vervangen, maar elke opvolging zou een
onderhandelde regeling hebben vereist, zoals een uitkoop van de vorige
exploitant door zijn vervanger. Ook zou het een overeenkomst hebben
vereist over de overname van de huur van de fabrieksruimte, en misschien
een nieuw leasecontract voor de lasmachine of de pers die werd gebruikt
voor het stansen van de achterlichtfittingen. Dit alles zou ingewikkeld
zijn geweest.

\subsection{Perverse prikkels}\label{perverse-prikkels}

Een andere cruciale moeilijkheid bij een assemblagelijn van
onafhankelijke aannemers onder de omstandigheden van het Industriële
Tijdperk was dat de kapitaalbehoeften voor de individuele aannemers
sterk zouden verschillen. Een mal die nodig was om een kunststoffen
schakelaar van het dashboard te produceren zou bijvoorbeeld relatief
goedkoop kunnen zijn, terwijl de apparatuur die nodig is om een
motorblok te gieten of het plaatwerk van een spatbord te stansen
miljoenen zou kunnen kosten. De grote hoeveelheid grondstoffen en de
sequentiële aard van assemblagelijnproductie maakten problemen door hoge
kapitaalkosten onvermijdelijk om redenen die in het vorige hoofdstuk
zijn geanalyseerd. Aannemers met kapitaalintensieve taken zouden in
wezen afhankelijk zijn geweest van de medewerking van anderen om hun
investeringen af te schrijven. Aannemers met grote kapitaaleisen konden
alleen winst maken en financiering krijgen als zij de samenwerking van
andere, minder kapitaalintensieve deelnemers wisten te verzekeren. In
veel gevallen zouden ze die medewerking niet hebben gekregen.

Er zou een substantiële prikkel zijn geweest voor de kleintjes om de
groten uit te buiten. Wie minder kapitaal nodig had voor hun taak op de
assemblagelijn, kon voordeel behalen door op belangrijke momenten niet
mee te werken. Net als stakende arbeiders konden ze de productie
stilleggen, waardoor zij nauwelijks verlies leden maar de degenen met
grote investeringen wel. Het productieproces zou voortdurend worden
beïnvloed door strategisch gedrag, waarbij kleine aannemers degenen met
hogere kapitaalkosten blootstelden aan gijzeling via hun vermogen om de
productie te dwarsbomen. Het gedrag van kleinere aannemers om extra geld
van de groten af te dwingen zou de efficiëntie van het systeem hebben
verminderd.

\subsection{The firm solution}\label{the-firm-solution}

Kortom, een groot deel van de efficiëntie die tijdens het Industriële
Tijdperk werd behaald door een assemblagelijn op grote schaal te
exploiteren, zou zijn verloren gegaan als de productie was verdeeld over
talloze individuele aannemers. Het enkele grote bedrijf was een
efficiënte manier om deze nadelen te overwinnen, ondanks de andere
beperkingen. Grote bedrijven waren bureaucratisch, maar tot op zekere
hoogte waren bureaucratie en hiërarchie precies wat nodig was tijdens
het Industriële Tijdperk. Administratieve en managementteams hielden
toezicht op en coördineerden de productie, waarbij talrijke
middenmanagers opdrachten naar beneden gaven en andere informatie terug
de hiërarchie in stuurden. De corporatieve bureaucratie zorgde ook voor
boekhoudkundige controles en minimaliseerde problemen, waarbij
werknemers niet handelen in het beste belang van het bedrijf dat hen in
dienst heeft. Om geavanceerde boekhouding te realiseren onder de
omstandigheden van het industriële tijdperk, was het werk van vele
mensen nodig. Het hebben van een dergelijke administratieve bureaucratie
was kostbaar. Deze moest betaald worden, of er nou actief geproduceerd
werd of niet. Omdat zulke beheerders over cruciale kennis beschikten die
nodig was om het bedrijf te runnen, kregen zij doorgaans een hoger
salaris dan hun vaardigheden op de vrije markt zouden opleveren.

\subsection{Organisational slack}\label{organisational-slack}

Het grote aantal professionele managers en administrateurs had ook het
nadeel dat zij de neiging hadden het bedrijf ``over te nemen'' en in hun
eigen belang te laten opereren in plaats van in dat van de
aandeelhouders. Het was in het Industriële Tijdperk bijvoorbeeld niet
ongewoon om bedrijven te zien die royale bedragen spendeerden aan
kantoorinrichting, lidmaatschappen van clubs en andere voordelen die
door het management konden worden genoten, maar die mogelijk geen direct
rendement voor investeerders opleverden.

In een complex bedrijf was het van buitenaf onmogelijk om gemakkelijk te
controleren welke overheadkosten essentieel waren en welke louter luxe
waren voor werknemers. Ook was het moeilijk te voorkomen dat een soms
aanzienlijk deel van de bedrijfsmedewerkers de kantjes er vanaf liep.
Het feit dat het technologisch lastig was om prestaties te monitoren,
maakte een groot middenmanagement noodzakelijk, en tegelijk maakte het
het moeilijk om de toezichthouders zelf te controleren.

Al deze omstandigheden droegen bij aan wat bekend werd als
``organisational slack,'' een term bedacht in 1963 door Richard Cyert en
James March in \emph{A Behavioral Theory of the Firm}. Zorgvuldig
onderzoek wees uit dat talrijke echte bedrijven substantieel
onderpresteerden.

\begin{quote}
Of je nu resultaten behaalt of niet, het loon is hetzelfde. Of je nu
hard werkt of niet, het loon is hetzelfde. Of het je iets kan schelen of
niet, het loon is hetzelfde. -- CHRIS DRAY
\end{quote}

\subsection{``Dat is mijn werk niet''}\label{dat-is-mijn-werk-niet}

Als een naar permanentie strevende entiteit had het grote industriële
bedrijf het nadeel dat het, zoals we al zagen, kwetsbaar was voor
afpersing door vakbonden. Het deelde ook enkele kenmerken van
bureaucratie, zoals we in nog sterkere mate zagen in overheidskantoren.
Bevelen kwamen van bovenaf. Taken waren gestandaardiseerd en in
compartimenten verdeeld. Deze taken waren vaak rigide gedefinieerd. Er
ontstonden grenzen tussen functiecategorieën, vergelijkbaar met degene
die werden afgedwongen door de kartels die de vrije beroepen
reguleerden. Het idee dat een boekhouder zelf het kapotte lampje in zijn
bureaulamp zou vervangen, leek in het Industriële Tijdperk voor velen
even vreemd als het inschakelen van een advocaat om je van griep te
genezen. Er werd niet van werknemers verwacht, en in veel gevallen was
het zelfs niet toegestaan om de grenzen tussen strikt afgebakende
functies te overschrijden.

``Dat is mijn werk niet'' was een veelgehoorde slogan die de
``organisational slack'' van het Industriële Tijdperk onderstreepte.
Ieders werk was exact omschreven in termen van gestandaardiseerde taken
waar niet van mocht worden afgeweken, hoezeer dat de productiviteit ook
had kunnen verbeteren. Elke werknemer in de bedrijfsbureaucratie werd
aangenomen op basis van ``kwalificaties'' die geacht werden prestaties
in zijn specifieke functie te voorspellen. Met weinig uitzonderingen
werd iedereen betaald volgens zijn functie, met min of meer uniforme
beloning binnen de hele organisatie. Omdat specifieke prestaties binnen
de administratieve hiërarchie van Big Business vaak niet werden gemeten,
net als in staatsbureaucratieën, verliep het werk in een rustig tempo.
Het bedrijf wist dus wel de schaalvoordelen van massaproductie te
benutten, maar droeg daarvoor wel de lasten van andere inefficiënties.

\begin{quote}
In een markt doe je iets niet omdat iemand je dat opdraagt of omdat het
op pagina dertig van het strategisch plan staat. Een markt kent geen
taakomschrijvingen. \ldots{} Er zijn geen bevelen, geen signalen van
bovenaf die vertaald moeten worden, niemand die het werk in pakketjes
verdeelt. In een markt heb je klanten, en de relatie tussen leverancier
en klant is fundamenteel niet-organisatorisch, omdat het gaat om twee
onafhankelijke entiteiten. -- WILLIAM BRIDGES
\end{quote}

\subsection{Nieuwe verplichtingen}\label{nieuwe-verplichtingen}

De nieuwe megapolitieke omstandigheden van het Informatietijdperk zullen
de logica van bedrijfsorganisatie aanzienlijk veranderen. Een deel
daarvan is duidelijk. Als informatietechnologie iets doet, dan is het
wel het drastisch verlagen van de kosten van het verwerken, berekenen en
analyseren van informatie. Eén gevolg van deze technologie is dat de
noodzaak om grote aantallen middenmanagers in te huren om
productieprocessen te monitoren, sterk afneemt. Geavanceerde
computergestuurde werktuigen vervangen in veel gevallen al werknemers
die per uur betaald krijgen. En waar het productieproces nog steeds door
mensen wordt bemand, is het controle- en coördinatieproces grotendeels
geautomatiseerd. Apparatuur met microprocessors kan de voortgang van de
lopende band veel beter monitoren dan managers ooit konden. Niet alleen
kan de nieuwe apparatuur de snelheid en nauwkeurigheid meten waarmee
mensen werken, het kan ook automatisch de boekhouding bijhouden en
onderdelen opnieuw bestellen zodra ze uit de voorraad worden gehaald.
Zelfs de kleinste bedrijven kunnen nu financiële controleprogramma's
betalen die hun financiën sneller en geavanceerder beheren dan zelfs de
grootste ondernemingen enkele decennia geleden konden bereiken via hun
productiehiërarchieën.

Het feit dat informatietechnologie verspreide, niet-sequentiële
productie met minder grondstoffen mogelijk maakt, verkleint de
kwetsbaarheid voor manipulatie en afpersing drastisch, zoals we al
hebben gezien. Maar dit zijn niet de enige kenmerken van
informatietechnologie die het steeds aantrekkelijker maken om functies
uit te besteden die voorheen door werknemers werden uitgevoerd.
Kapitaalkosten zijn lager en productcycli zijn korter. De onafhankelijke
contractanten zelf, inclusief de eenpersoonsbedrijven, hebben de
beschikking over veel geavanceerdere informatienetwerken. Binnenkort
zullen zij kunnen vertrouwen op een reeks digitale assistenten om een
breed scala aan kantoorfuncties uit te voeren, van telefoondiensten tot
secretariaatswerk. Digitale assistenten zullen optreden als
secretaresses, reclameadviseurs, reisagenten, bankloketten en
bureaucraten.

\subsection{Het verdwijnen van goede
banen}\label{het-verdwijnen-van-goede-banen}

In toenemende mate zullen individuen die in staat zijn om een grote
hoeveelheid economische waarde te creëren, het grootste deel van die
waarde voor zichzelf kunnen behouden. Ondersteunend personeel dat
voorheen een groot deel van de opbrengsten van de belangrijkste
waardecreërende krachten in een onderneming opslokte, zal worden
vervangen door goedkope geautomatiseerde informatiesystemen. Dit houdt
in dat een organisatie de hoogste kwaliteit van dienstverlening beter
kan garanaderen door deze uit te besteden, in plaats van de functie
binnen het bedrijf te behouden, waar het relatief moeilijker zal zijn om
individuen goed te belonen voor hun prestaties. Een virtuele onderneming
zal de meeste ``organisational slack'' elimineren door de organisatie
zelf te elimineren.

``Goede banen'' zullen tot het verleden behoren. Een ``goede baan,''
zoals Princeton-econoom Orly Ashenfelter het stelde, ``is een baan die
meer betaalt dan je waard bent.'' In het Industriële Tijdperk bestonden
er veel ``goede banen'' vanwege hoge informatie- en transactiekosten.
Bedrijven groeiden en internaliseerden een bredere waaier aan functies
omdat ze zo schaalvoordelen konden benutten. Corporatief overgewicht
werd eveneens gesubsidieerd door belastingwetten. In de latere
industriële periode versterkten hoge belastingen kunstmatig de
aantrekkelijkheid om langlevende bedrijven te vormen met vaste
werknemers. In de meeste landen verhoogden belastingwetten en -regels
aanzienlijk de kosten voor het oprichten en ontbinden van bedrijven op
projectbasis. Ze dwongen ondernemers er bovendien vaak toe om
onafhankelijke contractanten als werknemers aan te nemen. Juridische
interventies bliezen tijdelijk het aanbod van ``goede banen'' verder op
door het kostbaar en moeilijk te maken een werknemer te ontslaan,
ongeacht hoe weinig hij bijdroeg aan de productiviteit van het bedrijf.

Het was dan ook onvermijdelijk en logisch dat de aard van de
bedrijfsorganisatie in het Industriële Tijdperk ervoor zorgde dat de
meest bekwame en getalenteerde mensen, die een onevenredig groot deel
van de toegevoegde waarde in een onderneming creëerden, relatief minder
werden betaald dan hun bijdrage waard was. Dit zal veranderen in het
Informatietijdperk.

De microprocessorrevolutie verbetert de beschikbaarheid van informatie
enorm en verlaagt transactiekosten. Dit devolueert de firma. In plaats
van permanente bureaucratie zullen activiteiten rond projecten worden
georganiseerd, zoals filmmaatschappijen dat al doen. De meeste voorheen
``interne'' functies van het bedrijf zullen worden uitbesteed aan
onafhankelijke contractanten. Werknemers uit het Industriële Tijdperk
die ``goede banen'' hadden maar weinig bijdroegen en op collega's
vertrouwden om hen te ``dekken,'' zullen zichzelf spoedig aanbieden voor
contracten op de spotmarkt. En hetzelfde geldt voor vele loyale,
ijverige werknemers. ``Goede banen'' zullen een achterhaald idee zijn
omdat banen in het algemeen achterhaald zullen zijn.

In het extreme geval van grote Japanse ondernemingen verwachtten
werknemers een baan voor het leven te hebben. Zelfs wanneer zij geen
productieve taak te vervullen hadden, bleven zij in dienst, soms slechts
aanwezig achter ``een kaal bureau in de hoek van een fabriek.'' Nu wordt
zelfs in Japan het opgeblazen witteboordenapparaat afgeslankt. De kop
van een artikel in de \emph{International Herald-Tribune} vertelde het
verhaal: ``Bitter Afscheid: Het Pijnlijke Verval van Japans
Baan-voor-het-Leven-Cultuur.''

In de postindustriële periode zullen banen taken zijn die je doet, niet
iets dat je ``hebt.'' Voor het Industriële Tijdperk was permanente
werkgelegenheid vrijwel onbekend. Zoals William Bridges stelde: ``Voor
1800, en in veel gevallen nog lang daarna, verwees een job altijd naar
een bepaalde taak of onderneming, nooit naar een rol of positie in een
organisatie. \ldots{} Tussen 1700 en 1890 registreert de Oxford English
Dictionary veelvuldig het gebruik van termen als \emph{job-coachman},
\emph{job-doctor} en \emph{job-gardener}, allemaal verwijzend naar
mensen die op eenmalige basis werden ingehuurd. \emph{Jobwork} (een
andere veel voorkomende term) was occasioneel werk, geen vaste
werkgelegenheid.'' In het Informatietijdperk zullen de meeste taken die
voorheen binnen bedrijven werden opgevangen als middel om informatie- en
transactiekosten te verlagen, terug migreren naar de spotmarkt. ``Just
in time''-voorraadbeheer en outsourcing zijn beide praktisch geworden
door informatietechnologie. Het zijn stappen in de richting van de dood
van banen. Grote ondernemingen zoals AT\&T hebben inmiddels al alle
vaste functiecategorieën afgeschaft. Posities in dat grote bedrijf zijn
nu voorwaardelijk. In de woorden van Bridges: ``Werkgelegenheid wordt
opnieuw tijdelijk en situationeel, en categorieën verliezen hun
grenzen.'' In de nieuwe cybereconomie zullen ``onafhankelijke
contractanten'' via telewerken continenten overstijgen en zich groeperen
rond het equivalent van de lopende band van het Informatietijdperk.

\subsection{Hollywood neemt het over}\label{hollywood-neemt-het-over}

Het model voor de organisatie van bedrijven zou in de nieuwe
Informatiemaatschappij een filmproductiemaatschappij kunnen zijn.
Dergelijke ondernemingen kunnen zeer geavanceerd zijn, met budgetten van
honderden miljoenen dollars. Hoewel het vaak grote operaties zijn, zijn
ze tijdelijk van aard. Een filmmaatschappij die een film produceert voor
\$100 miljoen kan zich een jaar lang vormen en daarna weer ontbinden. De
mensen die aan de productie werken zijn getalenteerd, maar ze verwachten
niet dat werk aan het project gelijkstaat aan een ``vaste baan.''
Wanneer het project voorbij is, gaan de lichttechnici, cameramensen,
geluidstechnici en kostuumspecialisten hun eigen weg. Ze kunnen in een
volgend project weer bij elkaar komen, of ook niet.

Naarmate schaalvoordelen afnemen en kapitaalvereisten voor veel soorten
informatie-intensieve activiteiten tegelijkertijd dalen, zal er een
sterke prikkel zijn voor bedrijven om te ontbinden. Bedrijfsactiviteiten
zullen ad-hoc en tijdelijk zijn. Bedrijven zullen over het algemeen
korter bestaan. Virtuele ondernemingen die talenten voor specifieke
doelen samenbrengen zullen efficiënter zijn dan langdurige bedrijven.
Naarmate encryptie breed gebruikt wordt en de belasting op kapitaal door
concurrentie wordt gedrukt, zullen kunstmatige schaalvoordelen die het
bestaan van ``permanente'' bedrijven ondersteunen, verdwijnen. Dit zal
gebeuren, ongeacht of belastingen snel of langzaam worden verlaagd. Als
dit snel gebeurt, verdwijnen de kunstmatige kosten van projectmatig
functioneren sneller. Bij een trage verlaging dragen bestaande bedrijven
nog steeds de last van ouderwetse, hoge belastingen, terwijl nieuwe
ondernemingen als virtuele bedrijven opereren en beter in staat zijn om
dure lasten van de stervende natiestaat te vermijden.

Hoewel speciale vaardigheden en talenten belangrijker dan ooit zullen
zijn in de Informatiemaatschappij, zullen de meeste kunstmatige grenzen
tussen beroepen verdwijnen. Geavanceerde informatie- en
opslagtechnologieën zullen de bedrijfsgeheimen en gespecialiseerde
informatie van beroepen zoals recht, geneeskunde en boekhouding
toegankelijk maken voor iedereen. De economische waarde van memorisatie
als vaardigheid zal afnemen, terwijl het belang van synthese en
creatieve toepassing van informatie zal toenemen.

De volledige implicaties van deze verandering zullen worden vertraagd
door verouderde regelgeving. Maar op de lange termijn zal de macht van
overheden om de cybereconomie te reguleren tot vrijwel nul afnemen. Elke
kunstmatige regulering van professionele monopolies die kosten verhoogt
zonder marktwaarde te leveren, zal uiteindelijk worden genegeerd.

Er zijn nog andere implicaties van de verschuiving naar een
Informatiemaatschappij:

\begin{itemize}
\tightlist
\item
  Lokale regulering die hogere kosten oplegt, zal worden getransformeerd
  naar marktconforme systemen.
\item
  Concurrentie tussen rechtsgebieden zal toenemen om hoogwaardige
  activiteiten aan te trekken die in principe overal gevestigd kunnen
  worden. Geen enkele locatie is noodzakelijkerwijs aantrekkelijker dan
  de volgende.
\item
  Zakelijke relaties zullen steeds meer steunen op
  ``vertrouwenskringen.'' Doordat encryptie individuen in staat stelt om
  onopgemerkt te stelen, zal eerlijkheid een hoog gewaardeerde
  eigenschap van zakenpartners worden.
\item
  Octrooi- en auteursrechtssystemen zullen veranderen door de
  gemakkelijke toegang tot bepaalde informatie.
\item
  Bescherming zal steeds meer technologisch dan juridisch zijn. Lagere
  klassen zullen worden buitengesloten. De beweging naar afgesloten
  gemeenschappen is vrijwel onvermijdelijk. Mensen die last veroorzaken
  buitenhouden is een effectieve, en traditionele, manier om crimineel
  geweld te minimaliseren in tijden met zwak centraal gezag
\item
  Bulkgoederen zullen zwaar belast worden en lokaal worden vervoerd,
  terwijl luxegoederen licht belast zullen worden en over grote afstand
  zullen worden getransporteerd.
\item
  Politiefuncties zullen steeds meer door particuliere beveiligers
  worden uitgevoerd, gekoppeld aan handelsverenigingen.
\item
  Private bedrijven kunnen tijdelijk voordeel hebben tegenover
  beursgenoteerde bedrijven omdat zij meer vrijheid genieten om kosten
  van overheden te ontwijken.
\item
  Banen voor het leven zullen verdwijnen, aangezien ``banen'' steeds
  vaker taakgericht of projectwerk worden in plaats van posities binnen
  een organisatie.
\item
  Controle over economische middelen zal verschuiven van de staat naar
  personen met superieure vaardigheden en intelligentie, naarmate het
  steeds makkelijker wordt om vermogen te creëren door kennis aan
  producten toe te voegen.
\item
  Veel leden van geleerde beroepen zullen worden vervangen door
  interactieve informatie-opzoeksystemen.
\item
  Naarmate inkomensongelijkheid binnen rechtsgebieden toeneemt, zullen
  voor mensen met lagere intelligentie nieuwe overlevingsstrategieën
  ontstaan, gericht op vrijetijdsvaardigheden, sportieve bekwaamheden en
  criminaliteit, evenals dienstverlening aan de groeiende groep
  Soevereine Individuen.
\end{itemize}

Politieke systemen die ontstonden toen geweld hoge rendementen
opleverde, moeten ingrijpende aanpassingen ondergaan. Nu efficiëntie
belangrijker wordt in verhouding tot de macht van een systeem, zullen
kleine, efficiënte soevereiniteiten, die meer bescherming bieden tegen
lagere kosten, steeds beter houdbaar zijn.

Net als in de Middeleeuwen ontstaan er opnieuw toenemende schaalnadelen
bij het uitoefenen van geweld. Dit blijkt al uit het groeiende aantal
soevereine entiteiten sinds de val van het communisme. Het aantal
soevereiniteiten in de wereld zal naar verwachting snel toenemen
naarmate de logica van het Informatietijdperk door ervaring wordt
bevestigd.

Macht zal opnieuw op kleine schaal worden uitgeoefend. Bij het bieden
van aantrekkelijke voorwaarden voor soevereiniteit aan hun ``afnemers'',
zullen enclaves en provincies zelfs substantieel voordeel kunnen ervaren
ten opzichte van uitgestrekte landen. Dit zal heel anders zijn dan de
snel stervende moderne periode, waarin geen enkele entiteit kon
overleven zonder militaire macht, krachtig genoeg om een koninkrijk te
beheersen. Vroeger, toen er schaalnadelen gepaard gingen met het
uitoefenen van macht, hadden degenen die het meest profiteerden van
bescherming, zoals de rijke kooplieden in de late middeleeuwse
stadstaten, controle over de overheid. Naar onze mening kan iets
dergelijks opnieuw gebeuren. De verlaging van roofzuchtige lasten en
efficiëntere verdeling van middelen zal leiden tot snelle groei in
gebieden waar klanten daadwerkelijk controle uitoefenen over de lokale
soevereiniteiten.

Of deze ontwikkelingen kunnen of moeten doorgaan ondanks oppositie van
vele verliezers, zal, zoals we hierna verkennen, tot een van de
belangrijkste controverses van het Informatietijdperk behoren.

\bookmarksetup{startatroot}

\chapter{Nationalisme, reactie en de nieuwe
Luddieten}\label{nationalisme-reactie-en-de-nieuwe-luddieten}

\begin{quote}
Nationalisme is uiteraard van nature absurd. Waarom zou het toeval, of
het lot om geboren te worden als Amerikaan, Albanees, Schot of
eilandbewoner van \emph{Fiji}, verhoudingen van loyaliteit verplichten
die een individu volledig domineren en een samenleving zo inrichten dat
ze formeel met anderen in conflict komt? Vroeger leefden mensen met
lokale loyaliteitsverhoudingen, gehecht aan een plaats, clan of stam, en
hadden zij verplichtingen jegens een heer of grondbezitter, wat leidde
tot dynastieke of territoriale oorlogen. De voornaamste
loyaliteitsverhoudingen waren echter gericht op religie, op God of op de
godkoning, eventueel op een keizer of op een beschaving als geheel. Er
was geen natie. Men voelde wel een band met de patria, het land van je
voorouders, of kende patriottisme, maar over nationalisme spreken vóór
de moderne tijd is een anachronisme. -- WILLIAM PFAFF
\end{quote}

\url{http://www.ibm.com} Zeggen dat `de wereld kleiner wordt' is een
treffende beeldspraak, versterkt door autoriteiten zo gerenommeerd als
het reclamebureau van \emph{IBM}. Hun multiculturele reclamecampagnes
\emph{Solutions for a small planet} op het Internet herinneren
sportliefhebbers, die dit zelf wellicht niet doorhebben, eraan dat de
verhoudingen tussen individuen in wijd verspreide rechtsgebieden door
technologische ontwikkelingen ingrijpend zijn veranderd. We verwijzen
naar de vooraanstaande historicus William McNeill voor een waardevolle
voetnoot over de implicaties. Hij schrijft: ``De voortdurende
intensivering van communicatie en transport bevordert, in plaats van
nationale consolidatie, het tegenovergestelde, aangezien het bereik
bestaande politieke en etnische grenzen overschrijdt.'' Nu de wereld
steeds `kleiner wordt' en de communicatie verbetert, zullen de
willekeurige en inherent absurde aanspraken van naties en van het
nationalisme onvermijdelijk verzwakken.

\section{De grote transformatie}\label{de-grote-transformatie}

Het probleem met deze redelijke verwachting is dat alle eerdere
geschiedenis suggereert dat dit niet op een redelijke manier kan worden
gerealiseerd. De overgang die dit impliceert zal een crisis met zich
meebrengen. Het vereist een radicaal nieuwe manier van denken, een
nieuwe invulling van gemeenschap die verder gaat dan nationalisme en de
natiestaat. Zoals Michael Billig benadrukt: ``Onze overtuigingen over
nationaliteit en dat het natuurlijk is om te behoren tot een natie zijn
de producten van een bepaalde historische periode.'' Die periode, het
Moderne Tijdperk, is mogelijk al voorbij. De overheersende instellingen,
de natiestaten, bestaan nog wel, maar wankelen op een uitgehold
fundament. Wanneer de natiestaten instorten, verwachten we een felle
reactie, vooral in de rijke landen waar de ``nationale economie'' in de
twintigste eeuw hoge inkomens bracht voor ongeschoold werk. Wij geloven
dat de verandering in megapolitieke omstandigheden door
informatietechnologie zal leiden tot radicale institutionele
verandering. De stelling van dit boek is dat de massale macht van de
natiestaat zal worden geprivatiseerd en gecommercialiseerd. Zoals bij
alle echt radicale institutionele veranderingen, zal deze privatisering
en commercialisering van soevereiniteit een revolutie voortbrengen in
het ``gezond verstand'' waarmee de wereld wordt begrepen. Dergelijke
veranderingen verlopen zelden geleidelijk en lineair.

Integendeel. Zoals besproken in \emph{The Great Reckoning}, is een
geleidelijke overgang praktisch uitgesloten. Het Informatietijdperk zal
discontinuïteiten brengen, plotselinge breuken met de instellingen en
het bewustzijn van het verleden. Hier is wat we kunnen verwachten:

\begin{enumerate}
\def\labelenumi{\arabic{enumi}.}
\tightlist
\item
  Veranderingen in economische organisatie door de impact van
  microprocessing, zoals in eerdere hoofdstukken beschreven.
\item
  Snelle afname van het belang van organisaties die binnen geografische
  grenzen opereren. Overheden, vakbonden, gereguleerde beroepen en
  lobbyisten zullen minder belangrijk zijn dan tijdens de Industriële
  Tijd. Omdat gunsten en handelsbeperkingen van overheden minder nuttig
  worden, zullen minder middelen aan lobbying worden verspild.
\item
  Brede erkenning dat de natiestaat verouderd is, wat zal leiden tot
  wijdverspreide afscheidingsbewegingen wereldwijd.
\item
  Afname van status en macht van traditionele elites, evenals verminderd
  respect voor symbolen en overtuigingen die de natiestaat
  rechtvaardigen.
\item
  Intense en soms gewelddadige nationalistische reacties van degenen die
  status, inkomen en macht verliezen wanneer hun ``normale leven'' wordt
  verstoord door politieke devolutie en nieuwe marktarrangementen,
  waaronder: a) achterdocht tegenover en verzet tegen globalisering,
  vrije handel, ``buitenlands'' eigendom en penetratie van lokale
  economieën; b) vijandigheid tegen immigratie, vooral van zichtbaar
  verschillende groepen; c) breed gedragen haat jegens de
  informatie-elite, rijken, hoogopgeleiden en klachten over
  kapitaalvlucht en verdwijnende banen; d) extreme maatregelen door
  nationalisten om afscheiding van individuen en regio's van falende
  natiestaten te voorkomen, inclusief oorlogen en ``etnische
  zuiveringen'' die nationalistische identificatie met de staat
  versterken en de claims van de staat rationaliseren.
\item
  Omdat informatietechnologie het ontsnappen van soevereine individuen
  aan staatsmacht vergemakkelijkt, zal de reactie op het instorten van
  dwang ook een neo-ludditische aanval omvatten op deze technologieën en
  hun gebruikers.
\item
  De nationalistische-ludditische reactie zal regionaal en per
  bevolkingsgroep verschillen: a) In economieën die snel groeien en in
  de industriële periode lage inkomens hadden, zal de reactie minder
  sterk zijn doordat marktontwikkeling de inkomens van alle vaardigheden
  doet stijgen; b) In rijke landen, vooral bij gemeenschappen met een
  hoog percentage waardearme en vaardigheidsarme personen die voorheen
  hoge inkomens hadden, zal de reactie het meest voelbaar zijn; c)
  Neo-Luddieten zullen voornamelijk steun vinden bij de lagere
  inkomensgroepen in de vooraanstaande natiestaten, de Unabomber
  daargelaten; d) De nationalistische en Luddistische-reactie zal echter
  het sterkst zijn, niet bij de allerarmsten, maar bij mensen met
  gemiddelde vaardigheden, onderpresteerders met diploma's, die
  volwassen werden tijdens het Industriële Tijdperk en te maken hebben
  met dalende sociale mobiliteit.
\item
  Nieuwe megapolitieke omstandigheden zullen een nieuw bewustzijn van
  identiteit creëren, samen met complementaire ideologieën en moraal. De
  oude vanzelfsprekendheden van nationalisme verliezen hun
  aantrekkingskracht.
\item
  De nationalistische reactie zal pieken in de vroege decennia van het
  nieuwe millennium en daarna afnemen naarmate de efficiëntie van
  gefragmenteerde soevereiniteiten superieur blijkt aan de macht van de
  natiestaat. We vermoeden dat het aangeboren pestgedrag van
  alternatieve jurisdicties door natiestaten, zoals de Russische invasie
  van Tsjetsjenië, er waarschijnlijk toe zal leiden dat landen en
  nationalistische fanatici de sympathie verliezen van nieuwe generaties
  die volwassen worden onder de megapolitieke omstandigheden van het
  Informatietijdperk.
\item
  De natiestaat zal uiteindelijk instorten door fiscale crises.
  Systemische crises ontstaan typisch wanneer falende instellingen
  stijgende uitgaven en dalende inkomsten ervaren, iets wat
  onvermijdelijk zal gebeuren door de explosie van uitkeringen en
  medische kosten vroeg in de 21e eeuw. Zowel het Verenigd Koninkrijk
  als de Verenigde Staten hebben biljoenen dollar aan niet-gedekte
  pensioenverplichtingen die waarschijnlijk niet onder controle gebracht
  zullen worden. Andere leidende natiestaten kampen met vergelijkbare
  faillissementbedreigende lasten.
\end{enumerate}

\section{Parallellen met de
renaissance}\label{parallellen-met-de-renaissance}

Eerder hebben we redenen besproken om te denken dat de instorting van de
``nanny state'' gevolgen zal hebben die nauw parallel lopen met die van
de instorting van het institutionele monopolie van de Heilige Moederkerk
vijf eeuwen geleden. Net als de natiestaat van vandaag, had de Kerk toen
eeuwenlang een onuitgedaagde dominante positie. In sommige opzichten was
de Kerk zelfs steviger verankerd dan de staat vijfhonderd jaar later zou
worden. De Kerk had lange tijd beweerd te handelen als ``de universele
autoriteit aan het hoofd van de christelijke samenleving.'' Dat is de
karakterisering van de middeleeuwse intellectuele historicus John B.
Morrall. Toch, hoewel weinig Europeanen het voor de technologische
revolutie van de jaren 1490 zouden hebben betwist dat de Kerk de
oppermacht in het christendom had, overleefde de Kerk nauwelijks nog een
generatie in haar traditionele rol.

\subsection{De privatisering van het
geweten}\label{de-privatisering-van-het-geweten}

Aan het begin van de jaren 1520 hadden miljoenen goede Europeanen de
universele autoriteit van de Katholieke Kerk verworpen, een ketterij die
nog enkele decennia eerder bestraft werd met marteling en de dood. Veel
middeleeuwse Europese kathedralen en kerken waren versierd met
instructieve houtsnijwerken van ketters van wie de tong door demonen
werd uitgetrokken.

De les die deze martelingen overbrachten, moet indruk hebben gemaakt op
vele ongeletterde parochianen, die de slachtoffers als ketters konden
herkennen aan hun straf. De iconografie was eenduidig: ketters waren zij
wiens tong werd verminkt. Hoewel deze straf zwaar was, was het slechts
een opwarmer voor de ultieme straf voor ketterij: de brandstapel. Tot
grote teleurstelling van de Kerk bleek deze les echter onvoldoende
afschrikwekkend. De komst van de drukpers vergrootte de verspreiding van
ketterse argumenten zo dramatisch dat zelfs de dreiging van gruwelijke
straffen toekomstige ketters niet meer afschrikte. Het waren inderdaad
niet weinig ongelukkige pioniers van religieuze vrijheid in het
vroegmoderne Europa die voor hun uitspraak van spirituele
onafhankelijkheid betaalden door hun tong te laten uitrukken. Anderen
werden verbrand op de brandstapel. De agenten van de Inquisitie
verbrandden mensen letterlijk voor het uiten van wat wij als gewone
gewetensuitingen zouden beschouwen. Alles bij elkaar kostten de
Reformatie en de reactie die zij inspireerde miljoenen mensen het leven.
Alleen al de veldslagen in de laatste helft van de Dertigjarige Oorlog
eisten 1.151.000 doden. Veel meer stierven door hongersnood, ziekte en
door de hand van de Inquisitie en andere autoriteiten.

Niet al het geweld werd door katholieke autoriteiten gepleegd. In de
Tower of London zijn de botten van meer dan duizend vooraanstaande
Engelse katholieken aangetroffen die vermoedelijk op brute wijze door
koning Hendrik VIII werden vermoord. Sommigen, waaronder Sir Thomas More
en bisschop St.~John Fisher, werden openlijk geëxecuteerd omdat ze
weigerden het oude geloof los te laten. De katholieke dochter van
Hendrik VIII, koningin Maria, die krankzinnig was door syfilis, geërfd
van haar vader, verbrandde daarentegen driehonderd protestantse ketters
op de brandstapel in de laatste twee jaar van haar heerschappij.

Dit was de prijs die werd betaald toen individuen van verschillende
overtuigingen hun religieuze opvattingen en het lang ontziene recht om
hun kerk te kiezen, opeisten. Vanuit ons perspectief aan het eind van de
twintigste eeuw vielen deze uitingen van persoonlijk geloof ruimschoots
binnen het bereik van wat beschermd zou moeten worden door
godsdienstvrijheid en vrijheid van meningsuiting. Maar in het begin van
de zestiende eeuw bestond er noch godsdienstvrijheid, noch vrijheid van
meningsuiting. De autoriteiten van die tijd oriënteerden zich nog aan
het afnemende middeleeuwse wereldbeeld. Voor hen waren gebaren van
individuele autonomie tegenover gezag, vooral de plenitudo potestatis
(volheid van macht) van de paus, schandalig en ronduit subversief. Zoals
theologisch historicus Euan Cameron zei, namen religieuze hervormers
zoals Maarten Luther standpunten in die ``een bewuste en beslissende
breuk met de institutionele en spirituele continuïteit van de oude Kerk
betekenden.''

\subsection{Ketterij en verraad}\label{ketterij-en-verraad}

In diezelfde geest verwachten we ``een bewuste en beslissende breuk''
met de institutionele en ideologische continuïteit van de natiestaat.
Tegen het einde van het eerste kwart van de volgende eeuw zullen
miljoenen rechtschapen individuen het seculiere equivalent van
zestiende-eeuwse ketterij hebben begaan, een soort lage vorm van
verraad. Zij zullen hun trouw aan de wankelende natiestaat hebben
ingetrokken om hun eigen soevereiniteit te doen gelden, hun recht om
niet hun bisschoppen of hun gebedshuis te kiezen, maar hun vorm van
bestuur als klanten. De privatisering van soevereiniteit zal parallel
lopen aan de privatisering van het geweten van vijf eeuwen eerder. Beide
zijn de massale afvalligheid van voormalige aanhangers van dominante
instellingen. Zoals Albert O. Hirschman schreef, expert in ``reacties op
achteruitgang in bedrijven, organisaties en staten,'' is dit soort exit
moeilijk omdat ``exit vaak als crimineel is bestempeld, aangezien het
werd aangeduid als desertie, overlopen en verraad.''

Soevereine Individuen zullen niet langer slechts instemmen met wat hen
wordt opgelegd alsof ze menselijke hulpbronnen van de staat zijn.
Miljoenen zullen de verplichtingen van burgerschap afwerpen om klanten
te worden van de nuttige diensten die overheden leveren. Sterker nog, ze
zullen parallelle instellingen creëren en ondersteunen die de meeste
diensten die met burgerschap samenhangen volledig commercieel maken.
Gedurende het grootste deel van de twintigste eeuw zijn de productieven
door de staat behandeld als activa, net zoals een melkveehouder die zijn
koeien melkt. Ze werden steeds heftiger uitgeknepen. Nu zullen de koeien
vleugels krijgen.

\subsection{Burgerschap opgeven}\label{burgerschap-opgeven}

Net zoals in de zestiende eeuw nieuwe megapolitieke ontwikkelingen het
kerkelijke monopolie ondermijnden, verwachten wij dat de megapolitiek
van het Informatietijdperk uiteindelijk de voorwaarden voor het bestuur
in de eenentwintigste eeuw zal bepalen, ongeacht hoe extreem haar nieuwe
voorschriften ook lijken voor degenen die de waarden van de moderne
politiek als de hunne beschouwen. De transformatie van de status van
`burger' naar die van `klant' betekent een breuk met het verleden, zo
ingrijpend als de overgang van ridderlijkheid naar burgerschap in de
vroegmoderne tijd. Wanneer de informatie-elite haar band met het
burgerschap verbreekt, levert dat een prikkel die vergelijkbaar is met
de reden waarom vijfhonderd jaar geleden miljoenen Europeanen hun
vertrouwen in de paus verloren.

Als de vergelijking met de Reformatie niet overtuigend overkomt, ligt
dat misschien deels aan het feit dat men tegenwoordig niet direct inziet
dat het afzweren van loyaliteit aan religieuze instituties ooit net zo
ingrijpend was als de strenge bestraffing van verraad in de twintigste
eeuw. Buiten enkele islamitische landen beschouwt men ketterij aan het
einde van de twintigste eeuw als een geestelijke overtreding die iemands
reputatie net zo min aantast als een bekeuring voor het rijden met 70
kilometer per uur, waar 50 was toegestaan. Sterker nog, het komt
regelmatig voor in Europa en Noord-Amerika dat geestelijken, en zelfs
bisschoppen, openlijk aangeven niet in God te geloven of cruciale
fundamenten van het geloof dat zij belijden af te wijzen. Tegenwoordig
zal men ketterij bijna alleen nog constateren wanneer het neerkomt op
regelrechte duivelaanbidding. In de meeste westerse landen zijn de
religieuze doctrines zo onsamenhangend en losjes vastgelegd dat slechts
enkelen nog de theologische kernelementen kunnen aanwijzen die ooit het
middelpunt vormden van theologische geschillen. Dit weerspiegelt de
algemene aandachtsverschuiving weg van religieuze kwesties.

Religieuze leiders hebben er in zekere mate toe bijgedragen dat men in
de late twintigste eeuw spirituele onderwerpen niet langer serieus nam.
Zij verlegden hun energie van de spirituele sferen naar het worden van
lobbyisten en agitatoren. Als losse individuen, aangetrokken door de
kracht van de macht, richten zij zich voornamelijk op het onder druk
zetten van politieke leiders om herverdelingsmaatregelen in te voeren
die essentieel blijken voor het nationalistische compromis. Denk
bijvoorbeeld aan de luidruchtige acties van de katholieke Kerk in
Argentinië, die de regering van president Carlos Menem probeert te
dwingen economische hervormingen te laten varen ten gunste van
conventionele, inflatoire monetaire en Keynesiaanse fiscale maatregelen.
Vergelijkbare klachten hebben religieuze leiders ook geuit over pogingen
de opgeblazen begrotingen in Nieuw-Zeeland, en in vele andere landen, te
hervormen. Katholieke bisschoppen lobbyden fel tegen de hervorming van
de sociale zekerheid in de Verenigde Staten.

\subsection{Een fiscale inquisitie?}\label{een-fiscale-inquisitie}

Eenvoudig gezegd richten hedendaagse religieuze leiders hun afnemende
morele autoriteit vooral op seculiere verlossing en pogingen om de staat
te beïnvloeden, in plaats van op spirituele redding. Gezien dit patroon
kan worden verwacht dat zij medeplichtigen zullen zijn in de reactie
tegen de komende seculiere reformatie. Wanneer de natiestaat wordt
uitgedaagd en begint te wankelen, zal zij niet langer in staat zijn de
materiële beloften waar te maken die cruciaal zijn om de steun van het
volk te behouden. De feitelijke overeenkomst die tijdens de Franse
Revolutie werd gesloten, zal vervallen. De staat zal haar burgers geen
goedkope of gratis scholing, laat staan gezondheidszorg,
werkloosheidsuitkeringen en pensioenen meer kunnen garanderen in ruil
voor anders slecht betaalde militaire dienst. Hoewel veranderende
vereisten voor oorlogsvoering het zoor overheden mogelijk maken om
zichzelf en hun grondgebied te verdedigen zonder massale legers op te
stellen, zal dit hen nauwelijks vrijwaren van kritiek op het verbreken
van een inmiddels achterhaalde overeenkomst.

Naarmate de nieuwe megapolitieke logica zich doorzet, zullen de gevolgen
uiterst impopulair blijken bij de verliezers van de nieuwe
Informatiemaatschappij. Het is daarom vrijwel zeker dat veel religieuze
leiders, samen met de belangrijkste begunstigden van overheidsuitgaven,
voorop zullen lopen in een nostalgische reactie die nationalistische
claims opnieuw wil doen gelden. Zij zullen beweren dat geen enkele
Amerikaan, Fransman, Canadees of andere nationaliteit, vul zelf maar in,
met honger naar bed zou mogen gaan. Zelfs landen die vooroplopen in de
hervorming en onevenredig zouden profiteren van ``marktvriendelijke
globalisering,'' zoals Nieuw-Zeeland, zullen worden geplaagd door
reactionaire verliezers. Die zullen proberen de verplaatsing van
kapitaal en mensen over grenzen heen tegen te houden. En daar zal het
niet bij blijven. Demagogen zoals Winston Peters, leider van de New
Zealand First Party, zijn te lui om op een originele wijze na te denken
over hoe de nieuwe wereld zal functioneren. Vroeg of laat zullen Winston
en zijn volgelingen echter de logica van de Informatiemaatschappij
doorzien. Ze zullen proberen de verspreiding van computers, robotica,
telecommunicatie, encryptie en andere technologieën van het
Informatietijdperk, die de verdringing van werknemers in vrijwel elke
sector van de wereldeconomie versnellen, tot stilstand te brengen. Waar
je ook kijkt, er zijn politici die bereid zijn de vooruitzichten voor
lange termijn welvaart te saboteren, enkel om te voorkomen dat mensen
hun onafhankelijkheid van de politiek uitroepen.

\subsection{20/20 zicht}\label{zicht}

Tegen 2020, oftewel ongeveer vijf eeuwen nadat Martin Luther zijn 95
subversieve stellingen op de kerkdeur in Wittenberg had vastgenageld,
zal de perceptie van de kosten en baten van burgerschap een even
subversieve verheldering hebben doorgemaakt. De visie op de natiestaat
zal bij mensen met talent en vermogen, de soevereine individuen van de
toekomst, een politieke transformatie ondergaan die te vergelijken is
met laserchirurgie. Zij zullen 20/20 zicht hebben. In de twintigste
eeuw, en gedurende het hele Moderne Tijdperk, zorgden de aanhoudende
hoge opbrengsten uit geweld ervoor dat een grote overheid rendabel werd.
De kracht van geconcentreerde macht wist de loyaliteit van welgestelden
en ambitieuze mensen te winnen voor de \emph{OESO}-natiestaten, ondanks
de roofzuchtige belastingen op inkomen en kapitaal. Politici slaagden
erin om in elk OESO-land marginale belastingtarieven op te leggen die in
het decennium direct na de Tweede Wereldoorlog de 90 procent naderden of
zelfs overschreden.

Zoals we al bespraken, hadden de rijken nauwelijks een andere keus dan
zich neer te leggen bij dergelijke heffingen. De omstandigheden dwongen
hen immers hun veiligheid toe te vertrouwen aan overheden die
grootschalig geweld konden beheersen. Het deed er nauwelijks toe dat
\emph{OESO-overheden} monopolistische belastingen oplegden, behalve dan
wellicht voor Britse agenten die de kans kregen in Hongkong te gaan
werken. Mensen met groot verdienpotentieel, die van de beste economische
kansen gebruik wilden maken, hadden in de Industriële Periode nauwelijks
een andere keuze dan om zich in een sterk belaste economie te vestigen.
Dat betekende dat zij een belastingdruk moesten dragen die onevenredig
was ten opzichte van de geleverde diensten.

\subsection{Politieke rekenkunde}\label{politieke-rekenkunde}

Negentiende-eeuwse Amerikaanse vicepresident John J. Calhoun schetste
scherp de rekenkunde van de moderne politiek. Calhouns formule verdeelt
de totale bevolking van de natiestaat in twee klassen:
belastingbetalers, die meer bijdragen aan de kosten van
overheidsdiensten dan ze consumeren, en belastingconsumenten, die minder
bijdragen dan de voordelen die ze ontvangen, kosten. Met enkele
opvallende uitzonderingen waren de meeste ondernemers in de OESO-landen
aan het einde van de twintigste eeuw in sterke mate
netto-belastingbetalers. Bijvoorbeeld, in 1996 droeg de top 1\% van de
Britse belastingbetalers 17\% van de totale inkomstenbelasting. Ze
betaalden 30\% meer dan de onderste 50\% van de inkomensontvangers, die
slechts 13\% van de inkomstenbelasting bijdroegen. In de Verenigde
Staten was de last nog groter: de top 1\% betaalde in 1994 28\% van de
totale inkomstenbelasting. Niet alleen moesten de rijken betalen voor
diensten die, zoals Frederic C. Lane opmerkt, ``van slechte kwaliteit en
buitensporig duur waren,'' hun betalingen kwamen vaak geheel ten goede
aan anderen. In de meeste gevallen zouden de rijken zelfs hebben
verkozen om minder overheidsdiensten, die doorgaans van lage kwaliteit
waren, te consumeren. Overheidsbureaus waren in bijna elk land berucht
inefficiënt, grotendeels omdat ze werden beheerd door medewerkers die
weinig geprikkeld werden om de productiviteit te verbeteren. Volgens
vrijwel elke maatstaf betaalden de grootste belastingbetalers tijdens
het Industriële Tijdperk vele malen meer voor overheidsdiensten dan hun
marktwaarde zou rechtvaardigen.

Dit bleef niet onopgemerkt. Het inzicht dat betalingen voor bescherming
aan de overheid, zoals Lane het formuleerde ``verspilling volgens ideale
maatstaven'' waren, leverde in het midden van de twintigste eeuw zelden
praktische gevolgen op. Het werd eerder opgevat als een gebrek dat men
maar moest accepteren: ``één van de vele soorten verspilling die horen
bij de sociale organisatie.''

Voor de mensen die hier niet tevreden mee waren, was het alternatief
niet om bijvoorbeeld van Groot-Brittannië naar Frankrijk te verhuizen,
of van de Verenigde Staten naar Canada. Dat zou in de meeste gevallen
weinig hebben opgeleverd. De leidende natiestaten leden allemaal aan
hetzelfde mankement: ze hanteerden minder of meer onteigenende
belastingregimes. Wie werkelijk meer autonomie wilde, moest de
kernlanden van Europa en Noord-Amerika volledig verlaten en naar de
periferie trekken. De belastingdruk was daar, in delen van Azië,
Zuid-Amerika en op diverse afgelegen eilanden aanzienlijk lager. Om te
ontsnappen aan roofzuchtige belastingen moest meestal echter wel een
prijs worden betaald: verlies van economische kansen en vaak ook een
daling van de levensstandaard. In het Industriële Tijdperk, zoals we
hebben gezien, waren, in de meeste rechtsgebieden buiten de kernlanden
van de industriële natiestaten, die zich schuldig maakten aan
roofbelasting, de economische mogelijkheden beperkt en de
levensstandaard laag.

Neem de communistische systemen als voorbeeld. Net als veel regimes in
de Derde Wereld hieven zij doorgaans geen hoge inkomstenbelasting, of
helemaal geen. Toch zochten weinig, zo niet geen, ondernemers er tijdens
de driekwart eeuw van het bestaan de Sovjet-Unie hun toevlucht om
belastingen te ontlopen. Hoewel de Sovjet-inkomstenbelasting niet hoog
was, bood zij geen enkel voordeel, omdat de Sovjets er een deugd van
maakten om eigendomsrechten te ontkennen. Dat legde een nog zwaardere
last op dan belastingen. De communistische systemen maakten het vrijwel
onmogelijk een bedrijf te organiseren en serieus geld te verdienen. In
feite confisqueerde de communistische staat het inkomen vóór belasting.

Daarnaast, mocht iemand met een zeker inkomen om een excentrieke reden
ervoor gekozen hebben in Moskou of Havana te wonen, dan zou hij grote
moeite hebben gehad om met zijn geld een fatsoenlijke levensstandaard te
bereiken. Buiten toegang tot goede sigaren, kaviaar, uitstekende
orkesten en het ballet, bood het leven in de voormalige communistische
systemen weinig dingen om als consument van te genieten. Het merendeel
van de schaarse goede dingen des levens waren onverkrijgbaar of streng
gerantsoeneerd op basis van politieke invloed in plaats van vrije
uitwisseling. Al bevestigt het misschien het cliché van
postmodernisme-critici die ``het belang van consumptie in het
postmoderne leven'' benadrukken, toch heeft de toename van goederen en
diensten na de val van het communisme wereldwijd de concurrentie tussen
staten aangewakkerd en nationale en lokale bindingen verzwakt.

Onder het oude regime waren de consumentenkeuzes zo beperkt dat zelfs
Castro zelf, als hij stukjes cohiba uit zijn tanden had willen halen,
moeite zou hebben gehad om een fatsoenlijk pakje tandfloss te
bemachtigen. Tot voor kort konden zelfs de rijken in veel delen van de
wereld niet genieten van de levenskwaliteit die voor de middenklasse in
West-Europa of Noord-Amerika vanzelfsprekend was. De meeste mensen met
uitzonderlijk talent waren, geconfronteerd met deze sombere situatie,
geneigd dan maar de nationale verplichtingen van het Industriële
Tijdperk te accepteren. Ze bleven in het gebied waarin ze geboren waren
en betaalden buitensporig hoge belastingen voor de twijfelachtige
bescherming die geboden werd door de natiestaat met het monopolie op
geweld.

\begin{quote}
Het paradijs is nu gesloten en op slot, door engelen gebarricadeerd, dus
moeten we verder gaan, de wereld rond, en zien of er ergens toch nog een
achteringang te vinden is. -- HEINRICH VON KLEIST
\end{quote}

De val van het communisme lichtte een ``IJzeren Gordijn'' dat reizen
belemmerde en de globalisering van handel blokkeerde, waardoor de wereld
kunstmatig ``groot'' bleef. Het vliegtuig, gecombineerd met
communisme-ondermijnende informatietechnologie, wakkere concurrentie aan
voor het geld dat besteed werd aan reizen. Het komen en gaan van
bankiers, zelfs in de verste uithoeken, gaf een krachtige impuls aan de
kwaliteit van wonen en gastronomie overal ter wereld. Hiermee verwijzen
we niet naar de verspreiding van McDonald's hamburgers en Kentucky Fried
Chicken-filialen, zelfs niet in vroeger ontoegankelijke steden als
Moskou en Boekarest. Minder opgemerkt, maar belangrijker, is de
uitbreiding van toonaangevende hotelketens en hoogwaardige restaurants
die grand cru clarets serveren in plaats van vodka en cola.Dankzij deze
transformatie kan vrijwel iedereen die het zich kan veroorloven nu een
hoge levensstandaard genieten op bijna elke plek ter wereld. Er komen nu
bijna geen landen meer voor waar er geen eersteklas hotel is, en op zijn
minst één restaurant dat de interesse trekt van een Michelin-inspecteur.

Technologische ontwikkelingen hebben, zoals Hirschman eerder aangaf, het
veel aantrekkelijker gemaakt om te vertrekken als reactie op slechte
dienstverlening en prijsbeleid. Hij schreef: ``Loyaliteit aan het eigen
land daarentegen is iets wat we kunnen missen. \ldots{} Pas wanneer
landen op elkaar gaan lijken door de vooruitgang in communicatie en
algemene modernisering, ontstaat het gevaar van vroegtijdig en overmatig
vertrek, waarbij de `braindrain' een actueel voorbeeld is.'' Merk op,
zoals we in hoofdstuk 8 benadrukten, dat Hirschmans standaard van
``vroegtijdig en overmatig vertrek'' wordt bekeken vanuit het
perspectief dat de natiestaat wordt verlaten, niet vanuit het
perspectief dat het individu een beter leven zoekt.

Desalniettemin is zijn conclusie dat de gelijkenis tussen landen de
aantrekkingskracht van ontrouw en vertrek zal vergroten, onomstotelijk.
Het feit dat het nu gemakkelijker is om ergens goed te leven, maakt het
aantrekkelijk om te wonen waar de kosten het laagst zijn. Nog
belangrijker dan het feit dat je bijna overal goed kunt leven, is echter
dat je nu overal een hoog inkomen kunt verdienen. Het is niet langer
nodig om in een duur rechtsgebied te wonen om voldoende vermogen op te
bouwen om te leven zoals Lord Keynes adviseerde, ``wijs, aangenaam en
goed.'' Om redenen die we al hebben besproken, verandert
microtechnologie de onderliggende megapolitieke basis waarop de
natiestaat steunt. In het Informatietijdperk zal een nieuwe
cybereconomie ontstaan die buiten het vermogen van welke overheid dan
ook ligt om het te monopoliseren. Voor het eerst zal technologie
individuen in staat stellen om rijkdom te vergaren in een domein dat
niet gemakkelijk aan de eisen van systematische dwang kan worden
onderworpen.

De nieuwe samenleving, en dus de nieuwe cultuur, zal aan de ene kant
worden bepaald door wat machines beter kunnen dan mensen, door
automatisering die een toenemend aantal laaggeschoolde taken overneemt,
en aan de andere kant door de macht die informatietechnologie geeft aan
mensen die daadwerkelijk het talent hebben om daarvan te profiteren.
Zo'n samenleving zal grotere spanningen kennen tussen een kleine klasse,
die de informatie-aristocratie kan worden genoemd, en een groeiende
onderklasse, die de informatie-armen kan worden genoemd. Een van de
verschillen tussen hen is dat de informatie-armen geografisch gebonden
zullen zijn of weinig voordeel zullen hebben bij verhuizen. De
informatie-aristocratie, zoals we elders bespreken, zal extreem mobiel
zijn, aangezien zij overal geld kunnen verdienen waar het hen
aantrekkelijk lijkt, net zoals populaire romanschrijvers dat altijd
hebben kunnen doen. Robert Louis Stevenson kon honderd jaar geleden op
een eiland in de Stille Oceaan zijn brood verdienen; nu kan de
informatie-aristocratie overal hetzelfde doen.

\subsection{Marktconcurrentie tussen
rechtsgebieden}\label{marktconcurrentie-tussen-rechtsgebieden}

Omdat informatietechnologie de tirannie van plaats overstijgt, zal het
automatisch overal rechtsgebieden blootstellen aan wereldwijde
concurrentie op basis van kwaliteit en prijs. Met andere woorden,
overheden die lokale territoriale monopolies uitoefenen, zullen, net als
de meeste andere entiteiten, eindelijk onderworpen worden aan echte
marktconcurrentie op basis van hoe goed ze hun klanten bedienen. Dit zal
onvermijdelijk duidelijk maken dat de oude logica, die dure regimes in
het Industriële Tijdperk bevoordeelde, is omgekeerd. De vooraanstaande
natiestaten, met hun roofzuchtige, herverdelende belastingregimes en
zware regulering, zullen niet langer de voorkeursjurisdicties zijn.
Objectief bekeken bieden ze bescherming van lage kwaliteit en minder
economische kansen, en dat tegen monopolieprijzen. In de komende jaren
kunnen deze gebieden sociaal minder ontvankelijk en gewelddadiger zijn
dan delen van Azië en Latijns-Amerika, waar historisch een grotere
inkomensongelijkheid was. De vooraanstaande verzorgingsstaten zullen hun
meest getalenteerde burgers door desertie verliezen.

\subsection{Het ``extranationale'' tijdperk dat voor ons
ligt}\label{het-extranationale-tijdperk-dat-voor-ons-ligt}

Naarmate het tijdperk van het `Soevereine Individu' vorm krijgt, zullen
steeds meer bekwame mensen stoppen met zichzelf te definiëren als
onderdeel van een natie, als `brits', `amerikaans' of `canadees'. In het
nieuwe millennium ontdekken we een transnationaal of extranationaal
wereldbeeld en een geheel nieuwe manier om iemands plek daarin te
bepalen. Deze nieuwe identiteit komt niet voort uit de systematische
dwang die in de twintigste eeuw leidde tot de universele invoering van
natiestaten en het statelijke systeem.

Het simpele feit dat ontwikkelingen die de hele wereld omvatten
gewoonlijk als ``internationaal'' worden beschreven, laat zien hoe diep
het nationalistische paradigma is doorgedrongen in hoe we naar de wereld
kijken. Na twee eeuwen indoctrinatie in de mysteries van
``internationale betrekkingen'' en ``internationaal recht'' is het
gemakkelijk te vergeten dat ``internationaal'' geen oud Westers concept
is. In feite werd het woord \emph{international} uitgevonden door Jeremy
Bentham in 1789. Het werd voor het eerst gebruikt in zijn boek \emph{An
Introduction to the Principles of Morals and Legislation}. Bentham
schreef: ``Het woord `internationaal', moet worden toegegeven, is nieuw,
al wordt gehoopt dat het voldoende vergelijkbaar en begrijpelijk
is.''326 Het woord sloeg aan, maar niet alleen in de nauwe betekenis die
Bentham bedoelde. ``Internationaal'' werd een slordig synoniem voor
alles wat wereldwijd plaatsvond.

Het Internationale Tijdperk begon in 1789, hetzelfde jaar als de Franse
Revolutie. Het duurde twee eeuwen, tot 1989, toen de opstand tegen het
communisme in Europa begon. Wij geloven dat die tweede revolutie het
einde markeerde van het Internationale Tijdperk, en niet alleen omdat
het in diskrediet geraakte communistische volkslied ``De
Internationale'' heette. De planeconomie met staatsbezit was de meest
ambitieuze uitdrukking van de natiestaat. De nauwe relatie tussen
staatsmacht en nationalisme weerspiegelde zich in de taal. Het meest
agressieve werkwoord van de Moderne Tijd was ``nationaliseren,'' wat
betekende: onder staatsbezit en -controle brengen. Het was een woord dat
in het Internationale Tijdperk vlot van de tongen van demagogen rolde in
de meeste delen van de wereld. Nu behoort het tot de woordenschat van
het verleden. Nationalisatie is achterhaald geworden, juist omdat
staatsmacht achterhaald is.

In de schemering van het Moderne Tijdperk werd de geconcentreerde macht
van de staat ondermijnd door de wisselwerking tussen technologische
innovatie en marktkrachten. Nu staat de volgende fase in de triomf van
de markt op het punt zich te ontvouwen. Niet alleen zullen natiestaten
beginnen op te lossen, maar naar onze mening is zelfs de club van
natiestaten, de Verenigde Naties, gedoemd failliet te gaan. Het zou ons
niet verbazen als de VN kort na de eeuwwisseling geliquideerd zou
worden.

Als ``internationaal'' een aandeel was, zou dit het moment zijn om te
verkopen. Het concept zal waarschijnlijk in het nieuwe millennium worden
vervangen of op zijn minst worden teruggebracht tot de oorspronkelijke
betekenis, om de dwingende reden dat de wereld niet langer gedomineerd
zal worden door een systeem van interacterende soevereine naties.
Betrekkingen zullen de nieuwe ``extranationale'' vormen aannemen die
worden opgelegd door het groeiende belang van microjurisdicties en
Soevereine Individuen. Een conflict tussen een enclave aan de kust van
Labrador en een Soeverein Individu zal terecht niet als een
``internationaal'' geschil worden omschreven. Het zal extranationaal
zijn.

In het komende tijdperk zullen gemeenschappen en verbondenheid niet
territoriaal begrensd zijn. Identiteit zal eerder voortkomen uit echte
raakvlakken, zoals geloof, belangen en genen, niet uit de valse
verbanden waarop nationalisten hameren. Bescherming zal worden
georganiseerd op nieuwe manieren die niet kunnen worden vastgelegd met
een sextant, een schietlood of andere vroegmoderne instrumenten uit de
gereedschapskist van een landmeter die territoriale grenzen afbakenen.

\section{Uitgevonden gemeenschappen en
tradities}\label{uitgevonden-gemeenschappen-en-tradities}

Het idee dat mensen zich vanzelfsprekend moeten plaatsen in een
``uitgevonden'' gemeenschap genaamd een natie, zal in de volgende eeuw
door de kosmopolitische elite worden gezien als excentriek en
onredelijk, net zoals het dat zou zijn geweest gedurende het grootste
deel van het menselijke bestaan. De natiestaat, zoals socioloog Anthony
Giddens schreef, heeft ``geen precedent in de geschiedenis.'' Michael
Billig, een autoriteit op het gebied van nationalisme, verduidelijkte
dat punt:

\begin{quote}
In andere tijden hadden mensen niet de noties van taal en dialect, laat
staan die van grondgebied en soevereiniteit, die tegenwoordig zo
alledaags zijn en voor ``ons'' zo werkelijk lijken. Zo sterk zijn zulke
noties verankerd in het hedendaagse gezonde verstand dat het gemakkelijk
is te vergeten dat het uitgevonden permanenties zijn. De middeleeuwse
schoenmakers in de werkplaatsen van Montaillou of San Mateo lijken ons,
met een afstand van 700 jaar, misschien bekrompen, bijgelovige figuren.
Maar zij zouden onze ideeën over taal en natie vreemd mystiek hebben
gevonden. Zij zouden zich afgevraagd hebben waarom deze mystiek een
kwestie van leven en dood zou kunnen zijn.
\end{quote}

Wij vermoeden dat denkende mensen in de extranationale toekomst net zo
verbaasd zullen zijn. Zoals Benedict Anderson stelde, zijn naties
``ingebeelde gemeenschappen.'' Dat wil niet zeggen dat wat ingebeeld is
per se triviaal is. Zoals Dr.~Johnson opmerkte: als het niet om
verbeelding ging, zou een man ``even graag met een kamermeisje slapen
als met een hertogin.'' Toch kunnen ``naties'' voor degenen die in de
twintigste eeuw volwassen werden zo'n vanzelfsprekende
organisatie-eenheid lijken, dat het moeilijk is te bevatten dat zij
``ingebeeld'' en niet natuurlijk zijn. Om te begrijpen hoe verschillend
de toekomst kan zijn van de wereld die voor ons vertrouwd voelt, is het
noodzakelijk om te zien hoe nationalisme werd opgelegd aan het ``gezonde
verstand'' van het Industriële Tijdperk.

Men ziet vaak over het hoofd hoeveel de ``nationale gemeenschap''
afhankelijk is van een blijvende inspanning van de verbeelding. Er zijn
geen objectieve criteria om nauwkeurig te definiëren welke groep een
``natie'' zou moeten zijn en welke niet. Evenmin zijn er, strikt
genomen, ``natuurlijke grenzen,'' zoals vooraanstaande historici Owen
Lattimore en C. R. Whittaker hebben laten zien. ``Een grote imperiale
grens,'' schreef Lattimore over het keizerlijke China, ``is niet louter
een lijn die geografische regio's en menselijke samenlevingen verdeelt.
Het vertegenwoordigt ook de optimale grens voor de groei van één
bepaalde samenleving.'' Of zoals Ronald Findlay, econoom aan Columbia
University, het stelde: ``Voor zover ze in de economie al worden
beschouwd, worden de grenzen van een bepaald economisch systeem of
`land' over het algemeen als gegeven beschouwd, samen met de bevolking
die binnen die grenzen woont. Toch is het duidelijk dat, hoezeer deze
grenzen ook in het internationaal recht zijn geheiligd, ze ooit allemaal
betwist zijn geweest door rivaliserende aanspraken en uiteindelijk zijn
bepaald door de balans van economische en militaire macht tussen de
strijdende partijen.''

Iemand met alle beschikbare gegevens over de helft van de natiestaten in
de wereld en een verzameling gedetailleerde satellietkaarten zou niet in
staat zijn te voorspellen waar de grenzen van de andere natiestaten
zouden liggen. Er bestaat ook geen wetenschappelijke manier om
biologisch of taalkundig de leden van de ene nationaliteit van die van
een andere te onderscheiden. Geen enkele autopsie, hoe geavanceerd ook,
zou genetisch onderscheid kunnen maken tussen de resten van Amerikanen,
Canadezen en Soedanezen na een vliegtuigcrash. De grenzen tussen staten
en nationaliteiten zijn niet natuurlijk, zoals de grenzen tussen soorten
of de fysieke verschillen tussen dierenrassen. Het zijn artefacten van
vroegere en voortdurende pogingen om macht te projecteren.

\begin{quote}
Een taal is een dialect met een leger en een marine. -- MARIO PEI
\end{quote}

\section{Talen als artefacten van
macht}\label{talen-als-artefacten-van-macht}

Opmerkelijk genoeg geldt voor talen ongeveer hetzelfde principe. Na
eeuwenlange dominantie door natiestaten lijkt het onbezonnen, of zelfs
absurd om te beweren dat `taal' geen objectieve basis biedt om volkeren
van elkaar te onderscheiden. Maar bekijk het eens goed: de geschiedenis
van de moderne talen toont duidelijk hoe bewust ze zijn vormgegeven om
de nationalistische identiteit te versterken. De westerse `talen', zoals
wij ze vandaag de dag spreken en begrijpen, hebben zich niet op
natuurlijke wijze ontwikkeld tot hun huidige vorm. Men kan ze bovendien
niet objectief onderscheiden van `dialecten'. Tegenwoordig kiest vrijwel
niemand ervoor een `dialect' te spreken. Bijna iedereen verkiest dat
zijn moedertaal als authentiek wordt gezien, als een volwaardige `taal'.

\begin{quote}
Laat niemand zeggen dat het woord in zulke momenten weinig nut heeft.
Woord en daad vormen samen één. De krachtige, energieke bevestiging die
harten geruststelt, schept daden; dat wat gezegd wordt, wordt
gerealiseerd. De daad is hier de dienaar van het woord. Hij volgt
onderdanig, zoals op de eerste dag van de wereld: Hij zei, en de wereld
was. -- MICHELET, augustus 1792
\end{quote}

\subsection{`Woord en daad vormen samen
één'}\label{woord-en-daad-vormen-samen-uxe9uxe9n}

Voor de Franse Revolutie had de verfranste versie van het Latijn die in
Zuid-Frankrijk werd gesproken, la langue d'oc of Occitaans, bijvoorbeeld
meer gemeen met de volkstaal die in Catalonië in Noord-Spanje werd
gesproken dan met la langue d'oïl, de taal van Parijs die de basis werd
van het ``Frans.'' Toen de ``Verklaring van de Rechten van de Mens en de
Burger'' in Parijse stijl werd gepubliceerd, was die inderdaad
onbegrijpelijk voor de meerderheid van de bevolking binnen de huidige
grenzen van Frankrijk. Een van de uitdagingen waarmee de Franse
revolutionairen werden geconfronteerd, was uit te rekenen hoe zij hun
pamfletten en edicten konden vertalen in de patois van talloze dorpen
die elkaar slechts gedeeltelijk konden begrijpen.

De mensen die leefden binnen wat later ``Frankrijk'' werd, spraken zeer
verschillende talen die bewust beleidsmatig tot één officiële taal
werden samengevoegd. Geschreven Frans was sinds het Edict van
Villers-Cotterêts in 1539, uitgevaardigd door Frans I, de officiële taal
van de gerechtshoven. Maar dit betekende niet dat het wijdverbreid
verstaanbaar was, net zomin als het ``juridisch Frans'' dat in Engeland
na 1200 de officiële taal van de rechtbanken werd. Beide waren
``administratieve volkstalen,'' geen gestandaardiseerde talen die overal
in het gebied werden gesproken en begrepen.

De Franse revolutionairen wilden iets veelomvattenders creëren: een
nationale taal. Historicus Janis Langins merkt in \emph{The Social
History of Language} op dat ``een invloedrijke groep revolutionairen
geloofde dat de triomf van de Revolutie en de verspreiding van de
Verlichting zouden worden bevorderd door een bewuste poging om een
gestandaardiseerd Frans op te leggen in het grondgebied van de
Republiek.''

Deze ``bewuste poging'' omvatte een hoop gemierenneuk over het gebruik
van individuele woorden. Neem het sprekende voorbeeld van het
bijvoeglijk naamwoord ``revolutionair,'' voor het eerst gebruikt door
Marabou in 1789. Na een periode van ``enigszins wijd en willekeurig
gebruik,'' zoals Langins het formuleert, ``volgde tijdens de Terreur een
periode van onderdrukking en vergetelheid die enkele decennia duurde.
\ldots{} Op 12 juni 1795 besloot de Conventie de taal, net als de
instellingen van onze voormalige tirannen {[}d.w.z. de verslagen
Robespierristen{]}, te hervormen door het woord `revolutionair' in
officiële benamingen te vervangen.'' Deze traditie van taalengineering
leeft voort in de negatieve houding door de Franse autoriteiten ten
opzichte van woorden als ``weekend'' die vanuit het Engels het Frans
binnendringen.

Twee eeuwen geleden discrimineerden de nationale taalingenieurs in
Frankrijk niet alleen woorden van over het Kanaal, ze hadden een veel
grotere taak: het uitroeien van lokale taalvarianten binnen het
grondgebied van de republiek. Deze onderneming beperkte zich niet tot
het onderdrukken van la langue d'oc. Het ``Frans'' dat toen aan de
Rivièra werd gesproken, leek meer op het ``Italiaans'', dat verder naar
het oosten werd gesproken, dan op het Parijse Frans. De taal van de
Elzas had ook evengoed als een vorm van Duits kunnen worden
gecategoriseerd, dat zelf talrijke lokale varianten kende. In de
Pyreneeën werd Baskisch gesproken. Net als het Bretons, dat langs de
noordwestkust van Frankrijk werd gesproken, had Baskisch weinig gemeen
met de volkse ``dialecten'' van het Latijn die de basis vormden van het
``Frans.'' Ook waren er behoorlijk veel Vlaamssprekenden in het
noordoosten. ``De Parijse spreekstijl,'' herinnert Michael Billig ons,
``werd niet verspreid door spontane marktprocessen, maar wettelijk en
cultureel opgelegd als `Frans'.''

Wat in Frankrijk waar was, gold ook elders bij de opbouw van
natiestaten. Talen werden vaak door legers meegevoerd en opgelegd door
koloniale machten. Zo werd de kaart van Afrika na de onafhankelijkheid
gedefinieerd volgens de gebieden waar de administratieve talen van
Europese mogendheden de overhand hadden. Lokale dialecten werden zelden
onderwezen op scholen. De verschillen tussen erkende ``talen,'' die de
``naties'' definieerden, zelfs naties met arbitraire koloniale grenzen,
en ``dialecten,'' die dat niet deden, waren grotendeels politiek.

Kortom, de oplegging van een ``nationale taal'' was onderdeel van een
wereldwijd toegepast proces om de macht van de staat te versterken. Het
aanmoedigen of verplichten van het gebruik van de ``moedertaal'' binnen
het territorium van de staat, bracht grote voordelen voor de
machtsuitoefening.

\subsection{De militaire dimensie van
taaluniformiteit}\label{de-militaire-dimensie-van-taaluniformiteit}

In een wereld waarin geweld steeds meer opleverde, bood de invoering van
een nationale taal militaire voordelen. Een nationale taal was bijna een
voorwaarde voor de consolidatie van centrale macht in natiestaten.
Centrale autoriteiten die hun burgers aanmoedigden om dezelfde taal te
spreken, konden de militaire macht van lokale machthebbers beter
verzwakken. De standaardisering van taal na de Franse Revolutie maakte
de goedkoopste en meest effectieve vorm van modern militair geweld
haalbaar: nationale dienstplichtlegers. Een gemeenschappelijke taal
stelde troepen uit alle regio's van de ``natie'' in staat vloeiend met
elkaar te communiceren. Dit was een vereiste voordat massale
dienstplichtlegers de onafhankelijke bataljons konden verdringen die
niet door centrale autoriteiten, maar door machtige lokale magnaten
werden opgetrommeld en bestuurd.

Voor de Franse Revolutie, zoals we in hoofdstuk 5 hebben besproken,
werden troepen verzameld en aangevoerd door lokale machthebbers die wel
of geen gehoor gaven aan oproepen uit Parijs of een andere hoofdstad.
Hun houding werd in ieder geval bepaald na zorgvuldige onderhandeling.
Zoals Charles Tilly opmerkt, bood ``het vermogen om steun te geven of te
onthouden \ldots{} grote onderhandelingsmacht.'' Bovendien hadden
onafhankelijke militaire eenheden, vanuit het standpunt van de centrale
autoriteiten, het bijkomende nadeel dat ze in staat waren verzet te
bieden tegen pogingen van de overheid om lokale middelen in beslag te
nemen. Duidelijk is dat centrale autoriteiten, of het nu de koning of de
Revolutionaire Conventie was, grote moeite hadden om belastingen te
innen of anderszins middelen af te pakken van lokale machthebbers die
over privélegers beschikten die deze bezittingen konden verdedigen.

``Nationale legers'' versterkten de macht van de nationale regering
aanzienlijk waardoor ze haar wil op kon leggen in een heel gebied. Het
opleggen van een nationale taal speelde een duidelijke rol in het
vergemakkelijken van de vorming van nationale legers. Vooraleer
nationale legers konden ontstaan en effectief functioneren, was het
uiteraard nuttig dat hun leden vlot konden communiceren.

Het was dus een militair voordeel als iedereen binnen een rechtsgebied
bevelen en instructies kon begrijpen en ook bepaalde inlichtingen terug
kon koppelen langs de bureaucratische hiërarchie. De Franse
revolutionairen toonden dit voordeel vrijwel onmiddellijk aan. Naast het
opzetten van iets dat leek op een taalschool, organiseerden ze ook
speciale spoedcursussen van een maand waarin, zoals Langins schrijft,
``honderden studenten uit heel Frankrijk werden opgeleid in de
technieken van buskruit- en kanonnenproductie.''

Het militaire voordeel van de Franse aanpak werd aangetoond door hun
successen in de Napoleontische periode, maar ook door tegengestelde
voorbeelden van wat er gebeurde met regimes die niet konden rekenen op
de mobilisatievoordelen van een gemeenschappelijke taal tijdens een
oorlog. Een van de vele factoren die bijdroegen aan de rampzalige
nederlagen en demoralisatie van de Russische troepen in de beginfase van
de Eerste Wereldoorlog was het feit dat het aristocratische
officierskorps van de tsaar voornamelijk communiceerden in het Duits (de
andere hofstaal van de Romanovs was Frans), wat de gewone soldaten, laat
staan de burgers, niet begrepen.

Dit wijst op nog een belangrijk militair voordeel van een
gemeenschappelijke taal: het verlaagt de motivatiebarrières om oorlog te
voeren. Propaganda is nutteloos als ze onbegrijpelijk is. Ook in dit
opzicht waren de Franse revolutionairen goed afgestemd op de
mogelijkheden. Hun ``dominante idee,'' aldus Langins, was ``de wil van
het volk.'' Zij moesten zich dus identificeren met de volkswil door die
in de eigen taal uit te drukken. Voor 1789 was wederzijdse
onbegrijpelijkheid onder ``burgers'' een belemmering om de ``wil van het
volk'' te uiten en dus een rem op de machtsuitoefening op nationaal
niveau. Op meer dan één manier ondervonden meertalige staten en rijken
tijdens de industriële periode grotere uitdagingen bij de mobilisatie
voor een oorlog.

Aan de marge werden ze daarom vaak verdrongen door natiestaten die hun
burgers beter konden motiveren om te vechten en middelen te mobiliseren
voor oorlog. Dit wordt geïllustreerd door nationalistische consolidatie,
zoals de uitvinding van Frankrijk en de Fransen aan het einde van de
achttiende eeuw. Het wordt ook geïllustreerd door gevallen van
nationalistische devolutie, zoals de ineenstorting van het
Oostenrijks-Hongaarse Rijk na de Eerste Wereldoorlog. De nieuwe
natiestaten die ontstonden na de ondergang van het Habsburgse Rijk
(Oostenrijk, Hongarije, Tsjecho-Slowakije en Joegoslavië) waren, zoals
Keynes zei, ``onvolledig en onvolwassen.'' Toch wisten hun aanspraken op
het vormen van onafhankelijke natiestaten, gebaseerd op nationale
identiteiten die deels door taal werden bepaald, Woodrow Wilson en
andere geallieerde leiders bij het opstellen van het Verdrag van
Versailles te overtuigen.

Na de Eerste Wereldoorlog werd duidelijk dat taal zowel een hulpmiddel
als een probleem kon zijn bij het vormen van nieuwe staten in
Centraal-Europa. Wanneer de opbrengsten van geweld stegen,
vergemakkelijkte een gemeenschappelijke taal de machtsuitoefening en
consolideerde ze rechtsgebieden, maar wanneer de prikkels om te
consolideren zwakker waren, zorgden minderheden rond taalkwesties ook
vaak voor fragmentatie van meertalige staten. De opkomst van
separatistische gevoelens in de steden van het Oostenrijks-Hongaarse
Rijk in het midden van de negentiende eeuw volgde na epidemieën die de
Duitstalige bevolking hard raakten. Praag was een Duitstalige stad aan
het begin van de negentiende eeuw. Zoals andere steden groeide ze snel
in de loop van de eeuw, vooral door migratie, doordat grote aantallen
landloze Tsjechischtalige boeren uit het platteland kwamen. In het begin
moesten de nieuwkomers Duits leren om zich te redden, en dat deden ze.
Maar toen hongersnood en ziekte halverwege de eeuw veel Duitstalige
stedelingen wegvaagden, werden die vervangen door Tsjechischtalige
boeren. Plots waren er zoveel Tsjechischtaligen dat het voor de nieuwe
bewoners niet langer essentieel was om Duits te leren. Praag werd een
Tsjechischtalige stad en een broeinest van Tsjechisch nationalisme.

Hedendaagse separatistische bewegingen ontstaan nu vaak rond
taalkwesties in meertalige landen. Dit is duidelijk het geval in België
en Canada, twee landen die, zoals we eerder opmerkten, waarschijnlijk
tot de eerste in de OESO zullen behoren die in het nieuwe millennium
zullen uiteenvallen. Weinig overheden overtreffen de hardhandige
maatregelen die door de Parti Québécois in Quebec werden opgelegd om
taaleenheid af te dwingen. Meer verrassend is dat taalproblemen ook een
rol speelden bij de eerste activiteiten van de noordelijke separatisten
in Italië, dat eveneens met desintegratie wordt geconfronteerd. In het
begin van de jaren tachtig verklaarde de Lombardische Liga, zoals ze
toen heette, dat ``Lombardisch'' een aparte taal was en geen Italiaans.
Billig merkt op: ``Als het programma van de Liga in de vroege jaren
tachtig succesvol was geweest, en als Lombardije zich van Italië had
afgescheiden en zijn eigen staatsgrenzen had vastgesteld, zou een
voorspelling mogelijk zijn: Lombardisch zou steeds meer erkend zijn als
verschillend van Italiaans.'' Deze bewering is niet uit de lucht
gegrepen. Het weerspiegelt wat er in vergelijkbare gevallen is gebeurd.
Toen Noorwegen bijvoorbeeld in 1905 onafhankelijk werd, zetten Noorse
nationalisten een gecoördineerde poging in gang om kenmerken van de
``Noorse taal'' te identificeren en te benadrukken die verschilden van
het Deens en Zweeds. Evenzo veranderden activisten die een onafhankelijk
Wit-Rusland bepleitten de verkeersborden in het ``Wit-Russisch,'' maar
blijkbaar slaagden ze er niet in duidelijk te maken dat Wit-Russisch een
aparte taal is en geen dialect van het Russisch.

Nu de militaire noodzaak van taaluniformiteit grotendeels is verdwenen,
verwachten we dat de nationale talen zullen vervagen, maar niet zonder
strijd. Het is te verwachten dat het vaak herhaalde adagium dat ``oorlog
de gezondheid van de staat is'' zal worden getest als herstelmiddel.
Terwijl de natiestaat in irrelevantie wegzinkt, zullen demagogen en
reactionairen oorlogen en conflicten aanwakkeren, naar het voorbeeld van
de etnische en tribale gevechten die het voormalige Joegoslavië en
talrijke rechtsgebieden in Afrika, van Burundi tot Somalië, hebben
geteisterd. Conflicten zullen handig zijn als voorwendsel voor degenen
die de trend naar commercialisering van soevereiniteit willen
tegenhouden. Oorlogen zullen pogingen vergemakkelijken om strengere
belastingregimes in stand te houden en zwaardere straffen op te leggen
voor het ontduiken van de plichten en lasten van het staatsburgerschap.
Oorlogen zullen helpen de ``wij en zij''-dimensie van nationalisme te
ondersteunen. Voor de voorstanders van systematische dwang zal
commercieel georganiseerde soevereiniteit, waarbij individuen kunnen
kiezen op basis van prijs en kwaliteit, even zondig lijken als het
tijdens de Reformatie was dat individuen het recht opeisten om de
beslissingen van de paus te blokkeren en hun eigen weg naar verlossing
te kiezen.

De parallel wordt onderstreept door het feit dat zowel de nieuwe
technologie van de boekdrukkunst aan het einde van de vijftiende eeuw
als de nieuwe informatietechnologie aan het einde van de twintigste eeuw
voorheen verborgen kennis op een bevrijdende manier binnen het bereik
van individuen bracht. De drukpers bracht de Schriften en andere heilige
teksten rechtstreeks binnen het bereik van individuen die voorheen
afhankelijk waren van priesters en de kerkelijke hiërarchie om het Woord
van God te interpreteren. De nieuwe informatietechnologie brengt
informatie over handel, investeringen en actuele gebeurtenissen binnen
het bereik van iedereen met een computerverbinding, informatie die
voorheen alleen beschikbaar was voor mensen aan de top van de overheids-
en bedrijfshiërarchieën.

\begin{quote}
{[}D{]}e ontwikkeling van drukkunst en uitgeverij maakte het nieuwe
nationale bewustzijn mogelijk en bevorderde de opkomst van moderne
natiestaten. -- JACK WEATHERFORD
\end{quote}

\subsection{Rock and roll in
cyberspace}\label{rock-and-roll-in-cyberspace}

Vergis je niet, de opkomst van het Internet en het World Wide Web zal
nationalisme op een vergelijkbare manier verzwakken als dat buskruit en
de drukpers het ooit stimuleerden. Wereldwijde computernetwerken zullen
Latijn niet terugbrengen als universele taal, maar ze zullen de handel
helpen verplaatsen uit lokale dialecten, zoals Frans in Quebec, naar de
nieuwe wereldwijde taal van het internet en het World Wide Web, de taal
die Otis Redding en Tina Turner de wereld hebben geleerd, de taal van
rock-'n-roll, Engels.

Deze nieuwe media zullen nationalisme verzwakken door verbondenheden te
scheppen die de geografische grenzen overschrijden. Ze bereiken
wijdverspreide, opgeleide publieken en vormen niet-territoriale
affiniteiten die een nieuw soort ``patriottisme'' mogelijk maken, of
beter gezegd, nieuwe ``in-groups'' waar men zich mee kan identificeren
zonder hun economische rationaliteit op te moeten geven. De geschiedenis
van de Joden in de afgelopen tweeduizend jaar laat zien dat dit op de
lange termijn mogelijk is, zelfs onder vijandige lokale omstandigheden.
Zoals de opmerking van William Pfaff, geciteerd aan het begin van dit
hoofdstuk, suggereert, is het historisch gezien onjuist om te denken dat
loyaliteit aan het land van de voorouders, de patria, ook loyaliteit aan
een instelling die op een natiestaat lijkt, vereist. Geoffrey Parker en
Lesley M. Smith maken dit nog duidelijker in \emph{The General Crisis of
the Seventeenth Century}, waaruit blijkt dat wat lijkt op voorbeelden
van vroegmodern nationalisme vaker gevallen zijn van patriotten die een
veel beperktere patria verdedigen, vaak tegen de opmars van een staat.
Ze schrijven: ``Al te vaak blijkt een vermeende trouw aan een nationale
gemeenschap bij nader onderzoek helemaal niet zo te zijn. De patria kan
minstens zo waarschijnlijk een geboorteplaats of provincie zijn als de
hele natie.''

Zoals Jack Weatherford helder uitlegt in \emph{Savages and
Civilization}, had de opkomst van de drukpers, de eerste
massaproductietechnologie, dramatische effecten op de ontwikkeling van
politiek, met haar eisen van trouw aan een bredere natiestaat. Rond 1500
waren er drukpersen actief in 236 locaties in Europa, ``en zij hadden
samen zo'n 20 miljoen boeken gedrukt.'' Gutenbergs eerste gedrukte boek
was een editie van de Bijbel in het Latijn. Daarna volgden edities van
andere populaire middeleeuwse boeken in het Latijn. Zoals Weatherford
uitlegt, deed de drukpers de vroege verwachtingen dat de brede
beschikbaarheid van teksten het gebruik van Latijn en zelfs Grieks zou
verspreiden langzaam teniet. Integendeel, er waren twee belangrijke
redenen waarom de drukpers het gebruik van Latijn niet versterkte. Ten
eerste was de drukpers een massaproductietechnologie. Zoals Benedict
Anderson opmerkt: ``Waar manuscriptkennis zeldzaam en mysterieus was,
werd gedrukte kennis gekenmerkt door herhaalbaarheid en brede
verspreiding.'' Slechts een klein aantal Europeanen waren in 1500
meertalig. Dit betekende dat er geen massapubliek was voor werken in het
Latijn. De overgrote meerderheid die eentalig was, vormde een veel
grotere markt van potentiële lezers. Daarnaast gold hetzelfde nog meer
voor schrijvers. Uitgevers hadden producten nodig om te verkopen.

Omdat er weinig vijftiende- of zestiende-eeuwse auteurs waren die
interessante nieuwe werken in het Latijn konden schrijven, werden
uitgevers door marktwerking gedreven om werken in de volkstaal uit te
geven. Drukwerk droeg zo bij aan de differentiatie van Europa in
linguïstische subsets. Dit werd niet alleen aangemoedigd door de
publicatie van nieuwe werken die de identiteit van nieuwe talen
vestigden, zoals Spaans en Italiaans, maar ook door de adoptie van
karakteristieke lettertypes, zoals Romeins, Italic en het zware Gotische
schrift dat gebruikelijk was in de Duitse publicaties tot ver in de
twintigste eeuw. De nieuwe volkstaalpublicaties, door Anderson aangeduid
als ``printkapitalisme,'' waren zeer succesvol. De drukpers gaf vooral
ketterij de beslissende impuls die wij ook verwachten voor de
denationalisering van het individu via het internet. In het bijzonder
werd Luther ``de eerste bestverkopende auteur. Of om het anders te
zeggen, de eerste schrijver die zijn nieuwe boeken kon `verkopen' op
basis van zijn naam.'' Verbazingwekkend genoeg vertegenwoordigden
Luthers werken ``niet minder dan een derde van alle Duitstalige boeken
die tussen 1518 en 1525 werden verkocht.''

In veel opzichten zal de technologie van het Informatietijdperk de
impact van de vijftiende-eeuwse drukpers, die natiestaten hielp
opbouwen, gedeeltelijk tenietdoen. Het World Wide Web creëert een
commercieel platform met een wereldtaal, het Engels. Het zal
uiteindelijk worden versterkt met simultane vertalingssoftware, waardoor
bijna iedereen praktisch meertalig wordt en taal en verbeelding worden
denationaliseerd. Net zoals de technologie van de drukpers de trouw aan
de dominante instelling van de Middeleeuwen, de Heilige Moederkerk,
ondermijnde, zo verwachten wij dat de nieuwe communicatietechnologie van
het Informatietijdperk de autoriteit van de verzorgingsstaat zal
ondermijnen. Uiteindelijk zal bijna elk gebied meertalig worden. Lokale
dialecten zullen in belang toenemen. Propaganda vanuit het centrum zal
veel van haar samenhang verliezen naarmate immigranten en sprekers van
minderheidstalen zich gesterkt voelen om assimilatie in de natie te
weerstaan.

\section{Militaire mystiek}\label{militaire-mystiek}

In tegenstelling tot objectieve gemeenschappen, zoals bijvoorbeeld
``jagers-verzamelaarsbenden'' objectief zijn, worden naties ingebeeld
uit een soort mystiek, geïnspireerd door een verouderde militaire
noodzakelijkheid. De noodzaak om elke persoon binnen een territorium te
verbinden met een gevoel van identiteit dat belangrijker kan lijken dan
het leven zelf. Zoals Kantorowicz opmerkte, is het geen toeval dat ``op
een bepaald moment in de geschiedenis de staat in abstracto of de staat
als corporatie verscheen als een corpus mysticum, en dat de dood voor
dit nieuwe mystieke lichaam gelijk leek te zijn aan de dood van een
kruisvaarder voor de zaak van God!'' In deze zin kan de natiestaat
worden gezien als een mystieke constructie. Toch, zoals Billig opmerkt,
is nationalisme ``een banaal mysticisme, zo banaal dat al het mysticisme
al lang lijkt te zijn verdampt.'' Het ``bindt `ons' aan het thuisland,
die bijzondere plaats die meer is dan slechts een plaats, meer dan een
louter geofysisch gebied. In dit alles wordt het thuisland als huiselijk
voorgesteld, onbetwistbaar en, indien nodig, zoveel waard om jezelf
ervoor op te offeren. En mannen in het bijzonder, worden geconfronteerd
met hun unieke, genotrijke herinneringen aan wat opoffering kan
betekenen.''

De denkbeeldige link tussen natie en thuis wordt bij elke gelegenheid
door nationalisten benadrukt. Zoals Billig suggereert, wordt de natie
``ingebeeld als huiselijke ruimte, gezellig binnen haar grenzen, veilig
tegen de gevaarlijke buitenwereld. En `wij', de natie binnen het
thuisland, kunnen ons zo gemakkelijk inbeelden als een soort familie.''
De clichés van het nationalisme, onvermoeibaar en routinematig herhaald,
bevatten veel alledaagse metaforen van verwantschap en identiteit. Ze
koppelen de natie aan het gevoel van ``inclusive fitness,'' een
krachtige drijfveer voor altruïsme en opoffering.

\begin{quote}
Dat opofferend altruïsme bestaat bij sociale insecten, andere
niet-menselijke dieren en mensen, impliceert dat maximalisatie van
eigenbelang niet uitsluitend kan worden gedefinieerd in termen van de
wensen en behoeften van een individueel organisme. Inderdaad, de
aanwezigheid van altruïsme, met name tegenover verwanten, vereiste een
herziening van traditionele opvattingen over survival of the fittest in
de biologische wetenschappen. Dit heeft geleid tot een groeiende
overtuiging dat natuurlijke selectie uiteindelijk niet op het individu
opereert. -- R. Paul Shaw en Yuwa Wong
\end{quote}

\section{Nationalisme en inclusive
fitness}\label{nationalisme-en-inclusive-fitness}

We richten ons in dit boek voornamelijk op objectieve ``megapolitieke''
factoren die de kosten en opbrengsten van menselijke keuzes veranderen.
De onderliggende premisse waarop de voorspellende kracht van de analyse
rust, is dat mensen beloningen zullen zoeken en kosten zullen vermijden.
Dit is een essentiële waarheid die Charles Darwin ``de economie van de
natuur'' noemde. Het is echter niet de hele waarheid. Eenvoudige
beloningsoptimalisatie verklaart niet alles in het leven. Het werpt
echter wel licht op twee van de drie belangrijkste vormen van menselijke
socialiteit, door Pierre Van Den Berghe geïdentificeerd als
``wederkerigheid en dwang.'' Met ``wederkerigheid'' bedoelt Van Den
Berghe ``samenwerking voor wederzijds voordeel.'' De meest complexe en
verstrekkende voorbeelden van wederkerigheid zijn marktinteracties:
handel, kopen, verkopen, produceren en andere economische activiteiten.
``Dwang is het gebruik van geweld voor eenzijdig voordeel, oftewel voor
doeleinden van intraspecifiek parasitisme of roofzucht.'' Zoals we in
dit boek en in twee eerdere boeken hebben onderzocht, geloven wij dat
dwang een cruciaal element is in de menselijke samenleving, belangrijker
dan meestal wordt erkend. Dwang helpt de veiligheid van eigendom te
bepalen en beperkt het vermogen van individuen om wederzijds voordelige
samenwerking aan te gaan. Dwang ligt ten grondslag aan alle politiek.
Het derde element in Van Den Berghe's typologie van menselijke
socialiteit is ``verwantenselectie,'' het coöperatieve gedrag dat dieren
vertonen met hun verwanten. Verwantenselectie, dat hieronder
uitgebreider wordt beschreven, is ook een cruciaal kenmerk van de
``economie van de natuur.''

Zoals Jack Hirshleifer heeft geschreven: ``De hernieuwde toepassing van
Darwins selectietheorie op sociale gedragsvraagstukken, ook wel
sociobiologie genoemd,'' heeft ``een duidelijk economische aspect.'' En:

\begin{quote}
Kijkende naar de hele leefwereld, probeert de sociobiologie de algemene
wetten te vinden die de veelvormige associatiepatronen tussen organismen
bepalen. Bijvoorbeeld: Waarom zien we soms seks en families, soms seks
zonder families, soms noch seks noch families? Waarom vormen sommige
dieren groepen, terwijl anderen solitair blijven? Waarom zien we binnen
groepen soms hiërarchische dominantiepatronen en soms niet? Waarom delen
organismen in sommige soorten territoria op, en anderen niet? Wat
bepaalt de onbaatzuchtigheid van de sociale insecten, en waarom is dit
patroon zo zeldzaam in de natuur? Wanneer worden middelen vreedzaam
verdeeld, wanneer door middel van geweld? Dit zijn vragen die zowel
gesteld als beantwoord worden in een duidelijk economisch kader.
Sociobiologen vragen wat de netto voordelen zijn van de waargenomen
associatiepatronen voor de organismen die ze vertonen, en wat de
mechanismen zijn waardoor deze patronen voortduren in sociale
evenwichtstoestanden. Het is wellicht deze stelling van
economisch-gedragsmatige continuïteit tussen mens en andere levensvormen
(door een criticus ``genetisch kapitalisme'' genoemd) die de
vijandigheid van sommige ideologen tegenover de sociobiologie verklaart.
\ldots{}
\end{quote}

Wij introduceren sociobiologie in onze analyse van nationalisme omdat
het perspectief biedt op aspecten van de menselijke natuur die
systematische dwang helpen faciliteren. Wij zijn het eens met
natuurwetenschapper Colin Tudge, auteur van \emph{The Time Before
History}, dat we, voordat we de huidige wereld kunnen begrijpen, laat
staan een perspectief krijgen op wat nog komen gaat, eerst het voorwoord
van de geschiedenis moeten begrijpen. Dat betekent dat we ``onszelf
moeten bekijken op grote tijdschaal.'' Tudge herinnert ons eraan ``dat
onder de oppervlakkige trillingen van ons leven veel diepere en sterkere
krachten werkzaam zijn die uiteindelijk ons allemaal en al onze
medeschepselen beïnvloeden\ldots{} Wij vermoeden dat onder deze diepere
en sterkere krachten een genetisch beïnvloed component ligt dat
nationalisme motiveert. Hirshleifer parafraseert Adam Smith en R. H.
Coase:''Menselijke verlangens zijn uiteindelijk adaptieve reacties,
gevormd door de biologische natuur van de mens en zijn situatie op
aarde.'' Dit treedt duidelijk op de voorgrond door de evidente
biologische verwijzingen in de meeste discussies over nationalisme.
Zelfs in de Verenigde Staten, een opvallend multi-etnische natie, wordt
de overheid in familiale termen gepersonifieerd als ``Uncle Sam.''

\section{De biologische erfenis}\label{de-biologische-erfenis}

Kortom, de menselijke natuur, de oorsprong van soorten, en hun
ontwikkeling door natuurlijke selectie zijn elementen die overwogen
moeten worden wanneer we de voortdurende evolutie van de menselijke
samenleving willen begrijpen. In dit geval bekijken we de
waarschijnlijke menselijke reactie op nieuwe omstandigheden die
voortkomen uit informatietechnologie. In het bijzonder richten we ons op
de reactie op de komst van de cybereconomie en haar vele consequenties,
waaronder de opkomst van economische ongelijkheid die sterker is dan
alles wat in het verleden is gezien. Sleutels tot ten minste een deel
van de verwachte reactie liggen in onze genetische erfenis.

Wanneer een nieuwe soort wordt gevormd, gooit deze niet al het DNA weg
dat zij in haar eerdere vorm droeg, maar voegt eraan toe. Het gehele
verschil tussen een mens en een chimpansee zit in elke soort in minder
dan 2 procent van het DNA. Iets meer dan 98 procent van het DNA is
gemeenschappelijk, en een deel daarvan kan worden teruggevoerd tot zeer
primitieve vroege organismen, diep in de historische ontwikkelingsketen.

\section{Genetische inertie}\label{genetische-inertie}

Menselijke culturen bevatten op vergelijkbare wijze elementen die
universeel zijn, waarvan sommige inderdaad zijn geërfd van
pre-menselijke voorouders. Hoe we voedsel zoeken, hoe we ons
voortplanten, hoe we families vormen, hoe we ons verhouden tot vreemde
groepen, hoe we ons verdedigen, het zijn allemaal complexe mengsels van
instinct en cultuur, met zeer primitieve wortels. Ze hebben ook allemaal
de mogelijkheid tot moderne aanpassingen, zoals degene die de natiestaat
in de moderne periode hebben gekenmerkt. Als we culturen op deze manier
beschouwen, zullen we ze zien als parallel aan genetische ontwikkeling.
De drie grote verschillen zijn dat culturen worden overgedragen via de
informatieketen tussen mensen, niet via de genetische keten tussen
generaties; dat ze tot op zekere hoogte, wellicht minder dan we denken,
kunnen worden veranderd door bewuste, intelligente actie; en dat ze
veranderen met de heersende omstandigheden van kosten en baten, die veel
sneller muteert dan genetische verandering. Fysiek lijken we sterk op
onze voorouders van dertigduizend jaar geleden; cultureel zijn we daar
ver van verwijderd geraakt.

\subsection{Evolutionaire modellen}\label{evolutionaire-modellen}

Er bestaan twee biologische modellen die de evolutie van soorten
proberen te verklaren. De heersende wetenschappelijke consensus volgt de
neo-darwinistische benadering: willekeurige genetische mutaties leiden
tot diverse fysieke varianten. De meeste van deze vormen bieden geen
overlevingsvoordeel, zoals bijvoorbeeld te zien is bij de albino merel,
en sterven daarom doorgaans uit. Slechts enkele varianten bevorderen
overleving en verspreiden zich binnen de soort. Hoewel deze theorie nog
gepaard gaat met veel moeilijkheden, problemen die wetenschappers in de
komende eeuw wellicht oplossen, vormt het idee dat willekeur en het
overleven van voordelige aanpassingen centraal staan wel de basis van de
huidige wetenschappelijke orthodoxie. Een alternatief model is een
variant op de theorie van de vroege twintigste-eeuwse Franse filosoof
Henri Bergson, die stelde dat de natuur een niet-willekeurig, creatief
doel nastreeft, een intelligente kracht die voortdurend op zoek is naar
oplossingen. Dit idee vindt weerklank in het werk van hedendaagse
denkers zoals \emph{David Layzer} en \emph{Stephen Jay Gould}, die
benadrukken dat genetische variatie niet louter willekeurig verloopt,
maar duidelijke tendensen vertoont. Het gaat hier niet over creationisme
in de strikte bijbelse zin, maar het omzeilt wel veel van de problemen
die het orthodoxe darwinisme met zich meebrengt.

\begin{quote}
De belangrijkste theoretische bijdrage van de sociobiologie ligt in de
uitbreiding van het begrip van `fitness' naar dat van `inclusive
fitness.' Een dier kan immers zijn genen rechtstreeks doorgeven via zijn
eigen nageslacht of indirect via de voortplanting van verwanten met wie
het een bepaald aantal genen deelt. Daarom mag men verwachten dat
dieren, naarmate zij genetisch nauwer verwant zijn, zich coöperatiever
opstellen en zo elkaars overlevingskansen vergroten. Dit principe noemt
men verwantenselectie. Kortom, dieren vertonen nepotisme: zij geven de
voorkeur aan verwanten boven niet-verwanten, en aan naaste verwanten
boven verre verwanten. Bij mensen kan dit bewust gebeuren, maar vaker
verloopt het onbewust. -- PIERRE VAN DEN BERGHE*
\end{quote}

\section{Genetisch beïnvloede
motivatie}\label{genetisch-beuxefnvloede-motivatie}

Het biologische perspectief op menselijk gedrag werd in 1963 versterkt
door de introductie van het concept ``inclusive fitness'' door W. D.
Hamilton in \emph{The Evolution of Altruistic Behavior.} Hamilton merkte
op dat hoewel mensen van nature op zichzelf gericht zijn, ze ook af en
toe altruïstische of zelfopofferende handelingen verrichten die geen
schijnbare voordelen opleveren voor het individu zelf. Om deze
schijnbare tegenstrijdigheden te verklaren stelde Hamilton voor dat de
fundamentele eenheid die gemaximaliseerd wordt niet het individuele
organisme is, maar het gen. Individuen van elke soort zullen proberen
niet alleen hun eigen welzijn te maximaliseren, maar wat Hamilton hun
``inclusive fitness'' noemde. ``Inclusive fitness'' omvatte volgens hem
niet alleen persoonlijke overleving in Darwiniaanse zin, maar ook de
verbeterde voortplanting en overleving van naaste verwanten die dezelfde
genen delen. Hamiltons thesis van ``inclusive fitness'' helpt diverse
intrigerende kanten van menselijke samenlevingen te verklaren, waaronder
aspecten van de politiek in natiestaten.

\subsection{Altruïsme: misbenaming of fossiele
verwantenselectie?}\label{altruuxefsme-misbenaming-of-fossiele-verwantenselectie}

Volgens Van Den Berghe is ``altruïsme voornamelijk gericht op verwanten,
vooral op nauwe verwanten, en is het feitelijk een verkeerde benaming.
Het vertegenwoordigt het ultieme genetische egoïsme. Het is slechts de
blinde uitdrukking van het maximaliseren van inclusive fitness.'' Dit
betekent echter niet dat er geen altruïsme bestaat buiten de nauwe
genetische verwantschap die Hamilton en Van Den Berghe beschrijven. De
onzekerheden die voortkomen uit het feit dat mensen zich seksueel
voortplanten in plaats van via aseksuele kloning, zorgen ervoor dat een
neiging tot ``maximalisatie van inclusive fitness'' veel ``altruïsme''
kan stimuleren dat ten goede komt aan andere allelen dan het
``egoïstische gen.'' Zo kan iemand die een persoon helpt waarvan hij
veronderstelt dat hij verwant is, in feite geen nauwe verwant zijn.
Bijvoorbeeld, een vader die zichzelf opoffert voor zijn kinderen, kan
niet de werkelijke biologische vader zijn, maar slechts denken dat hij
dat is.

Zoals Hirshleifer opmerkt, zijn veel van de paradoxen van ``altruïsme''
semantische verwarringen die mensen misleiden door de context van
competitie uit het oog te verliezen, waarin ``helpen'' een
overlevingsvoordeel kan opleveren: ``Als een altruïstische strategie
levensvatbaar wil zijn in competitie met niet-altruïsme, moet altruïsme
meer bijdragen aan zelfoverleving dan niet-altruïsme, en daarom is het
eigenlijk geen echt altruïsme.'' Al deze verwarring zou kunnen worden
vermeden als we de term ``altruïsme'' laten vallen en in plaats daarvan
zouden vragen: wat zijn de bepalende factoren van het volledig
objectieve fenomeen dat ``helpen'' kan worden genoemd?

Het concept is vooral fascinerend bij ``verwantenhulp''. Hamiltons
inclusive fitness analyseert het biologisch: het individu, of het gen
dat hulpgedrag bepaalt, hecht evenveel belang aan het overleven van een
identieke kopie als aan zijn eigen overleven. De bereidheid om te helpen
of zich op te offeren varieert dus met de kans dat een ander individu
hetzelfde gen deelt. Concreet instrueert een gen voor het helpen van
verwanten dus een mens (alle andere omstandigheden gelijk) om zijn leven
te geven als hij daarmee twee broers/zussen, vier halfbroers/-zussen,
acht neven/nichten, enzovoort kan redden.

\section{Waarschijnlijkheidsproblemen van inclusieve
fitheid}\label{waarschijnlijkheidsproblemen-van-inclusieve-fitheid}

Hoewel deze biologische benadering in principe duidelijk lijkt, maskeert
ze bij nadere beschouwing een aantal moeilijkheden. Zo betekent het feit
dat iemands broers, zussen of kinderen een kans van 50 procent hebben om
een identiek gen te delen, strikt logisch gezien niet dat dit gen
daadwerkelijk in hen tot uitdrukking komt. Elk individu draagt twee sets
van elk gen, één van de vader en één van de moeder. Dit betekent echter
dat slechts de helft van de genen die een ouder draagt,
noodzakelijkerwijs aanwezig is in de nakomelingen. Bovendien is er
altijd het risico van mutatie bij de voortplanting, wat, hoe
onwaarschijnlijk ook, de zekerheid van genetische kosten-batenanalyse
vermindert. Als de metafoor van het ``gen als optimizer'' serieus wordt
genomen, is het geval van de vader die niet de biologische ouder is
slechts het meest duidelijke voorbeeld van een breder probleem. Als
inderdaad het overleven van het ``egoïstische gen'' wordt
geoptimaliseerd door opoffering voor naaste verwanten, dan kan elke
mogelijkheid die resulteert in de vervanging van een ander allel voor de
identieke kopie van het ``egoïstische gen'' worden beschouwd als een van
die ingewikkelde trucs van Moeder Natuur.

\subsection{Onzekere gevolgen}\label{onzekere-gevolgen}

Altruïsme gericht op verwanten brengt daarom problemen met zich mee.
Niet alleen is er het waarschijnlijkheidsprobleem voor het ``egoïstische
gen'' dat schijnbare verwanten van de gastheer mogelijk niet
daadwerkelijk identieke kopieën delen. Er is ook de moeilijkheid om
onder onzekere somstandigheden te bepalen of een gegeven daad van
opoffering daadwerkelijk primair ten goede zal komen aan verwanten in
plaats van aan anderen. (Een opoffering die primair anderen ten goede
komt, kan het inclusive fitness van het egoïstische gen schaden door de
kans te verkleinen dat het in volgende generaties aanwezig zal zijn.)
Overweeg een schrijnend voorbeeld geïnspireerd door het nieuws dat zich
terwijl we dit schreven afspeelt. Stel dat een ouder in Dunblane,
Schotland, hoort dat een gewapende gek een lokale school nadert met de
duidelijke intentie om schade aan te richten. Door onmiddellijk te
handelen, zou hij of zij het heroïsche maar mogelijk gedoemde gebaar
kunnen maken om de gek te confronteren en daarmee mogelijk zijn of haar
kinderen op de school te redden. Of misschien ook niet.

Zelfs een meedogenloze gek die erop uit is elk kind op aarde te doden,
zou beperkt zijn in de schade die hij kan aanrichten voordat hij zonder
munitie komt te zitten of door anderen wordt overmeesterd. Als de
opofferende ouder had besloten niet in te grijpen, zouden de meeste
kinderen waarschijnlijk toch hebben overleefd, zoals de meeste kinderen
op die school deden. Alle schade die een heldhaftige daad van opoffering
zou hebben voorkomen, zou waarschijnlijk anders zijn gevallen op de
kinderen van anderen. Door zijn of haar leven te riskeren, primair voor
de kinderen van anderen, zou de betreffende vader of moeder het
``inclusive fitness'' mogelijk hebben verminderd. Door al zijn kinderen
van een van hun ouders te beroven, zou hij waarschijnlijk die kinderen
in een slechtere positie hebben achtergelaten in de Darwiniaanse strijd.

Hoewel dit voorbeeld ongetwijfeld geforceerd is, is het ook realistisch.
Het weerspiegelt het feit dat er talloze omstandigheden in het leven
zijn waarin grote of kleine helpende handelingen gunstige effecten
hebben. In veel gevallen kunnen dergelijke handelingen niet gemakkelijk
alleen worden gericht op nauwe verwanten. En ironisch genoeg, zoals we
hieronder zullen bespreken, kan dit deel uitmaken van het
overlevingsvoordeel dat degenen met minder selectieve hulpgenen in staat
stelde alle millennia van ontbering tot nu toe te doorstaan.

\subsection{Altruïsme en genetische
inertie}\label{altruuxefsme-en-genetische-inertie}

Als de ``egoïstische gen''-these, zoals wij aannemen, een nauwkeurige
benadering is van wat menselijk gedrag motiveert, zou het te eenvoudig
zijn om te veronderstellen dat het helpende of opofferende gedrag dat
het voortbrengt uitsluitend voor het voordeel van echte verwanten werkt.
Onvolledige kennis maakt het in sommige omstandigheden onzeker om
verwanten te onderscheiden. En zelfs als de verwantschap bekend is, kan
het werkelijke tot uiting komen van een bepaald ``egoïstisch gen'' in de
populatie van verwanten slechts op basis van kansberekening worden
ingeschat. Tot voor kort zou het onmogelijk zijn geweest om
daadwerkelijke genetische merkpunten tussen individuen te onderscheiden.
En we zijn nog steeds ver verwijderd van praktische mogelijkheden om te
bepalen welke naaste verwanten daadwerkelijk het ``egoïstische gen'',
dat zijn overleving optimaliseert, tot expressie brengen.

Daarbovenop is het een groter probleem om voordelen te beperken tot
alleen verwanten. Bovendien is het uit ervaring duidelijk dat mensen hun
``zorginstincten'' soms richten op niet-verwanten wanneer eigenlijke
verwanten niet beschikbaar zijn. Het meest duidelijke voorbeeld hiervan
is het gedrag van ouders tegenover adoptiekinderen, of zelfs het gedrag
van bepaalde personen, meestal kinderloos, tegenover hun huisdieren. Het
is niet ongebruikelijk dat zulke personen ernstig gevaar lopen of zelfs
overlijden bij pogingen om katten uit bomen te redden. Wat geldt voor
huisdieren geldt des te meer voor adoptiekinderen. Het is zeker niet
overdreven te zeggen dat ouders van adoptiekinderen hen vaak behandelen
``alsof'' ze verwant zijn, waardoor het concept van
``verwantenselectie'' een bredere betekenis krijgt.

In tegenstelling tot wat sommige critici zouden wensen, brengen
dergelijke gevallen de theorie van het ``egoïstische gen'' niet in
diskrediet. Integendeel, wij zien voorbeelden van mensen die hun
inclusive fitness willen bevorderen door te handelen ``alsof'' ze zich
opofferen voor nauwe verwanten als voorbeelden van ``genetische
traagheid.'' Met andere woorden, ze weerspiegelen het feit, opgemerkt
door Howard Margolis in \emph{Selfishness, Altruism and Rationality},
dat ``de menselijke samenleving sneller veranderde'' dan het menselijke
genetische materiaal. Mensen blijven daarom handelen ``alsof ze in wezen
in een kleine jagers-verzamelaarsgroep leven.'' Een cruciaal kenmerk van
zulke groepen was, zoals Van Den Berghe het formuleerde:

\begin{quote}
``Ze waren kleine, inteelt-populaties van enkele honderden individuen.
\ldots{} Leden van de stam, hoewel onderverdeeld in kleinere
familiegroepen, zagen zichzelf als een enkel volk, alleen tegen de
buitenwereld, en verbonden door een netwerk van familiebanden en
huwelijken dat de stam in feite tot een superfamilie maakte. Een hoge
mate van inteelt verzekerde dat de meeste echtgenoten ook verwanten
waren.''
\end{quote}

Kortom, gedurende de hele menselijke geschiedenis vóór landbouw waren
etnische groepen ``inteelt-superfamilies.'' Gezien deze vroegere
identiteit tussen familie en in-group, kan er een genetisch beïnvloedde
neiging zijn om de in-group als verwanten te behandelen. Het is goed
voor te stellen dat dergelijk gedrag in het verleden een
overlevingswaarde had toen elk lid van de ``inteelt-superfamilie''
verwant was. Zoals Margolis suggereert, is het goed voorstelbaar dat
voor ``zulke kleine, nauw verwante jagers-verzamelaarsgroepen inclusief
egoïsme (naast enig vooruitzicht op wederkerigheid of wraak) op zichzelf
al een mate van toewijding aan groepsbelangen zou ondersteunen. Men kan
dan betogen dat een zekere neiging tot groepsgerichte motivatie als een
soort fossiel verwantschapsaltruïsme voortleeft.'' Met andere woorden,
omdat wij het genetische materiaal van jagers-verzamelaars behouden,
weerspiegelt ons gedrag tegenover in-groups het soort ``altruïsme'' dat
de overlevingskansen van in-groups bestaande uit
``inteelt-superfamilies'' zou optimaliseren.

Verondersteld wordt, zoals Margolis speculeert, dat deze neiging tot
groepsgericht gedrag, voortkomend uit ``fossiel verwantschapsaltruïsme''
of genetische traagheid, heeft bijgedragen aan het overleven van
\emph{Homo sapiens} ``terwijl andere humanoïde soorten uitsterven.''

\subsection{Epigenese}\label{epigenese}

Wij zien dit ``alsof''-gedrag als een duidelijk voorbeeld van
``epigenese,'' of de genetisch beïnvloede motivatie van mensen om
bepaalde keuzes boven andere te verkiezen. Met andere woorden, de
menselijke geest is geen \emph{tabula rasa}, of blanco blad, maar een
harde schijf met vooraf bedrade circuits die bepaalde reacties
gemakkelijker te leren en aantrekkelijker maken dan andere. Zo is de
veronderstelling dat de geest geneigd is om te denken in termen van een
out-group die vijandigheid of antagonisme oproept en een in-group
waarvoor men grote genegenheid of loyaliteit voelt, meestal gereserveerd
voor verwanten.

Deze epigenetische neiging om te handelen alsof de in-group uit nauwe
verwanten bestaat, creëert een kwetsbaarheid voor manipulatie die vaak
door nationalisten is uitgebuit om opofferende steun voor de staat te
genereren. In die zin is het geen toeval dat nationalistische propaganda
overal in de taal van verwantschap is verpakt.

\begin{quote}
Bij het alarm van haar kanonnen roept het mooie Frankrijk haar kinderen
op. Soldaten om ons heen bewapenen zich. Vooruit, vooruit, 't is onze
moeder die roept. -- LIED VAN FRANSE SOLDATEN
\end{quote}

\subsection{Vals verwantschap}\label{vals-verwantschap}

Beschouw de sterke neiging van politici overal om de staat te
beschrijven met termen ontleend aan het verwantschap. De natie is ``ons
vaderland'' of ``ons moederland.'' Haar burgers zijn ``wij,'' ``leden
van de familie,'' onze ``broers en zussen.'' Het feit dat staten die
cultureel zo verschillend zijn als Frankrijk, China en Egypte dergelijke
vergelijkingen gebruiken, is volgens ons geen retorisch toeval, maar een
duidelijk voorbeeld van ``epigenese,'' of de genetisch beïnvloede
motivatie van mensen om bepaalde keuzes boven andere te verkiezen.

Hoe werkt deze epigenese? Om emotionele trouw loyaliteit aan de
natiestaat te mobiliseren, gebruikt men identificatiemiddelen die
vroeger kenmerkend waren voor verwantschap, zodat de zorgen van een
individu over inclusive fitness aansluiten bij de belangen van de staat.
Shaw en Wong richten zich bijvoorbeeld op vijf identificatiemiddelen die
moderne natiestaten gebruiken om hun bevolking tegen ``out-groups'' te
mobiliseren. Dit zijn:

\begin{quote}
\begin{enumerate}
\def\labelenumi{\arabic{enumi}.}
\tightlist
\item
  een gemeenschappelijke taal
\item
  een gedeeld vaderland
\item
  vergelijkbare fenotypische kenmerken
\item
  een gedeeld religieus erfgoed en
\item
  het geloof in een gemeenschappelijke afstamming.
\end{enumerate}
\end{quote}

Natuurlijk zouden dergelijke kenmerken in het primitieve verleden de
etnische kerngroep hebben onderscheiden. Een groot deel van de
aantrekkingskracht van nationalisme kan worden herleid tot de manier
waarop deze identificatiemiddelen zijn overgenomen en aangekleed in de
taal van verwantschap, zoals geïllustreerd in het eerder geciteerde
Franse soldatenlied. Dergelijke mobilisatiemiddelen, die de staat
aanduiden als ``vaderland'' of ``moederland,'' zijn wereldwijd
gebruikelijk omdat ze effectief zijn.

\subsection{Genetische boekhouding}\label{genetische-boekhouding}

Het imaginaire karakter van de vermeende verwantschapsbanden tussen
burgers en de staat blijkt uit het gebrek aan variatie dat je bij echte
verwantschappen wel aantreft. Ook binnen grote families geldt dat de ene
verwantschap nauwer is dan de andere. Ouders en broers of zussen zijn de
meest nabije verwanten, grootouders en neven of nichten staan verder
weg, terwijl verre achterneven zo ver verwijderd zijn dat zij nauwelijks
meer kans hebben dan vreemden om een bepaald gen te delen. Tegenwoordig
delen echtgenoten doorgaans geen nauwe verwantschap, in tegenstelling
tot wat in het Stenen Tijdperk wel het geval was. Men definieert echte
verwantschap wiskundig via de `verwantschapscoëfficiënt', een maat voor
genetische overlap die Hamilton berekende.

De nationale `familie' wordt daarentegen voorgesteld alsof het volledig
samenvalt met de flexibele territoriale grenzen van de staat.
Nationaliteit verspreidt zich als een vloeistof gelijkmatig uit in
iedere spleet binnen die strikt afgebakende perken. Benedict Anderson
stelt: `In de moderne opvatting geldt de soevereiniteit van de staat
volledig, vlak en gelijkmatig over elke vierkante centimeter van een
wettelijk afgebakend territorium.' En, natuurlijk, wanneer het om
opoffering voor de staat gaat, speelt de coëfficiënt van de denkbeeldige
verwantschap altijd een rol.

Deze koppeling van inclusive fitness aan de natiestaat is interessant
omdat het inzicht kan geven in de houding van mensen om de veranderingen
van het nieuwe millennium te verwelkomen of af te wijzen. Zoals we
eerder hebben besproken, berustten alle samenlevingsvormen vóór het
Informatietijdperk op territoriale grenzen. Men organiseerde zich ofwel
rondom het thuisgebied van de etnische kerngroep, of, zoals bij de
natiestaat, maakten ze gebruik van hetzelfde gevoel van
groepssolidariteit om te mobiliseren voor de verdediging van een lokaal
grondgebied tegen buitenstaanders. Het was steeds de vreemdeling buiten
het eigen directe grondgebied die als vijand werd gevreesd, wat vanuit
de aannames van verwantenselectie in de oertijd volstrekt logisch was.
Toen de mens in zijn huidige genetische vorm opkwam, waren stamleden
nauw verwant en behoorden zij tot een etnische kern, een
`inteelt-superfamilie.'

Vanuit de principes van verwantenselectie was er bovendien daadwerkelijk
een praktische economische reden om de voorspoed en overleving van
directe verwanten gelijk te stellen aan die van de hele stam of
superfamilie. Een lid van een jager-verzamelaarsstam was voor zijn eigen
welzijn immers sterk afhankelijk van het succes van de gehele stam. Er
was geen sprake van onafhankelijk bezit, noch had een individu of
familie de mogelijkheid om los van de stam te overleven en te floreren.
Hierdoor raakte het eigenbelang van het individu nauw verweven met dat
van de groep. Zoals Hirshleifer verwoordde: `Voor zover de leden van een
groep een gemeenschappelijk lot delen, wordt het elkaar helpen gezien
als een vorm van zelfhulp.'

\begin{quote}
Blijkbaar stelt de primitieve mens, en de Lovedu kunnen worden gezien
als representatief voor honderden vergelijkbare volkeren, een
samenleving waarin op elk moment ieders positie exact gelijk is als
norm. -- HELMUT SCHOECK
\end{quote}

\subsection{Nieuwe omstandigheden, oude
genen}\label{nieuwe-omstandigheden-oude-genen}

Nu maakt microtechnologie de creatie mogelijk van heel andere
omstandigheden dan die waarvoor wij door de omstandigheden van het
Stenen Tijdperk genetisch waren toegerust. Informatietechnologie creëert
economische ongelijkheid van een omvang die ver buiten het bereik ligt
van alles wat onze voorouders in het oorspronkelijk egalitaire Stenen
Tijdperk hebben meegemaakt. Informatietechnologie schept ook
supraterritoriale bezittingen, die zullen helpen om de belichaming van
de in-groep, de natiestaat, te ondermijnen. Ironisch genoeg zullen deze
nieuwe cyberbezittingen waarschijnlijk meer waarde hebben juist omdat ze
buiten de eigen thuisbasis zijn gevestigd. Dit geldt des te meer als er,
zoals wij verwachten, een afgunstige tegenreactie komt tegen de
economische ongelijkheid die voortvloeit uit de toenemende penetratie
van informatietechnologie in de rijke industriële landen. Juist dat feit
zou bezittingen op grote afstand waardevoller maken. Ze zullen niet
alleen minder blootstaan aan afgunst, maar er is ook een grotere kans
dat ze buiten het bereik worden gehouden van de meest roofzuchtige groep
waarmee een individu moet omgaan, zijn eigen natiestaat.

\subsection{Onevenwichtigheden van Natuur en
Nationalisme}\label{onevenwichtigheden-van-natuur-en-nationalisme}

Het geringe besef van de ironieën rond in-groepidentificatie in relatie
tot de moderne natiestaat kan worden gezien als bewijs voor de rol van
epigenese bij het vormen van houdingen. De logica van geweld in de
moderne periode ondermijnde juist de impuls die oorspronkelijk
aanleiding gaf tot de neiging om fitness met de in-groep te
identificeren. Waarom? Omdat het koppelen van de individuele ``inclusive
fitness'' aan een nationale groep, in plaats van aan het voortbestaan en
de voorspoed van naaste verwanten, de betekenis van persoonlijke
opofferingen voor verwanten tot verwaarloosbare proporties verminderde.
De typische moderne natiestaat was eenvoudigweg te groot om een
statistisch significante ``verwantschapscoëfficiënt'' mogelijk te maken
tussen het individu en andere burgers die dezelfde natie claimen. Niet
alleen nam het aandeel naaste verwanten binnen de in-groep drastisch af
van bijna eenheid in het Stenen Tijdperk tot een vrijwel verwaarloosbaar
niveau in de twintigste eeuw, de ``verwantschapscoëfficiënt'' tussen de
individuele burger en de rest van de natie was in de meeste gevallen
nauwelijks hoger dan in relatie tot de gehele mensheid. Een in-groep van
tientallen miljoenen of zelfs honderden miljoenen (of in het geval van
de Chinezen meer dan een miljard leden) werd zo gigantisch dat het
inclusive fitness-effect van wat voor opoffering of voordeel dan ook
verwaterde tot de schaal van een druppel op een gloeiende plaat. In
strikte logica kon de moderne nationalist, in tegenstelling tot de
jager-verzamelaar uit het Stenen Tijdperk, dan ook niet redelijkerwijs
verwachten dat een daad van opoffering of hulp voor zijn ``in-groep'' de
overlevingskansen van zijn familie op een betekenisvolle manier zou
vergroten.

Ondanks het feit dat nationale economieën de fundamentele rekeneenheden
werden waarin welzijn in de moderne tijd werd gemeten, werd het grootste
obstakel voor het succes van het getalenteerde individu, en daarmee voor
dat van zijn verwanten, gevormd door de lasten die in naam van de natie,
de in-groep zelf, werden opgelegd. Dit gold in elk geval voor degenen
die zich primair bezighielden met wederkerigheid in plaats van met
dwingende socialiteit, om terug te grijpen op Van Den Berghes
categorieën van menselijk gedrag.

De logica van de natiestaat suggereert dat de ultieme prijs van
burgerschap opoffering en dood is. Zoals Jane Bethke Elshtain opmerkte,
indoctrineren natiestaten hun burgers meer voor opoffering dan voor
agressie: ``De jongeman gaat niet zozeer naar de oorlog om te doden,
maar om te sterven, om zijn eigen lichaam op te offeren voor dat van het
grote lichaam, het lichaamspolitiek.'' De impuls tot opoffering is niet
minder actief waar het de belastingbetaler betreft. Belasting betalen
is, net als het dragen van wapens, een plicht, geen ruil waarbij men
geld afstaat om een product of dienst van gelijke of grotere waarde te
verkrijgen. Dit wordt in het dagelijks spraakgebruik erkend. Men spreekt
van een ``belastingdruk'' zoals men niet spreekt van de ``voedseldruk''
bij het kopen van voedingsmiddelen, of de ``autodruk'' bij de aanschaf
van een auto, of een ``vakantiedruk'' bij het reizen, juist omdat
commerciële aankopen over het algemeen eerlijke ruilen zijn. Anders zou
men de aankoop niet doen.

In dit opzicht laat nationalisme zien hoe epigenese de logica van de
Darwiniaanse ``economie van de natuur'' kan omkeren. De natiestaat
faciliteerde systematische op territorium gebaseerde roofzucht. In
tegenstelling tot de situatie waarmee jagers-verzamelaars in het Stenen
Tijdperk werden geconfronteerd, was de belangrijkste parasiet en
roofdier van het individu aan het einde van de twintigste eeuw
waarschijnlijk niet de''buitenstaander,'' de buitenlandse vijand, maar
veeleer de veronderstelde belichaming van de ``in-groep,'' de lokale
natiestaat zelf. Het belangrijkste voordeel van activa die de
territoriale grenzen overstijgen in het Informatietijdperk is dus dat ze
buiten het bereik geplaatst kunnen worden van de systematische dwang van
de lokale natiestaat waarin het potentiële Soevereine Individu verbleef.

Als onze visie klopt, zal microtechnologie het technisch haalbaar maken
voor mensen om grotendeels te ontsnappen aan de lasten van ondergeschikt
burgerschap. Ze zullen in de nieuwe ``Virtuele Stad'' extranationale
soevereinen over zichzelf zijn, geen onderdanen, waarbij ze trouw
verschuldigd zijn via contract of privaat verdrag op een manier die meer
doet denken aan het premoderne Europa, waar kooplieden commerciële
verdragen en charters afsloten om zichzelf te beschermen ``tegen
willekeurige inbeslagnames van eigendom'' en om ``vrijstelling van
seigneuriaal recht'' te verkrijgen. In de cybercultuur zullen
succesvolle personen vrijstellingen krijgen van de burgerplichten die
hem door de geboorte werden opgelegd. Ze zullen zichzelf niet langer
primair zien als Britten of Amerikanen. Ze zullen extranationale
inwoners van de hele wereld zijn, die toevallig in een of meer van haar
locaties verblijven.

\section{De Cybereconomie en onze genetische
erfenis}\label{de-cybereconomie-en-onze-genetische-erfenis}

Het knelpunt is echter dat dit technologische en economische wonder, het
ontsnappen aan de tirannie van plaatsgebondenheid, afhangt van de
bereidheid van individuen om een groot deel van hun vermogen en toekomst
toe te vertrouwen aan onbekenden. Vanuit strikt genetisch perspectief
zouden die onbekenden niet per se genetisch minder verwant zijn dan de
meeste ``medeburgers'' waarvan we in recente eeuwen afhankelijk waren.

De vraag is of de perverse gevolgen van in-groepvriendelijkheid in het
geval van de natiestaat negatieve of positieve indicatoren zijn voor de
cybereconomie. Zullen de ``achterblijvers'' die de voordelen van
dwingende herverdeling verliezen, de ondergang van de natiestaat
beschouwen alsof het een aanval op hun verwanten is? Het eerste kwart
van de eenentwintigste eeuw zal dit uitwijzen. De emotionele reacties
kunnen complex zijn. Het feit dat 115 miljoen mensen in de twintigste
eeuw hun leven gaven voor natiestaten, is een duidelijke indicatie van
de kracht van epigenese. Het toont dat velen de overleving van hun
natiestaten beschouwden als een zaak van leven of dood. De vraag is of
deze houding zich zal voortzetten in een nieuw tijdperk met andere
megapolitieke imperatieven.

Het feit dat genetisch beïnvloede opoffering ten behoeve van de
natiestaat vaak tegen de evolutionaire doelstelling van
verwantenselectie inging, laat ook zien dat mensen flexibel genoeg zijn
om zich aan te passen aan omstandigheden waarvoor we genetisch niet
geprogrammeerd waren in het Stenen Tijdperk. Zoals Tudge beschrijft over
de ``extreme generaliteit'' van de mens: ``Wij zijn het dierlijke
equivalent van de Turingmachine: het universele apparaat dat op elke
taak kan worden ingesteld.'' Welke neiging aan de oppervlakte zal komen
tijdens de komende overgangscrisis? Waarschijnlijk beide.

De commercialisering van soevereiniteit hangt op haar beurt af van de
bereidheid van honderdduizenden Soevereine Individuen en vele miljoenen
anderen om hun bezittingen in de ``First Bank of Nowhere'' te zetten om
immuniteit tegen directe dwang te verkrijgen. Dit type vertrouwen heeft
geen duidelijk equivalent in het primitieve verleden. Er waren weinig
bezittingen in het Stenen Tijdperk. Wat er was, werd gehamsterd onder
controle van een stam, een ``inteelt-superfamilie'' die zich
achterdochtig opstelde tegenover buitenstaanders. Desalniettemin biedt
de cybereconomie, ondanks haar evolutionaire nieuwheid, de mens de kans
om ons meest innovatieve genetische erfgoed tot uitdrukking te brengen:
de intelligentie die gepaard gaat met onze grote hersenen. Leden van de
informatie-elite zullen zeker slim genoeg zijn om een goede kans te
herkennen wanneer ze die zien.

Verder zou de creatie van bezittingen die grotendeels immuun zijn voor
roof, in praktische zin daadwerkelijk de ``inclusive fitness'' van
Soevereine Individuen kunnen vergroten. Terwijl de economische logica
van deelname aan de cybereconomie de logica van de natiestaat op zijn
kop zet, is deze bijzonder overtuigend, vooral voor mensen met hoge
vaardigheidsniveaus.

Om hun voordeel te optimaliseren bij het ``shoppen'' tussen
jurisdicties, moeten mensen bereid zijn de natiestaat te verlaten en hun
persoonlijke bescherming toe te vertrouwen aan veiligheidsmedewerkers
die hoofdzakelijk door marktprikkels worden gemotiveerd, in gebieden die
mogelijk ver verwijderd zijn van waar ze geboren en opgegroeid zijn. Dit
impliceert een significant voordeel van meertaligheid en kosmopolitische
cultuur boven jingoïsme. Bovendien impliceert het dat iedereen die
serieus is over het realiseren van het bevrijdende potentieel van de
cybereconomie voor zichzelf en zijn familie, moet beginnen met zich in
meerdere jurisdicties buiten die waarin hij zijn belangrijkste loopbaan
heeft doorgebracht, te laten verwelkomen. Voor meer details, zie onze
bespreking van strategieën voor het bereiken van onafhankelijkheid in de
bijlagen.

\subsection{Oprechte affiniteiten}\label{oprechte-affiniteiten}

Een vernieuwd, extranationaal wereldbeeld en een andere manier om je
plek in de wereld te bepalen, kunnen onze culturele gewoonten, zo niet
zelfs onze aangeboren neigingen, ingrijpend veranderen. De
extranationale invulling van identiteit die in het nieuwe millennium
opkomt, maakt de aanpassing aan een veranderde wereld wellicht
eenvoudiger dan je zou denken. In tegenstelling tot nationaliteit
ontstaan deze nieuwe identiteiten niet door de systematische dwang die
in de twintigste eeuw natiestaten en het natiestaat-systeem universeel
oplegde. In het nieuwe tijdperk binden gemeenschappen en loyaliteiten
zich niet langer aan vaste territoria. Je identiteit zal steeds meer
gebaseerd zijn op oprechte affiniteiten, gedeelde belangen of werkelijke
verwantschap, in plaats van op de gefabriceerde banden van burgerschap
die de traditionele politiek onvermoeibaar nastreeft. Bescherming zal op
geheel nieuwe wijze georganiseerd worden, die niet te vergelijken is met
het meetgereedschap waarmee een landmeter territoriale grenzen afbakent.
Activa komen steeds vaker in de cyberspace terecht in plaats van op een
vaste locatie, wat een nieuwe vorm van concurrentie om de
`beschermingskosten', oftewel de belastingen, omlaag te brengen zal
stimuleren in de meeste territoriale jurisdicties.

\begin{quote}
`Ambitieuze mensen beseffen dat een migrerende levensstijl de prijs is
die je betaalt om vooruit te komen.' - Christopher Lasch
\end{quote}

\section{Ontsnappen aan de
natiestaat}\label{ontsnappen-aan-de-natiestaat}

Ondanks de sterke greep die de natiestaat als ``in-groep'' op de moderne
verbeelding heeft gehad, zullen bekwame mensen die nog niet twijfelen
aan het nut van aansluiting bij een schandalig dure ``ingebeelde
gemeenschap'' dat binnenkort wel doen. Inderdaad, de verdedigers van de
natiestaat zijn al begonnen te klagen over de groeiende loskoppeling van
de cognitieve elites. De overleden Christopher Lasch valt in zijn tirade
\emph{The Revolt of the Elites and the Betrayal of Democracy} degenen
aan ``wier levensonderhoud niet zozeer berust op het bezit van eigendom
maar op de manipulatie van informatie.'' Lasch beklaagt zich over het
extranationale karakter van de opkomende Informatiemaatschappij. Hij
schrijft:

\begin{quote}
De markten waarin de nieuwe elites opereren, zijn nu internationaal van
omvang. Hun fortuin is verbonden met ondernemingen die grenzen
overschrijden. Ze zijn meer begaan met het soepele functioneren van het
systeem als geheel dan met dat van de afzonderlijke delen. Hun
loyaliteiten, als die term in deze context al niet achterhaald is, zijn
internationaal in plaats van regionaal, nationaal of lokaal. Ze hebben
meer gemeen met hun tegenhangers in Brussel of Hong Kong dan met de
massa Amerikanen die nog niet aangesloten zijn op het netwerk van
wereldwijde communicatie.
\end{quote}

Hoewel Lasch verre van een onpartijdige observator was, en hij zijn
portret van de informatie-elite duidelijk onvleiend bedoelde, rust zijn
minachting voor hen die bevrijd zijn van de tirannie van plaats op een
perceptie van een aantal van dezelfde ontwikkelingen waar wij in dit
boek over schrijven. Wanneer wij de kritieken van Lasch lezen, of die
van Mickey Kaus (\emph{The End of Equality}), Michael Walzer
(\emph{Spheres of Justice}), of Robert Reich (\emph{The Work of
Nations}), zien wij delen van onze analyse, vaak ongelukkig, bevestigd
door auteurs die diep onsympathiek staan tegenover veel van de gevolgen
van de verdieping van markten, laat staan de denationalisatie van
Soevereine Individuen.

Lasch hekelt degenen met extranationale ambities ``die uit zijn op
lidmaatschap van de nieuwe hersen-aristocratie'' omdat ze ``banden
cultiveren met de internationale markt in snel geld, glamour, mode en
populaire cultuur.'' Hij vervolgt:

\begin{quote}
Het is de vraag of ze zichzelf überhaupt nog als Amerikanen beschouwen.
Patriottisme staat in ieder geval niet hoog in hun rangorde van deugden.
``Multiculturalisme'' daarentegen past hen perfect, omdat het het
aangename beeld oproept van een mondiale bazaar waar men zorgeloos kan
genieten van exotische keukens, kledingstijlen, muziek en tribale
gebruiken, zonder vragen of verplichtingen. De nieuwe elites voelen zich
alleen thuis onderweg, op weg naar een conferentie van hoog niveau, naar
de grootse opening van een nieuwe franchise, naar een internationaal
filmfestival of een onontdekte vakantiebestemming. Hun perspectief is in
wezen dat van een toerist, geen perspectief dat waarschijnlijk een
hartstochtelijke toewijding aan democratie zal bevorderen.
\end{quote}

\subsection{Economisch nationalisme}\label{economisch-nationalisme}

Achter de kritiek op de ``voorbijgangers'' die de virtuele
gemeenschappen van het Informatietijdperk vormen, schuilt de erkenning
dat voor velen in de elite de voordelen van voorbijgangerschap al groter
zijn dan de nadelen. Critici zoals Lasch en Walzer betwisten niet dat
een heldere kosten-batenanalyse burgerschap overbodig maakt voor mensen
met een hoog vaardigheidsniveau. Zij beweren niet dat degenen binnen de
informatie-elite, wier houding zij verachten, zich hebben vergist in wat
hun belangen zijn. Evenmin doen ze alsof de samengestelde-rentetabellen
werkelijk zouden aantonen dat het blijven pompen van geld in een
nationaal socialezekerheidsprogramma, laat staan inkomstenbelasting, een
beter rendement oplevert dan particuliere investeringen. Integendeel,
zij begrijpen de rekenkunde. Ze hebben de berekeningen tot hun
vanzelfsprekende conclusies gevolgd. Maar in plaats van de ondermijnende
logica van economische rationaliteit te erkennen, deinzen ze ervoor
terug en bestempelen ze het als ``verraad'' wanneer de informatie-elite
de tirannie van plaats overstijgt en de ``onverlichten'' achterlaat.

Net als Pat Buchanan zijn de sociaaldemocraten economische nationalisten
die de triomf van markten boven politiek verafschuwen. Zij hekelen ``de
nieuwe hersen-aristocratie'' omdat die losstaat van plaats en zich niet
vurig bekommert om hun opvatting van waar de belangen van de massa
zouden liggen. Hoewel zij de denationalisatie van het individu niet
expliciet erkennen, hekelen zij de eerste tekenen en manifestaties
ervan, wat Walzer omschrijft als ``het imperialisme van de markt,'' of
de neiging van geld om ``grenzen over te sijpelen'' om dingen te kopen
die, zoals Lasch benadrukt, ``niet te koop zouden mogen zijn,'' zoals
vrijstelling van militaire dienst. Let op de reactionaire
terugverwijzing naar de militaire eisen van de natiestaat als heilige
grond waarop geld en markten niet mogen treden.

Deze kritieken op de informatie-elite voorzien de voorwaarden van een
volksreactie tegen de opkomst van Soevereine Individuen in het volgende
millennium. Naarmate er nieuwe, meer marktgestuurde vormen van
bescherming beschikbaar komen, zal het voor grote aantallen bekwame
personen steeds duidelijker worden dat de meeste vermeende voordelen van
nationaliteit denkbeeldig zijn. Dit zal niet alleen leiden tot een
betere berekening van de opportuniteitskosten van burgerschap, maar ook
tot nieuwe manieren om zogenaamd ``politieke'' en zelfs ``economische''
vraagstukken te formuleren. Voor het eerst zal ``een individuele
ondernemer, voor en door zichzelf handelend'' zijn eigen
beschermingskosten kunnen veranderen door tussen jurisdicties te
bewegen, zonder te hoeven wachten tot deze door ``groepsbeslissing en
groepsactie'' worden doorgevoerd, om Frederic C. Lane's formulering van
een oud dilemma te citeren.

Wanneer de prijs die wordt betaald voor bescherming onderworpen wordt
``aan het substitutieprincipe,'' zal dit de rekenkunde van dwang
blootleggen en het conflict intensiveren tussen de nieuwe
kosmopolitische elite van het Informatietijdperk en ``de
informatie-armen,'' de rest van de bevolking die grotendeels eentalig is
en niet uitblinkt in probleemoplossend vermogen of beschikt over een
wereldwijd verhandelbare vaardigheid. Deze ``verliezers'' of
``achterblijvers,'' zoals Thomas L. Friedman ze noemt, zullen
ongetwijfeld hun welzijn blijven identificeren met het politieke leven
van bestaande natiestaten.

\section{De meeste politieke agenda's zullen reactionair
zijn}\label{de-meeste-politieke-agendas-zullen-reactionair-zijn}

De meeste mensen met een uitgesproken politieke agenda, of die nu
nationalistisch, milieugeoriënteerd of socialistisch is, zullen zich bij
de start van de eenentwintigste eeuw verzamelen om de wankelende
natiestaat te verdedigen. Na verloop van tijd zal het duidelijk worden
dat het behoud van de natiestaat en nationalistische gevoelens
noodzakelijk zijn om politieke dwang in stand te houden. Zoals Billig
stelt, is nationalisme ``de voorwaarde voor conventionele (politieke)
strategieën, ongeacht de specifieke politiek.'' Daarom zal het
nationalistische gehalte in alle politieke programma's de komende jaren
aanzwellen. Milieuactivisten zullen zich bijvoorbeeld minder richten op
de bescherming van ``Moeder Aarde'' en meer op de bescherming van het
``moederland.'' Voor wie gelijkheid hoog in het vaandel draagt, zullen
natie en burgerschap extra heilig worden. Zij zullen, meer dan zij nu
wellicht beseffen, instemmen met Christopher Lasch, die Hannah Arendt
volgde in de uitspraak: ``Het is burgerschap dat gelijkheid verleent,
niet gelijkheid dat een recht op burgerschap schept.''

Het privatiseren van soevereiniteit breekt de band tussen
welvaartscheppers en hun natie en plaats, waardoor de fixatie op
gelijkheid uit het Industriële Tijdperk zijn meerwaarde verliest.
Burgerschap zal niet langer functioneren als mechanisme om
inkomensherverdeling af te dwingen, via het principe van stemgelijkheid
binnen een territorium. Dit zal opnieuw een harde klap uitdelen aan het
progressieve beeld van de geschiedenis. Anders dan zogenaamd
vooruitziende mensen aan het begin van de twintigste eeuw verwachtte,
heeft de vrije markt de twintigste eeuw niet alleen overleefd maar er
zelfs als winnaar uitgekomen. De marxisten voorzagen dat de ondergang
van het kapitalisme, die nooit kwam, zou leiden tot de ondergang van de
natiestaten en het ontstaan van een universeel klassenbewustzijn onder
arbeiders. In werkelijkheid zal de staat verdwijnen, maar op een heel
andere manier. Het tegenovergestelde gebeurt: de triomf van het
kapitalisme leidt tot de opkomst van een nieuwe mondiale, of
extranationale, bewustzijnsvorm onder kapitalisten, van wie velen
Soevereine Individuen zullen worden. In tegenstelling tot de
marxistische verwachting dat de staat nodig zou zijn om arbeiders te
disciplineren, blijkt het dat juist de meest capabele en vermogende
individuen netto verliezers waren door de acties van de natiestaat. Zij
hebben dan ook het meest te winnen bij het overstijgen van nationalisme
nu markten het zullen winnen van dwang.

Misschien niet direct, maar spoedig, zeker binnen één generatie, zal
vrijwel iedereen uit de informatie-elite zijn inkomensactiviteiten
onderbrengen in jurisdicties met lage of geen belasting. Naarmate het
Informatietijdperk de wereld transformeert, zal het een onmiskenbare les
in samengestelde rente zijn. Binnen jaren, laat staan decennia, zal
algemeen bekend zijn dat bijna ieder getalenteerd persoon een veel hoger
nettovermogen kan opbouwen en een beter leven kan leiden door de
natiestaten met een hoge belastingdruk te verlaten. We hebben al gehint
naar de enorme kosten die deze natiestaten opleggen, maar omdat dit de
kern raakt van een vaak slecht begrepen probleem, is het nodig om de
opportuniteitskosten van een nationaliteit nogmaals te benadrukken.

\subsection{Opportuniteitskosten}\label{opportuniteitskosten}

Het verlies of de inperking van overheidsdiensten, die nu met hoge
belastingen worden gefinancierd, zullen de informatie-elite geen leed
veroorzaken maar juist ongeëvenaard doen floreren. Alleen al door te
ontsnappen aan de buitensporige belastingdruk die zij nu dragen,
realiseren ze een enorme marge om het materiële welzijn van hun families
te verbeteren. Zoals eerder aangegeven, zorgt elke €5.000 aan jaarlijkse
belastingbetalingen voor een vermindering van het levenslange
nettovermogen met €2,4 miljoen, als je 10 procent rendement per jaar
kunt behalen op je investeringen. Bij 20 procent rendement, zou elke
€5.000 aan jaarlijkse belastingbetalingen je zelfs €44 miljoen armer
maken over een periode van veertig jaar. Cumulatief zou €5.000 belasting
per jaar je dus meer dan een miljoen dollar per jaar kosten. Aan dat
tempo zou €250.000 per jaar aan belasting al snel resulteren in een
jaarlijks verlies van meer dan €50 miljoen, of €2,2 miljard in een
mensenleven. En natuurlijk betekenen incidenteel hogere opbrengsten,
zelfs als dit maar enkele jaren gebeurt, vooral vroeg in het leven, een
nog schokkender verlies aan vermogen door roofzuchtige belastingheffing.

De auteurs hebben tot hun eigen tevredenheid vastgesteld dat rendementen
hoger dan 20 procent mogelijk zijn. Hun collega's bij Lines Overseas
Management op Bermuda behaalden rendementen in de driecijferige orde,
gemiddeld 226 procent per jaar, tijdens de jaren dat dit boek werd
geschreven. Hun ervaring onderstreept wat de spreadsheet al suggereert:
dat veel grootverdieners en kapitaalbezitters gedurende hun levensloop
een groot fortuin kwijt zijn aan roofzuchtige belastingheffing.

Een individu met hoge inkomsten, belast volgens de tarieven van
Hongkong, kan uitkomen op een vermogen dat duizendmaal groter is dan dat
van iemand met gelijk inkomen in Noord-Amerika of Europa. Je kapitaal
onderwerpen aan terugkerende overvallen door een jurisdictie met hoge
belasting is als deelnemen aan een race waarbij iemand je neerschiet
telkens als je een stap zet. Als je dezelfde race zou kunnen lopen met
de juiste bescherming en zonder handicap, zou je uiteraard veel verder
komen, en sneller.

De Soevereine Individuen van de toekomst zullen als ``voorbijgangers''
shoppen voor de meest winstgevende jurisdicties om zich te vestigen,
iets dat Christopher Lasch en andere critici van de informatie-elite zo
verafschuwen. Hoewel dit indruist tegen de logica van het nationalisme,
strookt het met een interessante economische logica. Een verschil in
nettowinst van 10 procent, laat staan een tienvoudig verschil, zal vaak
al voldoende motivatie zijn voor winstmaximaliserende mensen om hun
levensstijl en productietechnieken, maar ook hun woonplaats, te
veranderen. De geschiedenis van de westerse beschaving is een verslag
van rusteloze verandering, waarin mensen en welvaart herhaaldelijk naar
nieuwe kansrijke gebieden migreerden onder invloed van veranderende
megapolitieke omstandigheden. Een duizendvoudig verschil in
nettoresultaten zou de krachtigste prikkel evenaren die rationele mensen
ooit in beweging heeft gebracht. Anders gezegd: de meeste mensen, vooral
degenen die Thomas L. Friedman de ``verliezers en achterblijvers''
noemt, zouden, als ze de kans kregen, met plezier elke natiestaat
verlaten voor \$50 miljoen, om nog maar te zwijgen over de nog grotere
kosten die natiestaten opleggen aan de top 1 procent van de
belastingbetalers. De opkomst van Soevereine Individuen die shoppen
tussen jurisdicties is daarom een van de meest zekere voorspellingen die
men kan doen.

\section{De commercialisering van
soevereiniteit}\label{de-commercialisering-van-soevereiniteit}

In termen van kosten en baten was burgerschap tegen het einde van de
twintigste eeuw al een rampzalige deal. Dit werd geïllustreerd door een
onbedoeld grappige Parliamentary Research Note getiteld \emph{``Is the
Queen an Australian Citizen?''}, in augustus 1995 opgesteld door Ian
Ireland van de Australian Parliamentary Research Service. Ireland
bespreekt de \emph{Australian Citizenship Act} van 1948 en geeft een
overzicht van de vier manieren waarop men het Australische
staatsburgerschap kan verkrijgen. Deze komen overeen met de opties voor
burgerschap in andere vooraanstaande natiestaten, namelijk:

\begin{itemize}
\tightlist
\item
  burgerschap door geboorte
\item
  burgerschap door adoptie
\item
  burgerschap door afstamming
\item
  burgerschap door verlening
\end{itemize}

Dit is allemaal niet heel bijzonder behalve dat het de aandacht vestigt
op het onderscheid tussen soevereiniteit en burgerschap. Zoals Ireland
zegt: ``Volgens traditionele juridische en politieke opvattingen is de
monarch soeverein en zijn de mensen zijn/haar onderdanen. Onderdanen
zijn aan de monarch gebonden door trouw en onderwerping.'' Met de
vaststelling dat koningin Elizabeth II de soeverein is, concludeert hij
dat ``er beargumenteerd kan worden dat de koningin geen Australisch
staatsburger is.''

Inderdaad, dat is ze niet. De koningin, moge ze lang leven, heeft het
geluk dat burgerschap voor haar niet van belang is. Ze is soeverein, de
Soeverein over haar onderdanen. Net als een handvol andere monarchen in
de wereld is de koningin soeverein door geboorte, en heeft ze haar
status geërfd als onderdeel van een traditie die ouder is dan de moderne
tijd. Het idee van monarchie is oeroud, teruggaand tot de vroegste
geschiedkundige bewijzen van het menselijk bestaan. Landen die hun
monarchie hebben behouden, danken hun constitutie aan hun oude
geschiedenis, maar die beïnvloedt nog steeds de vorm van hun
samenleving, in termen van klassenprestige of ook politieke macht.
Postmoderne individuen, zonder de voorsprong van de koningin, zullen
nieuwe juridische grondslagen moeten uitvinden om de feitelijke
soevereiniteit te rechtvaardigen die de informatietechnologie hen zal
verschaffen.

Soevereine Individuen zullen ook moeten omgaan met de vernietigende
gevolgen van afgunst, een probleem dat soms ook monarchen treft. Dit zal
echtert veel sterker gevoeld worden door mensen die niet traditioneel
vereerd worden, maar hun eigen soevereiniteit uitvinden. Zoals Helmut
Schoeck schreef in zijn uitgebreide studie \emph{Envy}: ``Waar er
slechts één koning is, één president van de Verenigde Staten, of met
andere woorden, slechts één lid van een bepaalde status, kan hij
betrekkelijk ongestraft een levensstijl leiden die, zelfs op veel
kleinere schaal, in dezelfde samenleving verontwaardiging zou opwekken
als ze werd nagevolgd door succesvolle leden van grotere professionele
of sociale groepen.'' Monarchen, als belichaming van de natie, genieten
een zekere immuniteit tegen afgunst die niet zal overgaan op Soevereine
Individuen.

De ``verliezers en achterblijvers'' in de Informatiemaatschappij zullen
zeker de winnaars benijden en hun succes misgunnen, vooral omdat de
verdieping van markten inhoudt dat dit steeds meer een wereld van ``the
winner takes all'' wordt. Beloningen zijn nu al steeds vaker gebaseerd
op relatieve prestaties, in plaats van absolute prestaties zoals in de
industriële productie. Een fabrieksarbeider werd betaald op basis van
aanwezigheid, gemeten met de prikklok, of volgens een criterium van
output, zoals het aantal gemaakte stuks, geassembleerde eenheden of een
vergelijkbare maatstaf. Gestandaardiseerde beloning was mogelijk doordat
de output vergelijkbaar was voor iedereen die dezelfde gereedschappen
gebruikte. Het creëren van conceptuele rijkdom, zoals artistieke
prestaties, verschilt echter enorm tussen personen die dezelfde middelen
gebruiken. In dit opzicht wordt de hele economie steeds meer zoals
opera, waar de hoogste beloningen gaan naar degenen met de beste
stemmen, en waar degenen die vals zingen, hoe oprecht ook, doorgaans
geen hoge beloning krijgen. Naarmate meer sectoren worden opengesteld
voor echte mondiale concurrentie, zal de opbrengst van middelmatige
prestaties onvermijdelijk dalen. Middelmatig talent zal in overvloed
aanwezig zijn, deels afkomstig van mensen die hun tijd verhuren voor een
fractie van de tarieven die in de leidende industrielanden gelden. De
verliezers zullen de outfielders uit de lagere leagues zijn met ``slider
speed bats'', wiens reflexen een halve seconde tekortkomen om een
fastball in de major league te raken. In plaats van een miljoen dollar
per jaar te verdienen met homeruns, zullen ze \$25.000 verdienen, zonder
extra inkomsten uit celebrity-reclames. Anderen zullen zelfs volledig
uitvallen.

\begin{quote}
Zodra een land zich openstelt voor de mondiale markt, worden burgers met
de vaardigheden om hiervan te profiteren de winnaars, en degenen zonder
worden verliezers of achterblijvers. Gewoonlijk beweert één
partij\ldots{} globalisering te kunnen weerstaan of de pijn ervan te
kunnen verzachten. Dat is Pat Buchanan in Amerika, de communisten in
Rusland en nu de Islamitische Welvaartspartij hier in Turkije. Wat er in
Turkije gebeurt is dus veel gecompliceerder dan enkel een
fundamentalistische machtsovername. Het is wat er gebeurt wanneer
toenemende globalisering steeds meer verliezers produceert, wanneer
toenemende democratisering hun allemaal een stem geeft, terwijl
religieuze partijen deze samenkomst van omstandigheden succesvol
uitbuiten om de macht te grijpen. -- THOMAS L. FRIEDMAN
\end{quote}

Wie zullen de verliezers zijn in het Informatietijdperk? In algemene zin
zullen de belastingconsumenten de verliezers zijn. Meestal zijn zij
degenen die hun vermogen niet kunnen vergroten door naar een ander
rechtsgebied te verhuizen. Een groot deel van hun inkomen ligt vast in
de regels van een nationale politieke jurisdictie in plaats van bepaald
te worden door marktwaarderingen. Het sterk verlagen of schrappen van de
belastingen die hun vermogen zo benadelen, zou in hun ogen
waarschijnlijk tot weinig verbetering leiden, want minder belasting
betekent een kleinere stroom aan uitkeringen. Ze zullen inkomen
verliezen omdat ze niet langer kunnen rekenen op politieke dwang om de
zakken van productievere mensen dan henzelf leeg te roven. Degenen
zonder spaargeld die afhankelijk zijn van de overheid voor hun pensioen
en medische zorg, zullen hoogstwaarschijnlijk een daling van hun
levensstandaard meemaken. Dit inkomensverlies vertaalt zich in een
waardevermindering van wat financieel schrijver Scott Burns
``transcendent'' of politiek kapitaal heeft genoemd. Dit
``transcendente'' of denkbeeldige kapitaal is niet gebaseerd op het
economische eigendom van activa, maar in feite op de aanspraak op de
inkomensstroom die door politieke regels en voorschriften is
vastgesteld. Het verwachte inkomen uit overheidsuitkeringen zou
bijvoorbeeld kunnen worden voorgesteld als een obligatie die wordt
gekapitaliseerd tegen de geldende rentetarieven. Deze denkbeeldige
obligatie, gefinancierd door de verbeelde gemeenschap, is transcendent
kapitaal. Dit zal plotseling in waarde dalen door de ``grote
transformatie'' die onvermijdelijk de greep van politieke autoriteiten
op de geldstromen zal verminderen, maar nodig is om hun beloften na te
komen.

\begin{quote}
In grensgebieden en op volle zee, waar niemand een blijvend monopolie
had op het gebruik van geweld, ontweken kooplieden heffingen die zo hoog
waren dat bescherming elders goedkoper kon worden verkregen. -- FREDERIC
C. LANE
\end{quote}

Het vergt niet veel fantasie om in te zien dat de informatie-elite
waarschijnlijk gebruik zal maken van de kansen op bevrijding en
persoonlijke soevereiniteit die de nieuwe cybereconomie biedt. Evenzeer
valt te verwachten dat de ``achterblijvers'' steeds meer chauvinistisch
en onaangenaam zullen worden naarmate de impact van
informatietechnologie in het nieuwe millennium groeit. Het is moeilijk
om precies te voorspellen op welk moment de reactie lelijk zal worden.
Onze gok is dat de verwijten zullen toenemen wanneer westerse landen
ondubbelzinnig uit elkaar beginnen te vallen, zoals de voormalige
Sovjet-Unie.

Elke keer dat een natiestaat uit elkaar valt, zal dat verdere devolutie
vergemakkelijken en de autonomie van Soevereine Individuen aanmoedigen.
We verwachten een aanzienlijke vermenigvuldiging van soevereine
entiteiten, wanneer tientallen enclaves en rechtsgebieden, meer
vergelijkbaar met stadstaten, opkomen uit het puin van naties. Veel van
deze nieuwe entiteiten zullen zeer concurrerende tarieven voor
beschermingsdiensten aanbieden, en lage of helemaal geen belastingen
heffen op inkomen en kapitaal. Deze nieuwe entiteiten zullen hun
beschermingsdiensten vrijwel zeker aantrekkelijker prijzen dan de
leidende OESO-natiestaten. Eenvoudig bekeken als een kwestie van
marktsegmentatie, is het deel van de markt dat het slechtst bediend
wordt het segment met hoge efficiëntie en lage kosten. Wie hoge
belastingen wil betalen in ruil voor een ingewikkeld pakket aan
staatsuitgaven, kan dat ruim doen. Daarom zal de meest voordelige en
winstgevende strategie voor een nieuwe mini-souvereiniteit vrijwel zeker
liggen bij een efficiënte en goedkope aanpak. Zo'n mini-souvereiniteit
zou het alleen met grote moeite kunnen opnemen tegen het volledige
dienstenaanbod van de bestaande natiestaten. Omdat niet alle natiestaten
tegelijkertijd zullen ineenstorten, zullen statelijke opties vooral in
het begin van de transitie goed beschikbaar blijven. Een basisregime met
redelijke wet en orde kan daarentegen relatief goedkoop worden geboden.
Als sociale onrust en criminaliteit zich in de oude industriële
kernlanden verspreiden in de mate die wij verwachten, zal verdraagbare
orde en veiligheid veel aantrekkelijker zijn dan een nationaal
ruimtevaartprogramma, een door de staat gesponsord vrouwenmuseum of
gesubsidieerde herscholingsprogramma's voor ontslagen managers.

\section{De denationalisatie van het
individu}\label{de-denationalisatie-van-het-individu}

Burgerschap zal minder aantrekkelijk en houdbaar worden naarmate nieuwe
instellingen ontstaan die keuze mogelijk maken in de diensten die
overheden nu monopoliseren, te beginnen met bescherming. Dit zal het
voor individuen praktisch maken om zich niet langer nationaal te
identificeren. Toch zal de demystificatie van burgerschap een langzaam
proces zijn. Je wordt voortdurend blootgesteld aan een stortvloed van
banale boodschappen in de routines van het dagelijkse leven, bedoeld om
je identificatie met je lokale natiestaat te versterken. Deze
boodschappen maken het zeer onwaarschijnlijk dat je ``je nationaliteit''
vergeet. Voor veel mensen is nationaliteit een cruciaal symbool van
identiteit. ``Wij'' leren de wereld te zien in termen van nationaliteit.
Het is ons land, ``onze'' atleten doen mee aan de Olympische Spelen.
Wanneer zij winnen, is het ``onze'' vlag die wappert tijdens de
ceremonie. ``Ons'' volkslied trekt de aandacht van de jury en andere
deelnemers tijdens de prijsuitreiking. ``Wij'' worden geleid te geloven
dat het ``onze'' overwinning is, hoewel het nooit helemaal duidelijk is
hoe ``wij'' hebben bijgedragen, anders dan door binnen hetzelfde
grondgebied een burger te zijn.

\subsection{Van de eerste persoon meervoud naar het
enkelvoud}\label{van-de-eerste-persoon-meervoud-naar-het-enkelvoud}

Informatietechnologie zal een wereldwijd perspectief mogelijk maken en
soevereine individuen in staat stellen de verborgen kansen ervan te
benutten om de druk van nationale belastingen te vermijden. Binnen de
komende tientallen jaren zal narrowcasting bijvoorbeeld broadcasting
vervangen als methode waarmee mensen hun nieuws verkrijgen. Dit heeft
belangrijke gevolgen: het verandert de verbeelding van miljoenen van
eerste persoon meervoud naar enkelvoud, oftewel van ``wij'' naar ``ik''.
Naarmate individuen zelf hun eigen nieuwsredacteur worden en gaan kiezen
welke onderwerpen en nieuwsverhalen voor hen van belang zijn, zal het
veel minder waarschijnlijk zijn dat ze zichzelf indoctrineren met de
zogenaamde noodzaak van offers voor de natiestaat. Een soortgelijk
effect zal ontstaan door privatisering van het onderwijs, ook
gefaciliteerd door technologie. In de middeleeuwen stond onderwijs onder
strikt toezicht van de Kerk. In de moderne tijd stond onderwijs onder
controle van de staat. In de woorden van Eric Hobsbawm:
``Staatsonderwijs transformeerde mensen tot burgers van een specifiek
land: `boeren tot Fransen.'\,'' In het Informatietijdperk zal onderwijs
worden geprivatiseerd en geïndividualiseerd. Het zal niet langer belast
zijn met de zware politieke bagage die het onderwijs tijdens de
industriële periode kenmerkte. Nationalisme zal niet constant in elk
aspect van het denken worden geïnjecteerd.

De overstap naar het internet en het World Wide Web zal ook het belang
van locatie in de handel verminderen. Het zal individuele adressen
creëren die niet territoriaal gebonden zijn. Digitale telefoniediensten
op basis van satellieten zullen verder evolueren dan locatiegebonden
vaste lijnen met een gemeenschappelijke internationale netcode. Het
individu zal zijn eigen unieke wereldwijde telefoonadres hebben,
vergelijkbaar met een internetadres, dat hem overal zal kunnen bereiken.
Op termijn zullen nationale postmonopolies instorten, waardoor
geprivatiseerde wereldwijde postdiensten ontstaan zonder specifieke
banden met bestaande natiestaten.

Deze en andere ogenschijnlijk kleine stappen zullen zowel de gewone
consument als de cognitieve elite bevrijden van automatische
identificatie met de natiestaat. De demystificatie van burgerschap zal
het meest dramatisch versneld worden door het ontstaan van praktische
alternatieven die het mogelijk maken om buiten de begrensde gebieden van
de staatsmonopolies te werken. De bouwstenen van de cybereconomie:
cybergeld, cyberbankieren en een ongereguleerde wereldwijde cybermarkt
in effecten; zullen vrijwel zeker op grote schaal bestaan. Hierdoor zal
het vermogen van hebzuchtige overheden om de rijkdom van ``burgers'' te
confisqueren sterk afnemen.

Hoewel de vooraanstaande staten ongetwijfeld zullen proberen om hoge
belastingen en fiatgeld te behouden door een kartel te handhaven, om
encryptie te beperken en te voorkomen dat burgers hun domeinen
ontvluchten, zullen de staten uiteindelijk falen. De meest productieve
mensen ter wereld zullen hun weg naar economische vrijheid vinden. Het
is onwaarschijnlijk dat de staat zelfs effectief zal zijn in het fysiek
vasthouden van mensen. De ineffectiviteit van pogingen om illegale
immigranten tegen te houden toont overtuigend aan dat natiestaten hun
grenzen niet kunnen afsluiten om succesvolle mensen tegen te houden. De
rijken zullen minstens zo ondernemend zijn om te ontsnappen als dat
aspirant-taxichauffeurs en obers zijn om binnen te komen.

Voor het eerst sinds de middeleeuwse periode van gefragmenteerde
soevereiniteit zullen grenzen niet duidelijk worden afgebakend. Zoals
eerder besproken, zal er geen duidelijk territorium zijn waarin veel
toekomstige financiële transacties plaatsvinden. In plaats van de
acceptatie van een erfenis van verplichtingen op basis van een
toevalligheid van geboorte, zullen steeds meer Soevereine Individuen van
deze ambiguïteit profiteren om hun belastinglasten te ontlopen. Ze
zullen burgerschap achter zich laten en klanten worden. Ze zullen
particuliere belastingverdragen onderhandelen als klanten, vergelijkbaar
met de huidige mogelijkheden in Zwitserland, zoals geanalyseerd in
hoofdstuk 8. Een typisch privébelastingverdrag met de Franstalige
Zwitserse kantons staat een individu of gezin toe te wonen in ruil voor
een vaste jaarlijkse belasting van 50.000 Zwitserse frank (momenteel
ongeveer \$45.000). Dit is geen vlaktaks, maar een vast bedrag
onafhankelijk van inkomen. Bij een jaarlijks inkomen van 50.000
Zwitserse frank (\$45.000) is een dergelijk belastingverdrag niet
voordelig, omdat de belastingdruk 100 procent zou zijn. Bij 500.000
Zwitserse frank is het 10 procent, bij 5.000.000 frank slechts 1
procent, en bij 50 miljoen frank slechts 0,1 procent. Vergeleken met een
marginale belasting van 58 procent in New York City, laat dit zien hoe
roofzuchtig en monopolistisch overheidsdiensten tijdens de industriële
periode werden geprijsd.

In feite is 50.000 Zwitserse frank een ruimschoots toereikende
jaarlijkse betaling voor de noodzakelijke en nuttige overheidsdiensten.
Zwitserland maakt waarschijnlijk een aanzienlijke winst door elke
miljonair te bedienen die zich vestigt en jaarlijks 50.000 frank betaalt
voor dit voorrecht. In veel gevallen zijn de marginale kosten voor de
overheid om een extra miljonair in het kanton te huisvesten ongeveer
nul. De jaarlijkse winst uit deze transactie nadert daarom 50.000 frank.
Elke dienst die kan worden onderboden en toch een winst van ongeveer 100
procent oplevert, is extreem gemonopoliseerd en overprijsd. Wat
opmerkelijk is, is niet dat het belastingtarief als percentage van het
inkomen daalt in dit specifieke geval, maar dat het ooit ``eerlijk''
leek dat verschillende personen zulke uiteenlopende bedragen betaalden
voor overheidsdiensten in de twintigste eeuw. Dit is des te opvallender
omdat degenen die de meeste overheidsdiensten gebruiken het minst
betalen, en degenen die ze het minst gebruiken het meest. Voor een
Amerikaan met een hoog inkomen biedt zo'n vestigingsplaats een voordeel
ten opzichte van de VS van tientallen miljoenen gedurende zijn leven.
Tenzij de Amerikaanse belastingen worden hervormd om concurrerender te
worden met andere jurisdicties en niet langer op nationaliteit worden
geheven, zullen denkende mensen hun Amerikaanse staatsburgerschap
opzeggen om minder belaste paspoorten te verkrijgen, ondanks de
obstakels van Clintons exit-tax.

Overheden in de industriële tijd prijsden hun diensten op basis van het
succes van de belastingbetaler, niet naar de kosten of waarde van de
geleverde diensten. De overgang naar commerciële prijsstelling van
overheidsdiensten zal leiden tot betere bescherming tegen een veel
lagere prijs dan die van conventionele natiestaten.

\subsection{Burgerschap gaat de weg van de
ridderlijkheid}\label{burgerschap-gaat-de-weg-van-de-ridderlijkheid}

Kortom, burgerschap is voorbestemd om hetzelfde lot te ondergaan als
ridderschap. Naarmate de basis waarop bescherming wordt geboden opnieuw
wordt herzien, zullen ook de rationalisaties en motiverende ideologieën
die het systeem aanvullen onvermijdelijk veranderen. Een half millennium
geleden, aan het einde van de Middeleeuwen, toen het bieden van
bescherming in ruil voor persoonlijke dienst over het algemeen niet
langer rendabel was, reageerden mensen op de voorspelbare manier: ze
lieten de ridderlijkheid varen. Eedafleggingen en persoonlijke trouw
werden niet langer zo serieus genomen als in de voorgaande vijf eeuwen.
Nu belooft informatietechnologie evenzeer subversief te zijn voor
burgerschap. De natiestaat en de claims van nationalisme zullen worden
ontmythologiseerd, net zoals dat de claims van de monopolistische Kerk
vijf eeuwen geleden werden ontmythologiseerd. Hoewel reactionairen
zullen proberen om innovators zwart te maken en nationalistisch
sentiment te herstellen, betwijfelen wij of de megapolitiek achterhaalde
natiestaat voldoende loyaliteit kan afdwingen om de competitieve druk
die door informatietechnologie wordt ontketend, te weerstaan. De meeste
weldenkende individuen in een wereld van failliete overheden zullen
liever goed behandeld worden als klanten van beschermingsdiensten, dan
beroofd te worden als burgers van natiestaten.

De rijke OESO-landen leggen zware belastinglasten en regulering op aan
individuen die binnen hun grenzen zaken doen. Deze kosten waren wellicht
te verdragen toen OESO-natiestaten de enige rechtsgebieden waren waarin
men comfortabel kon wonen en ondernemen. Die tijd is voorbij. De premie
die men betaalt om belast en gereguleerd te worden als inwoner van de
rijkste natiestaten betaalt zijn kosten niet langer terug. Naarmate de
concurrentie tussen jurisdicties toeneemt, zal dit steeds minder
verdraaglijk worden. Degenen met het verdienvermogen en kapitaal om de
competitieve uitdagingen van het Informatietijdperk aan te gaan, zullen
overal hun domicilie kunnen kiezen en overal zaken kunnen doen. Met een
keuze aan woonplaatsen zullen alleen de meest patriottische of domme
mensen blijven wonen in landen met hoge belastingen.

Om deze reden valt te verwachten dat één of meer natiestaten geheime
acties zullen ondernemen om de aantrekkingskracht van voorbijgangerschap
te ondermijnen. Reizen zou ontmoedigd kunnen worden door biologische
oorlogsvoering, zoals het uitbreken van een dodelijke epidemie. Dit zou
niet alleen de wens om te reizen kunnen ontmoedigen, maar ook
jurisdicties wereldwijd een excuus geven om hun grenzen te sluiten en
immigratie te beperken.

\subsection{Het nadeel van
nationaliteitsbelasting}\label{het-nadeel-van-nationaliteitsbelasting}

Tenzij er een verbazingwekkende en vrijwel miraculeuze
beleidsverandering plaatsvindt, zal de succesvolle belegger of
ondernemer in het Informatietijdperk een levenslange belastinglast van
tientallen miljoenen, honderden miljoenen of zelfs miljarden dollars
dragen om te wonen in landen met fiscale beleidsmaatregelen zoals die
van de landen met de hoogste levensstandaarden in de twintigste eeuw.

Zonder radicale verandering zal de belastingdruk het hoogst zijn voor
Amerikanen. De Verenigde Staten zijn een van slechts drie jurisdicties
op aarde die belastingen heffen op basis van nationaliteit in plaats van
verblijfplaats. De andere twee zijn de Filipijnen, een voormalige
Amerikaanse kolonie, en Eritrea, waarvan een van de verbannen leiders
onder de invloed van de IRS viel tijdens de lange opstand tegen
Ethiopisch gezag. Eritrea heft nu een nationaliteitsbelasting van 3
procent. Hoewel dat een zwakke imitatie van de Amerikaanse tarieven is,
maakt zelfs die last het Eritrees staatsburgerschap tot een nadeel in
het Informatietijdperk. De huidige wet maakt het Amerikaanse
staatsburgerschap een nog grotere last. De IRS is een van Amerika's
belangrijkste exportproducten geworden. Meer dan welk ander land dan ook
bereikt de Verenigde Staten de uithoeken van de aarde om inkomen van
haar staatsburgers te innen.

Als een 747 met één belegger uit elke jurisdictie ter wereld zou landen
in een pas onafhankelijk geworden land, en elke belegger €1.000
riskeerde in een start-up in de nieuwe economie, zou de Amerikaan veel
hogere belastingen op eventuele winsten betalen dan iemand anders.
Speciale, bestraffende belasting op buitenlandse investeringen, zoals de
zogenaamde PFIC-belasting, plus de Amerikaanse nationaliteitsbelasting,
kan leiden tot belastingverplichtingen van 200 procent of meer op
langetermijnactiva buiten de VS. Een succesvolle Amerikaan zou zijn
totale levenslange belastingdruk kunnen verlagen door burger te worden
van elk van de meer dan 280 andere jurisdicties wereldwijd.

De Verenigde Staten hebben het meest roofzuchtige,
``soak-the-rich''-belastingsysteem ter wereld. Amerikanen die binnen of
buiten de VS wonen, worden meer als activa behandeld en minder als
klanten dan burgers van enig ander land. Het Amerikaanse
belastingstelsel is daarom meer achterhaald en minder compatibel met
succes in het Informatietijdperk dan zelfs de beruchte zwaar belastende
verzorgingsstaten van Scandinavië. Burgers van Denemarken of Zweden
ondervinden weinig wettelijke belemmeringen om hun groeiende
technologische autonomie als individu te realiseren. Willen ze hun eigen
belastingtarieven onderhandelen, dan kunnen ze ervoor kiezen in
Zwitserland belasting te betalen via een privaat verdrag, of naar
Bermuda verhuizen en helemaal geen inkomstenbelasting betalen. Een Zweed
of Deen die vrijwillig hoge belastingen betaalt omdat hij gelooft dat de
Scandinavische verzorgingsstaat zijn kosten waard is, maakt
daadwerkelijk een keuze. Hij kan belasting betalen tegen elk tarief dat
in een andere jurisdictie geldt, geciviliseerd of ongeciviliseerd. Om
zijn belastingtarief te wijzigen hoeft hij alleen maar te verhuizen.
Technologie maakt deze keuze steeds gemakkelijker. Amerikanen wordt deze
optie echter ontzegd. Het hebben van een Amerikaans paspoort zal een
groot nadeel blijken voor het benutten van de mogelijkheden voor
individuele autonomie die de Informatierevolutie biedt. Geboren worden
als Amerikaan tijdens de industriële periode was een gelukkige
toevalligheid, maar zelfs in de vroege fase van het Informatietijdperk
is het een last geworden die miljoenen dollar kost.

Om te illustreren hoe groot deze last is: een Nieuw-Zeelander die net zo
goed presteert als het gemiddelde van de rijkste 1\% Amerikanen, zou
alleen al door het rendement op de belastingbesparingen rijker worden
dan de Amerikaan ooit zal zijn. Aan het einde van zijn leven zou de
Nieuw-Zeelander \$73 miljoen meer hebben om aan kinderen of
kleinkinderen na te laten. En Nieuw-Zeeland is nog niet eens een erkend
belastingparadijs. Meer dan veertig andere jurisdicties heffen lagere
inkomsten- en vermogensbelastingen dan Nieuw-Zeeland. Als ons argument
klopt, zal het aantal rechtsgebieden met lage belastingen eerder
toenemen dan afnemen. Al deze jurisdicties bieden een voordeel als
woonplaats ten opzichte van de VS, goed voor tientallen miljoenen, zo
niet honderden miljoenen, gedurende een leven. Tenzij de Amerikaanse
belastingen worden hervormd om concurrerender te zijn met andere
jurisdicties en niet langer op nationaliteit worden geheven, zullen
weldenkende mensen hun Amerikaanse staatsburgerschap opzeggen, ondanks
de obstakels van Clintons exit-tax.

De competitieve omstandigheden van het Informatietijdperk maken het
mogelijk om vrijwel overal hoge inkomens te verdienen. In feite zullen
de lokale monopolies, die natiestaten gebruiken om extreem hoge
belastingen op te leggen, door technologie worden doorbroken. Ze zijn al
aan het afbrokkelen. Naarmate ze verder afnemen, zal competitieve druk
ondernemende individuen ertoe drijven om landen die te veel belasting
heffen te ontvluchten. Zoals voormalig \emph{Economist}-redacteur Norman
Macrae zei, zullen zulke landen ``voornamelijk door dommeriken worden
bewoond.''

\begin{quote}
{[}T{]}egen het jaar 2012 zullen de geraamde uitgaven voor sociale
voorzieningen en rente op de nationale schuld alle belastinginkomsten
van de federale overheid opslokken. \ldots{} Er zal geen cent
overblijven voor onderwijs, kinderprogramma's, snelwegen, nationale
defensie of enig ander discretionair programma. -- BIPARTISAN U.S.
COMMISSION ON ENTITLEMENT AND TAX REFORM
\end{quote}

De vlucht van de rijken uit geavanceerde verzorgingsstaten zal,
demografisch gezien, precies op het verkeerde moment plaatsvinden. In
het begin van de 21e eeuw zullen grote vergrijzende populaties in Europa
en Noord-Amerika onvoldoende spaargeld hebben om medische kosten te
dekken en hun levensstijl tijdens hun pensioen te financieren. 65
procent van de Amerikanen heeft bijvoorbeeld helemaal geen spaargeld
voor hun pensioen. En degenen die sparen, sparen veel te weinig. Van de
gemiddelde Amerikaan wordt verwacht dat ze van 65-jarige leeftijd tot
overlijden meer dan \$200.000 aan medische rekeningen zullen hebben,
terwijl ze gemiddeld slechts een nettovermogen van minder dan \$75.000
hebben. Zelfs de minderheid met een privé-pensioen zal waarschijnlijk
niet comfortabel zijn. Het gemiddelde pensioen vervangt slechts 20
procent van het inkomen tijdens het werkende leven. Het grootste deel
van de activa van de typische gepensioneerde is geen werkelijk vermogen,
maar ``transcendent kapitaal'', de verwachte waarde van
overheidsuitkeringen. De meeste mensen zijn gewend om op deze betalingen
te rekenen voor het aanvullen van privétekorten. Het probleem is dat
deze waarschijnlijk niet volledig zullen worden uitgekeerd.
Pay-as-you-go-systemen zullen niet over voldoende kasstroom of middelen
beschikken om ze na te komen. Een studie van Neil Howe toonde aan dat,
zelfs als de belastbare inkomens in de VS sneller zouden stijgen dan in
de afgelopen twintig jaar, het gemiddelde inkomen na belasting in
Amerika tegen 2040 met 59 procent omlaag zou moeten worden gedrukt om
Social Security en overheidszorgprogramma's op het huidige niveau te
financieren.

Dit probleem kan niet marginaal worden opgelost. De verzorgingsstaat
staat op instorten. De financieringsproblemen zijn in Europa zelfs nog
ernstiger dan in Noord-Amerika. Italië is wellicht het slechtste geval,
gevolgd door Zweden en andere Noordse verzorgingsstaten die de standaard
voor genereuze inkomensondersteuning bepaalden. De \emph{Financial
Times} schat dat, als ``de huidige waarde van Italiaanse
staats­pensioenen wordt meegerekend, de staatsschuld van het land zou
stijgen tot meer dan 200 procent van het BBP.''

Schulden op dat niveau zijn realistisch gezien wiskundig hopeloos. Een
uitgebreide studie van commerciële schulden van bedrijven op de Toronto
Stock Exchange enkele jaren geleden toonde dat weinig bedrijven
overleven bij schuldverhoudingen een kwart zo extreem als die van de
vooraanstaande verzorgingsstaten vandaag. Simpel gezegd: ze zijn
failliet. Naarmate deze realiteit, zij het met tegenzin, wordt
onderkend, zullen letterlijk biljoenen aan niet-gefinancierde
uitkeringsverplichtingen worden afgeschreven.

Dit is de logica van de cybereconomie. Een mogelijke belemmering is
simpele inertie: de nestdrang die mensen terughoudend maakt om te
verhuizen. Andere belemmeringen kunnen ingebakken zijn in de menselijke
natuur. De economische logica van het inzetten van activa in cyberspace
kan botsen met biologisch gedrag, zoals de ingebakken achterdocht tegen
buitenstaanders. Kinderen in alle culturen tonen afkeer van vreemden.
Tegenstanders van de commercialisering van soevereiniteit zullen hun
best doen om twijfels aan te wakkeren over de nieuwe wereldcultuur van
het Informatietijdperk en het verval van de natiestaat dat dit
impliceert. Een andere mogelijke belemmering, voortkomend uit epigenese
of genetisch beïnvloede motivatie, is dat de ``verliezers en
achterblijvers'' zullen reageren op ontwikkelingen die de natiestaat
ondermijnen, met de woede van jagers-verzamelaars die hun families
beschermen. In een omgeving waarin gedesoriënteerde en vervreemde
individuen meer macht hebben om te verstoren en vernietigen, kan een
tegenreactie op de informatie-economie gewelddadig en onaangenaam zijn.

\begin{quote}
Historisch gezien is collectief geweld regelmatig voortgekomen uit de
centrale politieke processen van westerse landen. Mensen die de macht
wilden grijpen, behouden of herverdelen, hebben voortdurend collectief
geweld gebruikt in hun strijd. De onderdrukten sloegen toe in naam van
gerechtigheid, de bevoorrechten in naam van orde, degenen ertussen in
naam van angst. Grote verschuivingen in machtsverhoudingen produceerden
doorgaans, en waren vaak afhankelijk van, uitzonderlijke periodes van
collectief geweld. -- CHARLES TILLY
\end{quote}

\section{Geweld in perspectief}\label{geweld-in-perspectief}

Er zijn minstens twee concurrerende theorieën over wat geweld
veroorzaakt in omstandigheden van verandering. Historicus Charles Tilly
vat één theorie samen: ``{[}D{]}e prikkel tot collectief geweld komt
grotendeels voort uit de angsten die mensen ervaren wanneer gevestigde
instellingen instorten. Als ellende of gevaar de angst versterkt, wordt
de reactie volgens de theorie des te gewelddadiger.'' Volgens Tilly is
geweld echter niet zozeer een product van angst, maar eerder een veel
rationelere poging om autoriteiten onder druk te zetten hun
verantwoordelijkheden na te komen, gemotiveerd door een ``gevoel van
ontzegde rechtvaardigheid.'' Volgens Tilly's interpretatie stimuleren
``grote structurele veranderingen'' collectief geweld van een
``politieke'' aard. ``In plaats van een scherp breukpunt met het
`normale' politieke leven te vormen, gaan gewelddadige strubbelingen
bovendien vaak gepaard met, vullen ze aan en breiden ze georganiseerde,
vreedzame pogingen van dezelfde mensen uit om hun doelen te bereiken. Ze
behoren tot dezelfde wereld als niet-gewelddadige strijd.''

Welke theorie over geweld ook juist is, de vooruitzichten op sociale
vrede tijdens de Grote Transformatie lijken beperkt. De instorting van
de natiestaat is zeker een opvallend voorbeeld van een ``gevestigde
instelling die instort.'' Daarom zullen angsten hoogtij vieren, evenals
de politieke inspiratie voor geweld. Dit kan vooral gelden in de
vooraanstaande verzorgingsstaten, waar bevolkingen gewend zijn aan
relatieve inkomensgelijkheid. Aangezien bevolkingen in de vroege fasen
van de informatie-economie volwassen zijn geworden tijdens de
industriële periode, toen politieke autoriteiten wel de capaciteit
hadden om klachten te beantwoorden met materiële voordelen, is het
redelijk om te verwachten dat de ``achterblijvers'' blijven aandringen
op materiële voordelen. Waarschijnlijk is een langzaam, pijnlijk
leerproces over de cybereconomie nodig voordat OESO-burgers hun
verwachting loslaten dat ze inkomensherverdeling op grote schaal kunnen
afdwingen. In beide gevallen, of geweld nu voortkomt uit ``angst'' of
als een meer berekende poging om de voordelen van systematische dwang te
benutten, lijkt het erop dat de omstandigheden waarschijnlijk zullen
leiden tot de inzet van geweld.

\subsection{Achterban van verliezers}\label{achterban-van-verliezers}

De ineenstorting van de gedwongen inkomensherverdeling zal
onvermijdelijk degenen verontrusten die verwachten aan de ontvangende
kant te staan van de biljoenen aan uitkeringsprogramma's. Dit zullen
vooral de ``verliezers of achterblijvers'' zijn, mensen zonder de
vaardigheden om te concurreren op wereldmarkten. Net zoals de
gepensioneerden van de voormalige Sovjet-Unie de kern vormden van
Zuganovs communistische steun, zullen de teleurgestelde gepensioneerden
van de stervende verzorgingsstaten een reactionair kiespubliek vormen
dat er alles aan zal doen om te voorkomen dat de soevereiniteit van de
natiestaten wordt geprivatiseerd, waardoor de staat haar vergunning om
te stelen zou verliezen. Zodra ze beseffen dat overheden die voorheen
onder hun controle stonden hun soevereiniteit over middelen en het
vermogen tot grootschalige inkomensoverdrachten verliezen, zullen ze
even vastberaden worden als Franse ambtenaren in hun strijd tegen de
rekenkunde.

Je herinnert je misschien de gewelddadige reactie op de vrij bescheiden
voorstellen van premier Alain Juppé om de ``demografisch onhoudbare''
pensioenuitkeringen van staatswerknemers terug te schroeven en te
besparen op de exploitatie van het genationaliseerde spoorwegsysteem.
Symbolisch voor de absurditeit van de \emph{État Providence}, zoals de
Fransen hun sociale zekerheidssysteem noemen, is de regel die
``ingenieurs van de geautomatiseerde hogesnelheidstreinen TGV toestaat
om op hun vijftigste met pensioen te gaan, net als hun voorgangers die
zwoegden op de kolengestookte locomotief.'' Het is een reële
mogelijkheid dat er rumoerige reacties op het schrappen van onhoudbare
uitkeringen zullen ontstaan in elk OESO-land. En zelfs waar bevolkingen
minder heftig reageren, kun je verwachten dat de waarschijnlijke
verliezers alles zullen doen wat in hun macht ligt om de erosie van
staatsdwang tegen te houden.

Dit zal tot verrassende wendingen leiden. In de Verenigde Staten
bijvoorbeeld is het nativistische sentiment historisch gezien doordrenkt
met meer dan een vleugje racisme. Deze traditie begon met de
negentiende-eeuwse ``White Caps'' en de Ku Klux Klan. Toch zijn zwarten,
als groep, grote begunstigden van inkomensoverdrachten, positieve
discriminatie en andere vruchten van politieke dwang. Ze zijn ook
onevenredig vertegenwoordigd in het Amerikaanse leger. Daarom zullen
zij, samen met blanke arbeiders, waarschijnlijk uitgroeien tot de meest
fervente aanhangers van het Amerikaanse nationalisme.

Politici die bereid zijn in te spelen op de onzekerheden van degenen
wier relatieve talenten ver onderaan Ammons knol staan, zullen in bijna
elk land luidruchtig op de voorgrond treden. Van Slobodan Milosevic in
Servië tot Pat Buchanan in de Verenigde Staten, Winston Peters in
Nieuw-Zeeland en Necmettin Erbakan van Turkije's fundamentalistische
Islamitische Welvaartspartij, demagogen zullen oproepen tegen de
globalisering van markten, immigratie en investeringsvrijheid.

Degenen die zichzelf zien als de ``slachtoffers van de wereldeconomie'',
zullen bijzonder vijandig zijn tegen rijken en immigranten. In de
woorden van Andrew Heal zullen zij ``de instroom verachten van
immigranten wiens belangrijkste toelatingscriterium hun rijkdom of juist
het gebrek daaraan lijkt te zijn, wat volgens de bedrieglijke logica
maakt dat ze een last voor de verzorgingsstaat zijn.''

\subsection{Angst voor vrijheid}\label{angst-voor-vrijheid}

Het vooruitzicht dat de natiestaat aan het begin van het nieuwe
millennium zal verdwijnen, lijkt getimed om maximale ontwrichting te
veroorzaken in de levens van beïnvloedbare mensen. Dit zal leiden tot
wijdverspreid ongemak. Heel wat waarnemers hebben een reactiepatroon
herkend dat vaak voorkomt bij degenen die zich buitengesloten voelen
door het vooruitzicht van een wereld zonder grenzen. Terwijl de grotere,
meer inclusieve nationale verbanden uiteenvallen doordat de mobielere
``informatie-elite'' haar zaken globaliseert, vallen de ``verliezers en
achterblijvers'' terug op lidmaatschap van een etnische subgroep, stam,
bende of religieuze/taalkundige minderheid. Dit is deels een praktische
en pragmatische reactie op het instorten van overheidsdiensten, zoals
ordehandhaving. Voor mensen zonder verhandelbare middelen is het vaak
moeilijk om marktgebaseerde alternatieven voor falende publieke diensten
te kopen.

De transformatie van voormalige publieke goederen, zoals onderwijs,
schoon drinkwater, en buurtpolitie, tot private goederen is makkelijker
te doorstaan voor wie voldoende middelen heeft om kwalitatieve
alternatieven te betalen. Voor wie geld nodig heeft, is het meest
praktische alternatief vaak om te leunen op familie, of zich aan te
sluiten bij een onderlinge hulpgroep die langs etnische lijnen is
georganiseerd, zoals de vroegere Chinese ``Hokkien'' in Zuidoost-Azië,
of via een religieuze gemeenschap. In regio's met dynamische,
missionaire religies is de populariteit van hun programma's deels te
danken aan het feit dat zij teruggrijpen op premoderne mechanismen van
sociale zekerheid. Zo speelden door moslims geleide burgerwachten in
Kaapstad, Zuid-Afrika, een hoofdrol in de strijd tegen gewelddadige
bendes. Naast dit pragmatisme speelt er echter ook een psychologische
dimensie in de reactie op globalisering.

Dit argument verschilt niet veel van de psychologische verklaring voor
de aantrekkingskracht van het fascisme die Erich Fromm ontwikkelde in
zijn beroemde werk \emph{Fear of Freedom}, voor het eerst gepubliceerd
in 1941. Volgens Fromm had de sociale mobiliteit, die door het
kapitalisme werd geïntroduceerd, de vaste identiteiten van het
traditionele dorpsleven vernietigd. Een boerenzoon hoefde er niet meer
van uit te gaan dat hij zelf ook boer zou worden en oogstend zijn leven
zou moeten slijten op hetzelfde arme stuk land als zijn vader. Hij had
nu een brede keuze aan beroepen. Hij kon onderwijzer, koopman of soldaat
worden; geneeskunde studeren of de zee opgaan. Zelfs als boer kon hij
emigreren naar de Verenigde Staten, Canada of Argentinië en een leven
opbouwen ver van het land van zijn voorouders. Het kapitalisme bood
mensen de vrijheid ``om hun eigen identiteiten te creëren'', maar dit
bleek beangstigend voor hen die niet waren voorbereid om er creatief
gebruik van te maken. Zoals Billig zei, verlangden zij ``naar de
zekerheid van een vaste identiteit,'' en werden zij ``aangetrokken door
de simplismen van nationalistische en fascistische propaganda.''
Eveneens schrijft Billig over de schemering van het industriële
tijdperk: ``Er bestaat een globale psychologie die van bovenaf de natie
treft en bestaande loyaliteiten uitholt via een vrij spel van
identiteiten. Daartegenover staat de felle psychologie van kaste of
stam, die de zwakke plek van de staat aanvalt met een intolerante
toewijding en emotionele heftigheid.''

Andrew Heal bekijkt hetzelfde verschijnsel vanuit een ander perspectief.
Hij ziet twee grote ``globale politieke en economische trends. . . .
Trend één is de groei van de wereldeconomie. . . De tweede is de opkomst
van nationalistisch, etnisch en regionalistisch sentiment, of het nu
Maori, Schots, Welsh is of van anti-immigratiefracties, die, zelfs
terwijl hun overheden hen naar nieuwe, grenzeloze horizonten duwen,
zichzelf des te sterker de andere kant op trekken.'' Of je ze nu ziet
als grote ``trends'' of als ``psychologische thema's'', het is duidelijk
dat wereldwijd een krachtige reactionaire beweging opkomt die
nationalisme steunt en zich verzet tegen het vervagen van grenzen en de
groei van markten.

\section{multiculturalisme en
slachtofferschap}\label{multiculturalisme-en-slachtofferschap}

In zijn nadagen, met een wankel vermogen om beloftes van iets voor niets
uit een lege portemonnee waar te maken, vond de verzorgingsstaat het
nuttig om nieuwe discriminatiemythes te creëren. Veel categorieën van
officieel ``onderdrukte'' mensen werden aangewezen, vooral in
Noord-Amerika. Individuen in groepen met de status van ``slachtoffer''
werd verteld dat zij niet verantwoordelijk waren voor tekortkomingen in
hun eigen leven. De schuld zou liggen bij ``dode blanke mannen'' van
Europese afkomst, en bij de onderdrukkende machtsstructuur die zogenaamd
tegen de achtergestelde groepen was ingesteld. Was je zwart, vrouwelijk,
homoseksueel, Latino, Franstalig, gehandicapt, enz., dan maakte je
aanspraak op compensatie voor eerdere repressie en discriminatie.

Als we Lasch volgen, werd het benadrukken van slachtofferschap gebruikt
om staten te ondermijnen en de nieuwe, onafhankelijke informatie-elite
te bevrijden van hun burgerlijke plichten. We zijn niet helemaal
overtuigd dat de nieuwe elite, vooral de meesten in de massamedia, sluw
genoeg is om tot zo'n houding te redeneren. Het zou bijna geruststellend
zijn om te voelen dat ze dat zijn. We zien de toename van
slachtofferschap voornamelijk als een poging om sociale vrede te kopen,
zowel door meer mensen in de meritocratie op te nemen, zoals Lasch zegt,
maar ook door de rechtvaardigingen voor inkomensherverdeling opnieuw
vorm te geven. De nieuwe sport victimologie verscheen in zijn meest
overdreven vorm in Noord-Amerika omdat informatietechnologie daar dieper
doordrong. We vermoeden echter dat nieuwe discriminatiemythes, tot op
zekere hoogte, in de seniele toestand van alle industriële samenlevingen
gemeengoed zullen zijn. De multi-etnische verzorgingsstaten in
Noord-Amerika waren eenvoudigweg kwetsbaarder voor de verleiding om de
kosten van inkomensherverdeling op de private sector af te schuiven. Ze
konden dit doen, terwijl ze een gevoel van ongenoegen en aanspraak
aanwakkerden, door de structuur van de samenleving als geheel, en blanke
mannen in het bijzonder, de schuld te geven van de economische
tekortkomingen van diverse subculturen binnen de samenleving.

\subsection{De megapolitiek van
innovatie}\label{de-megapolitiek-van-innovatie-1}

Zelfs voordat informatietechnologie dreigde met de ``creatieve
vernietiging'' van de industriële economie, verouderde het al duidelijk
veel van de gekoesterde mythes van marxisten en socialisten. In een
eerder hoofdstuk onderzochten we de megapolitiek van innovatie. Wat we
daar benadrukten helpt om de maatschappelijke impact van de
Informatierevolutie in perspectief te plaatsen. Het feit dat technologie
in de afgelopen eeuwen steeds een uitbreiding van de
arbeidsmogelijkheden teweegbracht, lijkt wel een betrouwbare regel
binnen de economie, maar er is geen garantie dat dat zo zal blijven. Het
is namelijk mogelijk dat de opbrengsten zich zullen concentreren in de
handen van een welvarende minderheid.

\section{Reële lonen dalen met 50
procent}\label{reuxeble-lonen-dalen-met-50-procent}

Dat is precies wat gebeurde tijdens de eerste twee eeuwen of meer van de
moderne periode. Vanaf de tijd van de Buskruitrevolutie rond 1500 tot
1700 daalden de reële inkomens voor de onderste 60-80 procent van de
bevolking in het grootste deel van West-Europa met 50 procent of meer.
Op veel plaatsen bleef het reële inkomen dalen tot 1750 en herstelde het
zich pas tot het niveau van 1500 rond 1850.

In tegenstelling tot de ervaring van de afgelopen 250 jaar, was de
inkomensgroei in de eerste helft van de moderne periode, toen de
West-Europese economieën enorm groeiden, geconcentreerd bij een kleine
minderheid. De huidige innovatie van informatietechnologieën verschilt
sterk van de innovatie van industriële technologieën die de wereld in
recente eeuwen heeft ervaren. Het verschil zit hem in het feit dat het
merendeel van de huidige arbeidsbesparende innovaties vaak leidt tot
meer gespecialiseerde taken en minder schaalvoordelen. Dit is het
tegenovergestelde van wat sinds ongeveer 1750 het geval was.

Industriële innovatie creëerde doorgaans werkgelegenheid voor de
onopgeleiden en vergrootte de schaalvoordelen van ondernemingen. Dit
verhoogde niet alleen de inkomsten van de armen zonder inspanning van
hun kant, het vergrootte ook de macht van politieke systemen, waardoor
ze beter bestand werden tegen onrust. Degenen die werden verdrongen door
mechanisering en automatisering in de vroege fasen van de Industriële
Revolutie waren doorgaans geschoolde ambachtslieden en gezellen, in
plaats van ongeschoolde arbeiders. Dit was zeker het geval in de
textielindustrie, de eerste die mechanisatie en krachtige apparatuur op
grote schaal toepaste, wat leidde tot een gewelddadige reactie van
Luddieten, die textielmachines vernietigden en fabriekseigenaars
vermoordden tijdens een woeste uitbarsting in het begin van de
negentiende eeuw. Aan de andere kant waren de volgelingen van Captain
Swing, de mythische leider van een opstand in 1830 in Zuidoost-Engeland,
dagloners. Ze eisten onder ander het opleggen van een heffing aan de
lokale rijken die hen geld of bier zou moeten verschaffen, het opleggen
van een loonsverhoging aan de lokale werkgevers van dagloners, en ``het
vernietigen, of eisen van de vernietiging van; nieuwe landbouwmachines,
vooral dorsmachines'' die de vraag van boeren naar dagloners
verminderden.

In tegenstelling tot de romantische verhalen van marxisten en anderen
die de gewelddadige tegenstanders van arbeidsbesparende technologieën
tot helden hebben gemaakt, waren zij een onaangenaam en gewelddadig
gezelschap dat de invoering van technologie die wereldwijd de
levensstandaard verhoogde puur uit egoïstische motieven tegenstond.

Terwijl de gewelddadige volgelingen van Ned Ludd en Captain Swing vele
maanden de openbare orde in Engeland in gevaar brachten, waren hun
bewegingen, eenmaal onderdrukt door de centrale autoriteit, gedoemd te
mislukken. De arme, ongeschoolde meerderheid zou zich waarschijnlijk
niet lang blijven inzetten voor een zaak die beloofde machines te
vernietigen die hen werk boden en ook hun levensstandaard verhoogden
door de kosten van noodzakelijke goederen, zoals warme kleding en brood,
te verlagen.

\subsection{Hogere inkomsten voor de
ongeschoolden}\label{hogere-inkomsten-voor-de-ongeschoolden}

Met de tijd werd industriële en agrarische automatisering aantrekkelijk
voor de ``have-nots'', omdat het hen arbeidsmogelijkheden gaf en het de
kosten voor levensonderhoud verlaagde. Nieuwe gereedschappen stelden
degenen zonder vaardigheden in staat om goederen van gelijke kwaliteit
te produceren als heel bekwame personen. Een genie en een idioot zouden
op de assemblagelijn hetzelfde product produceren en hetzelfde loon
verdienen.

Gedurende de afgelopen twee eeuwen heeft industriële automatisering de
lonen voor ongeschoold werk enorm verhoogd, vooral in het kleine deel
van de wereld waar de omstandigheden eerst het kapitalisme lieten
floreren. De grootschaligheid van geavanceerde industriële ondernemingen
beloonde ongeschoolde arbeid niet alleen met ongekende lonen, het
faciliteerde ook inkomensherverdeling.

De verzorgingsstaat ontstond als een logische consequentie van de
technologie van het industrialisme. Vanwege hun grote schaal en hoge
kapitaalkosten waren de vooraanstaande industriële werkgevers de
gemakkelijkste doelwitten om te belasten, en men kon erop vertrouwen dat
ze administratie bijhielden en de inhouding van lonen handhaafden.
Hierdoor werd inkomstenbelasting technologisch haalbaar, zoals dat in
eerdere eeuwen niet het geval was toen economieën meer gedecentraliseerd
waren. Het netto-effect was dat overheden door de groei van
schaalvoordelen dankzij industriële innovatie rijker werden en
vermoedelijk beter in staat om de orde te handhaven.

\subsection{Het proces is omgekeerd}\label{het-proces-is-omgekeerd}

Volgens ons gebeurt het tegenovergestelde vandaag.
Informatie-technologie vergroot de verdienmogelijkheden voor de
geschoolden en ondermijnt instituties die op grote schaal opereren,
inclusief de natiestaat.

Dit wijst op een andere ironie van het Informatietijdperk: de
schizofrene en fundamenteel obstructieve houding van critici van de
vrije markt tegenover het ontstaan en verdwijnen van industriële banen.
In de vroege fasen van het industrialisme klaagden zij over het
vermeende kwaad van fabrieksbanen, die landloze boeren weglokten uit
``de wereld die we hebben verloren.'' Volgens de critici was de komst
van fabrieksbanen een ongekend kwaad en ``uitbuiting'' van de
arbeidersklasse. Maar nu lijkt het enige dat erger is dan de komst van
fabrieksbanen, hun verdwijning. De achterkleinkinderen van degenen die
klaagden over de introductie van fabrieksbanen klagen nu over het tekort
aan fabrieksbanen die goed betaalde laaggeschoolde arbeid bieden.

De enige consistente draad door deze klachten is een standvastige
weerstand tegen technologische innovatie en verandering van de markt. In
de vroege fasen van het fabriekssysteem leidde deze weerstand tot
geweld. Dat kan opnieuw gebeuren, maar niet omdat kapitalisten de
arbeiders ``uitbuiten.'' De opkomst van de computer als
paradigmatechnologie toonde de absurditeit van die bewering. Voor de
onoplettende leek het half geloofwaardig dat een nauwelijks geletterde
autofabrikant op enige wijze bij de productie van een auto ``uitgebuit''
werd door de eigenaren die de bedrijven bedachten en financierden.
Conceptueel kapitaal speelde vroeger een minder duidelijke rol bij de
productie en marketing van tastbare producten dan nu in het
Informatietijdperk, waarin duidelijk mentale arbeid wordt toegepast. De
veronderstelling dat ondernemers de door werknemers gemaakte
informatieproducten hadden ingepikt, werd hierdoor veel minder
geloofwaardig. Waar de waarde duidelijk door mentale arbeid werd
gecreëerd, zoals bij consumentensoftware, was het bijna belachelijk te
veronderstellen dat het product van iemand anders dan de geschoolde
personen die het bedachten was. Integendeel, in plaats van te
veronderstellen dat de werknemers alle waarde creëerden, zoals marxisten
en socialisten deden in de negentiende en twintigste eeuw, gaf de
duidelijke trend weg van laaggeschoolde arbeid aanleiding tot een
groeiende zorg over het tegenovergestelde probleem: hadden
laaggeschoolde arbeiders nog economische bijdrage te leveren?

Daarom verschoof de rechtvaardiging voor inkomensherverdeling van
``uitbuiting,'' wat een productieve competentie bij laagverdieners
veronderstelde, naar ``discriminatie,'' wat dat niet deed. Er werd
verondersteld dat ``discriminatie'' ervoor had gezorgd dat
laaggeschoolden er niet in slaagden om waardevollere vaardigheden te
ontwikkelen.

Deze discriminatie zou ook het opleggen van niet-optimale
wervingscriteria en andere standaarden voor het creeëren van ``kansen,''
of preciezer, het herverdelen van inkomen naar achterblijvende groepen,
rechtvaardigen. In de Verenigde Staten bijvoorbeeld, liet op ras
gebaseerde normering van prestatie- en vaardigheidstests zwarten hogere
scores behalen dan blanke en Aziatische kandidaten, terwijl de
objectieve scores lager waren. Via deze methode en andere dwong de
overheid werkgevers om meer zwarten en andere officieel ``victimized''
groepen aan te nemen tegen hogere lonen dan anders het geval zou zijn
geweest. Wie niet voldeed, riskeerde kostbare rechtszaken, inclusief
claims met hoge schadevergoedingen.

Het doel van het aanduiden van slachtoffers was niet het kweken van
paranoïde vervolgingswanen onder belangrijke subgroepen van de
industriële samenleving, of het subsidiëren van contraproductieve
waarden. Het was om de failliete staat te ontlasten van de fiscale druk
van inkomensherverdeling. Het aanwakkeren van vervolgingswanen was
slechts een ongelukkig neveneffect. Ironisch genoeg viel de toegenomen
bezorgdheid over ``discriminatie'' samen met de vroege fasen van een
technologische revolutie die willekeurige discriminatie veel minder
problematisch zou maken dan ooit tevoren. Niemand op het internet weet
of geeft om het ras, geslacht, seksuele geaardheid of andere kenmerken
van de maker van nieuwe software.

Hoewel de werkelijkheid van discriminatie in de toekomst minder
onderdrukkend zal zijn, zal dat de druk voor ``herstelbetalingen'' om
verschillende werkelijke of ingebeelde onrechtvaardigheden te
compenseren niet per se verminderen. Elke samenleving creëert een of
meer rationalisaties voor inkomensherverdeling, van subtiel tot absurd,
van bijbelse geboden tot zwarte magie. Hekserij en het boze oog zijn de
spirituele equivalenten van belastingheffing. Wanneer mensen niet door
liefde gemotiveerd kunnen worden om de armen te steunen, zullen de armen
proberen angst in te zetten. Soms gebeurt dit via directe afpersing,
soms op een verborgen of denkbeeldige manier. Het is geen toeval dat de
meeste ``heksen'' in de vroegmoderne periode weduwen of ongehuwde
vrouwen met weinig middelen waren. Ze terroriseerden hun buren met
vervloekingen, waardoor die buren vaak betaalden. Het was niet altijd
puur bijgeloof. Ook arme vrouwen konden vee schaden of huizen in brand
steken. In die zin waren de heksenprocessen niet geheel belachelijk.

Wij verwachten een heropleving van afpersing vanuit het verlangen om mee
te genieten van de beloningen van succes in het Informatietijdperk.
Groepen die zich benadeeld voelen door eerdere discriminatie zullen hun
status als slachtoffers niet snel opgeven, alleen omdat hun claims
minder gerechtvaardigd of moeilijker afdwingbaar worden. Ze zullen hun
eisen blijven doorzetten totdat het lokale bewijs duidelijk maakt dat ze
niet langer beloond worden.

Het toenemende sociopathische gedrag binnen de Afro-Amerikaanse en
Afro-Canadese gemeenschappen laat zien dat er nauwelijks evenwicht is
tussen woede en de erkenning dat veel problemen voortkomen uit eigen
antisociaal gedrag. De woede is toegenomen, terwijl de leefstijlen juist
verder achteruitgingen. Geboorten buiten het huwelijk zijn gestegen,
educatieve resultaten dalen, en een groeiend percentage jonge zwarten is
betrokken bij criminaliteit, waardoor er nu meer zwarte mannen in
gevangenissen zitten dan op universiteiten.

Mogelijk zorgden deze perverse uitkomsten er tijdelijk voor dat in de
laatste fase van het industrialisme meer middelen naar
onderklassegemeenschappen gingen, doordat de dreiging van afpersing op
de samenleving als geheel toenam. Door de competitie uit te bannen, die
onderpresteerders uitdaagde zich aan te passen aan de norm, heeft de
verzorgingsstaat legioenen disfunctionele, paranoïde en slecht
gesocialiseerde mensen gecreëerd, het sociale equivalent van een
kruitvat. De dood van de natiestaat en het verdwijnen van grootschalige
inkomensherverdeling zal sommige van deze ongelukkigen ertoe aanzetten
zich te keren tegen mensen die welvarender zijn dan zij. Het is redelijk
om te veronderstellen dat sociale vrede in gevaar zal zijn in het
Informatietijdperk, vooral in Noord-Amerika en multi-etnische enclaves
in West-Europa.

\begin{quote}
We zullen de wapens nooit neerleggen {[}totdat{]} het Lagerhuis een wet
aanneemt om alle machines die schadelijk zijn voor het gemeen af te
schaffen, en die wet herroept die voorziet in het ophangen van ``frame
breakers''. Maar wij, wij verzoeken niet langer, dat helpt niet. We
moeten vechten. -- Ondertekend door de Generaal van het Leger der
Herstellers, Ned Ludd, Klerk ``Hersteller-voor-altijd, Amen''
\end{quote}

\subsection{Neo-Luddiet}\label{neo-luddiet}

Gezien de antitechnologische opstanden in het begin van de negentiende
eeuw en de lange traditie van collectief geweld in zowel Europa als
Noord‑Amerika, zou het geen verrassing zijn als we een neo-ludditische
aanval op de informatietechnologie en haar gebruikers zouden zien. De
Luddieten, zoals eerder genoemd, waren textielarbeiders die zich in West
Yorkshire, Engeland, hadden verzameld en in 1811--1812 een
terroristische campagne voerden tegen geautomatiseerde snoeimachines en
tegen de fabriekseigenaren die deze in gebruik namen. Met zwartgemaakte
gezichten raasden de Luddieten door West Yorkshire, staken fabrieken in
brand en vermoordden fabriekseigenaren die de nieuwe technologie durfden
te omarmen. Het grootste deel van het geweld kwam van de ``croppers'',
vaklieden die met enorme scharen van tot wel vijftig pond een essentiële
rol speelden in de productie van wollen textiel. De afwerking die
croppers deden, ``het opruwen met distels en bijknippen met scharen,''
was, zoals Robert Reid in zijn standaardwerk \emph{Land of Lost Content:
The Luddite Revolt 1812} echter schreef, ``te eenvoudig om niet te
mechaniseren.'' Leonardo da Vinci had al een ontwerp voor zo'n
mechanische snoeimachine uitgetekend, maar zijn idee voor automatische
snoei bleef eeuwen onbenut. Uiteindelijk slaagden ingenieurs in 1787
erin om een apparaat heruit te vinden naar het model van Leonardo's
ontwerp, en brachten zij het in Engeland in productie. Reid merkt
daarbij op: ``Alle bouwstenen van de technologie waren al geruime tijd
bekend, waardoor het opmerkelijk is dat men deze niet eerder had
doorgevoerd\ldots{} De nieuwe machines uit de Industriële Revolutie
vereisten zo weinig kracht en vaardigheid dat veel vacatures werden
ingevuld door vrouwen en jonge kinderen, aanvankelijk tegen lage lonen.
Één van deze apparaten kon, zelfs wanneer zij bediend werd door relatief
ongeschoolden, in achttien uur verrichten wat een ervaren cropper met
handscharen in achtentachtig uur voor elkaar kreeg.''

Opmerkelijk is dat werknemers die zich fel tegen mechanisatie verzetten,
zich alleen tegen technologieën keerden die hun eigen banen overbodig
maakten of de vraag naar vakbekwame arbeid deden afnemen. Toen
ondernemer William Cooke tapijtweefmachines invoerde in West Yorkshire,
brak er geenszins geweld uit. Men trachtte zijn molen niet in brand te
steken, zijn machines niet te vernielen en zeker niet hem, laat staan
zijn leven, te ruïneren. Volgens Reid, die de Luddieten-opstanden
uitvoerig beschrijft, riep Cooke's nieuwe technologie geen verzet op,
omdat ``niemand in het dal tot dan toe gespecialiseerd was in
tapijtproductie''. Hij vervolgt: ``Doordat Cooke een nieuw product
introduceerde en banen creëerde die niet op traditionele methoden waren
gebaseerd, kende zijn molen een ware bloeiperiode.'' Dit voorbeeld heeft
grote implicaties voor de toekomst. Het duidt erop dat vooruitziende
ondernemers in het komende millennium ingrijpende, arbeidsefficiënte
automatisering zullen introduceren in regio's waar men nog geen traditie
had in de productie van het betreffende product of de dienst.

Als we de geschiedenis kunnen gebruiken als gids, dan zullen de meest
radicale terroristen in de eerste jaren van het nieuwe millennium
waarschijnlijk niet uit dakloze verarmden bestaan, maar uit werknemers
die ooit tot de middenklasse behoorden en aanzien genoten, maar nu hun
baan kwijt zijn geraakt. Dat bleek immers tijdens de Luddieten-opstand
van 1812, waar het merendeel van de betrokken Luddieten geen verarmd
proletariaat vormde, maar hoogopgeleide ambachtslieden waren die gewend
waren om een inkomen te verdienen dat ten minste vijfmaal hoger lag dan
dat van een gemiddelde arbeider. De hedendaagse variant zal vermoedelijk
bestaan uit verdrongen fabrieksarbeiders. Helaas toont een blik op de
demografie van de meeste OESO-landen dat er zich op meerdere plaatsen
gewelddadige reacties kunnen afspelen.

Wereldwijde natiestaten zullen pogingen doen om de cybereconomie te
dwarsbomen, en daarmee ook de soevereine individuen die er voordeel uit
halen en vermogen opbouwen. Deze pogingen zullen een woedende
nationalistische reactie oproepen, waarin onvermijdelijk ook een
antitechnologisch sentiment zal meespelen, vergelijkbaar met de
Luddieten- en andere antitechnologische opstanden in Groot-Brittannië
tijdens de Industriële Revolutie. Dit verdient serieuze aandacht, want
het kan bepalend zijn voor hoe bestuur zich in het nieuwe millennium zal
ontwikkelen. Een cruciale opgave in de grote transformatie die voor ons
ligt, is het bewaren van orde ondanks toenemend geweld, of het vermijden
ervan. Personen en ondernemingen die nauw verbonden zijn met het
ontstaan van het Informatietijdperk, waaronder de spelers in Silicon
Valley en zelfs de leveranciers van de elektriciteit die de nieuwe
technologie aandrijft, moeten extra waakzaam zijn voor freelance,
neo-ludditisch terrorisme.

Een krankzinnige zoals de Unabomber zal helaas waarschijnlijk legioenen
aan imitators aanwakkeren, nu de frustratie over dalende inkomens en de
toegenomen wrok tegen prestaties de kop opsteken. Wij vermoeden dat een
groot deel van het toekomstige geweld uit bomaanslagen zal bestaan.
Volgens de \emph{New York Times} is het binnenlands terrorisme in de
Verenigde Staten in de jaren negentig fors toegenomen. ``Het is de
afgelopen vijf jaar met meer dan vijftig procent toegenomen en in het
afgelopen decennium bijna verdrievoudigd. Het aantal criminele explosies
en pogingen steeg van 1.103 in 1985 tot 3.163 in 1994\ldots. In kleine
steden, voorsteden en binnenstedelijke straatbendes komt steeds vaker
een veelvoorkomende alledaagse bommenlegger voor.''

\subsection{Defensie wordt een privaat
goed}\label{defensie-wordt-een-privaat-goed}

Ondanks de strafbelastingen die door natiestaten worden opgelegd als
prijs voor bescherming, is het onwaarschijnlijk dat zij die in de
toekomst effectief zullen bieden. De dalende schaal van geweld, zoals
geïmpliceerd door de nieuwe informatietechnologie, maakt het onderhoud
van een massaal militair apparaat veel minder nuttig. Het betekent niet
alleen dat oorlogen minder doorslaggevend zullen worden, waardoor staten
hun burgers minder effectief kunnen beschermen, maar het impliceert ook
dat de schijnbare extraterritoriale hegemonie van de Verenigde Staten
als 's werelds supermacht in de volgende eeuw minder succesvol zal zijn
dan de hegemonie van Groot-Brittannië in de negentiende eeuw. Tot het
uitbreken van de Eerste Wereldoorlog kon macht effectief en beslissend
worden geprojecteerd van de kern naar de periferie tegen relatief lage
kosten. In de eenentwintigste eeuw zullen de dreigingen die grootmachten
vormen voor de veiligheid van leven en eigendom noodzakelijkerwijs
afnemen samen met de opbrengsten van geweld. Omdat geweld steeds minder
oplevert, is het onwaarschijnlijk dat grootschalige militaire machten in
de vorm van natiestaten of rijken zullen voortbestaan of ontstaan in het
Informatietijdperk.

Naarmate de fiscale vereisten voor de levering van een adequate defensie
afnemen, zal het steeds geloofwaardiger worden om beschermingsdiensten
te behandelen alsof het private goederen zijn. Immers, bedreigingen voor
de veiligheid zullen op kleinere schaal in toenemende mate kunnen worden
afgeweerd door beveiligingsmaatregelen die commercieel kunnen worden
ingeschakeld, zoals door muren, hekken en beveiligingsperimeters in te
zetten om onruststokers buiten te houden. Bovendien kan een vermogend
individu of bedrijf zich wellicht veroorloven bescherming in te huren
tegen de meeste bedreigingen die waarschijnlijk in het
Informatietijdperk zullen ontstaan. Aan de marge zal de verminderde
schaal van militaire dreigingen het gevaar van anarchie vergroten, of
van concurrerend geweld binnen één enkel grondgebied. Maar het zal ook
de concurrentie tussen jurisdicties om bescherming op concurrerende
voorwaarden intensiveren. ``Shoppen'' tussen jurisdicties voor
beschermingsdiensten, paspoort- en consulaire diensten, en juridische
hulp zal dus steeds meer voorkomen.

Op de lange termijn zullen Soevereine Individuen waarschijnlijk in staat
zijn te reizen met niet-gouvernementele documenten, uitgegeven zoals
kredietbrieven door private agentschappen en belangengroepen. De
veronderstelling dat er een groep zal ontstaan als een soort
koopmansrepubliek van de cyberspace, is niet ver gezocht. Ze zou
georganiseerd kunnen worden zoals de middeleeuwse Hanze, om de
onderhandeling van private verdragen en contracten tussen jurisdicties
te vergemakkelijken en bescherming te bieden aan haar leden. Stel je een
speciaal paspoort voor, uitgegeven door de Liga van Soevereine
Individuen, waarin de houder wordt geïdentificeerd als een persoon onder
de bescherming van de liga.

Een dergelijk document, mocht het ontstaan, zal slechts een tijdelijk
artefact zijn van de overgang weg van de natiestaat en het
bureaucratische tijdperk dat zij voortbracht. Vóór de moderne periode
waren paspoorten doorgaans onnodig om grenzen te passeren, die in de
meeste gevallen vaag waren gedefinieerd. Hoewel veiligegeleidebrieven
soms werden gebruikt in middeleeuwse grenssamenlevingen, werden ze
normaal gesproken uitgegeven door de autoriteiten van het gebied dat
bezocht werd, en niet door de jurisdictie waar de reiziger vandaan kwam.
Belangrijker dan een paspoort waren aanbevelings- en kredietbrieven, die
een reiziger in staat stelden onderdak te vinden en zaken te doen. Die
dag zal terugkeren. Uiteindelijk zullen invloedrijke personen helemaal
zonder documenten kunnen reizen. Ze zullen zich kunnen identificeren op
een onfeilbare biometrische manier via spraakherkenning of retinascans
die hen uniek herkennen.

Kortom, we verwachten dat ergens in de eerste helft van de volgende eeuw
de wereld de daadwerkelijke privatisering van soevereiniteit zal
meemaken. Dit zal samengaan met omstandigheden waarvan verwacht kan
worden dat ze het domein van dwang zullen doen verschrompelen tot het
logische minimum. Toch zal het vermarkten van de eens ``heilige''
attributen van nationaliteit, het kopen en verkopen als een kwestie van
kosten-batenberekening, voor de seculiere inquisiteurs en reactionairen
van het volgende millennium zowel woestmakend als bedreigend worden
ervaren.

We betogen in dit boek dat er geen natiestaat meer nodig zal zijn om een
Informatie-oorlog te voeren. Dergelijke oorlogen zouden kunnen worden
ondernomen door de inzet van grote aantallen ``bots'' of digitale
assistenten door computerprogrammeurs. Bill Gates bezit nu al een
grotere capaciteit om logische bommen te laten ontploffen in kwetsbare
systemen wereldwijd dan de meeste natiestaten ter wereld. In het
tijdperk van de Informatie-oorlog zou elk softwarebedrijf, of zelfs de
Scientologykerk, een formidabelere tegenstander zijn dan de gezamenlijke
dreiging die wordt gevormd door de meerderheid van de staten met een
zetel in de Verenigde Naties.

Dit machtsverlies van natiestaten is een logische consequentie van de
komst van goedkope, geavanceerde rekenkracht. Door microprocessing nemen
de voordelen van geweld af en ontstaat er voor het eerst een markt waar
concurrerende aanbieders bescherming kunnen leveren, iets waarvoor
staten vroeger monopolistische prijzen rekenden.

In de nieuwe wereld van gecommercialiseerde soevereiniteit zullen mensen
hun jurisdicties kiezen, net zoals dat velen nu hun
verzekeringsmaatschappijen of hun religies kiezen. Jurisdicties die er
niet in slagen een geschikt dienstenpakket te bieden, wat dat ook mag
zijn, zullen failliet gaan en geliquideerd worden, net zoals onbekwame
commerciële ondernemingen of mislukte religieuze congregaties.
Concurrentie zal daarom de inspanningen van lokale jurisdicties
mobiliseren om hun diensten zuiniger en efficiënter te leveren. In dit
opzicht zal concurrentie tussen jurisdicties in het leveren van publieke
goederen een soortgelijke impact hebben als die in andere sectoren.
Concurrentie verbetert meestal de klanttevredenheid.

\section{Concurrentie en anarchie}\label{concurrentie-en-anarchie}

Het is belangrijk om te beseffen dat de concurrentie die wij tussen
rechtsgebieden voorzien, niet hoofdzakelijk concurrentie is tussen
organisaties die geweld gebruiken in hetzelfde territorium. Zoals eerder
aangegeven, hebben concurrerende organisaties die geweld gebruiken de
neiging de doordrenking van geweld in het leven te vergroten, waardoor
economische kansen afnemen. Zoals Lane het stelde:

\begin{quote}
``Het toepassen van geweld bood duidelijk voordelen op grotere schaal,
zowel in de strijd tegen andere gewelddadige organisaties als bij het
opzetten van een territoriaal monopolie. Dit is een fundamenteel feit
voor de economische analyse van één aspect van de overheid: de
geweld-gebruikende, geweld-beheersende industrie was een natuurlijk
monopolie, althans op land. Binnen territoriale grenzen kon de dienst
die zij leverde veel goedkoper worden geproduceerd door een monopolie.
Zeker, er zijn tijden geweest waarin ondernemingen die geweld
gebruikten, in hetzelfde gebied concurreerden om beschermingsgeld, zoals
bijvoorbeeld tijdens de Dertigjarige Oorlog in Duitsland. Zo'n situatie
was economisch gezien echter nog ongunstiger dan concurrentie in
hetzelfde territorium tussen rivaliserende telefoonsystemen.''
\end{quote}

Het commentaar van Lane is op twee punten informatief. Ten eerste
stemmen we in met zijn algemene conclusie dat soevereiniteiten de
neiging zullen hebben om territoriale monopolieposities uit te oefenen,
omdat dit hen in staat stelt goedkopere en effectievere
beschermingsdiensten te leveren. Het tweede interessante aspect van
Lane's opmerking is zijn verouderde vergelijking met een monopolie op
telefoniediensten. Tegenwoordig weten we duidelijk dat telefoonsystemen
geen monopolie hoeven te zijn. Hierdoor moeten we voorzichtig zijn wat
betreft deze analyse. Veranderingen in technologische omstandigheden
kunnen tot op zekere hoogte de algemene conclusie dat anarchie binnen
territoriale grenzen niet levensvatbaar is, ondermijnen. Bijvoorbeeld,
als cyberassets op grote schaal groeien in een domein buiten het bereik
van dwang, kan de prijsstelling van beschermingsdiensten veel minder een
kwestie van ``vraag'' zijn en meer een kwestie van onderhandeling in de
markt.

Desalniettemin verwijzen wij hier naar iets anders dan gegeneraliseerde
anarchie, namelijk concurrentie tussen jurisdicties, elk met een
monopolie op geweld binnen hun eigen territorium. Wij voorzien dat zulke
jurisdicties zullen concurreren om de grootst mogelijke waarde te bieden
bij de kosteneffectieve levering van beschermingsdiensten die
aantrekkelijk zijn voor hun ``klanten.'' Ongetwijfeld zullen er grotere
ambiguïteiten zijn in de levering van beschermingsdiensten in het
Informatietijdperk, waarin er meer private levering van politiediensten
en defensie zal zijn dan we gewend zijn. Toch verschilt de concurrentie
die wij voorzien van een grootschalig gevecht tussen meerdere
beschermingsinstanties voor dienstverlening aan verschillende klanten in
hetzelfde territorium, wat anarchie is.

Hoe dan ook, de vermenigvuldiging van het aantal soevereiniteiten,
waarbij individuen steeds meer de rol van soeverein op zich nemen zodra
ze voldoende middelen hebben opgebouwd, impliceert onvermijdelijk dat de
kans op anarchie in de wereld zal toenemen. De betrekkingen tussen
soevereiniteiten zijn altijd anarchistisch. Er is nooit een
wereldregering geweest die het gedrag van individuele soevereiniteiten
reguleerde, of het nu ministaten, natiestaten of imperia betreft. Zoals
Jack Hirshleifer schrijft: ``{[}W{]}anneer associaties variërend van
primitieve stammen tot moderne natiestaten intern worden bestuurd door
enige vorm van wet, blijven hun externe betrekkingen met elkaar
grotendeels anarchistisch.'' Wanneer er meer soevereine entiteiten in de
wereld zijn, zullen er onvermijdelijk meer betrekkingen plaatsvinden in
meer dan één jurisdictie en zijn deze daarom anarchistisch.

Het is belangrijk om op te merken dat anarchie, of het ontbreken van een
overweldigende macht om geschillen te beslechten, niet gelijkstaat aan
totale chaos of de afwezigheid van vorm of organisatie. Hirshleifer
merkt op dat anarchie geanalyseerd kan worden: ``intertribale of
internationale systemen hebben ook hun regelmatigheden en systematisch
analyseerbare patronen.'' Met andere woorden, net zoals ``chaos'' in de
wiskunde een ingewikkelde en hoog georganiseerde vorm van organisatie
kan inhouden, is ``anarchie'' niet volledig vormloos of wanordelijk.

Hirshleifer analyseert een aantal anarchistische situaties. Deze
omvatten, naast betrekkingen tussen soevereiniteiten, bendeoorlogen in
het Chicago van de drooglegging en ``mijnwerkers versus claim-jumpers
tijdens de Californische goudkoorts.'' Merk op dat hoewel Californië
tegen het begin van de goudkoorts in 1849 deel uitmaakte van de
Verenigde Staten, de omstandigheden in de goudvelden terecht als
anarchie konden worden omschreven. Zoals Hirshleifer opmerkt: ``{[}D{]}e
officiële rechtsorganen waren machteloos.'' Hij betoogt dat de
combinatie van de topografische omstandigheden en de effectieve
burgerwacht, die door mijnwerkers werd georganizeerd om claimjumpers
tegen te gaan, het voor bendes moeilijk maakte om goudmijnen te
bemachtigen, ondanks het ontbreken van wettelijk toezicht. Met andere
woorden, onder bepaalde omstandigheden kan waardevol eigendom succesvol
worden beschermd, zelfs onder anarchie.

De vraag is of Hirshleifer's theoretische analyse van de dynamiek van de
spontane orde van de Darwiniaanse ``natuurlijke economie'' relevant is
voor de economie van het Informatietijdperk. Wij vermoeden van wel.
Hoewel wij geen gegeneraliseerde anarchie of goudmijnomstandigheden
verwachten, anticiperen wij wel een toename van het aantal
anarchistische betrekkingen in het wereldwijde systeem. In het licht van
deze verwachting is Hirshleifer's betoog over de omstandigheden
waaronder ``twee of meer anarchistische deelnemers levensvatbare delen
van de maatschappelijk beschikbare middelen in evenwicht kunnen
houden'', veelzeggend. In het bijzonder onderzoekt hij wanneer anarchie
geneigd is te ``vervallen'' tot tirannie of dominantiehiërarchieën, wat
gebeurt wanneer de anarchistische partijen kunnen worden onderworpen
door een overweldigende autoriteit.

Deze kwesties kunnen in het Informatietijdperk belangrijker zijn om te
begrijpen dan in het Industriële Tijdperk. Een deel van de reden dat de
fijnere nuances over de dynamiek van anarchie in recente eeuwen minder
cruciaal waren dan ze in het nieuwe millennium kunnen zijn, is precies
omdat de opbrengsten van geweld tijdens de moderne periode toenamen. Dit
betekende dat het samenbrengen van steeds grotere militaire krachten,
zoals natiestaten in recente eeuwen deden, vaak leidde tot
doorslaggevende oorlogsvoering. Doorslaggevende oorlogsvoering
onderwerpt anarchie bijna per definitie doordat het degenen die
concurreren om middelen onderwerpt aan een sterkere macht. Aan de andere
kant draagt afnemende doorslaggevendheid in gevechten, die samenhangt
met het voordeel van defensieve militaire technologie, bij aan de
dynamische stabiliteit van anarchie. Daarom zou de schijnbare impact van
informatietechnologie op het verminderen van de doorslaggevendheid van
militaire actie de anarchie tussen mini-soevereiniteiten stabieler
moeten maken en ze minder vatbaar moeten maken voor verovering door een
grote overheid. Minder doorslaggevendheid in de strijd impliceert ook
minder gevechten, wat een bemoedigende conclusie is voor de wereld in
het Informatietijdperk.

\subsection{Levensvatbaarheid}\label{levensvatbaarheid}

Een andere belangrijke voorwaarde voor het in stand houden van anarchie
is levensvatbaarheid of de toereikendheid van het inkomen. Personen die
niet over voldoende inkomen beschikken om te overleven, zullen
waarschijnlijk ofwel (1) veel energie steken in gevechten om genoeg
middelen te bemachtigen om te overleven, of (2) zich onderwerpen aan een
andere deelnemer in ruil voor voedsel en levensonderhoud. Iets
soortgelijks gebeurde bij de opkomst van het feodalisme tijdens de
transformatie rond het jaar 1000. Wij verwachten dat een toenemend
aantal personen met lage inkomens in westerse landen, die voorheen
afhankelijk waren van overheidsuitkeringen, zich zullen verbinden aan
rijke huishoudens als dienaren. Desalniettemin is het feit dat sommige
deelnemers in een Hobbesiaanse strijd (een oorlog van allen tegen allen)
niet levensvatbaar zijn, op zich nog geen bewijs dat anarchie niet kan
voortbestaan. Zoals Hirshleifer zegt: ``{[}H{]}et loutere feit van een
laag inkomen onder anarchie, \ldots{} geeft op zichzelf geen duidelijk
inzicht in wat er waarschijnlijk hierna zal gebeuren.''

\subsection{De aard van activa}\label{de-aard-van-activa}

Nog een andere interessante voorwaarde voor het in stand houden van
anarchie is dat middelen ``voorspelbaar en verdedigbaar'' moeten zijn.
In Hirshleifers analyse is ``{[}A{]}narchie een sociale ordening waarin
deelnemers strijden om duurzame middelen te veroveren en te
verdedigen.'' Hij definieert ``duurzame middelen'' als ``grondgebieden
of verplaatsbare kapitaalgoederen.'' In het Informatietijdperk kunnen
digitale middelen voorspelbaar blijken, maar zij zullen geen `duurzame
middelen' zijn van het type dat Hirshleifer met territorialiteit en
anarchie associeert. Immers, als digitaal geld overal op de planeet met
de snelheid van het licht kan worden overgedragen, kan de verovering van
het grondgebied waarin een cyberbank is gevestigd tijdverspilling zijn.
Natiestaten die Soevereine Individuen willen onderdrukken, zouden
tegelijkertijd zowel de wereldwijde banken- als datahavens moeten
veroveren. Zelfs dan zouden zij, als de versleutelde systemen correct
zijn ontworpen, slechts bepaalde hoeveelheden digitaal geld kunnen
saboteren of vernietigen, niet in beslag nemen.

De conclusie is dat de meest voorspelbare en kwetsbare bezittingen van
rijken in het komende Informatietijdperk hun fysieke persoon kunnen
zijn, met andere woorden, hun leven. Daarom vrezen wij ludditisch
terrorisme in de komende decennia, deels misschien stilzwijgend
aangemoedigd door provocateurs in dienst van natiestaten.

Op lange termijn betwijfelen wij echter dat de vooraanstaande
natiestaten zullen slagen in het onderdrukken van Soevereine Individuen.
Bestaande staten zullen, vooral in kapitaalarme regio's, merken dat ze
meer te winnen hebben bij het herbergen van Soevereine Individuen dan
bij het handhaven van solidariteit met de Noord-Atlantische natiestaten
en het respecteren van het ``internationale'' systeem. Het feit dat
failliete, hoogbelaste verzorgingsstaten ``hun burgers'' en ``hun
kapitaal'' in ``hun land'' willen houden, zal geen dwingende motivatie
zijn voor de honderden fragmenterende soevereiniteiten elders.

Dit zeggen wij, ondanks het feit dat er duizenden multinationale
organisaties zijn die het gedrag van de verschillende soevereiniteiten
ter wereld willen beïnvloeden. Sommige van deze organisaties, zoals de
Europese Unie en de Wereldbank, zijn zeker invloedrijk, maar bedenk dat
de rechtsgebieden die Soevereine Individuen welkom heten aanzienlijk
profiteren van hun aanwezigheid. Zelfs een eigenzinnige macht als de
Verenigde Staten, die volgens de huidige trends fel zal proberen een
cybereconomie buiten haar controle te voorkomen, zal uiteindelijk niet
de bewoners van de wereld met positieve banktegoeden willen uitsluiten
die geen Amerikaan willen zijn. Dit geldt des te meer omdat winkelen nu
een belangrijke fascinatie is van reizigers. Uiteindelijk, hoewel later
dan anderen, zullen de Verenigde Staten, of fragmenten daarvan,
deelnemen aan de commercialisering van soevereiniteit vanwege
competitieve druk.

\subsection{Vraag creëert aanbod}\label{vraag-creuxebert-aanbod}

Die druk zal in het begin sterker worden gevoeld in staten met de
zwakste financiële positie. Onder de nieuwe ``offshore''-centra zullen
fragmenten en enclaves van huidige staten zijn, zoals Canada en Italië,
die vrijwel zeker zullen uiteenvallen ruim voor het einde van het eerste
kwart van de eenentwintigste eeuw. De geboorte van een wereldmarkt voor
hoogwaardige, kostenefficiënte jurisdicties zal helpen dergelijke
jurisdicties tot stand te brengen. Zoals in de gewone handel zullen
kleinschalige concurrenten wendbaarder zijn en beter kunnen concurreren.
Een dunbevolkte jurisdictie kan zich gemakkelijker efficiënt
structureren.

De informatie-elite zal hoogwaardige bescherming op contractbasis tegen
een redelijke vergoeding zoeken. Hoewel deze vergoeding ruim
tekortschiet om een merkbaar voordeel te herverdelen onder de hele
bevolking van huidige natiestaten, met tientallen tot honderden
miljoenen burgers, zou het niet onbeduidend zijn in een jurisdictie met
tienduizenden of honderdduizenden inwoners. De belastingbetalingen en
andere economische voordelen, voortvloeiend uit de aanwezigheid van een
klein aantal buitengewoon rijke individuen, impliceren een veel hoger
per capita voordeel voor een rechtsgebied met een kleine in plaats van
een grote bevolking.

Aangezien het nagenoeg irrelevant zal zijn waar men zijn bedrijven
domicilieert, behalve in de puur negatieve zin dat sommige adressen
hogere verplichtingen impliceren dan andere, zullen kleine jurisdicties
het gemakkelijker vinden om commercieel succesvolle voorwaarden voor
bescherming vast te stellen. Daarom zullen rechtsgebieden met kleine
populaties een duidelijk voordeel genieten bij het formuleren van een
fiscaal beleid dat aantrekkelijk is voor Soevereine Individuen.

Wij geloven dat het tijdperk van de natiestaat voorbij is, maar dit
betekent niet dat de aantrekkingskracht van nationalisme als emotionele
trek onmiddellijk zal verdwijnen. Als ideologie speelt nationalisme erg
goed in op de universele emotionele behoeften. We hebben allemaal het
gevoel van ontzag ervaren, zoals bij het zien van een gigantische
waterval of het betreden van een grote kathedraal. We hebben allemaal
het gevoel van erbij horen ervaren, zoals bij een kerstfeest met familie
of als lid van een succesvol sportteam. De menselijke cultuur vraagt een
reactie op beide krachtige emoties. Wij worden verlicht door de
historische cultuur van ons eigen land, dat op zijn beurt deel uitmaakt
van de grotere cultuur van de mensheid. We worden getroost door het
besef dat wij tot een culturele groep behoren, wat ons zowel een gevoel
van participatie als van identiteit geeft.

De impact van deze culturele symbolen kan een heel sterk emotioneel
effect hebben. De Amerikaanse associaties van de vlag, het volkslied of
het familiefeest op Thanksgiving Day, de Engelse associaties van de
monarchie of cricket, allemaal hebben ze een echte greep op de
verbeelding van respectievelijk Amerikanen en Engelsen, een greep die
wordt versterkt door herhaling en is diep doordrongen in het
onderbewustzijn. Dergelijke symbolen helpen ons te vertellen wat voor
soort mensen we zijn en herinneren ons aan een nationale cultuur. Toen
demonstranten tegen de Vietnamoorlog de rest van de Verenigde Staten
wilden shockeren, verbrandden zij de vlag. Vervreemde Engelsen vallen de
monarchie aan en hebben zelfs gaten gegraven in cricketvelden.

Deze typische elementen zijn oppervlakkig, maar niet onbelangrijk. Het
zijn de associaties waarvoor wij zijn opgevoed om te bloeden. Ongeacht
de verandering in megapolitieke omstandigheden of de resulterende
verandering in instituties, zullen zij waarschijnlijk belangrijk blijven
in de verbeelding van mensen die, net zoals wij, volwassen werden in de
twintigste eeuw.

\bookmarksetup{startatroot}

\chapter{De schemering van de
democratie}\label{de-schemering-van-de-democratie}

\begin{quote}
Democratische politieke systemen zijn in historische termen een recent
verschijnsel. Ze hadden een kortstondig bestaan in Griekenland en Rome,
en kwamen daarna pas in de 18e eeuw opnieuw op, minder dan 200 jaar
geleden. \ldots{} Een cyclus van verwerping kan nu opnieuw zijn
begonnen. -- WILLIAM PFAFF
\end{quote}

Het is geen geheim dat democratie relatief zeldzaam en vluchtig is
geweest in de geschiedenis van overheden. In die perioden, oud en
modern, waarin democratie de overhand had, berustte haar succes op
megapolitieke omstandigheden die de militaire macht en het belang van de
massa versterkten. Historicus Carroll Quigley onderzocht deze kenmerken
in \emph{Weapons Systems and Political Stability}.

Deze omvatten:

\begin{enumerate}
\def\labelenumi{\arabic{enumi}.}
\tightlist
\item
  \textbf{Goedkope en wijdverspreide bewapening.} Democratie bloeit
  meestal wanneer de kosten voor het aanschaffen van bruikbare wapens
  laag zijn.
\item
  \textbf{Wapens die effectief gebruikt kunnen worden door amateurs.}
  Democratie is waarschijnlijker wanneer iedereen wapens doeltreffend
  kan hanteren zonder langdurige training.
\item
  \textbf{Een militair voordeel voor een groot aantal deelnemers te voet
  in de strijd.} Zoals Quigley stelt: ``Perioden van
  infanteriedominantie zijn perioden geweest waarin de politieke macht
  breder verspreid was binnen de gemeenschap en democratie een betere
  kans had om te gedijen.''
\end{enumerate}

Deze opsomming van de voorwaarden waaronder democratie kan bestaan, is
nauwelijks volledig. Als dat zo was, zou democratie niet het winnende
systeem zijn geworden aan het einde van de twintigste eeuw. Wapens waren
tegen de schemering van het Industriële Tijdperk wellicht duurder dan
ooit. En veel van de meest doeltreffende wapens vereisten specialisten
om ze goed te gebruiken. Bovendien bewees de Golfoorlog tussen de
Verenigde Staten, hun bondgenoten en Irak hoe kwetsbaar grote
infanteriecontingenten zijn, zelfs wanneer zij in loopgraven en
versterkte stellingen schuilen. Waarom leek democratie dan toch te
bloeien onder deze omstandigheden aan het einde van de twintigste eeuw?

\section{Democratie, de broederlijke tweeling van het
communisme?}\label{democratie-de-broederlijke-tweeling-van-het-communisme}

We stelden in Hoofdstuk 5 een paradoxale verklaring voor, namelijk dat
democratie opbloeide als een tweelingbroer van het communisme, juist
omdat het de overheid in staat stelde ongehinderd controle uit te
oefenen over middelen. Deze conclusie lijkt misschien absurd uitgaande
van het ``gezonde verstand'' van het Industriële Tijdperk. We ontkennen
niet dat binnen de terminologie van de industriële samenleving
democratische systemen en communisme elkaars tegenpolen waren.
Waarschijnlijk zal men in het Informatietijdperk er echter meer vanuit
een megapolitiek perspectief naar kijken, waarin blijkt dat de twee
systemen meer gemeen hebben dan je zou vermoeden.

In een context waarin wapens absurd duur waren, werd democratie het
besluitvormingsmechanisme dat de staatscontrole over middelen
maximaliseerde. Net als het staatssocialisme stelde democratie enorme
sommen geld beschikbaar om een massale militaire macht te financieren.
Het verschil was dat de democratische verzorgingsstaat zelfs grotere
middelen in handen van de staat bracht dan de staatssocialistische
systemen konden. Dat zegt wat, aangezien de staatssocialistische of
communistische systemen praktisch alle waardevolle bezittingen opeisten.

Onpartijdig beschouwd als louter een mechanisme om middelen te
verzamelen, was de democratische staat als recept voor staatsverrijking
superieur aan het staatssocialisme. Zoals eerder uitgelegd, maakte
democratie aanzienlijk meer geld beschikbaar voor het leger omdat ze
verenigbaar was met particuliere eigendom en kapitalistische
productiviteit.

Het staatssocialistische systeem was gebaseerd op de doctrine dat de
staat alles bezat. De democratische verzorgingssstaat daarentegen had
aanvankelijk meer bescheiden claims. Zij deed alsof ze particulier
eigendom toestond, zij het voorwaardelijk, en gebruikte daardoor
superieure prikkels om productie te mobiliseren. In plaats van alles van
meet af aan te mismanagen, lieten democratische regeringen in het Westen
hun burgers eigendom en vermogen opbouwen. Pas nadat de rijkdom was
gecreëerd, traden de democratische natiestaten op om er een groot deel
van te belasten.

Het woord ``groot'' moet met hoofdletters geschreven worden. In 1996
bedroeg het levenslange federale belastingtarief in de Verenigde Staten
bijvoorbeeld drieënzeventig cent per dollar. Voor eigenaars van
vennootschappen, die hun inkomen via dividenden ontvingen, was het
tarief drieëntachtig cent per dollar. En voor wie geld aan kleinkinderen
wilde nalaten of schenken, bedroeg het federale belastingtarief
drieënnegentig cent per dollar. Wanneer ook staats- en lokale
belastingen worden meegerekend, eigent de democratische overheid op alle
niveaus zich het grootste deel van elke dollar toe die in de Verenigde
Staten wordt verdiend.

Roofzuchtige belastingtarieven maakten van de democratische staat een
feitelijke partner met een aandeel van drie kwart tot negen tienden in
alle verdiensten. Dit was weliswaar niet hetzelfde als staatssocialisme,
maar wel een nauwe verwant.

De democratische staat overleefde langer omdat ze flexibeler was en
aanzienlijk grotere hoeveelheden middelen verzamelde dan beschikbaar
waren in Moskou of Oost-Berlijn.

\subsection{``Inefficiëntie, waar het
telde''}\label{inefficiuxebntie-waar-het-telde}

We hebben de megapolitieke voordelen van democratie als
besluitvormingsregel voor een machtige overheid omschreven als
``inefficiëntie, waar het telde.'' In vergelijking met het communisme
was de verzorgingsstaat inderdaad een veel efficiënter systeem, maar
vergeleken met een echte laissez-faire-enclave als Hongkong was de
verzorgingsstaat inefficiënt. De groeicijfers in Hongkong waren
verbluffend, maar hun superioriteit lag precies in het feit dat de
inwoner van Hongkong, en niet de overheid, 85 procent van de baten van
snellere groei in eigen zak kon steken.

Hongkong is uiteraard geen democratie. Sterker nog, het is een mentaal
model van het soort jurisdictie waarvan wij verwachten dat het zal
floreren in het Informatietijdperk. In het Industriële Tijdperk had
Hongkong geen noodzaak om een democratie te zijn, aangezien het gespaard
bleef van de onaangename noodzaak om middelen te vergaren ter
ondersteuning van een imposante militaire macht. Hongkong werd van
buitenaf verdedigd en kon zich daarom een werkelijk vrije economie
permitteren.

Het was juist het vermogen om middelen binnen te harken die democratie
superieur maakte onder de megapolitieke omstandigheden van het
Industriële Tijdperk. Massademocratie ging hand in hand met
industrialisme. Zoals Alvin Toffler zei, is massademocratie ``de
politieke uitdrukking van massaproductie, massadistributie,
massaconsumptie, massaal onderwijs, massamedia, massavermaak en al het
andere.''

Nu informatietechnologie de massaproductie verdringt, is het logisch om
het einde van de massademocratie te verwachten. De cruciale
megapolitieke noodzaak die massademocratie deed triomferen in het
Industriële Tijdperk is verdwenen. Het is daarom slechts een kwestie van
tijd voordat massademocratie hetzelfde lot ondergaat als haar
tweelingbroer, het communisme.

\subsection{Massademocratie onverenigbaar met het
Informatietijdperk}\label{massademocratie-onverenigbaar-met-het-informatietijdperk}

Een moment van reflectie maakt duidelijk dat de technologie van het
Informatietijdperk niet inherent een massatechnologie is. In militaire
termen, zoals we hebben aangegeven, maakt ze de weg vrij voor ``slimme
wapens'' en ``informatie-oorlog'', waarin ``logische bommen''
gecentraliseerde commandostructuren en controlesystemen kunnen
saboteren. Informatietechnologie leidt niet alleen tot de
perfectionering van wapens die door specialisten moeten worden bediend,
maar maakt oorlogvoering ook minder doorslaggevend, waardoor een
verdedigende positie relatief gezien voordeliger wordt. Microtechnologie
maakt dramatische toenames mogelijk in de militaire macht van
individuen, terwijl het belang van massale infanterieformaties afneemt.
Zoals de Rand Corporation aan de minister van Defensie rapporteerde:
``Onderling verbonden netwerken kunnen worden aangevallen en verstoord,
niet alleen door staten maar ook door niet-statelijke actoren, waaronder
decentrale groepen en zelfs individuen.'' Bovendien impliceert dit dat
cyberoorlog de potentiële schaalnadelen van massale gecentraliseerde
systemen tot uiting zal laten komen.

In de woorden van de Rand-experts: ``Op Informatie gebaseerde technieken
maken geografische afstand irrelevant; doelwitten op het continent van
de Verenigde Staten zijn net zo kwetsbaar als die in een
oorlogsgebied.'' Waar het vroeger als vanzelfsprekend leek om veilig te
kunnen wonen binnen de grenzen van grote supermacht-natiestaten zoals de
Verenigde Staten, kan de logica van machtsaggregatie in het
Informatietijdperk worden omgekeerd. Peoria mag dan ver van enig
potentieel militair front liggen, het zal niet langer veilig zijn voor
cyberaanvallen van vrijwel elke potentiële tegenstander. Wonen binnen de
grenzen van een supermacht betekent dat je jezelf in de roos plaatst. In
plaats van te fuseren, kunnen lokale gebieden zich veiliger maken door
te defragmenteren. De komst van cyberoorlog zal de kwetsbaarheid van
gecentraliseerde commandostructuren en controlesystemen vergroten,
terwijl de concurrentiële levensvatbaarheid van decentrale systemen
toeneemt.

De terugkoppelingsmechanismen die dit stimuleren, zouden het
devolutieproces kunnen versnellen. Zoals de Rand-experts suggereren, zal
de overheid, om de kwetsbaarheid voor cyberaanvallen van de
commandostructuren en controlesystemen die zich in de late fase van de
natiestaat hebben ontwikkeld te verminderen, verplicht zijn ``om het
gebruik van nieuwe encryptiesoftware'' te vergroten. Dit zal deze
voornamelijk private systemen veel minder kwetsbaar maken voor sabotage,
terwijl het de commerciële verspreiding van sterke encryptie versnelt,
wat zal bijdragen aan de bevrijding ervan van staatsdominantie. Ook dit
zal de devolutie stimuleren. Zoals we eerder betoogden, zal het
bovendien de verspreiding van middelen naar de cyberspace verder
bevorderen, waar ze buiten het bereik van de politiek zullen liggen.

Uiteindelijk betekent dit het einde van de massademocratie, vooral in
haar dominante vorm: representatief wanbestuur, of van het congres-
ofwel van het parlementaire type.

\section{De megapolitiek van
misvertegenwoordiging}\label{de-megapolitiek-van-misvertegenwoordiging}

Wanneer megapolitieke omstandigheden ingrijpend veranderen, zoals nu het
geval is, verandert onvermijdelijk ook de organisatie van het bestuur.
De representatieve overheid is in feite altijd een artefact geweest van
de verdeling van ruwe macht. Dit blijkt uit het simpele feit dat
vertegenwoordigers geografisch worden gekozen, en niet op een andere
manier.

Denk daar eens over na. In principe zou een wetgevend orgaan even
democratisch zijn als de leden volgens een willekeurige indeling van de
bevolking werden gekozen. Parlementaire kiesdistricten of
congresdistricten zouden kunnen worden gebaseerd op verjaardagen, of
zelfs op alfabetische kieskringen. Iedereen die op 1 januari geboren is,
zou dan stemmen op één lijst kandidaten. Degenen die op 2 januari
geboren zijn, op een andere. Of alle personen wier namen beginnen met
``Aa'' tot ``Af'' zouden kiezen uit een bepaalde lijst, terwijl degenen
met namen vanaf ``Ag'' uit een andere zouden kiezen.

Geen enkel systeem van die aard bestaat nu, om verschillende redenen. De
eerste en afdoende reden is dat het technologisch onpraktisch was in de
achttiende eeuw. Maar nog belangrijker is dat verjaardags- of
alfabetische kieskringen de verdeling van ruwe macht die de stem
destijds moest weerspiegelen niet zouden hebben weergegeven of zelfs
maar benaderd. Personen die niets meer deelden dan een geboortedag of
enkele beginletters van hun naam, zouden toen en nu uiterst moeilijk in
een coherente machtsbasis te organiseren zijn.

\subsection{Waarom wegen geografische indelingen
zwaarder?}\label{waarom-wegen-geografische-indelingen-zwaarder}

Het stemrecht begon feitelijk als een plaatsvervanger voor een militaire
confrontatie. En dat is het nog steeds, al dan niet verhuld. Zulke
conflicten kunnen worden georganiseerd langs geografische lijnen, en
soms ook langs verwantschaps- of religieuze lijnen. Ze kunnen echter
niet worden georganiseerd op basis van verjaardagen of initialen.
Evenmin volgens beroepen, behalve waar beroepen binnen erfelijke gildes
beperkt blijven, zoals de kasten in India, of lokaal clusteren zoals
boeren in Iowa.

Het hele punt van de huidige vormen van representatie is dat ze belangen
vertegenwoordigen die geografisch zijn verankerd, en niet volgens een
andere dimensie. Historisch was de sleutel tot militair succes het
beheersen van territorium. Alle militaire dreigingen ontstonden lokaal.
Representatieve systemen zijn zo ingericht dat ze die macht op een
andere manier laten gelden. Hun neiging om lokale belangen te dienen, is
simpelweg een product van het representatieve model. Door geografische
kiesdistricten richten politici zich op het bevoordelen van bepaalde
groepen, terwijl dit ten nadele gaat van de algemene belangen van de
bevolking.

\subsection{Nieuwe mogelijkheden in het
verschiet}\label{nieuwe-mogelijkheden-in-het-verschiet}

Zoals analyses door Public Choice-economen hebben aangetoond, hebben
schijnbaar kleine verschuivingen in de manier waarop een verkiezing
wordt georganiseerd of hoe de stemmen worden geteld, grote en
voorspelbare gevolgen voor de uitkomst. Dit is de reden waarom serieuze
studenten van politiek vandaag de dag ook serieuze studenten van
grondwetten moeten zijn. Het is een van de overwegingen die ons hebben
geleid om verder te kijken dan bestaande grondwetten, maar naar de
uiteindelijke metagrondwet, zoals bepaald door de heersende
megapolitieke factoren van een bepaalde omgeving.

Technologische veranderingen hebben al enkele van de fundamenten
weggespoeld die het stemmen tot geografische kiesdistricten beperkten.
Toen moderne representatieve systemen in de achttiende en negentiende
eeuw ontstonden, waren bijna alle communicatiekanalen lokaal. De meeste
mensen leefden en stierven binnen een paar kilometer van waar ze waren
geboren, en al hun handel en communicatie vond lokaal plaats.
Tegenwoordig is er onmiddellijke wereldwijde communicatie. Je kunt bijna
even gemakkelijk zaken doen met iemand vijfduizend kilometer verderop
als met een buurman. In toenemende mate overstijgt de economie
geografische beperkingen. De samenleving is veel mobieler.

En dat geldt ook voor vermogen in het Informatietijdperk. In
tegenstelling tot een staalfabriek kan een computerprogramma niet
gemakkelijk gegijzeld worden door het lokale politieke proces. Een
staalfabriek kan nauwelijks verplaatst worden als wetgevers besluiten
deze te belasten of de eigenaren te reguleren. Een computerprogramma kan
wereldwijd via een modem met lichtsnelheid worden verzonden. De eigenaar
kan zijn laptop pakken en wegvliegen. Dit ondermijnt ook de
megapolitieke fundamenten van geografische kiesdistricten.

Een groot probleem dat alle representatieve democratische systemen delen
in het licht van onze analyse, is dat hun geografische kiesdistricten de
gevestigde belangen van industriële ondernemingen te veel
vertegenwoordigen. De ``verliezers'' of ``achterblijvers'' zijn perfecte
kiezers: geografisch geconcentreerd en afhankelijk van politieke
aandacht. De geschiedenis van de industriële democratie bevestigt dit.
``Winnaars'' van nieuwe industrieën waren chronisch
ondervertegenwoordigd in wetgevende beraadslagingen, zelfs in de
hoogtijdagen van het Industriële Tijdperk in de jaren 1930. De neiging
van politici om bestaande, gevestigde concurrenten te vertegenwoordigen
in plaats van de nieuwe ondernemingen die zouden kunnen ontstaan of de
potentiële klanten van nieuwe ondernemingen, is waarschijnlijk een
inherent kenmerk van de representatieve overheid. Zoals Mancur Olson
betoogde in \emph{The Rise and Decline of Nations}, ontwikkelen
langdurige industrieën effectievere ``distributieve coalities'' om te
lobbyen en te strijden voor politieke voordelen.

Dit probleem wordt enorm vergroot in de economie van het
Informatietijdperk. De meer creatieve deelnemers aan de nieuwe economie
zijn geografisch verspreid. Daarom zullen ze waarschijnlijk niet
voldoende geconcentreerd zijn om de aandacht van wetgevers te trekken,
zoals zalmvissers in Schotland of tarweboeren in Saskatchewan dat doen.
Veel van de dynamische persoonlijkheden van de nieuwe economie zullen
waarschijnlijk zelfs geen burgers zijn van de meest omvattende
rechtsgebieden. Daarom zullen ze slechts een kleine ``stem'' hebben in
de wetgevende beraadslagingen van representatieve democratieën. De
dubieuze inspanningen van Amerikaanse wiskunde-PhD's om buitenlandse
wiskundigen ervan te weerhouden vacatures in te vullen in de Verenigde
Staten, is hier een veelzeggend voorbeeld van. Hun xenofobe voorstellen
aan het Congres om werkgevers te verhinderen mensen op basis van
verdienste in te huren, zullen al te vaak gehoor vinden. De verouderde
geografische vertegenwoordiging uit het Industriële Tijdperk houdt geen
rekening met buitenlandse wiskundigen of andere cruciale mensen die geen
kiezers zijn, maar bijdragen aan de welvaart.

\begin{quote}
Waarom geloven mensen in de legitimiteit van democratische instellingen?
Het beantwoorden van die vraag is bijna net zo moeilijk als uitleggen
waarom mensen in bepaalde religieuze dogma's geloven, want net als bij
religieuze overtuigingen varieert de mate van begrip, scepticisme en
geloof sterk in de samenleving en in de tijd. -- JUAN J. LINZ
\end{quote}

Slechts enkelen hebben serieus nagedacht over hoe technologische
veranderingen het industrialisme ondermijnen en de inkomensverdeling
beïnvloeden. Het is duidelijk dat democratie, als de inkomens zo sterk
uit elkaar liggen als in de informatie-economie mogelijk is,
waarschijnlijk niet veel meer zal zijn dan een recept voor gelegaliseerd
parasitisme. Nog minder mensen hebben de impliciete incompatibiliteit
opgemerkt tussen sommige instellingen van industriële overheden en de
megapolitiek van de postindustriële samenleving. Of deze tegenstellingen
nu expliciet worden erkend of niet, hun gevolgen zullen steeds
duidelijker worden naarmate voorbeelden van politieke mislukkingen zich
wereldwijd opstapelen. Instellingen van de overheid die in de moderne
periode zijn ontstaan, weerspiegelen de megapolitieke omstandigheden van
een of meerdere eeuwen geleden. Ze overleefden de overgang van een
agrarische samenleving naar stedelijk industrialisme, maar het
Informatietijdperk kan nieuwe mechanismen van vertegenwoordiging
vereisen om chronische disfunctionaliteit en zelfs een instorting zoals
de Sovjet-Unie te voorkomen.

Je kunt verwachten dat er in veel landen crises van wanbestuur zullen
plaatsvinden naarmate politieke beloften worden afgezwakt en overheden
zonder krediet en institutionele steun komen te zitten. Uiteindelijk
zullen nieuwe institutionele vormen moeten ontstaan die in staat zijn om
vrijheid te behouden onder de nieuwe technologische omstandigheden,
terwijl ze tegelijkertijd uitdrukking en leven geven aan de
gemeenschappelijke belangen die individuen delen.

Dit alles wijst op het einde van massademocratie zoals wij die in de
twintigste eeuw hebben gekend. De vraag is: wat zal het vervangen? Als
de enige alternatieve optie voor massademocratie een dictatuur zou zijn
waarin het individu geen zeggenschap heeft over zijn eigen lot, dan zou
men misschien geneigd zijn zich bij de neo-Ludditische ``opstand tegen
de toekomst'' aan te sluiten.

\subsection{Nieuwe instellingen}\label{nieuwe-instellingen}

Gelukkig is dictatuur niet het enige alternatief voor massademocratie.
Informatietechnologie maakt het mogelijk om eigen keuzes te maken. In
plaats van een collectieve beslissing binnen de beperkende context van
`massaproductie, massaconsumptie, massa-onderwijs, massamedia,
massa-vermaak en al het andere', faciliteert informatietechnologie een
werkelijk consumentgerichte keuze voor op maat gemaakte
soevereiniteitsdiensten. Dit wordt mogelijk doordat grootschalig
opereren niet langer noodzakelijk is. Wij zijn ervan overtuigd dat de
technologie van het Informatietijdperk zal leiden tot nieuwe vormen van
bestuur, net zoals de Agrarische Revolutie en later het Industriële
Tijdperk elk hun eigen specifieke vormen van sociale organisatie
voortbrachten.

Wat voor nieuwe instellingen zouden dat kunnen zijn? Als je het wilt
snappen, moet je vergeten wat er in literatuur van wat misleidend de
``politieke wetenschap'' genoemd wordt. De nieuwe bestuursvormen voor
het Informatietijdperk zullen de grenzen van conventioneel denken
overschrijden. De overgang naar zulke instellingen is al in gang gezet.
Het gaat om nauwelijks erkende improvisaties die gericht zijn op het
herstructureren van onderbenutte voordelen van soevereiniteit. De
natiestaten van de wereld, bezorgd over afscheidingsbewegingen en de
ingrijpende effecten van devolutie, hebben zich verenigd tot het
krachtigste grensbewaringskartel dat zij kunnen vormen. Hoewel het
aantal nieuwe staten in de jaren negentig wereldwijd is toegenomen,
gebeurde dit vooral in twee clusters dankzij de ineenstorting van
multi-etnische, communistische dictaturen in de voormalige Sovjet-Unie
en Joegoslavië. Het valt op dat andere vooraanstaande natiestaten,
waaronder de Verenigde Staten, strategieën hanteerden om de Sovjet-Unie
zo lang mogelijk in stand te houden. En maar weinig overheden stonden
positief tegenover de ontbinding van Joegoslavië. Pas toen de
afscheidingsbewegingen hun macht met eigen legers konden handhaven, werd
hun onafhankelijkheid erkend. De grootmachten namen er genoegen mee dat
ongewapende of slecht bewapende separatisten werden afgeslacht door hun
Servische belagers. Zelfs het verre China, een machtige natiestaat
zonder direct belang bij het voortbestaan van Joegoslavië, verzette zich
fel tegen de pogingen tot zelfbeschikking van de onderdrukte etnische
Albanezen in Kosovo. Ironisch genoeg zal deze obsessie met grensbehoud
juist de fragmentatie van soevereiniteit bevorderen in plaats van
voorkomen. Het felle verzet van kwetsbare natiestaten tegen openlijke
afscheiding en politieke fragmentatie maakt erkende soevereiniteit tot
een waardevolle vorm van transcendent kapitaal, wat staten vrijwillig
kunnen opsplitsen en onderverhuren.

Een voorbeeld van het vrijwillig fragmenteren van soevereiniteit,
waardoor in feite een particuliere, belastingvrije jurisdictie ontstaat,
is de Agulhas Bay Concession Free Zone, die vijftig vierkante kilometer
van het eiland Principe voor de kust van West-Afrika omvat. Hoewel dit
grondgebied binnen de grenzen van de Democratische Republiek São Tomé en
Principe valt, is het beheer van de zone geprivatiseerd. Het bestuur
wordt contractueel geregeld en verzorgd door WADCO, de \emph{West
African Development Corporation Ltd.}, een particulier bedrijf dat in
Zuid-Afrika is opgericht. In de zone is de voertaal Engels in plaats van
het officiële Portugees van São Tomé en Principe, en wordt de handel
niet gevoerd in het lokale Tomeaanse monopoliegeld, maar in de dollar,
het wereldgeld, de Amerikaanse dollar. De veiligheid wordt niet
gewaarborgd door de nationale politie van São Tomé en Principe, maar
door private politiediensten die door WADCO in dienst zijn genomen. Op
de handel binnen de zone is het commerciële recht van São Tomé niet van
toepassing en hebben de rechtbanken van São Tomé geen rechtsmacht. Alle
geschillen lossen de partijen op via transnationale arbitrage conform de
Parijse ICC-regels. Behalve een paar strikt gecontroleerde en triviale
uitzonderingen geldt de belastingwetgeving van São Tomé niet en
functioneren officiële monopolies daar eveneens niet. Zo worden
bijvoorbeeld telecommunicatiediensten binnen de zone automatisch
gedereguleerd. Onder voorbehoud van tijdige huurbetaling en naleving van
de overige concessievoorwaarden mag WADCO haar huurcontract op deze
particuliere, gefragmenteerde soevereiniteit automatisch en telkens met
vijftig jaar verlengen, gerekend vanaf de eerste verlengingsdatum in
2047.

Wat WADCO in São Tomé en Principe heeft bereikt, kan door anderen in
uiteenlopende rechtsgebieden worden gekopieerd, en dat zal ook gebeuren.
Een ware pionier van de ontwikkeling in de eenentwintigste eeuw, Joaquin
Aguirre, heeft in de Centrale Aguirre Portuaria in oostelijk Bolivia een
soortgelijke zone van particuliere soevereiniteit opgericht. Aguirre,
een multimiljonair, romanschrijver, uitvinder, medeoprichter van de
Verenigde Naties én voormalig senator van de Boliviaanse Republiek, is
op vele vlakken een echte pionier. Een halve eeuw nadat hij een
belangrijke rol speelde bij de oprichting van de Verenigde Naties,
belichaamt hij nu het archetype van het Soevereine Individu in de
eenentwintigste eeuw. Zijn Zona Franca, vrij van de meeste Boliviaanse
belastingen, heffingen en regelgevende beperkingen, wijst de weg naar
een nieuwe soort geprivatiseerde stadstaten die door succesvolle
individuen in het Informatietijdperk steeds vaker gerealiseerd zullen
worden. Het laat duidelijk zien dat vrijhandelszones zoals die van Señor
Aguirre, door economische groei te stimuleren, de levens van de massa's
ingrijpend kunnen verbeteren; massa's die zo vaak door apologeten van de
grote overheid worden geprezen. Geleidelijk zal het aantal de facto
stadsstaten wereldwijd aanzienlijk toenemen. Inderdaad, als je als
individu voldoende financiële onafhankelijkheid bereikt, zul je de
ultieme mate van onafhankelijkheid kunnen verwerven, net zoals Joaquin
Aguirre. Mocht een gefragmenteerde, gecommercialiseerde soevereiniteit
van een ander je geen comfortabele thuisbasis bieden, dan kun je altijd
je eigen ministaat opzetten, net zo onafhankelijk als een middeleeuws
hertogdom. In plaats van met demagogen en politieke opportunisten de
strijd aan te gaan om te voorkomen dat je bezittingen worden weggeharkt
en verdeeld onder de talloze kravende handen van de massademocratie, kun
je je eigen particuliere domein van bestuur creëren.

De ingrijpende overgang van massademocratie naar de ultieme vorm van
zelfbestuur, individuele soevereiniteit, hoeft niet gepaard te gaan met
een radicale verandering in de publieke opinie, noch met een miraculeuze
stemmingsuitslag waardoor ontgoochelde kiezers massademocratie
afschaffen. Zo'n revolutie kan namelijk al onopgemerkt zijn ingezet door
grondgebied te verhuren voor soevereine belastingvrije zones, `Zona
Francas', en vrije havens. Op termijn zal soevereiniteit steeds verder
worden gefragmenteerd, tot het punt waarop verdere opsplitsing niet meer
genoeg waarde oplevert om de transactiekosten van de devolutie te
compenseren. Gelet op de wet van Moore en Gilder's corollarium, dat
stelt dat de bandbreedte elk jaar verdrievoudigt, is er momenteel geen
reden om te verwachten dat de devolutietrend vroegtijdig zal stoppen.
Integendeel, wij voorzien dat de schijnbaar solide macht van de
natiestaten die vandaag de dag massademocratie belichamen, zal
uiteenvallen in tienduizenden fragmenten, in een systeem dat meer doet
denken aan de Middeleeuwen dan aan het moderne Industriële Tijdperk.

Op termijn zullen zelfs natiestaten die nog restanten van
massademocratie in zich dragen, een ingrijpende beleidsverschuiving
doormaken om in lijn te komen met de nieuwe metaconstitutionele
realiteiten. Zoals William Keech, een vurig voorstander van democratie,
in \emph{Economic Politics: The Costs of Democracy} betoogt: `Mensen
leren te willen wat ze zien dat ze kunnen krijgen, maar ze kunnen ook
van gedachten veranderen als ze merken dat ze niet houden van wat ze
wilden en wat ze kregen.' Met andere woorden, dat massademocratie met
haar conventionele instituties overal wordt geprezen nu de twintigste
eeuw ten einde loopt, zou wel eens een ``verkoopsignaal'' kunnen zijn.
Dit garandeert echter geenszins dat deze besluitvormingsregels de tand
des tijds zullen doorstaan, zelfs niet volgens hun eigen maatstaven.
Bedenk dat, als je buiten de politiek kijkt, er nauwelijks bewijs is dat
bestuurders, beheerders, coaches of andere professionele leiders via
democratische keuzes worden aangesteld. Integendeel, de meest
succesvolle leiders worden routinematig door eigenaars voorgedragen via
selectieprocessen, waarbij degenen met het grootste belang een
onevenredig grote stem hebben. Als democratische selectie werkelijk een
universele methode zou zijn om capabele leiders te vinden, zou je
verwachten dat dit vrijwel uitsluitend in de politieke sfeer gebeurt.
Kortom, op basis van de huidige gegevens lijkt het aannemelijker dat de
levering van soevereine diensten wordt belemmerd door de overheersing
van democratische besluitvorming, dan dat bedrijven en
handelsorganisaties eronder lijden dat zij geleid worden door door
eigenaars aangestelde bestuurders in plaats van door een stemming.

Tegen de helft van de eenentwintigste eeuw zal de toename van
particuliere jurisdicties, gebaseerd op gefragmenteerde soevereiniteit,
onomstotelijk de voordelen van particuliere administratie bewijzen.
Kiezers zullen merken dat zij de lasten van massademocratie dragen.
Daarom zullen zij, zoals professor Keech suggereert, concluderen dat de
voordelen van werknemerscontrole over de overheid de kosten niet
compenseren en zetten zij de stap richting hervorming. Zelfs electoraten
in Europa en Noord-Amerika, die nu schijnbaar fel tegen hervorming zijn,
zouden uiteindelijk kunnen stemmen om hun regio's meer open te stellen
voor particuliere bestuursvormen. Meerderheden zullen wellicht
vrijwillig, zelfs graag, de politieke schijnvertoning opgeven ten gunste
van een particuliere bestuursvorm die er werkelijk op gericht is de
optimale voorwaarden te scheppen voor het sluiten en afdwingen van
contracten.

Voor zover de overheid met al haar vertrouwde toebehoren überhaupt nog
zal bestaan, zou ze op totaal nieuwe manieren kunnen worden
geïnformeerd. Ergens, in een bepaald rechtsgebied, nog voordat het te
laat is, zal iemand het potentieel van computertechnologie doorgronden,
waardoor een werkelijk representatieve overheid mogelijk wordt. We
kunnen het probleem van buitensporige campagne-uitgaven en de
onmiskenbare irritatie over voortdurende politieke campagnes binnen een
oogwenk oplossen. In plaats van verkozen te worden, selecteren we de
vertegenwoordigers volledig willekeurig via trekking, zodat hun talenten
en zienswijzen nauw aansluiten bij die van de algemene bevolking.

Dit is in feite een moderne variant van het oude Griekse systeem waarbij
men door loting functies vervulde. Zoals E. S. Staveley beschrijft in
zijn gezaghebbende geschiedenis \emph{Greek and Roman Voting and
Elections} selecteerde Athene talrijke ambtspersonen, van magistraten
tot archons, via trekking in plaats van via verkiezingen.{[}13{]} Ze
deden dit op een slimme manier, ondanks de mechanische beperkingen bij
het willekeurig genereren van kansen, door gebruik te maken van een
sorteermachine, of zoals de Atheners het noemden: de cleroterion.

Men gebruikte een rij zwart-witte bonen als willekeurige tellers om te
bepalen wie voor diverse ambten in aanmerking kwam en om de volgorde
vast te stellen waarin de tribale afdelingen als prytaneis in de Raad
aan de beurt waren. Het feit dat dit idee uit de klassieke oudheid
stamt, geeft het extra geloofwaardigheid. Het meest aantrekkelijke aan
dit systeem is echter dat het de nadelen van zelfselectie in de politiek
voorkomt. Statistisch gezien kunnen minder advocaten en egomaniakken
hierdoor de publieke aangelegenheden domineren.

Wetgevende lichamen kunnen zo worden gevormd uit ware
vertegenwoordigers. Doordat zij niet worden samengebracht door een
honger naar macht en vrijwel nooit opnieuw per toeval worden gekozen,
hebben zij de vrijheid om staatszaken te beheren en beleid te formuleren
op basis van een rationele analyse van de kwesties.

\subsection{Rechtstreekse commissie}\label{rechtstreekse-commissie}

Tegenwoordig worden politici die erop uit zijn stemmen te maximaliseren,
weinig geprikkeld om problemen op een samenhangende manier te
analyseren. Het is dan ook nauwelijks verrassend dat hun resultaten bij
het daadwerkelijk oplossen van problemen zo erbarmelijk zijn in
vergelijking met ondernemers, bedrijfsleiders en coaches van sportteams,
die worden beloond op basis van prestaties. Prestatiegerichte beloning
voor wetgevers zou niet iedereen die willekeurig wordt gekozen even
doeltreffend maken als Lee Kuan Yew. Maar er is alle reden om te geloven
dat de prestaties sterk zouden verbeteren als het salaris van wetgevers
werd gekoppeld aan een objectieve maatstaf voor prestaties, zoals de
groei van het reëel beschikbaar inkomen per hoofd van de bevolking.
Betaal hen op basis van prestaties, en de kans dat ze daadwerkelijk
zouden presteren zou duizendvoudig toenemen.

De winst voor de samenleving van beleid dat het reëel inkomen na
belastingen verbetert, zou enorm kunnen zijn. Waarom zouden we premiers
en presidenten zelfs geen klein aandeel geven in de winst die hun beleid
oplevert? De financiering van zulke betalingen zou kunnen worden
verzameld via een kleine, onopvallende belasting. Een dergelijke
regeling zou de samenleving bevrijden van de dreiging die zij nu
ondervindt van ambitieuze mannen met een gespecialiseerd politiek talent
zoals Richard Nixon en Bill Clinton.

\begin{quote}
``Ze brachten hem goud, zilver en kleding; maar de `Christus' verdeelde
al deze dingen onder de armen. Wanneer geschenken werden aangeboden,
wierpen hij en zijn vrouwelijke metgezel zich ter aarde en baden; maar
vervolgens, wanneer hij opstond, beval hij de menigte hem te aanbidden.
Later organiseerde hij een gewapende bende, die hij door het platteland
leidde en reizigers beroofde die ze onderweg tegenkwamen. Maar ook hier
was zijn ambitie niet om rijk te worden, maar om aanbeden te worden. Hij
verdeelde alle buit onder degenen die niets hadden, waaronder, mag men
aannemen, zijn eigen volgelingen.'' NORMAN COHN
\end{quote}

\subsection{Messiaanse
persoonlijkheden}\label{messiaanse-persoonlijkheden}

Er wordt te weinig stilgestaan bij het feit dat verkiezingspolitiek
chaotische, messiaanse persoonlijkheden aantrekt. Zulke figuren hebben
altijd bestaan en vormden vaak een serieuze bedreiging voor de sociale
orde, zelfs in agrarische samenlevingen lang voordat democratische
systemen ontstonden. Kijk je naar de loopbanen van Eudo de Stell, de
Bretonse Christus, Adelbert in de achtste eeuw, Eon in de elfde eeuw,
Tanchelm uit Antwerpen, Melchior Hoffman, Bernt Rothmann en
soortgenoten, valt één ding op: hoe opvallender hun politieke talenten
leken te zijn, hoe groter de schade die zij aanrichtten. Omdat de staat
nog niet de handen had uitgeslagen in het organiseren van grootschalige,
systematische dwang, namen deze vroege protopolitici vaak zelf het
initiatief om te plunderen en te beroven. Zo vergaarden zij geld, dat ze
vervolgens onder hun volgelingen, de armen, verdeelden.

\subsection{Protopolitici in actie}\label{protopolitici-in-actie}

De verhalen over hun capriolen doen denken aan talenten die hun tijd
vooruit waren, alsof je leest over mannen van twee meter die over een
veld rennen, lang voordat basketbal was uitgevonden. Tegenwoordig
verdienen deze reuzen dankzij het \emph{NBA} miljoenen met dribbelen en
dunken. Mocht basketbal verdwijnen, dan zouden zij weer in de kieren van
de samenleving verdwijnen, hoogstwaarschijnlijk als circusacts en
kermisattracties.

Voordat de politiek was uitgevonden, werden demagogen aangetrokken tot
het dichtstbijzijnde equivalent dat de agrarische wereld te bieden had:
rondtrekkende prediking. Ze spraken menigten toe en beloofden, net als
politici, welsprekend een beter leven aan degenen die hen zouden volgen.
Toen en nu waren de armen het belangrijkste doelwit van demagogen. In
\emph{The Pursuit of the Millennium} vertelt Norman Cohn de geschiedenis
van vele messiaanse leiders die leefden vóór de invoering van
verkiezingen. In zijn beschrijvingen zijn de sterke overeenkomsten met
het persoonlijkheidstype van de charismatische politicus uit de moderne
tijd gemakkelijk te herkennen.

\begin{quote}
``{[}D{]}e leider bezit, zoals de farao en vele andere `goddelijke
koningen', alle eigenschappen van een ideale vader: hij is volmaakt
wijs, volmaakt rechtvaardig, en hij beschermt de zwakken. Anderzijds is
hij echter ook de zoon wiens taak het is de wereld te transformeren, de
Messias die een nieuwe hemel en een nieuwe aarde zal vestigen en die van
zichzelf kan zeggen: `Aanschouw, Ik maak alle dingen nieuw!' En zowel
als vader als zoon is deze figuur kolossaal, bovenmenselijk, almachtig.
Men schrijft hem een overvloed aan bovennatuurlijke krachten toe, die
worden voorgesteld als een licht dat van hem uitstraalt. \ldots{}
Doordrongen van deze goddelijke geest bezit de eschatologische leider
unieke wonderbaarlijke krachten. Zijn legers zullen altijd en
onvermijdelijk zegevieren. Zijn aanwezigheid zal de aarde vruchtbaar
maken; zijn heerschappij zal een tijdperk van volmaakte harmonie brengen
zoals de oude, verdorven wereld nooit heeft gekend.

Dit beeld was natuurlijk pure fantasie, in die zin dat het geen enkele
relatie had tot de werkelijke aard en capaciteiten van enig mens die
ooit heeft bestaan of zou kunnen bestaan. Toch was het een beeld dat kon
worden geprojecteerd op een levend mens; en er waren altijd mensen die
maar al te bereid waren zo'n projectie te aanvaarden, die er zelfs vurig
naar verlangden te worden gezien als onfeilbare, wonderverrichtende
verlossers. \ldots{} En het geheim van de macht die zij uitoefenden lag
nooit in hun geboorte, noch in hun opleiding, maar altijd in hun
persoonlijkheid. Tijdgenoten benadrukken in hun beschrijvingen van deze
messiassen van de armen doorgaans hun welsprekendheid, hun
indrukwekkende voorkomen en hun persoonlijke uitstraling. Bovenal krijgt
men de indruk dat, zelfs als sommigen van deze mannen misschien bewuste
bedriegers waren, de meesten zichzelf werkelijk zagen als geïncarneerde
goden. \ldots{} En deze totale overtuiging deelde zich gemakkelijk mee
aan de menigten wier diepste verlangen juist dat was: een
eschatologische verlosser.''
\end{quote}

Hoewel het fragment bewonderenswaardig beknopt de zelfverklaarde
millenaristische redders die het middeleeuwse leven opschudden
beschrijft, kan het de volle diepgang van Cohns grootse werk niet
overbrengen. Wie het hele werk leest, herkent in het gedrag van deze
\emph{prophetae} de vertrouwde kenmerken van de moderne demagoog: de
welsprekendheid, het ``persoonlijk magnetisme,'' de ``messiaanse
pretenties,'' en het terugkerende verlangen om te worden aanbeden als
tribuun van de armen.

Het belangrijkste verschil tussen de ontvangst van zulke bedriegers in
de middeleeuwse samenleving en die in de democratie aan het einde van de
twintigste eeuw is dat zij in de Middeleeuwen meestal werden
geëxecuteerd, terwijl de moderne democratische politiek hun aan het
einde van de twintigste eeuw een legitiem kanaal biedt om de macht in de
natiestaat te grijpen. Een systeem dat routinematig de controle over de
grootste en dodelijkste ondernemingen ter wereld overdraagt aan de
winnaar van populariteitswedstrijden tussen charismatische demagogen,
moet daar op de lange termijn de gevolgen van ondervinden.

\subsection{Betaal leiders om goed werk te
leveren}\label{betaal-leiders-om-goed-werk-te-leveren}

Zoals hierboven al werd aangegeven, is het vrij eenvoudig om een betere
manier te vinden om talentvol leiderschap binnen een organisatie veilig
te stellen, namelijk door deze leiders daadwerkelijk aan te stellen.
Deze werkwijze wordt in concurrerende economieën op grote schaal en met
succes toegepast.

Een doordacht selectieproces, gecombineerd met een stimulerend
beloningssysteem voor positief leiderschap, zorgt ervoor dat bekwame
mensen de overheid gaan leiden. Het trekt bovendien nieuw talent aan dat
zich anders nauwelijks zou interesseren voor bestuurlijke vraagstukken.

De meest getalenteerde topfunctionarissen wereldwijd zouden
wanpresterende regeringen met enthousiasme overnemen als hun beloning
afhangt van de concrete resultaatsverbetering die zij voor de
samenleving realiseren. Een leider die het reële inkomen in elk
vooraanstaand Westers land fors weet te verhogen, verdient terecht veel
meer dan Michael Eisner. In een betere wereld is iedere succesvolle
regeringsleider multimiljonair.

\subsection{Elektronische
plebiscieten}\label{elektronische-plebiscieten}

Een andere voor de hand liggende oplossing voor representatief
wanbestuur is het inzetten van elektronische plebiscieten, waarbij
burgers, bijvoorbeeld een representatieve groep die via een
manipulatiebestendige loting wordt samengesteld, rechtstreeks stemmen op
wetgevende voorstellen. Dankzij computertechnologie kunnen we op een
democratische wijze besluiten vaststellen via deze plebiscieten. Je kunt
ze bovendien eenvoudig combineren met een toewijzingssysteem, zodat per
specifiek vraagstuk slechts een beperkt aantal stemmers meedoet. Hoe dan
ook, het is in principe veel minder moeilijk voor potentiële kiezers om
politieke kwesties zelf te begrijpen dan om politici te doorgronden, hun
oordelen over die kwesties te beoordelen, laat staan te weten wat zij
daadwerkelijk zouden doen eenmaal aan de macht. Dit wordt alleen maar
ingewikkelder nu politici en hun medewerkers steeds bedrevener worden in
het verpakken en manipuleren van de beelden die ze aan het publiek
presenteren.

\section{Gecommercialiseerde
soevereiniteit}\label{gecommercialiseerde-soevereiniteit}

We verwachten dat er een nieuw systeem ontstaat dat de traditionele
politiek vervangt. Hoewel we in elk van de eerder genoemde opties wel
wat voordelen zien, verwachten we niet dat hervormingen de politiek
nieuw leven inblazen. Integendeel, zij raakt verouderd en verliest op
veel vlakken haar relevantie. We bedoelen hiermee niet dat wij een
dictatuur voor ogen hebben, maar dat de overheid een ondernemende,
commerciële vorm aanneem: een commercialisering van soevereiniteit.

In tegenstelling tot een dictatuur of zelfs democratie sluit
gecommercialiseerde soevereiniteit de individuele keuzevrijheid niet
uit. Integendeel, zij geeft ieder mens meer ruimte om zijn ideeën te
uiten. Bovendien biedt dit systeem diegenen die er hun voordeel mee
weten te doen een praktischere omgeving voor besluitvorming en
zelfbepaling dan alle andere vormen van sociale organisatie die we tot
nu toe kenden.

\subsection{Op maat gemaakte overheid}\label{op-maat-gemaakte-overheid}

Mocht dit wat millenaristisch klinken, onthoud dan dat microtechnologie
verkleint en uit elkaar haalt. Het faciliteert maatwerk in plaats van
massaproductie. Je kunt tegenwoordig een winkel binnengaan en een
spijkerbroek kopen die wordt gesneden volgens een patroon dat op jouw
maten is afgestemd en die aan de andere kant van de wereld in elkaar
wordt genaaid. Wanneer nieuwe instituties zich uiteindelijk ontwikkelen
om te passen bij de nieuwe megapolitieke werkelijkheden van het
Informatietijdperk, zul je bestuur kunnen verkrijgen dat minstens even
goed is afgestemd op jouw persoonlijke behoeften en voorkeuren als een
spijkerbroek.

Alvin Toffler, van alle mensen, heeft kritiek geuit op het idee dat
informatietechnologie burgers tot klanten zou kunnen maken. Toffler zegt
naar onze mening ten onrechte: ``Dat is een veel te beperkt model. Of we
het nu leuk vinden of niet, er bestaat een wereld van religie en gevoel
die niet eenvoudig kan worden teruggebracht tot contractuele relaties.''
Wij zouden, om redenen die we eerder hebben besproken, ermee instemmen
dat het moeilijk zal zijn ``de wereld van nationalistische gevoelens''
te ``reduceren tot contractuele relaties.'' Maar dat zeggen is nog niet
hetzelfde als beweren dat het onmogelijk is, laat staan dat het een
slechte regeling zou zijn. Iets minder irrationeel enthousiasme in het
nationalisme zou miljoenen levens kunnen redden.

\subsection{Toetreding, uittreding en
inspraak}\label{toetreding-uittreding-en-inspraak}

Natuurlijk is de commercialisering van soevereiniteit een onbekend
concept, blijkbaar zelfs voor Alvin Toffler. Toch is het kernidee, dat
van economische expressie, gewoon in het dagelijks leven van eind
twintigste-eeuwse mensen. In elke enigszins vrije economie kunnen
consumenten hun wensen rechtstreeks uitdrukken door diensten en
producten te kopen, of door hun klandizie in te trekken. Wanneer je
ontevreden raakt over een bepaalde versie van een product of over een
aanbieder van een dienst, kun je je ontevredenheid direct uitdrukken via
een ``exit''. Met andere woorden: je kunt je zaken elders onderbrengen.

In de voorgaande hoofdstukken hebben we geanalyseerd hoe de vooruitgang
van informatietechnologie het binnenkort mogelijk zal maken om
bezittingen in de cyberspace te creëren die praktisch immuun zijn voor
roofzuchtige invasie door natiestaten. Dit zal in feite een
metaconstitutionele voorwaarde scheppen dat overheden je daadwerkelijk
tevredenstellende diensten moeten leveren vóórdat je hun rekeningen
betaalt. Waarom? Omdat inkomstenbelasting in de praktijk bijna even
vrijwillig zal worden als het in theorie verondersteld wordt te zijn.

\subsection{Het vermijden van ``logge politieke
kanalen''}\label{het-vermijden-van-logge-politieke-kanalen}

In feite, als informatietechnologie zich ontwikkelt zoals verwacht, zal
dit garanderen dat overheden daadwerkelijk door hun klanten worden
bestuurd. Als klant zul je eerst honderden, en later duizenden opties
hebben om je beschermingskosten rechtstreeks te verlagen via het
afsluiten van een privaat belastingcontract met een natiestaat of door
volledig over te stappen van natiestaten naar opkomende
minisoevereiniteiten. Deze contractuele opties voor ``entry'' of
``exit'' zijn economische uitdrukkingen van jouw wensen als klant.
Stemmen met je voeten en je geld heeft het grote voordeel dat het leidt
tot de resultaten die je wenst.

Hoe verhouden je opties om als klant toe of uit te treden zich tot de
politieke expressie in een democratie? Personen die ontevreden zijn over
een product of dienst, vooral als deze door de overheid wordt geleverd
of sterk gereguleerd wordt, kunnen hun mening kenbaar maken door brieven
te schrijven aan de president in de Verenigde Staten, of een ontmoeting
te vragen met hun parlementslid of een andere gekozen functionaris
elders. Soms werken zulke brieven, maar niet altijd. Meestal niet.
Indien dit niet lukt, kunnen personen die hun ``stem'' willen gebruiken
voor verandering een demonstratie organiseren, een advertentie plaatsen
in een krant, of zelfs zelf een verkiezingskandidaat worden.

De politieke vorm van expressie biedt een kanaal voor welsprekende
verklaringen en oratorische vaardigheden. Het nadeel is echter dat je
zelden tevredenheid of verbetering van je positie kunt verkrijgen door
je eigen handelingen. Wanneer je geconfronteerd wordt met een ondermaats
product of dienst van de overheid, ben je verplicht ervoor te blijven
betalen totdat je het hele politieke proces zover krijgt om je verzoek
tot verandering in te willigen.

In westerse landen, en nu vrijwel de gehele aarde, betekent dit dat je
de meerderheid in een democratisch systeem moet overtuigen. De noodzaak
om een meerderheid te betrekken legt enorme transactiekosten op tussen
jou en het bereiken van wat waarschijnlijk een relatief eenvoudig en
rationeel doel is.

Milton Friedman besprak de voordelen van de economische, in plaats van
de politieke expressie in zijn voorstel voor schoolvouchers in
\emph{Capitalism and Freedom}:

\begin{quote}
Ouders zouden hun mening over scholen rechtstreeks kunnen uitdrukken
door hun kinderen van de ene naar de andere school te sturen, veel meer
dan nu mogelijk is. Over het algemeen kunnen ze deze stap nu alleen
nemen door van woonplaats te veranderen. Voor de rest kunnen ze hun
mening alleen uitdrukken via omslachtige politieke kanalen.
\end{quote}

Albert O. Hirschman, als voorstander van politiek, keerde zich tegen
Friedmans voorkeur voor ``exit'' als de `directe' manier om
ontevredenheid over een organisatie te uiten: ``een persoon die minder
goed is opgeleid in economie zou naïef kunnen suggereren dat de directe
manier om je mening te uiten, is om ze te uiten!''

Of het directer of effectiever is om je mening te uiten via
marktmechanismen, zoals het geven of intrekken van je steun als klant,
of via ``omslachtige politieke kanalen'', is een complexe en
controversiële kwestie. Verschillende personen zullen hier verschillend
over denken. Voor degenen wier primaire betrokkenheid bij politieke
expressie bestaat uit het opeisen van voordelen ten koste van anderen,
kan overschakelen naar de economische expressie inderdaad een armoedig
alternatief lijken voor het schrijven aan en het eisen van meer van een
politicus.

\subsection{Economische expressie en ``wederkerige
socialiteit''}\label{economische-expressie-en-wederkerige-socialiteit}

Voor degenen die hun medemensen willen betrekken bij ``reciproke'' in
plaats van ``coërcieve'' of parasitaire sociale interacties, opent de
economische vorm van expressie het vooruitzicht om veel grotere
tevredenheid te bereiken tegen lagere tijds- en inspanningskosten.
Ondanks professor Hirschfield is dit eenvoudig aan te tonen.

Elke reeks economische uitdrukkingen, bestaande uit toetreding, lopende
contracten en uittreding, kan worden omgezet in een uitdrukking van een
politieke ``stem'' door een menigte mensen bij de besluitvorming te
betrekken. Probeer het als experiment. Je hebt alleen enkele honderden
mensen voor het experiment nodig die vinden dat er niet genoeg politiek
in hun leven is. In plaats van hun besteedbaar inkomen te besteden aan
duizenden afzonderlijke aankopen gedurende een jaar, zullen ze deze
veelheid aan economische beslissingen omzetten in een handvol politieke
beslissingen.

Om te beginnen zullen ze allemaal ermee in moeten stemmen hun
besteedbaar inkomen te bundelen en daarna individuele aankopen op te
geven. In plaats van individueel duizenden dollars uit te geven op
duizenden manieren, zal iedereen één of enkele stemmen krijgen,
afhankelijk van het aantal te vervullen functies. In plaats van geld
direct uit te geven om te verkrijgen wat je op elk moment wil, zal je je
stem of stemmen besteden aan de handvol gelegenheden waarop verkiezingen
worden gehouden om vertegenwoordigers te kiezen die dan zullen beslissen
hoe de nu gigantische collectieve portemonnee zal worden besteed.

Jij, samen met de anderen, zal dan delen in de consumptie van die items,
en alleen die items die het bestuur in naam van de meerderheid
goedkeurde.

Lijkt dat al op een ``log politiek kanaal'' voor expressie? Wacht maar.
Dit model biedt al het potentieel voor welsprekendheid en overtuiging
dat op nationaal niveau in de politiek te vinden is, en ook het grootste
potentieel voor frustratie.

Bijvoorbeeld, als je van verse broccoli houdt, en de groep heeft een
gewone verdeling van voedselvoorkeuren, zit je in de problemen. De kans
is groot dat sommigen of de meesten in je groep liever meer van het
gemeenschappelijke voedselbudget aan rood vlees zouden besteden dan aan
verse groenten. Om te voorkomen dat het kantinecomité naar een
groothandel gaat en het hele jaarlijkse groentebudget verkwist aan
ingeblikte erwten en maïs, zou je misschien je ``stem'' moeten laten
horen. Je zou de groep kunnen wijzen op de relatieve voordelen van het
consumeren van meer vitaminen en fytonutriënten zoals sulforafaan in
broccoli, vergeleken met meer verzadigde vetten en cholesterol uit rood
vlees.

Precies hoe je dit of enig ander punt duidelijk maakt, zou in dit
geconstrueerde politieke model net zo'n puzzel zijn als voor
voorstanders van welke politieke zaak of kandidaat dan ook. Je zou een
toespraak kunnen houden, maar dat vereist dat een groot deel van de
groep die je moet overtuigen al ergens is verzameld en bereid is te
luisteren. Je zou flyers kunnen drukken, mits zo'n ``campagne-uitgave''
door de huisregels van je politieke spel is toegestaan. Je zou brieven
kunnen schrijven. Deze beide opties zijn er echter van afhankelijk dat
de andere deelnemers geletterd genoeg zijn om te lezen.

\begin{quote}
Het schetst een beeld van een samenleving waarin de overgrote
meerderheid van de Amerikanen niet weet dat ze niet over de vaardigheden
beschikken die ze nodig hebben om hun brood te verdienen in onze steeds
technologischere samenleving en internationale markt. -- RICHARD RILEY,
Amerikaanse minister van onderwijs, in ``Adult Literacy in America''.
\end{quote}

\subsection{Negentig miljoen
Alzheimerpatiënten?}\label{negentig-miljoen-alzheimerpatiuxebnten}

Als jouw groep in dit politieke experiment toevallig uit Amerikanen zou
bestaan, zou het buitengewoon moeilijk zijn om een overtuigende
boodschap te laten doordringen, vooral als de groepsleden vergelijkbaar
zouden zijn met het Amerikaanse electoraat als geheel. De vaststelling
dat onevenredig veel burgers van 's werelds machtigste natiestaat
onderpresteren, is somber genoeg bevestigd door het meest grondige
onderzoek dat ooit uitgevoerd is naar de competentie van volwassen
Amerikanen. De studie \emph{Adult Literacy in America} toont aan dat het
vinden van een geletterd publiek voor wat voor politiek betoog dan ook
alles behalve eenvoudig is. Een groot deel, wellicht zelfs een
meerderheid, van de Amerikanen ouder dan vijftien mist de
basisvaardigheden die essentieel zijn om ideeën te evalueren en oordelen
te vormen. Volgens het Amerikaanse ministerie van Onderwijs kunnen 90
miljoen Amerikanen geen brief schrijven, een busdienstregeling begrijpen
of zelfs optellen en aftrekken met een rekenmachine. Dat is ongeveer wat
je zou verwachten als 90 miljoen Amerikanen zich in verschillende stadia
van de ziekte van Alzheimer zouden bevinden. Dertig miljoen werden zo
incompetent geacht dat ze niet eens in staat waren op vragen te
antwoorden.

Dus als jouw gezondheidsboodschap het tij niet kon keren, dat anders
vanzelf zijn weg zou hebben gevonden, dan zou je hulp kunnen inroepen
van dierenrechtenactivisten. Misschien zou je hen ertoe kunnen brengen
om te protesteren tegen je tegenstanders in het kantinecomité of om bij
invloedrijke leden thuis ophef te maken over het kwaad van het doden van
koeien.

Dit voorbeeld kan oneindig worden uitgebreid, waarschijnlijk veel langer
dan het geduld van rationele mensen zou toestaan. Het toont duidelijk
aan dat (1) elke economische expressie van toe of uittreding kan worden
omgezet in een politieke uitdrukking door er een collectieve beslissing
van te maken; en (2) dat collectieve beslissingen, ondanks hun
uitnodiging tot welsprekendheid, inderdaad omslachtig en vaak moeilijk
hanteerbaar zijn.

Dat is precies wat de ervaring heeft uitgewezen. Het is verre van
eenvoudig om de inspanning te mobiliseren die nodig is om de koers van
een democratie te veranderen. Dit is waarschijnlijk precies de reden
waarom democratische verzorgingsstaten eeuwenlange concurrentie met
alternatieve bestuursvormen hebben overleefd en aan het einde van het
Industriële Tijdperk zijn gaan domineren. Democratie slaagde als
politiek systeem juist omdat de werking ervan het voor burgers moeilijk
maakte om de overheid te controleren of het opeisen van middelen door de
staat te beperken.

Aangezien een onbeperkt partnerschap van de staat in jouw zaken in het
Informatietijdperk geen militair voordeel meer oplevert, zullen
vindingrijke mensen superieure manieren vinden om de weinige waardevolle
diensten die overheden daadwerkelijk leveren, elders te verkrijgen.
Feitelijke macht zal waarschijnlijk worden uitbesteed aan collectieve
mechanismen die hun kosten niet langer rechtvaardigen. We verwachten dat
efficiëntie de overhand zal krijgen boven massamacht. Zoals Neil Munro
kernachtig stelde:

\begin{quote}
``{[}Het is{]} gecomputeriseerde informatie, niet mankracht of
massaproductie, die de Amerikaanse economie steeds meer aandrijft en die
oorlogen zal winnen in een wereld met 500 televisiezenders. De
gecomputeriseerde informatie bestaat in cyberspace, de nieuwe dimensie
die is ontstaan door de eindeloze vermenigvuldiging van
computernetwerken, satellieten, modems, databanken en het publieke
internet.''
\end{quote}

In zo'n wereld zullen massale legers weinig betekenen. Efficiëntie zal
belangrijker zijn dan ooit. Omdat microtechnologie een nieuwe dimensie
van bescherming creëert, zoals besproken in hoofdstuk 6 en elders,
zullen individuen voor het eerst in de menselijke geschiedenis in staat
zijn om activa te creëren en te beschermen die volledig buiten het
territoriale geweldsmonopolie van welke overheid dan ook vallen. Deze
activa zullen dus veel ruimte laten voor individuele controle. Het zal
volkomen redelijk zijn voor jou, en voor aanzienlijke aantallen
toekomstige Soevereine Individuen, om met je voeten te ``stemmen'' door
uit vooraanstaande natiestaten te stappen en persoonlijke bescherming te
contracteren bij een andere natiestaat of een nieuwe minisouvereiniteit
die slechts een commercieel aanvaardbaar bedrag rekent in plaats van het
grootste deel van je nettovermogen. Kortom: je zou waarschijnlijk \$50
miljoen aannemen om naar Bermuda te verhuizen.

\subsection{Eerst ``exit'', later
contracteren}\label{eerst-exit-later-contracteren}

De eerste stimulans voor de commercialisering van soevereiniteit zal
moeten komen van individuen die zich economisch uitdrukken via een exit.
Deze optie zal het moeilijkst zijn in de Verenigde Staten, waar ze
tegelijk het meest waardevol zal blijken. De ``Berlijnse Muur'' voor
kapitalisten, opgelegd door president Bill Clinton en het Republikeinse
Congres, staat haaks op de slogan die Amerikaanse nationalisten in de
jaren zestig met zoveel zelfvertrouwen verkondigden: \emph{``Love it or
leave it.''} Door strafheffingen op te leggen aan degenen die
vertrekken, is de exit tax bedoeld om loyaliteit af te dwingen. Toch kan
deze wraakzuchtige wetgeving, die doet denken aan de sancties tegen
vluchtende grondbezitters in de laatste dagen van het Romeinse Rijk,
onbedoeld het kader scheppen voor een rationelere beleidsvorm later in
het Informatietijdperk.

Op een bepaald moment, wanneer voldoende bekwame personen zijn
vertrokken en in het buitenland genoeg grote fortuinen hebben opgebouwd,
zal het voor de Amerikaanse autoriteiten aantrekkelijk worden om burgers
of houders van een verblijfsvergunning toe te staan zich \emph{vrij te
kopen} van toekomstige belastingverplichtingen, door een exitbelasting
te betalen, maar zonder het land te verlaten. Met andere woorden: de
exit tax zou kunnen uitgroeien tot het model van een eenmalige
afkoopsom. De overheid die een dergelijke belasting oplegt, zou veel
meer voordeel halen door de vertrekkenden toe te staan het verblijf te
hervatten onder voorwaarden van een privaat verdrag, vergelijkbaar met
de verdragen die momenteel in Zwitserland en elders beschikbaar zijn.

Dergelijke stappen van de Verenigde Staten of andere overheden zouden
rationele, inkomensoptimaliserende maatregelen zijn. Uiteindelijk zal
concurrentie op het gebied van beschermingsdiensten ertoe leiden dat
belastingtarieven dalen en de belastingvoorwaarden worden aangepast aan
meer beschaafde normen. In plaats van afhankelijk te zijn van wetgevers
om aanvaardbare belastingregimes uit te vaardigen, zullen toekomstige
Soevereine Individuen in staat zijn om via private verdragen
aanvaardbare, op maat gemaakte beleidsarrangementen te onderhandelen.

\section{Het beledigen van de ware
gelovigen}\label{het-beledigen-van-de-ware-gelovigen}

Natuurlijk beweren we geenszins dat veel hiervan populair zal zijn. De
denationalisering van het individu en de commercialisering van
soevereiniteit die dit impliceert, zullen de overgebleven ware gelovigen
in de clichés van de twintigste-eeuwse politiek beledigen. Zoals de
inmiddels overleden Christopher Lasch zien zij de verschrompeling van de
politiek als een bedreiging voor het welzijn van de meerderheid van de
bevolking. In hun ogen zou een heropleving van de politiek uit het
Industriële Tijdperk, met haar inzet voor inkomensherverdeling, een
oplossing kunnen bieden voor de onrust die velen voelen door de
competitieve druk die de informatietechnologie veroorzaakt.

E. J. Dionne Jr., politiek verslaggever van de \emph{Washington Post},
grijpt net als Lasch nostalgisch terug naar de politiek. Hij verwoordt
ook een sociaal-democratisch egalitair streven dat in de komende
decennia ongetwijfeld luider gehoord zal worden naarmate de nieuwe
megapolitieke realiteiten van het Informatietijdperk de overgebleven
instituties van de moderne wereld verder ondermijnen. Dionne beschouwt
de materiële verbeteringen in de levensstandaard, die in de twintigste
eeuw binnen rijke jurisdicties breed werden gedeeld, voornamelijk als
het resultaat van democratische politiek en niet van technologische of
economische ontwikkeling. Volgens hem vereist hoop voor de toekomst dat
de politiek haar invloed uitbreidt over de technologieën van het
Informatietijdperk:

\begin{quote}
``De overheersende noodzaak in de Verenigde Staten en in de hele
democratische wereld is een nieuwe betrokkenheid bij democratische
hervorming, de politieke motor die het industriële tijdperk zo succesvol
maakte. De technologieën van het informatietijdperk zullen op zichzelf
geen succesvolle samenleving vormen, net zomin als het industrialisme
uit zichzelf de wereld beter maakte. \ldots{} Zelfs de meest
buitengewone technologische doorbraken en de meest ingenieuze
toepassingen van het internet zullen ons niet redden van
maatschappelijke ontwrichting, misdaad of onrechtvaardigheid. Alleen de
politiek, de kunst van hoe wij onszelf organiseren, kan überhaupt aan
zulke taken beginnen.''
\end{quote}

Dionne, en anderen als hij, begrijpen niet dat de omstandigheden die het
twintigste-eeuwse leven bijzonder gevoelig maakten voor systematische
dwang niet door enig menselijk handelen zijn gekozen. De ``kunst van hoe
wij onszelf organiseren'' is een uitspraak die vóór de moderne periode
onbegrijpelijk zou zijn geweest. Samenlevingen zijn te complex om
terecht te worden beschouwd als het product van een bewuste daad van
zelforganisatie. De natiestaten van de moderne periode ontstonden
spontaan als toevallig bijproduct van industriële technologie die het
rendement op geweld verhoogde. Nu informatietechnologie het rendement op
geweld verlaagt, wordt politiek onherroepelijk achterhaald, hoe graag
mensen het ook in het volgende millennium zouden willen behouden.

\begin{quote}
Niet van vandaag of gisteren zijn zij, Door alle tijden leven zij voort;
en vanwaar zij kwamen, weet niemand. -- SOPHOCLES, Antigone
\end{quote}

\section{`Ze maken ze niet zoals
vroeger'}\label{ze-maken-ze-niet-zoals-vroeger}

Het vurige verlangen om ``wetten te maken,'' dat zo vanzelfsprekend
lijkt binnen het gezonde verstand van de twintigste-eeuwse politiek, is
allerminst universeel voor alle culturen. Het verdwijnen ervan in de
toekomst kan worden gezien als onderdeel van een cyclus die door de
eeuwen heen is opgekomen en verdwenen. Zo geloofden bijvoorbeeld de
vroege Grieken, onder anderen, dat wetten niet gemaakt konden worden. In
de woorden van de filosoof Ernst Cassirer geloofden de Grieken dat ``de
`ongeschreven wetten,' de wetten van gerechtigheid, geen begin in de
tijd hebben.'' Zoals andere vóórpolitieke volkeren waren zij van mening
dat niemand de natuurlijke ``geometrische'' wetten van gerechtigheid kon
verbeteren, die niet door enig menselijk gezag waren geschapen.

Zij geloofden niet in een ``wetgever.'' Zoals Cassirer het uitdrukte:
``Het is door rationeel denken dat wij de normen van moreel gedrag
moeten vinden, en het is de rede, en alleen de rede, die hen gezag kan
verlenen.'' In die zin zou elke poging om wetten aan de samenleving op
te leggen via wetgeving gelijkstaan aan een poging om de meetkunde bij
wet te veranderen.

\subsection{Wetgeving als
heiligschennis}\label{wetgeving-als-heiligschennis}

Om heel andere redenen heerste een vergelijkbare weerstand tegen
``wetten maken'' gedurende een groot deel van de Middeleeuwen. Zoals
John B. Morrall schrijft: ``{[}V{]}oor de Germanen was wetgeving iets
dat sinds mensenheugenis bestond.'' Het was ``een garantie van de
rechten'' van individuele leden van de stam.

\begin{quote}
Koningen en raden waren nog niet van plan nieuwe wetten te scheppen.
Zo'n intentie zou vanuit het oogpunt van die vroege middeleeuwse tijden
niet alleen overbodig, maar zelfs semi-heiligschennis zijn geweest, want
wet, net als koningschap, bezat een eigen heilige aura. In plaats
daarvan beschouwden koning en raadsleden zichzelf als enkel degene die
de ware betekenis van het reeds bestaande en volledige wetboek uitlegden
of verduidelijkten. De Germaanse gewoonte schonk het middeleeuwse denken
een idee dat het nooit kon vergeten, zelfs wanneer het in de praktijk
anders handelde. Dit idee was dat goede wetten herontdekt of opnieuw
ingevoerd werden, maar nooit opnieuw gemaakt.
\end{quote}

Na de buitensporigheden van de twintigste-eeuwse wetgeving is er iets
aandoenlijks aan die oude houding. Het verlangen om de dwangmacht van de
staat in te zetten voor privédoeleinden, met name inkomensherverdeling,
werd bijna een tweede natuur.

\subsection{Spijt}\label{spijt}

Het is dan ook geen wonder dat er droevige liedjes zijn over politiek in
haar laatste dagen. Ze zijn volledig voorspelbaar, en niet alleen omdat
de meeste denkers blind zijn voor het dwingende karakter van de
megapolitieke omstandigheden. Weinig politieke verslaggevers, zoals
Dionne, accepteren het schijnbare verval en de ondergang van de
politiek, omdat dit negatieve gevolgen zou kunnen hebben voor hun
carrière. Aan het einde van de Middeleeuwen riepen mensen op tot een
heropleving van de ridderlijkheid. Denk bijvoorbeeld aan \emph{Il Libro
del Cortegiano} oftewel \emph{Het boek van de Hoveling}, dat graaf
Baldassare Castiglione in 1514 schreef en in 1528 in Venetië door Aldus
werd uitgegeven.

Castiglione voelde een diep verlangen naar de heropleving van
ridderlijke deugden, maar zo'n sentiment kon in de zestiende eeuw het
verleden dat uitsterfde niet weer tot leven brengen. Evenmin zal dat in
de eenentwintigste eeuw gebeuren.

Zoals we in onze uitleg van de theorie van megapolitiek probeerden over
te brengen, vormen technologische drijfveren, en niet de publieke
opinie, de voornaamste bronnen van verandering. Als onze theorie van
megapolitiek klopt, komt de vervanging van het feodale systeem en de
ridderlijkheid, die steunden op persoonlijke eden en relaties, door het
Moderne Tijdperk, met zijn burgerschapsconcept en staatgerichte
politiek, niet voort uit ideeën. De cruciale drijfveer zijn de
verschuivingen in kosten en baten die nieuwe technologie met zich
meebrengt. Ridderlijkheid verdween niet omdat Castiglione of anderen er
niet in slaagden een onverschillige bevolking, die bovendien geen
invloed had op deze zaak, ervan te overtuigen dat eer en moraliteit in
de politiek overbodig waren. Integendeel, in \emph{de Hoveling}
bekritiseert Castiglione vorsten en het gedrag dat zijn tijdgenoot
Niccolo Machiavelli in zijn \emph{Il Principe}, oftewel \emph{De vorst},
juist toejuichte. En dan? Machiavelli bereikte uiteindelijk een veel
breder publiek met zijn boek, niet doordat zijn betoog in \emph{De
vorst} overtuigender was, maar omdat zijn adviezen beter aansloten bij
het grootschalige politieke landschap van het Moderne Tijdperk.

Zoals de invloedrijke twintigste-eeuwse filosoof Ernst Cassirer opmerkte
in zijn bespreking van `het morele probleem in Machiavelli',

\begin{quote}
Het boek beschrijft, met volledige onverschilligheid, de manieren en
middelen waarmee politieke macht verkregen en behouden dient te worden.
Over het juiste gebruik van deze macht zegt het geen woord. \ldots{}
Niemand had ooit getwijfeld aan het feit dat het politieke leven, zoals
het ervoor stond, vol misdaden, verraad en zware misdrijven zat. Maar
geen enkele denker vóór Machiavelli had ooit de kunst van deze misdaden
onderwezen. Deze daden werden wel verricht, maar zij werden niet
onderwezen. Het feit dat Machiavelli beloofde een leermeester te worden
in de kunst van sluwheid, verraad en wreedheid, was ongehoord.
\end{quote}

Kortom, \emph{De vorst} presenteerde een radicaal recept waarmee een
aspirant-heerser zijn carrière ten koste van anderen kon bevorderen.
Machiavelli keurde gedragingen goed die perfect aansloten op de aard van
de politiek in een tijdperk waarin macht centraal stond. De kunst van
het dubbelspel, een sluwe strategie van politici in het Moderne
Tijdperk, was in de context van de riddercultuur van de voorgaande
eeuwen echter schandalig en subversief.

Zoals we eerder zagen, draaiden de deugden van de ridderlijkheid er
vooral om dat iedereen zich strikt aan zijn eden hield. Dat vereiste men
in een samenleving waar bescherming werd gegarandeerd in ruil voor
persoonlijke diensten. De afspraken waarop het feodale systeem was
gebouwd, zouden niet vanzelf herleven als mensen alleen hun eigen
belangen nastreefden. Daarom moesten de feodale verplichtingen, de
ruggengraat van de ridderlijkheid, altijd steunen op een sterk gevoel
van eer. In dat opzicht was Machiavelli's advies, dat een vorst niet zou
moeten aarzelen om te liegen, bedriegen en stelen wanneer dat in zijn
voordeel werkte, ronduit subversief.

Toen de twintigste eeuw ten einde kwam, bleven Machiavelli's argumenten
in de belangstelling als middel om de moderne politiek en de
uiteenlopende misdaden en tirannieën van die tijd te doorgronden. In
tegenstelling daarmee raakte het werk van Castiglione vrijwel in de
vergetelheid. Over een jaar lezen waarschijnlijk slechts een handjevol
masterstudenten literatuur en enkele kenners van de
etiquettegeschiedenis \emph{Il Libro del Cortegiano} van kaft tot kaft.

Binnen enkele decennia raakt de megapolitiek van het Informatietijdperk,
zoals vervat in \emph{De vorst}, duidelijk achterhaald. Het Soevereine
individu zal een nieuw succesrecept moeten hanteren, eentje die, bij het
inzetten van middelen buiten de greep van de staat, sterk inzet op eer
en integriteit. We verwachten dat E. J. Dionne, Jr.~en de overige nog
levende sociaaldemocraten zo'n advies met weinig enthousiasme zullen
oppikken.

\subsection{Door klanten vastgesteld
beleid}\label{door-klanten-vastgesteld-beleid}

Dit zal vooral gelden vroeg in de overgang, wanneer de meeste
jurisdicties nog steeds belast zijn met de noodzaak om beleid te
formuleren waarvan de voorstanders instemming kunnen verkrijgen van een
meerderheid van de bevolking. Later, naarmate de democratie vervaagt en
de markt voor soevereiniteitsdiensten zich uitbreidt, zullen de
marktomstandigheden die ``beleid'' beperken breder worden begrepen.

Wat we nu beschouwen als ``politiek'' leiderschap, dat altijd wordt
opgevat in termen van een natiestaat, zal steeds meer ondernemend van
aard worden in plaats van politiek. Onder deze omstandigheden zal de
haalbare keuzevrijheid bij het samenstellen van een ``beleids-''regime
voor een rechtsgebied beperkt zijn, vergelijkbaar met de opties voor
ondernemers bij het ontwerpen van een eersteklas resort of soortgelijk
product of dienst, die worden bepaald door wat mensen bereid zijn te
betalen. Een resorthotel zou bijvoorbeeld zelden proberen te opereren op
basis van voorwaarden die vereisen dat gasten zwaar werk verrichten om
de faciliteiten te repareren en uit te breiden. Zelfs een resorthotel
dat eigendom is van of wordt beheerd door de werknemers, zoals de
typische moderne democratie, zou tevergeefs proberen klanten te dwingen
aan dergelijke eisen te voldoen, vooral na het beschikbaar stellen van
betere voorzieningen. Als de klanten liever golfen dan zwaar werk in de
hete zon verrichten, biedt de markt op dat punt weinig ruimte om
willekeurige alternatieven op te leggen. Onder dergelijke omstandigheden
zullen momenteel ``politieke'' kwesties vervagen tot ondernemende
afwegingen, terwijl jurisdicties proberen te ontdekken welke
beleidspakketten klanten zullen aantrekken.

\subsection{Het verval van de
politiek}\label{het-verval-van-de-politiek}

Als dit eenmaal wordt begrepen, zal er een fundamentele verandering in
houding optreden. Bevolkingen in jurisdicties die afnemen in omvang
zullen niet langer verwachten te kiezen uit hetzelfde scala aan
wensvervullende beleidsopties die het politieke debat in de twintigste
eeuw beheersten. Met een inkomensverdeling die schever is dan in het
Industriële Tijdperk, zullen jurisdicties de neiging hebben te voorzien
in de behoeften van de klanten met de meest waardevolle zaken, en die de
grootste keuze hebben om deze onder te brengen.

Onder dergelijke omstandigheden zal het veel minder van belang zijn dan
we gewend zijn om te veronderstellen of beleid dat commercieel optimaal
is voor een jurisdictie, aantrekkelijk zou zijn voor de ``gemiddelde
kiezer'' in een focusgroep. Kortom, de commercialisering van
soevereiniteit zal de controle van regeringen door hun klanten
vergemakkelijken. Dit zal ertoe leiden dat de meningen van niet-klanten
irrelevant, of minder relevant, worden, net zoals de meningen van Big
Mac-eters over foie gras irrelevant zijn voor het succes van Franse
drie-sterrenrestaurants, zoals L'Arpège in Parijs.

\section{`Het verraad van de
democratie'}\label{het-verraad-van-de-democratie}

Net als de overleden Christopher Lasch klagen tegenstanders niet alleen
dat informatietechnologie banen vernietigt, maar stellen zij ook dat
deze technologie de democratie ondermijnt doordat mensen hun middelen
buiten het bereik van politieke dwang kunnen houden. Daarom ervaren
reactionairen in het nieuwe millennium de via informatietechnologie
mogelijk gemaakte financiële privacy als een ernstige bedreiging. Ze
zullen terugdeinzen bij de gedachte dat inkomsten- en
vermogensbelastingen daadwerkelijk op `vrijwillige naleving' moeten
berusten. Ook steunen zij nieuwe, zelfs ingrijpende methoden om van
iedereen die welvarend lijkt middelen af te romen, zoals het invoeren
van een `presumptieve belasting' en het letterlijk gijzelen van
vermogende personen.

\subsection{Gemeenschappelijk
eigendom}\label{gemeenschappelijk-eigendom}

Tekenen van wat komen gaat liggen al aan de oppervlakte terwijl we dit
boek schrijven. Vroeg bewijs dat de capaciteit van regeringen om
internationale markten te beheersen afneemt, stuit op weerstand bij
degenen die geloven dat individuen van rechtswege activa van natiestaten
zijn. Ze willen hun macht handhaven om de burgers van een land als
middelen, niet als klanten, te behandelen. De reactionairen geloven dat
alle inkomens als gemeenschapsinkomsten moeten worden beschouwd, wat
betekent dat ze ter beschikking van de staat moeten staan.

We hebben eerder de argumenten besproken die Lasch naar voren bracht in
\emph{Revolt of the Elites} en \emph{The Betrayal of Democracy}. Maar
zijn werk is niet het enige pamflet ter ondersteuning van de natiestaat.
Politiek theoreticus Michael Sandel van de Harvard-universiteit stelt in
\emph{Democracy in Discontent} dat ``Democratie vandaag de dag niet
mogelijk is zonder een politiek die wereldwijde economische krachten kan
beheersen, omdat het zonder een dergelijke controle niet uitmaakt op wie
mensen stemmen, de bedrijven zullen heersen.'' Met andere woorden, de
staat moet zijn parasitaire macht over individuen behouden om ervoor te
zorgen dat politieke uitkomsten kunnen afwijken van marktkrachten.
Anders zouden collectieve beslissingen om niet-economische uitkomsten af
te dwingen zinloos zijn.

Naar onze mening is Sandels klacht, net als die van Lasch, slechts half
juist. We erkennen dat democratie veel van haar belang zal verliezen als
overheden niet in staat zijn individuen te dwingen te handelen zoals
politici dat eisen. Dat is duidelijk. Inderdaad, democratie zoals het
bekend was in de negentiende en twintigste eeuw is voorbestemd om te
verdwijnen. Maar Sandel mist het werkelijke belang van de overwinning
van markten op dwang. Zijn verwijzing naar ``bedrijfsoverheersing'' als
gevaar bij de ineenstorting van de natiestaat is opvallend achterhaald.

Bedrijven zullen nauwelijks in staat zijn de markten van de nieuwe
wereldeconomie te beheersen. Zoals we hebben gesuggereerd, is het
allesbehalve zeker dat bedrijven zelfs in hun bekende moderne vorm
zullen blijven bestaan. Integendeel, bedrijven zullen vrijwel zeker
worden getransformeerd in de megapolitieke revolutie die gepaard gaat
met de introductie van het Informatietijdperk. Zoals eerder besproken,
zal microprocessing de ``informatiekosten'' veranderen die helpen
bepalen wat het ``nexus van contracten'' is dat bedrijven definieert.
Zoals economen Michael C. Jensen en William H. Meckling stellen, zijn
bedrijven slechts een juridische vorm die ``een nexus biedt voor een
reeks contractuele relaties tussen individuen.''

Of een bedrijf zelfs kan overleven, laat staan ``heersen'' als ``een
domein van bureaucratische sturing dat beschermd is tegen
marktkrachten,'' zal volgens de woorden van economen Louis Putterman en
Randall S. Kroszner waarschijnlijk worden bepaald door ``de volledigheid
van de marktkrachten en het vermogen van marktkrachten om door te
dringen in relaties binnen het bedrijf.''

Zoals we eerder betoogden, is het twijfelachtig of bedrijven het zullen
overleven naarmate de marktkrachten steeds meer doordringen in wat tot
nu toe ``intra-bedrijfsrelaties'' waren. Als gevolg daarvan zullen
bedrijven de neiging hebben te verdwijnen naarmate informatietechnologie
het lonender maakt om te vertrouwen op het prijsmechanisme en de
veilingmarkt om taken uit te voeren in plaats van ze te internaliseren
binnen een formele organisatie. Naarmate informatietechnologie het
productieproces steeds meer automatiseert, wordt een deel van de
bestaansreden van het bedrijf weggenomen: de noodzaak om managers in te
zetten en te motiveren om individuele werknemers te controleren.

\subsection{``Waarom bestaan
bedrijven?''}\label{waarom-bestaan-bedrijven}

Onthoud dat de vraag ``Waarom bestaan bedrijven?'' niet zo triviaal is
als het op eerste zicht lijkt. De micro-economie gaat er doorgaans van
uit dat het prijsmechanisme het meest effectieve middel is om middelen
te coördineren voor hun meest gewaardeerde toepassingen. Zoals Putterman
en Kroszner opmerken, impliceert dit dat organisaties zoals bedrijven
geen inherente ``economische bestaansreden'' hebben. In die zin zijn
bedrijven vooral artefacten van informatie- en transactiekosten, die
door informatietechnologieën drastisch zullen afnemen.

Daarom zal het Informatietijdperk neigen het tijdperk van onafhankelijke
opdrachtnemers zonder ``banen'' bij langlevende ``bedrijven'' te worden.
Naarmate technologie de transactiekosten verlaagt, zal het proces dat
individuen in staat stelt te ontsnappen aan dominantie door politici,
ook ``heerschappij door bedrijven'' verhinderen. Bedrijven zullen
concurreren met ``virtuele bedrijven'' van over de hele wereld in een
mate die iedereen, behalve enkelen, zal verontrusten en bedreigen. De
meeste corporaties zullen als instituties het geluk hebben te overleven
in een intensievere concurrentie naarmate markten vollediger worden.

Het te verwachten gevolg is niet dat individuen aan de genade van
bedrijven worden overgeleverd. Integendeel, bedrijven op zich zullen
niet meer macht hebben om markten te manipuleren dan politici. Het is
eerder zo dat individuen eindelijk vrij zullen zijn om hun eigen lot te
bepalen in een werkelijk vrije markt, geregeerd door noch grote
overheden, noch hiërarchieën van bedrijven.

Deze afname van transactiekosten zal ook de recente populaire
opvattingen over ``stakeholderkapitalisme'' weerleggen. Dergelijke
opvattingen, geliefd bij Tony Blair van de Britse Labour Party en
sommige leden van Bill Clintons entourage, zijn gebaseerd op het
vermogen van de staat om het bedrijf te manipuleren. Nu het socialisme
is ingestort, dromen interventionisten er nu van om de doelen van het
socialisme te bereiken via markt-efficiëntere middelen door het bedrijf
zwaar te reguleren. Deze nieuwe herverdelende theorie stelt dat het
management, de aandeelhouders, de werknemers en de ``gemeenschap''
allemaal ``stakeholders'' van bedrijven zijn. Het argument is dat zij
allemaal profiteren van langlevende bedrijven en zelfs afhankelijk zijn
van deze voordelen. Daarom zou regelgeving de belangen moeten beschermen
die managers, werknemers en lokale belastingautoriteiten hebben in het
voortbestaan van hun historische relaties met de bedrijven.

``Stakeholderkapitalisme'' is een doctrine die uiteindelijk niet alleen
de macht van de staat veronderstelt om de besluitvorming van bedrijven
te manipuleren, maar zelfs nog fundamenteler veronderstelt dat bedrijven
bestaan als langdurige organisaties die onafhankelijk van prijssignalen
op de veilingmarkt kunnen functioneren.

We vermoeden dat de verdieping van markten niet alleen het
belastingvermogen van de natiestaat zal verminderen, maar ook het
vermogen van politici zal uithollen om hun wil arbitrair op te leggen
aan de eigenaars van middelen via regulering. In een wereld waarin
jurisdictievoordelen aan testen van de markt onderworpen zullen worden
en veel lokale markten openstaan voor concurrentie van overal, is het
nauwelijks te verwachten dat lokale ``gemeenschappen'' veel effectieve
manieren zullen hebben om bevoordeelde bedrijven te isoleren van
wereldwijde concurrentiedruk. Daarom zullen ze weinig middelen hebben om
te verzekeren dat bedrijven die hogere kosten dragen (bijvoorbeeld om
onnodige werknemers en managers te behouden, en onnodige faciliteiten
open te houden om lokale politieke druk te accommoderen) die kosten
kunnen compenseren en in bedrijf kunnen blijven. In het Industriële
Tijdperk konden politici markten sluiten en de toegang beperken tot
enkele bevoordeelde bedrijven om werkgelegenheids- en andere
doelstellingen te bereiken. In de toekomst, wanneer informatie overal
ter wereld vrij verhandelbaar is, zal de macht van regeringen om lokale
bedrijven te beschermen tegen wereldwijde concurrentiedruk minimaal
zijn.

Evenmin is het waarschijnlijk dat oproepen tot een ``nieuw sociaal
contract'', gericht op een zogenaamd onafhankelijke of
vrijwilligerssector, om de tijd van anders werklozen of
gemarginaliseerde werknemers ``in de gemeenschap'' op te nemen, haalbaar
zullen blijken. Jeremy Rifkin stelt zich ``een nieuw partnerschap tussen
de overheid en de derde sector voor om de sociale economie te herbouwen.
\ldots{} Het voeden van de armen, het bieden van basisgezondheidszorg,
het onderwijzen van de jeugd van het land, het bouwen van betaalbare
huisvesting en het behouden van het milieu. \ldots{}''

\subsection{De ondergang van openbare
goederen}\label{de-ondergang-van-openbare-goederen}

Voorstanders van dwang zullen ongetwijfeld stellen dat een afnemend
staatsgezag leidt tot een onvermogen om openbare goederen te leveren of
ervan te genieten. Maar dat lijkt om zowel competitieve als andere
redenen onwaarschijnlijk. Ten eerste verliezen rechtsgebieden, nu
technologie plaatsgebonden voordelen grotendeels heeft weggenomen, hun
klanten snel als ze essentiële openbare goederen, zoals het handhaven
van orde, niet kunnen aanbieden. In uiterste gevallen, zoals te zien was
in Somalië, Liberia, Rwanda en het voormalige Joegoslavië, zullen
waarschijnlijk hordes blutte vluchtelingen de landsgrenzen overschrijden
op zoek naar een betrouwbaardere waarborging van orde en veiligheid.
Deze extreme vormen van desertie, of stemmen met de voeten, wijken in
urgentie nauwelijks af van gewoon shoppen voor rechtsgebieden. Hoe dan
ook, ondernemingen zullen lokale rechtsgebieden ertoe dwingen om aan de
behoeften van hun klanten te voldoen.

\subsection{``Competitieve territoriale
clubs''}\label{competitieve-territoriale-clubs}

``Dit is meer dan louter een theorie'', verkondigde Econoom Charles
Tiebout voor het eerst in 1956. Fred Foldvary documenteerde in
\emph{Public Goods and Private Communities: The Market Provision of
Social Services} dat er geen fundamentele reden bestaat om sociale
diensten en andere openbare goederen uitsluitend via politieke kanalen
te leveren. De voorbeelden die Foldvary aandraagt, bevestigen tevens de
controversiële stelling van de Nobelprijswinnende econoom Ronald Coase,
namelijk dat overheidsinterventie niet nodig is om externaliteiten,
zoals vervuilingsproblemen, op te lossen. Ondernemers leveren
collectieve goederen via marktmechanismen en velen hebben dat al in
praktijk gebracht binnen lokale gemeenschappen. Foldvary's casestudies
tonen aan dat het privatiseren van gemeenschappen kan leiden tot
innovatieve manieren om openbare goederen en diensten te leveren en te
financieren.

\subsection{De weg naar voorspoed}\label{de-weg-naar-voorspoed}

Microtechnologie zal zelf nieuwe manieren faciliteren om de levering van
goederen, die tot nu toe als collectieve goederen werden behandeld, te
financieren en te reguleren. Achteraf zullen sommige van deze goederen
privégoederen in vermomming blijken te zijn. Snelwegen vormen een
belangrijk voorbeeld. Zolang congestie een klein probleem bleef, konden
wegen en snelwegen worden behandeld alsof het collectieve goederen
waren, zij het onderworpen aan de kritiek van Adam Smith dat zij
onevenredig voordeel opleveren voor nabijgelegen bewoners ten koste van
degenen in afgelegen gebieden, die gedwongen worden ervoor te betalen
terwijl ze weinig van de voordelen genieten.

In het Informatietijdperk zal het technologisch haalbaar zijn om
tolheffingen, inclusief congestiekosten, op te leggen die toegang tot
snelwegen, landingsbanen en andere infrastructuur nauwkeurig prijzen,
zonder het verkeer te onderbreken. Zo kan de levering van
transportinfrastructuur discreet worden geprivatiseerd en rechtstreeks
worden gefinancierd door degenen die van de dienst gebruikmaken. Econoom
Paul Krugman schat dat marktprijsstelling van de Amerikaanse
transportinfrastructuur jaarlijks tussen \$60 miljard en \$100 miljard
aan het BBP in de Verenigde Staten zou toevoegen, terwijl het efficiënt
gebruik van middelen wordt verbeterd en vervuiling wordt verminderd.

Bovendien mag niet worden vergeten dat het kostbaarste deel van wat
moderne natiestaten doen, het herverdelen van inkomen, helemaal geen
collectief goed is, maar de levering van privégoederen op publieke
kosten. ``Publieke kosten'' is hier een eufemisme voor ``ten koste van
degenen die belastingen betalen.''

Wat betreft een echt collectief goed, zoals de levering van een
militaire macht die in staat is een aanval door een grootmacht af te
schrikken: zo'n kracht is traditioneel duur geweest. Zoals eerder
besproken, zou een regering die niet onbeperkt het inkomen en eigendom
van haar burgers kan confisqueren, niet in staat zijn deel te nemen aan
nog een groot machtsconflict zoals de Tweede Wereldoorlog.

Toch vormt deze fiscale beperking een kleinere dreiging dan de
reactionairen zouden doen geloven, om de eenvoudige reden dat er geen
conflicten zoals de Tweede Wereldoorlog meer zullen plaatsvinden. De
technologie die individuen bevrijdt, zal daar zorg voor dragen.

\subsection{Los van de politiek}\label{los-van-de-politiek}

In plaats van de kwaliteit en aard van zulke diensten over te laten aan
de grillen van de politiek, kun je `regeringen' op een ondernemende
wijze besturen en omvormen tot wat Foldvary omschrijft als
`concurrerende territoriale clubs.' Wij vermoeden dat de manier waarop
deze clubs hun besluitvorming inrichten uiteindelijk minder belangrijk
zal blijken dan hun succes in de prestatietesten door de markt.
Tegenwoordig maakt het voor de meeste consumenten nauwelijks uit of een
bedrijf dat een product of dienst levert een eenmanszaak is, een
besloten vennootschap of een onderneming die door externe bestuurders,
voorgedragen door pensioenfondsen, wordt geleid. Wij twijfelen er
bovendien niet aan dat de rationele afnemer van soevereiniteitsdiensten
in dit Informatietijdperk zich niet druk zal maken over de vraag of
Singapore een massademocratie vormt of het particuliere bezit is van Lee
Kwan Yew.

\bookmarksetup{startatroot}

\chapter{Moraal en misdaad in de `natuurlijke economie' van het
Informatietijdperk}\label{moraal-en-misdaad-in-de-natuurlijke-economie-van-het-informatietijdperk}

\begin{quote}
Corruptie \ldots{} is veel wijdverspreider en universeler dan eerder
werd gedacht. Bewijs ervan is overal, in ontwikkelingslanden en, met
toenemende frequentie, in industrielanden. \ldots{} Prominente politieke
figuren, waaronder presidenten en ministers, zijn beschuldigd van
corruptie. \ldots{} In zekere zin vertegenwoordigt dit een privatisering
van de staat, waarbij de macht niet wordt overgedragen aan de markt,
zoals privatisering normaal impliceert, maar aan
regeringsfunctionarissen en bureaucraten. -- VITO TANZI
\end{quote}

Wij geloven dat, naarmate de moderne natiestaat uiteenvalt, barbaren van
de nieuwe tijd steeds meer reële macht achter de schermen zullen
uitoefenen. Groepen zoals de Russische maffia's die de resten van de
voormalige Sovjet-Unie plunderen, andere etnische misdaadbendes,
nomenklatoera's, drugsbaronnen en afvallige geheime diensten zullen
steeds meer hun eigen wetten stellen. Dat doen ze nu al.~Veel meer dan
algemeen wordt beseft, hebben de moderne barbaren de vormen van de
natiestaat al geïnfiltreerd zonder haar uiterlijk wezenlijk te
veranderen. Zij zijn microparasieten die zich voeden met een stervend
systeem. Even gewelddadig en gewetenloos als een staat in oorlog
gebruiken deze groepen de technieken van de staat op kleinere schaal.
Hun groeiende invloed en macht maken deel uit van de inkrimping van de
politiek. Microprocessing verkleint de omvang die groepen moeten hebben
om effectief te zijn in het gebruik en de controle van geweld. Naarmate
deze technologische revolutie zich ontvouwt, zal roofzuchtig geweld
steeds vaker buiten centrale controle worden georganiseerd. Pogingen om
geweld te beteugelen zullen eveneens decentraliseren, op manieren die
meer afhankelijk zijn van efficiëntie dan van de omvang van de macht.

De opkomst van heimelijke criminele activiteiten en corruptie binnen
natiestaten zal een belangrijk subverhaal vormen in de wereldwijde
verandering. Wat je zult zien, zou kunnen lijken op een duistere versie
van een slechte film, \emph{Invasion of the Body Snatchers}. Nog vóór de
meeste natiestaten zichtbaar instorten, zullen ze worden gedomineerd
door barbaren van de nieuwe tijd. Vaak zullen ze, zoals in de beroemde
B-film uit de jaren vijftig, barbaren in vermomming zijn. De ``Pod
People'' van de toekomst zullen echter geen buitenaardse wezens zijn,
maar criminelen van diverse pluimage die officiële posities bekleden
terwijl ze minstens gedeeltelijke loyaliteit verschuldigd zijn aan
machten buiten de constitutionele orde.

Het einde van een tijdperk is gewoonlijk een periode van intense
corruptie. Terwijl de banden van het oude systeem oplossen, lost ook de
sociale ethos op, wat een omgeving creëert waarin mensen op hoge
posities publieke doeleinden kunnen combineren met particuliere
criminele activiteiten.

Helaas zul je niet kunnen vertrouwen op normale informatiekanalen voor
een nauwkeurig en tijdig begrip van het verval van de natiestaat.
``Aanhoudend zelfbedrog'' van het soort dat de val van het Romeinse Rijk
maskeerde, is waarschijnlijk een typisch kenmerk van het uiteenvallen
van grote politieke entiteiten. Het maskeert nu de ineenstorting van de
natiestaat. Om verschillende redenen kan niet altijd worden vertrouwd
dat de nieuwsmedia de waarheid vertellen. Velen zijn conservatief in de
zin dat ze de partij van het verleden vertegenwoordigen. Sommigen zijn
verblind door achterhaalde ideologische trouw aan het socialisme en de
natiestaat. Anderen zullen bang zijn om de groeiende corruptie in een
aftakelend systeem te onthullen uit angst voor hun veiligheid, hun baan
of andere vormen van vergelding. En natuurlijk is er geen reden om aan
te nemen dat verslaggevers en redacteuren minder vatbaar zouden zijn
voor corruptie dan bouwinspecteurs of Italiaanse wegwerkers. In grotere
mate dan men zou verwachten, zullen belangrijke informatie-organen die
erop lijken te staan alles te onthullen, uiteindelijk minder betrouwbare
bronnen blijken dan algemeen wordt aangenomen. Velen zullen vooral
proberen een falend systeem overeind te houden, ook als dat betekent dat
ze je niet eerlijk informeren. Ze zullen weinig zien en nog minder
uitleggen.

\section{Voorbij de realiteit}\label{voorbij-de-realiteit}

Als technologieën voor kunstmatige realiteit en computerspellen blijven
verbeteren, zul je zelfs een avondjournaal kunnen bestellen dat het
nieuws simuleert dat jij graag zou willen horen. Wil je een reportage
zien waarin jijzelf de winnaar bent van de tienkamp op de Olympische
Spelen? Geen probleem. Het kan morgen de hoofdheadline zijn. Je zult elk
verhaal dat je maar wilt, waar of onwaar, op je televisie of computer
zien ontvouwen, met een geloofwaardigheid die groter is dan alles wat
NBC of de BBC momenteel kunnen produceren.

We bewegen ons snel naar een wereld waarin informatie zo volledig zal
worden bevrijd van de grenzen van de werkelijkheid als menselijke
vindingrijkheid mogelijk kan maken. Dit zal uiteraard enorme gevolgen
hebben voor de kwaliteit en het karakter van de informatie die je
ontvangt. In een wereld van kunstmatige realiteit en onmiddellijke
overdracht van alles overal, zullen vermogen om informatie scherp te
beoordelen en waarheid van leugen te onderscheiden belangrijker zijn dan
ooit.

Dit zal echter een minder grote verandering zijn, ten opzichte van onze
huidige omstandigheden, dan velen denken. De grenzen tussen waarheid en
onwaarheid zijn nu al vaak vervaagd, door oorzaken die door technologie
zijn versterkt. We zeggen dit terwijl we erkennen dat veel gevolgen van
de Informatierevolutie bevrijdend zijn geweest.

Technologie is al begonnen geografische nabijheid en politieke
overheersing te overstijgen. Regeringen kunnen nog barrières opwerpen om
handel in goederen te belemmeren, maar ze kunnen veel minder doen om de
verspreiding van informatie te stoppen. Vrijwel elke eter in een
restaurant in Hongkong staat via een mobiele telefoon in verbinding met
de hele wereld. De hardliner-coup in Moskou in augustus 1991 kon
Jeltsins communicatie niet stilleggen, omdat hij over mobiele telefoons
beschikte.

\subsection{Meer informatie, minder
begrip}\label{meer-informatie-minder-begrip}

Nu de barrières voor informatieoverdracht zijn weggevallen, is er meer
informatie beschikbaar gekomen, wat goed is. Maar er is ook meer
verwarring ontstaan over wat die informatie betekent. De moderne
technologie die informatie bevrijdt van politieke controle en
beperkingen van tijd en plaats, verhoogt tegelijkertijd de waarde van
ouderwets beoordelingsvermogen. Het inzicht waarmee je het echte van het
onechte kunt scheiden in een wereld vol informatie, wordt steeds
waardevoller, en dat om minstens drie redenen:

\begin{enumerate}
\def\labelenumi{\arabic{enumi}.}
\tightlist
\item
  De overvloed aan informatie legt een premie op beknoptheid.
  Beknoptheid leidt tot afkorting. Afkorting laat weg wat onbekend is.
  Wanneer men veel feiten moet verwerken en talloze telefoontjes moet
  beantwoorden, is de natuurlijke neiging om elk
  informatieverwerkingsmoment zo kort mogelijk te maken. Helaas biedt
  verkorte informatie vaak een zwakke basis voor een goed begrip. De
  diepere en rijkere lagen van de geschiedenis zijn juist de delen die
  worden weggeknipt in de ``soundbites'' van vijfentwintig seconden en
  verkeerd worden weergegeven op CNN. Het is veel gemakkelijker een
  boodschap over te brengen die een variatie is op een reeds bekend
  thema, dan een geheel nieuw denkkader uit te leggen. Je kunt een
  honkbal- of cricketscore veel eenvoudiger rapporteren dan uitleggen
  hoe het spel wordt gespeeld en wat het betekent.
\item
  Snel veranderende technologie ondermijnt de megapolitieke basis van
  sociale en economische organisatie. Daardoor verouderen brede
  paradigma's, de onuitgesproken theorieën over hoe de wereld werkt,
  sneller dan vroeger. Dit vergroot het belang van een breed overzicht
  en vermindert de waarde van losse ``feiten'' van het soort dat voor
  vrijwel iedereen met een informatiesysteem beschikbaar is.
\item
  De toenemende tribalisering en marginalisering van het leven hebben
  een verstikkend effect gehad op het discours en zelfs op het denken.
  Veel mensen zijn daardoor gewend geraakt om conclusies te vermijden
  die duidelijk uit hun eigen feiten voortvloeien. Een recente
  psychologische studie, vermomd als opiniepeiling, toonde aan dat leden
  van beroepsgroepen bijna zonder uitzondering weigerden een conclusie
  te aanvaarden die inkomensverlies voor hen impliceerde, hoe sluitend
  de logica erachter ook was. Door de toegenomen specialisatie is de
  meeste interpretatieve informatie over gespecialiseerde beroepen
  gericht op de belangen van die groepen zelf. Zij hebben weinig
  belangstelling voor opvattingen die onbeleefd, onrendabel of politiek
  incorrect zouden kunnen zijn. Er is geen beter voorbeeld van deze
  tendens dan de constante stroom berichten die optimisme over de
  aandelenmarkt in stand houdt. Het merendeel van die informatie komt
  van brokers, die zelden zullen zeggen dat aandelen overgewaardeerd
  zijn, aangezien hun inkomsten afhangen van transacties en van het feit
  dat de meeste klanten willen kopen. Onafhankelijke, afwijkende stemmen
  worden zelden gehoord.
\end{enumerate}

Om deze en nog andere redenen is het Informatie­tijdperk nog niet het
Tijdperk van Begrip geworden. Integendeel, de strengheid van het
publieke discours is sterk afgenomen. De wereld zou nu meer kunnen weten
dan ooit tevoren, maar er is bijna geen publieke stem meer die de
betekenis van gebeurtenissen beoordeelt en zegt wat waar is. Dit
verklaart onze fascinatie voor de matige belangstelling, vooral in de
Amerikaanse media, voor aanwijzingen van sensationele corruptie op hoge
niveaus van de Amerikaanse regering.

Een centraal thema dat wij in dit boek hebben onderzocht, is hoe
veranderende technologie en andere ``megapolitieke'' factoren de
``natuurlijke economie'' veranderen. De ``natuurlijke economie'' is de
Darwinistische ``natuurtoestand,'' waarin uitkomsten, soms oneerlijk,
worden bepaald door fysieke macht. In deze ``natuurlijke economie'' is
een belangrijk gedragspatroon wat biologen ``interferentieconcurrentie''
noemen.

\subsection{Interferentieconcurrentie}\label{interferentieconcurrentie}

``Interference competitors,'' zoals Jack Hirshleifer het formuleerde,
``verwerven en behouden controle over middelen door hun rivalen direct
af te weren of te belemmeren.'' Hoezeer we ook zouden willen dat
menselijk gedrag altijd onderworpen was aan de wet en ``andere sociaal
afgedwongen spelregels'' (``politieke economie''), er is overvloedig
bewijs dat veel mensen ``zich aan de regels houden'' alleen wanneer het
hen uitkomt. Hirshleifer, een autoriteit op het gebied van conflicten,
stelde het als volgt: ``{[}D{]}e persistentie van misdaad, oorlog en
politiek leert ons dat actuele menselijke aangelegenheden nog steeds
grotendeels onderhevig blijven aan de onderliggende werking van de
natuurlijke economie.''

Met andere woorden, economische uitkomsten worden slechts deels bepaald
door het vreedzame en wettengehoorzamende gedrag van de in leerboeken
beschreven Homo economicus, die eigendomsrechten respecteert ``en niet
simpelweg zal nemen wat niet van hem is.'' Werkelijke uitkomsten worden
ook beïnvloed door conflict, inclusief openlijk geweld. Zoals
Hirshleifer opmerkt: ``Zelfs onder wet en overheid zal het rationele,
eigenbelang nastrevende individu een balans vinden tussen legale en
illegale middelen om middelen te verwerven, tussen productie en ruil aan
de ene kant en diefstal, fraude en afpersing aan de andere.''

\section{Overvallen in het
Informatietijdperk}\label{overvallen-in-het-informatietijdperk}

In een waardevol boek over geweld, misdaad en politiek, \emph{The
Political Economy of Conflict and Appropriation}, stellen Michelle R.
Garfinkel en Stergios Skaperdas dat individuen en groepen óf rijkdom
creëren door te produceren óf de door anderen opgebouwde welvaart in
beslag nemen. Zij citeren een verhaal over moderne
interferentieconcurrentie, oorspronkelijk gerapporteerd door \emph{The
Economist}: ``Een Amerikaanse zakenman, die onlangs in Moskou was
aangekomen om een kantoor te openen, werd bij zijn hotel opgewacht door
vijf mannen met gouden horloges, pistolen en een afdruk van de
nettowaarde van zijn bedrijf. Zij eisten 7\% van de toekomstige
inkomsten. Hij nam de eerstvolgende vlucht naar New York, waar
overvallers minder geraffineerd zijn.'' Dit verhaal over beroving in het
Informatietijdperk is meer te danken aan nieuwe technologie dan aan het
simpele feit dat criminelen in Rusland nu via Internet toegang hebben
tot financiële profielen en kredietrapporten van hun slachtoffers.

\subsection{Dalende beslissendheid van militaire
macht}\label{dalende-beslissendheid-van-militaire-macht}

In zowel positieve als negatieve zin verzwakt informatietechnologie,
doordat grootschalige militaire macht minder doorslaggevend is, de
mogelijkheid van de natiestaat om haar gezag in een instabiele wereld op
te leggen. Waar men ooit, zoals Voltaire zei, geloofde dat `God aan de
kant van de grotere bataljons' stond, lijkt de goddelijke steun voor
grootschalig geweld met de dag af te nemen. Integendeel, alle signalen
wijzen op dalende opbrengsten van geweld, wat erop duidt dat grote
entiteiten zoals de natiestaat hun enorme vaste kosten niet langer
kunnen rechtvaardigen.

Het meest voor de hand liggende bewijs voor de afnemende effectiviteit
van gecentraliseerde macht is de opkomst van terrorisme. De
grootschalige bomaanslagen in de Verenigde Staten midden in de jaren
negentig tonen aan dat zelfs de militaire supermacht van de wereld niet
immuun is voor aanvallen.

Een andere duidelijke uiting van de dalende winstgevendheid van geweld
blijkt uit de wereldwijde toename van gangsterisme en georganiseerde
misdaad, vergezeld door politieke vriendjespolitiek en corruptie. Deze
ontwikkelingen karakteriseren een overwegend amorele sfeer waarin de
staat weliswaar kan dwingen, maar weinig in staat is bescherming te
bieden. Nu haar geweldsmonopolie afbrokkelt, dringen nieuwe spelers zich
op, denk maar aan de pestkoppen die geprobeerd hebben particuliere
belastingen op te leggen aan de Amerikaanse zakenman in Moskou.

Kleine groeperingen, stammen, triaden, bendes, gangsters, mafia's,
milities en zelfs individuele strijders hebben hun militaire
effectiviteit versterkt. In de `natuurlijke economie' van het komende
millennium zullen zij waarschijnlijk meer reële macht uitoefenen dan in
de twintigste eeuw. Wapens die microchips inzetten, verschuiven de
machtsbalans in het voordeel van de verdediging, waardoor beslissende
agressie minder winstgevend en daarmee minder waarschijnlijk wordt.
Slimme wapens, zoals Stinger-raketten, neutraliseren een groot deel van
het voordeel dat grote, welvarende staten vroeger haalden met een dure
luchtmacht tegenover armere, kleinere groepen.

\subsection{Informatieoorlog in het
verschiet}\label{informatieoorlog-in-het-verschiet}

Wat aan de horizon opdoemt, is de veelbesproken maar weinig begrepen
mogelijkheid van ``Informatieoorlog.'' Dit wijst ook op dalende
rendementen van geweld. ``Logische bommen'' kunnen
luchtverkeersleidingssystemen, spoorwissels, stroomgeneratoren en
distributienetwerken, water- en rioolsystemen, telefoonschakelaars en
zelfs militaire communicatiesystemen uitschakelen of saboteren. Naarmate
samenlevingen meer afhankelijk worden van gecomputeriseerde besturing,
kunnen ``logische bommen'' bijna evenveel schade aanrichten als fysieke
explosies.

In tegenstelling tot conventionele bommen kunnen ``logische bommen'' op
afstand worden geactiveerd, niet alleen door vijandige overheden, maar
ook door groepen freelance programmeurs of getalenteerde individuele
hackers. Een Argentijnse tiener werd bijvoorbeeld in 1996 gearresteerd
voor herhaaldelijk inbreken op Pentagon-computers. Tot nu toe hebben
hackers systemen zelden op destructieve wijze aangetast, maar dat komt
niet doordat er echt effectieve manieren zijn om hen tegen te houden.

Wanneer het tijdperk van Informatieoorlog aanbreekt, zullen de
tegenstanders waarschijnlijk niet alleen overheden zijn. Een bedrijf als
Microsoft heeft bijvoorbeeld een grotere capaciteit voor
Informatieoorlogvoering dan 90 procent van de wereldwijde staten.

\subsection{Het tijdperk van het Soevereine
Individu}\label{het-tijdperk-van-het-soevereine-individu}

Dit is deels de reden waarom we dit boek \emph{Het Soevereine Individu}
hebben genoemd. Nu de schaal van oorlogvoering afneemt, zullen
verdediging en bescherming op kleinere schaal worden georganiseerd.
Daardoor zullen ze steeds vaker particuliere in plaats van publieke
goederen worden, geleverd door private aannemers die winst nastreven.
Dit is nu al zichtbaar in de privatisering van politiediensten in
Noord-Amerika. Een van de snelst groeiende beroepen in de Verenigde
Staten is dat van `beveiligingsbeambte'. Prognoses geven aan dat het
aantal particuliere beveiligers tussen 1990 en 2005 met 24 tot 40
procent zal stijgen.

De privatisering van politietaken is inmiddels een duidelijk herkenbare
trend. Toch, zoals de Anglo-Ierse goeroe Hamish McRae opmerkt, is dit
nauwelijks het gevolg van een bewuste beslissing van de overheid. Hij
schrijft het volgende in \emph{The World in 2020}:

\begin{quote}
Geen enkele regering heeft expliciet besloten om zich terug te trekken
uit bepaalde politietaken, en geen enkele heeft dat ook daadwerkelijk
gedaan; de private sector is er gewoon ingestapt. Gedeeltelijk als
gevolg van het falen van de politie, en gedeeltelijk door andere
maatschappelijke veranderingen, hebben particuliere
beveiligingsbedrijven geleidelijk een groot deel van de taak overgenomen
om gewone burgers in hun kantoren of winkelcentra te beschermen. Zoals
de ommuurde gemeenschappen van Los Angeles laten zien, keren mensen
zelfs enigszins terug naar het middeleeuwse concept van een stad, waar
burgers achter stadsmuren wonen die door wachters worden bewaakt, en
waar toegang alleen mogelijk is via gecontroleerde toegangspoorten.
\end{quote}

Wij denken dat dit slechts een voorproefje is van een veel verdergaande
privatisering van bijna alle functies die overheden in de twintigste
eeuw hebben uitgevoerd. Informatie­technologie heeft het vermogen
ondermijnd van gecentraliseerde autoriteiten om macht te projecteren en
fysieke veiligheid te bieden aan systemen die op grote schaal opereren.
Daardoor neemt de optimale omvang van vrijwel elke onderneming in de
`natuurlijke economie' af.

Om op deze technologische verandering te reageren, zal een enorme
investeringsbehoefte (lees kans) ontstaan om kwetsbare systemen opnieuw
te ontwerpen met gedistribueerde in plaats van geconcentreerde
capaciteiten. Als kwetsbaarheden op grote schaal niet worden verwijderd,
zullen de systemen die ze behouden vatbaar zijn voor catastrofale
mislukkingen. Vroeg of laat, bij toeval of opzettelijk, zullen diensten
en producten die nu nog worden geleverd door grote bureaucratische
instanties en corporaties uiteenvallen in sterk concurrerende markten,
beheerd via een gedistribueerd, gedecentraliseerd netwerk in plaats van
vanuit een `hoofdkantoor'. De onderneming met een hoofdkantoor dat
omsingeld kan worden door stakers of gesaboteerd kan worden door
terroristen, zal kwetsbaar blijven totdat ze uiteindelijk verandert in
een `virtuele onderneming' zonder vaste locatie, ``tegelijk op vele
plaatsen aanwezig'', zoals Kevin Kelly, hoofdredacteur van \emph{Wired
Magazine}, schrijft in \emph{Out of Control}. Kelly begrijpt dat
technologie de noodzaak heeft veranderd om productieprocessen onder
centrale controle te brengen. ``Gedurende het grootste deel van de
Industriële Revolutie werd serieus geld verdiend door processen onder
één dak te brengen. Groter was efficiënter.'' Nu niet meer.

Kelly voorziet dat een toekomstige auto, de \emph{Upstart Car},
ontworpen en geproduceerd zou kunnen worden door slechts een dozijn
mensen die samenwerken binnen een virtuele onderneming.

In de toekomst kan een te grote schaal niet alleen contraproductief maar
ook gevaarlijk blijken. Grotere ondernemingen zijn aantrekkelijkere
doelwitten. Zoals deelnemers aan de ondergrondse economie laten zien, is
een van de geheimen van belastingontwijking om onopgemerkt te blijven.
Dat zal veel eenvoudiger zijn voor kleinschalige `virtuele
ondernemingen' dan voor traditionele bedrijven die opereren vanuit een
wolkenkrabber met hun naam in neonletters. Zulke bedrijven blijven
kwetsbaar voor de aandacht van `mannen met gouden horloges, pistolen en
een uitdraai van de waarde van het bedrijf', de gangsters die hun eigen
vorm van belastingheffing opleggen in andere delen van de wereld, zoals
ze dat al doen in Rusland. Ondernemingen van elke omvang zullen
kwetsbaar zijn voor afpersing en heffingen door georganiseerde
misdaadgroepen.

\begin{quote}
Beschouw de definitie van een afperser als iemand die een dreiging
creëert en vervolgens geld vraagt om die te verminderen. Volgens die
maatstaf kwalificeert het leveren van bescherming door overheden vaak
als afpersing. -- CHARLES TILLY
\end{quote}

\subsection{De natuur haat monopolies}\label{de-natuur-haat-monopolies}

Terwijl het geweldsmonopolie waarover de `grotere bataljons' beschikken
verder afbrokkelt, mogen we als eerste gevolg verwachten dat de
georganiseerde misdaad zal floreren. Immers, georganiseerde misdaad
concurreert met natiestaten als het gaat om het inzetten van geweld voor
roofzuchtige doeleinden. Hoe onbeleefd dat ook klinkt, moeten we dit
niet vergeten. Zoals politicoloog Charles Tilly ons eraan herinnert,
gelden overheden eigenlijk als onze grootste voorbeelden van
georganiseerde misdaad, oftewel ``het toonbeeld van een
beschermingsmaffia, maar met het voordeel van legitimiteit''.

Als je verder niets van de wereld zou weten behalve dat een belangrijk
monopolie aan het instorten is, dan zou je vrij zeker kunnen zijn van je
simpele voorspelling dat de directe concurrenten het meeste zullen
profiteren. Het is dan ook geen toeval dat drugskartels, criminele
bendes, maffia's en triaden wereldwijd in opmars zijn.

\subsection{Sistema del potere}\label{sistema-del-potere}

Van Rusland tot Japan en de Verenigde Staten beïnvloedt georganiseerde
misdaad economieën veel meer dan economische leerboeken doen vermoeden.
Wat de Sicilianen `sistema del potere' noemen, oftewel het machtssysteem
van de georganiseerde misdaad, speelt een steeds belangrijkere rol in
het functioneren van economieën.

Europese politiefunctionarissen melden dat internationale
misdaadsyndicaten, waaronder Russische en Italiaanse maffia's, een
`dominante rol' hebben gespeeld bij het financieren van de genocidale
oorlogen die de Balkan de afgelopen jaren teisterden.

Ook drugssmokkelaars hebben een cruciale rol gespeeld in het financieren
van recente burgeroorlogen en opstanden in andere delen van de wereld.
Julio Fernandez, chef van de drugseenheid van de Spaanse nationale
politie in Catalonië, zegt: ``Van 1986 tot 1988 vervoerden Tamil
Tiger-guerrilla's, in samenwerking met Pakistaanse inwoners in Barcelona
en Madrid, 80 procent van de heroïne in Spanje. Zodra we dat netwerk via
arrestaties hadden ontmanteld, namen Koerden uit Turkije het over en
domineerden zij volledig gedurende de daaropvolgende twee jaar.''
Wanneer een nieuwe burgeroorlog of opstand uitbreekt, zullen wanhopig
arme strijders hun militaire inzet waarschijnlijk financieren door drugs
te verhandelen en het geld wit te wassen.

\subsection{Door drugs gefinancierde
kortingen}\label{door-drugs-gefinancierde-kortingen}

Georganiseerde criminele syndicaten oefenen een neerwaartse druk uit op
de prijzen van goederen, met uitzondering van drugs. Op microniveau
financieren deze groepen schijnbaar legitieme ondernemingen met de buit
uit hun illegale activiteiten. Ze wassen drugsinkomsten en geld uit
andere onwettige bronnen wit door reguliere producten onder de kostprijs
te verkopen, waardoor ze de prijzen van hun eerlijke concurrenten
ondermijnen en talrijke bedrijven in de problemen brengen.

\subsection{Yakuza-deflatie}\label{yakuza-deflatie}

In Japan vervulden invloedrijke Yakuza-bendes een sleutelrol tijdens de
hyperactieve vastgoedzeepbel van de late jaren tachtig. Hoewel
negentigduizend Yakuza-leden jaarlijks tussen de \$10,19 miljard
(volgens officiële schattingen) en \$71,35 miljard (volgens professor
Takatsugu Nato) genereren, kwam een aanzienlijk deel van de oninbare
leningen die Japanse banken bedreigden voort uit Yakuza-gerelateerde
investeringen. Daardoor ervaart de Japanse economie een deflatoire druk,
of ``prijsvernietiging'' zoals de Japanners het noemen, die kenmerkend
is voor die periode.

\subsection{Een oogje dichtknijpen}\label{een-oogje-dichtknijpen}

De Russische mafiya's, zoals Yeltsin zelf heeft erkend, zijn samengegaan
met ``commerciële structuren, administratieve instanties, ministeries
van Binnenlandse Zaken, stadsautoriteiten \ldots{}'' Door de immuniteit
die de maffia's hebben verworven door deze samenvoeging met de politie,
kunnen ze de inning van hun private belastingen afdwingen via flagrante
geweldpleging. Autoritatieve bronnen geven aan dat vier van de vijf
Russische bedrijven nu bescherming betalen. Volgens sommige berichten
moeten lokale kleine bedrijven in Rusland 30 tot 50 procent van hun
winst afdragen aan afpersers, en niet slechts de magere 7 procent die
van Amerikaanse zakenlieden wordt geëist.

In 1993 waren er officieel 355.500 misdrijven in Rusland geregistreerd
als voorbeelden van ``afpersing'', waaronder bijna ``30.000 moorden met
voorbedachte rade'', meestal moorden op zakenlieden door gangsters.
Volgens een voormalig minister van Binnenlandse Zaken, generaal Viktor
Yerin, ``waren de meeste contractmoorden, vanwege conflicten in de
commerciële en financiële sfeer.'' In de meeste gevallen kozen de
autoriteiten ervoor om ``een oogje dicht te knijpen.'' Criminele
organisaties spelen, ``door hun controle over dwang en corruptie,''
zoals economen Gianluca Fiorentini en Sam Peltzman schrijven in
\emph{The Economics of Organized Crime}, een sleutelrol in de economie.
In theorie kan deze invloed soms voordelig zijn, omdat het regelgeving
beperkt en overheden kan aanmoedigen om de levering van publieke
goederen te verbeteren. De aanwezigheid van een krachtige maffia
``beperkt de monopolistische rol van overheidsautoriteiten'', omdat
overheden in gebieden met krachtige georganiseerde misdaadgroepen
slechts met grote moeite beleid kunnen voeren dat de maffia tegenstaat.

\subsection{Collusie}\label{collusie}

Het is inderdaad opvallend hoe zelden de meeste overheden bereid zijn om
de maffia's direct te confronteren, terwijl dat hun belangrijkste
concurrenten zijn in het gebruik van georganiseerde dwang. In strikt
economische termen is dat niet verrassend. De meest winstgevende
regeling die ``de gekozen leden van het openbaar bestuur'' kunnen
treffen, is een ``samenspanningsakkoord'' met de georganiseerde misdaad.
Fiorentini en Peltzman merken op dat ``er bewijs is van grootschalige
afspraken waarbij de georganiseerde misdaad politieke steun verzekert
voor groepen kandidaten, terwijl deze op hun beurt de gunst terugbetalen
via een welwillend beheer van overheidsopdrachten en de verstrekking van
publieke diensten of subsidies.''

In tegenstelling tot het beeld dat Hollywood schetst, lijkt het
doordringen in en oplichten van overheden tegenwoordig een van de
belangrijkste activiteiten van criminele organisaties zoals de
Siciliaanse maffia te zijn. ``De meeste onderzoekers denken dat de
grootste bron van inkomsten van de Siciliaanse maffia tegenwoordig juist
bestaat uit het zich toe-eigenen van verschillende vormen van
overheidsuitgaven en het organiseren van fraude met lokale, nationale en
Europese subsidieregelingen.''

\subsection{Narco-republieken}\label{narco-republieken}

Zoals we in \emph{The Great Reckoning} hebben gewaarschuwd, zijn vele
regeringen wereldwijd volledig gecorrumpeerd door drugsbaronnen. Mexico
vormt daar een onbetwistbaar voorbeeld van. Voormalig Mexicaans federaal
adjunct-aanklager Eduardo Valle Espinosa plaatste het Mexicaanse systeem
in perspectief in zijn ontslagverklaring: ``Niemand kan een politiek
project opzetten waarin de kopstukken van de drugshandel en hun
geldschieters niet voorkomen, want als je dat wel doet, sterf je.''
Valle verklaarde dat steekpenningen het dienen als Mexicaanse
politiechef zo lucratief maken dat kandidaten tot \$2 miljoen betalen om
aangenomen te worden. Als je het strikt vanuit een winst- en
verliesberekening bekijkt, kan het kopen van een functie binnen de
lokale politie al gauw een winstgevende investering blijken te zijn.
Drugskartels betalen fortuinen aan zelfs laaggeplaatste Mexicaanse
ambtenaren, omdat dat hen immuniteit biedt tegen vervolging voor hun
misdaden.

Colombia is een ander land waar de hoogste regionen van de overheid
gedomineerd worden door drugsbaronnen. Onlangs trokken de Amerikaanse
autoriteiten het visum van de Colombiaanse president Ernesto Samper voor
de Verenigde Staten in, omdat hij opzettelijk politieke bijdragen van
drugsdealers aannam in ruil voor gunsten.

\subsection{De pot verwijt de ketel dat hij zwart
ziet}\label{de-pot-verwijt-de-ketel-dat-hij-zwart-ziet}

Iedereen die in de jaren negentig de berichten in onze nieuwsbrief
\emph{Strategic Investment} heeft gevolgd, zal onmiddellijk de ironie
herkennen in het optreden van Clinton's regering tegenover Samper. Er is
geloofwaardig bewijs dat de Amerikaanse president Bill Clinton alles
heeft gedaan waarvan Samper wordt beschuldigd, en erger. Zelfs als je
ons niet op ons woord gelooft, wordt Clintons achtergrond in opvallend
detail beschreven in twee goed onderzochte boeken, geschreven door
auteurs van tegenovergestelde politieke overtuigingen.

Roger Morris, die over het algemeen een linkse invalshoek heeft, was een
nationaal veiligheidsfunctionaris in de regering van Nixon en een hoge
medewerker van Dean Acheson, president Lyndon Johnson en Walter Mondale.
Morris heeft een doctoraat van Harvard University. Zijn boek
\emph{Partners in Power} beschrijft een schimmig verleden van Clinton
dat Samper als een padvinder doet lijken.

Morris vertelt over Clintons vaderloze jeugd in Hot Springs, Arkansas,
een broeinest voor gokken, prostitutie en georganiseerde misdaad.
Activiteiten waarmee het merendeel van zijn familie op een of andere
manier verbonden was. Clintons stiefoom, Raymond Clinton, die Bill als
een ``vaderfiguur'' beschouwde, stond bekend als een belangrijke
``Godfather''-figuur binnen de Dixie-maffia. Morris beweert dat Bill
Clinton door de CIA werd gerekruteerd en zijn studententijd aan Oxford
doorbracht met het bespioneren van activisten tegen de Vietnamoorlog.
Volgens Morris bleef Clinton een CIA-informant tijdens zijn periode als
gouverneur en hielp hij een CIA-operatie in stand houden die draaide om
drugshandel en wapenverkoop vanuit Mena, Arkansas. Morris lijkt de
gehele CIA te beschuldigen van betrokkenheid bij drugshandel, in plaats
van de mogelijkheid te overwegen dat Clinton samenwerkte met een
corrupte factie binnen het agentschap, wat ons waarschijnlijker lijkt.
Welke interpretatie je ook volgt, beide wijzen erop dat de belangrijkste
geheime inlichtingendienst van de Amerikaanse overheid direct of
indirect betrokken was bij grootschalige drugshandel. Als de CIA geen
verlengstuk van de georganiseerde misdaad is, balanceert ze gevaarlijk
dicht tegen die grens aan.

\subsection{Eén kans op 250.000.000}\label{euxe9n-kans-op-250.000.000}

Toch bevat \emph{Partners in Power} details die interessant zijn voor
iedereen die de corruptie in de moderne Amerikaanse politiek bestudeert.
En Morris richt zijn beschuldigingen zeker niet alleen op Bill Clinton;
ook zijn vrouw krijgt scherpe kritiek. Zo schrijft hij over Hillary
Clintons wonderbaarlijke handel in grondstoffen: ``In 1995 voerden
economen van de universiteiten van Auburn en North Florida een
geavanceerd computermodel van de transacties van de First Lady uit, voor
publicatie in het \emph{Journal of Economics and Statistics}. Ze
gebruikten alle beschikbare gegevens en ook marktdata uit het \emph{Wall
Street Journal}. De kans dat Hillary Rodham haar transacties op
legitieme wijze had uitgevoerd, berekenden ze op minder dan één op
250.000.000.''

Morris verzamelde tal van belastende details over de, onder Clinton in
Arkansas florerende, drugshandel en het witwassen van geld. ``Door de
enorme hoeveelheid drugs en geld die de vluchten opleverden, werd het
kleine Mena, Arkansas, in de jaren tachtig een van de wereldcentra van
de drugshandel.'' Morris citeert een vertrouweling die over Clinton
verklaarde: ``Hij wist het.''

Clinton wist niet alleen van de cocaïnesmokkel, maar vertelde aan
staatspolitieagent L.D. Brown, een voormalige lijfwacht die dankzij
Clinton een baan bij de CIA kreeg, dat de drugshandel geen CIA-operatie
was. ``\,`Oh nee,' zei Clinton, `dat is Lasaters zaak.'\,''

Dan Lasater, een veroordeelde cocaïnedistributeur, was een van Clintons
belangrijkste geldschieters, een man die miljoenen verdiende aan
staatsopdrachten in Arkansas en naar verluidt ooit \$300.000 in
contanten in een bruine papieren zak aan de toenmalige gouverneur van
Kentucky gaf, John Y. Brown. Volgens Morris was Lasater ``niet zomaar
een grote donor die met respect behandeld moest worden, maar een
buitengewoon intieme vertrouweling die Clinton regelmatig bezocht op
zijn makelaarskantoor en die het gouverneurs­huis in- en uitliep wanneer
hij wilde.'' Morris schrijft dat Lasaters chauffeur, die hem vaak naar
het gouverneurshuis bracht, ``een veroordeelde moordenaar was die een
wapen droeg en bijverdiende als drugsdealer.'' Volgens Morris' verslag
had de president van de Verenigde Staten dus nauwere banden met een
drugsdealer dan de vermeende relatie tussen de Colombiaanse president
Ernesto Samper en het Cali-kartel.

\begin{quote}
Wow! Bob zegt dingen over Bill Clinton die zelfs Hillary niet zou
zeggen. -- P.J. O'Rourke
\end{quote}

R. Emmett Tyrell Jr., hoofdredacteur van \emph{The American Spectator},
is geen linkse liberaal zoals Morris, maar zijn boek \emph{Boy Clinton}
bevat veel van dezelfde details die Clinton neerzetten als een corrupte
politicus, nauw verbonden met drugshandel en andere misdrijven. In de
proloog van \emph{Boy Clinton} wordt L.D. Brown, Clintons voormalige
lijfwacht, geciteerd met de sensationele beschuldiging dat Clinton
medeplichtig was aan doodseskaders, bedoeld om getuigen van de
drugshandel in Mena te elimineren.

Brown verklaart specifiek dat hij op 18 juni 1986 persoonlijk naar
Puerto Vallarta, Mexico, werd gestuurd met een Belgisch F.A.L.-geweer.
Reizend onder het alias Michael Johnson moest hij Terry Reed
neerschieten en doden.

Reed kwam in 1994 in de publiciteit als coauteur van \emph{Compromised:
Clinton, Bush and the CIA}. De stelling van \emph{Compromised} is dat de
CIA ``het presidentschap heeft overgenomen'' en dat haar ``zwarte
operaties, als een kankergezwel, zich hebben uitgezaaid naar de organen
van de overheid.'' Concreet beweren Reed en zijn coauteur dat zowel
Clinton als Bush diep betrokken waren bij illegale activiteiten in
Arkansas, waaronder drugshandel.

Brown voerde de moord niet uit. Hij en Reed overleefden om hun verhaal,
gedeeltelijk, te vertellen, wat hen gelukkiger maakt dan anderen die met
Clinton te maken hadden, toen of later. Denk aan de inmiddels overleden
Jerry Parks, die in 1992 de beveiliging verzorgde voor het
Clinton-Gore-hoofdkwartier en in september 1993 werd doodgeschoten in
een gangstermoord. In een bizarre wending onthulde de S\emph{unday
Telegraph} op basis van exclusieve informatie van Parks' weduwe dat
Parks door de inmiddels overleden Vincent Foster was ingehuurd om Bill
Clinton te bespioneren.

Waarom Foster een dossier met belastende informatie over Clinton wilde
samenstellen, blijft gissen. (Hij zei dat hij het voor Hillary deed.)
Hoe dan ook, dit ondermijnt het officiële beeld van Foster als een
naïeve plattelandsjongen die zo geschokt was door de meedogenloze
politiek van Washington dat hij uit wanhoop zelfmoord pleegde. Dat
verhaal, dat nooit echt geloofwaardig was, wordt met elke nieuwe
onthulling nog minder aannemelijk.

\subsection{De president van de
maffia}\label{de-president-van-de-maffia}

Hoewel de wereld als geheel terugschrikt voor de verontrustende
conclusie dat de president van de Verenigde Staten besmet zou zijn door
nauwe banden met de georganiseerde misdaad, is dat wat het bewijs
suggereert. Morris citeert een voormalige federale aanklager die figuren
uit de georganiseerde misdaad volgde. Hij stelt dat Clintons verkiezing
tot gouverneur in 1984 ``het moment was waarop de maffia echt zijn
intrede deed in de politiek van Arkansas, de mensen van de hondenrennen,
de jongens uit gokcircuit, de omkopers die hun kans zagen \ldots{} het
ging verder dan onze oude Dixie-maffia, die daarbij verbleekte. Dit was
geld van misdaadsyndicaten aan de oost- en westkust, die net als de
legitieme bedrijven de mogelijkheden zagen.''

Blijkbaar zagen anderen met dezelfde instelling eveneens mogelijkheden
bij Clinton. \emph{New York Magazine}, na een eerder artikel in
\emph{Reader's Digest}, meldde dat ``de belangrijkste bondgenoten van de
president in de vakbeweging ook mannen zijn die verbonden zijn met wat
zonder twijfel enkele van schimmigste, meest door de maffia doordrongen
vakbonden in Amerika zijn.'' In het bijzonder wordt gewezen op Clintons
nauwe relatie met Arthur Coia, een van zijn ``belangrijkste
fondsenwervers'' en voorzitter van de \emph{Laborers International Union
of North America}, ``een van de meest opzichtig corrupte vakbonden in de
geschiedenis van de arbeidersbeweging.''

Blijkbaar sloot het ministerie van Justitie onder Clinton, volgens
\emph{New York}, een ``merkwaardig royaal akkoord'' met Coia, zodat hij
zijn functie kon behouden ondanks overtuigende beschuldigingen van
datzelfde ministerie dat hij al jarenlang een vertrouweling van
misdaadfiguren was.

Of Terry Reeds stelling klopt dat ``de CIA het presidentschap heeft
overgenomen'', valt te betwijfelen, maar het is duidelijk dat individuen
binnen een geheime organisatie die gemachtigd is om ``zwarte operaties''
uit te voeren, sterk verleid worden om, zoals professor Hirshleifer
stelt, ``onwettige middelen te gebruiken om hun middelen te verwerven.''

Gezien de technologische veranderingen die de doorslaggevendheid van
massale militaire macht verminderen, mag je verwachten dat corruptie
toeneemt, zo niet een regelrechte overname van overheden door
georganiseerde misdaad. Hirshleifer stelt, en wij zijn het daarmee eens,
dat ``de instellingen van de politieke economie nooit zo volmaakt kunnen
zijn dat ze volledig de onderliggende werkelijkheden van de natuurlijke
economie verdringen.'' De macht devolueert in de ``natuurlijke
economie'', wat ingrijpende verschuivingen in de machtsverhoudingen
binnen de samenleving impliceert.

Politieke corruptie, zoals Vito Tanzi scherp opmerkt, ``betekent een
privatisering van de staat waarbij de macht niet verschuift naar de
markt, zoals privatisering normaal impliceert, maar naar ambtenaren en
bureaucraten.'' Dit is in feite gebeurd met de FBI en andere
politiediensten onder Clinton. De ``rechtsstaat'' wordt steeds meer wat
Clinton en zijn trawanten ervan willen maken. Tot dusver lijkt er weinig
bewijs dat details over deze corrupte banden enig gewicht hebben bij
kiezers, zelfs niet als de massamedia ze zouden oppakken. Integendeel,
er lijkt nauwelijks bezorgdheid te bestaan over aanwijzingen dat de
president van de Verenigde Staten betrokken is bij drugshandel,
witwassen en erger.

Dit doet denken aan Walter Lippmanns vrees dat kiezers het onderscheid
niet kunnen maken tussen echte en fictieve persoonlijkheden. Volgens hem
``doen vleierij en verheerlijking kiezers geen goed. Ze worden misleid
door de onderdanige hypocrisie die suggereert dat goed en kwaad,
waarheid en leugen, afhangen van hun stem.''

Lippmann zag een ``instorting van de constitutionele orde'' die kon
leiden tot ``de plotselinge en catastrofale neergang van de westerse
beschaving. We zijn ver gevallen in korte tijd \ldots{} Wat we zien is
niet alleen verval, al is veel van de oude structuur op aan het lossen,
maar iets dat een historische ramp genoemd kan worden.''

Het probleem is dat politieke oordelen steeds minder een reactie zijn op
de echte wereld, en meer op een pseudowerkelijkheid die het publiek
heeft geconstrueerd over verschijnselen buiten hun directe waarneming.
Maar het zou een fout zijn om je te laten leiden door de beperkingen van
wat anderen zien. Zelfs als het je niets kan schelen of Vincent Foster
is vermoord en zijn dood door de hoogste politie-autoriteiten en
functionarissen van de Amerikaanse regering in de doofpot werd gestopt,
inclusief de toenmalige speciale aanklager Kenneth Starr, zou je toch
het bredere patroon van banden tussen georganiseerde misdaad en het
Witte Huis in overweging moeten nemen.

Op de lange termijn maakt politieke corruptie op het hoogste niveau de
traditionele lofzangen over de democratie als middel tot doelbewuste
oplossing van publieke problemen betekenisloos. In het
Informatie­tijdperk zal het veel minder belangrijk zijn dat de overheid
groot en machtig is, dan dat ze eerlijk is. De meeste diensten die
overheden historisch hebben geleverd, zullen in het volgende millennium
overgaan naar de private markt. Gezien het wereldwijde bewijs is het
echter te betwijfelen of je lang kunt vertrouwen op een corrupt systeem
met corrupte leiders voor de veiligheid van je gezin en je
investeringen.

Zoals Morris zegt: ``{[}D{]}e Clintons zijn niet slechts een symptoom,
maar het symbool van het grotere, tweepartijen­systeem op zijn
eind-twintigste-eeuwse doodlopende weg.'' Vito Tanzi laat in zijn essay
over corruptie zien dat ``de enige manier om corruptie te bestrijden is
door de schaal van overheidsinterventie aanzienlijk te verkleinen.'' De
Informatierevolutie zal die schaal drastisch verkleinen en biedt daarmee
hoop op een heropleving van moraal en eerlijkheid. Een andere duidelijke
implicatie van de Informatierevolutie voor de moraal is de toegenomen
kwetsbaarheid die gepaard gaat met cyberhandel en virtuele
ondernemingen, die communiceren via onkraakbare encryptie. Interne
dieven binnen een organisatie, zelfs een virtuele, zullen moeilijker te
ontdekken zijn, en het zal vrijwel onmogelijk zijn om gestolen geld of
geheime betalingen voor de verkoop van handelsgeheimen, patenten of
andere waardevolle activa terug te halen.

Misdaad loont, en velen vinden het aantrekkelijk om legale, productieve
activiteiten aan te vullen met illegale, roofzuchtige. In tegenstelling
tot de situatie die in de afgelopen twee eeuwen in het Westen gold, zijn
criminelen niet langer slechts buitenstaanders zonder aanzien. Wanneer
misdaad loont, krijg je simpelweg een beter soort crimineel, omdat er
weinig sociale schande aan kleeft. De Siciliaanse maffia bijvoorbeeld,
en veel drugshandelaren die lokaal personeel tegen hoge lonen in dienst
nemen, genieten in hun eigen omgeving respect en steun.

\section{De morele orde en haar
vijanden}\label{de-morele-orde-en-haar-vijanden}

Alle sterke samenlevingen rusten op een degelijke morele basis. Elk
onderzoek naar de geschiedenis van economische ontwikkeling illustreert
de nauwe relatie tussen morele en economische factoren. Landen en
groepen die succesvol ontwikkelen, slagen er deels in omdat zij een
ethiek hanteren die zelfredzaamheid, hard werken, familie en
maatschappelijke verantwoordelijkheid, spaarzaamheid en eerlijkheid
bevordert. Dit geldt ook voor kleinere sociale subgroepen. Het zakelijk
succes van Joden, met name religieuze Joden, van de Puriteinen in New
England, van de Quakers in de Britse zakenwereld in de achttiende en
negentiende eeuw, en van de Mormonen in het moderne Amerika, toont stuk
voor stuk de economische voordelen aan die voortvloeien uit een cultuur
met een stevig moreel raamwerk.

Neem als voorbeeld de Quakers. Deze groep werd zakelijk zeer succesvol
en stond om diverse redenen vooral bekend als bankiers. Zij hanteerden
de hoogste norm van betrouwbaarheid: in plaats van eden af te leggen,
beschouwden zij elke zakelijke toezegging als even bindend als een eed.
``Mijn woord is mijn borg'' was voor hen een onwrikbaar principe. Ze
kozen voor een rustige, fatsoenlijke en zuinige levensstijl en
beschouwden het als hun religieuze plicht om geen geld te verspillen aan
de ijdelheden van deze wereld. Ze vermeden ruzies en waren ervan
overtuigd dat oorlog altijd zondig is. Ze vonden dat een zakenman de
morele verplichting had om ``fair value'' te leveren, wat hen als
handelaren een reputatie opleverde voor het bieden van hoge kwaliteit
tegen gematigde prijzen. De uitdrukking ``caveat emptor'', oftewel de
koper moet maar opletten, volstond voor hen niet. In een tijd waarin de
meeste kooplieden hoge prijzen en grote marges hanteerden, leidde de
Quaker-moraal onvermijdelijk tot een strategie met lage winstmarges en
een hoge omloopsnelheid. Zoals Henry Ford later aantoonde, kan zo'n
beleid potentieel veel winstgevender zijn. Ze kozen bewust voor deze
benadering omdat zij het als hun plicht zagen om hun klanten niet te
bedriegen en ontdekten al snel dat dit tegelijkertijd de beste manier
bleek om hun ondernemingen uit te breiden. Zo bleken de Quakers
uitstekende zakenpartners te zijn: hun klanten keerden keer op keer
terug, waardoor beide partijen voordeel hadden. Bovendien gaven ze als
een zuinige gemeenschap die haar verplichtingen nakwam de Quakers een
voorsprong als bankiers, en het lidmaatschap van de Quakers op zich
vormde al een betrouwbaar bedrijfsinstrument.

Helaas kan het succes waarmee zulke zakelijke voordelen worden behaald,
er uiteindelijk voor zorgen dat ze zichzelf ondermijnen. Landen
doorlopen immers een cyclus, zoals beschreven in de sociologische
theorie van Adam Ferguson uit de achttiende eeuw, die begint bij armoede
en hard werken, overgaat in rijkdom, luxe en decadentie, en uiteindelijk
leidt tot achteruitgang. De oude Romeinen keken met heimwee terug op de
deugden van de republikeinse periode waarin het rijk werd opgebouwd en
betreurden de luxe en luiheid die zij als oorzaak van hun ondergang
beschouwden. Deze ondermijning van de ijverige deugden door welvaart kan
verrassend snel plaatsvinden. Hoewel de Duitsers nog steeds een capabel
en efficiënt volk zijn, werken zij totaal niet meer zo hard als toen zij
hun land herbouwden na de verwoestende nederlaag in 1945. In slechts
twee generaties stapten zij over van lange werkdagen in tijden van acute
armoede, naar korte werkdagen in ruil voor het hoogste loon en de beste
sociale voorzieningen op aarde. In oktober 1995 tekenden zestien Duitse
werkgeversverenigingen de Petersburg-verklaring, een overzicht van goed
onderbouwde klachten die de achteruitgang van de industriële moraal in
Duitsland weerspiegelen.

\begin{quote}
In 1995 bereikte de belastingdruk in Duitsland recordhoogten, mede door
de solidariteitstoeslag en de bijdragen aan de verpleegzorgverzekering.
Met een totale ondernemingsbelasting van meer dan 60 procent ligt
Duitsland ver boven het internationale gemiddelde van 35 tot 40 procent.
Praktijken in de publieke sector, zoals gereguleerde promoties, een baan
voor het leven en hogere pensioenuitkeringen, moeten wijken voor
vrijemarkteisen op het gebied van meritocratische promotie en beloning.
Omdat Duitsland de hoogste arbeidskosten ter wereld kent, moet het
loonbeleid bijdragen aan het terugdringen van de werkloosheid door de
kosten voor ondernemingen te verlagen. Loonstijgingen zouden gebaseerd
moeten worden op concurrentievermogen en productiviteit, en het gedrag
van de vakbonden dient te veranderen. Het jaarlijkse ritueel van
campagnes, eisen, mobilisatie van werknemers, dreigementen en
waarschuwingsstakingen is schadelijk.
\end{quote}

Die bezorgdheid dat vooral de Duitsers, in het bijzonder de jongeren en
de erfgenamen van de welvaart, de gewoonte van hard werken dreigen te
verliezen, wordt door bondskanselier Kohl gedeeld.

Het arbeidscontract bij Volkswagen kent autowerkers ter wereld het
hoogste loon toe, en daarbovenop komen nog socialezekerheidsbijdragen,
in ruil voor een werkweek van 28 uur, vier dagen van elk zeven uur. Het
naoorlogse Duitsland exporteert nu massaal banen. Halverwege de
negentiende eeuw stonden de Britten bekend als de meest efficiënte
industriële natie, een reputatie die zij een eeuw later zeker hadden
ingeleverd. De cyclus van welvaart ondermijnt onvermijdelijk de deugden
van hard werken en bescheiden verwachtingen die kenmerkend zijn voor de
vroege fasen van een succesvolle industriële opbouw. Naties slagen er
vaak niet in hun oorspronkelijke deugden te behouden, net zoals
individuen bij een te gemakkelijke vorm van succes hebberig en lui
kunnen worden.

Wereldwijde investeringen prijzen terecht deze ijverige deugden en
straffen wie hebberig en lui worden, zoals het hoort. Men kan stellen
dat een verstandige investering zowel moreel als financieel verantwoord
moet zijn. Die Engelsman uit de achttiende eeuw, die investeerde in het
kapitaal van een Quakerbank, boekte naar alle waarschijnlijkheid groot
succes. In de negentiende eeuw stopten de Quakers hun geld in
chocoladebedrijven, ervan overtuigd dat cacao gezonder was dan alcohol.
Dat was waarschijnlijk zo, maar investeren in Fry's of Cadbury's bleek
eveneens een slimme zet. Investeerders moeten oppassen dat ze niet
verstrikt raken in perioden van decadentie. Hoewel Duitsland op de
Europese markt sterk staat en beschikt over geavanceerde industriële
kennis, beperken hoge arbeidskosten en korte werktijden het toekomstige
potentieel van het land.

Sociale moraal en economisch succes gaan hand in hand. Maar welke
factoren versterken juist de sociale moraal en welke ondermijnen die?
Arnold Toynbee, de invloedrijke filosofisch-historicus uit de eerste
helft van de twintigste eeuw, introduceerde de theorie van uitdaging en
respons. Uitdagingen geven samenlevingen nieuwe kracht en brengen
kwaliteiten naar boven die ze niet wisten te hebben.

Mensen beseffen al lang dat moeilijke tijden vaak gezondere reacties
oproepen dan periodes van overvloed. We streven er allemaal naar om het
comfortabel te hebben. We dromen van een fijne woning, een baan waar we
voldoening uit halen en een stevige spaarpot. De strijd om deze doelen
te bereiken levert ons veel op. We studeren hard, trainen onszelf en
werken keihard, ons loonbeleid moet de werkloosheid bestrijden door
bedrijfskosten te verlagen\ldots{} Loonstijgingen moeten gemeten worden
aan de hand van de concurrentiekracht en productiviteit\ldots{} het
gedrag van de vakbonden moet veranderen. Het jaarlijkse ritueel van
campagnes, eisen, werknemersmobilisaties, dreigementen en
waarschuwingstakingen is immers schadelijk.

Voor velen blijkt het bereiken van die doelen een soort valkuil. De
strijd is beter dan de overwinning. Begin deze eeuw behandelde de
gerenommeerde Zwitserse psycholoog Carl Jung een Amerikaanse zakenman
die al van jongs af aan de volgende ambities koesterde: hij werkte
onvermoeibaar om een eigen bedrijf op te zetten en genoeg geld te
verdienen zodat hij op veertigjarige leeftijd met pensioen kon. Hij
trouwde met een jonge, aantrekkelijke vrouw, kocht een prachtig huis,
stichtte een gezin en verkocht uiteindelijk met groot succes zijn
bedrijf, zodat hij als rijke, onafhankelijke man zonder zorgen met
pensioen ging. In het begin omarmde hij zijn vrijheid en deed hij alles
wat hij zichzelf had beloofd. Hij nam zijn gezin mee op reis door
Europa, bezocht kunstgalerijen en dergelijke. Maar na verloop van tijd
vervaagden die interesses en het gevoel van vrijheid zelf. Hij keek
terug op de periode dat hij `niet vrij' was, toen hij dag en nacht aan
zijn bedrijf werkte en alle gebruikelijke zakelijke zorgen had, en
beschouwde die tijd als de gelukkigste van zijn leven. Uiteindelijk
raakte hij in een depressie, waarna zijn vrouw hem naar Jung bracht.
Jung stelde vast dat hij geen uitlaatklep had voor zijn creatieve
energie, die naar binnen was gekeerd en hem verwoestte. Hoewel de
diagnose waarschijnlijk klopte, leidde dat niet tot genezing; de
zakenman herstelde nooit van zijn zenuwinzinking.

Voor veel mensen is de strijd belangrijker dan de uiteindelijke
overwinning. Wij zijn gemaakt voor actie, en succes kan uiteindelijk
teleurstellend blijken te zijn. Ambitie, wat voor vorm die ook aanneemt,
stuwt ons voort in de strijd, maar die strijd blijkt vaak bevredigender
dan het beoogde eindresultaat, zelfs wanneer we ons doel bereiken.
Natuurlijk slagen de meesten van ons er slechts gedeeltelijk in onze
dromen waar te maken. We hebben niet zoveel geld als we zouden willen en
wonen niet in het perfecte droomhuis. We moeten genoegen nemen met
minder.

Men kwam tot het besef dat deugdzaamheid een dynamisch begrip is, dat ze
vooral wordt gevormd door de inzet en niet louter door het behaalde
resultaat, en dit inzicht ontwikkelde zich op uiteenlopende manieren in
de negentiende eeuw. Er is een bekend gedicht van Arthur Hugh Clough dat
tijdens de Tweede Wereldoorlog veel troost bood te midden van de strijd
tussen leven en dood. Het valt op dat het zelfmoordcijfer in de
strijdende landen in die periode daalde. Zelfs het voeren van oorlog
bleek soms beter dan wegzinken in de depressie van inactiviteit.

\begin{quote}
\emph{Zeg niet: de strijd is zinloos,}\\
\&\#xNAN;\emph{Het zwoegen en de littekens zijn niet voor niets,}\\
\&\#xNAN;\emph{De vijand verslapt niet, en faalt niet,}\\
\&\#xNAN;\emph{En zoals dingen waren, zullen ze blijven.}

\emph{Als hoop je in de steek laat, kunnen angsten je misleiden;}\\
\&\#xNAN;\emph{Misschien, verborgen in die rook daar,}\\
\&\#xNAN;\emph{Achtervolgen je kameraden nu de vluchtenden,}\\
\&\#xNAN;\emph{Die, als jij er niet zou zijn geweest, het veld zouden
bezitten.}

\emph{Want terwijl de vermoeide golven tevergeefs beuken,}\\
\&\#xNAN;\emph{Lijkt het alsof hier geen pijnlijke centimeter te winnen
is;}\\
\&\#xNAN;\emph{Ver weg, via kreken en inhammen,}\\
\&\#xNAN;\emph{stroomt geruisloos het zeewater het land in.}

\emph{En niet uitsluitend via oosterse ramen,}\\
\&\#xNAN;\emph{Wanneer het daglicht doorbreekt, schijnt het licht
binnen,}\\
\&\#xNAN;\emph{Voor ons, klimt de zon langzaam omhoog, zo langzaam,}\\
\&\#xNAN;\emph{Maar kijk naar het westen, licht schijnt over het land.}
\end{quote}

Deze actieve vorm van competitie spreekt nog steeds de moderne
gevoeligheid aan. Velen van ons leiden immers hun leven als een
voortdurende strijd om kansen in een vaak vijandige omgeving te
benutten. We leven in een competitieve wereld en bijna niemand wil zich
daaraan onttrekken. Natuurlijk bestaat er ook een contemplatieve,
spirituele inslag, maar die komt zelden voor.

Een vergelijkbare 19e-eeuwse visie op deze dynamische moraal verwoordde
William James, de vooraanstaande Amerikaanse filosoof, in een toespraak
voor de Yale Philosophical Club in 1891 als volgt:

\begin{quote}
Het fundamentele verschil in het morele leven van de mens ligt in het
onderscheid tussen een ontspannen en een inspannende stemming. Als we in
een ontspannen stemming verkeren, staat het afwijken van het actuele
kwaad als doorslaggevende overweging voorop. De inspannende stemming
daarentegen zorgt ervoor dat we het actuele kwaad links laten liggen,
mits het hogere ideaal wordt bereikt. Waarschijnlijk bezit iedere mens
het vermogen voor een inspannende stemming, maar bij sommigen komt dat
moeilijk tot uiting. Het heeft de heftigere hartstochten nodig om het te
wekken, de grote angsten, liefdes en verontwaardiging, of anders de diep
doordringende aantrekkingskracht van een hogere trouw, zoals
rechtvaardigheid, waarheid en vrijheid. Voor de visie ervan is een
krachtige verlichting noodzakelijk; en een wereld waarin bergen
instorten en valleien oprijzen, biedt geen gastvrije omgeving voor die
stemming. Daarom kan deze stemming bij een eenzame denker eeuwig
sluimeren zonder ooit ontwaakt te worden. Zijn uiteenlopende idealen,
waarvan hij weet dat het slechts persoonlijke voorkeuren zijn, dragen
vrijwel dezelfde symbolische waarde, waardoor hij er naar hartenlust mee
kan jongleren. Dit verklaart bovendien dat in een puur menselijke wereld
zonder God de oproep tot onze morele energie niet zijn maximale
stimulerende kracht bezit. Het leven vormt, zelfs in zo'n wereld, een
oprechte ethische symfonie; maar die speelt zich af binnen het beperkte
bereik van een paar arme octaven, waarbij de oneindige rijkdom aan
waarden niet de ruimte krijgt zich te openbaren.
\end{quote}

William James was ervan overtuigd dat de dynamische moraal, waarin doen
boven zijn en handelen boven afzien van actie centraal staan, ook
toepasbaar is binnen de religieuze sfeer. Er is ook een sterke
ontwikkeling van de moraal van competitie en overleving in het werk van
Adam Smith (1776), waarvan het centrale thema, de morele leer van de
huidige wereld­economische orde, uitgebreide aandacht verdient.

Het fundamentele idee van darwinisme is dat soorten overleven doordat ze
zich aanpassen aan hun omgeving en dat het proces van natuurlijke
selectie hun kenmerken vormgeeft. Bij dieren komt dat proces door
willekeurige mutaties tot stand, waarvan inmiddels bekend is dat ze
onderdeel zijn van een genetisch proces waar Darwin zelf slechts naar
kon raden. Het voortbestaan van menselijke samenlevingen berust echter
op culturele keuzes, gebaseerd op menselijke intelligentie. Cultuur
transformeert de samenleving op dezelfde wijze als genen andere soorten
veranderen. Hierdoor kunnen in onze samenlevingen veranderingen veel
sneller plaatsvinden; het draait immers niet om vele generaties, zoals
bij willekeurige genetische mutaties het geval is. In tegenstelling tot
dieren, die via natuurlijke selectie evolueren, ontwikkelden mensen een
vorm van culturele selectie. Sommige samenlevingen voerden in een
bepaald tijdvak baanbrekende technologieën in die hen een doorslaggevend
voordeel gaven bij het vergaren van welvaart of het verkrijgen van
macht. Het culturele voordeel van nieuwe technologieën blijkt vaak
bepalend, denk maar aan de mens in de Ijzertijd ten opzichte van die in
het Bronzen Tijdperk, of de elektronische mens ten opzichte van de
mechanische mens. Adam Smith was misschien niet de eerste econoom die
het welzijn van naties reduceerde tot individuele handelingen, maar hij
verwoordde dit op de meest bondige en overtuigende wijze:

\begin{quote}
Ieder individu streeft er voortdurend naar om het beste uit het kapitaal
te halen dat hij bezit. Hij werkt immers voor zijn eigenbelang, niet
voor dat van de samenleving. Toch leidt zijn zoektocht naar persoonlijk
gewin er onvermijdelijk toe dat hij kiest voor hetgeen uiteindelijk het
meeste voordeel voor de gemeenschap oplevert.
\end{quote}

Thomas Malthus, de pionier op het gebied van bevolkingsstudies, stelde
vast dat Adam Smiths redenering niet alleen op de economische
ontwikkeling van naties van toepassing is, maar ook op het voortbestaan
van menselijke populaties. Hij is vooral beroemd om zijn stelling dat
``de bevolking, wanneer zij niet wordt ingeperkt, toeneemt volgens een
exponentieel tempo, terwijl het levensonderhoud slechts lineair groeit.
Een geringe kennis van getallen volstaat om het enorme verschillen
tussen deze groeipatronen te zien.''

Lang voor Darwin wees Malthus erop dat hetzelfde principe in de gehele
natuur geldt:

\begin{quote}
De natuur verspreidt de zaden van het leven in dieren- en plantenrijken
met een ongekende overvloed en vrijgevigheid. Tegelijkertijd gaat ze
zuinig om met de beschikbare ruimte en de voeding die nodig is om die te
laten groeien. Die levenskiemen, gehuisvest in een klein stukje aarde en
omgeven door overvloedig voedsel en ruime groeimogelijkheden, zouden in
enkele duizenden jaren miljoenen werelden kunnen vormen. Maar de
allesomvattende wet van de natuur houdt hen binnen vaste grenzen.
\end{quote}

Aan het einde van de achttiende eeuw bleek al dat de wereld zich op een
dynamische manier ontwikkelt, een inzicht dat al in de tijd van Adam
Smith en Malthus werd opgemerkt. De mens, als slechts één van de vele
levensvormen, moet concurreren vanwege de spanning tussen een onbeperkte
voortplantingscapaciteit en een beperkt vermogen om voedsel te
produceren. Het voortbestaan van menselijke samenlevingen, net als dat
van diersoorten, vraagt om een succesvolle aanpassing aan de omgeving.
Een dynamische moraal pakt daarom de uitdagingen van deze aanpassing
actief aan, doordat mensen hun handelen afstemmen op de kansen die hun
omgeving biedt en zo de beschikbare middelen optimaal benutten.

Malthus merkte op dat de ideeën van Adam Smith de wereld wezenlijk
hadden veranderd en stelde dat zijn eigen betoog over de bevolking niet
geheel origineel was: ``De principes waarop het berust, zijn deels
uitgelegd door Hume en deels door Dr.~Adam Smith.'' Tevens benadrukte
hij dat deze voortdurende strijd om te overleven niet louter een
praktische, maar in essentie een morele zaak is. De afsluitende alinea
van het ``Essay'' uit 1798 luidt als volgt:

\begin{quote}
Het kwaad bestaat in de wereld niet om wanhoop te zaaien, maar om ons
tot actie te bewegen. Wij mogen het niet passief accepteren, maar moeten
ons inspannen het te vermijden. Het is zowel in ons eigen belang als
onze plicht om met alle macht het kwaad uit onszelf en uit de kring
waarop wij invloed uitoefenen weg te nemen; hoe vastberadener wij deze
taak volbrengen, des te wijzer en succesvoller onze inspanningen worden
en hoe meer wij onze eigen geest weten te verheffen, zodat wij
vollediger lijken te voldoen aan de wil van onze Schepper.
\end{quote}

We kunnen het belang van dit betoog verder illustreren met Darwins
samenvatting van hoofdstuk 3 uit zijn baanbrekende boek \emph{On the
Origin of Species}, voor het eerst gepubliceerd in 1859. Hij noemde dit
cruciale hoofdstuk ``\emph{Struggle for Existence}.'' De
onderwerpstitels luiden als volgt: Betrekking tot natuurlijke selectie,
De term in ruime zin gebruikt, Wiskundige vermogens tot vermeerdering,
Snelle toename van ingevoerde dieren en planten, Aard van de remmingen
op vermeerdering, Concurrentie is universeel, Invloed van klimaat,
Bescherming door het aantal individuen, Complexe relaties tussen alle
dieren en planten in de natuur, Strijd om het bestaan het hevigst tussen
individuen en variëteiten van dezelfde soort; vaak ook hevig tussen
soorten van hetzelfde geslacht, De relatie van organisme tot organisme
is de belangrijkste van alle relaties.

Sinds 1776 blijkt al duidelijk dat naties hun welvaart het best
optimaliseren wanneer zij mensen de vrijheid bieden hun eigen
kapitaalrendement te maximaliseren in een systeem van vrije
concurrentie. Sinds 1798 weten we dat het voortbestaan van populaties
afhangt van samenlevingen die economisch en politiek voldoende succes
boeken om in hun eigen onderhoud te voorzien, zich te beschermen tegen
infectieziekten en hun bevolking in oorlogstijd te verdedigen. Sinds
1859 is gebleken dat het hele drama van het leven, of het nu om mensen,
dieren of planten gaat, een voortdurende strijd om te overleven inhoudt,
waarbij nauw verwante soorten of culturen vaak elkaars grootste rivalen
vormen. Deze strijd vraagt om een dynamische moraal die het kwaad actief
afweert in plaats van er enkel reactief mee om te gaan.

Deze ideeën bleken zo krachtig dat, sinds hun ontstaan, niemand nadacht
over de aard van de mensheid of morele vraagstukken zonder er meteen op
in te springen. Karl Marx hechtte evenveel waarde aan de strijd om te
overleven als Charles Darwin, maar hij interpreteerde die als een oorlog
tussen sociale klassen, gevormd door economische krachten. Ook Adolf
Hitler geloofde in de strijd om te overleven en bekeek zijn politieke
carrière bijna uitsluitend vanuit dat perspectief. Hij vond echter dat
de strijd tussen verschillende rassen werd uitgevochten. Marx, Lenin,
Stalin, Mao en Hitler vertoonden allemaal kenmerken van
sociaal-Darwinisme, omdat zij de strijd om te overleven, `Mein Kampf',
zoals Hitler dat noemde, als de kern van de politieke realiteit
beschouwden. Marxisten zagen sociale klassen als afzonderlijke groepen,
terwijl nazi's rassen op een vergelijkbare wijze benoemden.

Dit leidt echter niet tot een dynamische moraal, zoals Malthus had
beoogd, maar juist tot een dynamische immoraliteit. Zowel het marxisme
als het nazisme wilden de strijd om te overleven aanpakken, maar deden
dat door de concurrentie te vernietigen. Ze drongen op onbekend terrein
door conflicten te zaaien tussen klassen die om sociale macht streden of
tussen rassen die zij beschouwden als economische uitbuiters (zoals vaak
door antisemieten jegens joden wordt beweerd) of als een gevaarlijke
onderklasse (zoals de angst die blanke vijanden voor zwarten
koesterden). De Tweede Wereldoorlog vormde voor Adolf Hitler een
mislukte poging om voor het Duitse volk een overlevingsvoordeel te
creëren door potentiële rivalen, met name Slavische volkeren en joden,
uit te schakelen. Paradoxaal genoeg kwam Duitsland er met een nederlaag
in de oorlog uiteindelijk beter vanaf dan wanneer de nazi's zouden
hebben gewonnen.

Het alternatief voor destructieve ``interferentieconcurrentie'' is
collaboratieve concurrentie, het kernidee van Adam Smith, maar ook van
Malthus en William James. Het archetype van destructieve concurrentie is
de veroveraar, die zijn rivalen verplettert om hun bezittingen voor zich
te nemen, wat soms inhoudt dat hij hun landen overneemt en hun volk tot
slaaf maakt. Daartegenover staat de handelaar, het archetype van
collaboratieve concurrentie. Een handelaar zorgt ervoor dat de klant
tevreden is, want alleen tevreden klanten komen terug voor meer
transacties. Daarnaast is het gunstig als de klant welvarend is,
aangezien een welvarende klant meer kan kopen. Verovering betekent de
vernietiging van de ander, terwijl handel draait om het vervullen van de
behoeften van de ander. Nu moderne technologie verovering tot een
buitengewoon risicovol beleid heeft gemaakt, vormt handel de enige
rationele benadering van de overlevingsvraagstukken.

Deze wederzijdse afhankelijkheid krijgt extra kracht door een ander
belangrijk principe van Adam Smith, namelijk de arbeidsverdeling.
\emph{The Wealth of Nations} opent met een beroemde passage waarin Smith
opmerkt dat ``de grootste vooruitgang in de productieve krachten van
arbeid en het merendeel van de vaardigheid, behendigheid en het
beoordelingsvermogen dat ergens wordt toegepast, schijnbaar voortkomt
uit de arbeidsverdeling. Hij merkt bijvoorbeeld op dat de productie van
een speld is opgedeeld in zo'n achttien verschillende handelingen, die
in sommige fabrieken door diverse arbeiders worden uitgevoerd. Hoe
verder de specialisatie, hoe efficiënter de productie, maar zo'n
economie is vanzelfsprekend sterk afhankelijk van samenwerking: succes
vraagt om gezamenlijk optreden.

Een succesvolle sociale moraal moet aan een aantal voorwaarden voldoen.
Ze moet krachtig zijn, een zwakke moraal is immers kwetsbaar en
ineffectief, en de strijd om te overleven bevorderen op een
collaboratieve in plaats van een destructieve manier. Hitler hanteerde
een krachtige overlevingsmoraal, maar zijn destructieve drang heeft
bijna zijn eigen samenleving verwoest. Verder moet een degelijke sociale
moraal dynamisch zijn, zodat hij meebeweegt met de voortdurende
veranderingen in moderne technologie en alle hedendaagse sociale
systemen, en bovendien economisch efficiënt opereren. De mix van
egalitaire en autoritaire ideeën binnen het Leninistische systeem bleek
simpelweg niet te werken. Dit zijn echter niet alle aspecten die zo'n
sociale moraal zou moeten hebben. Het bredere doel is om prettig in de
samenleving te kunnen leven en om mensen aan elkaar te binden. Bovendien
moeten moralen zich flexibel kunnen aanpassen en overleven. Een fragiele
moraal zou in onze generatie acceptabel kunnen zijn, maar in de volgende
kunnen worden afgewezen, terwijl een starre, traditionele moraal te
inflexibel zou kunnen zijn om mee te bewegen met de veranderende sociale
structuur. Tegelijkertijd levert een puur relativistische benadering
geen moraal op, omdat zij geen duidelijke richtlijnen verschaft voor
gewenst gedrag.

Allereerst kunnen we alle sociale moraliteit in een bredere context
plaatsen. Een hechte gemeenschap, zelfs een virtuele, is afhankelijk van
een moraal die breed wordt gedeeld. De meest succesvolle perioden in de
geschiedenis ontstaan wanneer de collectieve moraal algemeen gedragen
wordt. Zo'n moraal vervult niet alleen specifieke functies, zoals het
verminderen van criminaliteit en het versterken van familie- en sociale
structuren, maar geeft burgers ook een gevoel van betekenis en richting.
Historische eensgezindheid over moraliteit hangt vaak samen met het
bestaan van een dominante religie, of het nu gaat om een
staatsgodsdienst die in vroege tijden het overleven van een verspreid
volk waarborgde, de islam met haar sociale regels, het middeleeuwse
katholicisme of het protestantisme in het vroege New England. De
begrippen volk, moraal en religie zijn onlosmakelijk verbonden en
versterken elkaar.

In zo'n morele samenleving kan iedere burger binnen een kader van
sociale steun aan persoonlijke doelen werken. Natuurlijk kunnen de
morele voorschriften soms arbitrair overkomen, of op zijn minst zo
worden ervaren door buitenstaanders. Een orthodoxe jood verliest de
vrijheid om varkensvlees of schelpdieren te eten en op de sabbat te
werken. Een trouwe katholiek wordt beroofd van de mogelijkheid om
kunstmatige anticonceptie te gebruiken, om nog maar te zwijgen van
abortus. Een moslim mag geen alcohol nuttigen. En een vrome confuciaan
moet mogelijk een langdurige, oncomfortabele rouwperiode doorstaan na
het verlies van zijn eerbiedwaardige vader, waarbij zelfs Confucius
waarschuwde dat rouwrituelen te ver zouden kunnen gaan. Toch zien de
volgelingen van deze geloofssystemen hun voorschriften als een geringe
opoffering voor de voordelen van een gedeeld en samenhangend
wereldbeeld, waarin ieder individu een vaste plaats heeft. Zo kan een
orthodoxe jood stellen dat het in acht nemen van de sabbat een kleine
opoffering is in ruil voor de kracht van de wet en de saamhorigheid
binnen de joodse familie. John Locke en de vroege vrijheidsfilosofen
streefden naar een gedeelde moraal in een tolerante samenleving. Zij
waren ervan overtuigd dat elke maatschappij, ongeacht haar aard, regels
nodig heeft, maar wel dat regels gebaseerd moeten zijn op de hoogste
rationaliteit en dat dwang slechts mag gelden voor de noodzakelijke
regels. Ze erkenden dat enige dwang in de sociale moraal onvermijdelijk
is, zeker ter bescherming van leven en eigendom, omdat geen enkele
samenleving kan voortbestaan zonder veiligheid. Tegelijkertijd pleitten
zij voor vrijwel absolute tolerantie betreffende persoonlijke keuzes die
het welzijn van anderen niet raken. Een confuciaan die veertig dagen om
zijn vader rouwt, zou naast een jood kunnen leven die de sabbat eert,
zonder elkaar in de weg te lopen of de ander te dwingen de eigen
religieuze gebruiken over te nemen.

Uit deze gecombineerde leer van sociale moraliteit in essentiële zaken
en tolerantie in persoonlijke keuzes, ontstaat een kernmoraal die aan
alle burgers moet worden opgelegd, en een vrijwillige ethiek die burgers
als individu of als lid van subgroepen accepteren. Wanneer een
benedictijnse monnik geloften van armoede, kuisheid en gehoorzaamheid
aflegt, doet hij dat als lid van zijn kloostergemeenschap. Hij verwacht
niet dat alle katholieken, laat staan al zijn medeburgers, dezelfde
geloften afleggen of dezelfde regels naleven. Hij volgt de bevelen van
zijn abt, maar gaat er niet van uit dat iemand buiten het klooster hier
aandacht aan besteedt. Het naleven van minder essentiële aspecten van de
moraal kan vrijwillig blijven, maar de kernwaarden moet iedereen delen.
Wie die fundamentele moraal weigert, schaadt zowel zichzelf als de
samenleving. In het uiterste geval biedt een maatschappij die overspoeld
wordt door meedogenloze, moordzuchtige rovers, zoals in grote delen van
Europa na de val van het Romeinse Rijk, niemand een bevredigend bestaan,
zelfs de rovers zelf niet, omdat zij voortdurend onder de druk van
andere moordenaars komen te staan. Dit scenario doet zich ook in
bepaalde stedelijke gebieden in de huidige Verenigde Staten voor.
Anarchie vormt immers geen ideale maatschappij, want zonder handhaving
van de wet ontbreekt de menselijke veiligheid.

Als we kijken naar de krachten die de moraal van de samenleving
ondermijnen, moeten we de kernmoraal als uitgangspunt nemen, een moraal
die in de meeste moderne religies globaal min of meer hetzelfde is. Ten
minste twee van de Tien Geboden, namelijk `Gij zult niet doden' en `Gij
zult niet stelen', gelden als universele normen, zowel voor christenen
als voor joden. Zelfs bijna alle serieuze agnosten zien zowel moord als
diefstal, de ultieme bedreigingen voor leven en eigendom, als
ontoelaatbaar en erkennen dat de samenleving het recht heeft om hen die
beroven of vermoorden te straffen. Ze zijn het wel eens oneens over de
redelijke strafmaat voor een bepaalde misdaad, maar niet over het recht
van de samenleving om deze te bestraffen.

De originele frase van John Locke zegt het treffend. Ieder mens heeft
recht op ``leven, vrijheid en bezit.'' In 1776 voegde Thomas Jefferson
hieraan toe: ``het nastreven van geluk.'' Hoewel dit een mooie en nobele
uitdrukking is, spreken ``leven, vrijheid en bezit'' meer over de
praktische basis dan ``het nastreven van geluk.'' De maatschappij is
immers in hoge mate afhankelijk van de rechten op leven en eigendom. De
geschiedenis leert ons dat deze fundamentele rechten uitsluitend in een
vrije samenleving beschermd kunnen worden. Wanneer de staat almachtig
is, wordt zij de grootste vijand van het leven, zoals in
agressie-oorlogen, en van individueel eigendom, doordat zij een
onevenredig deel van de nationale welvaart opeist voor eigen, vaak
ongewenste en verspillende doeleinden.

De kernmoraal krijgt in de meest ontwikkelde landen echter flinke
kritiek, mede dankzij de krachten van de moderniteit die deze naties hun
technologische voordeel geven. De Verenigde Staten vormen 's werelds
leidende technologische macht. Tot ver in de vroege jaren zestig
beschouwden velen, waaronder de meeste Amerikanen, het land als een
moreel voorbeeld voor de rest van de wereld. Vandaag de dag komt zo'n
oordeel zelden meer naar voren, zelfs niet onder Amerikanen die trots op
hun land zijn. Het was voor velen van hen, net als voor de rest van de
wereld, onmogelijk om het proces tegen O.J. Simpson aan te horen en
tegelijkertijd de VS te zien als de eenvoudige, deugdzame republiek die
het aan het begin was.

Als we kijken naar de benamingen uit het oude Amerika, blijkt dat ze de
noden van een grensmaatschappij weerspiegelden en zelfs in de grote
steden de houding van de burgers bepaalden. De grensgebieden ademen een
intrinsiek democratische sfeer uit: mensen voelen zich gelijkwaardig en
de vroege Amerikanen wierpen de Europese klassehiërarchieën overboord.
Zelfs dienstmannen, die vanuit Engeland als gevangenen naar Amerika
werden gestuurd, vestigden zich als zelfstandige ambachtslieden, boeren
of vrije arbeiders zodra hun contract afliep. De lonen waren hoger dan
in Europa en de kosten van levensonderhoud laag, al waren geïmporteerde
productiegoederen prijzig. Mensen waren in de grensgebieden sterk op
elkaar aangewezen. Ondanks de hardheid van het bestaan kende het leven
voorspoed, volgens Europese standaarden. Immigranten begonnen vaak als
laagbetaalde arbeiders in de sloppenwijken van Boston of New York, maar
wisten daar doorgaans al snel aan te ontsnappen en generatie na
generatie vond voorspoed. Na de burgeroorlog zagen zwarte Amerikanen
zich als een andere van de groepen immigranten, en velen omarmden de
Amerikaanse waarden en idealen. Hieruit groeide uiteindelijk de zwarte
middenklasse.

Die ambitie, gesterkt door de ervaringen aan de grens en de invloed van
zowel protestantse als katholieke kerken, vormde het patriottisme van de
Amerikanen. Ze waren ervan overtuigd dat ze in Gods eigen land woonden,
een opvatting die volledig werd gevoed door democratische idealen en het
christelijke geloof en die leidde tot 's werelds eerste en meest
succesvolle democratie. Dat beeld kennen we allemaal. Het komt tot
uiting in de bijna iconische voorstelling van Abraham Lincoln, hoewel
sommigen in het zuiden hem nog steeds zien als de man die de gruwelen
van de eerste moderne oorlog ontketende om te voorkomen dat vrije staten
een unie verlieten die ze niet langer vertrouwden.

Toch blijft het beeld van Lincoln, ruig, eenvoudig, eerlijk en
welbespraakt, het hoogste Amerikaanse ideaal en dient hij in wezen als
moreel voorbeeld. Veel Amerikanen ervaren nog steeds de krachtige
oorspronkelijke tegenstelling tussen de democratische energie van het
nieuwe land en de versleten hiërarchieën uit Europa. Buitenlanders zien
dit ideaal van een intrinsiek dynamische meritocratie nauwelijks terug
in hedendaagse steden als Los Angeles, New York, Houston of Washington,
hoewel er in de grote voorstedelijke gebieden en op het platteland nog
meer dan alleen sporen van te vinden zijn. De Amerikaanse puriteinse
ethiek, met al haar historische betekenis, houdt het best stand ten
noorden van de sneeuwgrens, terwijl het ondernemende dynamisme wijder
verspreid is.

Veel Amerikanen wijzen op het verval van de grote steden, die
broedplaatsen zijn geworden voor criminaliteit, met name in verband met
de narco-industrie, als het meest ernstige symptoom van een afnemend
gemeenschappelijk moreel besef. Ook erkennen zij dat er een botsing
plaatsvindt tussen uiteenlopende morele culturen die wedijveren om hun
gezag en aanspraken. De `politiek correcte' cultuur wijst vele, maar
niet alle, morele beginselen van de oude cultuur af. Zij legt sterke
nadruk op de rol en rechten van groepen die historisch gezien uitgebuit
werden door een dominante blanke mannencultuur en verwerpt die, hoewel
deze juist de oprichtingscultuur van de Verenigde Staten vormt.

De dominante mannencultuur in de eerste helft van de twintigste eeuw
draaide om het behoud van ``the nuclear family'', waarin de echtgenoot
en vader, zij het vaak slechts symbolisch, de leiding in huis had,
terwijl de echtgenote of moeder in de praktijk het dagelijkse beheer
voerde en haar man als nominale meester onderdanig accepteerde. Het gaf
de mannelijke baas een echte dominantie op de werkvloer, een dominantie
die de feministische beweging tot nu toe wel heeft uitgedaagd maar niet
omgekeerd. Het grote belang van het gezin, en de historisch christelijke
leer, betekenden een verbod op abortus. Vroeger beschouwde de oude
moraal abortus als een onrechtmatige doding, volstrekt ontoelaatbaar,
een visie die haar aanhangers nog steeds delen, terwijl voorstanders van
de nieuwe moraal juist het tegendeel inzien. Het Hooggerechtshof beriep
zich in \emph{Roe v. Wade} op het recht op privacy om abortus
grondwettelijk te maken, hoewel dat recht niet expliciet in de Grondwet
of haar amendementen staat en abortus eerder tot de bevoegdheid van de
staten behoorde.

Er werd gesteld dat de privacy van de vrouw ook het recht inhield om te
bepalen of zij al dan niet kinderen wilde krijgen, ongeacht de gevolgen
van het embryo. Het Hooggerechtshof achtte het embryo namelijk niet als
drager van grondwettelijke rechten, in de late twintigste eeuw
beschouwde men embryo's als extraconstitutionele entiteiten,
vergelijkbaar met de positie van slaven in de eerste helft van de
negentiende eeuw. ``Leven, vrijheid en de zoektocht naar geluk'' gold
immers niet voor slaven, en de rechters in \emph{Roe v. Wade} pasten
deze bewoordingen dan ook niet toe op embryo's.

Het abortusdebat is een extreem voorbeeld van het conflict tussen de
oude en de nieuwe moraal, al zijn er tal van andere domeinen waarin de
traditionele sociale organisatie uitgedaagd wordt. De traditionele
christelijke moraal, in zowel protestantse als katholieke kerken, legde
strikte normen vast voor seksualiteit: geen heteroseksuele gemeenschap
buiten of vóór het huwelijk, en geen homoseksuele relaties. Lesbianisme
kwam minder nadrukkelijk aan bod, daar men het bestaan ervan nauwelijks
erkende. Toen koningin Victoria er voor het eerst van hoorde, weigerde
zij koppig te geloven dat vrouwen zulke relaties zouden hebben.
Politieke correctheid vertegenwoordigt de moraal van zogenaamd
onderdrukte groepen. Homoseksuelen eisten gelijke erkenning voor hun
levensstijl en tartten de traditionele afkeuring van hun seksuele
gedrag. Men bestempelde ``homofobie'' als een verwerpelijk vooroordeel,
vergelijkbaar met rassendiscriminatie. Kritiek hebben op homo's wordt
door de nieuwe moraal net zo onaanvaardbaar geacht als kritiek hebben op
zwarten, joden of vrouwen.

Tegelijkertijd verdwenen andere seksuele taboes juist, of werden ze
afgeschaft. In de jaren zestig ontwaakte een nieuwe golf van vrije
liefde, deels dankzij de vermeende betrouwbaarheid van de
anticonceptiepil voor vrouwen, maar ook gestimuleerd door
stemmingsveranderende drugs en popmuziek. Hierdoor werd samenwonen
buiten het huwelijk steeds gangbaarder. Tegen de jaren negentig
beschouwde men in Groot-Brittannië, een samenleving die vergeleken met
de meeste delen van de Verenigde Staten wat ouderwets was, als volstrekt
normaal dat prins Edward in Buckingham Palace met zijn vriendin sliep in
een stabiele, ongehuwde relatie, vergelijkbaar met de manier waarop
studenten in de jaren zestig in hun studentenkamers samenleefden. Weinig
mensen vonden het raar dat koningin Elizabeth II, hoofd van de
\emph{Church of England}, het gedrag van haar jongste zoon tolereerde,
terwijl de huwelijken van haar drie oudste kinderen al waren verbroken.
Wie hierover klaagde, noemde men ouderwets en betweterig, terwijl velen
nog steeds de traditionele moraal prefereerden, ook al leefden zij er
niet naar en verwachtten zij niet dat hun kinderen dat vanaf jonge
leeftijd zouden doen.

De politiek correcte beweging had een puriteinse kant. Doordat deze
beweging voortkwam uit de vermeende belangen van vrouwen, beschouwd als
de meest onderdrukte groep, ontwikkelde zij een uitgesproken
vijandigheid tegenover de mannelijke seksualiteit, zowel tegenover
agressieve uitingen als tegen gedragingen die men voorheen als
onschadelijk beschouwde. Sommige vrouwen waren ervan overtuigd dat alle
mannen van nature verkrachters zijn, waardoor de natuurlijke afschuw
voor verkrachting uitmondde in een algemene veroordeling van het
mannelijk geslacht. Anderen legden de nadruk op seksuele intimidatie, op
zich terecht, want veel mannen gedragen zich grof, maar bij onbeduidende
gevallen werd het soms lachwekkend. Sommigen beweerden zelfs dat een
enkele blik al als seksuele intimidatie kon worden aangemerkt, zonder
dat iemand ook maar één woord zei of fysiek contact maakte. Hierdoor
kwam de nieuwe moraal soms over als bijzonder censurerend. Blanke mensen
konden beschuldigd worden van raciale vooroordelen, niet omdat zij per
se bevooroordeeld waren, maar simpelweg vanwege hun huidskleur. Mannen
konden voor seksuele intimdatie worden beschuldigd, alleen maar omdat ze
een vrouw aantrekkelijk vinden, iets dat in voorgaande generaties als
een compliment in plaats van als belediging werd beschouwd.

De politiek correcte groepen en de fundamentalistische christenen zijn
erg kritisch ten opzichte van elkaar, maar in de moderne wereld vertonen
zij opvallend veel overeenkomsten. Beide kampen vinden de autoriteit van
hun eigen morele doctrine universeel, ook al verschillen hun morele
doctrines wezenlijk. Ze hebben beiden een overdreven, zelfverzekerde
moraal die vaak ontbreekt aan diepgang, historisch perspectief en
tolerantie. Beiden worden aangevallen vanwege hun vermeende gelijkenis
met het puritanisme van de zeventiende eeuw, met zelfverzekerde
moralisten als Oliver Cromwell in Engeland, die bijna naar New-England
emigreerde, of met de heksenjagers van Salem. Noch de vrouwenbeweging,
in haar meer dogmatische vorm, noch de conservatieve predikanten van de
\emph{Bible Belt} kunnen worden verweten een gebrek aan moraal te
hebben. Integendeel, hun moraal blijkt juist overontwikkeld en
inflexibel. Soms lijkt het hart van hun morele overtuigingen tot steen
geworden te zijn. Zo'n verharding schaadt de gemeenschappelijke moraal
in de samenleving net zozeer als de `alles mag'-anarchie waartegen zij
zich verzetten.

Het vervormt morele energie tot zelfingenomen morele superioriteit. Het
farizeïsme, het onwrikbare geloof in de eigen deugd, is zo oud als de
mensheid en werd door Jezus Christus krachtig veroordeeld. Daartegenover
staat de recente opvatting dat ethische keuzes louter een kwestie van
persoonlijke voorkeur zijn, net zo individueel als de keuze van kleding.
Dit standpunt weerspiegelt het ontbreken van een gedeelde moraal,
herdefinieert de klassieke vrijheidsleer en verandert ``het nastreven
van geluk'', zoals John Locke dat oorspronkelijk bedoelde en Jefferson
in 1776 verwoorde, in een hedonistisch streven dat roekeloos omgaat met
de gevolgen.

De uitdrukking ``het nastreven van geluk'' vindt zijn oorsprong in John
Locke's \emph{Essay on Human Understanding} (1691). Daarin schrijft hij:
``De hoogste volmaaktheid van de intellectuele natuur ligt in een
zorgvuldige zoektocht naar waar en solide geluk, dus de zorg voor
onszelf, opdat wij het denkbeeldige niet verwarren met het reële, is de
noodzakelijke basis van onze vrijheid.'' Vervolgens merkt hij op: ``Niet
iedereen vindt zijn geluk in hetzelfde\ldots{} de geest heeft een andere
smaak dan het gehemelte\ldots{} Mensen mogen voor verschillende dingen
kiezen, maar allen maken de juiste keuze, indien men hen beschouwt als
een gezelschap arme insecten, waarvan sommigen bijen zijn, die zich
verheugen op bloemen en hun zoetheid, en anderen kevers, die genieten
van andere soorten voedsel.'' Toch betoogt hij dat het verkiezen van
ondeugd boven deugd ``onmiskenbaar een foutieve afweging is.'' Hij legt
daarbij bijzondere nadruk op religieuze argumenten en stelt dat de
``slechten het het zwaarst hebben.'' Hij is ervan overtuigd dat
``wanneer een moraal op degelijke fundamenten rust, deze de keuzes van
iedereen die erover nadenkt, onvermijdelijk zal bepalen.''

De Lockeaanse doctrine van vrijheid biedt ongetwijfeld meer ruimte voor
individuele voorkeuren dan autoritaire morele systemen die trachten
iedereen gelijk te behandelen en uniform gedrag af te dwingen. Toch
erkent de klassieke vrijheidsleer al snel de noodzaak van collectieve
morele imperatieven, zoals respect voor medemensen, in het bijzonder
voor hun leven en het vreedzaam bezit van hun eigendommen volgens de
wet. Erosie van de gedeelde moraal vormt een bedreiging voor de
vrijheid, enerzijds doordat het een element van anarchie introduceert en
anderzijds doordat het autoritaire krachten in de hand werkt. De
geschiedenis van de publieke moraal verloopt in een cyclus van wanorde
en autoritarisme; hedendaagse autoritaire morele stromingen, zoals het
feminisme en het fundamentalisme, zijn ontstaan als reactie op het
hedonisme van de jaren zestig.

We hebben al enkele kenmerken geschetst van de wereld van de komende
eeuw. Deze zal worden vormgegeven door twee primaire krachten: de
technologische verschuiving die de Aziatische economieën opent en de
opkomst van wereldwijde elektronische communicatiemiddelen die burgers
steeds minder afhankelijk maken van de lokale overheid. Deze nieuwe
technologie vervangt veel middelmatige menselijke vaardigheden, zoals
die van de productielijnarbeider, de kantoormedewerker en steeds vaker
de middenmanager, of heeft dat al gedaan. Tegelijkertijd beloont het
zeldzamere competenties en creëert het een internationale cognitieve
elite van extreem bekwame mensen, voor wie de nieuwe
communicatiemiddelen een zo breed mogelijke markt openen voor hun
talenten. Net als de meeste elites beschouwt deze cognitieve elite
zichzelf vaak als superieur, gaat ze arrogant te werk en meent ze haar
eigen normen te mogen stellen. Daardoor vervreemdt ze zich van de
samenleving.

Tijdens de eerste helft van de volgende eeuw vindt een enorme overdracht
van rijkdom plaats van het oude Westen naar het nieuwe Oosten. Politiek
falen, en de politieke achtergesteldheid van China, kunnen deze
verschuiving wellicht vertragen, maar zullen haar vrijwel zeker niet
tegenhouden. Ze kunnen het proces niet terugdraaien.

Deze rijkdomsoverdracht legt ongetwijfeld de grootste druk op de
voornamelijk blanke landen op het noordelijk halfrond, zoals in Europa
en Noord-Amerika. Momenteel telt dit gebied ongeveer 750 miljoen
inwoners in de ontwikkelde landen. Tot voor kort was Japan het enige
Aziatische, niet-blanke land dat de Euro-Amerikaanse levensstandaard had
bereikt, al waren er etnisch Europese bevolkingsgroepen in
Nieuw-Zeeland, Australië en in de blanke gemeenschappen van Zuidelijk
Afrika. In 1990 behoorden de geavanceerde industriële landen naar
schatting slechts tot 15 procent van de wereldbevolking van 5 miljard.
Het wereldwijde vermogen was als volgt verdeeld: ongeveer 15 procent
rijken en 85 procent armen, een verhouding die sterk overeenkomt met de
inkomensverdeling in de geavanceerde industriële samenlevingen van een
eeuw geleden. Tegen 2050 wordt verwacht dat de geavanceerde economieën
ongeveer 3 miljard van de wereldwijde 7 miljard mensen zullen omvatten,
een situatie die neerkomt op een vermogensverdeling van 40 procent
rijken en 60 procent armen. Tegen het einde van de eeuw kunnen de
aantallen zelfs omkeren, met 60 procent rijken en 40 procent armen,
waarbij armoede vooral geconcentreerd blijft in Afrika. Hoewel deze
verschuiving een grotere gelijkheid in rijkdom tussen naties bevordert,
zal de ongelijkheid binnen landen waarschijnlijk juist toenemen. Wie
zijn talent en kapitaal efficiënt benut, krijgt een aanzienlijk voordeel
ten opzichte van mensen met gemiddelde vaardigheden of weinig middelen.
Bovendien beweegt deze welvaart zich uiterst mobiel. De armen in de
ontwikkelde landen zullen de rijken niet langer kunnen belasten zoals in
de twintigste eeuw. Landen die dat toch proberen, lopen het risico
achterop te raken in een felle, competitieve race.

Natuurlijk zal de totale productiviteit van de wereldeconomie blijven
groeien, waarschijnlijk met gemiddeld 3 procent wereldwijd, mits er geen
wereldoorlogen uitbreken. Als dat standhoudt, verdubbelt de totale
wereldproductie elke vijfentwintig jaar, waardoor het in 2050 meer dan
vier keer zo hoog ligt als nu en tegen het jaar 2100 zestien tot twintig
keer zo groot zal zijn. Zelfs als de wereldbevolking tegen 2100 de 8
miljard heeft bereikt, resulteert dat per hoofd in een wereld-BBP dat
aan het einde van de eeuw tien keer het huidige niveau bedraagt. Een
dergelijke toename van rijkdom kan zowel de opkomst van nieuwe
industriële samenlevingen als de miljoeneninkomens van de cognitieve
elite opvangen, terwijl het tegelijkertijd voor de overige
arbeidskrachten in de ontwikkelde landen een stijgende en fatsoenlijke
levensstandaard waarborgt. De inkomensverschillen zullen daarbij echter
radicaal anders zijn dan in de twintigste eeuw. Op wereldschaal zullen
de inkomens in arme landen veel sneller groeien dan in rijke landen,
terwijl nationaal gezien juist de inkomens van de rijke, zoals in het
Amerika van de jaren negentig, veel sneller zullen stijgen dan die van
de midden- en lagere inkomenslagen. In de komende eeuw zullen we getuige
zijn van de opkomst van een wereldsuperklasse, mogelijk bestaande uit
500 miljoen zeer rijke mensen, waarvan er 100 miljoen rijk genoeg zullen
zijn om zich te ontpoppen als Soevereine Individuen.

Dit proces heeft onvermijdelijk consequenties. Samenlevingen zullen veel
minder homogeen worden, de natiestaat zal verzwakken of zelfs volledig
uiteenvallen, en de cognitieve elite zal zichzelf als kosmopolitisch
gaan beschouwen. Mensen die wereldwijd in vergelijkbare functies werken,
ontwikkelen immers een cultuur die veel meer overeenkomt met die van hun
collega's in andere delen van de wereld dan met die van hun medeburgers
in traditionele natiestaten. Een investeringsbankier uit Londen voelt
zich waarschijnlijk meer thuis in Seoel dan in Glasgow, en een ambtenaar
uit Washington voelt zich wellicht prettiger in Bonn dan in de zwarte
wijken van Washington zelf. We zien nu al hoe dit proces de morele
waarden versplintert. De moraal van een individu wordt immers deels
bepaald door zijn opvoeding, door wat iemand als kind geleerd heeft, en
deels door zijn levenservaringen. Zowel de scholing als de ervaringen
binnen de cognitieve elite hebben een duidelijke kosmopolitische inslag,
die mensen losmaakt van hun lokale gemeenschappen.

Naarmate de volgende eeuw dichter nadert, valt op dat veel
vertegenwoordigers van de groeiende cognitieve elite nauwelijks enige
religieuze of morele vorming binnen het gezin hebben meegekregen. De
overheersende levensbeschouwing in deze kringen is agnostisch humanisme.
Bovendien zien we dat deze families vaak verscheurd worden door
echtscheidingen en hertrouw, met daaropvolgende derde huwelijken. Hoewel
het huwelijksmodel dat in Hollywood heerst niet representatief is voor
de gehele Verenigde Staten, kampt de cognitieve elite in Euro-Amerika
met een hoog percentage echtscheidingen, gemiddeld wel een derde of
meer. Kinderen van gescheiden ouders krijgen zelden een fundamentele
religieuze vorming mee en raken al vroeg bewust van de uiteenlopende
morele opvattingen die hun ouders, stiefouders, en stiefbroers en
-zussen hanteren. Vergelijk je deze initiële morele vorming met die in
een traditioneel Iers of Pools dorp, dan valt op dat de boerenopvoeding
een veel sterkere religieuze basis biedt. Een elite die goddeloos,
wortelloos en welvarend is, zal waarschijnlijk niet gelukkig of geliefd
zijn.

Dit tekort in de morele vorming van wat de dominante economische groep
van de volgende eeuw zal worden, zal waarschijnlijk nog versterkt worden
door hun levenservaring. Deze mensen zullen een gedisciplineerde,
geavanceerde technische opleiding hebben gevolgd om zich voor te
bereiden op hun rol als leiders van het nieuwe elektronische universum.
Maar daaruit leren ze slechts enkele van de morele lessen die historisch
de basis vormden voor menselijk sociaal gedrag. Gemeten naar de
maatstaven van Confucius, Boeddha of Plato (500 v.Chr.), Paulus (50
n.Chr.) of Mohammed (600 n.Chr.), zullen ze moreel analfabeet zijn. Ze
leren economische efficiëntie, optimaal gebruik van middelen en het
najagen van geld, maar niet de deugden van nederigheid of
zelfopoffering, laat staan kuisheid. In wezen zullen de meesten van hen
als heidenen zijn opgevoed, met waarden die dichter liggen bij die van
de late Romeinse Republiek dan bij het christendom. Zelfs die waarden
zullen sterk individualistisch zijn, eerder persoonlijk dan
gemeenschappelijk. Samenlevingen kunnen echter alleen sterk zijn als
echte morele waarden breed gedeeld worden. De ontwikkelde landen bewegen
zich juist richting een toestand waarin veel mensen zwakke of beperkte
morele overtuigingen hebben, anderen dat compenseren met fanatieke trouw
aan irrationele waarden, en er nog maar weinig waarden overblijven die
door de hele samenleving worden gedeeld. Sommige van de eerder
beschreven ``concurrerende territoriale clubs'' zullen ongetwijfeld
strenge morele normen opleggen voor wie er mag wonen.

Verschillen in rijkdom hebben historisch gezien nooit op zichzelf geleid
tot fundamentele verschillen in religieuze waarden. In dichte, stabiele
samenlevingen met sterke tradities en een uitgesproken hiërarchie, ``de
rijke in zijn kasteel, de arme aan zijn poort'', kunnen er waarden
bestaan die door alle lagen heen lopen. Maar dat hangt af van de kracht
van het gemeenschapsgevoel tussen rijk en arm, en van de vitaliteit van
de sociale tradities. Beide voorwaarden ontbreken vandaag, en zowel
gemeenschap als traditie worden verzwakt door de economische en
technologische revolutie die gaande is. Het leven van de velen en dat
van de enkelen raakt steeds verder van elkaar verwijderd. De
technologische revolutie is juist tot stand gekomen door te breken met
oude manieren van doen. In elk domein is het de radicaal die wint, en de
conventionele denker die achterblijft, die letterlijk uit de race valt.
Onze politiek mag dan nog geleid worden door conventionele denkers,
zoals Bill Clinton, Helmut Kohl en John Major, maar onze succesvolste
bedrijven staan onder leiding van radicalen met een scherp inzicht in de
nieuwe technologische wereld. Het archetype hiervan is Bill Gates.
Conventioneel denken is in diskrediet geraakt door het onvermogen ervan
om om te gaan met de snelheid en de kracht van verandering.

Toch werkt moraal niet zo. Als we de wetenschap van Mozes nemen, gevormd
rond 1000 v.Chr., heeft die ons weinig te zeggen. Het scheppingsverhaal
in Genesis mag theologisch waar zijn, maar het biedt geen
wetenschappelijke verklaring voor het ontstaan van het universum en de
mensheid, maar de moraal van Mozes, de Tien Geboden, heeft ons veel te
bieden.

Eerbied voor ouders en trouw in het huwelijk behouden het gezinsleven.
Een sterk gezinsleven vormt moreel gezonde kinderen. Diefstal schaadt
zowel de dader als het slachtoffer en ontmoedigt arbeid en sparen. De
maatschappelijke orde steunt op waarheidsgetrouwe getuigenissen. Moord
is verkeerd, en zo verder.

De afgelopen drieduizend jaar heeft op wetenschappelijk vlak alles
veranderd, maar op vlak van moraal zijn we mogelijk zelfs achteruit
gegaan. De gemiddelde psychotherapeut geeft slechter moreel advies dan
een Joodse leraar in de tijd van Mozes. Het christendom bestaat nog,
maar is een schim geworden van zijn vroegere gedaante. Weinig mensen
hebben nog het geloof van vroegere tijden, of zelfs van eenvoudigere
gemeenschappen; je zoekt geen heiligen op Park Avenue.

De vernietiging van traditie was een noodzakelijke voorwaarde voor
wetenschappelijke vooruitgang. Als men nog steeds geloofde dat de zon om
de aarde draaide, zou satellietcommunicatie onmogelijk zijn. Wetenschap
is immers slechts een reeks hypothesen, imperfecte verklaringen die door
andere verklaringen vervangen zullen worden die beter zijn, maar nooit
volmaakt. Toch is het verlies van traditie rampzalig geweest voor de
wereldwijde morele orde.

Confucius predikte gematigdheid in gedrag (Hij noemde de Gulden
Middenweg ``chum yum'', ten minste volgens vertalers uit de zeventiende
eeuw). Hij leerde ons ook autoriteit te respecteren en anderen te
behandelen zoals je zelf behandeld wilt worden. Die leer is 2500 jaar
oud en bepaalde eeuwenlang de Chinese beschaving, maar veel moderne
Chinezen achten Confucianisme ouderwets. Ze zien geen waarde in
matiging, waarderen macht boven gezag en behandelen anderen al helemaal
niet zoals ze zelf behandeld zouden willen worden. Met het verlies van
traditie verliest een samenleving ook de vocabulaire van haar morele
consensus. Zo is China, ondanks haar groeiende macht, moreel
achterlijker dan Tibet, hoe arm en onderdrukt ze daar ook zijn.

Een goede sociale moraal heeft bepaalde kenmerken. Het moet bijdragen
aan het voortbestaan van zowel de samenleving als het individu, op een
dynamische in plaats van een statische manier. Het moet verdraagzaamheid
bevorderen en zelfgenoegzaamheid vermijden. Het moet religieus zijn,
niet slechts agnostisch. Het moet zich niet bezighouden met
wetenschappelijke feiten. Het mag noch anarchistisch, noch autoritair
zijn. Het moet breed gedeeld en diep verankerd zijn. Zo'n moraal is
vooral van belang voor het gezin en voor de opvoeding van kinderen tot
zelfstandige en verantwoordelijke volwassenen. Het vormt de kern van een
gezonde samenleving.

Een dergelijke moraal wordt ondersteund door de logica van onderlinge
afhankelijkheid die voortkomt uit handel en gevoelens van
medemenselijkheid, maar wordt bedreigd door oppervlakkig sciëntisme,
door de vervreemding van een superklasse en een onderklasse, en door het
verlies van de verankering van de oude geografische economieën.
Misschien zal er een reactie komen tegen deze ontwikkelingen, die als
uiterst gevaarlijk beschouwd moeten worden voor de samenlevingen van de
volgende eeuw.

Nu de ``meest verschrikkelijke eeuw in de westerse geschiedenis,'' zoals
Isaiah Berlin het noemde, ten einde loopt, komt ook het tijdperk van
sociaal gigantisme ten einde. De laatste dagen van de twintigste eeuw
luiden een periode in van verkleining, decentralisatie en
herstructurering, de tijd van de sociale dinosaurussen die vastzitten in
de teerput, en van aaseters die hun botten zullen afkluiven. Overheden,
bedrijven en vakbonden zullen zich moeten aanpassen aan nieuwe
metaconstitutionele omstandigheden, gevormd door de opkomst van
microtechnologie, dat de grenzen van geweldsuitoefening ingrijpend heeft
verschoven. De wereld van vandaag is al veel ingrijpender veranderd dan
de media beseffen, en precies in de richting die wordt voorspeld door de
studie van megapolitieke omstandigheden. Zoals eerder betoogd in
\emph{Blood in the Streets} en \emph{The Great Reckoning}, wanneer
technologie en andere factoren de grenzen van machtsuitoefening
veranderen, verandert onvermijdelijk ook het karakter van de
samenleving. Alles wat te maken heeft met menselijke interactie,
inclusief moraal en gezond verstand, zal mee veranderen. Na een periode
van moreel verval, kenmerkend voor het einde van een tijdperk, volgt
onvermijdelijk de opkomst van een strengere moraal, met hogere eisen,
passend bij een wereld vol concurrerende soevereiniteit.

Enkele trekken van die nieuwe moraal zijn te voorzien. Ze zal het belang
benadrukken van productiviteit en het recht dat verdiensten toekomen aan
wie ze genereert. Ook efficiëntie in investeringen zal een moreel
principe worden. De moraal van het Informatietijdperk prijst efficiëntie
en erkent dat middelen het best worden ingezet waar ze de hoogste waarde
opleveren. Met andere woorden: de moraal van het Informatietijdperk zal
de moraal van de markt zijn. Zoals James Bennett stelt, zal die moraal
ook gebaseerd zijn op vertrouwen. De cybereconomie zal een ``high-trust
society'' zijn. Omdat onkraakbare versleuteling het mogelijk maakt voor
fraudeurs om hun buit buiten bereik te verbergen, zal er een sterke
prikkel zijn om alleen zaken te doen met betrouwbare partijen. Net als
de Quakers vroeger, zal een reputatie van eerlijkheid erg waardevol
zijn. In de anonimiteit van de cyberspace zal die reputatie niet altijd
aan een bekende persoon kleven, maar ze zal wel verifieerbaar zijn via
cryptografische sleutels. Het gevaar dat criminelen deze systemen zouden
corrumperen is zo groot, dat dit werkgevers zal aansporen om uiterst
zorgvuldig te zijn met wie ze vertrouwen. Bennett voorziet ``een
herenclub van de cyberspace'', beveiligde zones waarin deelname alleen
mogelijk is met strikte identificatie, bijvoorbeeld via stemherkenning.
De beheerders zouden kunnen borg staan voor de identiteit en
betrouwbaarheid van deelnemers, zodat transacties veiliger verlopen dan
op het open internet. Zo zou de 21e eeuw een heropleving kunnen brengen
van de Victoriaanse nadruk op karakter en betrouwbaarheid, in een
omgeving die geen enkele Victoriaan had kunnen voorstellen.

De beschermde zones van cyberspace zouden ook garanties kunnen bieden,
vergelijkbaar met de extraterritoriale bescherming die de graven van
Champagne verleenden aan handelaren die van en naar de
Champagne-jaarmarkten reisden. Andere rechtsgebieden ``vergoedden zelfs
de reizende kooplieden voor eventuele verliezen die ze zouden lijden
tijdens hun doortocht door het grondgebied onder het gezag van de
betreffende edelman.''

De ``wachters van de jaarmarkt,'' oorspronkelijk aangesteld door de
graven, zorgden voor veiligheid en een ``gerechtshof'' voor de
kooplieden op de markt. Uiteindelijk ontwikkelden zij zich tot meer
onafhankelijke instanties, met een eigen zegel, die contracten notarieel
bekrachtigden en de naleving ervan afdwongen. Zij hadden de bevoegdheid
om ``handelaren die schuldig bevonden waren aan het niet betalen van hun
schulden of het niet nakomen van hun contractuele verplichtingen uit te
sluiten van toekomstige jaarmarkten. Dit was kennelijk zo'n zware straf
dat maar weinigen het risico namen deze kans op toekomstige winst te
verliezen. Als lichtere maatregel konden de wachters de goederen van een
wanbetalende schuldenaar in beslag nemen en verkopen ten voordele van
zijn schuldeisers.''

Uitsluiting als middel om naleving van contracten af te dwingen verloor
aan belang toen het aantal alternatieve markten toenam. Met de nieuwe
informatietechnologie die nu beschikbaar is, zou het weren van
bedriegers en wanbetalers echter opnieuw een krachtig
handhavingsmechanisme kunnen worden binnen de gefragmenteerde
soevereiniteiten van de volgende maatschappelijke fase.
Computernetwerken kunnen de cyberspace bewaken met onvervalsbare
informatie over krediet en fraude. In die zin zal de wereld een kleine
gemeenschap vormen, waardoor bedrog en oplichting worden ontmoedigd.

Naast de nadruk op de moraal van verdiensten en efficiëntie en de
hernieuwde focus op karakter en betrouwbaarheid, zal de nieuwe moraal
waarschijnlijk ook het kwaad van geweld benadrukken, in het bijzonder
van ontvoering en afpersing, middelen die aantrekkelijker zullen worden
om individuen met ontoegankelijke middelen alsnog af te persen. Nog een
waarschijnlijke prikkel tot een strengere moraal zal het einde zijn van
uitkeringen en inkomensherverdeling. Wanneer hulp aan achterblijvers
vooral afhangt van de vrijgevigheid van particulieren en
liefdadigheidsorganisaties, zal het belangrijker zijn dan in de
twintigste eeuw dat de ontvangers van die hulp moreel waardig lijken in
de ogen van degenen die hen vrijwillig helpen.

\begin{quote}
Subsidies, meevallers en de verwachting van economische kansen
verminderen de directe noodzaak om te sparen. De mantra's van
democratie, herverdeling en economische ontwikkeling verhogen
verwachtingen en geboortecijfers, stimuleren bevolkingsgroei en
verdiepen daarmee de neerwaartse ecologische en economische spiraal. --
VIRGINIA ABERNATHY
\end{quote}

Op bepaalde terreinen zal de nieuwe informatiewereld beter in staat zijn
om morele vraagstukken serieus aan te pakken. De herverdelingsbeloften
die hoop wekten bij de pechvogels en mislukkingen in de VS, Canada en
West-Europa, hebben internationaal ook een pervers effect gehad. Er is
overtuigend bewijs dat buitenlandse hulp en interventiebeloftes, bedoeld
om hongersnood te voorkomen en de levensstandaard te verhogen, er in
belangrijke mate voor heeft gezorgd dat de bevolkingsgroei tot zelfs
voorbij de draagkracht van achterblijvende economieën is toegenomen. De
opzienbarende groei van de wereldbevolking sinds de Tweede Wereldoorlog,
met de vaak verwoestende gevolgen voor bossen, bodems en watervoorraden,
vloeit voort uit wereldwijde interventies die de negatieve
terugkoppelingsmechanismen, waardoor lokale bevolkingen en de
beschikbare hulpbronnen lange tijd in evenwicht werden gehouden, teniet
hebben gedaan.

Degenen die in kleine gemeenschappen leefden met weinig middelen en
amper economische groei, waren natuurlijk zeer tevreden dat de
beperkingen van hun dorpsleven zouden verdwijnen. Zij namen met
enthousiasme de positieve boodschap over, verkondigd door internationale
hulpverleners, vrijwilligers van het Peace Corps, lokale revolutionairen
en de strijdende ideologen tijdens de Koude Oorlog, die iedereen
verzekerden dat er betere tijden in het verschiet lagen. Dit bleek
echter precies de verkeerde boodschap te zijn.

Een belangrijk gevolg van de herverdeling tussen verschillende culturen
is dat mensen die in niet-geïndustrialiseerde samenlevingen leefden en
trouw bleven aan traditionele waarden, kunstmatig competitief werden.
Internationale hulp, reddingsoperaties om hongersnood en epidemieën te
bestrijden en technologische interventies hebben velen voor de gek
gehouden en doen geloven dat hun vooruitzichten drastisch waren
verbeterd, zonder dat ze hun waarden hoefden aan te passen of hun gedrag
wezenlijk te veranderen.

Internationale inkomensherverdeling heeft niet alleen geleid tot een
onhoudbare bevolkingsgroei wereldwijd, maar heeft ook in belangrijke
mate bijgedragen aan cultureel relativisme en tot wijdverspreide
verwarring over de cruciale rol die cultuur speelt in het
aanpassingsvermogen van mensen aan hun leefomgeving. Vandaag de dag ziet
men cultuur vooral als een kwestie van voorkeur, niet als een gids voor
gedrag die zowel kan informeren als misleiden. We nemen te snel aan dat
alle culturen gelijkwaardig zijn en herkennen de nadelen van
contraproductieve cultuurvormen te laat. Dit geldt vooral voor de
hybride culturen die zich in deze eeuw in verschillende delen van de
wereld ontwikkelen door subsidies en staatsinterventies. Net als de
criminele subcultuur in de Amerikaanse steden houden ze nog fragmenten
vast van oude culturen die ooit pasten bij een vroegere economische
fase, en vermengen die met moderne waarden uit het Informatietijdperk.

De Informatierevolutie zal niet alleen genialiteit tot bloei brengen,
maar ontketent ook vijandigheid. In het komende millennium zullen beide
krachten op ongekende wijze met elkaar botsen.

De overgang van een industriële naar een Informatiemaatschappij belooft
werkelijk adembenemend te worden. Elke verschuiving van het ene stadium
van economisch leven naar het volgende ging altijd gepaard met een
revolutie. Wij zijn ervan overtuigd dat de Informatierevolutie
waarschijnlijk de meest ingrijpende wordt van allemaal. Ze zal het leven
grondiger herstructureren dan zowel de agrarische als de industriële
revolutie, en de impact zal binnen een fractie van de tijd voelbaar
zijn. Riemen vast.


\backmatter


\end{document}
