% Options for packages loaded elsewhere
\PassOptionsToPackage{unicode}{hyperref}
\PassOptionsToPackage{hyphens}{url}
%
\documentclass[
  a5paper,
  smalldemyvopaper,10pt,twoside,onecolumn,openright,extrafontsizes,hidelinks]{memoir}

\usepackage{amsmath,amssymb}
\usepackage{iftex}
\ifPDFTeX
  \usepackage[T1]{fontenc}
  \usepackage[utf8]{inputenc}
  \usepackage{textcomp} % provide euro and other symbols
\else % if luatex or xetex
  \usepackage{unicode-math}
  \defaultfontfeatures{Scale=MatchLowercase}
  \defaultfontfeatures[\rmfamily]{Ligatures=TeX,Scale=1}
\fi
\usepackage{lmodern}
\ifPDFTeX\else  
    % xetex/luatex font selection
\fi
% Use upquote if available, for straight quotes in verbatim environments
\IfFileExists{upquote.sty}{\usepackage{upquote}}{}
\IfFileExists{microtype.sty}{% use microtype if available
  \usepackage[]{microtype}
  \UseMicrotypeSet[protrusion]{basicmath} % disable protrusion for tt fonts
}{}
\makeatletter
\@ifundefined{KOMAClassName}{% if non-KOMA class
  \IfFileExists{parskip.sty}{%
    \usepackage{parskip}
  }{% else
    \setlength{\parindent}{0pt}
    \setlength{\parskip}{6pt plus 2pt minus 1pt}}
}{% if KOMA class
  \KOMAoptions{parskip=half}}
\makeatother
\usepackage{xcolor}
\setlength{\emergencystretch}{3em} % prevent overfull lines
\setcounter{secnumdepth}{5}
% Make \paragraph and \subparagraph free-standing
\makeatletter
\ifx\paragraph\undefined\else
  \let\oldparagraph\paragraph
  \renewcommand{\paragraph}{
    \@ifstar
      \xxxParagraphStar
      \xxxParagraphNoStar
  }
  \newcommand{\xxxParagraphStar}[1]{\oldparagraph*{#1}\mbox{}}
  \newcommand{\xxxParagraphNoStar}[1]{\oldparagraph{#1}\mbox{}}
\fi
\ifx\subparagraph\undefined\else
  \let\oldsubparagraph\subparagraph
  \renewcommand{\subparagraph}{
    \@ifstar
      \xxxSubParagraphStar
      \xxxSubParagraphNoStar
  }
  \newcommand{\xxxSubParagraphStar}[1]{\oldsubparagraph*{#1}\mbox{}}
  \newcommand{\xxxSubParagraphNoStar}[1]{\oldsubparagraph{#1}\mbox{}}
\fi
\makeatother


\providecommand{\tightlist}{%
  \setlength{\itemsep}{0pt}\setlength{\parskip}{0pt}}\usepackage{longtable,booktabs,array}
\usepackage{calc} % for calculating minipage widths
% Correct order of tables after \paragraph or \subparagraph
\usepackage{etoolbox}
\makeatletter
\patchcmd\longtable{\par}{\if@noskipsec\mbox{}\fi\par}{}{}
\makeatother
% Allow footnotes in longtable head/foot
\IfFileExists{footnotehyper.sty}{\usepackage{footnotehyper}}{\usepackage{footnote}}
\makesavenoteenv{longtable}
\usepackage{graphicx}
\makeatletter
\def\maxwidth{\ifdim\Gin@nat@width>\linewidth\linewidth\else\Gin@nat@width\fi}
\def\maxheight{\ifdim\Gin@nat@height>\textheight\textheight\else\Gin@nat@height\fi}
\makeatother
% Scale images if necessary, so that they will not overflow the page
% margins by default, and it is still possible to overwrite the defaults
% using explicit options in \includegraphics[width, height, ...]{}
\setkeys{Gin}{width=\maxwidth,height=\maxheight,keepaspectratio}
% Set default figure placement to htbp
\makeatletter
\def\fps@figure{htbp}
\makeatother

% typographical packages
\usepackage{microtype}
\usepackage{setspace}
\tolerance=6000
\hyphenpenalty=1000

% typographical settings for the body text
\setlength{\parskip}{0em}
\setlength{\parindent}{1em}
\linespread{1}

% DEFINITIONS TITLE PAGE / COPYRIGHT
\newcommand{\titleoriginal}{The Sovereign Individual}
\newcommand{\subtitleoriginal}{Mastering the Transition to the Information Age}
\newcommand{\yearoriginal}{1999}
\newcommand{\subtitletranslation}{XXX}
\newcommand{\yeartranslation}{XXX}
\newcommand{\stringtranslation}{XXX}
\newcommand{\stringlicense}{XXX}
\newcommand{\stringpublisher}{XXX}
\newcommand{\ISBNHC}{XXX}
\newcommand{\ISBNSC}{XXX}
\newcommand{\ISBNEBOOK}{XXX}
\newcommand{\ISBNAUDIO}{XXX}
\newcommand{\press}{Konsensus Network}
\newcommand{\translatorone}{XXX}
\newcommand{\translators}{
\large\textit{\stringtranslation:}\\
\translatorone\\
}

% PHYSICAL DOCUMENT SETUP
\setstocksize{210mm}{148mm}
\settrimmedsize{210mm}{148mm}{*}
\setbinding{7mm}
\setlrmarginsandblock{15mm}{16mm}{*}
\setulmarginsandblock{16mm}{16mm}{*}
\setlength{\skip\footins}{18pt} % More space between the text and the footnote line

% FONTS
\usepackage{fontspec}
\setmainfont{stone-serif}[
    Path=./fonts/stone-serif-itc-pro/,
    Scale=0.83,
    Extension=.OTF,
    UprightFont=*-Regular,
    BoldFont=*-SemiBd,
    ItalicFont=*-MediumIt,
    BoldItalicFont=*-SemiBdIt
    ]

\setsansfont{stone-sans}[
    Path=./fonts/stone-sans/,
    Scale=0.85,
    Extension=.otf,
    UprightFont=*-Medium,
    BoldFont=*-Semibold,
    ItalicFont=*-MediumItalic,
    BoldItalicFont=*-SemiBoldItalic
    ]

\usepackage{lettrine}
\setcounter{DefaultLines}{3}
\renewcommand{\DefaultLoversize}{0.1}
\renewcommand{\DefaultLraise}{0}
\renewcommand{\LettrineTextFont}{}
\setlength{\DefaultFindent}{\fontdimen2\font}
\setlength{\DefaultNindent}{0em}

% custom second title page
\makeatletter
\newcommand*\halftitlepage{\begingroup % Misericords, T&H p 153
  \setlength\drop{0.1\textheight}
  %\begin{center}
  \vspace*{\drop}
  \rule{\textwidth}{0in}\par
  {\Large\sffamily\thetitle\par}
  \rule{\textwidth}{0in}\par
  \vfill
  %\end{center}
\endgroup}
\makeatother

% custom title page
\makeatletter
\newlength\drop
\newcommand*\titleM{\begingroup % Misericords, T&H p 153
  \setlength\drop{0.15\textheight}
  %\begin{center}
  \vspace*{\drop}
  {\HUGE\sffamily\thetitle\par}
  \vspace{2em}
  {\Large\sffamily\textit\subtitletranslation\par}
  \vspace{4em}
  \rule{5.5cm}{0.3mm}\par
  \vspace{4em}
  {\Large\sffamily\textit\theauthor\par}
  \vspace{6em}
  % {\footnotesize\sffamily\textit\translators\par}
  \vfill
  \includegraphics[width=3.5cm]{figures/knw.png}\par
  %\end{center}
\endgroup}
\makeatother

% copyright page
\makeatletter
\newcommand*\copyrightpage{\begingroup
  \setlength\drop{0.1\textheight}
  \vphantom{just for the drop}
  \vfill
  \begin{footnotesize}
  \noindent \copyright\space \yearoriginal: \theauthor
  \par\noindent \textit{\titleoriginal: \subtitleoriginal}
  \vspace{0.5\baselineskip}
  \par\noindent \copyright\space \yeartranslation\space \stringtranslation: \translatorone
  \par\noindent \textit{\thetitle: \subtitletranslation}
  \vspace{\baselineskip}
  \par\noindent \textit{\stringlicense}
  \vspace{0.5\baselineskip}
  \par\noindent \stringpublisher: \href{https://konsensus.network}{\textit{konsensus.network}}
  \vspace{0.5\baselineskip}
  \par\noindent v1.0.0
  \vspace{0.5\baselineskip}
  \setlength{\parindent}{2em}% default 20pt
  \par\noindent ISBN \ISBNHC \:Hardcover
  \par\hspace{0.28\parindent}\ISBNSC \:Paperback
  \par\hspace{0.28\parindent}\ISBNEBOOK \:E-book\par
  \setlength{\parindent}{0pt}
  \end{footnotesize}
  \vspace{3em}
  \par\noindent \href{https://konsensus.network}{\includegraphics[width=1cm]{figures/freestarfish.png}}
  \par\noindent \href{https://konsensus.network}{\includegraphics[width=3.5cm]{figures/knw.png}}
  \setcounter{footnote}{0}
  \clearpage
\endgroup}
\makeatother

% HEADER AND FOOTER MANIPULATION
% for normal pages
\nouppercaseheads
\headsep = 4mm
\makepagestyle{mystyle} 
\makeevenhead{mystyle}{\scriptsize\sffamily\mdseries\thepage}{}{}
\makeoddhead{mystyle}{\scriptsize\sffamily\mdseries\leftmark}{}{\scriptsize\sffamily\mdseries\thepage}
\makeevenfoot{mystyle}{}{}{}
\makeoddfoot{mystyle}{}{}{}
\makeatletter

% for pages where chapters begin
\makepagestyle{plain}
\makerunningwidth{plain}{\headwidth}
\makeevenfoot{plain}{}{}{}
\makeoddfoot{plain}{}{}{}
\pagestyle{mystyle}

\newif\ifmainmatter
\appto\mainmatter{\mainmattertrue}
\appto\backmatter{\mainmatterfalse}
\appto\appendix{\mainmatterfalse}

\renewcommand\chaptermark[1]{%
  \markboth{\MakeUppercase{%
    \ifmainmatter~\oldstylenums\thechapter.~\fi#1}}{}}%

% TOC
\usepackage[]{tocloft}
\renewcommand{\cftsectiondotsep}{\cftnodots}
\renewcommand{\cftpartfont}{\Large\sffamily\MakeUppercase}
\renewcommand{\cftchapterfont}{\small\sffamily}
\renewcommand{\cftsectionfont}{\Small\sffamily}
\renewcommand{\cftpartpagefont}{\Large\sffamily}
\renewcommand{\cftchapterpagefont}{\small}
\renewcommand{\cftchapterpresnum}{KAPITEL\space}
\renewcommand{\cftchapternumwidth}{7em}
\setlength{\cftchapterindent}{0em}
\setlength{\cftsectionindent}{5em}
\setlength{\cftbeforechapterskip}{0.8em}
\setsecnumdepth{chapter}
\setcounter{tocdepth}{0}


% Redefine footnote presentation
\makeatletter
\renewcommand\@makefntext[1]{%
  \noindent\hb@xt@2em{% <-- Box of fixed size for footnote number and space
    \@thefnmark\quad}% <-- Footnote number followed by a quad space
  \parbox[t]{\dimexpr\linewidth-2em}{#1}% <-- Parbox to control the width of footnote content
}
\makeatother

% layout check and fix
\checkandfixthelayout

% COUNTERS FOOTNOTES
\usepackage{chngcntr}
\counterwithout*{footnote}{chapter}

% TITLE FORMATTING
\usepackage{titlesec}

% Define chapter format with titlesec
\titleformat
    {\chapter}[display]
    {\huge\sffamily} % Main title font style
    {\Large\sffamily\chaptertitlename~\thechapter} % "Chapter N" format
    {0pt} % Space between the chapter number and title
    {\Huge} % Chapter title formatting
    [\vspace{10pt}\Large\textit{\chaptersubtitle}] % Subtitle formatting

% Command to set the subtitle (empty by default)
\newcommand{\chaptersubtitle}{}

% Automatically render the subtitle (if set) after the chapter title
\titleformat{\chapter}[display]
  {\huge\sffamily}
  {\Large\sffamily\chaptertitlename\ \thechapter}
  {0pt}
  {\Huge}
  [\ifx\chaptersubtitle\empty\else\vspace{10pt}\Large\textit{\chaptersubtitle}\fi]

% Command to set subtitle manually after chapter rendering
\newcommand{\setsubtitle}[1]{%
  \renewcommand{\chaptersubtitle}{#1}%
  \chaptermark{\chaptersubtitle} % Update subtitle for header/footer
}

\titleformat
  {\section}[block]
  {\sffamily\large\bfseries}
  {}
  {0pt}
  {}
  
\titlespacing*{\section}{0pt}{2em}{0.5em}

\titleformat{\subsection}{\sffamily\bfseries}{}{}{}
\titlespacing*{\subsection}{0pt}{2em}{0em}

% QUOTE FORMATTING
\renewenvironment{quote}%
               {\list{}{\rightmargin=.6cm\leftmargin=.6cm}%
                \itshape \item[]}% <- The effect of \samepage is local!!!
               {\endlist}

% LAYOUT CHECK AND FIX
\checkandfixthelayout

% CUSTOM TITLE PAGE
\makeatletter
\def\@maketitle{%
  % the half title page
  \pagestyle{empty}
  \halftitlepage
  \cleardoublepage

  % the title page
  \titleM
  \clearpage

  % the copyright page
  \copyrightpage
  \cleardoublepage
  \pagestyle{mystyle}
}
\makeatother
% END PREAMBLE
\makeatletter
\@ifpackageloaded{bookmark}{}{\usepackage{bookmark}}
\makeatother
\makeatletter
\@ifpackageloaded{caption}{}{\usepackage{caption}}
\AtBeginDocument{%
\ifdefined\contentsname
  \renewcommand*\contentsname{Table of contents}
\else
  \newcommand\contentsname{Table of contents}
\fi
\ifdefined\listfigurename
  \renewcommand*\listfigurename{List of Figures}
\else
  \newcommand\listfigurename{List of Figures}
\fi
\ifdefined\listtablename
  \renewcommand*\listtablename{List of Tables}
\else
  \newcommand\listtablename{List of Tables}
\fi
\ifdefined\figurename
  \renewcommand*\figurename{Figure}
\else
  \newcommand\figurename{Figure}
\fi
\ifdefined\tablename
  \renewcommand*\tablename{Table}
\else
  \newcommand\tablename{Table}
\fi
}
\@ifpackageloaded{float}{}{\usepackage{float}}
\floatstyle{ruled}
\@ifundefined{c@chapter}{\newfloat{codelisting}{h}{lop}}{\newfloat{codelisting}{h}{lop}[chapter]}
\floatname{codelisting}{Listing}
\newcommand*\listoflistings{\listof{codelisting}{List of Listings}}
\makeatother
\makeatletter
\makeatother
\makeatletter
\@ifpackageloaded{caption}{}{\usepackage{caption}}
\@ifpackageloaded{subcaption}{}{\usepackage{subcaption}}
\makeatother

\ifLuaTeX
\usepackage[bidi=basic]{babel}
\else
\usepackage[bidi=default]{babel}
\fi
\babelprovide[main,import]{english}
% get rid of language-specific shorthands (see #6817):
\let\LanguageShortHands\languageshorthands
\def\languageshorthands#1{}
\ifLuaTeX
  \usepackage{selnolig}  % disable illegal ligatures
\fi
\usepackage{bookmark}

\IfFileExists{xurl.sty}{\usepackage{xurl}}{} % add URL line breaks if available
\urlstyle{same} % disable monospaced font for URLs
\hypersetup{
  pdftitle={The Sovereign Individual},
  pdfauthor={James Dale Davidson \& Lord William Rees-Mogg},
  pdflang={en},
  hidelinks,
  pdfcreator={LaTeX via pandoc}}


\title{The Sovereign Individual}
\usepackage{etoolbox}
\makeatletter
\providecommand{\subtitle}[1]{% add subtitle to \maketitle
  \apptocmd{\@title}{\par {\large #1 \par}}{}{}
}
\makeatother
\subtitle{Mastering The Transition to the Information Age}
\author{James Dale Davidson \& Lord William Rees-Mogg}
\date{2024-09-10}

\begin{document}
\frontmatter
\maketitle

\renewcommand*\contentsname{Contents}
{
\setcounter{tocdepth}{0}
\tableofcontents
}

\mainmatter
\bookmarksetup{startatroot}

\chapter*{About this book}\label{about-this-book}

\markboth{About this book}{About this book}

\bookmarksetup{startatroot}

\chapter{De overgang in het jaar
2000}\label{de-overgang-in-het-jaar-2000}

\begin{quote}
`Het lijkt wel alsof er iets monumentaals op til is: grafieken tonen de
jaarlijkse groei van bevolkingsaantallen, de concentratie van
koolstofdioxide in de atmosfeer, het aantal webadressen en de megabytes
per dollar. Alle cijfers stijgen naar een asymptoot net na de
eeuwwisseling: de singulariteit. Het einde van alles wat we kennen. Het
begin van iets wat we wellicht nooit zullen doorgronden.' \footnote{Danny
  Hillis, \emph{The Millennium Clock}, Wired, Special Edition, Fall
  1995, p.48.} - Danny Hillis
\end{quote}

\section{Voorgevoelens}\label{voorgevoelens}

Het jaar 2000 houdt de westerse verbeelding al eeuwenlang in zijn greep.
Nadat men destijds verwachtte dat de wereld bij de millenniumwisseling
van het eerste millennium na Christus zou instorten, keken theologen,
evangelisten, dichters en zieners reikhalzend uit naar een decennium vol
ingrijpende gebeurtenissen. Zelfs de beroemde Isaac Newton speculeerde
ooit dat de wereld in het jaar 2000 ten onder zou gaan. Michel de
Nostradamus, wiens profetieën al sinds 1568 door generaties zijn
bekeken, voorspelde de komst van de derde antichrist in juli
1999.\footnote{Ericka Cheetham, \emph{The Final Prophecies of
  Nostradamus} (New York: Putnam, 1989), p.~424.} De Zwitserse
psycholoog Carl Jung, expert op het gebied van het `collectieve
onbewuste', zag al in 1997 de geboorte van een nieuw tijdperk opdoemen.
Men maakt zulke voorspellingen vaak belachelijk, net als de nuchtere
prognoses van economen zoals Dr.~Edward Yardeni van \emph{Deutsche Bank
Securities}, die verwacht dat computerstoringen op middernacht van het
millennium de hele wereldeconomie compleet zullen
ontwrichten.\footnote{Dr.~Edward Yardeni, \emph{Year 2000 Recession:
  `Prepare for the worst. Hope for the best,' Version 5.0} (13 mei 1998,
  B1.2).} Of je het Y2K-probleem nu beschouwt als ongegronde hysterie,
aangewakkerd door programmeurs en adviseurs op het gebied van
informatietechnologie, of als een mysterieus fenomeen waarin technologie
zich ontvouwt in samenhang met profetische ideeën, je kunt niet
ontkennen dat de drempel van het millennium voor veel meer opwinding
zorgt dan de gebruikelijke, morbide onzekerheid over de toekomst.

Het optimisme dat westerse samenlevingen de afgelopen 250 jaar
kenmerkte, wordt nu overschaduwd door een gevoel van toekomstonheil.
Overal zie je het terug: in de gezichtsuitdrukkingen van mensen, in hun
gesprekken, in peilingen en zelfs op de stembus. Net zoals een subtiele
verandering in de ionen in de atmosfeer al verraadt dat een onweersbui
eraan komt, nog voordat de wolken donker worden en de bliksem slaat,
hangt er in de schemering van het millennium een voorgevoel van
verandering in de lucht. Ieder voelt op zijn eigen manier dat de tijd
dringt voor de levenswijze die hij gewend is. Naarmate het decennium
vordert, nemen zowel een moorddadige eeuw als een glorieuze periode van
menselijke verwezenlijkingen afscheid. Alles vindt zijn einde in het
jaar 2000.

\begin{quote}
Want er is niets verborgen dat niet op een dag onthuld wordt, en geen
geheim dat niet bekend wordt.\\
-- Mattheüs 10:26
\end{quote}

Wij zijn ervan overtuigd dat de moderne fase van de westerse beschaving
ten einde loopt -- en in dit boek leggen we uit waarom. Net als vele
andere werken poogt het via een donkere spiegel te kijken en de vage
contouren en proporties van een nog te komen toekomst te schetsen. Op
die manier hanteren wij `apocalyptisch' in de letterlijke zin van het
woord, want `apokalypsis' betekent in het Grieks `onthulling'. Wij
geloven dat een nieuw tijdperk -- het informatietijdperk -- op het punt
staat onthuld te worden.

\begin{quote}
Wij maken de eerste tekenen waar van een nieuwe logische ruimte, een
directe elektronische alomtegenwoordigheid waar iedereen toegang toe
heeft en die we intens kunnen ervaren. Kortom, we bevinden ons in de
begindagen van een nieuwe vorm van gemeenschap. De virtuele gemeenschap
fungeert als model voor een seculier koninkrijk der hemelen; zoals Jezus
zei dat er vele huizen waren in het koninkrijk van zijn Vader, zo
ontstaan er talloze virtuele gemeenschappen, elk afgestemd op hun eigen
wensen en behoeften.\\
-- Michael Grasso\footnote{Michael Grasso, \emph{The Millennium Myth:
  Love and Death at the End of Time} (Wheaton, Illinois: Quest Books,
  1995).}
\end{quote}

\section{The fourth stage of human
society}\label{the-fourth-stage-of-human-society}

Het thema van dit boek richt zich op de opkomst van een nieuwe
machtsrevolutie die individuen bevrijdt en de traditionele natiestaat
van de twintigste eeuw verdringt. Innovaties die de logica van geweld
radicaal herschrijven, verschuiven de grenzen waarbinnen de toekomst
vorm krijgt. Als onze conclusies kloppen, sta je op de drempel van de
meest ingrijpende revolutie in de geschiedenis. Veel sneller dan de
meesten zich voorstellen, zal microverwerking de natiestaat ondermijnen
en uiteindelijk vernietigen, waarna nieuwe vormen van sociale
organisatie zullen ontstaan. Deze transformatie wordt allesbehalve
eenvoudig.

De uitdaging die eraan verbonden is, wordt des te groter omdat deze zich
in een ongekend tempo voltrekt, veel sneller dan we in het verleden
hebben gezien. In de gehele menselijke geschiedenis -- van de oertijd
tot nu -- kennen we slechts drie fundamentele ontwikkelingsstadia: (1)
jager-verzamelaarsmaatschappijen, (2) agrarische samenlevingen en (3)
industriële samenlevingen. Aan de horizon doemt nu iets compleet nieuws
op: informatiemaatschappijen, het vierde stadium van sociale
organisatie.

Bij elk van de voorgaande samenlevingsstadia kende men een eigen,
duidelijk afgebakende fase in de evolutie en beheersing van geweld.
Zoals we uitvoerig toelichten, zullen informatiemaatschappijen de
voordelen van geweld drastisch verkleinen, mede omdat ze niet langer
gebonden zijn aan één vaste locatie. De virtuele wereld van cyberspace
-- die romanschrijver William Gibson omschreef als een `consensuele
hallucinatie' -- zal voor pestkoppen onbereikbaar blijven, zoveel de
verbeelding reikt. In dit nieuwe millennium valt de winst uit
grootschalige geweldsbeheersing veel lager uit dan op enig moment sinds
vóór de Franse Revolutie. Dit heeft verstrekkende gevolgen. Eén daarvan
is een toename van criminaliteit. Als de beloning voor grootschalig
geweld instort, zal de prikkel voor kleinschalig geweld naar alle
waarschijnlijkheid juist groeien. Geweld krijgt dan een meer
willekeurige, lokale aard, terwijl georganiseerde criminaliteit in
omvang toeneemt; wij lichten hierover verder toe.

Een andere logische consequentie van de afnemende voordelen van geweld
is het vervagen van de politieke rol. Er is veel bewijs dat het
vertrouwen in de burgerlijke mythen van de twintigste-eeuwse natiestaat
snel afneemt. De ondergang van het communisme is slechts één opvallend
voorbeeld. Zoals we in detail onderzoeken, bewijzen de afbrokkelende
moraliteit en de groeiende corruptie onder leiders van westerse
regeringen dat het potentieel van de natiestaat is uitgeput -- het gaat
hier niet om een willekeurige ontwikkeling. Zelfs vele leiders geloven
de holle frasen die zij verkondigen niet meer, en ook het publiek
schenkt die woorden geen vertrouwen meer.

\subsection{Geschiedenis herhaalt
zich}\label{geschiedenis-herhaalt-zich}

Deze situatie vertoont opvallende overeenkomsten met het verleden.
Telkens wanneer technologische veranderingen de oude structuren losmaken
van de nieuwe krachten die de economie aandrijven, veranderen ook de
morele normen en gaan mensen degenen die de oude instituties leiden met
groeiende minachting behandelen. Deze wijdverspreide afkeer komt vaak al
duidelijk naar voren voordat er zelfs een nieuwe, samenhangende
ideologie van verandering ontstaat. Zo was dat bijvoorbeeld in de late
vijftiende eeuw, toen de middeleeuwse kerk nog het overheersende
instituut van het feodalisme vormde. Ondanks het brede geloof in `de
heiligheid van het sacerdotale ambt' behandelden mensen zowel de hoge
als de lage geestelijkheid met grote minachting -- vergelijkbaar met de
populaire houding ten opzichte van politici en bureaucraten in deze
tijd.

Wij zijn ervan overtuigd dat we veel kunnen leren door een vergelijking
te maken tussen het einde van de vijftiende eeuw, toen het leven
doordrenkt was met georganiseerde religie, en de hedendaagse wereld,
waarin politiek overal de boventoon voert. Destijds waren de kosten om
de geïnstitutionaliseerde religie in stand te houden historisch extreem
hoog, net zoals tegenwoordig de kosten voor het ondersteunen van de
overheid absurd ver zijn opgelopen.

We hebben gezien hoe de georganiseerde religie veranderde na de
buskruitrevolutie. Technologische innovaties stimuleerden het verkleinen
van religieuze instellingen en dwongen tot drastisch kostenbesparingen.
Een vergelijkbare technologische revolutie dreigt aan het begin van dit
millennium de natiestaat radicaal te verkleinen.

\begin{quote}
Vandaag, na meer dan een eeuw aan elektronische technologie, hebben we
ons centrale zenuwstelsel zelf uitgebreid in een mondiale omarming,
waarbij we zowel ruimte als tijd, voor zover onze planeet betreft,
hebben afgeschaft.\footnote{Marshall McLuhan, \emph{Understanding Media}
  (New York: Signet, 1964), p.~19.}
\end{quote}

\subsection{De informatierevolutie}\label{de-informatierevolutie}

Naarmate grote systemen verder instorten, verliest systematische dwang
zijn bepalende rol in het vormgeven van het economisch leven en de
inkomensverdeling. Efficiëntie krijgt al snel voorrang boven de bevelen
van de machthebbers bij het organiseren van sociale instituties. Dit
houdt in dat provincies en zelfs steden die effectief eigendomsrechten
bewaken en de rechtspraak draaiende houden, terwijl ze weinig middelen
verbruiken, in het informatietijdperk als levensvatbare soevereiniteiten
kunnen functioneren -- iets wat de afgelopen vijf eeuwen zelden het
geval was.

Binnen cyberspace ontstaat een geheel nieuw terrein voor economische
activiteiten dat niet vatbaar is voor fysiek geweld. De grootste
voordelen zullen vooral de `cognitieve elite' treffen, die zich steeds
minder laat beperken door politieke grenzen. Zij voelen zich net zo
thuis in Frankfurt, Londen, New York, Buenos Aires, Los Angeles, Tokio
en Hong Kong. Binnen de rechtsgebieden zal de inkomensongelijkheid
toenemen, terwijl tussen deze gebieden juist meer gelijkheid ontstaat.

Het soevereine individu onderzoekt de maatschappelijke en financiële
gevolgen van deze revolutionaire ommekeer. Wij willen je helpen de
kansen van dit nieuwe tijdperk te grijpen en ervoor zorgen dat de impact
ervan je niet verwoest. Zelfs als slechts de helft van onze
verwachtingen uitkomt, ondervind je een verandering van ongekende
proporties, zonder werkelijk historisch precedent.

De transformatie in het jaar 2000 verandert radicaal het karakter van de
wereldeconomie en gebeurt bovendien veel sneller dan elke voorgaande
fasewisseling. In tegenstelling tot de agrarische revolutie vergt de
informatie-revolutie geen millennia om haar volledige impact te
bereiken. In tegenstelling tot de industriële revolutie verspreidt haar
effect zich niet over eeuwen; je ervaart de volledige reikwijdte ervan
binnen één mensenleven.

Daarbij vindt deze transformatie vrijwel gelijktijdig over de hele
wereld plaats. Technologische en economische innovaties beperken zich
niet langer tot geïsoleerde uithoeken van de wereld. De ommekeer raakt
vrijwel alle hoeken van de aarde en markeert een breuk met het verleden
die zo ingrijpend is dat hij het bijna mythische rijk van de goden tot
leven wekt, zoals de oude Grieken ooit voorstelden.

Veel meer dan men zich nu durft voor te stellen, zal blijken dat het in
het nieuwe millennium buitengewoon lastig -- zo niet onmogelijk -- is om
veel van onze hedendaagse instellingen te behouden. Zodra
informatiesamenlevingen opkomen, onderscheiden ze zich van industriële
samenlevingen op dezelfde manier als het Griekenland van Aeschylus
verschilde van de wereld van de grotbewoners.

\section{\texorpdfstring{\emph{`Prometheus unbound'}: De opkomst van het
soevereine
individu}{`Prometheus unbound': De opkomst van het soevereine individu}}\label{prometheus-unbound-de-opkomst-van-het-soevereine-individu}

\begin{quote}
Ik ken geen bemoedigender feit dan het onmiskenbare vermogen van de mens
om zijn leven te verrijken door doelgerichte inzet -- Henry David
Thoreau
\end{quote}

De aanstaande transformatie brengt zowel goed als slecht nieuws met zich
mee. Het positieve is dat de informatierrevolutie mensen als nooit
tevoren bevrijdt. Voor het eerst geniet degene die zichzelf kan
bijscholen de volledige vrijheid om eigen ideeën te ontwikkelen en
maximaal te profiteren van zijn productiviteit. Genialiteit komt tot
bloei wanneer overheidsinmenging stopt en raciale en etnische
vooroordelen hun invloed verliezen. Als je echt bekwaam bent, laat je je
in de informatiemaatschappij niet tegenhouden door bekrompen
opvattingen. Het maakt niet uit wat de meeste mensen van je ras,
uiterlijk, leeftijd, seksuele voorkeur of kapsel vinden. In de
cybereconomie blijft je identiteit anoniem. Minder fraai, mollig, ouder
of met een beperking wedijveren zij op gelijke voet met de jonge en
knappe, dankzij de volledige kleurblinde anonimiteit op de nieuwe
frontlinies van cyberspace.

\subsection{Ideeën worden rijkdom}\label{ideeuxebn-worden-rijkdom}

Waar echte verdienste ontstaat, beloon je dat als nooit tevoren. In een
wereld waarin ideeën de grootste bron van rijkdom vormen in plaats van
louter fysiek kapitaal, kan iedereen met een heldere geest potentieel
rijk worden. Het informatietijdperk belooft een tijdperk van sociale
mobiliteit te worden en opent talloze gelijke kansen voor de miljarden
mensen in regio's die nooit ten volle hebben geprofiteerd van de
welvaart van de industriële samenleving. De slimsten, succesvolste en
meest ambitieuze treden op als ware zelfstandigen.

In eerste instantie bereikt slechts een enkeling volledige financiële
soevereiniteit, maar dat doet niets af aan de voordelen van financiële
onafhankelijkheid. Dat niet iedereen een gigantisch fortuin vergaart,
betekent immers niet dat het streven naar rijkdom zinloos is. Er zijn
25.000 miljonairs op elke miljardair. Als je miljonair bent en geen
miljardair, ben je zeker niet arm. In de toekomst meet je je financiële
succes niet alleen aan het aantal nullen op je nettowaarde, maar ook aan
de mate waarin je je zaken zo organiseert dat je volledige persoonlijke
autonomie bereikt. Hoe slimmer je bent, hoe minder energie je nodig hebt
om de overgang naar financiële onafhankelijkheid te maken. Zelfs mensen
met bescheiden middelen zullen vooruitkomen zodra de politieke druk op
de wereldeconomie afneemt. Onovertroffen financiële onafhankelijkheid
wordt dan een haalbaar doel voor jou en je kinderen.

Op het hoogste plateau van productiviteit concurreren en communiceren
onafhankelijke individuen op een wijze die doet denken aan de relaties
tussen de Griekse goden. De ongrijpbare Mount Olympus van het komende
millennium bevindt zich in cyberspace -- een rijk zonder fysieke vorm
dat desalniettemin tegen het tweede decennium van dit nieuwe millennium
uitgroeit tot 's werelds grootste economie. Tegen 2025 telt de
cybereconomie miljoenen deelnemers. Sommigen zullen een vermogen
ontwikkelen vergelijkbaar met dat van Bill Gates -- elk met een waarde
van meer dan 10 miljard dollar -- terwijl de cyberarmen bestaan uit
mensen die minder dan 200.000 dollar per jaar verdienen. Er komt geen
cyberwelzijn, geen cybertaksen én geen cyberregering. De cybereconomie
kan, in plaats van China, wel eens het grootste economische fenomeen van
de komende dertig jaar worden.

Het goede nieuws is dat politici in dit nieuwe rijk de handel niet
zullen beheersen, onderdrukken of reguleren -- net zoals de wetgevers
van de oude Griekse stadsstaten zeker niet in staat waren om aan Zeus'
baard te knippen. Dat komt de rijken ten goede en is nog beter nieuws
voor de minder vermogenden. De door de politiek opgelegde obstakels en
lasten belemmeren immers vooral het rijk worden eerder dan het rijk
blijven. Dankzij de dalende opbrengsten uit geweld en de decentralisatie
van rechtsgebieden kan elke energieke en ambitieuze persoon profiteren
van het verval van de politiek. Ook consumenten van overheidsdiensten
zullen hier voordeel uit halen, omdat ondernemers de vruchten van de
concurrentie verder benutten. Al decennialang richtte de concurrentie
tussen rechtsgebieden zich op militaire ondernemingen. Maar de opkomst
van de cybereconomie zorgt op nieuwe grondslagen voor concurrentie in de
levering van soevereiniteitsdiensten. Een toename van het aantal
rechtsgebieden stimuleert experimenten met vernieuwende manieren om
contracten af te dwingen en de veiligheid van mensen en hun eigendommen
te waarborgen. De bevrijding van een groot deel van de wereldeconomie
van politieke controle dwingt de overgebleven overheden om meer
marktgerichte werkwijzen toe te passen. Op den duur blijven zij niets
anders over dan de inwoners van hun rechtsgebieden als klanten te
behandelen, in plaats van hen te onderwerpen aan afpersing zoals bij
georganiseerde misdaad.

\subsection{Voorbij de politiek}\label{voorbij-de-politiek}

Wat de mythologie ooit als het domein van de goden afschilderde, wordt
voor het individu een haalbare keuze -- een leven buiten de invloed van
koningen en raden. Eerst zullen tientallen mensen, daarna honderden en
uiteindelijk miljoenen zich bevrijden uit de greep van de politiek.
Terwijl zij zich hiervan ontdoen, hervormen zij de werking van
overheden, waardoor de sfeer van dwang afneemt en particuliere controle
over middelen toeneemt.

De opkomst van het autonome individu zal opnieuw de verrassende
voorspellende kracht van mythen onderstrepen. Vroege agrarische
volkeren, die nauwelijks inzicht hadden in de natuurwetten, verbeeldden
zich dat `krachten die wij bovennatuurlijk noemen' overal aanwezig
waren. Soms zetten zij deze krachten in en soms belichaamden zij ze als
menselijke goden die op henzelf leken en met de mensen omgingen -- zoals
Sir James George Frazer in \emph{The Golden Bough} omschreef als `een
grote democratie'.

Toen de antieken zich voorstelden dat de kinderen van Zeus onder hen
leefden, lieten zij zich meevoeren door een diep geloof in magie. Zoals
andere primitieve agrarische volkeren bewonderden zij de natuur en waren
zij ervan overtuigd dat de kracht van de individuele wil -- oftewel
magie -- de natuurlijke orde in beweging zette. Hun visie op de natuur
en de goden straalde geen zelfbewuste, profetische inslag uit. Op het
gebied van microtechnologie waren zij totaal onvoorbereid. Zij konden
zich niet voorstellen dat deze technologie, duizenden jaren later, de
marginale productiviteit van mensen drastisch zou beïnvloeden en de
balans tussen macht en efficiëntie radicaal zou verschuiven, waardoor de
manier waarop bezittingen worden gecreëerd en beschermd volledig zou
worden getransformeerd. Toch vindt u een vreemde overeenkomst tussen hun
mythologische voorstellingen en de wereld die u waarschijnlijk zult
tegenkomen.

\subsection{Alt.abracadabra}\label{alt.abracadabra}

De `abracadabra' van de magische oproep lijkt sterk op het wachtwoord
dat toegang verleent tot een computer. Dankzij de snelle verwerking van
computers kunnen we de magie van de geest al nabootsen. Vroege digitale
dienaren gehoorzamen immers al de bevelen van degenen die de computers
besturen, net zoals men vroeger geesten in magische lampen plaatste. De
virtuele realiteit van informatietechnologie breidt het rijk van
menselijke verlangens uit, zodat bijna alles wat men zich maar kan
voorstellen werkelijkheid lijkt te worden. Telepresence stelt mensen in
staat om afstanden met een bijna bovennatuurlijke snelheid te
overbruggen en op afstand gebeurtenissen te volgen, precies zoals de
Grieken beweerden dat Hermes en Apollo dat konden. Na verloop van tijd
genieten de soevereine individuen van het informatietijdperk, net als de
goden uit oude mythen, van een soort `diplomatieke immuniteit' tegen de
politieke beproevingen die sterfelijke mensen door de eeuwen heen hebben
gekend.

Het nieuwe, soevereine individu opereert in dezelfde fysieke omgeving
als de gewone burger, maar politiek gezien leeft hij in een eigen sfeer.
Met een veel groter arsenaal aan middelen tot zijn beschikking en zonder
de beperkingen van talrijke vormen van dwang, gaat dit individu
overheden hervormen en economieën herinrichten in het nieuwe millennium.
De volledige impact van deze verschuiving is bijna onvoorstelbaar.

Genialiteit en nemesis

Voor wie houdt van menselijke ambitie en succes, opent het
informatietijdperk een schat aan kansen. Zonder twijfel is dit het beste
nieuws van generaties, maar het gaat ook gepaard met minder positieve
ontwikkelingen. Door de triomf van individuele vrijheid en echte gelijke
kansen op basis van verdienste herstructureert onze samenleving zich op
zo'n manier dat prestaties royaal beloond worden en de persoonlijke
autonomie sterk toeneemt. Dat betekent dat mensen veel meer
verantwoordelijkheid voor hun eigen leven krijgen dan in het industriële
tijdperk. Tegelijk zal het onterecht genoten levensstandaardvoordeel,
waar de inwoners van geavanceerde industriële samenlevingen in de
twintigste eeuw van profiteerden, sterk afnemen. Op dit moment verdient
de top 15 procent van de wereldbevolking gemiddeld \$21.000 per persoon
per jaar, terwijl de overige 85 procent gemiddeld slechts \$1.000 tot
hun beschikking heeft. Dit enorme, in de loop der tijd opgepotte
voordeel zal onvermijdelijk verdwijnen in de nieuwe realiteit van het
informatietijdperk.

Naarmate dit gebeurt, stort het vermogen van natiestaten om op grote
schaal inkomen te herverdelen in. Informatietechnologie zorgt voor een
scherpe toename van de concurrentie tussen rechtsgebieden. Als
technologie mobiel wordt en transacties in de cyberspace plaatsvinden --
wat steeds vaker het geval zal zijn -- kunnen overheden niet langer meer
in rekening brengen dan de werkelijke waarde van hun diensten voor de
betalende burgers. Iedereen met een laptop en een satellietverbinding
kan vrijwel elk informatiebedrijf overal runnen, wat vrijwel de gehele
multitriljoenen-dollar financiële handel wereldwijd omvat.

Dit betekent dat u zich niet langer gedwongen voelt in een rechtsgebied
met hoge belastingen te wonen om een hoog inkomen te realiseren. In de
toekomst, wanneer bijna alle rijkdom wereldwijd verzameld en uitgegeven
kan worden, zullen overheden die te hoge tarieven rekenen voor domicilie
simpelweg hun beste klanten doen wegtrekken. Als onze redenering opgaat
-- en daar zijn wij van overtuigd -- blijft de natiestaat in de huidige
vorm niet overleven.

\section{Het einde van naties}\label{het-einde-van-naties}

Veranderingen die de macht van gevestigde instellingen ondermijnen, zijn
zowel verontrustend als riskant. Net zoals monarchen, grote heren,
pausen en potentaten in de vroegmoderne periode meedogenloos vochten om
hun traditionele voorrechten te behouden, zal de regering van vandaag
geweld inzetten -- vaak op een heimelijke en willekeurige wijze -- om de
klok terug te draaien. Verzwakt door de technologische uitdagingen gaat
de staat steeds autonomer wordende burgers benaderen met dezelfde
meedogenloosheid en diplomatie als waarmee zij tot nu toe met andere
regeringen omging.

Op 20 augustus 1998 luidde de komst van deze nieuwe fase in de
geschiedenis met een knal in, toen de Verenigde Staten voor ongeveer 200
miljoen dollar Tomahawk BGM-109 kruisraketten afvuurden op doelen die
naar verluidt in verband stonden met de verbannen Saoedische miljonair
Osama bin Laden. Bin Laden werd de eerste persoon in de geschiedenis op
wiens satelliettelefoon raketaanvallen werden gericht. Tegelijkertijd
verwoestten de Verenigde Staten een farmaceutische fabriek in Khartoem,
Soedan, die met Bin Laden werd geassocieerd. Bin Ladens opkomst als de
belangrijkste vijand van de Verenigde Staten illustreert een
fundamentele verandering in de aard van oorlogsvoering. Één enkel
individu, al bezit hij honderden miljoenen dollars, kan nu als een
geloofwaardige bedreiging worden gezien voor de machtigste militaire
macht van het industriële tijdperk. In uitspraken die doen denken aan de
Koude Oorlogpropaganda over de Sovjet-Unie portretteerden de president
van de Verenigde Staten en zijn nationale veiligheidsspecialisten Bin
Laden -- een privépersoon -- als een transnationale terrorist en de
voornaamste vijand van de Verenigde Staten.

Dezelfde militaire logica die ervoor zorgde dat Osama bin Laden als
oppervijand van de Verenigde Staten werd gezien, zal zich ook aftekenen
in de verhouding tussen overheden en hun burgers. Steeds strengere
inningstechnieken volgen als logisch gevolg van de opkomst van een nieuw
soort onderhandeling tussen overheden en individuen. Door technologische
vooruitgang worden individuen meer dan ooit soeverein, waardoor men hen
op uiteenlopende wijze behandelt -- soms met geweld als vijanden, soms
als gelijkwaardige onderhandelingspartners en af en toe als bondgenoten.
Hoe meedogenloos overheden zich ook opstellen tijdens deze
overgangsperiode, een fusie van de IRS met de CIA levert hen weinig op.
Overheden staan er niet omheen dat zij met autonome individuen moeten
onderhandelen, omdat hun middelen niet langer even eenvoudig te
controleren zijn.

De veranderingen die de Informatie-Revolutie teweegbrengt, veroorzaken
niet alleen fiscale crises bij overheden, maar zorgen er ook voor dat
alle grote structuren uiteenvallen. In de twintigste eeuw zijn al
veertien rijken verdwenen. Het verval van rijken maakt deel uit van een
proces dat uiteindelijk ook de natiestaat zal doen instorten. Overheden
moeten zich aanpassen aan de groeiende zelfstandigheid van het individu.
De capaciteit om belastingen te innen daalt met 50 tot 70 procent, wat
kleinere rechtsgebieden bevoordeelt. Het vaststellen van concurrerende
voorwaarden om bekwame individuen en hun kapitaal aan te trekken, slaagt
in enclaves veel eenvoudiger dan op continentale schaal.

Wij zijn ervan overtuigd dat hedendaagse barbaren, zodra de moderne
natiestaat uiteenvalt, steeds vaker op de achtergrond de macht grijpen.
Groepen zoals de Russische \emph{mafiya}, die de overblijfselen van de
voormalige Sovjet-Unie uitbuit, andere etnische misdaadbendes,
\emph{nomenklaturen}{[}\^{}27{]}, drugsbaronnen en afvallige geheime
diensten gaan al volgens hun eigen regels te werk. Veel meer dan men
doorgaans denkt, hebben deze moderne barbaren de vorm van de natiestaat
al binnengedrongen zonder dat haar uiterlijk wezenlijk verandert. Ze
fungeren als microparasieten die zich tegoed doen aan een vervalend
systeem. Op dezelfde wijze als een oorlogvoerende staat zetten deze
groepen staatsmethoden op een kleinere schaal in -- vaak gewelddadig en
zonder principes. Dankzij efficiëntere, kleinschalige organisatie kan
een kleine groep even effectief geweld inzetten en beheersen. Naarmate
de technologische revolutie vordert, organiseert geweld zich steeds
vaker buiten centrale controle en richt de aanpak om het in te dammen
zich meer op efficiëntie dan op de omvang van macht.

\subsection{Geschiedenis in omkering}\label{geschiedenis-in-omkering}

De nieuwe logica van het informatietijdperk keert het proces om dat de
natiestaat zich in de afgelopen vijf eeuwen heeft ontwikkeld. Lokale
machtscentra houden zich staande, terwijl de staat uiteenvalt in
gefragmenteerde, overlappende soevereine eenheden.\footnote{Voor meer
  informatie over gefragmenteerde soevereiniteiten als voorloper én
  alternatief voor de natiestaat, zie Charles Tilly, \emph{Coercion,
  Capital and European States AD 990-1992} (Oxford: \emph{Blackwell},
  1993).} De groeiende invloed van de georganiseerde misdaad is slechts
één afspiegeling van deze trend. Multinationale ondernemingen moeten
inmiddels al het niet-essentiële werk uitbesteden. Sommige
conglomeraatbedrijven, zoals \emph{AT\&T}, \emph{Unisys} en \emph{ITT},
splitsen zich op in meerdere ondernemingen om winstgevender te kunnen
opereren. De natiestaat valt uiteen als een onhandelbaar conglomeraat,
maar dat gebeurt waarschijnlijk pas nadat financiële crises hem hiertoe
hebben gedwongen.

Niet alleen verandert de machtsbalans in de wereld, maar verandert ook
de aard van het werk radicaal. Dit betekent dat de manier waarop
bedrijven opereren onvermijdelijk zal veranderen. De `virtuele
onderneming' illustreert een fundamentele transformatie in de
bedrijfsvoering, mogelijk gemaakt door de dalende kosten voor informatie
en transacties. We gaan in op de gevolgen van de Informatie-Revolutie,
die leidt tot het uiteenvallen van traditionele ondernemingsstructuren
en het vaarwel zeggen tegen de traditionele `goede baan'. In het
informatietijdperk betekent een `baan' simpelweg een taak die je
uitvoert, in plaats van een vaste positie. Microverwerking opent geheel
nieuwe mogelijkheden voor economische activiteiten die alle territoriale
grenzen overschrijden. Deze grensoverschrijdende ontwikkeling is
wellicht de meest revolutionaire sinds Adam en Eva, verbannen uit het
paradijs onder het vonnis van hun Schepper: `In het zweet van uw
voorhoofd zult gij brood eten.' Naarmate technologie onze hulpmiddelen
radicaal vernieuwt, veroudert ons rechtssysteem, verandert onze moraal
en verschuift onze perceptie. Dit boek legt uit hoe.

Dankzij microverwerking en de razendsnelle vooruitgang in communicatie
kan iedereen nu zelf bepalen waar hij of zij werkt. Transacties via
internet of het wereldwijde web worden steeds beter versleuteld en
zullen binnenkort vrijwel onmogelijk voor belastingambtenaren te
onderscheppen zijn. Offshore groeit belastingvrij kapitaal al veel
sneller dan onshorefondsen, die nog steeds worstelen met de hoge
belastingdruk die de natiestaten van de twintigste eeuw oplegden. Na de
millenniumwisseling zal een groot deel van de wereldhandel verhuizen
naar het nieuwe domein van cyberspace, een gebied waar regeringen even
weinig invloed hebben als over de zeebodem of de verre planeten. In
cyberspace verdwijnen de dreigingen van fysiek geweld, de alfa en omega
van de oude politiek. Hier ontmoeten de minderbedeelden en de
machthebbers elkaar op gelijke voet. Cyberspace vormt de ultieme
offshore jurisdictie: een economie zonder belastingen, een soort Bermuda
in de lucht vol diamanten.

Wanneer dit ultieme belastingparadijs volledig toegankelijk is voor het
bedrijfsleven, zullen vrijwel alle fondsen als offshorekapitaal
fungeren, precies naar wens van hun eigenaar. Dit zal een kettingreactie
in gang zetten. De staat is er inmiddels aan gewend geraakt haar
belastingbetalers te behandelen als een boer zijn koeien, die hij op een
weiland plaatst om gemolken te worden. Voor je het weet, krijgen de
koeien vleugels.

\subsection{De wraak van naties}\label{de-wraak-van-naties}

Als een kwaadwillende boer zal de staat in eerste instantie wanhopige
pogingen ondernemen om haar ontsnapte kudde in toom te houden. Ze zet
heimelijke, zelfs gewelddadige middelen in om de toegang tot bevrijdende
technologieën te beperken. Dergelijke noodmaatregelen werken slechts
tijdelijk, als ze al effect sorteren. De natiestaat van de twintigste
eeuw, met al haar grootse claims, zal verhongeren terwijl haar
belastinginkomsten dalen.

Wanneer de staat haar uitgaven niet kan dekken door de belastingen te
verhogen, grijpt ze naar nog wanhopigere maatregelen. Daaronder valt ook
het drukken van geld. Overheden hebben zich er inmiddels aan gewend dat
ze een monopolie op de valuta hebben en deze naar eigen inzicht kunnen
devalueren. Deze willekeurige inflatie kenmerkt het monetaire beleid van
vrijwel alle staten in de twintigste eeuw. Zelfs de sterkste munteenheid
van de naoorlogse periode, de Duitse mark, verloor tussen 1 januari 1949
en het einde van juni 1995 maar liefst 71 procent van haar waarde,
terwijl in dezelfde periode de Amerikaanse dollar 84 procent in waarde
daalde.\footnote{Op 31 december 1948 bedroeg de Duitse GPI-index 33,20
  en op 30 juni 1995 112,90, wat neerkomt op een samengestelde
  jaarlijkse waardevermindering van 2,7 procent. De Amerikaanse CPI
  steeg in dezelfde periode van 24 naar 152,50, wat resulteerde in een
  cumulatieve inflatie van 635 procent.} Die inflatie werkt als een
belasting op iedereen die de valuta aanhoudt. Zoals we later zullen
zien, zal inflatie als inkomstenbron voor de overheid grotendeels
verdwijnen door de opkomst van cybergeld. Nieuwe technologieën geven
mensen de mogelijkheid om het nationale monopolie, dat overheden in de
moderne periode op het uitgeven en reguleren van geld hebben, te
omzeilen. De kredietcrisissen die in 1997 en 1998 Azië, Rusland en
andere opkomende economieën teisterden, tonen immers aan dat nationale
valuta en kredietbeoordelingen achterhaald zijn en de soepele werking
van de wereldeconomie verstoren. Dat alle transacties binnen een
jurisdictie in een nationale valuta moeten verlopen, maakt de economie
kwetsbaar voor fouten van centrale bankiers en aanvallen van
speculanten, wat deflatoire crises van de ene regio in de andere
veroorzaakt. In het informatietijdperk kunnen individuen cybervaluta
gebruiken en daarmee hun monetaire onafhankelijkheid terugwinnen.
Wanneer zij hun eigen monetaire beleid via het World Wide Web voeren,
wordt het minder relevant dat de staat de drukkers van het industriële
tijdperk blijft beheersen. Hun invloed op de wereldwijde rijkdom wordt
dan ingehaald door wiskundige algoritmen die geen fysieke vorm kennen.
In het nieuwe millennium zal cybergeld, beheerd door particuliere
markten, het fiatgeld -- uitgegeven door overheden -- vervangen. Alleen
de minderbedeelden zullen de dupe worden van inflatie en de
daaropvolgende deflatoire instortingen die het gevolg zijn van de
kunstmatige hefboomwerking waarmee fiatgeld in de economie wordt
gespoten.

Zonder de gebruikelijke bevoegdheid om belastingen te heffen en inflatie
op te wekken, zullen regeringen -- zelfs in traditioneel burgerlijke
landen -- gemeen worden. Naarmate de inkomstenbelasting steeds minder
opbrengst genereert, zullen ouderwetse en arbitrair ingestelde heffingen
weer de kop opsteken. De ultieme vorm van bronbelasting -- in de
praktijk of zelfs openlijk gijzelingspraktijken -- krijgt vorm bij
overheden die wanhopig proberen te voorkomen dat rijkdom aan hun greep
ontsnapt. Ongelukkige burgers voelen zich dan bijna middeleeuws
uitgekozen en als gijzelaars aan hun losgeld overgelaten. Ondernemingen
die diensten verlenen ter bevordering van de individuele vrijheid raken
verstrikt in infiltratie, sabotage en ontwrichting. Willekeurige
onteigening van eigendom -- alledaags in de Verenigde Staten, waar dit
vijfduizend keer per week gebeurt -- zal alleen maar toenemen.
Regeringen zullen mensenrechten schenden, de vrije informatiestroom
censureren, nuttige technologieën saboteren en soms nog veel erger
handelen. Net zoals de inmiddels verdwenen Sovjet-Unie tevergeefs
probeerde de toegang tot personal computers en Xerox-machines te
beperken, zullen westerse regeringen met totalitaire middelen proberen
de cybereconomie te onderdrukken.

\section{Terugkeer van de luddites}\label{terugkeer-van-de-luddites}

Dergelijke methoden kunnen bij bepaalde bevolkingsgroepen in de smaak
vallen. Het goede nieuws over de bevrijding van het individu en meer
autonomie zal voor velen die niet tot de cognitieve elite behoren, als
slecht nieuws overkomen. Het meeste verzet komt vermoedelijk van mensen
met gemiddeld talent in de huidige rijke landen; zij gaan ervaren dat
informatietechnologie een bedreiging vormt voor hun levensstijl. De
begunstigden van de georganiseerde dwang -- waaronder miljoenen die een
herverdeeld inkomen ontvangen van de overheid -- kunnen de nieuwe
vrijheid, zoals gerealiseerd door soevereine individuen,
verafschuwen.\footnote{Op 31 december 1948 bedroeg de Duitse GPI-index
  33,20 en op 30 juni 1995 112,90, wat neerkomt op een samengestelde
  jaarlijkse waardevermindering van 2,7 procent. De Amerikaanse CPI
  steeg in dezelfde periode van 24 naar 152,50, wat resulteerde in een
  cumulatieve inflatie van 635 procent.} Hun verontwaardiging
onderstreept de waarheid dat `waar je staat, wordt bepaald door waar je
zit.'

\begin{quote}
Soms vroeg ik me af hoe ik zo'n diepe meelevende pijn kon voelen voor
het lot van een paar mannen die ik niet kende, terwijl zij in een
honkbalstadion honderden mijlen verderop een wedstrijd speelden tegen
een groep vreemden. Het antwoord is eenvoudig. Ik hield van mijn teams.
Hoewel er risico aan verbonden was, was betrokkenheid de moeite waard.
Sport deed mijn bloed sneller stromen, maakte me opgewonden en liet mijn
hart bonzen. Ik vond het fijn iets op het spel te hebben. Het leven kwam
tot leven tijdens een wedstrijd. -- Craig Lambert
\end{quote}

Het zou misleidend zijn alle negatieve gevoelens die tijdens de
aankomende transitiecrisis opkomen, louter toe te schrijven aan een
kille hunkering om ten koste van anderen te leven. Er spelen echter nog
meer factoren mee. De aard van de menselijke samenleving wijst erop dat
er onvermijdelijk een verkeerd geplaatste morele dimensie meespeelt in
de naderende ludditische reactie. Zie het als een kaal verlangen dat
zich achter een morele façade verschuilt. Wij belichten de morele en
moralistische aspecten van de transitiecrisis. Bewust zelfzuchtig
grijpen zet mensen echter veel minder in beweging dan een op
zelfrechtvaardigde woede gebaseerde impuls. Hoewel de band met de
burgerlijke mythes uit de twintigste eeuw snel verzwakt, bestaan er nog
altijd echte gelovigen. Iedereen die in die eeuw volwassen werd, raakte
doordrenkt met de plichten en verantwoordelijkheden van het burgerschap
uit die tijd. De overgebleven morele imperatieven uit de industriële
samenleving zullen ten minste enkele neo-luddistische aanvallen op
informatietechnologieën in gang zetten.

In dit opzicht blijkt het te verwachten geweld deels een uiting te zijn
van wat wij `moreel anachronisme' noemen: het toepassen van morele
richtlijnen uit de ene economische periode op de omstandigheden van een
andere. Iedere fase van de samenleving vraagt om eigen morele normen die
mensen helpen de specifieke prikkelvalkuilen te overwinnen die bij dat
levenspatroon horen. Net zoals een agrarische samenleving niet
functioneerde volgens de morele regels van een rondtrekkende Eskimostam,
kan de informatiesamenleving niet voldoen aan de imperatieven die ooit
het succes van een militante industriële staat in de twintigste eeuw
mogelijk maakten. Daar lichten we graag toe.

De komende jaren zal moreel anachronisme zich ook in de kernlanden van
het Westen duidelijk manifesteren, net zoals het de afgelopen vijf
eeuwen aan de wereldranden te zien was. Westerse kolonisten en militaire
expedities veroorzaakten dit soort crises zodra zij in aanraking kwamen
met inheemse jagers- en verzamelaarsstammen en volkeren wiens
samenlevingen nog volgens agrarische modellen functioneerden. Het
invoeren van nieuwe technologieën in anachronistische omgevingen leidde
tot verwarring en morele crisisbeelden. Het succes van christelijke
zendingswerken bij de bekering van miljoenen inheemse mensen is voor een
groot deel te danken aan de lokale crises die ontstonden door de
plotselinge invoering van nieuwe externe machtsverhoudingen. Dergelijke
botsingen herhaalden zich keer op keer, van de zestiende eeuw tot de
vroege decennia van de twintigste eeuw. Wij verwachten vergelijkbare
conflicten in het begin van het nieuwe millennium, nu de
informatiesamenleving de industriële systemen begint af te lossen.

\subsection{De nostalgie voor dwang}\label{de-nostalgie-voor-dwang}

De opkomst van het soevereine individu zal niet door iedereen als een
veelbelovende, nieuwe fase in de geschiedenis worden omarmd; zelfs
degenen die er het meeste voordeel uit halen, hebben bedenkingen. Veel
mensen zullen innovaties afkeuren die de territoriale natiestaat
ondermijnen. Het ligt in de menselijke natuur dat iedere ingrijpende
verandering vrijwel altijd als een dramatische achteruitgang wordt
beschouwd. Vijfhonderd jaar geleden beweerden de hofleden rondom de
hertog van Bourgondië dat de opkomende innovaties, welke het feodalisme
ondermijnden, slecht waren. Zij meenden dat de wereld snel in verval zou
raken, precies op het moment dat latere historici tijdens de Renaissance
een explosie van menselijk potentieel constateerden. Evenzo zal dat wat
later door het volgende millennium als een nieuwe Renaissance wordt
gezien, voor de vermoeide geesten van de twintigste eeuw beangstigend
overkomen.

Er is een grote kans dat degenen die door deze nieuwe methoden gekrenkt
worden, en vele anderen die erdoor benadeeld raken, onaangenaam zullen
reageren. Hun nostalgie naar dwang zal waarschijnlijk in geweld
uitmonden. Ontmoetingen met deze nieuwe `Luddieten' zorgen er in ieder
geval voor dat de overstap naar radicaal nieuwe vormen van sociale
organisatie voor iedereen voor wat ellende zal zorgen. Zet je schrap en
bereid je voor op de duik. Nu de snelheid van verandering de morele en
economische aanpassingscapaciteit van velen in onze generatie te boven
gaat, kun je rekenen op een felle, verontwaardigde weerstand tegen de
Informatie-Revolutie, ondanks de grote belofte die zij inhoudt om de
toekomst te bevrijden.

Je moet beseffen dat je je moet voorbereiden op dit soort vervelende
ontwikkelingen. Er wacht een transitiecrisis op ons. Sporadisch zullen
deflatoire beproevingen uitbreken -- denk bijvoorbeeld aan de Aziatische
besmetting die in 1997 en 1998 vanuit het Verre Oosten Rusland en andere
opkomende economieën overspoelde -- omdat de verouderde nationale en
internationale instituties uit het industriële tijdperk de uitdagingen
van de nieuwe, verspreide, transnationale economie simpelweg niet
aankunnen. De nieuwe informatie- en communicatietechnologieën
ondermijnen de moderne staat sterker dan elke andere politieke
bedreiging voor haar overheersing sinds Columbus de zee bevoer. Dit is
van groot belang, want machthebbers reageren zelden vreedzaam op
ontwikkelingen die hun autoriteit ondermijnen. Daarom kun je er nu ook
vanuit gaan dat zij niet op een vreedzame manier zullen optreden.

De botsing tussen het nieuwe en het oude zal de eerste jaren van het
nieuwe millennium bepalen. Wij voorzien een periode van groot gevaar en
grote beloningen, waarin op sommige terreinen de beschaving sterk
achteruitgaat en op andere terreinen ongekende kansen ontstaan. Steeds
autonomere individuen en failliete, wanhopige regeringen zullen elkaar
tegenoverkomen aan beide zijden van een nieuwe kloof. Wij verwachten
dat, nog voordat de transitie afgerond is, de aard van soevereiniteit
radicaal zal worden heringericht en de politiek vrijwel zal verdwijnen.
In plaats van staatsdominantie en de beheersing van hulpbronnen zal
vrijwel elke overheidsdienst uiteindelijk naar de private sector
verschuiven. Om onontkoombare redenen -- die we in dit boek uitvoerig
bespreken -- zal informatietechnologie de capaciteit van de staat
zodanig ondermijnen dat zij niet meer in staat is om voor haar diensten
meer te vragen dan deze daadwerkelijk waard zijn voor de burgers.

\begin{quote}
Regeringen zullen zich moeten bezighouden met wat soevereiniteit
betekent. -- Robert Martin
\end{quote}

\subsection{Soevereiniteit door
markten}\label{soevereiniteit-door-markten}

In een mate die we tien jaar geleden nauwelijks konden voorstellen,
verkrijgen individuen via marktmechanismen steeds meer vrijheid ten
opzichte van territoriale natiestaten. Deze staten zien hun gezag snel
verdwijnen en lopen het risico failliet te gaan. Hoe machtig ze ook
lijken, behouden zij enkel de mogelijkheid om te vernietigen, niet om
bevelen uit te delen. Hun intercontinentale raketten en vliegdekschepen
functioneren inmiddels als relikwieën, even imposant als zinloos,
vergelijkbaar met het laatste oorlogspaard uit het feodalisme.

Informatietechnologie breidt de markten drastisch uit doordat zij de
wijze verandert waarop activa worden gecreëerd en beschermd. Dit is
werkelijk revolutionair; het belooft zelfs een nog ingrijpendere impact
te hebben op de industriële samenleving dan de introductie van buskruit
ooit had op het feodale landbouwsysteem. De transformatie rond het jaar
2000 betekent de commercialisering van soevereiniteit en het einde van
de traditionele politiek -- precies zoals wapens ooit de ondergang van
het eedgebonden feodalisme inluidden. Het traditionele burgerschap zal
verdwijnen, net zoals de ridderlijkheid ooit verdween.

Wij zijn ervan overtuigd dat het tijdperk van de individuele economische
soevereiniteit aanbreekt. Net zoals staalfabrieken,
telefoonmaatschappijen, mijnen en spoorwegen die ooit genationaliseerd
waren, wereldwijd in rap tempo werden geprivatiseerd, zul je spoedig de
ultieme vorm van privatisering zien: de ingrijpende denationalisering
van het individu. Het soevereine individu van het nieuwe millennium
behoort niet langer tot de staatsbezittingen, noch vormt het een
onderdeel op de balans van de schatkist. Na de transitie van het jaar
2000 worden voormalige burgers geen burgers meer, maar klanten.

\subsection{Bandbreedte overwint
grenzen}\label{bandbreedte-overwint-grenzen}

Door soevereiniteit te commercialiseren, krijgt het traditionele
burgerschap binnen de natiestaat een ouderwetse uitstraling,
vergelijkbaar met ridderlijke eeden na de ineenstorting van het feodale
systeem. In plaats van als burgers, verplicht tot belastingbetalingen en
gebonden aan een machtige staat te zijn, worden de soevereine individuen
van de eenentwintigste eeuw klanten van overheden die opereren in een
`nieuwe logische ruimte'. Zij onderhandelen over het minimale
overheidsapparaat dat zij wensen en betalen daarvoor via contractuele
overeenkomsten. De overheden van het informatietijdperk organiseren zich
volgens geheel andere principes dan we de afgelopen eeuwen gewend waren.
Sommige rechtsgebieden en soevereiniteitsdiensten ontstaan via
`assortieve matching', een systeem waarbij affiniteiten -- ook
commerciële -- de basis vormen voor de opbouw van loyaliteit in virtuele
rechtsgebieden.

In zeldzame gevallen kunnen de nieuwe soevereiniteiten restanten zijn
van middeleeuwse organisaties, zoals de 900 jaar oude Soevereine
Militaire Hospitaalorde van St.~Johannes van Jeruzalem, van Rhodos en
van Malta. Beter bekend als de Ridders van Malta, vormt deze orde een
affiniteitsgroep van rijke katholieken met 10.000 leden en een
jaarlijkse inkomstenstroom van enkele miljarden. Zij geven hun eigen
paspoorten, postzegels en geld uit en onderhouden volledige diplomatieke
betrekkingen met zeventig landen. Momenteel onderhandelt de orde met de
Republiek Malta over de terugname van Fort St.~Angelo. Door het fort in
bezit te nemen, verkrijgen de Ridders het ontbrekende element van
territorialiteit, zodat zij als soevereine entiteit erkend kunnen
worden. De Ridders van Malta zouden zo opnieuw een soevereine microstaat
kunnen vormen, gesteund door hun lange geschiedenis. Vanuit Fort
St.~Angelo staken zij in 1565 tijdens de Grote Belegering de Turken de
rug toe en regeerden zij Malta vele jaren, totdat Napoleon hen in 1798
verdreef. Mocht men zien hoe de Ridders van Malta binnen enkele jaren
terugkeren, dan toont dat onweerlegbaar dat het moderne
natiestatenstelsel -- dat na de Franse Revolutie zijn intrede deed --
slechts een tussentijdse fase was in een lange periode waarin
uiteenlopende vormen van soevereiniteit tegelijkertijd bestonden.

Een ander, totaal verschillend model voor postmoderne soevereiniteit op
basis van assortieve matching zie je terug in het
Iridium-satelliettelefoonnetwerk. In eerste instantie lijkt het vreemd
om een mobiele telefoniedienst als een vorm van soevereiniteit te
beschouwen, maar internationale autoriteiten hebben Iridium al erkend
als een virtueel land. Zoals je wellicht weet, biedt Iridium een
wereldwijde mobiele telefoniedienst waardoor abonnees via één nummer
oproepen kunnen ontvangen, waar ze zich ook bevinden -- of je nu in
Featherston, Nieuw‑Zeeland bent of in de Boliviaanse Chaco. Om te
garanderen dat de oproepen overal de juiste abonnees bereiken, stemden
internationale telecomautoriteiten ermee in Iridium als een virtueel
land te erkennen, compleet met een eigen landcode: 8816. Het is dan ook
een kleine sprong van een virtueel land van satelliettelefoonabonnees
naar de soevereiniteit van meer samenhangende virtuele gemeenschappen op
het World Wide Web, die alle grenzen overstijgen.

Bandbreedte -- oftewel de draagcapaciteit van een communicatiemedium --
is sinds de uitvinding van de transistor sneller gegroeid dan de
rekenkracht. Als deze trend zich voortzet, zijn wij ervan overtuigd dat
de bandbreedte binnen enkele jaren -- kort na de millenniumwisseling --
zodanig toeneemt dat men de `metaverse' kan realiseren, de alternatieve
cyberspacewereld zoals bedacht door sciencefictionauteur Neal
Stephenson. Stephenson's `metaverse' staat voor een hechte virtuele
gemeenschap met haar eigen regels. Wij geloven dat, naarmate de
cybereconomie groeit, de deelnemers uiteindelijk vrijgesteld worden van
de verouderde wetten van traditionele natiestaten. De nieuwe
cybergemeenschappen zullen minstens even welvarend zijn en hun belangen
net zo effectief behartigen als de \emph{Soevereine Militaire
Hospitaalorde van St.~Johannes van Jeruzalem, van Rhodos en van Malta}.
Dankzij geavanceerde communicatietechnologieën en de mogelijkheden
binnen informatieoorlogvoering weten zij zich krachtiger te profileren.
Wij onderzoeken daarnaast andere modellen van gefragmenteerde
soevereiniteit, waarin kleine groepen de soevereiniteit van zwakkere
natiestaten in huur nemen en hun eigen economische toevluchtsoorden
exploiteren -- vergelijkbaar met hoe vrije havens en vrijhandelszones
tegenwoordig een licentie ontvangen.

Wij hebben een nieuwe morele woordenschat nodig om de relaties tussen
soevereine individuen en de rest van de overheid te beschrijven.
Naarmate de contouren van deze nieuwe verhoudingen duidelijker worden,
zullen veel mensen zich beledigd voelen als `burgers' van de
twintigste-eeuwse natiestaten. Het verdwijnen van naties en de
`denationalisering van het individu' doen enkele gekoesterde opvattingen
vervagen, zoals het principe van `gelijke bescherming onder de wet', dat
uitgaat van machtsverhoudingen die spoedig tot het verleden behoren.
Naarmate virtuele gemeenschappen sterker samenhang ontwikkelen, eisen
zij dat hun leden volgens hun eigen regels ter verantwoording worden
geroepen in plaats van volgens de wetten van de voormalige natiestaten
waarin zij wonen. Binnen hetzelfde geografische gebied zullen opnieuw
meerdere rechtsstelsels naast elkaar functioneren, zoals we in de
oudheid en middeleeuwen zagen.

Net zoals pogingen om de macht van ridders in wapenrusting te handhaven
gedoemd waren te mislukken tegenover buskruitwapens, raken de
hedendaagse opvattingen van nationalisme en burgerschap snel irrelevant
door de opmars van microtechnologie. Ze zullen uiteindelijk belachelijk
lijken, net zoals de heilige principes van het vijftiende-eeuwse
feodalisme in de zestiende eeuw tot spot werden blootgelegd. De
gekoesterde burgerlijke waarden van de twintigste eeuw blijken
lachwekkende anachronismen voor nieuwe generaties na de transformatie
van het jaar 2000. De Don Quichot van de eenentwintigste eeuw wordt geen
dwaalridder die de glorie van het feodalisme doet herleven, maar
transformeert in een bureaucraat in een bruin pak -- een belastinginner
die erop uit is om burgers te controleren.

\section{Herleving van de wetten van de
mark}\label{herleving-van-de-wetten-van-de-mark}

Wij beschouwen overheden zelden als concurrerende spelers, behalve in de
zeer algemene zin, waardoor onze moderne intuïtie over de reikwijdte en
mogelijkheden van soevereiniteit verzwakt raakt. Vroeger, toen
machtsverhoudingen het voor groepen bemoeilijkten een stabiel monopolie
op dwang te vestigen, was de macht vaak verspreid, lagen rechtsgebieden
deels over elkaar en oefenden diverse entiteiten één of meer kenmerken
van soevereiniteit uit. Vaak bleek de zogenaamde heerser in de praktijk
nauwelijks enige macht te bezitten. Tegenwoordig concurreren overheden
die minder sterk zijn dan natiestaten actief om op lokaal niveau een
dwangmonopolie uit te oefenen. Deze concurrentie heeft geleid tot
veranderingen in de wijze waarop geweld wordt beheerst en loyaliteit
wordt gewonnen -- veranderingen die spoedig weer zichtbaar zullen
worden.

Toen de macht van heren en vorsten nog beperkt was en de aanspraken van
één of meer groepen op grensgebieden elkaar vermengden, kon geen van
beide partijen de overhand krijgen. In de middeleeuwen kende men
talrijke grensgebieden -- de zogenaamde `marches' -- waar
soevereiniteiten in elkaar overliepen. Deze conflictrijke gebieden
hielden decennialang, zo niet eeuwenlang, stand in de grenzen van
Europa. Men vond marches tussen gebieden onder Keltische en Engelse
invloed in Ierland; tussen Wales en Engeland; tussen Schotland en
Engeland; tussen Italië en Frankrijk; tussen Frankrijk en Spanje; tussen
Duitsland en de Slavische grensstreken van Centraal-Europa; en tussen de
christelijke koninkrijken van Spanje en het islamitische koninkrijk
Granada. Dergelijke grensgebieden ontwikkelden kenmerkende
institutionele en juridische vormen, en we verwachten dat we deze in het
komende millennium opnieuw zullen aantreffen. Doordat twee rivaliserende
autoriteiten fungeerden, betaalden de inwoners van deze streken zelden
belasting. Bovendien konden zij doorgaans zelf bepalen wiens wetten zij
zouden volgen, door middel van juridische instrumenten als `avowal' en
`distraint' -- concepten die tegenwoordig vrijwel geheel zijn verdwenen.
Wij voorzien dat zulke principes een prominente rol zullen spelen in het
recht van informatiesamenlevingen.

Het overstijgen van nationaliteit

Voor de opkomst van de natiestaat was het lastig om precies vast te
stellen hoeveel soevereine entiteiten de wereld telde, omdat zij op
ingewikkelde wijze in elkaar overliepen en allerlei vormen van
organisatie hanteerden. Dat zal in de toekomst opnieuw zo zijn. Binnen
de natiestaat werden de grenzen tussen territoria scherp afgebakend,
maar in het informatietijdperk zullen deze lijnen weer vervagen. In het
nieuwe millennium raakt soevereiniteit opnieuw gefragmenteerd en duiken
er nieuwe entiteiten op die slechts enkele, maar niet alle kenmerken
bezitten die we met regeringen associëren.

Sommige van deze nieuwe entiteiten -- vergelijkbaar met de Tempeliers en
andere religieuze militaire orden uit de middeleeuwen -- kunnen met
aanzienlijke rijkdom en militaire kracht opereren, ook al beschikken zij
niet over een vast grondgebied. Ze organiseren zich op basis van
principes die niets met nationaliteit te maken hebben. In de
middeleeuwen ontleenden leden en leiders van religieuze organisaties,
die in delen van Europa soevereine macht uitoefenden, hun gezag niet aan
een nationale identiteit. Zij vertegenwoordigden diverse etnische
achtergronden en verklaarden hun trouw aan God te verschuldigen, in
plaats van aan een band die geacht wordt eigen te zijn aan mensen van
een bepaalde nationaliteit.

Koopliedenrepublieken van cyberspace

U zult tevens zien dat samenwerkingsverbanden tussen kooplieden en
vermogende individuen met semisouvereine bevoegdheden weer opbloeien,
zoals de Hanze -- een middeleeuwse confederatie van kooplieden -- deed.
De Hanze, actief op de Franse en Vlaamse markten, groeide uit tot een
organisatie die kooplieden uit zestig steden verenigde.\footnote{Janet
  L. Abu-Lughod, Voorafgaand aan de Europese hegemonie: het
  wereldsysteem A.D.1250-1350 (Oxford: \emph{Oxford University Press},
  1991), p.~62.} De `Hanseatie League' -- zoals men haar in het Engels
op een overbodige wijze aanduidt (letterlijke vertaling: `Leaguely
League') -- vormde een verbond van Germaanse koopmansgilden dat haar
leden bescherming bood en handelsverdragen onderhandelde. Ze kreeg in
diverse steden in Noord-Europa en de Baltische regio semisouvereine
bevoegdheden. Vergelijkbare entiteiten zullen in het nieuwe millennium
opkomen als vervanging van de stervende natiestaat, doordat zij
bescherming bieden en bijdragen aan het handhaven van contracten in een
onveilige wereld.

Kortom, de toekomst zal waarschijnlijk niet voldoen aan de verwachtingen
van degenen die de burgerlijke mythes van de industriële samenleving van
de twintigste eeuw hebben geïnternaliseerd. Onder hen bevinden zich ook
de illusies van de sociale democratie, die ooit de meest getalenteerde
geesten opzwepen en motiveren. Zij gaan ervan uit dat samenlevingen zich
ontwikkelen op een wijze die past bij de wensen van overheden -- het
liefst als reactie op opiniepeilingen en nauwkeurig getelde stemmen. Dit
bleek echter nooit zo evident als vijftig jaar geleden werd gedacht.
Vandaag de dag is het een anachronisme, net zozeer een overblijfsel van
het industrialisme als een roestige schoorsteen. De burgerlijke mythes
laten niet alleen een denkwijze zien die maatschappelijke problemen als
oplosbaar via technische ingrepen beschouwt, maar onthullen ook een vals
vertrouwen dat hulpbronnen en individuen in de toekomst even kwetsbaar
blijven voor politieke dwang als in de twintigste eeuw. Wij betwijfelen
dat. Het zijn marktkrachten -- niet politieke meerderheden -- die
samenlevingen zullen dwingen zich opnieuw in te richten op manieren die
de publieke opinie noch zal begrijpen, noch zal verwelkomen. Als dat
gebeurt, blijkt de naïeve opvatting dat geschiedenis is wat mensen
willen dat zij is, buitengewoon misleidend.

Het is daarom essentieel dat u de wereld met vernieuwde ogen bekijkt.
Dat betekent dat u de zaken van buitenaf benadert en veel
vanzelfsprekend geachte aannames in vraag stelt, zodat nieuwe inzichten
kunnen ontstaan. Als u er niet in slaagt het conventionele denken te
overstijgen in een tijd waarin dit denken de aansluiting met de
werkelijkheid verliest, loopt u het risico ten prooi te vallen aan een
epidemie van desoriëntatie. Die desoriëntatie leidt tot fouten die uw
bedrijf, uw investeringen en uw levensstijl in gevaar kunnen brengen.

\begin{quote}
Het universum beloont ons als we het doorgronden en straft ons als we
dat niet doen. Als wij het universum doorgronden, slagen onze plannen en
voelen wij ons op ons best. Maar als wij proberen te vliegen door van
een klif te springen en simpelweg met onze armen te fladderen, zal het
universum ons ten val brengen.\footnote{Jack Cohen en Ian Stewart, De
  ineenstorting van chaos (New York: \emph{Viking}, 1994).} -- JACK
COHEN AND IAN STEWART
\end{quote}

\subsection{De wereld met nieuwe ogen
zien}\label{de-wereld-met-nieuwe-ogen-zien}

Om u voor te bereiden op de toekomst, moet u begrijpen waarom de wereld
die gaat komen anders is dan de voorspellingen van de meeste experts.
Dit betekent dat u grondig moet kijken naar de onzichtbare oorzaken van
verandering. Wij hebben geprobeerd dit inzichtelijk te maken met een
onconventionele analyse, die wij de studie van megapolitics noemen. In
de twee eerdere delen, Blood in the Streets en \emph{The Great
Reckoning}, betoogden we dat de voornaamste oorzaken van verandering
niet te vinden zijn in politieke manifesten of in de uitspraken van
overleden economen, maar in de verborgen factoren die de grenzen bepalen
waarover macht wordt uitgeoefend. Vaak zorgen subtiele wijzigingen in
klimaat, topografie, micro-organismen en technologie ervoor dat de
logica van geweld verschuift. Deze veranderingen transformeren ook de
manier waarop mensen hun levensonderhoud organiseren en zich verdedigen.

Bovendien wijkt onze benadering van het doorgronden van
wereldverandering sterk af van die van de meeste voorspellers. Wij
beweren niet dat wij meer weten over bepaalde `onderwerpen' dan degenen
die hun hele carrière hebben gewijd aan diepgaande specialisatie.
Integendeel, wij bekijken de zaken juist van buitenaf en beschikken over
kennis van de onderwerpen waarop wij onze voorspellingen baseren. Voor
ons draait het er vooral om te herkennen waar de grenzen van noodzaak
liggen. Als die grenzen veranderen, verandert de samenleving
onvermijdelijk, ongeacht wat mensen erover wensen.

Wij zijn van mening dat de sleutel tot het begrijpen van de ontwikkeling
van samenlevingen schuilt in het doorgronden van de factoren die de
kosten en baten van geweld bepalen. Elke menselijke samenleving, van
jagerstammen tot grote rijken, wordt gevormd door de wisselwerking van
megapolitieke krachten die de dominante versie van de `wetten van de
natuur' vastleggen. Het leven is overal complex. Lammeren en leeuwen
bewaken een fragiele balans, waarbij zij subtiel op elkaar inwerken. Als
leeuwen ineens sneller zouden worden, vangen zij prooien die voorheen
ontsnapten. En stel dat lammeren plotseling vleugels kregen, dan zouden
leeuwen gaan verhongeren. Het vermogen om geweld in te zetten en zich
ertegen te verdedigen is de cruciale factor die het leven aan de rand
beïnvloedt.

Om gegronde redenen plaatsen wij geweld centraal in onze theorie over
megapolitics. Het beheersen van geweld vormt het grootste dilemma voor
iedere samenleving. Zoals we schreven in \emph{The Great Reckoning}:

\begin{quote}
De reden dat mensen tot geweld overgaan, is simpelweg dat het vaak
loont. Op zekere wijzen is het meest voor de hand liggende wat een man
kan doen als hij geld wil, het het gewoon pakken ervan. Dit geldt
evenzeer voor een leger dat een olieveld verovert als voor een enkele
crimineel die zomaar een portemonnee pakt. Macht zoekt, zoals William
Playfair schreef, `altijd de gemakkelijkste weg naar rijkdom door
degenen aan te vallen die erover beschikken.'
\end{quote}

\begin{quote}
De uitdaging voor voorspoed ligt juist in het feit dat roofzuchtig
geweld onder bepaalde omstandigheden zeer winstgevend is. Oorlog
herschrijft de spelregels, verandert de verdeling van eigendommen en
inkomen en bepaalt zelfs wie er leeft en wie sterft. Juist het feit dat
geweld loont, maakt het zo moeilijk te beheersen.\footnote{Zie James
  Dale Davidson en Lord William Rees-Mogg, \emph{The Great Reckoning},
  2e editie (New York: \emph{Simon \& Schuster}, 1993), p.~53.}
\end{quote}

Het denken in deze termen hielp ons ontwikkelingen te voorspellen
waarover zelfs de meest doorgewinterde experts ervan overtuigd waren dat
ze nooit zouden plaatsvinden. Zo was bijvoorbeeld \emph{Blood in the
Streets}, dat begin 1987 verscheen, onze poging de eerste signalen in
kaart te brengen van de grootschalige megapolitieke revolutie die zich
nu ontvouwde. Destijds stelden wij dat technologische vernieuwing de
mondiale machtsverhoudingen radicaal zou verstoren. Onze belangrijkste
stellingen waren:

\begin{itemize}
\item
  Wij voorspelden dat het Amerikaanse overwicht afnam, wat zou leiden
  tot economische onevenwichtigheden en tegenspoed, waaronder een
  beurscrash in de stijl van 1929. Hoewel de experts vrijwel unaniem
  weigerden te geloven dat zoiets kon gebeuren, schokten de markten in
  oktober 1987 -- binnen zes maanden -- met de meest heftige verkoopgolf
  van de eeuw.
\item
  Wij waarschuwden de lezers voor de ineenstorting van het communisme.
  Ook deze voorspelling riep schertsaligheden op bij de experts, maar in
  1989 deden zich onverwachte gebeurtenissen voor: de Berlijnse Muur
  viel en revoluties deden de communistische regimes, van de Baltische
  staten tot zelfs Boekarest, verdwenen.
\item
  Wij legden uit waarom het multi-etnische rijk, dat door de
  bolsjewistische nomenklatura de tsaristische erfenis had overgenomen,
  onvermijdelijk uiteen zou vallen. Eind december 1991 hing de hamer- en
  sikkelvlag voor de laatste keer boven het Kremlin, waarna de
  Sovjetunie voorgoed ophield te bestaan.
\item
  Midden in de periode van intensieve wapenopbouw onder Reagan
  voorspelden wij dat de wereld op de drempel stond van een ingrijpende
  ontwapening. Ook dit werd als onwaarschijnlijk -- zo niet belachelijk
  -- afgedaan, maar in de daaropvolgende zeven jaar vond wel de meest
  ingrijpende ontwapening plaats sinds het einde van de Eerste
  Wereldoorlog.
\item
  Terwijl experts in Noord-Amerika en Europa naar Japan wezen als bewijs
  dat overheden markten succesvol konden manipuleren, stelden wij het
  tegendeel vast. Wij voorspelden dat de bloeiperiode van de Japanse
  financiële activa zou eindigen in een ineenstorting. Niet lang na de
  val van de Berlijnse Muur stortte de Japanse aandelenmarkt in en
  verloor bijna de helft van haar waarde. Wij blijven ervan overtuigd
  dat dit dieptepunt het verlies van 89 procent, zoals Wall Street na
  1929 heeft geleden, kan evenaren of zelfs overtreffen.
\item
  Terwijl bijna iedereen -- van het middenklassegezin tot de grootste
  vastgoedinvesteerders ter wereld -- ervan overtuigd leek dat
  vastgoedmarkten enkel zouden stijgen, waarschuwden wij voor een
  naderende vastgoedcrisis. Binnen vier jaar verloren
  vastgoedinvesteerders wereldwijd meer dan \$1 biljoen toen de
  vastgoedwaarden kelderden.
\item
  Al ruim voordat experts de achteruitgang in de inkomens van arbeiders
  onderkenden, voorspelden wij in \emph{Blood in the Streets} dat deze
  daling structureel zou blijven bestaan. Nu, bijna een decennium later,
  begint een slapende wereld eindelijk in te zien dat dit klopt. Het
  gemiddelde uurloon in de Verenigde Staten is gedaald tot een niveau
  lager dan tijdens de tweede Eisenhower-administratie. In 1993 bedroeg
  het geannualiseerde gemiddelde uurloon in constante dollars \$18.808,
  terwijl dat in 1957, tijdens Eisenhower's tweede ambtstermijn, nog
  \$18.903 was.
\end{itemize}

Hoewel de centrale thema's uit \emph{Blood in the Streets} achteraf
gezien verrassend nauwkeurig bleken, noemden de hoeders van het
conventionele denken deze inzichten nog enkele jaren geleden pure onzin.
In 1987 bestempelde een recensent in \emph{Newsweek} onze analyse als
`een ondoordachte aanval op de rede', waarmee hij de bekrompenheid van
de laat-industriële mentaliteit treffend illustreerde.

Je zou denken dat \emph{Newsweek} en vergelijkbare publicaties inmiddels
inzagen dat onze analysemethode waardevolle inzichten bood in hoe de
wereld veranderde. Maar helemaal niet. De eerste uitgave van \emph{The
Great Reckoning} werd met dezelfde gniffelende vijandigheid ontvangen
als \emph{Blood in the Streets}. Zelfs het \emph{Wall Street Journal}
wees onze analyse resoluut af als het geklets van `je domme tante'.

Alle gegiechel terzijde, bleken de thema's van \emph{The Great
Reckoning} minder belachelijk dan de hoeders van de orthodoxie deden
vermoeden.

\& We hebben onze voorspelling over de ondergang van de Sovjet-Unie
verder uitgewerkt door te onderzoeken waarom Rusland en de andere
voormalige Sovjetrepublieken een toekomst tegemoet zouden gaan vol
toenemende burgerlijke onrust, hyperinflatie en dalende
levensstandaarden.

\begin{itemize}
\item
  We legden uit waarom de jaren negentig een decennium van krimp zouden
  worden, met voor het eerst een wereldwijde inkrimping van zowel
  overheidsinstanties als bedrijven.
\item
  We voorspelden tevens dat een ingrijpende herdefiniëring van de
  voorwaarden voor inkomensherverdeling op handen stond, wat zou leiden
  tot flinke bezuinigingen op uitkeringen. Overal -- van Canada tot
  Zweden -- deden de eerste signalen van een fiscale crisis zich voor,
  en Amerikaanse politici begonnen te spreken over `het beëindigen van
  de welvaartsstaat zoals wij die kennen.'
\item
  Wij voorspelden en legden uit dat de `nieuwe wereldorde' uiteindelijk
  als een `nieuwe wereldwanorde' zou uitpakken. Lange tijd vóórdat de
  gruweldaden in Bosnië de krantenkoppen overspoelden, waarschuwden wij
  dat Joegoslavië in een burgeroorlog zou instorten.
\item
  Nog vóórdat Somalië in volledige anarchie verviel, legden wij uit hoe
  de dreigende ineenstorting van regeringen in Afrika ervoor zorgde dat
  externe machten in sommige landen de touwtjes in handen kregen.
\item
  Wij voorspelden en verduidelijkten dat de militante islam het marxisme
  als leidende ideologie in de confrontatie met het Westen zou
  verdringen. Langer nog vóórdat de bomaanslag in Oklahoma plaatsvond en
  men probeerde het \emph{World Trade Center} op te blazen, lieten wij
  zien waarom de Verenigde Staten te maken kregen met een toename van
  terrorisme.
\item
  Nog vóórdat de krantenkoppen over rellen in Los Angeles, Toronto en
  andere steden verschenen, legden wij uit hoe de opkomst van criminele
  subculturen onder stedelijke minderheden de weg vrijmaakte voor
  wijdverspreid crimineel geweld.
\item
  Wij voorspelden ook `de laatste depressie van de twintigste eeuw', die
  in 1989 in Azië begon en zich vanaf de periferie naar het centrum van
  het mondiale systeem verspreidde. Wij stelden dat de Japanse
  aandelenmarkt het voorbeeld van Wall Street na 1929 zou volgen, wat
  uiteindelijk zou uitmonden in een kredietcrisis en depressie. Hoewel
  overheidsinterventies in Japan en elders tijdelijk verhinderden dat de
  markten de verslechtering van de kredietvoorwaarden volledig
  weerspiegelden, verplaatsten en verergerden die maatregelen de
  economische problemen slechts -- waardoor de druk voor competitieve
  devaluaties en een systemische kredietinstorting, zoals in de jaren
  dertig wereldwijd optrad, gestaag toenam.
\end{itemize}

\emph{De grote afrekening} belichtte ook een reeks controversiële these
die noch bevestigd zijn, noch het door ons voorspelde
ontwikkelingsniveau hebben bereikt:

\begin{itemize}
\item
  Wij stelden dat de Japanse aandelenmarkt het spoor van Wall Street na
  1929 zou volgen, wat zou resulteren in een kredietcrisis en depressie.
  Hoewel de werkloosheidscijfers in Spanje, Finland en enkele andere
  landen hoger waren dan in de jaren dertig en sommige landen --
  waaronder Japan -- wel lokale depressies kenden, heeft zich tot op
  heden geen systemische kredietinstorting voltrokken die de
  wereldeconomie, zoals in de jaren dertig, deed instorten.
\item
  Wij betoogden dat het instorten van het bevel- en controlesysteem in
  de voormalige Sovjet-Unie zou leiden tot de verspreiding van
  kernwapens in handen van ministaatjes, terroristen en criminele
  bendes. Tot grote opluchting van de wereld is dat niet gebeurd,
  althans niet in de mate die wij vreesden. Persberichten melden dat
  Iran diverse tactische kernwapens op de zwarte markt heeft aangeschaft
  en dat de Duitse autoriteiten meerdere pogingen tot de verkoop van
  nucleaire materialen hebben verijdeld. Er is overigens geen melding
  gemaakt van daadwerkelijke inzet of gebruik van kernwapens uit het
  arsenaal van de voormalige Sovjet-Unie.
\item
  Wij legden uit waarom de `oorlog tegen drugs' een recept is voor het
  ondermijnen van politie- en rechtssystemen in landen met wijdverspreid
  drugsgebruik, vooral in de Verenigde Staten. Dankzij jaarlijks
  tientallen miljarden dollars aan verborgen monopoliewinsten beschikken
  drugshandelaren over zowel de middelen als de prikkel om zelfs
  schijnbaar stabiele landen te corrumperen. Hoewel wereldmedia af en
  toe berichten publiceren die wijzen op een hoge mate van infiltratie
  van het Amerikaanse politieke systeem door drugsgeld, is het volledige
  verhaal nog niet verteld.
\end{itemize}

\subsection{Kijken waar anderen niet
kijken}\label{kijken-waar-anderen-niet-kijken}

Hoewel sommige van onze voorspellingen achteraf fout bleken of met de
huidige kennis als onjuist worden bestempeld, blijft ons geheel bij
nader inzien staan. Veel van de ontwikkelingen die waarschijnlijk een
rol gaan spelen in de economische geschiedenis van de jaren negentig
werden al eerder voorspeld en toegelicht in \emph{The Great Reckoning}.
Wij voorspelden niet simpelweg een voortzetting van de bestaande trends,
maar wezen op ingrijpende breuken met wat sinds de Tweede Wereldoorlog
als normaal werd beschouwd. We waarschuwden dat de jaren negentig
drastisch anders zouden zijn dan de voorgaande vijf decennia. Als je het
nieuws van 1991 tot en met 1995 volgt, zie je dat de thema's uit
\emph{The Great Reckoning} bijna dagelijks werkelijkheid werden.

Wij interpreteren deze ontwikkelingen niet als losse tegenslagen, maar
als schokgolven langs één doorlopende breuklijn. De oude orde wordt op
zijn kop gezet door een megapolitieke aardbeving die instellingen
revolutioneert en de manier waarop kritische geesten de wereld
beschouwen, radicaal verandert.

Ondanks de cruciale rol die geweld speelt in het functioneren van onze
wereld, krijgt het opvallend weinig serieuze aandacht. De meeste
politieke analisten en economen doen alsof geweld slechts een
onbeduidend irritantje is -- vergelijkbaar met een vlieg die rond een
taart zoemt -- en niet de chef-kok die de taart heeft gebakken.

\subsection{Een andere megapolitieke
pionier}\label{een-andere-megapolitieke-pionier}

Sterker nog, er is zo weinig helder nagedacht over de rol van geweld in
de geschiedenis dat een bibliografie met alle megapolitieke analyses op
één vel papier zou passen. In \emph{The Great Reckoning} baseerden we
onze redenering onder meer op een bijna volledig vergeten klassieker
over megapolitieke analyse, namelijk \emph{An Enquiry into the Permanent
Causes of the Decline and Fall of Powerful and Wealthy Nations} van
William Playfair, gepubliceerd in 1805. Eén van onze uitgangspunten is
tevens het werk van Frederic C. Lane. Lane, een historicus die zich
onder meer richtte op de middeleeuwen, publiceerde in de jaren '40 en
'50 verschillende scherpe essays over de rol van geweld in de
geschiedenis. Wellicht was zijn essay \emph{Economic Consequences of
Organized Violence}, dat in 1958 in het \emph{Journal of Economic
History} verscheen, het meest omvattende van deze werken. Buiten de
kringen van professionele economen en historici heeft slechts een
enkeling het gelezen en lijken de meesten de ware betekenis ervan niet
te hebben doorgrond. Net als Playfair richtte Lane zich tot een publiek
dat op dat moment nog niet eens bestond.

\subsection{Inzichten voor het
informatietijdperk}\label{inzichten-voor-het-informatietijdperk}

Lane publiceerde zijn werk over geweld en de economische betekenis van
oorlog ruim voor de intrede van het informatietijdperk. Hij schreef
zeker niet met het oog op microverwerking of andere technologische
revoluties die zich tegenwoordig ontvouwen. Toch bieden zijn inzichten
in geweld een kader om te begrijpen hoe de informatie-revolutie de
samenleving herstructureert.

Het raam dat Lane naar de toekomst opende, bood hem juist een blik op
het verleden.

Hij was een middeleeuwse historicus, met name gespecialiseerd in de
handelsstad Venetië, waar het fortuin op en neer schommelde in een
wereld vol geweld.

Toen hij nadacht over de opkomst en ondergang van Venetië, viel hem op
dat de manier waarop geweld wordt georganiseerd en gecontroleerd een
cruciale rol speelt bij het benutten van schaarse middelen.\footnote{Frederic
  C. Lane, `Economische gevolgen van georganiseerd geweld', \emph{The
  Journal of Economic History}, vol.~18, nr. 4 (december 1958), p.~402.}

Wij zijn ervan overtuigd dat Lanes analyses over de concurrerende
toepassingen van geweld ons veel inzicht geven in hoe het leven in het
informatietijdperk waarschijnlijk zal verlopen.

Maar verwacht niet dat de meeste mensen zo'n extreem abstract betoog
zullen opmerken, laat staan het daadwerkelijk volgen.

Terwijl de wereldsfeer zich richt op partijdige debatten en excentrieke
persoonlijkheden, glipt de dynamiek van de megapolitiek bijna
onopgemerkt voorbij.

De gemiddelde Noord-Amerikaan besteedt waarschijnlijk honderd keer meer
aandacht aan O.J. Simpson dan aan de nieuwste microtechnologieën, die
zijn baan overbodig lijken te maken en het politieke systeem, waarop hij
vertrouwt voor werkloosheidsuitkeringen, ondermijnen.

\section{De ijdelheid van wensen}\label{de-ijdelheid-van-wensen}

Het fundamenteel belangrijke over het hoofd zien is niet alleen
voorbehouden aan de bankzitter die tv kijkt. Conventionele denkers van
alle pluimage merken op dat één van de voorwendselen van de natiestaat
is dat de overtuigingen die mensen koesteren, bepalen hoe de wereld
verandert. Blijkbaar schieten zelfs schijnbaar verfijnde analisten
tekort als het gaat om verklaringen en voorspellingen, en interpreteren
zij ingrijpende historische ontwikkelingen alsof deze op een wenselijke
en voorbestemde wijze tot stand komen. Een opvallend voorbeeld van dit
type redenering verscheen op de opiniestaf van de \emph{New York Times}
in een artikel van Nicholas Colchester, precies op het moment dat wij
`Vaarwel, natiestaat, hallo\ldots wat?' schreven. Niet alleen behandelde
hij het onderwerp -- de ondergang van de natiestaat, precies het thema
dat wij bespreken -- maar plaatste hij zichzelf ook als een treffend
voorbeeld van hoe ver ons denken van de norm afwijkt. Colchester is geen
eenvoudige denker; hij was redactioneel directeur van de \emph{Economist
Intelligence Unit}. Als iemand een realistisch wereldbeeld belichaamt,
dan is dat ongetwijfeld hij. Toch betoogt hij in zijn artikel op
meerdere punten dat `de komst van een internationale overheid' nu
onvermijdelijk is.

Waarom? Omdat de natiestaat wankelt en niet langer in staat is de
economische krachten in toom te houden.

Wij vinden dat deze aanname bijna tot het absurde leidt. Het is een
drogreden om te veronderstellen dat er een nieuw bestuursmodel zal
ontstaan enkel omdat het bestaande gefaald heeft. Volgens die logica
hadden Haïti en Zaïre al lang beter bestuur moeten kennen, simpelweg
omdat wat zij hadden zo overduidelijk ontoereikend was.

Het standpunt van Colchester -- breed gedeeld door de weinigen in
Noord-Amerika en Europa die over zulke zaken nadenken -- houdt totaal
geen rekening met de grotere megapolitieke krachten die bepalen welke
politieke systemen werkelijk levensvatbaar zijn. Dat vormt de kern van
dit boek. Wanneer we de technologieën die het nieuwe millennium
vormgeven meenemen, zullen we waarschijnlijk niet één wereldregering
krijgen, maar eerder microregeringen of zelfs anarchistische toestanden.

Als men de rol van geweld in het vaststellen van de regels waar iedereen
zich aan houdt serieus wil onderzoeken, valt het op dat er tientallen
boeken zijn geschreven over de ingewikkeldheden van tarwe-subsidies en
honderden over obscure aspecten van monetair beleid. Het gebrek aan een
doordachte benadering van de cruciale kwesties die echt de loop van de
geschiedenis bepalen, weerspiegelt waarschijnlijk de relatieve
stabiliteit van de machtsstructuur in de afgelopen eeuwen. De vogel die
op de rug van een nijlpaard in slaap valt, maakt zich geen zorgen over
zijn zitplek totdat het dier daadwerkelijk in beweging komt. Dromen,
mythen en fantasieën spelen een veel grotere rol in het vormgeven van
wat men de `sociale wetenschappen' noemt dan men gemeenlijk vermoedt.

Dit komt vooral duidelijk naar voren in de overvloed aan literatuur over
economische rechtvaardigheid. Bij elke pagina die nauwgezet beschrijft
hoe geweld de samenleving vormgeeft en daarmee de grenzen bepaalt
waarbinnen economieën opereren, vinden we miljoenen woorden die ingaan
op economische (on)rechtvaardigheid. Toch gaan hedendaagse opvattingen
uit van een samenleving die wordt beheerst door een dwingend instrument
met de kracht om het goede in het leven ontnemen en vervolgens te
herverdelen. Een dergelijke macht bestaat in de moderne tijd maar enkele
generaties, en nu neemt die macht af.

\subsection{Big Brother over de sociale
zekerheid}\label{big-brother-over-de-sociale-zekerheid}

De industriële technologie voorzag overheden in de twintigste eeuw van
meer controle-instrumenten dan ooit tevoren. Voor een tijd leek het
onvermijdelijk dat overheden het geweld zo effectief monopoliseerden dat
er nauwelijks nog ruimte bleef voor individuele autonomie. Halverwege de
eeuw verlangde niemand naar een triomf van het soevereine individu.

Enkele van de scherpste waarnemers uit de midden twintigste eeuw
concludeerden dat de drang van natiestaten om de macht te centraliseren
uiteindelijk zou leiden tot totalitaire overheersing van alle
levensdomeinen. In George Orwells \emph{1984} (1949) hield Big Brother
het individu nauwlettend in de gaten, terwijl hij vergeefs probeerde ook
maar een spoor van autonomie en eigenwaarde te bewaren. Het leek een
verloren strijd. Friedrich von Hayeks \emph{The Road to Serfdom} (1944)
benaderde het onderwerp op een wetenschappelijke manier en betoogde dat
vrijheid ten onder ging aan een nieuwe vorm van economische controle,
waardoor de staat de absolute macht kreeg. Deze werken schreven men vóór
de opkomst van microprocessoren, die later een scala aan technologieën
introduceerden waarmee zelfs kleine groepen en individuen onafhankelijk
van de centrale autoriteit konden opereren.

Hoe scherp Hayek en Orwell ook waren, ze bleken uiteindelijk te
pessimistisch. De geschiedenis bracht verrassingen aan het licht. Het
totalitaire communisme hield nauwelijks stand tot 1984. Een nieuwe
variant van lijfeigenschap zou in het volgende millennium nog kunnen
ontstaan als overheden erin slagen de bevrijdende kracht van
microtechnologie te onderdrukken. Maar veel waarschijnlijker is dat we
ongekende kansen en meer individuele autonomie zullen ervaren. De zorg
en zekerheid die onze ouders hen boden, blijkt mogelijk helemaal geen
probleem te vormen. Wat zij als vaste en blijvende kenmerken van het
sociale leven beschouwden, lijkt nu gedoemd te verdwijnen. Wanneer
noodzaken de grenzen van onze keuzes bepalen, passen we ons aan en
organiseren we ons leven ernaar.

\subsection{De risico's van
voorspellen}\label{de-risicos-van-voorspellen}

Zonder twijfel riskeren we een deel van onze schaarse waardigheid door
te trachten zowel ingrijpende veranderingen in onze levensorganisatie
als in de cultuur die ons verbindt te voorspellen en te verklaren. De
meeste voorspellingen blijken uiteindelijk belachelijk; hoe dramatischer
de voorspelde verandering, hoe beschamender zij uitpakken. De wereld
gaat niet ten onder, de ozonlaag verdwijnt niet en de voorspelde ijstijd
maakt plaats voor wereldwijde opwarming. Ondanks alle waarschuwende
signalen zit er nog steeds olie in de tank. Meneer Antrobus, de
doorsnee-man uit \emph{The Skin of Our Teeth}, weet te voorkomen dat hij
bevriest, overleeft oorlogen en dreigende economische rampen en wordt
oud, terwijl hij de weldoordachte waarschuwingen van experts links laat
liggen.

De meeste pogingen om de toekomst te `onthullen' komen al snel komisch
over. Zelfs als eigenbelang ons scherp zet, blijft onze blik op de
toekomst vaak kortzichtig. In 1903 verklaarde \emph{Mercedes}: `Er
zullen nooit zoveel als 1 miljoen auto's wereldwijd zijn, omdat het
onwaarschijnlijk is dat 1 miljoen ambachtslieden tot chauffeurs kunnen
worden opgeleid.'\footnote{Norman Macrae, \emph{regeringen in verval},
  \emph{Cato Policy Report}, juli/augustus 1992, p.~10.}

Dit besef zou ons tot zwijgen moeten brengen, maar dat gebeurt niet. We
schuwen het niet om onze welverdiende portie spot te incasseren. Als het
ons ernstig misloopt, mogen toekomstige generaties naar hartenlust
lachen, op voorwaarde dat zich nog iemand herinnert wat wij ooit
verkondigden. Het durven uiten van een gedachte houdt altijd het risico
in dat je het mis hebt. We zijn beslist niet zo stijve en inflexibele
figuren dat we bang zijn fouten te maken. Integendeel, we verkiezen het
om ideeën te durven delen die u wellicht van nut kunnen zijn, in plaats
van ze repressief onder te drukken uit angst dat ze achteraf overdreven
of gênant blijken te zijn.

Arthur C. Clarke merkte scherp op dat er twee essentiële redenen zijn
waarom voorspellingen over de toekomst doorgaans mislukken: `gebrek aan
moed en gebrek aan verbeeldingskracht.' Hij voegde toe dat `gebrek aan
moed wel vaker voorkomt; het treedt op wanneer, ondanks dat alle
relevante feiten bekend zijn, de voorspeller niet inziet dat ze
onvermijdelijk tot één conclusie leiden. Sommige van deze voorspellingen
zijn zo belachelijk dat ze bijna ongelooflijk
lijken.'\footnote{Arthur C. Clarke, \emph{profielen van de toekomst: een
  onderzoek naar de grenzen van het mogelijke} (Londen: \emph{Victor
  Gollancz Ltd.}, 1962), p.~13.}\footnote{Ibid.}

Als onze benadering van de informatierevolutie tekortschiet -- en dat
zal onvermijdelijk gebeuren -- komt dat eerder door een gebrek aan
verbeeldingskracht dan door een gebrek aan moed. Het voorspellen van de
toekomst is immers altijd een gedurfde onderneming die terecht scepsis
oproept. Wellicht zal de tijd aantonen dat onze conclusies compleet mis
liggen. In tegenstelling tot Nostradamus gedragen wij ons niet als
profeten. Wij doen geen vage voorspellingen door met een toverstok in
een kom water te roeren of door horoscopen op te stellen, en wij
schrijven ook geen cryptische verzen. Wij willen u een heldere en
objectieve analyse bieden van vraagstukken die mogelijk van groot belang
voor u blijken te zijn.

Wij vinden het onze plicht onze ideeën uiteen te zetten -- ook al komen
ze voor sommigen als ketters over -- omdat ze anders wellicht
onopgemerkt blijven. In de gesloten denkwijze van de
laat-industrialiseerde samenleving gaan ideeën niet zo vrij rond als via
de gevestigde media zou moeten.

Dit boek is tot stand gekomen vanuit een constructieve instelling. Het
is het derde boek dat we gezamenlijk schrijven, waarin we de
verschillende fasen van de ingrijpende verandering die momenteel gaande
is analyseren. Net als \emph{Blood in the Streets} en \emph{The Great
Reckoning} vormt dit een denkexperiment. Het onderzoekt hoe de
industriële samenleving ten onder gaat en vervolgens opnieuw vorm
krijgt. Wij voorzien dat de komende jaren verbluffende paradoxen aan het
licht komen. Enerzijds ervaar je de opkomst van een nieuwe vrijheid,
waarin het soevereine individu centraal staat, en mag je verwachten dat
de productiviteit vrijwel volledig bevrijd wordt. Tegelijkertijd zien
wij het verval van de moderne natiestaat voor ons. Veel zekerheden over
gelijkheid -- die in de twintigste eeuw als vanzelfsprekend werden
beschouwd -- zullen met die staat verdwijnen. Wij zijn ervan overtuigd
dat de representatieve democratie, zoals we die nu kennen, zal
wegsterven en plaatsmaakt voor een keuzedemocratie in de digitale
marktplaats. Indien onze bevindingen kloppen, wordt de politiek van de
volgende eeuw veel diverser en minder belangrijk dan wat we nu gewend
zijn.

Wij geloven dat ons betoog helder te volgen zal zijn, ook al betreedt
het het intellectuele equivalent van de achterwouden en slechte buurten.
Mocht op bepaalde plaatsen onze intentie niet glashelder overkomen, dan
komt dat niet doordat wij speels te werk gaan of door de eeuwenoude
gewoonte van dubbelzinnigheid aannemen -- zoals die vaak wordt
gehanteerd door mensen die doen alsof ze de toekomst kunnen voorspellen
met cryptische uitspraken. Wij hanteren geen dubbelzinnigheid. Indien
onze argumenten onduidelijk lijken, ligt dat er vooral aan dat we er
niet in geslaagd zijn onze overtuigende ideeën simpel en toegankelijk te
verwoorden. In tegenstelling tot veel voorspellers willen wij dat je
onze gedachtegang doorgrondt en zelfs overneemt. Onze visie steunt niet
op spirituele zweverijen of kosmische schommelingen, maar op ouderwetse,
ongepolijste logica. Om zeer logische redenen zijn we ervan overtuigd
dat microprocessing de natiestaat onvermijdelijk ondermijnt en
vernietigt, terwijl het tegelijkertijd nieuwe vormen van sociale
organisatie voortbrengt. Het is zowel noodzakelijk als mogelijk dat jij
al enkele aspecten van de nieuwe levenswijze voor je ziet -- aspecten
die wellicht sneller werkelijkheid worden dan je verwacht.

\subsection{Ironieën van een voorspelde
toekomst}\label{ironieuxebn-van-een-voorspelde-toekomst}

Al eeuwenlang wordt het einde van dit millennium beschouwd als een
beladen moment in de geschiedenis. Meer dan 850 jaar geleden stelde
St.~Malachy vast dat het jaar 2000 de datum van het Laatste Oordeel zou
zijn. In 1934 voorspelde de Amerikaanse helderziende Edgar Cayce dat de
aarde in 2000 op haar as zou draaien, waardoor Californië in tweeën zou
splijten en New York City en Japan overstromingen zouden ondervinden. In
1980 verklaarde de Japanse raketwetenschapper Hideo Itokawa dat de
uitlijning van de planeten in een `Grand Cross' op 18 augustus 1999
grootschalige milieucatastrofes zou veroorzaken, met als uiteindelijk
gevolg het einde van het menselijk leven op aarde.\footnote{A.~T. Mann,
  \emph{millenniumprofetieën: voorspellingen voor het jaar 2000}
  (Shafiesbury, Engeland: Element Books, 1992), pp.~88, 112, 117.}

Apocalyptische visioenen worden al snel tot spot gedonderd. Toch is het
jaar 2000 -- ondanks het indrukwekkende ronde getal -- slechts een
willekeurig onderdeel van de door het Westen overgenomen christelijke
kalender. Andere kalenders en tijdsmaatstaven rekenen eeuwen en
millennia vanaf verschillende referentiepunten. Volgens de islamitische
kalender komt 2000 A.D. overeen met het jaar 1378, wat op het eerste
gezicht als een gewoon jaartal lijkt. In de Chinese kalender, die zich
elke zestig jaar herhaalt, valt 2000 in een drakenjaar, onderdeel van
een voortdurende cyclus die al millennia doorgaat. Maar het jaar 2000
draagt meer in zich dan louter religieuze betekenis, want zijn symboliek
put hij niet alleen uit de christelijke traditie, maar ook uit de
beperkingen van de informatietechnologie midden in de eeuw.

Het zogenaamd Y2K-probleem -- een potentieel verwoestende logische fout
in miljarden regels computercode -- dreigde op middernacht van het
millennium de vitale onderdelen van de industriële samenleving plat te
leggen. Veel computers en microprocessoren draaien op software uit de
allereerste dagen van de computer, een tijd waarin geheugenruimte,
verhandeld voor ongeveer \$600.000 per megabyte, kostbaarder was dan
goud. Om dure opslagruimte te besparen noteerden vroege programmeurs
jaartallen uitsluitend met de laatste twee cijfers. Deze werkwijze,
waarbij datumvelden uit twee cijfers bestonden, vond zijn weg in de
meeste software voor mainframecomputers en werd eveneens toegepast in
personal computers en in embedded chips -- microprocessoren die vrijwel
alle apparaten aansturen, van videorecorders tot
auto-ontstekingssystemen, van beveiligingssystemen tot telefoons, en
zelfs de systemen die het telefonienetwerk regelen, alsmede in de
proces- en besturingssystemen in fabrieken, elektriciteitscentrales,
olieraffinaderijen, chemische fabrieken, pijpleidingen en meer. Kortom,
als 1999 in een twee-cijferig veld wordt weergegeven, verschijnt dat als
`99'. Het probleem doet zich voor wanneer het jaar 2000 als `00' wordt
afgekort, waarna veel computers dit interpreteren als 1900. Hierdoor
kunnen tal van niet-geüpdatete computers en andere digitale apparaten
het jaar 2000 onjuist registreren.

Het gevolg kan grove datacorruptie zijn, een fenomeen dat onbedoeld het
nieuwe potentieel voor informatieoorlogvoering illustreert. In het
informatietijdperk kunnen tegenstanders aanzienlijke schade aanrichten
door `logische bommen' te laten ontploffen, die de werking van cruciale
systemen ondermijnen door de gegevens waarop zij vertrouwen te
corrumperen.

Tijdens een militaire oefening hoef je niet per se een vliegtuig neer te
schieten als je de gegevens, die essentieel zijn voor een veilige
werking, kunt verstoren. Datacorruptie kan bijna net zo verwoestend zijn
als fysieke wapens wanneer het gaat om het ontregelen van een moderne
samenleving. De mogelijke verstrekkende gevolgen hiervan zouden bij
nader inzien voor de hand moeten liggen. Zo meldde de \emph{Mail} van
Londen op 14 december 1997 dat luchtvaartmaatschappijen wereldwijd van
plan waren op 1 januari 2000 honderden vluchten te annuleren, uit angst
dat de luchtverkeersleidingssystemen zouden falen. Niet alleen deze
systemen, maar ook de datumgevoelige functies in de vliegtuigen zelf
lopen risico. Volgens Boeing hebben veel vliegtuigen Y2K-aanpassingen
nodig. Bovendien kunnen talrijke apparaten storingen vertonen wanneer ze
een gebeurtenis op een ongeldige datum registreren. De computergestuurde
fly-by-wire-systemen die vliegtuigen aansturen, kunnen defect raken als
zij zijn geprogrammeerd te veronderstellen dat cruciaal onderhoud voor
het laatst in 1900 heeft plaatsgevonden; ze kunnen zelfs in een foutlus
vastlopen en daardoor uitschakelen.

De potentieel dodelijke kettingreacties van een logische tijdbom die
cruciale besturingssystemen platlegt, zouden de eeuwwisseling op een
onvergetelijk onaangename wijze kunnen markeren. Bedenk dat allerlei
apparaten in een foutlus kunnen raken en uitschakelen -- en dat kan
gebeuren, ook als je het geluk hebt niet midden in de lucht te zitten
wanneer het nieuwe millennium aanbreekt.

Zorg ervoor dat je ongelukken voorkomt, of ze nu ontstaan door
pacemakers die niet Y2K-conform zijn of door dronken millenniumvierders.
Als zulke pacemakers falen, kan ook het telefoonsysteem het begeven,
waardoor de ambulance mogelijk niet arriveert.

Tenzij je in Brazilië of Oekraïne woont, ben je er toch aan gewend dat
je de telefoon oppakt of de autotelefoon inschakelt en meteen een
kiestoon hoort. Gelukkig hoef je je maar zelden te verdiepen in de
technische werking van het systeem. Maar uit onderzoek blijkt dat de
schakelapparatuur en routers van telefoonnetwerken sterk afhankelijk
zijn van datumvelden. Alle verbindingen worden opgeslagen met datum en
tijd, wat essentieel is voor het berekenen van gespreksduur en de daarop
gebaseerde facturering. Als je op 31 december 1999 om 11:59:30 een
gesprek van precies één minuut voert en het systeem om 12:00:00 dit
gesprek registreert met een negatieve duur van meer dan 99 jaar, kunnen
foutlussen ontstaan en schakelt het systeem mogelijk uit. Hoewel
langafstandsfirma's enorme bedragen investeren in het updaten van hun
schakelaars zodat deze Y2K-conform werken -- en men er van uitgaat dat
lokale dienstverleners dat ook doen -- kan het hele netwerk in de
problemen komen als zelfs een paar kleinere bedrijven niet voldoen en
uitvallen. Dan heb je geluk als je op 1 januari 2000 nog een kiestoon
krijgt.

Y2K-expert Peter de Jager verwoordt het als volgt: `Als we het vermogen
verliezen om een telefoongesprek te voeren, verliezen we alles. We
verliezen elektronische geldtransfers, de handel en het bankieren via
filialen.' De gevolgen van Y2K-fouten zouden bovendien nog verder
reiken.

Tegenwoordig weet niemand precies in hoeverre cruciale systemen door het
millenniumprobleem ten onder zullen gaan. Embedded systemen -- die niet
herprogrammeerd kunnen worden en dus vervangen moeten worden zodra zij
datumgevoelige fouten vertonen -- vind je in kaarten, vrachtwagens en
bussen die na 1976 gebouwd zijn. Misschien raak je dan niet betrokken
bij een ongeluk met voertuigen bestuurd door mensen met niet-conforme
pacemakers, aangezien hun voertuigen mogelijk niet starten. Ook in
energiecentrales, water- en rioleringssystemen, medische apparatuur,
militaire uitrusting, vliegtuigen, offshore olieplatforms, olietankers,
alarmsystemen en liften komen embedded systemen veel voor. Hoewel veel
installaties met microprocessors zelf geen datumgevoelige functies
uitvoeren, zijn ze vaak wel afhankelijk van een klok -- die mogelijk
Y2K-gevoelig is -- voor hun interne werking.

\section{Mainframes en de
Y2K-tijdbom}\label{mainframes-en-de-y2k-tijdbom}

De grootschalige commando- en controlesystemen van de overheid en grote
bedrijven, waarin enorme transacties via mainframecomputers worden
verwerkt, stonden aanvankelijk centraal in de Y2K-bezorgdheid. Deze
systemen draaien op grote machines met software die vaak decennialang
oud en grotendeels niet Y2K-conform is. Daarom richtten de eerste
Y2K-waarschuwingen -- voor het eerst geuit door Peter de Jager begin
jaren negentig -- zich voornamelijk op de noodzaak om de
besturingssystemen van grote, multiprocessor-mainframes te moderniseren.
De heer de Jager uitte zijn zorg dat er mogelijk niet genoeg
programmeurs bedreven in COBOL -- de oude mainframetaal -- beschikbaar
zouden zijn om de noodzakelijke patches en reparaties aan datumgevoelige
code uit te voeren, ook al had elk bedrijf en elke overheidsinstantie
met een kwetsbaar systeem enkele jaren geleden een noodprogramma
opgestart. Aangezien dit niet is gebeurd en veel beheerders van
datumgevoelige informatiesystemen pas recent begonnen zijn met hun
kwetsbaarheidsbeoordeling, mag je er met grote zekerheid van uitgaan dat
veel mainframesystemen niet goed voorbereid zullen zijn om soepel te
blijven draaien in het jaar 2000.

Dit vormt uiteraard een groot probleem, omdat er echt geen alternatief
bestaat voor computerverwerking, zoals de economie tegenwoordig
functioneert. De meeste bedrijven die groot genoeg zijn om een mainframe
te gebruiken voor het verwerken van hun transacties, zijn namelijk
afhankelijk van een transactievolume dat met de ouderwetse papieren
systemen uit de negentiende eeuw niet beheersbaar zou zijn. Als die
ondernemingen noodgedwongen weer op papieren documenten zouden moeten
vertrouwen, zouden zij slechts een fractie van hun gebruikelijke
transacties kunnen afhandelen. De inkomensschok als gevolg van zo'n
dramatische daling in bedrijfsactiviteiten zou het voortbestaan van
bijna alle bedrijven -- op enkele van de best gekapitaliseerde na --
ernstig in gevaar brengen.

Bijna alle processen rondom geld -- zoals facturering, inkoop- en
loonadministratie, voorraadbeheer en naleving van regelgeving -- raken
compleet in de war. Enorme hoeveelheden data gaan verloren als computers
crashen of als, door het Y2K-probleem, verkeerde gegevens beginnen te
stromen. In sommige gevallen blijkt het zelfs een zegen als systemen
direct crashen, in plaats van dat hun data geleidelijk worden
gecorrumpeerd tot een massale storing het probleem aan de kaak stelt.
Wat gebeurt er met bestanden wanneer een backup-utility bestanden die
oorspronkelijk op 07/04/99 zijn aangemaakt, kopieert naar een update op
01/04/00? Wie kan dat zeggen? Zal de computer een betaling uitvoeren op
4 januari `1900' voor een verzekeringspolis interpreteren als een
signaal dat de polis al een eeuw in gebreke is, met als gevolg dat de
polis wordt geannuleerd en uit de administratie verdwijnt? Zullen
computers van banken en financiële instellingen proberen om honderd jaar
rente in rekening te brengen voor leningen die de overgang naar het
nieuwe millennium overspannen? Zullen uw banken en
effectenmaatschappijen nauwkeurig uw rekeningstanden bijhouden en u
tijdig toegang geven tot uw tegoeden? Dit zijn slechts enkele van de
intrigerende vraagstukken waarmee u te maken krijgt als gevolg van het
Y2K-probleem.

\begin{quote}
`Dit is mogelijk het meest destructieve onderdeel van het jaar
2000-probleem. Dit betreft niet het ongemak wanneer uw salaris enkele
dagen te laat komt. Dit gaat over echte chaos op straat. - Dr.~Leon
Kappelman, medevoorzitter van de Year 2000-werkgroep van de
\emph{Society For Information Management}'
\end{quote}

Ook moet u zich afvragen wat er gebeurt als de elektriciteit uitvalt
door Y2K-gerelateerde storingen. Zonder stroom functioneren zelfs de
meeste systemen -- ook die welke niet direct door Y2K-problemen
getroffen zijn -- niet, zoals uw koelkast, vriezer en mogelijk zelfs uw
verwarming. Y2K-nalevingskwesties kunnen de veiligheidsgerelateerde
toegangs- en beheerssystemen in kerncentrales verstoren. Zo dragen
medewerkers in kerninstallaties dosimetrie-apparaten die de hoeveelheid
stralingsblootstelling bijhouden die zij in de faciliteit ontvangen.
Deze apparaten worden regelmatig gecontroleerd en de gegevens over de
blootstellingsniveaus worden opgeslagen in een computersysteem dat de
toegang van medewerkers tot de faciliteit beheert. Uiteraard, als deze
bestuurlijke computers uitvallen, raken al die uitgebreide controles --
die bedoeld zijn om een veilige werking en goed onderhoud te garanderen
-- volledig in de war. Belangrijker nog: een memo van de \emph{Nuclear
Regulatory Commission} merkt op dat veel `niet-veiligheidsgerelateerde,
maar belangrijke computergebaseerde systemen -- hoofdzakelijk databases
en de dataverzameling die onmisbaar is voor de werking van de centrale'
datumgevoelig zijn.

Ook reguliere elektriciteitscentrales lopen net zo goed risico op
Y2K-verstoringen. Kolen gestookte centrales zijn immers gevoelig voor
onderbrekingen in het transportsysteem dat de kolen naar de ketels
vervoert. In het winterverwarmingsseizoen van 1997--1998 moesten
exploitanten van kolengestookte elektriciteitscentrales in sommige
gevallen hun productie verminderen door vertragingen in de spoorlevering
van Westerse kolen, veroorzaakt door de fusie van de spoorwegsystemen
van Southern Pacific en Union Pacific. Het probleem ontstond door
incompatibiliteiten tussen de computerbestuurde controle- en
dispatchsystemen die beide spoorwegmaatschappijen hanteerden. Een
woordvoerder van Union Pacific noemde de integratie van de twee systemen
een `nachtmerrie', ondanks dat \emph{Union Pacific Technologies} als
koploper wordt gezien in de ontwikkeling van geautomatiseerde
transportsystemen. Door de programmeerproblemen kon de
spoorwegmaatschappij de bewegingen van haar goederenwagons niet
nauwkeurig volgen. Het onvermogen van Union Pacific om de systemen van
Southern Pacific succesvol te integreren, is een slecht voorteken voor
wat er kan gebeuren wanneer Y2K-logische tijdbommen het transport, de
elektriciteitsopwekking en andere economische sectoren verstoren.

De grootste zorg rond het elektriciteitsnet komt voort uit het feit dat
computers voortdurend toezicht houden op en het hele systeem aansturen,
zodat zij elektriciteit efficiënt verplaatsen van overschotgebieden naar
regio's met een tekort. Computers monitoren dit proces nauwgezet om
spanningspieken en storingen te voorkomen. Elke overdracht wordt
geregistreerd met vermelding van tijd en datum, vergelijkbaar met de
registratie bij een telefoongesprek. Robuuste mechanische relais maken
de verbindingen weliswaar mogelijk, maar computersystemen sturen ze aan.
Die computercontroles, die essentieel zijn voor de vermogensbalancering,
kunnen echter om dezelfde redenen falen als telefonienetwerken. In
Noord-Amerika verbinden T‑1‑lijnen en telefonische microwavelinks de
vermogensverdelings- en regelingssystemen onderling. Als het
telefonienetwerk uitvalt, kun je erop rekenen dat ook de
elektriciteitsvoorziening instort. En bedenk: de ervaring in Canada in
januari 1998 toont aan hoe lastig het is om het systeem weer aan de
praat te krijgen zodra de elektriciteit over een groot gebied wegvalt.
Zo kan een stroomuitval oncomfortabel lang aanhouden.

\section{Y2K en het nucleaire
arsenaal}\label{y2k-en-het-nucleaire-arsenaal}

In moderne economieën zorgt een stroomstoring in de diepste winter voor
enorme ontwrichting en vormt deze een potentiële bedreiging voor de
gezondheid, vooral voor mensen die afhankelijk zijn van elektrische
verwarming en medische apparatuur. Het ergste scenario is echter nog
grimmiger. John Koskinen, hoofd van het \emph{Y2K Conversion Council}
van president Clinton, waarschuwt dat de Amerikaanse militaire arsenalen
mogelijk niet meer functioneren zodra de klok op 31 december 1999
middernacht slaat. Hoewel hij geen onnodige paniek wil zaaien, voegt
Koskinen eraan toe: `Er moet zich zorgen over worden gemaakt.' Men
vreest dat nucleaire raketten daadwerkelijk kunnen afgaan als de
gegevens niet correct werken. Natuurlijk geldt dezelfde zorg -- zo niet
meer -- voor de Russische nucleaire raketten. Ruslands faillissement
heeft de noodzakelijke upgrades voor Y2K-compatibiliteit nog
problematischer gemaakt dan in de Verenigde Staten, en er is bewijs dat
Rusland de Y2K-conversie nog niet serieus neemt. Hoewel iedereen hoopt
dat er geen toevallige lanceringen plaatsvinden, bestaat er weinig
twijfel dat de jaarwisseling naar 2000 de wereldwijde veiligheid in
gevaar kan brengen, simpelweg doordat militaire communicatiesystemen in
veel landen mogelijk niet naar behoren functioneren. Zoals Koskinen het
verwoordt: `Als je in een land zit en plotseling niet precies kunt
achterhalen wat er gaande is, omdat je communicatiesystemen niet
optimaal werken, raak je nog nerveuzer.' Neem dat dus zeker mee in je
lijst met Y2K-zorgen. Deze logische tijdbom kan de lancering van echte
explosieve wapens bespoedigen, wat het gevaar aantoont dat voortkomt uit
een informatieoorlogvoering tegen gecentraliseerde commando- en
controlesystemen.

Terroristen die een gecentraliseerd systeem willen treffen, kunnen juist
31 december 1999 kiezen als actiedatum, omdat op dat moment veel
systemen hun grootste kwetsbaarheid vertonen. Niet alleen raken de
communicatiesystemen zwaar overbelast -- met als gevolg dat de stroom
uitvalt, voertuigen niet starten en de 911-diensten van politie,
brandweer en ambulance in de lappenmand worden gestuurd -- maar ook tal
van andere voorzieningen, zoals de luchtverkeersleiding, kunnen het
begeven. Zonder stroom ontbreekt het aan kraanwater, vallen
rioolinstallaties uit en kunnen verkeerslichten platgaan. Binnen enkele
uren nadat het transportsysteem instort, kopen mensen het voedsel in
supermarkten op of plunderen zij de winkels. Recente ervaringen in
Amerikaanse steden laten zien dat het ontbreken van stroom, water, voor
velen verwarming, licht en een adequate bereikbaarheid van hulpdiensten
-- waaronder politie en brandweer -- samen het verval van de beschaving
kunnen inluiden. Hoewel niemand met zekerheid kan vaststellen wat de
impact van het Y2K-probleem zal zijn, zou dit kunnen uitmonden in
plunderingen en rellen in de straten, zeker als er breed opschudding
ontstaat over mogelijke problemen bij het uitgeven van loon-, bijstands-
en pensioencheques.

\begin{quote}
`Wij zullen niet zijn wat wij geweest zijn, maar wij zullen beginnen
anders te zijn.'\footnote{Geciteerd in Frooso, op. cit., p.~40.} -
Joachim de Fiore
\end{quote}

Voorspellingen over een ondergang rondom het nieuwe millennium berusten
niet per se op een christelijke theologie, maar sluiten naadloos aan bij
de millennialtraditie van Joachim de Fiore. Zijn meditaties deden hem
immers geloven dat Christus slechts `het tweede scharnier in de
geschiedenis' vormde en dat er onvermijdelijk een volgend scharnier zou
volgen.\footnote{Ibid.} Zo stelt filosoof Michael Grosso dat de
informatierevolutie de menselijke geschiedenis stuurt naar de
verwezenlijking van de profetische visie van de Westerse wereld. Hij
noemt dit `technocalypse.' Of technologische ontwikkelingen nu wel of
niet door millennialvisies worden beïnvloed, blijft het Y2K-fenomeen
immers een product van de overheersende westerse tijdsverbeelding. Op
merkwaardige wijze vullen dromen, dagdromen en visioenen -- of hun
numerieke interpretaties, zoals Newtons commentaar op de profetieën van
Daniël illustreert -- elkaar aan. Deze intuïtieve sprongen ontstaan
vanuit een perspectief dat de geboorte van Christus als middelpunt van
de geschiedenis beschouwt. Bovendien versterken grote, ronde getallen
dit perspectief, want elke handelaar herkent de bijzondere
aantrekkingskracht die zij uitoefenen.

Het tweeduizendste jaar van onze jaartelling kan niet anders dan een
brandpunt worden voor de verbeelding van intuïtieve denkers.

Critici kunnen deze voorspellingen gemakkelijk belachelijk vinden,
zonder eens in te gaan op de dubbelzinnige en omstreden theologische
begrippen van de Apocalyps en het Laatste Oordeel, die juist zoveel
kracht aan deze visioenen geven. Opmerkelijk genoeg overstijgt de
Y2K-computerfout zelfs de rekenfouten die anders het belang van het jaar
2000 binnen het christelijke kader zouden ondermijnen. Het jaar 2000 kan
een keerpunt in de geschiedenis betekenen, simpelweg omdat het de
overgang naar een nieuw millennium markeert. Logischerwijs begint het
volgende millennium pas in 2001, aangezien het jaar 2000 slechts het
tweeduizendste jaar sinds de geboorte van Christus aanduidt. Dat zou
kloppen als Christus in het eerste jaar van ons huidige tijdperk was
geboren -- maar dat was niet zo. In 533, toen men de geboorte van
Christus als uitgangspunt nam in plaats van de oprichtingsdatum van Rome
voor de westerse jaartelling, maakten de monniken die deze nieuwe
conventie invoerden een rekenfout betreffende zijn geboorte.
Tegenwoordig gaan we ervan uit dat hij in 4\,v.Chr. werd geboren.
Volgens deze berekening eindigt de volledige periode van tweeduizend
jaar sinds zijn geboorte al rond 1997. Zo stelt ook Carl Jung -- op zijn
eigen, schijnbaar vreemde wijze -- de startdatum voor een Nieuwe Tijd
vast.

Lach erom als je wilt, maar wij wijzen intuïtieve inzichten in de
geschiedenis beslist niet af. Hoewel ons betoog op logica rust en niet
op vooringenomen veronderstellingen, raken we diep onder de indruk van
de profetische kracht van het menselijk bewustzijn. Keer op keer
bevestigt het de visioenen van krankzinnigen, helderzienden en heiligen.
Hetzelfde geldt voor de transformatie die met het jaar 2000 op handen
lijkt te komen. De datum, die al lang verankerd is in de westerse
verbeelding, lijkt het keerpunt te markeren dat bewijst dat de
geschiedenis een bestemming heeft. Wij kunnen niet precies verklaren
waarom dat zo moet zijn, maar we hebben er desondanks alle vertrouwen
in.

Wij hebben het gevoel dat de geschiedenis een bestemming kent en dat
vrije wil en determinisme twee zijden van dezelfde medaille vormen. De
menselijke interacties die de geschiedenis vormgeven, lijken erop te
wijzen dat een bepaald lot hen stuurt. Net zoals een elektronenplasma --
een dicht samengepakt gas van elektronen -- zich als een complex systeem
gedraagt, handelen mensen op een vergelijkbare manier. De individuele
bewegingsvrijheid van elektronen blijkt immers verenigbaar te zijn met
een sterk geordend collectief gedrag. Zoals David Ohm over een
elektronenplasma opmerkte, is de menselijke geschiedenis `een hoogst
georganiseerd systeem dat zich als een geheel gedraagt.'

Het doorgronden van de werking van de wereld betekent dat je leert
aanvoelen hoe de menselijke samenleving handelt volgens de wiskundige
wetten van de natuur. De werkelijkheid verloopt niet lineair, terwijl de
meeste mensen juist een lineair patroon verwachten. Om de dynamiek van
verandering echt te begrijpen, moet je inzien dat onze samenleving --
net als andere complexe systemen in de natuur -- cyclische patronen en
abrupte breuken vertoont. Dit houdt in dat bepaalde aspecten van de
geschiedenis de neiging hebben zich te herhalen en dat ingrijpende
veranderingen vaak plotseling plaatsvinden in plaats van geleidelijk.

Binnen de cycli die ons leven doordringen, markeert een raadselachtige
cyclus van vijfhonderd jaar keerpunten in de geschiedenis van de
westerse beschaving. Nu het jaar 2000 naderde, viel het op dat het
laatste decennium van elke eeuw die deelbaar is door vijf, altijd
gepaard gaat met een diepgaande transitie binnen de Westerse wereld. Dit
patroon van dood en wedergeboorte wijst op nieuwe fasen van sociale
organisatie, vergelijkbaar met de manier waarop geboorte en dood de
cyclus van menselijke generaties bepalen. Al sinds ten minste 500 v.Chr.
kwam de Griekse democratie op door de constitutionele hervormingen van
Kleisthenes in 508 v.Chr. De volgende vijfhonderd jaar kenden een
periode van groei en versterking van de oude economie, die culmineerde
in de geboorte van Christus in 4 v.Chr. Deze periode bleek tevens de
tijd van grootste welvaart te zijn, waarin de rentevoeten hun laagste
niveau bereikten vóór de aanvang van de moderne tijd.

In de daaropvolgende vijfhonderd jaar nam de welvaart geleidelijk af,
wat uiteindelijk leidde tot de ineenstorting van het Romeinse Rijk tegen
het einde van de vijfde eeuw na Chr. Het is de moeite waard om de
samenvatting van William Playfair hier te herhalen: ``Toen Rome op het
hoogtepunt van zijn grootheid was -- dat bleek samen te vallen met de
geboorte van Christus, dat wil zeggen tijdens het bewind van Augustus --
zal blijken dat het op dezelfde wijze geleidelijk afnam tot het jaar
490.''\footnote{William Playfair, Een onderzoek naar de permanente
  oorzaken van de neergang en val van machtige en rijke naties:
  Ontworpen om aan te tonen hoe de welvaart van het Britse rijk verlengd
  kan worden (Londen: Greenland and Norris, 1805), p.~79.} Toen vielen
de laatste legioenen uiteen en zonk de westerse wereld weg in de donkere
middeleeuwen.

De daaropvolgende vijfhonderd jaar werden gekenmerkt door een krimpende
economie; de langafstandshandel kwam tot stilstand, raakten steden
verlaten, verdween geld uit de omloop en verdwenen kunst en
geletterdheid bijna volledig. De ineenstorting van het West-Romeinse
Rijk leidde tot het verval van een effectief rechtsstelsel, waardoor er
primitieve regelingen ontstonden voor het oplossen van geschillen. Tegen
het einde van de vijfde eeuw kreeg bloedwraak steeds meer gewicht, en in
het jaar 500 vond het eerste geregistreerde geval van een beproeving
door ordeal plaats.

Nog eens: duizend jaar geleden markeerde het laatste decennium van de
tiende eeuw een andere enorme omwenteling in sociale en economische
systemen. Wellicht is de minst bekende van deze transities de feodale
revolutie, die begon in een periode van totale economische en politieke
onrust. In \emph{The Transformation of the Year One Thousand} stelt Guy
Bois, professor middeleeuwse geschiedenis aan de Universiteit van
Parijs, dat deze breuk aan het einde van de tiende eeuw zorgde voor de
volledige ineenstorting van de overblijfselen van oude instituties en de
opkomst van iets nieuws uit de anarchie van het feodalisme.\footnote{Guy
  Bois, De transformatie van het jaar duizend: Het dorp Lournard van de
  oudheid tot het feodalisme (Manchester, Engeland: Manchester
  University Press, 1992).} Zoals Raoul Glaber het verwoordde: `Er werd
gezegd dat de hele wereld in één klap de lompen van de oudheid van zich
afschudde.'\footnote{Ibid., p.~150.} Het nieuwe systeem dat plotseling
opkwam, stimuleerde de geleidelijke heropleving van de economische groei
en zorgde ervoor dat de daaropvolgende vijfhonderd jaar -- die we nu de
middeleeuwen noemen -- gepaard gingen met een heropleving van het
gebruik van geld en internationale handel, naast de herontdekking van
rekenkunde, geletterdheid en tijdsbesef.

In het laatste decennium van de vijftiende eeuw vond een nieuw keerpunt
plaats. Europa herstelde zich uit het door de Zwarte Dood veroorzaakte
demografische tekort en begon vrijwel onmiddellijk zijn heerschappij
over de wereld uit te breiden. De `kruitrevolutie', de `renaissance' en
de `reformatie' benoemden elk een aspect van deze overgangsperiode die
de moderne tijd inluidde. Het begon spectaculair toen Karel~VIII Italië
binnenviel met nieuwe bronzen kanonnen. Dit opende de weg naar nieuwe
werelddelen, zoals bleek uit Columbus' reis naar Amerika in 1492. Die
ontdekking stimuleerde de meest dramatische economische groei die de
mensheid ooit heeft gekend en ging gepaard met een revolutie in
natuurkunde en astronomie, wat de moderne wetenschap deed ontstaan.
Bovendien verspreidde de boekdruktechnologie de ideeën op grote schaal.

Wij bevinden ons nu op de drempel van een nieuwe millenaire
transformatie. De uitgebreide systemen voor commando en controle die we
uit het industriële tijdperk hebben geërfd, kunnen bij de overgang naar
het nieuwe millennium instorten als een eenpaardskar. Of de
Y2K-logicabom nu wel of niet leidt tot een onmiddellijke ineenstorting
van de industriële samenleving, de dagen daarvan zijn geteld. Wij
verwachten dat de opkomst van de informatiesamenleving de wereld
ingrijpend zal veranderen, op manieren die in dit boek worden
toegelicht. Uw twijfels hierin zijn begrijpelijk, daar een cyclus die
zich slechts tweemaal per millennium herhaalt niet genoeg herhalingen
kent voor statistische significantie. Inderdaad, economen benaderen
zelfs kortere cycli met scepsis, omdat zij meer overtuigend statistisch
bewijs verlangen. Professor Dennis Robertson schreef ooit dat we enkele
eeuwen moeten afwachten alvorens met zekerheid te kunnen spreken over de
bestaan van handelscycli van vier en van acht tot tien jaar.\footnote{Geciteerd
  in S. B. Saul, \emph{The Myth of the Great Depression} (London:
  Macmillan, 1985), p.~10.} Volgens dat criterium zou hij ongeveer
dertigduizend jaar moeten wachten om zeker te weten dat de
vijfhonderdjarige cyclus geen toevalligheid betreft. Wij zijn minder
dogmatisch en erkennen liever dat de patronen van de werkelijkheid
complexer zijn dan de statische, lineaire evenwichtsmodellen die de
meeste economen hanteren.

Wij menen dat de komst van het jaar 2000 meer betekent dan slechts een
praktische onderverdeling van een eindeloos tijdscontinuüm. Wij geloven
dat het een keerpunt vormt tussen de Oude Wereld en een opkomende Nieuwe
Wereld. Het industriële tijdperk verdwijnt in hoog tempo, en ironisch
genoeg kan de ondergang ervan worden versneld doordat vroege computers
door hun dure geheugen tweecijferige datumvelden gingen gebruiken. Toen
Hallerith-punchkaarten nog maar tachtig tekens konden bevatten, vond men
het verstandig om datums af te korten. In tegenstelling tot de
verwachtingen van de vroege programmeurs hield deze afkorting vier
decennia stand tot het einde van het millennium, als een potentiële
onopgemerkte logicabom die een groot deel van de industriële samenleving
zou kunnen verwoesten. Het \emph{Office of Management and Budget} van de
Amerikaanse overheid omschreef het probleem in \emph{Getting Federal
Computers Ready for 2000}, een rapport van 7 februari 1997. De
\emph{OMB} concludeert over computers: `Tenzij zij gerepareerd of
vervangen worden, falen zij bij de eeuwovergang op één van drie
manieren: ofwel wijzen zij legitieme invoer af, ofwel berekenen zij
foutieve resultaten, ofwel functioneren zij simpelweg niet.' Deze drie
gevolgen samen zouden de industriële samenleving compleet kunnen
lamleggen. In elk geval staat massaproductietechnologie op het punt te
worden overschaduwd door een nieuwe miniaturisatietechnologie, en een
crisis op korte termijn zal dat proces alleen maar versnellen. Dankzij
de nieuwe informatietechnologie is een wetenschap op het gebied van
niet-lineaire dynamica ontstaan, waarvan de verbluffende conclusies nog
slechts afzonderlijke elementen zijn die samen moeten worden geweven tot
een omvattend wereldbeeld. We leven in het tijdperk van de computer,
maar onze dromen worden nog steeds op het weefgetouw gesponnen. We
houden vast aan de metaforen en denkbeelden van het industrialisme. Onze
politieke arena balanceert nog steeds langs de industriële scheidslijn
tussen rechts en links, zoals denkers als Adam Smith en Karl Marx, die
stierven lang voordat bijna iedereen die nu leeft geboren werd, hebben
aangetoond. Wij stellen echter dat het `gezonde verstand' van het
industriële tijdperk op vele terreinen niet meer opgaat, nu de wereld
radicaal verandert.

Meer dan vijfentachtig jaar na de dag in 1911 waarop Oswald Spengler
werd getroffen door een intuïtie over een naderende wereldoorlog en `de
ondergang van het Westen', zien wij eveneens `\emph{een historische
fasewisseling} plaatsvinden \ldots{} op precies het moment dat daarvoor
al honderden jaren geleden was voorbestemd.'\footnote{Oswald Spengler,
  \emph{The Decline of the West}, vertaald door Charles Francis
  Atkinson, geciteerd in I. F. Clark, \emph{The Pattern of Expectation,
  1644-2001} (London: Jonathan Cape, 1979), p.~220.} Net als Spengler
voorzien wij de naderende ondergang van de westerse beschaving en
daarmee de ineenstorting van de wereldorde die de afgelopen vijf eeuwen
heeft gedomineerd, sinds Columbus naar het westen zeilde om contact te
leggen met de Nieuwe Wereld. Maar in tegenstelling tot Spengler zien wij
in het komende millennium juist de geboorte van een nieuwe fase binnen
de westerse beschaving.

{[}\^{}27{]} De nomenklatura vormt de gevestigde elite die in de
voormalige Sovjet-Unie en andere door de staat beheerde economieën de
leiding voerde.

\bookmarksetup{startatroot}

\chapter{Megapolitieke transformaties in historisch
perspectief}\label{megapolitieke-transformaties-in-historisch-perspectief}

\begin{quote}
`In de geschiedenis, net als in de natuur, zijn geboorte en dood in
evenwicht.' - Johan Huizinga\footnote{Huizinga, op. cit., p.~7.}
\end{quote}

\section{Het verval van de moderne
wereld}\label{het-verval-van-de-moderne-wereld}

Wij zijn ervan overtuigd dat je getuige bent van niets minder dan het
verval van onze moderne tijd. Een meedogenloze, maar verborgen logica
stuurt deze ontwikkeling aan. Meer dan we doorgaans aannemen -- of dan
ook meer dan CNN en de kranten ons doen geloven -- zal het komende
millennium niet langer `modern' zijn. We zeggen dit niet om te
suggereren dat je een barbaars of ouderwets tijdperk te wachten staat --
al is dat mogelijk -- maar om te benadrukken dat het historische toneel
dat zich nu ontvouwt, wezenlijk verschilt van het toneel waarin je bent
opgegroeid.

Er staat iets nieuws te gebeuren. Net zoals landbouwsamenlevingen
fundamenteel verschilden van jager-verzamelaarsgroepen en industriële
samenlevingen radicaal afweken van feodale of kleinschalige
landbouwsystemen, markeert de opkomende Nieuwe Wereld een breuk met
alles wat we eerder kenden.

In het nieuwe millennium houdt de natiestaat op het economische en
politieke leven op gigantische schaal te domineren, zoals dat in de
moderne eeuwen wel het geval was. De beschaving waaraan u de
wereldoorlogen, de lopende band, sociale zekerheid, de
inkomstenbelasting, deodorant en de toasteroven te danken hebt, gaat ten
onder. Misschien blijven deodorant en toasteroven behouden, maar de rest
niet. Net als een oude, eens machtige man telt de toekomst van de
natiestaat nu in jaren en dagen, in plaats van in eeuwen en decennia.

Overheden hebben al veel van hun vermogen om te reguleren en bevelen
ingeleverd. De val van het communisme betekende het einde van een lange
cyclus van vijf eeuwen, waarin de omvang van de macht de efficiëntie van
bestuur overschaduwde. In die periode nam de opbrengst van geweld
gestaag toe -- vandaag de dag geldt dat niet meer. We bevinden ons
midden in een faseovergang van wereldhistorische proporties. De
toekomstige historicus Gibbon, die in het komende millennium de
ondergang van ons ooit moderne tijdperk beschrijft, zou er zelfs voor
kunnen kiezen te stellen dat dat tijdperk al ten einde was op het moment
dat u dit boek leest. Achteraf gezien kunnen wij -- en hij -- stellen
dat het moderne tijdperk eindigde met de val van de Berlijnse Muur in
1989 of met de ineenstorting van de Sovjet-Unie in 1991. Beide momenten
vormen bepalende mijlpalen in de evolutie van de beschaving, het einde
van wat wij nu het Moderne Tijdperk noemen.

De vierde fase in de ontwikkeling van de mensheid staat voor de deur, en
wellicht blijft het minst voorspelbare element daarvan de naam die eraan
vast komt te zitten. Noem het 'postmodern', de 'cybersamenleving' of het
'informatietijdperk'. Of verzin zelf een benaming -- niemand weet welke
conceptuele lijm de volgende fase van de geschiedenis uiteindelijk
samenbindt.

We weten niet eens of de zojuist afgesloten periode van vijfhonderd jaar
nog als 'modern' wordt bestempeld. Als toekomstige historici iets weten
van woordafleidingen, valt dat vast tegen. Een beschrijvende titel had
kunnen luiden: 'Het tijdperk van de staat' of 'Het tijdperk van geweld.'
Maar zo'n benaming past niet binnen het huidige tijdsschema waarin wij
historische perioden indelen. Volgens het Oxford English Dictionary
verwijst 'modern' immers naar het heden en de recente tijden, in
tegenstelling tot het verre verleden. Historisch gezien gebruiken we de
term -- in tegenstelling tot 'oud' of 'middeleeuws' -- meestal voor de
periode na de middeleeuwen.\footnote{\emph{The Compact Edition of the
  Oxford English Dictionary}, vol.~1 (Oxford: \emph{Oxford University
  Press}, 1971), p.~1828.}

Westerlingen begonnen zichzelf pas 'modern' te noemen toen zij inzagen
dat de middeleeuwse periode tot het verleden behoorde. Vóór 1500 zag
niemand de feodale eeuwen als een overgangsperiode binnen de westerse
beschaving. De reden is duidelijk: om een tijdperk als een tussenvorm te
beschouwen, moet dat tijdperk reeds tot een einde zijn gekomen. Mensen
die in de feodale periode leefden, konden zich niet voorstellen dat zij
tussen de oudheid en de moderne beschaving in zaten, totdat het hen
duidelijk werd dat de middeleeuwen voorbij waren en dat hun cultuur
radicaal verschilde van zowel de donkere eeuwen als de
oudheid.\footnote{Michael Hicks, \emph{Bastard Feudalism} (London:
  Longmans, 1995), p.~1.}

Menselijke culturen hebben blinde vlekken. Wij beschikken niet over een
adequaat vocabulaire om de paradigmawisselingen aan de diepste grenzen
van het leven te beschrijven, zeker niet de veranderingen die om ons
heen plaatsvinden. Ondanks de talloze dramatische verschuivingen sinds
de tijd van Mozes heeft slechts een enkeling van de ketters zich
afgevraagd hoe de overgang van de ene beschavingsfase naar de andere
werkelijk plaatsvindt.

Hoe zet je deze transities in gang? Wat hebben ze met elkaar gemeen?
Welke patronen helpen je vaststellen wanneer ze beginnen en wanneer ze
eindigen? Wanneer komen Groot-Brittannië of de Verenigde Staten tot
stilstand? Voor zulke vragen vind je bijna geen conventionele
antwoorden.

\subsection{Het taboe op
vooruitziendheid}\label{het-taboe-op-vooruitziendheid}

Buiten het bestaande systeem treden voelt als wanneer een
toneelmedewerker wanhopig een gesprek probeert af te dwingen met een
personage in een toneelstuk. Dat doorbreekt een conventie die het
systeem in stand houdt. Elke sociale orde koestert het taboe dat haar
burgers niet na zouden moeten denken over hoe het systeem zal eindigen
en welke regels in het nieuwe systeem, dat haar opvolgt, de plaats
zullen innemen. Men gaat impliciet ervan uit dat het huidige systeem het
laatste, ja misschien wel het enige systeem is dat ooit zal bestaan.
Niet dat men dat zo bot uitspreekt -- weinigen die ooit een
geschiedenisboek hebben gelezen, zouden zo'n veronderstelling
realistisch vinden als die expliciet werd geuit. Toch regeert die
conventie onze wereld. Elk sociaal systeem doet alsof zijn regels voor
eeuwig gelden, hoe hard het ook zijn macht probeert te behouden. Het
lijkt het definitieve woord te zijn, of misschien wel het enige.
Primitieve gemeenschappen gaan er namelijk vanuit dat hun werkwijze de
enige manier is om het leven in te richten. Economisch complexere
systemen met historisch besef plaatsen zichzelf vrijwel altijd aan de
top. Of het nu gaat om de Chinese mandarijnen aan het keizerlijk hof, de
marxistische nomenklatura in het Kremlin van Stalin of de leden van het
Huis van Afgevaardigden in Washington, machthebbers veronderstellen óf
dat geschiedenis niet bestaat, óf dat zij zelf de absolute top vormen --
de voorhoede van alles wat nog komt.

Dit gebeurt om bijna onvermijdelijke redenen. Hoe duidelijker men ziet
dat een systeem op zijn einde afstevent, hoe minder mensen zich aan de
bestaande regels zullen houden. Daarom heeft iedere sociale organisatie
de neiging om uitspraken die haar ondergang voorspellen, te ontmoedigen
of als onbelangrijk weg te wuiven. Daardoor worden de grote overgangen
in de geschiedenis zelden opgemerkt terwijl ze gaande zijn. Als je
verder niets weet over de toekomst, kun je er zeker van zijn dat
conventionele denkers dramatische veranderingen noch omarmen, noch
actief aankondigen.

Je kunt de reguliere informatiebronnen niet vertrouwen als het gaat om
een objectieve en tijdige waarschuwing over de veranderingen in de
wereld en de achterliggende redenen. Wil je de grote transitie die
momenteel gaande is echt begrijpen, dan moet je het zelf ontdekken.

\subsection{Voorbij het voor de hand
liggende}\label{voorbij-het-voor-de-hand-liggende}

Dat houdt in dat je verder moet kijken dan wat op het eerste gezicht
evident is. De geschiedenis leert ons dat mensen zelfs wanneer een
transitie achteraf onmiskenbaar echt blijkt te zijn, deze vaak
decennialang of zelfs eeuwenlang niet herkennen zodra zij zich voordoet.
Denk eens aan de val van Rome -- waarschijnlijk de belangrijkste
historische ontwikkeling van het eerste millennium na Christus. Toch
hielden velen, lang nadat Rome was ingestort, vast aan de mythe dat Rome
nog bestond, vergelijkbaar met de manier waarop men het balsemde lichaam
van Lenin tentoonstelde. Wie het nieuws ontving via de voorwendselen van
ambtenaren, kreeg nooit te horen dat Rome ten onder was gegaan, totdat
die informatie al volkomen irrelevant bleek te zijn.

Dat lag niet alleen aan de gebrekkige communicatie in de oudheid. Het
effect zou vergelijkbaar zijn geweest met een situatie waarin \emph{CNN}
op miraculeuze wijze had kunnen uitzenden en in september 476
videobeelden had afgespeeld. In datzelfde moment namen barbaren in
Ravenna Romulus Augustulus, de laatste Romeinse keizer in het Westen,
gevangen en stuurden hem met geweld naar een villa in Campanië, zodat
hij daar als gepensioneerde moest gaan wonen. Zelfs als Wolfe Blitzer
daar met minicams klaarstond om er in 476 verslag van te doen, zou hij
-- net als een ander -- de gebeurtenissen vrijwel zeker niet hebben
aangeduid als het einde van het Romeinse Rijk. En precies dat beweerden
latere historici uiteindelijk te hebben vastgesteld.

Waarschijnlijk had \emph{CNN} nooit een kop keuren met de tekst `Rome is
vanavond gevallen.' De machthebbers ontkenden resoluut de ondergang van
Rome. Nieuwsmedia kiezen zelden een controversiële koers die hun eigen
winst in gevaar zou brengen. Ze mogen wel partijdig zijn, zelfs
overmatig, maar ze presenteren bijna nooit uitspraken die abonnees
zouden doen besluiten hun abonnement op te zeggen en naar de heuvels te
vluchten. Daarom zou men nauwelijks verslag hebben gedaan van de val van
Rome, zelfs als de benodigde technologie dat in 476 toestond. Experts
zouden vervolgens opstaan en beweren dat het belachelijk was om van een
val van Rome te spreken. Anders gezegd, zo'n bewering zou nadelig zijn
geweest voor de bedrijfsbelangen en wellicht zelfs schadelijk voor de
gezondheid van de verslaggevers. In het Rome van de late vijfde eeuw,
waar de machthebbers barbaren waren, ontkenden zij dan ook stellig de
val van Rome.

Maar het ging niet alleen om autoriteiten die verkondigden: `Doe er geen
melding van, anders doden we je.' Een deel van het probleem lag er ook
in dat Rome in de latere decennia van de vijfde eeuw zo in verval was
geraakt dat haar `ondergang' vrijwel onopgemerkt bleef door de meeste
getuigen. Pas een generatie later suggereerde graaf Marcellinus als
eerste dat `het West-Romeinse Rijk met deze Augustulus ten onder ging.'
Nog vele decennia -- misschien wel eeuwen -- verliepen voordat men breed
erkende dat het Romeinse Rijk in het Westen niet langer bestond.
Natuurlijk geloofde Karel de Grote dat hij in het jaar 800 een legitieme
Romeinse keizer was.

Het gaat er niet om dat Karel de Grote en iedereen die na 476 op de
traditionele wijze over het Romeinse Rijk nadacht, dwaas waren.
Integendeel. Het is vaak lastig om sociale ontwikkelingen eenduidig te
karakteriseren. Wanneer de gevestigde machten worden ingezet om een voor
de hand liggende conclusie te staven -- ook al berust die grotendeels op
schijn -- durven alleen mensen met een sterk karakter en uitgesproken
overtuigingen daartegenin te gaan. Als je je in de positie van een
Romein in de late vijfde eeuw plaatst, kun je je gemakkelijk voorstellen
hoe verleidelijk het zou zijn geweest om te concluderen dat er niets was
veranderd. Die zekerheid vormde immers de optimistische conclusie.
Anders denken zou beangstigend zijn geweest. En waarom tot een
angstaanjagende visie komen als er een bemoedigend alternatief
voorhanden lag?

Hoe dan ook had men kunnen stellen dat het zakenleven gewoon doorging
zoals gebruikelijk -- zo ging het immers in vroegere tijden. Het
Romeinse leger -- en dan met name de grensgarnizoenen -- was al
eeuwenlang barbaar geworden. In de derde eeuw riep het leger
routinematig een nieuwe keizer uit. Al in de vierde eeuw werden zelfs de
officieren steeds meer Germaans en vaak ongeletterd. Al vóórdat Romulus
Augustulus werd afgezet, had men talloze keizers met geweld
omvergeworpen. Voor zijn tijdgenoten leek zijn vertrek dan ook niet meer
dan één van de vele omwentelingen in een chaotische periode. En men
stuurde hem weg met een pensioen. Het feit dat hij een pensioen ontving
-- hoewel hij al snel stierf -- bood de geruststelling dat het systeem
in stand bleef.

Voor een optimist bracht Odoacer, die Romulus Augustulus afzette, het
rijk juist samen in plaats van het te verwoesten. Odoacer, zoon van
Attila's handlanger Edecon, was een slimme man. Hij kroonde zichzelf
niet tot keizer. In plaats daarvan riep hij de Senaat bijeen en
overtuigde hij de leden ervan om het keizerschap -- en daarmee de
soevereiniteit over het gehele rijk -- toe te kennen aan Zeno, de
Oost-Romeinse keizer in verre Byzantium. Zo bestuurde Odoacer Italië als
Zeno's patricius.

Zoals Will Durant schreef in \emph{The Story of Civilization}, leken
deze veranderingen niet te duiden op de `ondergang van Rome', maar op
`verwaarloosbare verschuivingen aan de oppervlakte van het nationale
toneel.' Toen Rome ten val kwam, verzekerde Odoacer dat Rome bleef
voortbestaan. Hij deed, net als bijna iedereen, alsof er niets was
veranderd. Men wist dat `de glorie die Rome was' veel meer waard was dan
de barbarij die haar intrede deed. Zelfs de barbaren onderschreven dat.
Zoals C. W. Previte-Orton schreef in \emph{The Shorter Cambridge
Medieval History}, werd het einde van de vijfde eeuw -- toen `de keizers
werden vervangen door barbaarse Germaanse koningen' -- gekenmerkt door
`aanhoudend doen alsof.'

\subsection{`Aanhoudend doen alsof'}\label{aanhoudend-doen-alsof}

Dit `verzinsel' hield in dat men de façade van het oude systeem in stand
hield, ook al was de kern ervan `vervormd door barbarisme.'\footnote{Ibid.,
  p.~131.} De oude regeringsvormen bleven onveranderd toen een barbaarse
`luitenant' de laatste keizer verving. De senaat vergaderde nog steeds.
`De pretorianenprefectuur en andere hoge ambten bleven bestaan en
vooraanstaande Romeinen vervulden deze functies.'\footnote{Ibid.,
  p.~137.} Men stelde nog steeds jaarlijks consuls voor. `De Romeinse
burgerlijke administratie bleef intact in stand.'\footnote{Ibid.} In
zekere zin hield deze administratie stand tot aan het ontstaan van het
feodalisme aan het einde van de tiende eeuw. Bij openbare gelegenheden
hield men de oude keizerlijke insignes in ere. Het christendom bleef de
officiële staatsgodsdienst. Ook gedroegen de barbaren zich alsof zij
trouwplicht hadden aan de Oosterse keizer in Constantinopel en de
tradities van het Romeinse recht in stand hielden. In feite, in de
woorden van Durant, `in het Westen was het grote rijk er niet
meer.'\footnote{Durant, op. cit., p.~43.}

\subsection{So what?}\label{so-what}

Het verre voorbeeld van de val van Rome is om verschillende redenen
relevant wanneer je de huidige wereldomstandigheden bekijkt. Veel boeken
over de toekomst behandelen eigenlijk het heden. Wij hebben geprobeerd
dit gemis weg te werken door ons boek over de toekomst in eerste
instantie als een boek over het verleden te presenteren. Wij geloven dat
je een beter perspectief krijgt op wat de toekomst in petto heeft
wanneer we met concrete historische voorbeelden belangrijke
megapolitieke inzichten over de logica van geweld illustreren.
Geschiedenis is een uitstekende leraar en haar verhalen zijn
fascinerender dan alles wat we ons kunnen voorstellen. Veel van de meest
boeiende verhalen gaan over de val van Rome en leveren lessen op die ook
voor jouw toekomst in het Informatietijdperk van belang kunnen zijn.

Allereerst vormt de val van Rome een van de meest levendige voorbeelden
uit de geschiedenis van wat er gebeurt tijdens een ingrijpende
transitie, wanneer de schaal van het bestuur instort. Ook de transities
rond het jaar 1000 gingen gepaard met het wegvallen van de centrale
autoriteit en vonden plaats op een wijze die de complexiteit en
reikwijdte van economische activiteiten vergrootte. Aan het einde van de
vijftiende eeuw veroorzaakte de buskruitrevolutie ingrijpende
veranderingen in de institutionele structuren, waarbij eerdere
ontwikkelingen juist leidden tot een uitbreiding van het bestuur in
plaats van tot een krimp ervan. Vandaag, voor het eerst in duizend jaar,
ondermijnen en vernietigen megapolitieke omstandigheden in het Westen
overheden en talloze andere instellingen die op grote schaal opereren.

De oorzaken van het instorten van de bestuurlijke schaal aan het einde
van het Romeinse rijk verschilden uiteraard sterk van die aan het begin
van het Informatietijdperk. Een deel van de reden dat Rome viel, lag er
simpelweg in dat het rijk zich had uitgestrekt tot een omvang die de
geweldseconomie niet meer kon ondersteunen. De kosten om de ver
uitgestrekte grenzen van het rijk te garnizoeneren, overtroffen de
economische baten die een traditioneel agrarisch systeem kon leveren. De
druk van belastingen en regelgeving, die nodig was om de militaire
inspanningen te bekostigen, groeide tot een niveau dat de draagkracht
van de economie overschreed. Corruptie raakte endemisch. Zoals
historicus Ramsay MacMullen heeft aangetoond, richtten veel militaire
bevelhebbers zich primair op het vergaren van `illegale winsten uit hun
commando.'\footnote{Ramsay MacMullen, \emph{Corruption and the Decline
  of Rome} (New Haven: Yale University Press, 1988), p.~192.} Ze deden
dit door de bevolking af te persen -- iets wat de vierde-eeuwse
waarnemer Synesius omschreef als `de vredestijd-oorlog, die bijna erger
was dan de barbaarse oorlog en voortkwam uit de ondiscipline van het
leger en de hebzucht van de officieren.'\footnote{Geciteerd in Ibid.,
  p.~193.}

Een belangrijke factor in de ondergang van Rome was het demografische
tekort dat de Antonijnse plagen veroorzaakten. In veel gebieden leidde
de forse afname van de Romeinse bevolking tot zowel economische als
militaire zwakte. Dergelijke ontwikkelingen hebben we in onze tijd nog
niet waargenomen, of in ieder geval niet op vergelijkbare schaal. Als we
de zaken op lange termijn bekijken, bestaat de mogelijkheid dat nieuwe
`plagen' de uitdagingen van technologische achteruitgang in dit
millennium nog verder versterken. De ongekende bevolkingsgroei in de
twintigste eeuw maakt ons tot een aantrekkelijk doelwit voor snel
muterende microparasieten. De bezorgdheid dat het ebolavirus -- of iets
dergelijks -- stedelijke bevolkingen zou binnendringen, neemt daarom
toe. Dit is echter niet de plaats om in te gaan op de coevolutie van
mens en ziekte. Hoe boeiend dat onderwerp ook is, ons betoog gaat hier
niet over de oorzaken van Rome's val, noch over de vraag of onze wereld
vandaag kwetsbaar is voor invloeden die destijds tot Rome's verval
leidden. We richten ons op hoe we grote historische transformaties
waarnemen -- of, beter gezegd, hoe we die vaak verkeerd interpreteren
wanneer ze plaatsvinden.

Mensen blijven overal en altijd, in zekere mate, conservatief -- met een
kleine `c'. Dit betekent dat we terughoudend zijn om oude sociale
conventies los te laten, gevestigde instituties overboord te gooien en
de wetten en waarden waarop we ons kompas baseren ter discussie te
stellen. Weinigen geloven dat ogenschijnlijk geringe veranderingen in
klimaat, technologie of andere factoren zo ingrijpend kunnen zijn dat ze
de band met de wereld van onze voorouders volledig verbreken. Ook de
Romeinen erkenden de om hen heen plaatsvindende veranderingen
nauwelijks; wij doen dat op dezelfde manier.

Of je het nu wilt erkennen of niet, we bevinden ons midden in een
ingrijpende historische transformatie, waarin de manier waarop mensen
voor hun levensonderhoud zorgen en zich verdedigen radicaal verandert.
De impact is zo diepgaand dat vrijwel niets meer vanzelfsprekend is.
Bijna voortdurend gaan we ervan uit dat de opkomende
informatiesamenlevingen veel gemeen hebben met de industriële
samenleving waarin we zijn opgegroeid. Wij betwijfelen dat. De kracht
van microprocessing zal als het ware het cement tussen de stenen doen
oplossen en de logica van geweld zo drastisch transformeren dat de hele
organisatie van ons levensonderhoud en onze verdediging heroverwogen
moet worden. Toch blijven we de onvermijdelijkheid van deze
veranderingen vaak bagatelliseren of debatteren over hun wenselijkheid,
alsof het de taak is van industriële instituties om te bepalen hoe de
geschiedenis zich ontvouwt.

\subsection{De grote illusie}\label{de-grote-illusie}

Auteurs die op veel vlakken juist beter geïnformeerd lijken dan wij,
misleiden je toekomstvisie door te oppervlakkig te werk te gaan in hun
onderzoek naar de werking van samenlevingen. David Kline en Daniel
Burstein hebben bijvoorbeeld een grondig onderbouwd boek geschreven met
de titel \emph{Road Warriors: Dreams and Nightmares Along the
Information Highway}. Hoewel het boek vol bewonderenswaardige details
zit, gebruiken ze veel van die details om een illusie te creëren: het
idee dat burgers samen en bewust de spontane economische en natuurlijke
processen om hen heen kunnen vormgeven.\footnote{Geciteerd in David
  Kline en Daniel Burstein, `Is de overheid overbodig?' \emph{Wired},
  januari 1996, p.~105.} Zo gemakkelijk als dat klinkt, komt dit neer op
zeggen dat het feodalisme had kunnen blijven bestaan als iedereen zich
opnieuw had toegewijd aan de ridderlijkheid. Niemand aan het hof in de
late vijftiende eeuw zou zo'n opvatting hebben verworpen; het zou zelfs
ketterij zijn geweest om dat te doen. Tegelijkertijd is het een
volstrekt misleidende vergelijking, een voorbeeld van de slang die de
toekomst in zijn oude vel wilde persen.

De fundamentele oorzaken van verandering schuilen in die factoren waarop
we geen bewuste invloed hebben. Het gaat om elementen die de condities
aanpassen waardoor geweld loont. Inderdaad, ze bevinden zich zo ver
buiten de mogelijkheden voor manipulatie dat je ze nauwelijks terugziet
als onderwerp in politieke manoeuvres in een door politiek verzadigde
maatschappij. Tijdens demonstraties hoor je nooit mensen marcheren onder
de kreet: `Vergroot de schaalvoordelen in het productieproces.' Evenmin
zie je een banner met de eis: `Bedenk een wapensysteem dat het belang
van de infanterie vergroot.' Geen enkele kandidaat heeft ooit beloofd
`de balans te veranderen tussen efficiëntie en omvang in de bescherming
tegen geweld.' Zulke slogans klinken belachelijk, omdat hun
doelstellingen simpelweg niet binnen het bereik liggen van wat we
doelbewust kunnen beïnvloeden. Toch bepalen deze variabelen, zoals we
straks zullen aantonen, in veel grotere mate hoe de wereld functioneert
dan welke politieke koers dan ook.

Als je er goed over nadenkt, valt meteen op dat belangrijke
omwentelingen in de geschiedenis zelden hun oorsprong vinden in
menselijke verlangens. Veranderingen vinden niet plaats omdat mensen
genoeg krijgen van het ene bestaan en plotseling kiezen voor een ander.
Denk er maar eens over na: als louter onze gedachten en wensen bepalend
waren, verklaarden wilde stemmingswisselingen zomaar alle abrupte
veranderingen in de geschiedenis, losgekoppeld van reële veranderingen
in de levensomstandigheden. In werkelijkheid zie je dit nooit; alleen
bij individuele medische aandoeningen komen stemmingswisselingen voor
die schijnbaar onafhankelijk zijn van een objectieve oorzaak.

Mensen besluiten in groten getale niet plotseling en gezamenlijk hun
levenswijze op te geven, louter omdat ze dat grappig zouden vinden. Je
hoort nooit een jager-verzamelaar verkondigen: `Ik heb genoeg van het
prehistorische bestaan; ik verkies het leven als boerenknecht in een
agrarisch dorp.' Elke ingrijpende verschuiving in gedragspatronen en
waarden vormt een directe reactie op een daadwerkelijke verandering in
de leefomstandigheden. Als mensen plotseling anders gaan denken, wijst
dat er vrijwel zeker op dat zij geconfronteerd worden met een breuk in
de vertrouwde situatie -- bijvoorbeeld door een invasie, een pest, een
onverwachte klimaatsverandering of een technologische revolutie die hun
bestaansmiddelen of verdedigingsvermogen aangepast.

Beslissende historische veranderingen ontstaan niet uit menselijke
verlangens; ze tarten juist de behoefte aan stabiliteit die de meeste
mensen koesteren. Als veranderingen plaatsvinden, brengen ze doorgaans
brede desoriëntatie met zich mee, vooral voor degenen die hun inkomen of
sociale status verliezen. Je zult tevergeefs in opiniepeilingen en
andere stemmingindicatoren proberen te achterhalen hoe de aankomende
megapolitieke transitie zich zal ontvouwen.

\section{Het leven zonder
vooruitziendheid}\label{het-leven-zonder-vooruitziendheid}

Als we de enorme transitie die om ons heen plaatsvindt over het hoofd
zien, komt dat deels omdat we er bewust voor kiezen om er niet naar te
kijken. Onze voorouders, die voedsel verzamelden, waren misschien net zo
koppig, maar zij hadden daar een gegronde reden voor. Tienduizend jaar
geleden kon niemand de gevolgen van de landbouwrevolutie voorzien;
niemand keek verder dan waar hij direct zijn volgende maaltijd kon
vinden. Toen de landbouw op gang kwam, beschikte men niet over
historische verslagen die een toekomstperspectief boden. Er bestond
zelfs niet het Westerse tijdsbesef, ingedeeld in ordelijke eenheden
zoals seconden, minuten, uren en dagen om de jaren te meten.
Jagers-verzamelaars leefden in het `eeuwige nu', zonder kalenders en
helemaal zonder geschreven documenten. Zij hadden geen wetenschap of
ander intellectueel hulpmiddel om oorzaak en gevolg te doorgronden; ze
vertrouwden enkel op hun intuïtie. Als het erop aankwam de toekomst te
voorzien, waren onze oeroude voorouders compleet blind. Met een knipoog
naar de Bijbel: zij hadden nog niet van de `vrucht der kennis' geproefd.

\subsection{Leren van het verleden}\label{leren-van-het-verleden}

Gelukkig beschikken we tegenwoordig over een breder perspectief. In de
afgelopen vijfhonderd generaties hebben we analytische vaardigheden
ontwikkeld die onze voorouders ontbeerden. Dankzij wetenschap en
wiskunde ontrafelen we nu talloze geheimen van de natuur, waardoor ons
begrip van oorzaak en gevolg bijna wonderbaarlijk aandoet vergeleken met
dat van de vroege jagers-verzamelaars. Computeralgoritmes, ontwikkeld
met behulp van hogesnelheidscomputers, bieden nieuwe inzichten in het
functioneren van complexe, dynamische systemen zoals de menselijke
economie. De moeizame ontwikkeling van de politieke economie zelf -- al
is die verre van perfect -- heeft ons inzicht vergroot in de factoren
die het menselijk handelen beïnvloeden. Daarbij is het essentieel te
erkennen dat mensen, altijd en overal, gevoelig reageren op prikkels. Ze
handelen wellicht niet zo mechanisch als economen vaak veronderstellen,
maar kosten en beloningen spelen wel degelijk een rol. Als externe
omstandigheden zodanig veranderen dat de beloningen voor bepaald gedrag
stijgen of de kosten dalen, neemt dat gedrag -- onder gelijke overige
omstandigheden -- toe.

\subsection{Prikkels doen ertoe}\label{prikkels-doen-ertoe}

Dat mensen reageren op kosten en beloningen vormt de kern van veel
voorspellingen. Je weet vrijwel zeker dat als je een biljet van honderd
dollar op straat laat liggen, iemand het al snel oppakt -- of je nu in
New York, Mexico-Stad of Moskou bent. Dit is minder vanzelfsprekend dan
het lijkt en toont aan waarom degenen die beweren dat voorspellen
onmogelijk is, ongelijk hebben. Hoe groter de verwachte verandering in
kosten en beloningen, des te zwaarder de impliciete voorspelling zal
wegen.

De meest verregaande voorspellingen ontstaan waarschijnlijk doordat we
de implicaties van veranderende megapolitieke variabelen doorgronden.
Geweld bepaalt immers als geen ander de grenzen van menselijk gedrag;
als je begrijpt hoe de logica van geweld verandert, kun je redelijk
inschatten waar mensen in de toekomst het equivalent van
honderd-dollarbiljetten zullen laten vallen of oppikken.

Hiermee bedoelen we niet dat je het onkenbare volledig kunt begrijpen.
We kunnen je immers niet uitleggen hoe je de winnende loterijnummers of
een werkelijk willekeurig evenement moet voorspellen. We hebben geen
methode om te weten wanneer een terrorist in Manhattan een atoombom zal
laten ontploffen of wanneer een asteroïde Saudi‐Arabië zal treffen.
Evenmin kunnen wij voorspellen wanneer een nieuwe ijstijd begint, een
plotselinge vulkaanuitbarsting plaatsvindt of een nieuwe ziekte opduikt.
Het aantal onkenbare gebeurtenissen dat de loop van de geschiedenis kan
veranderen, is enorm. Toch verschilt het doorgronden van het onkenbare
wezenlijk van het afleiden van de implicaties van hetgeen we al kennen.
Zie je op de afstand een flits van bliksem, dan weet je vrijwel zeker
dat er spoedig een donderklap volgt. Het voorspellen van de gevolgen van
megapolitieke transities bestrijkt weliswaar veel langere tijdsbestekken
en kent minder zekere verbanden, maar blijft in essentie een
vergelijkbare oefening.

Megapolitieke katalysatoren voor verandering komen doorgaans ruim voor
de daadwerkelijke manifestatie van hun effecten. Het duurde ongeveer
vijfduizend jaar voordat alle implicaties van de Agrarische Revolutie
duidelijk werden. De transitie van een agrarische naar een industriële
samenleving, gedreven door productie en chemische energie, vond binnen
enkele eeuwen plaats. De overgang naar de Informatie‐Samenleving zal
naar verwachting nog sneller verlopen, waarschijnlijk binnen één
mensenleven. Zelfs als we de verkorting van de geschiedkundige tijd in
ogenschouw nemen, mag je ervan uitgaan dat het nog decennialang duurt
voordat de volledige megapolitieke impact van de huidige
informatietechnologie zichtbaar wordt.

\subsection{Grote en kleine megapolitieke
transities}\label{grote-en-kleine-megapolitieke-transities}

In dit hoofdstuk bespreek ik enkele gemeenschappelijke kenmerken van
megapolitieke transities. In de volgende hoofdstukken duiken we dieper
in op de Agrarische Revolutie en de overgang van boerderij naar fabriek
-- de tweede van de eerdere grote faseveranderingen. Binnen de
agrarische periode van de beschaving vonden talloze kleinere
megapolitieke transities plaats, zoals de val van Rome en de feodale
revolutie rond het jaar 1000. Deze gebeurtenissen illustreren de opkomst
en ondergang van machtsverhoudingen, naarmate overheden kwamen en gingen
en de vruchten van de landbouw van de ene naar de andere hand schoven.
Eigenaren van uitgestrekte landgoederen in het Romeinse Rijk,
zelfstandige boeren in de donkere middeleeuwen en de heren en
lijfeigenen uit de feodale periode haalden allemaal hun graan van
dezelfde akkers. Zij leefden onder zeer uiteenlopende regeringsvormen,
als gevolg van de cumulatieve invloed van diverse technologieën,
klimaatfluctuaties en ontwrichtende invloeden van ziekte.

We gaan niet in op elk detail van deze veranderingen, al hebben we wel
enkele voorbeelden gegeven van hoe veranderende megapolitieke variabelen
de manier waarop in het verleden macht werd uitgeoefend, beïnvloedden.
Overheden groeiden en krompen naarmate de megapolitieke schommelingen de
kosten voor het uitoefenen van macht verlaagden of verhoogden.

Hieronder vindt u enkele kernpunten die u in gedachten moet houden bij
het begrijpen van de informatie-revolutie:

\begin{enumerate}
\def\labelenumi{\arabic{enumi}.}
\item
  Een verschuiving in de megapolitieke fundamenten van macht vindt
  doorgaans plaats lang voordat de daadwerkelijke revoluties in de
  machtsuitoefening losbarsten.
\item
  Inkomens dalen meestal zodra een ingrijpende transitie begint, vaak
  omdat een samenleving door bevolkingsdruk haar middelen marginaliseert
  en zichzelf daarmee crisisgevoelig maakt.
\item
  Buiten het systeem kijken valt doorgaans onder het taboe. Mensen zien
  vaak niet door de bestaande logica van geweld heen en merken daardoor
  nauwelijks op hoe die logica -- zowel subtiel als openlijk --
  verandert. Vaak herkennen mensen megapolitieke transities niet voordat
  ze zich voltrekken.
\item
  Grote transities gaan altijd vergezeld van een culturele revolutie,
  wat doorgaans leidt tot botsingen tussen aanhangers van oude en nieuwe
  waarden.
\item
  Megapolitieke transities zijn nooit geliefd. Ze verouderen het
  zorgvuldig opgebouwde intellectuele kapitaal en ondermijnen de
  gevestigde morele imperatieven. Ze ontstaan niet uit volksverlangen,
  maar vormen een reactie op externe veranderingen die de logica van
  geweld op lokaal niveau aanpassen.
\item
  Overgangen naar nieuwe manieren om het levensonderhoud te organiseren
  of naar andere vormen van bestuur beperken zich in eerste instantie
  tot de gebieden waar de megapolitieke katalysatoren actief zijn.
\item
  Behalve in de vroege stadia van de landbouw brachten eerdere
  transities altijd periodes van sociale chaos en een toename van geweld
  met zich mee, als gevolg van desoriëntatie en het instorten van het
  oude systeem.
\item
  Corruptie, moreel verval en inefficiëntie kenmerken doorgaans de
  laatste fasen van een systeem.
\item
  Door de toenemende invloed van technologie op de logica van geweld
  versnelt de geschiedenis; elke volgende transitie krijgt steeds minder
  tijd om zich aan te passen dan voorheen.
\end{enumerate}

De geschiedenis versnelt

Nu gebeurtenissen vele malen sneller plaatsvinden dan tijdens eerdere
transformaties, is het van groot belang om op tijd door te dringen tot
de manier waarop de wereld verandert. Dit inzicht kan u veel meer
opleveren dan wat uw voorouders ooit hadden kunnen realiseren in een
vergelijkbare fase. Zelfs als de eerste boeren op wonderbaarlijke wijze
alle megapolitieke gevolgen van landbouw hadden doorgrond, had die
kennis weinig nut, omdat het duizenden jaren zou duren voordat de
overgang naar een nieuwe samenlevingsfase compleet was.

Vandaag is het anders. De geschiedenis raast in een stroomversnelling en
nauwkeurige voorspellingen van de megapolitieke impact van nieuwe
technologieën blijken tegenwoordig van enorme waarde. Als we de gevolgen
van de huidige transitie naar de informatiesamenleving net zo grondig
kunnen doorgronden als iemand in het verleden de overstap naar landbouw
en industrie wist te vatten, dan wordt die kennis nu vele malen
kostbaarder. Kortom, de periode waarin megapolitieke voorspellingen van
nut zijn, heeft zich teruggebracht tot het tijdsbestek van één
mensenleven.

\begin{quote}
`Als we terugkijken over de eeuwen, of zelfs als we alleen naar het
heden kijken, kunnen we duidelijk vaststellen dat veel mannen hun brood
hebben verdiend -- vaak zeer goed -- door hun bijzondere vaardigheid in
het hanteren van geweldwapens, en dat hun activiteiten een grote rol
hebben gespeeld in het bepalen van de wijze waarop schaarse middelen
werden benut.'\footnote{Lane, `Economische gevolgen van georganiseerd
  geweld', op.~cit.} FREDERIC C. LANE
\end{quote}

Onze studie naar megapolitiek tracht inzicht te krijgen in de
veranderende factoren die bepalen binnen welke grenzen geweld wordt
toegepast.

Deze megapolitieke elementen bepalen in hoge mate wanneer en waar het
inzetten van geweld winstgevend is. Ze beïnvloeden tevens de verdeling
van inkomens op de markt. Zoals de economische historicus Frederic Lane
al duidelijk verwoordde, speelt de manier waarop geweld georganiseerd en
gecontroleerd wordt een cruciale rol bij de benutting van schaarse
middelen.\footnote{Ibid.}

\section{Een snelcursus megapolitiek}\label{een-snelcursus-megapolitiek}

Het begrip megapolitiek is krachtig omdat het helpt enkele van de grote
mysteries uit de geschiedenis te ontrafelen: hoe regeringen ontstaan en
instorten en welke instituties eruit voortvloeien; de timing en afloop
van oorlogen; én de patronen van economische voorspoed en neergang.
Megapolitiek beïnvloedt de kosten en opbrengsten van machtsshow en
bepaalt daarmee in welke mate mensen hun wil kunnen opleggen aan
anderen. Dit fenomeen speelt al sinds de vroegste samenlevingen een rol
en doet dat tot op de dag van vandaag. In Blood in the Streets en
\emph{The Great Reckoning} hebben we vele belangrijke, maar vaak
onopgemerkte megapolitieke factoren blootgelegd die de loop van de
geschiedenis sturen. De sleutel tot het doorgronden van de implicaties
van deze veranderingen ligt bij het begrijpen van de factoren die
revoluties in geweldgebruik in gang zetten. Deze variabelen kun je
globaal indelen in vier categorieën: topografie, klimaat, microben en
technologie.

\begin{enumerate}
\def\labelenumi{\arabic{enumi}.}
\tightlist
\item
  Topografie speelt een cruciale rol, want de controle over geweld op
  zee ligt nooit zo geconcentreerd als op het land. Geen enkele regering
  heeft daar ooit de volledige controle gehad over haar wetten. Dit
  inzicht is essentieel om te begrijpen hoe de organisatie van geweld en
  bescherming zich zal ontwikkelen wanneer de economie richting
  cyberspace verschuift.
\end{enumerate}

Topografie, samen met het klimaat, speelde een cruciale rol in de vroege
geschiedenis. De eerste staten ontstonden op vruchtbare
overstromingsvlakten omgeven door woestijn, zoals in Mesopotamië en
Egypte. Daar stroomde overvloedig irrigatiewater, terwijl de omliggende
gebieden te droog bleven voor een zelfstandige landbouw. In zulke
omstandigheden moesten boeren samenwerken om de hoge kosten van
irrigatie te dragen; lukte dat niet, dan zouden de gewassen niet
groeien, wat onvermijdelijk tot hongersnood zou leiden. Degene die het
water beheersten, stonden hierdoor in een sterke positie en vestigden
vaak rijke, despotische regeringen.

Zoals we in \emph{The Great Reckoning} hebben aangetoond, bepaalden de
topografische omstandigheden ook in het oude Griekenland de welvaart van
de zelfstandige boeren. Hierdoor groeide de regio uit tot de wieg van de
westerse democratie. Drie duizend jaar geleden waren de
vervoersmogelijkheden in het mediterrane gebied zo beperkt dat mensen
die slechts enkele mijl van de zee verwijderd woonden een aanzienlijk
concurrentievoordeel hadden bij de productie van hooggewaardeerde
gewassen als olijven en druiven. Als olie en wijn over land moesten
worden vervoerd, waren de kosten te hoog om er winst mee te maken. De
lange kustlijn van Griekenland zorgde er immers voor dat bijna elk deel
van het land op minder dan twintig mijl van de zee lag, wat de Griekse
boeren een beslissend voordeel gaf ten opzichte van rivalen in de
binnenlanden.

Dankzij dit voordeel in de handel met hoogwaardige producten kenden de
Griekse boeren hoogwaardige inkomsten, ondanks dat zij vaak over slechts
kleine percelen land beschikten. Deze welvaart stelde hen in staat om
dure bepantsering aan te schaffen. De beroemde hoplieten, doorgaans
boeren of landeigenaren, bewapenden zich op eigen kosten. Sterk bewapend
en gedreven vormden zij een indrukwekkende militaire kracht die niemand
over het hoofd kon zien. Zo legden de topografische omstandigheden de
basis voor de Griekse democratie, terwijl vergelijkbare factoren in
Egypte en elders leidden tot despotische heerschappij.

\begin{enumerate}
\def\labelenumi{\arabic{enumi}.}
\setcounter{enumi}{1}
\tightlist
\item
  Het klimaat bepaalt ook de grenzen waarbinnen brute kracht effectief
  kan worden ingezet. Klimaatverandering fungeerde als katalysator voor
  de eerste grote omschakeling van jagen en verzamelen naar landbouw.
\end{enumerate}

Het einde van de laatste ijstijd, zo'n dertienduizend jaar geleden,
bracht ingrijpende veranderingen in de vegetatie teweeg. In het Nabije
Oosten, waar de kou als eerste plaatsmaakte voor warmere omstandigheden,
zorgden een gestage stijging van temperatuur en neerslag ervoor dat
bossen zich uitbreidden naar gebieden die voorheen uit graslanden
bestonden. De snelle verspreiding van beukenbossen had echter een
ernstige impact op het menselijke dieet. Zoals Susan Alling Gregg het
verwoordde in \emph{Foragers and Farmers}:

\begin{quote}
De vestiging van beukenbossen heeft ongetwijfeld ingrijpende gevolgen
gehad voor de lokale populaties van mensen, planten en dieren. Terwijl
het open bladerdak van een eikenbos veel zonlicht op de bosvloer
toelaat, ontstaat er een weelderige ondergroei van diverse struiken,
kruiden en grassen die talrijke wilde dieren ondersteunt. Daarentegen
sluit het bladerdak van een beukenbos het zonlicht af, waardoor de grond
vrijwel permanent in de schaduw blijft. Alleen in de vroege lente, nog
vóórdat de bladeren verschijnen, zie je een opleving van eenjarigen; in
die periode komen uitsluitend schaduwminnende zegels, varens en enkele
grassoorten voor.\footnote{Susan Ailing Gregg, \emph{Foragers and
  Farmers: Population Interaction and Agricultural Expansion in
  Prehistoric Europe} (Chicago: University of Chicago Press, 1988),
  p.~9.}
\end{quote}

\begin{quote}
Na verloop van tijd drongen dichte bossen de open vlaktes binnen en
verspreidden zich over heel Europa, tot in de oostelijke
steppe.\footnote{Stephen Boyden, \emph{Western Civilization in
  Biological Perspective} (Oxford: Clarendon Press, 1987), p.~89. Zie
  ook Marvin Harris, \emph{Cannibals and Kings} (New York: Vintage,
  1978), pp.~29--32.} Deze bossen verkleinden het beschikbare
weidegebied voor grote dieren, waardoor het voor jagers-verzamelaars
steeds moeilijker werd in hun levensonderhoud te voorzien.
\end{quote}

\begin{quote}
Tijdens de welvarende periode van de ijstijd groeide de populatie
jagers-verzamelaars zo sterk dat zij niet langer in hun eigen
levensonderhoud konden voorzien op basis van de steeds krimpende kuddes
grote zoogdieren, waarvan vele soorten door jacht vrijwel naar de rand
van uitsterven werden gedreven. De overstap naar landbouw was dan ook
geen kwestie van voorkeur, maar een gedwongen maatregel om tekorten in
het dieet op te vangen. In de noordelijkere gebieden, waar de opwarming
de leefgebieden van grote zoogdieren niet negatief beïnvloedde, en in
tropische regenwouden, waar de wereldwijde opwarming niet leidde tot
verminderde voedselvoorraden, bleef jagen en verzamelen de overhand
houden. Sinds de opkomst van de landbouw veroorzaken afkoelingen vaak
ingrijpendere veranderingen dan opwarmingen.
\end{quote}

\begin{quote}
Een basaal inzicht in de dynamiek van klimaatverandering in vroegere
samenlevingen kan bijzonder waardevol blijken als het klimaat blijft
schommelen. Als je weet dat een daling van gemiddeld één graad Celsius
de groeiperiode met drie tot vier weken inkort en de maximale hoogte
waarop gewassen verbouwd kunnen worden met ongeveer 500 voet verlaagt,
begrijp je de randvoorwaarden die het toekomstige handelen van mensen
beperken.\footnote{Geoffrey Parker en Lesley Ni.~Smith, red., \emph{The
  General Crisis of the Seventeenth Century} (London: Routledge \& Kegan
  Paul, 1985), p.~8.} Met deze kennis kun je veranderingen voorspellen,
variërend van graanprijzen tot grondwaarden, en zelfs gegronde
conclusies trekken over de waarschijnlijke impact van dalende
temperaturen op reële inkomens en politieke stabiliteit. In het verleden
vielen regeringen ten prooi aan opstanden toen jarenlange mislukte
oogsten de voedselprijzen opdreven en het besteedbare inkomen sterk
daalde.
\end{quote}

\begin{quote}
Het is dan ook geen toeval dat de zeventiende eeuw -- de koudste periode
in de moderne geschiedenis -- samenviel met een tijdperk van wereldwijde
revoluties. Een onderliggende megapolitieke oorzaak van dit ongenoegen
was het aanzienlijk koudere weer. Het was zelfs zo streng dat de wijn op
de \emph{Sun King's}-tafel in Versailles bevroor. Kortere groeiseizoenen
leidden tot mislukte oogsten en verzwakten het reële inkomen. Daardoor
nam de welvaart fors af in een langdurige, wereldwijde depressie die
rond 1620 inzette, met catastrofaal destabiliserende gevolgen. De
economische crisis van de zeventiende eeuw leidde tot een golf van
opstanden, waarvan vele in 1648 samensmolten -- precies tweehonderd jaar
voor een omstreden en beruchte opstandencyclus. Tussen 1640 en 1650
braken in Ierland, Schotland, Engeland, Portugal, Catalonië, Frankrijk,
Moskou, Napels, Sicilië, Brazilië, Bohemen, Oekraïne, Oostenrijk, Polen,
Zweden, Nederland en Turkije opstanden uit. Zelfs China en Japan werden
door de onrust getroffen.
\end{quote}

Het is wellicht geen toeval dat het mercantilisme in de zeventiende eeuw
dominant was, te midden van een periode van afnemende handel.
Economische isolatie kwam het meest duidelijk naar voren aan het einde
van die eeuw, ``toen een verschrikkelijke hongersnood
uitbrak.''\footnote{Zie Charles Woolsey Cole, \emph{French Mercantilism:
  1683-1700} (New York: \emph{Octagon Books}, 1971), p.~6.} In de
achttiende eeuw, vooral na 1750, zorgden mildere temperaturen en hogere
oogstopbrengsten voor een aanzienlijke stijging van het reële inkomen in
West-Europa, waardoor de vraag naar geproduceerde goederen toenam. Vrije
marktmogelijkheden werden in toenemende mate omarmd. Dit leidde tot een
zelfversterkende groeispurt, waarbij de industrie op grote schaal
uitbreidde in wat wij nu de Industriële Revolutie noemen. Het toenemende
belang van technologie en industriële producten deed de weersinvloeden
op de economische cycli afnemen.

Zelfs tegenwoordig mag je de invloed van plotselinge kou op reële
inkomens niet onderschatten -- ook in welvarende regio's zoals
Noord-Amerika. Samenlevingen maken zichzelf vaak kwetsbaar voor crises
zodra hun bestaande instellingen hun maximaal haalbare potentieel lijken
te hebben uitgeput. In het verleden uitte die tendens zich vaak in een
bevolkingsgroei die de draagkracht van het land tot het uiterste testte.
Dit gebeurde zowel vóór het jaar 1000 als aan het einde van de
vijftiende eeuw. De scherpe daling van reële inkomens, als gevolg van
mislukte oogsten en lagere opbrengsten, speelde in beide gevallen een
cruciale rol bij het omverwerpen van de heersende instituties.
Tegenwoordig manifesteert die marginalisatie zich in de
consumentenkredietmarkten. Als plotselinge kou de oogstopbrengsten doet
dalen en het besteedbare inkomen verlaagt, kan dat leiden tot
wanbetalingen en opstanden over belastingen. Als het verleden een
leidraad is, kunnen zowel economische isolatie als politieke
instabiliteit de nasleep zijn.

\begin{enumerate}
\def\labelenumi{\arabic{enumi}.}
\setcounter{enumi}{2}
\tightlist
\item
  Microben beschikken over het vermogen om zowel schade aan te richten
  als bescherming te bieden op manieren die vaak bepalend zijn voor het
  uitoefenen van macht. Dat bleek zeker tijdens de Europese verovering
  van de Nieuwe Wereld, zoals we onderzochten in \emph{The Great
  Reckoning}. Europese kolonisten, afkomstig uit gevestigde
  landbouwsamenlevingen die door ziekten werden geteisterd, beschikten
  over een relatieve immuniteit tegen kinderziekten zoals mazelen. De
  Indianen die zij tegenkwamen, leefden grotendeels in dunbevolkte,
  voedselzoekende gemeenschappen. Zij waren niet immuun en raakten in
  grote aantallen gedecimeerd. Vaak nam de sterfte haar hoogtepunt nog
  voordat blanke kolonisten arriveerden, omdat de Indianen de infecties
  landinwaarts verspreidden.
\end{enumerate}

Er bestaan bovendien microbiologische barrières die de uitoefening van
macht beperken. In \emph{Blood in the Streets} bespraken we hoe
krachtige stammen van malaria tropisch Afrika jarenlang ondoordringbaar
maakten voor blanke invasies. Voordat kinine halverwege de negentiende
eeuw werd ontdekt, konden blanke legers in malariagebieden niet
overleven, hoe superieur hun bewapening ook was.

De interactie tussen mensen en microben heeft ook belangrijke
demografische gevolgen gehad, waardoor de kosten en baten van geweld
veranderden. Wanneer de sterfte sterk fluctueert door epidemieën,
hongersnoden of andere oorzaken, daalt het relatieve sterfterisico in
oorlogvoering. De afnemende pieken in sterfte sinds de zestiende eeuw
verklaren zowel de verschuiving naar kleinere gezinnen als de veel
geringere tolerantie voor plotselinge dood tijdens oorlogen vergeleken
met vroeger. Dit heeft ertoe geleid dat de acceptatie van imperialisme
afneemt en dat in samenlevingen met lage geboortecijfers de kosten voor
het projecteren van macht stijgen.

Moderne samenlevingen, waarin gezinnen klein zijn, vinden zelfs geringe
aantallen slagveldsterfgevallen onaanvaardbaar. Daarentegen toonden
vroegmoderne samenlevingen veel meer tolerantie voor de levensverliezen
die met imperialisme gepaard gingen. Vóór deze eeuw kregen de meeste
ouders vele kinderen, waarbij men ervan uitging dat enkelen willekeurig
en plotseling aan ziekte zouden overlijden. In een tijdperk waarin
vroege dood alledaags was, gingen toekomstige soldaten en hun families
met minder terughoudendheid de gevaren van het slagveld tegemoet.

\begin{quote}
`Machinerie is agressief. De wever wordt een web, de machinist een
machine. Als je geen gereedschap gebruikt, gebruiken zij jou.' - Emerson
\end{quote}

\begin{enumerate}
\def\labelenumi{\arabic{enumi}.}
\setcounter{enumi}{3}
\tightlist
\item
  Technologie heeft tot nu toe de doorslaggevende rol gespeeld in het
  bepalen van de kosten en baten van machtprojectie in de moderne
  eeuwen. Het uitgangspunt van dit boek is dat dit zo blijft.
  Technologie kent immers diverse cruciale dimensies:
\end{enumerate}

A. \emph{Balans tussen aanval en verdediging}. De verhouding tussen
aanval en verdediging, zoals bepaald door de geldende wapentechnologie,
draagt bij aan het vaststellen van de schaal van politieke organisaties.
Wanneer de offensieve mogelijkheden toenemen en macht op afstand kan
worden uitgeoefend, bundelen rechtsgebieden zich en ontstaan grotere
overheidsstructuren. Op andere momenten, zoals in de huidige situatie,
versterken de defensieve mogelijkheden zich, wat de kosten verhoogt om
macht buiten de kerngebieden uit te oefenen. Rechtsgebieden neigen
daardoor naar versnippering en grote overheden vallen uiteen in kleinere
eenheden.

B. \emph{Gelijkheid en de overheersing van de infanterie}. Een
fundamenteel kenmerk dat de mate van gelijkheid tussen burgers
beïnvloedt, is de aard van de wapentechnologie. Wapens die goedkoop
zijn, door niet-professionals ingezet kunnen worden en de effectiviteit
van de infanterie vergroten, egaliseren de machtsbalans. Toen Thomas
Jefferson verklaarde dat `alle mensen gelijk geschapen zijn', bedoelde
hij een stelling die destijds meer waarheid in zich droeg dan
vergelijkbare beweringen uit voorgaande eeuwen. Een boer met zijn
jachtgeweer was niet alleen even goed bewapend als de typische Britse
soldaat met zijn \emph{Brown Bess}, maar overtrof hem er ook in doordat
hij op grotere afstand kon richten en nauwkeuriger terugvuurde. Dit
stond in scherp contrast met de middeleeuwse situatie, waarin een boer,
die zich niets anders kon veroorloven dan een hooivork, nauwelijks
opbokste tegen een zwaarbewapende ridder te paard. Niemand schreef in
1276 dat `alle mensen gelijk geschapen zijn'. In die tijd had één ridder
veruit meer brute kracht dan tientallen boeren tezamen.

C. \emph{Voors en tegens van schaal in geweld}.\\
Een bijkomende factor die bepaalt of er een paar grote staten bestaan of
juist talloze kleine, is de mate van organisatie die nodig is om de
beschikbare wapens in te zetten. Wanneer geweld meer oplevert, loont het
om op grote schaal te besturen; overheden groeien dan vanzelfsprekend in
omvang. Als een kleine groep over doeltreffende middelen beschikt om een
aanval van een grotere groep af te weren -- zoals vaak in de
middeleeuwen gebeurde -- leidt dat tot een opsplitsing van de
soevereiniteit. Kleine, onafhankelijke autoriteiten nemen dan veel taken
van de overheid op zich. Zoals we in een later hoofdstuk toelichten,
zijn wij ervan overtuigd dat het informatietijdperk de opkomst van
cybersoldaten zal inluiden, wat het begin betekent van meer
decentralisatie. Cybersoldaten zetten niet uitsluitend natiestaten in,
maar ook kleine organisaties en zelfs individuen. In het volgende
millennium zullen conflicten waarschijnlijk enkele bijna bloedloze
veldslagen kennen, uitgevochten met behulp van computers.

D. \emph{Schaalvoordelen in de productie}.\\
Een andere belangrijke factor die bepaalt of de uiteindelijke macht
lokaal of op afstand wordt uitgeoefend, is de grootte van de dominante
ondernemingen waar mensen hun brood mee verdienen. Als cruciale
ondernemingen alleen optimaal functioneren wanneer ze in een groot,
overkoepelend handelsgebied georganiseerd zijn, kunnen overheden die
zich uitbreiden om zo'n omgeving voor ondernemingen te creëren, extra
rijkdom vergaren om de kosten van een omvangrijk politiek systeem te
dekken. Onder zulke omstandigheden werkt de wereldeconomie doorgaans
efficiënter wanneer één wereldmacht alle anderen domineert, zoals het
Britse Rijk in de negentiende eeuw liet zien. Maar overkoepelende,
megapolitieke factoren zorgen er soms voor dat deze schaalvoordelen
afnemen. Zodra de economische baten van een groot handelsgebied
wegvallen, kunnen grotere overheden -- die voorheen profiteerden van de
voordelen van een uitgebreid gebied -- uiteen gaan vallen, zelfs als de
balans tussen aanvallende en verdedigende wapentechnologie min of meer
onveranderd blijft.

E. \emph{Verspreiding van technologie}.\\
Een andere factor die de machtsverdeling beïnvloedt, is hoe
wijdverspreid sleuteltechnologieën zijn. Als wapens of productiemiddelen
effectief kunnen worden beheerst of gemonopoliseerd, centraliseert dat
vaak de macht. Zelfs technologieën die in principe defensief zijn, zoals
het machinegeweer, bleken tegelijkertijd krachtige offensieve wapens te
zijn en droegen bij aan een toenemende bestuurlijke schaal in een
periode waarin ze nog niet algemeen verspreid waren.

Aan het eind van de negentiende eeuw hadden de Europese mogendheden een
monopolie op het machinegeweer en zetten ze deze wapens in tegen
volkeren aan de periferie om hun koloniale rijken ingrijpend uit te
breiden. In de twintigste eeuw, vooral in de nasleep van de Tweede
Wereldoorlog, kwamen machinegeweren op grote schaal beschikbaar en
werden ze gebruikt om de macht van bestaande rijken geheel teniet te
doen. Als alle andere factoren gelijk blijven, geldt dat hoe verder
sleuteltechnologieën zich verspreiden, des te meer de macht gedeeld zal
worden en de optimale schaal van de overheid kleiner uitvalt.

\section{De snelheid van megapolitieke
verandering}\label{de-snelheid-van-megapolitieke-verandering}

Hoewel technologie tegenwoordig veruit de belangrijkste factor is -- en
kennelijk steeds meer de overhand krijgt -- hebben alle vier de grote
megapolitieke factoren vroeger bijgedragen aan de bepaling van de schaal
waarop macht werd uitgeoefend.

Deze factoren bepalen samen of de baten van geweld blijven toenemen als
men het op grotere schaal inzet. Daarmee komt naar voren hoe belangrijk
het is om over een ruime hoeveelheid vuurkracht te beschikken, tegenover
de efficiëntie waarmee middelen worden ingezet. Ze beïnvloeden tevens
sterk de verdeling van inkomen. De vraag blijft: welke rol spelen zij in
de toekomst? Een belangrijke aanwijzing om dat in te schatten is te
erkennen dat deze megapolitieke variabelen op dramatisch uiteenlopende
tempo's veranderen.

In de hele geschreven geschiedenis is de topografie vrijwel onveranderd
gebleven. Afgezien van kleine lokale effecten -- zoals havens die
dichtslibben, land dat wordt opgevuld of oppervlakte-erosie -- lijkt de
indeling van de aarde vandaag de dag vrijwel hetzelfde als in de tijd
dat Adam en Eva uit Eden strompelden. Zo blijft het waarschijnlijk
totdat een nieuwe ijstijd de continenten opnieuw vormgeeft of een andere
ingrijpende gebeurtenis het aardoppervlak verstoort. Op een nog dieper
niveau verschuiven de geologische tijdperken -- vermoedelijk als reactie
op grote meteorietinslagen -- over perioden van 10 tot 40 miljoen jaar.
Op een dag vinden we wellicht opnieuw geologische omwentelingen die de
topografie van onze planeet ingrijpend veranderen. Als dat gebeurt, kun
je er met zekerheid van uitgaan dat zowel het honkbal- als het
cricketseizoen niet door zal gaan.

Het klimaat verandert veel dynamischer dan de topografie. In het
afgelopen miljoen jaar heeft klimaatverandering grotendeels bijgedragen
aan de variaties die we op het aardoppervlak waarnemen. Tijdens
ijstijden hebben gletsjers nieuwe valleien uitgesleten, de loop van
rivieren heringericht en eilanden van continenten afgesneden of juist
met elkaar verbonden door de zeespiegel te laten dalen.

Klimaatschommelingen speelden een cruciale rol in de geschiedenis: zij
brachten na het einde van de laatste ijstijd de agrarische revolutie op
gang en destabiliseerden later regimes tijdens perioden van lagere
temperaturen en droogte.

Recentelijk werd er bezorgdheid geuit over de mogelijke gevolgen van
`opwarming van de aarde'. We mogen deze alarmbellen niet zomaar negeren.
Op de lange termijn lijkt het echter aannemelijker dat we een
verschuiving naar een kouder klimaat ervaren in plaats van een verdere
opwarming. Onderzoek naar temperatuurschommelingen -- gebaseerd op de
analyse van zuurstofisotopen in kernmonsters van de oceaanbodem -- wijst
erop dat de huidige periode de op één na warmste is geweest in meer dan
twee miljoen jaar.\footnote{Chris Scarre, red., \emph{Past Worlds: The
  Times Atlas of Archaeology 314} (New York: \emph{Random House}, 1995),
  p.~58.} Als de temperaturen weer zouden dalen, zoals in de zeventiende
eeuw gebeurde, zou dat kunnen leiden tot grootschalige politieke onrust.
In dat opzicht bieden de huidige waarschuwingen over opwarming van de
aarde, mits juist, de geruststelling dat de temperaturen binnen het
abnormaal warme en relatief milde bereik blijven waarin we de afgelopen
drie eeuwen leefden.

Hoe snel microben hun invloed op machtsstructuren aanpassen, blijft een
raadsel. Micro-organismen, en in het bijzonder virussen, passen zich
namelijk razendsnel aan. Zo muteert het verkoudheidsvirus op een bijna
caleidoscopische manier. Ondanks dat deze mutaties zich in hoog tempo
voordoen, verschuiven ze de machtsgrenzen minder abrupt dan
technologische ontwikkelingen. Dit komt deels doordat de natuurlijke
balans er doorgaans voor zorgt dat microben er beter aan doen hun
gastheren te besmetten zonder ze volledig uit te roeien. Infecties die
hun gastheren te snel doden, roken immers hun eigen voortbestaan;
microparasieten overleven alleen als zij voorkomen dat zij hun gastheren
in één klap fataal worden.

Dit betekent echter niet dat uitbraken van dodelijke ziekten, die de
machtsverhoudingen op zijn kop kunnen zetten, uitgesloten zijn.
Dergelijke gebeurtenissen hebben in de geschiedenis een prominente rol
gespeeld. De Zwarte Dood rukte enorme delen van de Euraziatische
bevolking weg en bezorgde de internationale economie van de veertiende
eeuw een zware klap.

\subsection{Wat had kunnen zijn}\label{wat-had-kunnen-zijn}

Geschiedenis begrijpen we zowel in termen van wat had kunnen gebeuren
als in termen van wat werkelijk is voorgevallen. Er is geen reden om te
veronderstellen dat microparasieten in de moderne tijd niet enorme
verwoesting in de menselijke samenleving hadden kunnen ontketenen. Zo
had het bijvoorbeeld kunnen dat microbiologische barrières voor
machtsuitoefening -- vergelijkbaar met malaria, maar virulenter -- de
westerse expansie in perifere gebieden hadden kunnen stoppen. De eerste
onbevreesde Portugese avonturiers die door de Afrikaanse wateren voeren,
hadden een dodelijk retrovirus kunnen oplopen -- een besmettelijker
variant van aids -- waardoor de opening van een nieuwe handelsroute naar
Azië nog vóór aanvang had kunnen worden verijdeld. Ook Columbus en de
eerste groep kolonisten in de Nieuwe Wereld hadden ziekten kunnen
tegenkomen die hen massaal hadden kunnen decimeren, net zoals inheemse
bevolkingen getroffen werden door mazelen en andere westerse
kinderziekten. Toch is niets dergelijks gebeurd, wat onze intuïtie
bevestigt dat geschiedenis een eigen verloop kent.

Microben beperkten het consolideren van macht in de moderne periode niet
zozeer, ze ondersteunden het juist. Westerse soldaten en kolonisten aan
de periferie ontdekten vaak dat hun technologische voorsprong werd
versterkt door microbiologische voordelen. Zij beschikten over
onzichtbare biologische wapens, zoals een relatief sterke immuniteit
voor kinderziekten die inheemse bevolkingsgroepen vaak verwoestten. Dit
verschaft de westerse reizigers een duidelijk voordeel waar hun
tegenstanders uit dunbevolkte regio's niet over beschikten. Gaandeweg
verliep de overdracht van ziekten vrijwel uitsluitend van Europa naar de
periferie; de omgekeerde richting bleef uit.

Sommigen hebben als mogelijk tegenvoorbeeld aangevoerd dat westerse
ontdekkingsreizigers syfilis uit de Nieuwe Wereld naar Europa brachten.
Hoewel dat bespreekbaar is, vormde syfilis -- indien de aantijgingen
klopten -- geen wezenlijke belemmering voor de machtsuitoefening.
Syfilis had vooral invloed op een wijziging in de seksuele moraal in het
Westen. Van het einde van de vijftiende eeuw tot het laatste kwart van
de twintigste eeuw begon de impact van microben op de industriële
samenleving geleidelijk minder nadelig te blijken. Ondanks de
persoonlijke tragedies en het leed door uitbraken van tuberculose, polio
en griep, zijn er in de moderne tijd geen ziekten opgekomen die in de
buurt kwamen van de megapolitieke gevolgen van de Antonijnse plagen of
de Zwarte Dood. De verbetering van de volksgezondheid en de introductie
van vaccinaties en antidota hebben het belang van besmettelijke microben
teruggedrongen, waardoor technologische middelen een relatief
belangrijkere rol kregen bij het bepalen van de grenzen van
machtsuitoefening.

De recente opkomst van \emph{AIDS} en de waarschuwingen over de
mogelijke verspreiding van exotische virussen geven aan dat microben in
de toekomst wellicht niet zo megapolitiek onschadelijk blijven als in de
afgelopen vijfhonderd jaar. Wanneer een nieuwe plaag de wereld zal
treffen, blijft echter onvoorspelbaar. Een uitbraak van microparasieten,
bijvoorbeeld in de vorm van een virale pandemie, zou waarschijnlijk
eerder de megapolitieke dominantie van technologie ondermijnen dan
ingrijpende veranderingen in klimaat of terrein.

Wij kunnen ingrijpende veranderingen in de aard van het leven op aarde,
zoals we die tot nu toe hebben gekend, niet monitoren of voorspellen.
Wij kruisen onze vingers en vertrouwen erop dat de voornaamste
megapolitieke factoren in het komende millennium technologisch zullen
zijn in plaats van microbiologisch. Als het geluk de mensheid blijft
gunstig gezind, zal technologie steeds prominenter opgaan als de
voornaamste megapolitieke variabele.

Het is echter nooit altijd zo geweest, zoals duidelijk wordt uit een
overzicht van de eerste grote megapolitieke transformatie: de agrarische
revolutie.

\setsubtitle{Subtitle}

\bookmarksetup{startatroot}

\chapter{East of Eden}\label{east-of-eden}

\begin{quote}
`En de Heer zei tegen Kaïn: ``Waar is Abel, jouw broer?'' Kaïn
antwoordde: ``Ik weet het niet. Ben ik dan wel de hoeder van mijn
broeders?'' En Hij vervolgde: ``Wat heb je gedaan? De stem van het bloed
van jouw broeders roept uit de aarde tot mij.''\,' - GENESIS 4:9-10
\end{quote}

Vijfhonderd generaties geleden zette de organisatie van de menselijke
samenleving haar eerste ingrijpende fase in gang.\footnote{Boyden, op.
  cit., p.~4.}

Onze voorouders in diverse regio's pakten met tegenzin ruwe werktuigen,
aangeslepen palen en geïmproviseerde hakken op en gingen aan het werk.
Terwijl zij de eerste gewassen zaaiden, legden zij tevens de basis voor
een nieuwe machtsstructuur in de wereld. De agrarische revolutie
markeerde de eerste grote economische en sociale omwenteling. Ze begon
met de verdrijving uit Eden en verliep zo traag dat de landbouw in alle
geschikte gebieden op aarde de jacht en het verzamelen nog niet volledig
had vervangen, toen de twintigste eeuw aanbrak. Deskundigen menen dat
zelfs in het Nabije Oosten -- waar de landbouw voor het eerst opkwam --
de invoering daarvan het resultaat was van `een langdurig, stapsgewijs
proces' dat `vijfduizend jaar of langer zou kunnen hebben
geduurd.'\footnote{Gregg, op. cit., xv.}

Het lijkt misschien overdreven om een proces dat zich over millennia
afspeelde een `revolutie' te noemen. Maar dat is precies wat de opkomst
van de landbouw inhield: een revolutie die zich in een langzaam tempo
voltrok en het menselijk leven ingrijpend veranderde door de logica van
geweld te herscheppen. Naarmate landbouw terrein won, kreeg geweld een
belangrijkere rol in het sociale leven. Hiërarchieën die bedreven waren
in het manipuleren of beheersen van geweld, kregen uiteindelijk de
overhand.

Inzicht in de agrarische revolutie vormt de eerste stap naar het
doorgronden van de informatierrevolutie. De invoering van ploegen en
oogstmethoden is een treffend voorbeeld van hoe een schijnbaar
eenvoudige wijziging in de aard van arbeid de organisatie van de
samenleving radicaal kan hervormen. Wanneer je deze vroege revolutie in
perspectief plaatst, ben je veel beter in staat te voorspellen hoe de
geschiedenis zich mogelijk ontvouwt als reactie op de nieuwe logica van
geweld die met microprocessors is geïntroduceerd.

Om het revolutionaire karakter van de landbouw te waarderen, heb je
eerst een beeld nodig van hoe de oeroude samenleving functioneerde vóór
de komst van de landbouw. We hebben dit behandeld in \emph{The Great
Reckoning} en hieronder volgt een nadere schets. Jagers-verzamelaars
samenlevingen waren gedurende een lange, prehistorische periode de enige
vorm van sociale organisatie, waarbij het menselijk bestaan van
generatie op generatie nauwelijks veranderde. Antropologen stellen dat
mensen sinds hun ontstaan op aarde gedurende 99 procent van de tijd
jagers en verzamelaars zijn geweest. Cruciaal voor zowel het langdurige
succes als het uiteindelijke falen van deze groepen was dat zij op zeer
kleine schaal binnen een uitgestrekt gebied opereerden.

Verzamelaars konden alleen overleven op plekken waar de
bevolkingsdichtheid laag bleef. Om te begrijpen waarom, kun je je de
problemen voorstellen die grotere groepen met zich meebrachten. Stel je
bijvoorbeeld voor dat duizend jagers gezamenlijk door het landschap
trekken: hun rumoer zou het wild dat zij wilden vangen direct verjagen.
En nog erger: als een klein leger jagers er af en toe in slaagde een
enorme kudde wild te omsingelen, dan zou het voedsel -- waaronder
vruchten en eetbare wilde planten -- al snel schaars worden. Een grote
groep jagers-verzamelaars zou het landschap verwoesten door
buitensporige jacht, vergelijkbaar met een hongerig leger in de
Dertigjarige Oorlog. Daarom moesten de jachtpartijen klein blijven.
Zoals Stephen Boyden schrijft in \emph{Western Civilization in
Biological Perspective}, `bestaan jagers-verzamelaarsgroepen doorgaans
uit tussen de vijfentwintig en vijftig personen.'\footnote{Boyden, op.
  cit., p.~62.}

Tegenwoordig is het leven op tienduizend acres in een gematigd klimaat
een luxe die enkel de allerrijksten zich kunnen veroorloven. Een familie
van jagers‐verzamelaars zou met minder nauwelijks hebben overleefd; zij
hadden gemiddeld duizenden acres per persoon nodig, zelfs in de
vruchtbaarste verzamelgebieden. Dit verklaart hoe de bevolkingsgroei
tijdens bijzonder gunstige landbouwperiodes uiteindelijk kon leiden tot
voedingscrises. Omdat er zoveel land nodig was om één individu te
onderhouden, bleef de bevolkingsdichtheid in deze samenlevingen extreem
laag. Vóór de landbouw leefden mensen ongeveer zo verspreid als beren.

Het menselijke dieet vertoonde nauwelijks verschillen met dat van beren.
Jagers‐verzamelaars waren volledig afhankelijk van het voedsel dat zij
in openvelden of langs waterwegen verzamelden. Hoewel enkelen ook als
vissers fungeerden, bestond hun dieet voor een derde tot een vijfde uit
eiwitten afkomstig van groot zoogwild. Buiten een handje eenvoudige
werktuigen en persoonlijke bezittingen beschikten deze groepen over
vrijwel geen technologie. Zij vonden geen effectieve manier om grote
hoeveelheden vlees of ander voedsel voor later gebruik te bewaren; het
merendeel moest vrijwel direct na het verzamelen worden verorberd om
bederf te voorkomen.

Dit houdt echter niet in dat sommige jagers‐verzamelaars geen bedorven
etenswaren aten. Volgens Boyden hebben Eskimo's naar verluidt een
uitgesproken voorliefde voor ontbonden voedsel.\footnote{Ibid., p.~67.}
Hij citeert deskundigen die opmerken dat Eskimo's `viskoppen begraven en
laten rotten totdat de botten dezelfde consistentie bereiken als het
vlees. Vervolgens kneden zij de stinkende massa tot een pasta en eten
die', en dat zij ook genieten van `de vette, wormige larven van de
rendiervlieg, rauw geserveerd; van hertenuitwerpselen, gekauwd als
bessen \ldots{} en van merg dat meer dan een jaar oud is en bezaaid is
met wormen'.\footnote{Ibid.}

Behalve voor zulke lekkernijen produceerden jagers‐verzamelaars
nauwelijks overtollig voedsel. Zoals antropoloog Gregg opmerkt: `Mobiele
populaties slaan zelden voedselvoorraden aan voor seizoensgebonden of
onverwachte tekorten.' Bijgevolg hadden zij weinig extra's om mee te
werken. Een arbeidsverdeling waarin zelfs specialisatie in het gebruik
van geweld aan de orde was, bleek onhoudbaar in situaties waarin
overtollig voedsel niet kon worden opgeslagen. Bovendien bepaalde de
logica van de jacht dat geweld binnen deze groepen nooit boven een
beperkt niveau mocht uitgroeien, omdat de groepen zelf klein moesten
blijven.

De kleinschaligheid van de jagers‐verzamelgroepen had nog een ander
voordeel: de leden kenden elkaar door en door, wat de samenwerking ten
goede kwam. Besluitvorming wordt immers complexer naarmate meer mensen
erbij betrokken raken -- bedenk maar hoe lastig het is om een dozijn
mensen samen uit eten te laten gaan. Stel je eens voor hoe hopeloos het
zou zijn geweest om honderden of zelfs duizenden mensen bij elkaar te
krijgen voor een rondtrekkend feest. Zonder een permanente, aparte
politieke organisatie of bureaucratie -- zoals vaak nodig is bij
specialisatie voor oorlogvoering -- moesten deze groepen vertrouwen op
overreding en consensus; methoden die het beste werken binnen kleine,
relatief ontspannen gemeenschappen.

Of jagers‐en verzamelgroepen werkelijk ontspannen waren, blijft
onderwerp van discussie. Sir Henry Maine wijst op `de universele
strijdlust van de primitieve mens' en stelt: `Het is niet de vrede die
van nature voorkomt, maar de oorlog.' Zijn visie wordt ondersteund door
evolutionaire biologen, zoals R. Paul Shaw en Yuwa Wong, die sterke
aanwijzingen vonden dat veel verwondingen bij resten van
Australopithecus, Homo erectus en Homo sapiens uit de Europese vierde en
pre-vierde ijstijd het gevolg waren van gevechten. Anderen betwijfelen
dit echter en beweren, zoals Stephen Boyden, dat primitieve groepen
doorgaans niet oorlogszuchtig of geneigd tot geweld waren. Zij
ontwikkelden sociale conventies om interne spanningen te beperken en de
verdeling van de buit bij de jacht te stroomlijnen. Vooral in gebieden
waar op groot wild werd gejaagd -- wat voor een enkele jager vrijwel
onmogelijk was -- ontstonden religieuze en sociale doctrines die
bepaalden hoe de buit over de hele groep verdeeld moest worden.
Medejagers maakten als eersten aanspraak op de caloriebronnen; noodzaak,
en niet sentiment, dreef hen. Zij met de beste economische vaardigheden
en de grootste militaire kracht claimden de hulpbronnen, niet de zieken
en zwakkeren. Uiteraard speelde het ook mee dat jagers in de bloei van
hun leven vaak de militair sterkste leden van de groep waren. Door hen
het eerste recht op de jacht toe te kennen, verkleinde de groep de kans
op dodelijke interne conflicten.

Zolang de bevolkingsdichtheid laag bleef, waren de goden bij
jagers‐verzamelaars niet oorlogszuchtig, maar eerder personificaties van
natuurelementen of van de dieren die zij jaagden. De schaarste aan
kapitaal en de open grensoverschrijdende relaties maakten oorlog vaak
overbodig, omdat er nauwelijks buren buiten de eigen kleinfamilie of
stam waren die als bedreiging konden gelden. Omdat jagers‐verzamelaars
vaak rondzwierven op zoek naar voedsel, waren persoonlijke bezittingen
boven een absoluut minimum juist een last. Degenen die weinig bezaten,
raakten daardoor vrijwel onvermijdelijk niet betrokken bij
vermogensdelicten. Bij conflicten kozen de betrokkenen er vaak voor
gewoon weg te lopen, omdat zij nauwelijks ergens in hadden geïnvesteerd.
Weglopen bleek een makkelijke oplossing voor persoonlijke vetes of
buitensporige eisen van andere aard. Dit betekent echter niet dat de
vroege mens per definitie vredelievend was. Ze konden verrassend
gewelddadig en onaangenaam optreden, in een mate die we ons nauwelijks
kunnen voorstellen. En als ze geweld gebruikten, deed dat meestal voor
persoonlijke redenen of -- wat wellicht nog erger is -- puur voor de
lol.

Het levensonderhoud van jagers‐verzamelaars hing af van hun functioneren
in kleine groepjes, waar nauwelijks ruimte was voor een andere
arbeidsverdeling dan langs traditionele geslachtslijnen. Ze kenden geen
georganiseerde overheid, woonden meestal niet permanent en hadden geen
mogelijkheid om echt rijkdom op te bouwen. Zelfs bouwstenen van
beschaving zoals een geschreven taal waren in die oeroude economie
onbekend. Zonder geschreven taal ontstonden er geen formele documenten
en dus ook geen geschiedenis.

\subsection{Overkill}\label{overkill}

De dynamiek van het foerageren bood totaal andere prikkels om te werken
dan wat we kennen sinds de opkomst van de landbouw. De kapitaalinzet
voor een foeragerslevensstijl bleef minimaal; een paar primitieve
gereedschappen en wapens volstonden. Investeren kwam nauwelijks aan de
orde, zelfs niet in de vorm van privébezit van land -- afgezien van
incidentele investeringen in steengroeven waar vuursteen of zeepsteen
gewonnen werd.\footnote{Zie Carleton S. Coon, \emph{The Hunting Peoples}
  (New York: Nick Lyons Books, 1971), p.~275.} Zoals antropoloog Susan
Alling Gregg schreef in \emph{Foragers and Farmers}, `het eigendom van
en de toegang tot hulpbronnen werd gezamenlijk beheerd door de
groep.'\footnote{Gregg, op. cit., p.~23.} Met uitzondering van enkele
gevallen, zoals vissers die langs meren woonden, hadden foeragers
doorgaans geen vaste woonplaats. Omdat ze geen permanente woningen
bezaten, behoefden ze niet hard te werken om eigendom te vergaren of in
stand te houden. Ze hoefden geen hypotheken of belastingen te betalen en
kochten geen meubels aan. Hun weinige consumentengoederen bestonden uit
dierenhuiden en persoonlijke versieringen, vervaardigd door de
groepsleden zelf. Er was weinig stimulans om iets te bezitten of op te
hopen dat als geld had kunnen gelden, omdat er vrijwel niets te koop
was. Onder dergelijke omstandigheden bleef sparen voor foeragers hooguit
een rudimentair concept.

Zonder reden om te verdienen en met nauwelijks enige arbeidsverdeling,
lag het idee van hard werken als deugd ver van de realiteit voor
jagers‐verzamelaars. Behalve tijdens perioden van zware ontbering,
waarin langdurige inspanning nodig was om voedsel te vinden, werkte men
doorgaans zo weinig als mogelijk, omdat er nauwelijks iets wint bij
harder werken dan het absolute minimum om te overleven. Voor de typische
jagers‐verzamelaarsgroep betekende dit dat ze zo'n acht tot vijftien uur
per week werkten.\footnote{Boyden, op. cit., p.~69.} Aangezien de
inspanningen van een jager de voedselvoorraad niet vergrootten maar
juist verminderden, bracht het heldhaftig overuren maken om extra dieren
te doden of meer fruit te verzamelen dan tijdig geconsumeerd kon worden,
niets voor de welvaart. Integendeel, overkill verzwakte de
vooruitzichten op toekomstige voedselvoorziening en had daardoor een
nadelig effect op het welzijn van de groep. Daarom bestraften of
bespotten sommige foeragers -- zoals de Eskimo's -- groepsleden die zich
schuldig maakten aan overdaad in de jacht.

Het voorbeeld van de Eskimo's die buitensporige jacht bestraffen is
bijzonder treffend, want zij zouden veel meer dan anderen in staat zijn
geweest vlees te conserveren door het te bevriezen. Ten minste had men
opslag kunnen realiseren voor oliën die uit grote zeezoogdieren werden
gewonnen. Dat foeragers er doorgaans voor kozen dit niet te doen, wijst
op hun veel meer passieve omgang met de natuur. Dit duidt tevens op de
mate waarin cultuur het denken en de mentale processen beïnvloedt.
Beperkingen in het leervermogen en gedrag in complexe omgevingen maken
het toepassen van bepaalde strategieën een stuk lastiger dan op het
eerste gezicht lijkt. Zoals R. Paul Shaw en Yuwa Wong schreven: `omdat
niches op vele vlakken verschillen, doen ook de vooroordelen in het
leren dat.'\footnote{Shaw en Wong, op. cit., p.~69.}

In dit opzicht betekende de opkomst van de landbouw niet alleen een
verandering in het dieet, maar luidde zij tevens een ware revolutie in
de organisatie van het economisch en cultureel leven in en leidde tot
een transformatie van de logica van geweld. Door landbouw ontstonden op
grote schaal kapitaalgoederen, zoals land en irrigatiesystemen. De
gewassen en gedomesticeerde dieren die boeren verbouwden, waren
waardevolle bezittingen. Ze konden worden opgeslagen, opgepot en
gestolen. Doordat er gedurende het volledige groeiseizoen zorg nodig was
-- van het planten tot de oogst -- werd migratie als reactie op
bedreigingen minder aantrekkelijk, met name in droge gebieden waar de
kans op succesvolle gewasverbouwing beperkt bleef tot kleine stukken
land met betrouwbare watervoorzieningen. Naarmate het lastiger werd om
te vluchten, groeiden de mogelijkheden voor georganiseerde afpersing en
plundering. Tijdens de oogst werden boeren geconfronteerd met raids,
waardoor de omvang van oorlogsvoering geleidelijk toenam.

Dit leidde tot grotere samenlevingen, omdat gewelddadige conflicten
vrijwel altijd door de sterkste, grotere groep werden gewonnen. Naarmate
de concurrentie om land en de controle over de opbrengst intenser werd,
vestigden samenlevingen zich steeds vaker op één plek. Er ontstond een
duidelijke arbeidsverdeling. Voor het eerst kwamen zowel
gespecialiseerde arbeid als slavernij op; boeren en herders
specialiseerden zich in voedselproductie, pottenbakkers maakten
aardewerk en andere bewaardozen voor de opslag van voedsel en priesters
baden voor regen en overvloedige oogsten. Specialisten in geweld --
voorlopers van de staat -- waren steeds actiever in zowel plunderen als
in het bieden van bescherming tegen plunderingen. Samen met de priesters
werden zij de eerste welvarende personen in de geschiedenis.

In de beginfase van agrarische samenlevingen verwierven deze krijgers
een deel van de jaarlijkse oogst als vergoeding voor hun beschermende
diensten. Op plaatsen waar de dreiging gering was, konden zelfstandige
boeren vaak een relatief hoge mate van zelfstandigheid behouden. Maar
naarmate de bevolkingsdichtheid groeide en de concurrentie om voedsel
intenser werd -- vooral in gebieden rondom woestijnen, waar vruchtbaar
land schaars was -- kon de krijgersgroep een aanzienlijk deel van de
totale oogst opeisen. Met de opbrengsten van deze afpersingspraktijken,
die wel 25 procent van de graanoogst en tot de helft van de toename van
de kuddes gedomesticeerde dieren bedroegen, stichtten de krijgers de
eerste staten. Op die manier nam het belang van dwang sterk toe; de
groei in de middelen die geplunderd konden worden, leidde tot een
significante stijging van plunderingen.

Het kostte duizenden jaren voordat de volledige logica van de agrarische
revolutie tot uiting kwam. Lange tijd leefden verspreide
boerenpopulaties in gematigde streken mogelijk nog op dezelfde manier
als hun jagers-verzamelaarsvoorouders. In gebieden waar land en neerslag
overvloedig waren, oogstten boeren op kleine schaal, doorgaans zonder
veel gewelddadige inmenging. Echter, naarmate de bevolking in de loop
van enkele duizenden jaren groeide, werden zelfs boeren in dunbevolkte
gebieden getroffen door onvoorspelbare plunderingen, waardoor zij soms
met te weinig zaad achterbleven om de volgende oogst aan te planten.
Enerzijds dreigde voortdurende, concurrerende plundering -- oftewel
anarchie -- en anderzijds leefden gemeenschappen onbeschermd, zonder
enige gespecialiseerde organisatie om geweld te reguleren.

Na verloop van tijd verspreidde de inherent aan de landbouw zijnde
geweldslogica haar invloed over een steeds groter wordend gebied. Vrije
landbouw en veeteelt konden namelijk alleen in enkele werkelijk
afgelegen streken floreren, vrij van overheidsroof. Als extreem
voorbeeld boden de Kafir-regio's in Afghanistan tot de laatste tien jaar
van de negentiende eeuw stug verzet tegen de vestiging van een overheid.
Zij hadden zich al eeuwenlang ontwikkeld tot een behoorlijk militante
samenleving, georganiseerd langs verwantschapslijnen, en slaagden er
niet in om op grote schaal hun kracht te bundelen. Totdat de Britten
moderne wapens naar de regio brachten, behielden de Kafirs hun
onafhankelijkheid in de afgelegen Bashgal- en Waigal-valleien, beschermd
door natuurlijke barrières als hoge bergen en uitgestrekte woestijnen
die hen van buitenlandse veroveraars scheidden.

In de samenlevingen waar landbouw voet aan de grond kreeg, drong de
basislogica van de agrarische revolutie door. De landbouw breidde de
mogelijkheden voor het vormen van menselijke gemeenschappen fors uit.
Rond tienduizend jaar geleden ontstonden de eerste steden. Hoewel deze
in moderne maatstaven nog klein waren, fungeerden zij als centra van de
eerste `beschavingen' -- een term afgeleid van `civitas', wat in het
Latijn `burgerschap' of `stadsbewoners' betekent. Omdat de landbouw
bezittingen voortbracht die geplunderd en beschermd moesten worden,
ontwikkelde zich de behoefte aan een systematische inventarisatie en
boekhouding. Je kunt immers geen belastingen heffen zonder een gedegen
administratie bij te houden en kwitanties uit te reiken. De symbolen in
het grootboek van de boekhouder legden de basisprincipes van de
geschreven taal -- een innovatie die bij jagers en verzamelaars nog
onbekend bleef.

De landbouw verbreedde tevens de reikwijdte van de problemen die mensen
moesten oplossen. Jagersstammen opereerden binnen een zeer beperkte
tijdshorizon en startten zelden projecten die langer dan enkele dagen
duurden. Daarentegen duurde het planten en oogsten van gewassen vaak
maanden. Het uitvoeren van dergelijk langetermijnproject dwong boeren om
hun blik tot in de sterren te richten. Nauwkeurige astronomische
observaties vormden daarbij de basis voor het opstellen van almanakken
en kalenders, die als gids dienden om het juiste moment voor planten en
oogsten te bepalen. Zelfs de tijdshorizon van de jagers breidde zich met
de komst van de landbouw uit.

\section{Eigendom}\label{eigendom}

De overgang naar een gevestigde landbouwsamenleving leidde
onvermijdelijk tot het ontstaan van privébezit. Niemand zou genoegen
nemen met ploeteren door een heel groeiseizoen om een gewas te
verbouwen, alleen om erachter te komen dat iemand anders opdook en
oogstte wat hij had geproduceerd. Het idee van eigendom was een directe
consequentie van de landbouw, maar de duidelijkheid van het begrip
privébezit werd overschaduwd door de geweldslogica die met de landbouw
opkwam. De opkomst van eigendom werd tevens bemoeilijkt doordat de
machtsbalans niet langer gelijk verdeeld bleef, zoals in
verzamelgemeenschappen waar iedere gezonde volwassen man even goed
bewapend was. De landbouw stimuleerde bovendien een specialisatie in
geweld. Juist omdat zij iets voortbracht dat gestolen kon worden, maakte
deze ontwikkeling investeringen in betere wapens winstgevend. Het
resultaat was diefstal, vaak in zeer georganiseerde vorm.

De machthebbers slaagden erin een nieuwe vorm van roofzucht te
organiseren, namelijk een monopolie op geweld: de overheid. Hierdoor
ontstond een scherpe verdeling in de samenlevingen. De omstandigheden
verschilden radicaal voor degenen die profiteerden van plunderingen en
voor de grote massa armen die op de akkers werkten. Degenen met
militaire macht konden rijkdom vergaren, samen met anderen die hun gunst
genoten. De godkoningen en hun bondgenoten -- de diverse lagere, lokale
potentaten die de eerste staten in het Nabije Oosten bestuurden --
beschikten over bezittingen in allerlei vormen die we tegenwoordig als
eigendom beschouwen, terwijl de overgrote meerderheid hiervoor
ploeterde.

Het is uiteraard anachronistisch om in vroege landbouwsamenlevingen een
onderscheid te maken tussen privé- en publiek bezit. De heersende
godkoning beheerste alle staatsmiddelen alsof hij een uitgestrekt
landgoed bezat. Net zoals in de Europese feodale periode lag al het
eigendom onder het gezag van hogere potentaten. Zij die lager in de
hiërarchie stonden, moesten zien hoe de heerser naar eigen zin hun
bezittingen aanpakte.

Dat de potentaat geen wettelijke rem voelde, betekende zeker niet dat
hij zich willekeurig alles toe-eigende wat hij maar wilde. Ook kosten en
beloningen beperkten de vrijheid van de farao, net zo zwaar als die van
de hedendaagse premier van Canada. Daarnaast ondervond de farao veel
meer beperkingen door problemen met transport en communicatie dan
moderne leiders vandaag de dag ervaren. Het verplaatsen van buit --
zeker wanneer het grotendeels om landbouwproducten ging -- ging gepaard
met aanzienlijke verliezen door bederf en diefstal. Hoewel een groter
aantal functionarissen de kleine diefstallen kon tegengaan, dwong dat de
farao tot hogere vaste kosten. Ondertussen stimuleerde
gedecentraliseerde bevoegdheid, die in bepaalde situaties de productie
optimaliseerde, het ontstaan van sterkere lokale machten die weleens
volwaardige rivalen werden in de strijd om dynastieke controle. Zelfs
oosterse despoten handelden niet naar eigen goeddunken; zij moesten de
bestaande machtsbalans aanvaarden.

Hoewel iedereen -- zelfs de rijken -- het risico liep op willekeurige
onteigening, slaagden sommigen er desondanks in eigen bezit te vergaren.
Net als nu besteedde de staat vroeger een groot deel van haar inkomsten
aan openbare werken. Projecten als irrigatiesystemen, religieuze
monumenten en grafkelders voor de koningen gaven architecten en
ambachtslieden volop kansen om een inkomen te verdienen. Een handjevol
gunstig gepositioneerde personen vergaarde aanzienlijke particuliere
bezittingen. Sterker nog, een groot aantal bewaarde
spijkerschrifttabletten uit Soemer -- een vroege Mesopotamische
beschaving -- documenteert uiteenlopende handelsacties, waarvan de
meeste de overdracht van eigendomsrechten vastleggen.

Hoewel in de vroege landbouwsamenlevingen sprake was van privébezit,
kwam dit nauwelijks voor aan de onderzijde van de sociale piramide. De
overgrote meerderheid bestond uit boeren die zo arm waren dat ze
nauwelijks vermogen konden vergaren. Met enkele uitzonderingen waren de
meeste boeren tot in de moderne tijd zo arm dat zij voortdurend dreigden
te verhongeren wanneer droogte, overstromingen of plagen hun oogst
vernielden. Daarom organiseerden de boeren hun zaken zodanig dat zij in
slechte jaren de neerwaartse risico's zoveel mogelijk konden beperken.
Binnen de brede, arme laag van de samenleving domineerde een meer
primitieve vorm van eigendomsorganisatie. Dit bevorderde de
overlevingskansen, maar ging ten koste van de mogelijkheden om kapitaal
te vergaren en economisch door te groeien.

\subsection{Boerenverzekering}\label{boerenverzekering}

De regeling kreeg vorm door de invoering van wat antropologen en sociale
historici het `gesloten dorp' noemen. In bijna elke boerenmaatschappij
in de premoderne tijd vormde dit gesloten dorp de kern van de
economische organisatie. Bij moderne systemen hebben individuen te maken
met talloze kopers en verkopers op een open markt, maar in een gesloten
dorp bundelden huishoudens hun krachten en werkten ze samen als een
informele corporatie of als één grote familie. Ze handelden binnen een
afgesloten systeem, waarbij vrijwel alle economische transacties via één
monopolist liepen -- namelijk de plaatselijke landeigenaar of diens
vertegenwoordiger, meestal het dorpshoofd.

Het hele dorp sloot een contract met de landeigenaar, doorgaans op basis
van betaling in natura: een groot deel van de oogst werd ingeleverd in
plaats van dat een vaste huur werd betaald. Deze afspraak hield in dat
de landeigenaar een deel van het risico van een mislukte oogst op zich
nam, terwijl hij uiteraard het merendeel van de winst verzilverde.
Bovendien verzorgde hij meestal ook het zaad.

Deze regeling beperkte tevens het risico op hongersnood. Men ging ervan
uit dat juist de landeigenaar -- en niet de boer -- een onevenredig
groot deel van zijn aandeel in de oogst opzij zou zetten. Doordat de
landbouwopbrengsten vroeger op veel plaatsen afschuwelijk laag waren,
hanteerde men soms zo'n verhouding dat er voor elke drie geoogste een
extra zaadlaag nodig was. In zulke omstandigheden zou een slechte oogst
leiden tot massale hongersnood. Rationeel kozen de boeren er daarom voor
een regeling waarbij de landeigenaar investeerde in hun overleving.
Hoewel zij opdraaiden voor het kopen tegen monopolistische prijzen, het
goedkoop verkopen en het leveren van arbeid in natura voor de
landeigenaar, vergrootte deze opstelling hun overlevingskansen.

Een vergelijkbare drijfveer zorgde er ook voor dat de typische boer in
een gesloten dorpsgemeenschap afstand liet doen van de zekerheid van
vrij eigendom. Door zich over te geven aan de grillen van het
dorpshoofd, vergrootte een boerengezin zijn kansen om te profiteren van
de regelmatige herverdeling van akkers. Niet zelden bemachtigde het
dorpshoofd de beste velden voor zichzelf en zijn favorieten, maar dat
risico accepteerden de boeren in ruil voor de overlevingsverzekering die
het complexe systeem van dorpsbezit bood. Wanneer de oogstopbrengsten
erbarmelijk laag waren, konden zelfs kleine verschillen in
teeltomstandigheden tussen velden die slechts honderd meter van elkaar
verwijderd waren, het verschil betekenen tussen hongersnood en
overleving. Boeren verkiezen dan vaak de regeling die het risico op
verliezen beperkt, ook al betekent dat dat zij alle hoop op groeiende
welvaart moeten opgeven.

Risicomijdend gedrag kenmerkte over het algemeen alle groepen die op de
rand van overleving leefden. De pure noodzaak om te overleven in
premoderne samenlevingen beperkte immers altijd het handelingsrepertoire
van de armen. Een interessant aspect van deze risicomijding, zoals
besproken in \emph{The Great Reckoning}, is dat zij het palet aan
sociaal aanvaardbaar economisch gedrag verkleinde. Taboes en sociale
beperkingen beperkten experimentatie en innovatie, zelfs als dat inhield
dat mensen potentiële en voordelige verbeteringen in gevestigde
werkwijzen moesten opgeven. Dit weerspiegelde op rationele wijze dat
experimenteren niet alleen de kans op hogere winsten vergrootte, maar --
wat voor hen die op het randje van overleving leefden des te dreigender
was -- ook leidde tot potentieel ruinieuze verliezen. Een groot deel van
de culturele energie in arme landbouwsamenlevingen ging dan ook uit naar
het ontmoedigen van experimenten, wat in feite diende als vervanging van
verzekeringspolissen. Beschikten de boeren over een verzekering of
voldoende spaargeld om de risico's van experimenten zelf op te vangen,
dan waren zulke strenge sociale beperkingen overbodig om hun overleving
veilig te stellen.

Culturen zijn niet louter een kwestie van smaak, maar systemen van
aanpassing aan specifieke omstandigheden die in andere contexten
irrelevant of zelfs contraproductief kunnen blijken te zijn. Mensen
leven in een enorme diversiteit aan habitats, en het overweldigende
aantal potentiële niches waarin we ons bevinden vraagt om
gedragsvariaties die te complex zijn om louter op instinct te baseren.
Daarom wordt ons gedrag cultureel geprogrammeerd. In veel
landbouwsamenlevingen zorgde de cultuur er dan ook voor dat mensen zich
vrijwel uitsluitend op overleving richtten, terwijl de luxe om deel te
nemen aan open markten meestal voorbehouden bleef aan anderen.

Taboes en sociale beperkingen beperkten de persoonlijke bekwaamheid en
keuzevrijheid -- het moderne ideaal van `het nastreven van geluk' --
vooral bij de armen. In samenlevingen met lage productiviteit lukte men
er maar zelden in deze beperkingen een halt toe te roepen. Waar de
landbouwproductiviteit hoger lag, zoals in het oude Griekenland,
ontstonden kleinere megapolitieke revoluties. Eigendom kreeg een
modernere invulling: zo ontstond `allod', oftewel vrij eigendom. Land
werd doorgaans verhuurd tegen een vaste vergoeding, waarbij de huurder
zowel het economische risico als een groter deel van de winst op zich
nam wanneer de oogst goed was. Hogere besparingen stelden hen in staat
om risico's zelf te dragen. Zo konden vrije boeren boven de gewone
boeren uitgroeien en soms zelfs een zelfstandig vermogen opbouwen.

Het feit dat marktgerichte eigendomsrechten en -verhoudingen aan de top
van de economische hiërarchie, of in zeldzame gevallen in de gehele
economie, naar voren kwamen naarmate samenlevingen uit armoede kwamen,
kenmerkt de sociale organisatie. Even belangrijk is te constateren dat
de meest voorkomende organisatiestructuur van landbouwsamenlevingen
historisch gezien in wezen feodaal was, met marktrelaties aan de top en
een gesloten dorpssysteem als basis. Bijna alle boeren in premoderne
landbouwsamenlevingen waren aan het land gebonden. Zolang de
landbouwproductiviteit laag bleef of een hogere productiviteit
afhankelijk was van gecentraliseerde waterbeheersystemen, genoten
individuele boeren aan de onderkant nauwelijks vrijheid of
eigendomsrechten. In die omstandigheden domineerden feodale
eigendomsverhoudingen: men verkreeg land via erfpacht in plaats van via
vrij eigendom, en de rechten op verkoop, schenking en erfopvolging waren
sterk beperkt.

Feodalisme reageerde in al zijn vormen niet alleen op de voortdurende
dreiging van roofzuchtig geweld, maar ook op extreem lage
productiviteitsniveaus. In landbouwsamenlevingen gingen beide vaak hand
in hand en versterkten ze elkaar. Toen de openbare autoriteit instortte,
zakten zowel eigendomsrechten als welvaart mee. Het verval van de
productiviteit ondermijnde bovendien het gezag. Hoewel niet elke droogte
of ongunstige klimaatverandering leidde tot de val van de publieke orde,
gebeurde dat in veel gevallen wel.

\section{De feodale revolutie van het jaar
1000}\label{de-feodale-revolutie-van-het-jaar-1000}

Precies dat voltrekte zich tijdens de transformatie rond het jaar 1000,
die de feodale revolutie in gang zette.\footnote{Zie Bois, op.~cit.} In
die periode verschilden de megapolitieke en economische omstandigheden
wezenlijk van wat men tegenwoordig met de Middeleeuwen associeert. In de
eerste eeuwen na de val van Rome verzwakte de economie van West-Europa
aanzienlijk. De Germaanse koninkrijken die zich in de gebieden van het
voormalige Romeinse Rijk vestigden, namen veel functies van de Romeinse
staat over, zij het op een veel minder ambitieus niveau. De
infrastructuur raakte vrijwel verwaarloosd; in de loop der eeuwen
vervielen bruggen en aquaducten en werden ze onbruikbaar. Hoewel de
Romeinse munten nog enige tijd in omloop bleven, verdween hun circulatie
bijna geheel. De grondmarkten, die in Romeinse tijden bloeiden, droogden
op en steden -- ooit centra van Romeinse administratie -- verdwenen
vrijwel samen met de belastingkracht van de staat. Het merendeel van de
beschavingskenmerken verdween eveneens.

De `donkere middeleeuwen' hebben hun naam niet voor niets gekregen.
Geletterdheid werd zo schaars dat bijna iedereen die nog kon lezen en
schrijven, voor vrijwel elk misdrijf -- zelfs voor moord -- bespaard
bleef van vervolging. De kunstzinnige, wetenschappelijke en technische
vaardigheden die in de Romeinse tijd zo ver ontwikkeld waren, verdwenen
volledig. Of het nu ging om de aanleg van wegen of om het enten van
wijnstokken en vruchtbomen: West-Europa deed geen beroep meer op de
technieken die ooit algemeen bekend en op hoog niveau beoefend werden.
Zelfs een oud instrument als de pottenbakkersschijf raakte op talrijke
plaatsen in de vergetelheid. Mijnbouwwerkzaamheden krompen in, de
metallurgie kwam tot stilstand en de irrigatiewerken in de mediterrane
regio stortten door verwaarlozing in elkaar.\footnote{Zie Frances en
  Joseph Gies, Cathedral, Forge, and Waterwheel: Technology and
  Invention in the Middle Ages (New York: HarperCollins, 1994), p.~40.}
Zoals historicus Georges Duby opmerkte: `Aan het eind van de zesde eeuw
was Europa een diep ongeciviliseerde plaats.'\footnote{Geciteerd in
  ibid., p.~42.} Rond het jaar 800, onder de heerschappij van Karel de
Grote, deed een korte renaissance van centrale autoriteit zich voor,
maar na zijn dood verviel alles al snel weer in desorganisatie.

Een verrassend gevolg van dit sombere landschap bleek dat de
ineenstorting van de Romeinse staat de levensstandaard van kleine boeren
enkele eeuwen lang waarschijnlijk verbeterde. De Germaanse koninkrijken
die tijdens de donkere middeleeuwen West-Europa domineerden, namen
enkele relatief soepele sociale kenmerken over uit hun voorouderlijke
stammen, zoals de wettelijke gelijkheid van vrije landeigenaren.
Hierdoor genoten de kleine boeren in deze periode van meer vrijheid dan
zij in latere feodale tijden zouden ervaren -- wat erop wijst dat zij
tevens welvarender waren. Zoals eerder opgemerkt bij de analyse van
eigendomsvormen onder verschillende omstandigheden van productiviteit,
ging vrije eigendom historisch gezien altijd gepaard met een hogere
welvaart onder de kleine boeren. Gesloten dorps- en feodale
eigendomsstructuren ontwikkelden zich juist op plekken waar de kansen
voor kleine boeren om in hun levensonderhoud te voorzien beperkter
waren.

De vrijwel volledige ineenstorting van de handel tijdens de donkere
middeleeuwen beroofde kleine boeren niet alleen van de vruchten van
bredere markten, maar ook van de voordelen die met uitgebreide
handelsbetrekkingen gepaard gingen. Het verval van de steden verzwakte
de contante economie, maar zorgde er tegelijkertijd voor dat de
landelijke bevolking niet langer de zware last van bureaucratie hoefde
te dragen. Guy Bois omschreef de Romeinse stad namelijk als een
parasitaire gemeenschap -- geen centrum van productie: `In de Romeinse
periode was de dominante functie van een stad politiek van aard. Zij
leefde hoofdzakelijk van de opbrengsten die via de grondbelasting uit
haar omgeving naar binnen stroomden\ldots. In wezen produceerde de stad
weinig of niets ten behoeve van het omliggende platteland.'\footnote{Bois,
  op. cit., p.~78.} Met de ondergang van de Romeinse heerschappij werden
de boeren op het platteland grotendeels bevrijd van belastingen die
tussen een kwart en een derde van het bruto product van het land
opslokten, naast de vele afpersingen waarmee kleine en middelgrote
landeigenaren werden geconfronteerd.\footnote{Ibid., p.~118.} De
belastingdruk was zo ondragelijk -- soms zelfs met de dood als straf
afgedwongen -- dat eigenaren massaal hun land in de steek lieten. De
barbaren lieten deze lasten genadig vervallen.

\subsection{Agri Deserti}\label{agri-deserti}

Door de barbaarse veroveringen namen de staatslasten zodanig af dat er
voor de armen kansen ontstonden om grond te verwerven en te behouden.
Sommige van de agri deserti -- verlaten boerderijen die eigenaren hadden
achtergelaten om de roofzuchtige belastingen in de laatste jaren van het
Romeinse Rijk te ontvluchten -- kwamen weer in productie. Ondanks de
barre omstandigheden en het feit dat de oogstopbrengsten volgens moderne
maatstaven ver onder de maat waren, vormden de donkere middeleeuwen een
periode van relatieve welvaart voor de kleine boeren in Europa. Sterker
nog, hun positie bleek sterker dan die in de daaropvolgende moderne tijd
zou zijn. Enerzijds waren er minder handen beschikbaar om het vruchtbare
land te bewerken, doordat grote delen onbewerkt waren geraakt. Plagen,
oorlogen en het verlaten van land door eigenaars die het instortende
Romeinse Rijk ontvluchtten, leidden tot een aanzienlijke ontvolking van
voorheen bebouwde gebieden. Anderzijds bood de introductie van een
nieuwe landbouwtechniek in de zesde eeuw -- de zware ploeg, vaak
voorzien van wielen -- een groot voordeel. In combinatie met verbeterd
tuig, waarmee boeren meerdere ossen in de omgang konden nemen, maakte
deze innovatie het een stuk eenvoudiger om bosrijk terrein in
Noord-Europa te ontginnen.\footnote{Gies, op. cit., p.~45.}

Onder deze omstandigheden kromp de markt voor landbouwgrond bijna tot
het nulpunt. Boeren kregen nieuwe landbouwgrond door simpelweg
onontgonnen gebieden te bewerken en een deel van elk perceel af te staan
aan de plaatselijke autoriteiten. Dit proces, `assarting' genoemd, bood
eeuwenlang na de val van Rome een effectieve manier om de
bevolkingsgroei op te vangen. Vooral in dunbevolkte noordelijke streken
werd assarting aantrekkelijk, nadat de warmere temperaturen in de
achtste eeuw de landbouwproductie verhoogden.

De leiders van de Germaanse stammen die de voormalige Romeinse gebieden
binnenvielen, vestigden zich als grote landeigenaren. De meeste mensen
werkten op kleine percelen, maar dat gebeurde onder omstandigheden die
radicaal verschilden van die in het latere feodale tijdperk. Rijkere
landeigenaren, oftewel heren, vormden zo'n 7 tot 10 procent van de
bevolking. Uit gegevens blijkt dat vóór het jaar 1000 tweederde van de
dorpelingen in een typisch gebied in Frankrijk vrije landeigenaren
waren. \footnote{Bois, op. cit., p.~116.} Zij bezaten ongeveer de helft
van het in landbouw gebruikte land. \footnote{Ibid., p.~26.} Lijfeigenen
kwamen nauwelijks voor. Coloni, oftewel pachtkwekers, maakten niet meer
dan 5 procent van de bevolking uit. Slavernij bestond nog, maar op een
veel kleinere schaal dan in de Romeinse tijd.

De Germaanse koninkrijken, die als opvolgers van Rome fungeerden, werden
militair verdedigd door alle vrije mannen die in actie kwamen op de roep
van hun plaatselijke vertegenwoordiger, de graaf. Ook van de `kleine en
middelgrote eigenaren' verwachtte men dat zij zich verenigden en één van
hen uitzonden om in de infanterie te vechten. \footnote{Ibid., p.~64.}
In het Edict van Pitres beval Karel de Kale dat iedereen die hiertoe in
staat was, zich te verzamelen voor een strijd te paard. Een eeuw eerder
had paus Gregorius II in 732 geprobeerd dit militaire imperatief te
stimuleren door de consumptie van paardenvlees te verbieden. \footnote{Gies,
  op. cit., p.~47.} Er bestond nauwelijks onderscheid in status of wet
tussen de infanterie van de vrije landeigenaren en de cavalerie. Alle
vrije mannen namen deel aan lokale juridische vergaderingen en konden
bij de graaf een verzoek indienen voor geschillenbeslechting -- een ambt
dat al sinds de late Romeinse tijd bestond. Er was immers geen adel in
de traditionele zin.

\begin{quote}
`Een sociaal fenomeen, nieuw als massaal fenomeen, verscheen plotseling
in de jaren 980 aan de horizon: neerwaartse sociale mobiliteit. De
eerste slachtoffers waren de kleine allodhouders.' \footnote{Bois, op.
  cit., p.~52.} - Guy Bois
\end{quote}

Naarmate de donkere eeuwen voortduurden, deden zich echter meerdere
ontwikkelingen voor die de onafhankelijkheid van de boeren en vrije
landeigenaren in de Germaanse koninkrijken -- erfgenamen van de Romeinse
macht -- ondermijnden.

\begin{enumerate}
\def\labelenumi{\arabic{enumi}.}
\item
  Langzaam gingen de bevolkingen weer op, waardoor de vraag naar land
  toenam. In de loop van enkele eeuwen nam men het meest vruchtbare,
  voorheen onbenutte land -- vooral in Noord-Europa -- in gebruik.
  Doordat het aantal boeren steeg terwijl het beschikbare land gelijk
  bleef, verloor de arbeid van iedere boer wat van zijn waarde.
  Eigendommen werden via erfenissen telkens verder opgedeeld in steeds
  kleinere percelen, want in de donkere eeuwen verdeelden kinderen de
  nalatenschap van hun ouders over en weer. Deze versnippering van
  bezittingen, in combinatie met de bevolkingsgroei, deed het land weer
  in trek raken en leidde rond het midden van de tiende eeuw tot een
  heropleving van actieve landmarkten.
\item
  In de laatste decennia van de tiende eeuw zakten de temperaturen
  plotseling, wat een verwoestend effect had op de landbouwopbrengsten.
  Drie opeenvolgende mislukte oogsten veroorzaakten een ernstige
  hongersnood tussen 982 en 984. Na weer een mislukte oogst brak de
  hongersnood in 994 opnieuw uit.\footnote{Ibid., p.~150.} In 997
  verergerde een pest, die kleinschalige familiebedrijven hard trof --
  omdat zij niet de middelen hadden om de verloren arbeid van overleden
  familieleden te vervangen -- het probleem van de dalende
  oogstopbrengsten. Deze opeenstapeling van mislukte oogsten en
  tegenslagen bracht de boeren in de schulden, en zonder herstel van de
  opbrengsten konden zij hun hypotheken niet aflossen.
\item
  De machtsverhoudingen raakten geleidelijk verstoord door het
  toenemende belang van de zware cavalerie. Middeleeuwse historicus
  Frances Gies beschrijft hoe de bepantserde cavalerist transformeerde
  in de middeleeuwse ridder:
\end{enumerate}

\begin{quote}
Aanvankelijk was de ridder iemand van middelmatige status, die met zijn
kostbare paard en harnas boven de boer uitstak en zo langzaam zijn
positie in de samenleving verbeterde totdat hij tot de adel behoorde.
Hoewel ridders de laagste schakel binnen de hogere klasse bleven,
ontwikkelde het ridderschap een bijzondere allure, waardoor het
ridderslaan een eer werd die zelfs door de grote adel en het koningshuis
hoog in aanzien werd gehouden. Deze aantrekkingskracht kwam vooral door
het beleid van de Kerk, die het ridderambt christianiseerde door de
rituelen rond het ridderslaan te heiligen en een gedragscode, bekend als
ridderlijkheid, te sponsoren -- een code die wellicht vaker werd
overtreden dan nageleefd, maar onmiskenbare invloed had op het denken en
handelen van latere generaties.\footnote{Gies, op. cit., p.~2.}
\end{quote}

Zoals we in \emph{The Great Reckoning} beschreven, gaf de uitvinding van
de stijgbeugel de bewapende ruiter op paard een enorme
aanvalskapaciteit. Hij kon nu met hoge snelheid aanvallen zonder dat de
klap van zijn lans hem uit het zadel smeet. De militaire kracht van de
zware cavalerie nam verder toe door een Aziatische innovatie die in de
tiende eeuw West-Europa bereikte: het metijzeren hoefijzer, wat de
uithoudingskracht van het paard op ruwe wegen aanzienlijk
verbeterde.\footnote{Ibid., p.~46.} Daarnaast maakten het gevormde zadel
-- dat het hanteren van zware wapens vereenvoudigde --, de schop en het
kurkenbit -- waarmee de ruiter het paard met één hand kon besturen
tijdens een gevecht -- de bewapende ridder nog effectiever.\footnote{Ibid.,
  pp.~56--57.} Samen zorgden deze schijnbaar kleine technologische
innovaties ervoor dat de militaire waarde van kleine landeigenaren sterk
afnam, omdat zij zich zowel de oorlogspaarden als de bijbehorende
bewapening niet konden veroorloven. De voor oorlogsvoering speciaal
gefokte, goedkopere paarden -- de grote strijdpaarden, beter bekend als
destriers -- waren vier ossen of veertig schapen waard, terwijl de
duurdere oorlogspaarden tien ossen of honderd schapen kostten. Daarnaast
veristen harnassen een bedrag dat voor een kleine landeigenaar
onbetaalbaar was, namelijk de prijs van zestig schapen.\footnote{Ibid.,
  p.~58.}

\begin{enumerate}
\def\labelenumi{\arabic{enumi}.}
\setcounter{enumi}{3}
\tightlist
\item
  Het feit dat het weer kouder werd, oogsten mislukten en hongersnoden
  en plagen zich in de aanloop naar het jaar 1000 voordeden, beïnvloedde
  eveneens het gedrag van mensen. Velen waren ervan overtuigd dat het
  einde der tijden of de Tweede Komst nabij was. Zowel vrome als
  angstige landeigenaren, groot en klein, schonken hun land aan de Kerk
  in afwachting van de apocalyps.
\end{enumerate}

\subsection{`Alleen een arme man verkoopt
land'}\label{alleen-een-arme-man-verkoopt-land}

De onstabiele omstandigheden van de late tiende eeuw effenden de weg
voor de feodale revolutie. Aaneengesloten mislukte oogsten en rampen
zorgden ervoor dat kleinschalige boeren in de schulden raakten. Toen de
oogstopbrengsten niet herstelden, belandden vrije landeigenaren in een
uitzichtloze situatie. Markten leggen immers de zwaarste druk op de
minst vermogenden; dat is een intrinsiek principe, want zij verhogen de
efficiëntie door activa van de zwakkeren over te hevelen. In het Europa
van die tijd was zelfvoorzienende landbouw vrijwel de enige bron van
bestaan. Families die hun land verloren, raakten daarmee ook hun enige
middel tot levensonderhoud. Confronterend met dit ondragelijke
vooruitzicht gaven veel -- zo niet de meeste -- vrije landeigenaren
tijdens de feodale revolutie hun velden op. Zoals Guy Bois het
verwoordde: `De enige zekere manier voor een boer om het land dat hij
bebouwde te behouden, was door het eigendom ervan over te dragen aan de
Kerk, zodat hij het vruchtgebruik ervan kon blijven
genieten.'\footnote{Bois, op. cit., p.~87.} Anderen kenden (een deel
van) hun land toe aan welvarendere boeren in wie zij vertrouwen hadden
-- of het nu bevriende buren of familieleden betrof.

Deze eigendomsoverdrachten vonden plaats op voorwaarde dat de boer, zijn
familie en nakomelingen de akkers bleven bewerken. De arme boeren konden
daarnaast rekenen op de wederzijdse steun van rijkere bezitters, de
zogenaamde `nobelen' die zich paard en harnas konden veroorloven en
daardoor in staat waren hun grote landgoederen te beschermen. Voor de
nieuwe horigen leek zo'n regeling vaak een tussentijdse stap tussen
blijvend economisch eigendom en onteigening. In veel gevallen konden zij
deze overeenkomst niet weigeren.

De dalende productiviteit plaatste de arme boeren in een uitzichtloos
economisch dilemma en leidde tot een golf van roofzuchtig geweld die het
vertrouwen in eigendom ondermijnde. Degenen die de middelen misten om
een deel van de schaarse -- en vaak ontoereikende -- voorraden paarden
en vee te bemachtigen, ontdekten plots dat noch zij, noch hun
bezittingen veilig waren. Om het in hedendaagse termen te vatten: het
was alsof je je tegenwoordig met een gloednieuw type wapen moest
bewapenen, dat \$100.000 kostte. Als je dat bedrag niet kon betalen,
bleef je overgeleverd aan degenen die dat wel konden.

Binnen enkele jaren stortte de macht van de koning en zijn rechtbanken
in, waardoor zij niet langer in staat waren de orde te
handhaven.\footnote{Ibid. Het ontbreken van bronnen bemoeilijkt het
  vaststellen van de precieze volgorde van gebeurtenissen tijdens de
  feodale revolutie, maar de brede schets van de these, zoals Guy Bois
  die voorstelt, lijkt hoogstwaarschijnlijk juist. Deze schets is niet
  alleen op zichzelf aannemelijk; hij verklaart ook op logische wijze
  afwijkende feiten en past bovendien binnen onze theorieën.} Iedereen
met een harnas en een paard deed in wezen zelf dienst als wetgever. Het
resultaat leek wel op een late tiende-eeuwse variant van Blade Runner:
een chaotische uitbarsting van gevechten en plunderingen waartegen de
gevestigde autoriteiten machteloos stonden. Gewapende ridders
verstoorden met hun plunderingen en aanvallen het platteland. Het is
echter allesbehalve vanzelfsprekend dat uitsluitend de armen het
onderspit doden in deze roofacties. Integendeel, vooral ouderen,
lichamelijk zwakkeren en slecht voorbereiden binnen de rangen van de
grotere landeigenaren vormden aantrekkelijkere doelwitten, simpelweg
omdat zij meer bezaten.

Het viel geen toeval dat dit plaatsvond op een moment dat kouder weer,
hongersnood en pest de middelen onder druk zetten. De grootschalige
politieke ontwikkelingen die het gezag langzaam ondermijnden, waren al
lange tijd aan de gang, maar pas toen de crisis losbarstte, kwam hun
potentie om de machtsverhoudingen in de samenleving radicaal te
veranderen, duidelijk naar voren. Mislukte oogsten en hongersnoden
lijken precies dat te hebben veroorzaakt. Hoewel we het exacte verloop
van de gebeurtenissen nauwelijks kunnen reconstrueren, wijst alles erop
dat de plunderingen -- althans gedeeltelijk -- hun oorsprong vonden in
de wanhopige omstandigheden. Toen het geweld uitbarstte, bleek dat
niemand de kracht bezat om het tegen te houden. De overgrote meerderheid
van de slecht bewapende boeren kon er weinig aan uitrichten; zelfs
tientallen boeren te voet zouden geen partij opmaken tegen één gewapende
ridder te paard. Zowel de vrije landeigenaren als de gevestigde
autoriteiten -- de koningen met hun graven -- hadden simpelweg niet de
middelen om te voorkomen dat gewapende strijders lokaal land veroverden.

``The Peace of God''

In deze uitzichtloze tijden hielp de kerk de opkomst van het feodalisme
te bevorderen door te proberen op het door geweld geteisterde platteland
een wapenstilstand af te dwingen. De historicus Guy Bois verwoordde de
situatie als volgt: ``De machteloosheid van de politieke autoriteiten
was zodanig dat de kerk voor hen opdook in de poging de orde te
herstellen, binnen de beweging die bekendstaat als `The Peace of God.'
`Councils of Peace' vaardigden een reeks verboden uit die door
anathemata werden bekrachtigd; omvangrijke `assemblies of peace'
verzamelden de eden van de krijgers. De beweging ontstond in de Franse
Midi (Council of Charroux in 989, Council of Narbonne in 990) en
verspreidde zich geleidelijk\ldots{}'' \footnote{Ibid., blz. 136.}

De overeenkomst die de kerk tot stand bracht, hield in dat men het gezag
van gewapende ridders in lokale gemeenschappen erkende, in ruil voor een
stilstand of zwakkere uitbundigheid van het geweld en de plunderingen.
Na de uitbraak van het geweld aan het einde van de tiende eeuw labelde
men grondbewijzen ineens met de aanduidingen ``nobilis'' of ``miles''
als teken van herenbezit. De feodale revolutie creëerde daarmee de adel
als een aparte stand, terwijl eigendomsoverdrachten enkele jaren eerder
aan dezelfde personen nog zonder dergelijk onderscheid werden
geregistreerd.\footnote{Ibid., blz. 57 en passim.}

Door de dalende productiviteit en de economische onzekerheid onder de
kleine landeigenaren zorgde de megapolitieke macht van de gewapende
ridders er onvermijdelijk voor dat grondbezit op basis van het feodale
leenstelsel ontstond. Tegen het einde van het eerste kwart van de elfde
eeuw waren de vrije boeren vrijwel verdwenen. Hun eigen grondbezit kromp
terug tot een fractie van hun vroegere omvang en werd slechts deels
bewerkt. De kleine boeren of hun afstammelingen werden horigen die het
grootste deel van hun arbeid verrichtten op de landgoederen van zowel
seculiere als kerkelijke feodale heren.

De wanorde tijdens de feodale revolutie leidde tot gedragsaanpassingen
die het feodalisme verder verstevigden. Zo nam de bouw van kastelen
sterk toe. Eerst verschenen kastelen in Noordwest-Europa als eenvoudige
houten constructies, opgebouwd in de nasleep van de Vikingaanvallen in
de negende eeuw. Aanvankelijk dienden zij als commandocentra voor
Karolingische functionarissen; later, na de omwenteling in het feodale
systeem, werden ze erfelijk bezit. Hoewel deze vroege bolwerken veel
primitiever waren dan hun latere opvolgers, bleven zij desalniettemin
moeilijk te overwinnen. Eenmaal opgericht, liet men ze met de grootste
moeite vernietigen. Naarmate ze het platteland begonnen te sieren, werd
het steeds onwaarschijnlijker dat de koning of zijn graven de lokale
overheersing van de heren effectief konden uitdagen.

\subsection{Bijdragen van de kerk aan de
productiviteit}\label{bijdragen-van-de-kerk-aan-de-productiviteit}

Het feodalisme ontstond als reactie van de agrarische samenleving op het
wegvallen van de orde in een tijdperk van lage productiviteit. In de
beginfase van het feodale systeem speelde de kerk een belangrijke en
economisch productieve rol. Tot haar bijdragen telde onder meer:

\begin{enumerate}
\def\labelenumi{\arabic{enumi}.}
\tightlist
\item
  In een omgeving waarin de militaire macht sterk gedecentraliseerd was,
  nam de kerk een unieke positie in. Zij hield de vrede in stand en
  stelde ordebepalende regels op die verder reikten dan de
  gefragmenteerde, lokale soevereiniteiten -- een taak die geen enkele
  wereldlijke macht op zich kon nemen. De inzichten van de prominente
  religieuze autoriteit A. R. Radcliffe-Brown zijn hierbij direct van
  belang; hij merkte op dat `de sociale functie van een religie
  onafhankelijk is van haar waarheid of onwaarheid.' Zelfs religies die
  absurd en afstotelijk lijken -- zoals die van sommige wilde stammen --
  kunnen belangrijke en effectieve onderdelen vormen van de
  maatschappelijke machine.\footnote{A. R. Radcliffe-Brown, `Religie en
    samenleving', in \emph{Structuur en functie in de primitieve
    samenleving} (London: \emph{Cohen \& West}, 1952), blz. 153--177.}
  Dit gold zeker voor de kerk in de vroege fase van het feodalisme. Zij
  stelde, als enige instantie die dat kon, regels op waarmee mensen
  prikkelvalstrikken en gedragsdilemma's konden overwinnen. Sommige van
  deze dilemma's betroffen morele vraagstukken die inherent verbonden
  zijn met het menselijke bestaan, terwijl andere specifiek lokaal
  waren, als gevolg van de heersende megapolitieke omstandigheden. De
  middeleeuwse kerk speelde een bijzondere rol in het herstellen van de
  orde op het platteland in de laatste jaren van de tiende eeuw. Door
  lokale autoriteiten zowel religieuze als ceremoniële steun te bieden,
  verlaagde zij de kosten voor het vestigen van, al is het maar een
  zwak, lokaal geweldmonopolie. Daarmee leverde de kerk een waardevolle
  bijdrage aan de voorwaarden die uiteindelijk tot stabielere
  machtsverhoudingen leidden.
\end{enumerate}

De kerk bleef lange tijd een cruciale rol spelen in het beteugelen van
particuliere oorlogen en excessief geweld, dat de burgerlijke
autoriteiten anders niet in de hand kregen. Het aanzien van de kerk ten
opzichte van wereldlijke machthebbers blijkt er in de elfde eeuw
duidelijk uit dat in het grootste deel van West‑Europa parochies de
voornaamste bestuurlijke indelingen vormden, in plaats van de oude
civiele structuren -- de ager en pagus (stad) -- die sinds de Romeinse
tijd tot in de donkere middeleeuwen in gebruik waren.

De kerk fungeerde als de voornaamste bewaker en doorgever van technische
kennis en informatie. Zij steunde universiteiten en verzorgde het
basisonderwijs waarover de middeleeuwse samenleving over beschikte.
Bovendien bood zij een methode om boeken en manuscripten te kopiëren,
waaronder vrijwel alle kennis over landbouw en veeteelt. De scriptoria
van de benedictijner kloosters vormden een alternatief voor drukpersen,
die toen nog niet bestonden. Hoe kostbaar en arbeidsintensief deze
scriptoria ook waren, zij bleek in de feodale periode het enige middel
om schriftelijke kennis vast te leggen en te bewaren.

De kerk droeg aanzienlijk bij aan een hogere productiviteit in de
Europese landbouw, mede doordat haar landgoedbeheerders geletterd waren
-- iets wat vooral in de vroege fase van het feodalisme opviel. Voor de
dertiende eeuw hielden de beheerders van wereldlijke landgoederen hun
administratie vrijwel uitsluitend bij met een ingewikkeld systeem van
tekens, omdat zij over het algemeen analfabetisch waren. Hoe kundig deze
boeren ook waren, zij konden niet profiteren van verbeterde
productiemethoden die zij niet zelf bedachten of aanschouwden. Daarom
speelde de kerk een onmisbare rol bij het verbeteren van de kwaliteit
van granen, fruit en fokdieren. Dankzij haar omvangrijke bezittingen,
die zich over heel Europa uitstrekten, kon zij het meest vruchtbare zaad
en de beste fokdieren naar regio's sturen waar de opbrengst achterbleef.
De vraag naar sacramentele wijn in Noord‑Europa stimuleerde monniken om
te experimenteren met robuustere druivensoorten die in koelere klimaten
konden overleven. Daarbij herschikte zij vele economisch ineffectieve,
kleine percelen die tijdens de feodale revolutie aan haar waren
geschonken, zodat ze makkelijker te bewerken waren. Bovendien bood de
kerk aanvullende diensten aan die kleine landbouwgemeenschappen nodig
hadden. In vele regio's maalden de door de kerk beheerde molens graan
tot meel.

De kerk vervulde talloze taken die tegenwoordig door de overheid worden
uitgevoerd, zoals het aanleggen van openbare infrastructuur. Op deze
wijze hielp zij de door economen `dilemma's rond publieke goederen' te
overwinnen in een tijdperk met gefragmenteerd gezag. Bepaalde religieuze
orden in de vroegmiddeleeuwse kerk wijdde zich aan praktische technische
werkzaamheden, zoals het aanleggen van wegen, het herbouwen van
ingestorte bruggen en het herstellen van vervallen Romeinse aquaducten.
Daarnaast ruimden zij land op, bouwden zij dammen en droogden zij
moerassen. De Kartusiaanse orde groef in Artois, Frankrijk, met
percussieboringen de eerste `artesische' put -- een gat dat diep genoeg
bleek te zijn om zonder pomp te functioneren.\footnote{Gies, op. cit.,
  blz. 112.} De cisterciënzerorde nam in de Lage Landen de taak op zich
om stabiele zeewallen en dijken te bouwen en te onderhouden. Lokale
boeren schonken land aan cisterciënzer kloosters en huurden het
vervolgens weer terug, terwijl de monniken de volledige
verantwoordelijkheid namen voor onderhoud en reparaties. Ook gingen de
cisterciënzers voorop bij de ontwikkeling van waterkrachtmachines, die
zij inzetten voor uiteenlopende werkzaamheden als stampen, hijsen, malen
en persen.\footnote{Ibid., blz. 114.} Het klooster van Clairvaux groef
een kanaal van twee mijl langs de rivier de Aube.\footnote{Ibid.,
  p.~117.} Verder zette de kerk zich in om nieuwe wegen en bruggen aan
te leggen op locaties waar bevolkingscentra zich buiten het bereik van
de oude Romeinse garnizoenswegen hadden verplaatst. Bisschoppen
verleenden aflaten aan lokale heren die rivierovergangen bouwden of
herstelden en hospices voor reizigers in stand hielden. Een monastieke
orde, opgericht door Sint Benezet -- de Frères Pontifes of `Broeders van
de Brug' -- bouwde enkele van de langste bruggen die er toen bestonden,
waaronder de Pont d'Avignon, een imposante brug met twintig bogen over
de Rhône waaraan een kapel met tolpost was gekoppeld. Zelfs London
Bridge, die tot in de negentiende eeuw overeind bleef, werd met
initiatief van een kapelaan gebouwd en deels gefinancierd met een
bijdrage van 1.000 mark uit het pauselijke legaat.\footnote{De details
  omtrent bruggen en infrastructuur zijn voornamelijk afkomstig uit
  Ibid., pp.~148--154.}

De kerk speelde tevens een rol in het ontstaan van een complexere markt.
De bouw van kathedralen verschilde immers wezenlijk van de aanleg van
openbare infrastructuur, zoals bruggen en aquaducten. Kerkgebouwen
dienden in principe uitsluitend voor religieuze diensten en niet als
handelsroutes. Toch droeg hun bouw bij aan de ontwikkeling en verbreding
van markten voor tal van ambachtelijke en technische vaardigheden. Net
zoals de militaire uitgaven van de natiestaten tijdens de Koude Oorlog
onbedoeld de ontwikkeling van het internet stimuleerden, leidde de bouw
van middeleeuwse kathedralen tot positieve neveneffecten, met name het
stimuleren van de handel. De kerk behoorde tot de grootste afnemers in
de bouwsector en bij ambachtslieden. Haar aankopen van zilver voor
communiediensten, kandelaars en kunstwerken ter decoratie van kerken
hielpen een markt voor luxe goederen tot stand te brengen die anders
niet had bestaan.

In veel opzichten temperde de kerk het meedogenloze geweld dat door
bewapende ridders werd ontketend tijdens en na de `feodale revolutie'.
Vooral in de eerste eeuwen van het feodalisme droeg de kerk in hoge mate
bij aan een toename van de landbouwproductiviteit. Ze vormde een
onmisbare instelling die naadloos aansloot op de behoeften van de
agrarische samenleving aan het einde van de duistere middeleeuwen.

\subsection{Kwetsbaarheid voor geweld}\label{kwetsbaarheid-voor-geweld}

In `dertig of veertig jaar gewelddadige onrust, de feodale revolutie van
het jaar 1000,' \footnote{Bois, op. cit., p.~136.} net als de val van
Rome vijf eeuwen daarvoor, ontstond een uniek fenomeen door een complexe
wisselwerking van invloeden. Tegelijkertijd illustreren de triomf van
`malihamines' (slechte mannen) en de onderdrukkingen die zij
teweegbrachten op treffende wijze de fundamentele kwetsbaarheid van de
agrarische samenleving voor geweld. In tegenstelling tot de periode van
jagen en verzamelen leidde de landbouw tot een sprong in georganiseerd
geweld en onderdrukking.

Vanaf het allereerste begin kwam dit tot uiting in de meer
militaristische culturen van landbouwvolkeren. De goden van de vroege
agrarische samenlevingen waren goden van regenval en overstromingen,
wier taken de zorgen over de factoren die de oogstopbrengst bepaalden,
weerspiegelden. De god die de regen of het water zond, werd vaak tevens
als oorlogsgod aangeroepen door de eerste koningen, die in wezen
krijgsheren waren.\footnote{Zie Norman Cohn, \emph{Cosmos, Chaos, and
  the World to Come: The Ancient Roots of the Apocalyptic Faith} (New
  Haven: Yale University Press, 1993), hoofdstukken I--3, met name
  p.~60.}

De hechte band tussen landbouw en oorlogsvoering komt tevens tot uiting
in de religieuze verbeelding van mensen wier leven radicaal veranderde
door de innovaties van de agrarische revolutie. De verdrijving uit de
Hof van Eden dient als een symbolische weergave van deze
maatschappelijke transformatie: van een bestaan waarin men met weinig
inspanning voedsel plukte uit de overvloed van de natuur, naar een
bestaan dat werd gekenmerkt door zware arbeid.

\section{Paradijs verloren}\label{paradijs-verloren}

De landbouw bracht de mensheid op een totaal nieuw pad. De eerste boeren
plantten daadwerkelijk de zaden van de beschaving. Door hun arbeid
ontstonden steden, legers, rekenkunde, astronomie, kerkers, wijn en
whiskey, het geschreven woord, koningen, slavernij en oorlog. Toch
beschouwden velen de overgang van de oer-economie vanaf het allereerste
begin als onwenselijk, ondanks al het drama dat de landbouw met zich
meebracht. Lees het verslag dat in het boek Genesis is bewaard gebleven,
waarin het verhaal van de verdrijving uit het paradijs wordt verteld. De
bijbelse gelijkenis van de Hof van Eden roept de warme herinnering op
aan de zorgeloze levensstijl die men in de wildernis kende.
Wetenschappers vermoeden dat het woord `Eden' is afgeleid van een
Sumerisch woord voor `wildernis.'\footnote{Bruce Ni. Metzger en Michael
  D. Coogan (red.), \emph{The Oxford Companion to the Bible} (Oxford:
  Oxford University Press, 1993), p.~178.}

De omschakeling van een vrij en dunbevolkt bestaan in de natuur naar een
sedentair leven in een landbouwdorp bracht diepe spijt teweeg -- zo
blijkt niet alleen uit de Bijbel, maar ook uit de aanhoudende ergernis
over het `s ochtends opstaan en naar het werk gaan. Zoals Stephen Boyden
schreef in \emph{Western civilization in biological perspective}, noemde
hij de nieuwe levenswijze, die de landbouw met zich meebracht,
'evodeviant.'\footnote{Boyden, op. cit., p.~118.} Vóór de komst van de
landbouw leefden duizenden generaties mensen, zoals Adam in Eden, op
uitnodiging van zijn Schepper: `Van alle bomen in de tuin magt gij vrij
eten.' Jagers en verzamelaars hoefden geen gewassen te onderhouden,
hadden geen kudde om te bewaken en betaalden geen belastingen. Net als
rondtrekkers zwierven zij waarheen zij wilden, werkten weinig en legden
aan niemand verantwoording af.

Met de komst van de landbouw begon een geheel nieuwe levenswijze, maar
wel onder veel ingrijpendere voorwaarden. `Thorns also and thistles
shall it bring forth to thee; and thou shalt eat the herb of the field;
In the sweat of thy face shalt thou eat bread.' Landbouw vergt hard
werk. De herinnering aan het bestaan vóór de landbouw deed men denken
aan een verloren paradijs.

Meer dan men zich had kunnen voorstellen, creëerden boeren nieuwe
omstandigheden die de logica van geweld radicaal wijzigden. Het is geen
toeval dat Genesis Kaïn, de eerste moordenaar, afbeeldt als `a tiller of
the ground.' Inderdaad, de Bijbel, die haar verhaal toevertrouwde aan
herders die als geen ander inzagen hoe landbouw geweld versterkte,
draagt daarmee een onheilspellende, profetische kracht. Enkele verzen
vatten een logica samen die zich over duizenden jaren heeft ontwikkeld.
Landbouw fungeerde als broedplaats voor conflicten. Zij creëerde op
grote schaal stationair kapitaal, waardoor de opbrengst van geweld
toenam en het beveiligen van eigendommen een veel grotere uitdaging
werd. Voor het eerst konden zowel misdaad als overheidsingrijpen
lucratieve ondernemingen worden.

\bookmarksetup{startatroot}

\chapter{de laatste dagen van de
politiek}\label{de-laatste-dagen-van-de-politiek}

\begin{quote}
`Ik geloof -- en hoop -- dat politiek en economie in de toekomst minder
centraal zullen staan dan vroeger. Er komt een tijd waarop de meeste van
onze huidige debatten over deze onderwerpen volstrekt banaal of
betekenisloos zullen lijken, net als de theologische discussies waaraan
de scherpste geesten van de middeleeuwen hun energie
verspilden.'\footnote{Clarke, op. cit., p.~9.} - ARTHUR C. CLARKE
\end{quote}

Over de vermeende naderende ondergang van de politiek praten sommige
mensen op manieren die, afhankelijk van je zienswijze, belachelijk of
juist optimistisch lijken. Toch verwacht de informatierevolutie dit te
brengen. Wie is opgegroeid in een tijdperk waarin politiek
alomtegenwoordig is, vindt het idee dat men zonder politiek kan leven
bijna fantasierijk -- vergelijkbaar met beweren dat we ons uitsluitend
via de in de lucht aanwezige voedingsstoffen zouden kunnen voeden. Maar
politiek in de moderne zin -- het besturen en rationaliseren van
staatsmacht -- is in feite een zuiver moderne uitvinding. Wij zijn ervan
overtuigd dat dit systeem zal verdwijnen met het einde van de moderne
wereld, net zoals de wirwar van feodale plichten en verplichtingen, die
in de middeleeuwen ieders aandacht opslokte, ten einde kwam toen die
periode voorbij was. Zoals historicus Martin van Creveld opmerkt:
`bestond politiek niet (het concept moest nog uitgevonden worden en
dateert pas uit de zestiende eeuw).' \footnote{Martin van Creveld,
  \emph{The Transformation of War} (New York: \emph{The Free Press},
  1991), p.~52.}

Het idee dat de politiek, zoals wij die tegenwoordig kennen, voor de
moderne tijd niet bestond, kan verrassend overkomen -- zeker als je
bedenkt dat Aristoteles in de tijd van Alexander de Grote al een essay
met die titel schreef. Laat je echter niet misleiden: woorden die in
oude teksten werden gebruikt, vertegenwoordigen niet automatisch moderne
begrippen. Zo schreef Aristoteles ook een werk getiteld Sophistical
Refutations, een term die tegenwoordig ongeveer net zo betekenisloos is
als de invulling van `politics' in de middeleeuwen. Destijds was die
term simpelweg niet gangbaar; de eerste gedocumenteerde vermelding in
het Engels dateert pas uit 1529.\footnote{\emph{The Compact Edition of
  The Oxford English Dictionary}, op. cit., p.~1074.} Zelfs toen kende
`politics' al een pejoratieve lading, afgeleid van het Oudfranse
politique, dat men gebruikte om `opportunisten en tempelaars' mee aan te
duiden.\footnote{Zie T.C. Onions (red.), \emph{The Oxford Dictionary of
  English Etymology} (Oxford: \emph{Oxford University Press}, 1966),
  p.~693.}

Het duurde bijna tweeduizend jaar voordat Aristoteles' latente concept
de betekenis kreeg die we nu kennen. Waarom? De moderne wereld moest
eerst zorgen voor grootschalige politieke omstandigheden die de
opbrengsten van geweld drastisch deden stijgen. De buskruitrevolutie --
zoals we in \emph{The Great Reckoning} hebben beschreven -- deed precies
dat: het bracht de opbrengst van geweld op een ongekend niveau. Hierdoor
won de vraag wie de staat bestuurde aan urgentie. Zo ontstond de moderne
politiek als een logische en onvermijdelijke uitkomst van de strijd om
de sterk toegenomen buit van de macht te beheersen.

De moderne politiek kreeg vijf eeuwen geleden vorm in de vroege
industriële fase, maar lijkt nu zijn vaarwel te zeggen. Over de hele
wereld groeit de afkeer van zowel politiek als politici. Je ziet het in
berichten en roddels over vermeende geheimhoudingen rondom Whitewater en
in de slecht vermomde moord op Vincent Foster. Ook in tal van andere
schandalen waarin president Bill Clinton verstrikt raakte, komt dit tot
uiting. Bovendien hoor je verhalen over verduistering door invloedrijke
congresleden bij het postkantoor van het Huis van Afgevaardigden.
Vergelijkbare schandalen dwalen rond in de kring van John Major en in
Frankrijk, waar twee recente premiers -- Eduard Balladur en Alain Juppe
-- betrokken waren. Nog grotere onthullingen deden de ronde in Italië,
waar de zevenvoudige premier Giuho Andreotti voor de rechtbank moest
verschijnen op aanklachten als banden met de maffia en het bevel tot
moord op onderzoeksjournalist Mino Pecorelli. Ook in Spanje is de
reputatie van premier Filipe Gonzales aangetast door schandalen. In de
eerste helft van de jaren negentig verloren vier Japanse premiers hun
positie door corruptieaanklachten. Het Canadese ministerie van Justitie
stuurde zelfs een brief aan Zwitserse autoriteiten waarin werd gesteld
dat voormalig premier Brian Mulroney steekpenningen had ontvangen bij de
verkoop van Airbus-vliegtuigen aan Air Canada, ter waarde van C\$l.8
miljard.\footnote{John Urquhart, `Voormalig premier klaagt Canada aan
  wegens laster in onderzoek naar vermeende Airbus-stekpenningen,'
  \emph{Wall Street Journal}, 21 november 1995, p.~A11.} Willy Claes,
secretaris-generaal van de \emph{NAVO}, moest afzien van zijn functie
onder een wolk van corruptieaanklachten. Zelfs in Zweden werd
vicepremier Mona Sahlm gedwongen af te treden nadat bleek dat zij
overheidscreditcards gebruikte om luiers en andere huishoudelijke
artikelen te kopen. Bijna overal -- in landen met eens zo goed bestede
welvaartsstaten -- groeit de afkeer van de politieke leiders.

\subsection{Minachting als leidende
indicator}\label{minachting-als-leidende-indicator}

Morele verontwaardiging over corrupte leiders vormt geen op zichzelf
staand historisch fenomeen, maar gaat vaak vooraf aan ingrijpende
veranderingen. Telkens weer zien we dat het ene tijdperk plaatsmaakt
voor het andere. Als technologische vernieuwingen de oude structuren
scheiden van de nieuwe economische dynamiek, verschuiven de morele
normen en behandelen mensen degenen die de oude instituties besturen met
toenemende minachting. Deze alomtegenwoordige afkeer wordt al zichtbaar
nog voordat men een samenhangende ideologie van verandering heeft
ontwikkeld. Op dit moment vinden we nog weinig bewijs van een duidelijk
gearticuleerde afwijzing van de politiek -- dat zal later komen. De
meesten van je tijdgenoten hebben nog niet ingezien dat een leven zonder
politiek mogelijk is. Wat we in de laatste jaren van de twintigste eeuw
waarnemen, is een onuitgesproken minachting.

Aan het einde van de vijftiende eeuw speelde zich iets soortgelijks af,
maar toen lag de focus niet op de politiek, maar op de religie. Ondanks
de overtuiging over `de heiligheid van het sacerdotale ambt' behandelden
mensen zowel de hoge als de lage geestelijkheid met grote minachting,
vergelijkbaar met hoe we vandaag de dag tegen politici en bureaucraten
aankijken. Men meende algemeen dat de hoge geestelijkheid corrupt,
werelds en venaal was -- en dat was niet zonder reden. Verschillende
pausen uit die periode brachten openlijk buitenechtelijke kinderen
voort. De lage geestelijkheid stond nog lager aangeschreven, omdat zij
in overvloed aanwezig was in zowel het platteland als in de steden. Ze
bedelden aalmoezen en boden vaak Gods genade en vergeving van zonden te
koop aan voor contant geld.

Onder de `korst van oppervlakkige vroomheid'\footnote{Ibid., p.~150.}
schuilde een corrupt en steeds disfunctioneler wordend systeem. Veel
mensen verloren al lang hun respect voor degenen die het bestuurden, nog
voordat iemand de moed had op te merken dat het systeem niet meer
functioneerde. Een leven dat volledig doordrenkt was van religie --
waarin men geen onderscheid maakte tussen het geestelijke en het
wereldse -- had al zijn mogelijkheden opgebruikt. Het einde daarvan was
dan ook onvermijdelijk, ruim lang voordat Luther zijn 95 stellingen op
de kerkdeur van Wittenberg spijkerde.

\section{Een seculiere reformatie}\label{een-seculiere-reformatie}

Wij zijn ervan overtuigd dat de reactie op de verzadigingspolitiek een
vergelijkbaar traject volgt.

De val van de Sovjet-Unie en de afwijzing van het socialisme maken deel
uit van een alomvattend patroon van depolitisering dat de wereld
teistert. Dit blijkt vooral uit de groeiende minachting voor de
regeringsleiders wereldwijd. Men beseft steeds vaker dat zij corrupt
handelen en aflaat verkopen om politieke problemen te sussen -- in ruil
voor campagnebijdragen of bijzondere hulp bij de handel in grondstoffen
om hun persoonlijke financiën te ondersteunen.

De toenemende afkeer van politici ontstaat daarnaast doordat velen
inzien dat hun inspanningen, ondanks enorme kosten, vaak zinloos
blijken. Het is vergelijkbaar met het organiseren van een pelgrimstocht
waarbij boetelingen blootsvoets door de sneeuw marcheren, of met het
opnieuw oprichten van een orde van bedelaarsmonniken in de late
vijftiende eeuw. Beide praktijken dragen nauwelijks bij aan een hogere
productiviteit of het verlichten van de druk op de levensstandaard.

\subsection{De laatste dagen van de heilige
moederkerk}\label{de-laatste-dagen-van-de-heilige-moederkerk}

Aan het einde van de middeleeuwen ontwikkelde de monolithische kerk zich
tot een verouderde en contraproductieve instelling -- een duidelijke
ommekeer ten opzichte van haar positieve economische bijdrage vijf
eeuwen eerder. Zoals we in het vorige hoofdstuk al benoemden, speelde de
kerk tegen het einde van de tiende eeuw een cruciale rol in het
herstellen van de orde en het stimuleren van economisch herstel na de
anarchie waarmee het einde van de donkere eeuwen werd gekenmerkt. In die
tijd was de kerk onmisbaar voor het voortbestaan van talloze kleine
vrije landeigenaren en lijfeigenen, die samen de meerderheid van de
West-Europese bevolking vormden. Tegen het einde van de vijftiende eeuw
belemmerde de kerk de productiviteit, en de lasten die zij oplegde,
zogen een daling van de levensstandaard met zich mee.

Hetzelfde geldt voor de moderne natiestaat. Zij paste zich noodzakelijk
aan op de nieuwe megapolitieke omstandigheden die vijf eeuwen geleden
door de buskruitrevolutie werden gecreëerd. De natiestaat vergrootte het
marktgebied en verdrong de gefragmenteerde lokale autoriteiten op een
moment dat grotere handelsgebieden aanzienlijke opbrengsten opleverden.
Het feit dat handelslieden zich vrijwel overal in Europa spontaan met de
centrale vorst verbonden terwijl hij zijn gezag consolideerde, levert
duidelijk bewijs dat de vroege natiestaat gunstig was voor het
bedrijfsleven. Bovendien verlichtte zij de handelslasten die feodale
landeigenaren en lokale magnaten eerder hadden opgelegd.

In een wereld waarin de opbrengsten van geweld zowel hoog als groeiend
waren, bleek de natiestaat een waardevolle instelling te zijn. Vijf
eeuwen later -- nu dit millennium ten einde loopt -- zijn de
geopolitieke omstandigheden drastisch veranderd. De opbrengsten van
geweld nemen af en de natiestaat fungeert, net als de middeleeuwse kerk
in haar schemering, als een anachronisme dat groei en productiviteit
belemmert.

Net zoals de middeleeuwse kerk destijds, heeft de hedendaagse natiestaat
haar mogelijkheden uitgeput. Ze verkeert in faillissement en is
uitgegroeid tot een seniel systeem. Vijf eeuwen lang domineerde ze als
de overheersende vorm van sociale organisatie, maar doordat ze de
omstandigheden waarin zij ontstond heeft overleefd, is ze nu rijp voor
ondergang -- en dat zal onvermijdelijk gebeuren. Technologie ontketent
een revolutie in de uitoefening van macht die de natiestaat net zo zeker
zal vernietigen als buskruitwapens en de drukpers ooit het middeleeuwse
kerkmonopolie deden instorten.

Als onze redenering klopt, vervangt een nieuwe vorm van soevereiniteit
de natiestaat. Sommige vormen zullen uniek blijken in de geschiedenis,
terwijl andere doen denken aan de stadstaten en middeleeuwse
koopmansrepublieken uit de premoderne wereld. Wat eens als oud werd
beschouwd, blijkt na het jaar 2000 weer vernieuwd te zijn, en wat
voorheen onvoorstelbaar was, wordt alledaags. Naarmate technologische
ontwikkelingen versnellen, ontdekken regeringen dat zij als bedrijven
moeten concurreren om inkomsten -- door hun diensten niet boven hun
werkelijke waarde te prijzen. De gevolgen van deze verandering zijn
bijna onvoorstelbaar.

\section{Toen en nu}\label{toen-en-nu}

Dit had men vijfhonderd jaar geleden kunnen zeggen, aan het begin van de
vijftiende eeuw. Toen, net als nu, stond de westerse beschaving aan de
vooravond van een ingrijpende transformatie. Bijna niemand doorzag
destijds dat de middeleeuwse samenleving op het punt stond te vervagen;
haar ondergang werd noch breed verwacht, noch begrepen. Desondanks hing
er een sfeer van diepe somberheid. Dit komt vaak voor aan het einde van
een tijdperk, wanneer conventionele denkers het gevoel krijgen dat de
dingen uit elkaar vallen, dat `de valk het geluid van de valker niet
meer hoort.' Toch blijkt hun mentale traagheid vaak zo groot dat zij de
implicaties van de opkomende machtsconfiguraties niet kunnen
doorgronden. De middeleeuwse historicus Johan Huizinga schreef over de
laatste dagen van de middeleeuwen: `De kroniekschrijvers van de
vijftiende eeuw zijn, bijna allen, ten prooi gevallen aan een volstrekte
misvatting van hun tijden, waarin de werkelijke drijvende krachten aan
hun aandacht ontsnapten.'\footnote{Ibid., p.~56.}

\subsection{Mythen verraden}\label{mythen-verraden}

Grote veranderingen in de diepgewortelde machtsstructuren brengen
conventionele denkers vaak in de war, omdat ze mythen blootleggen die de
oude orde verantwoorden, maar weinig verklarende kracht bezitten. Aan
het einde van de middeleeuwen lag -- net als nu -- een enorme kloof
tussen de overgeleverde mythen en de werkelijkheid. Huizinga zei over de
Europeanen in de late vijftiende eeuw: `Hun gehele ideeënstelsel was
doordrenkt met de fictie dat de ridderlijkheid de wereld
regeerde.'\footnote{Ibid., p.~65.} Tegenwoordig heerst een vergelijkbare
opvatting, namelijk dat alles wordt geregeerd door stemmen en
populariteitswedstrijden. Beide stellingen doorstaan nadere beschouwing
niet. Het idee dat de loop der geschiedenis voortkomt uit een
democratische optelsom van wensen is net zo absurd als de middeleeuwse
notie dat zij door een uitgebreide gedragscode -- de ridderlijkheid --
wordt bepaald.

Dat zo'n uitspraak bijna als ketterij wordt bestempeld, toont hoe ver
het conventionele denken verwijderd is van een realistisch begrip van de
machtdynamiek in de laat-industriële samenleving. We onderzoeken dit
vraagstuk uitvoerig in dit boek. Naar ons idee volgt stemmen eerder als
gevolg dan als oorzaak van de megapolitieke ontwikkelingen die de
moderne natiestaat mogelijk maakten. Massademocratie en het concept van
burgerschap bloeiden op met de groei van de natiestaat en zullen
instorten zodra deze faalt, waardoor in Washington evenveel onvrede zal
ontstaan als de ineenstorting van de ridderlijkheid ooit in het hof van
de hertog van Bourgondië, vijfhonderd jaar geleden.

\section{Parallellen tussen ridderlijkheid en
burgerschap}\label{parallellen-tussen-ridderlijkheid-en-burgerschap}

Als u doorgrondt hoe en waarom het belang van ridderlijke eden afnam
tijdens de overgang naar een industriële samenleving, herkent u ook
beter hoe het burgerschap -- zoals wij dat nu kennen -- mogelijk
verdwijnt in het informatietijdperk. Beide vervulden immers een
vergelijkbare functie: zij maakten de uitoefening van macht mogelijk
onder twee totaal verschillende megapolitieke omstandigheden.

Feodale eden bepaalden de orde in een tijd waarin defensieve technologie
cruciaal was, soevereiniteiten versnipperd waren en particuliere
individuen en bedrijven op eigen kracht militaire macht uitoefenden.
Voordat buskruit gangbaar werd, vochten oorlogen doorgaans met kleine
contingenten gewapende mannen. Zelfs de machtigste vorsten beschikten
niet over een staand leger, het zogenaamde `militum perpetuum'. Ze
putten hun militaire kracht uit een netwerk van vazallen: de grote heren
vertrouwden op de steun van minder invloedrijke edelen, die op hun beurt
op hun ridders konden rekenen. De loyaliteitsketen liep zo door de hele
hiërarchie, tot aan de laagste sociale laag die men als waardig
beschouwde om wapens te dragen.

\subsection{Uniformen of afwijkingen?}\label{uniformen-of-afwijkingen}

In tegenstelling tot moderne legers trok een middeleeuws leger -- in de
periode vóór de opkomst van het burgerschap -- het slagveld op zonder
uniforme kledij. Elke dienaar of vazal -- of hij nu een ridder, baronet
of heer betrof -- droeg zijn eigen karakteristieke dienstkledij waarmee
hij zijn positie in de hiërarchie kenbaar maakte. In plaats van
uniformiteit benadrukte de diversiteit in dienstkledij de verticale
opbouw van de samenleving, waarbij iedere positie uniek was. Zoals
Huizinga opmerkte, werden middeleeuwse krijgers onderscheiden door
`uitwendige tekenen van \ldots{} verschillen: dienstkledij, kleuren,
emblemen, strijdkreten.'\footnote{Ibid., p.~22.}

Bovendien voerden niet uitsluitend overheden of naties oorlog. Martin
van Creveld wijst erop dat moderne opvattingen over oorlog -- zoals deze
door strategen als \emph{Carl von Clausewitz} worden neergezet -- een
vertekend beeld geven van premoderne conflicten. Van Creveld schrijft:

\begin{quote}
Duizend jaar na de val van Rome werd gewapend conflict gevoerd door
verschillende soorten sociale entiteiten. Onder hen bevonden zich
barbaarse stammen, de Kerk, feodale baronnen van alle rangen, vrije
steden en zelfs particuliere individuen. En de ``legers'' van die
periode waren totaal niet te vergelijken met de legers die we
tegenwoordig kennen; het is immers moeilijk een woord te vinden dat
recht doet aan hun aard. Oorlog werd gevoerd door zwermen dienaren die
in militaire kledij verschenen en hun heer volgden.\footnote{van
  Creveld, op. cit., p.~52.}
\end{quote}

Onder deze omstandigheden vond de heer het van essentieel belang dat
zijn dienaren daadwerkelijk `hun militaire kledij aantrokken en hun heer
volgden.' Vandaar de grote nadruk op de ridderlijke eed.

De eer van de middeleeuwse ridder en de plicht van de dienstplichtige
soldaat vervulden een gelijke functie. Net zoals de moderne burger door
zijn burgerschap aan de natiestaat is verbonden, waren middeleeuwse
mensen met eden verplicht tegenover zowel hun medemensen als de Kerk.
Het schenden van een eed betekende in die tijd hoogverraad. In de
laatmiddeleeuwen deden mensen er alles aan om hun eden niet te breken,
net zoals miljoenen moderne burgers tijdens de Wereldoorlogen zich tot
het uiterste inspanden door machinegeweerbatterijen te bestormen om hun
burgerplicht na te komen.

Zowel ridderlijkheid als burgerschap gaven een extra dimensie aan de
eenvoudige afweging die anders niet-ingedoctrineerde mensen ervan zou
weerhouden het slagveld op te gaan en daarin te blijven wanneer het
zwaar werd. Deze waarden zorgden er immers voor dat mensen bereid waren
te doden en hun leven te riskeren. Alleen strenge, overdreven waarden,
krachtig ondersteund door vooraanstaande instellingen, kunnen zo'n
functie vervullen.

\subsection{Het omzeilen van de
kosten-batenanalyse}\label{het-omzeilen-van-de-kosten-batenanalyse}

Het succes en voortbestaan van elk systeem hangt af van het vermogen om
in tijden van conflict en crisis militaire krachten te bundelen.
Uiteraard baseerden middeleeuwse ridders en soldaten in de loopgraven
tijdens de Eerste Wereldoorlog hun besluit om hun leven in de strijd te
riskeren niet op een nuchtere kosten-batenberekening. Zelden verloopt
een oorlog zo eenvoudig of vallen de opbrengsten voor degenen die de
hoofdlast dragen zo fors dat een leger vol economische optimaliseerders
spontaan naar het slagveld zou opkomen. Bijna elke oorlog -- en
werkelijk de meeste veldslagen -- kent momenten waarop de balans in een
oogwenk kan omslaan. Zoals studenten van militaire geschiedenis weten,
maakt de moed, dapperheid en vurigheid waarmee individuele soldaten hun
taak op zich nemen vaak het verschil tussen nederlaag en overwinning.
Als de strijders niet bereid zijn te sterven voor een strook land die,
zodra de strijd voorbij is, nauwelijks enige waarde heeft, zullen zij
waarschijnlijk niet zegevieren tegen een even sterke tegenstander.

Dit heeft grote consequenties. Hoe effectiever een soevereiniteit
desertie tegengaat en de inzet van zijn troepen bevordert, des te groter
is de kans op militaire overwinning. Op het slagveld zorgen de meest
praktische waardesystemen ervoor dat mensen handelen op manieren die een
zuiver kortetermijnkosten-batenanalyse nooit zou voorspellen. Geen
enkele organisatie kan haar militaire macht effectief benutten als de
strijders zelf kunnen bepalen wat in hun eigen belang is en daardoor
besluiten wel of niet mee te vechten. In zo'n geval zouden zij vrijwel
nooit strijden. Alleen in de meest gunstige of juist uitzichtloze
omstandigheden neemt een rationeel denkend individu het risico van een
potentieel dodelijke confrontatie, op basis van een
kortetermijnkosten-batenrekening. Misschien gaat de \emph{Homo
economicus} op een zonovergoten dag het gevecht aan, wanneer zijn
troepen overweldigend aanwezig zijn, de vijand zwak is en de beloningen
verleidelijk. Misschien. Hij zou ook kunnen vechten als hij door
rondtrekkende kannibalen in het nauw komt.

Maar dat betreft extreme situaties. Hoe zit het met de meer alledaagse
omstandigheden van oorlogvoering, die noch aantrekkelijk genoeg zijn om
via een kosten-batenanalyse tot vechten te leiden, noch zo uitzichtloos
dat er geen alternatieve uitweg is? Juist in die gevallen spelen waarden
als ridderlijkheid en burgerschap een cruciale rol bij het succesvol
inzetten van militaire macht. Nog voordat een veldslag losbarst, moeten
de dominante machtstructuren de individuen ervan overtuigen dat het
nakomen van bepaalde verplichtingen ten opzichte van hun heer of
natiestaat belangrijker is dan hun eigen leven. De mythen en
rationalisaties die samenlevingen hanteren om het risico op het slagveld
te rechtvaardigen, vormen een essentieel onderdeel van hun militaire
slagkracht.

Voor hun effectiviteit moeten deze mythen aansluiten bij de heersende
geopolitieke omstandigheden. Het idee dat ridderlijkheid de wereld
regeert, heeft tegenwoordig weinig betekenis, zeker niet in een stad als
New York. Maar in feodale tijden was het juist de gekoesterde mythe van
het feodalisme. Het rechtvaardigde en verklaarde de verplichtingsbanden
die mensen met elkaar verbonden onder de overheersing van de Kerk en een
oorlogszuchtige adel. In een periode waarin particuliere oorlogen,
voortkomend uit hebzucht, de norm waren, hing het uitoefenen van macht
en het voortbestaan van individuen af van de bereidheid van anderen om
hun beloften tot militaire dienst -- vaak onder dwang -- na te komen.
Het was natuurlijk cruciaal dat die beloften betrouwbaar waren.

\subsection{Voor nationaliteit}\label{voor-nationaliteit}

In tegenstelling tot tegenwoordig speelde nationaliteit in de
Middeleeuwen nauwelijks -- zo niet helemaal -- een rol bij het vestigen
van soevereiniteit. Monarchen, evenals enkele prinsen van de Kerk en
machtige heren, bezaten gebieden op grond van privaatrecht. Op een
manier die we tegenwoordig niet meer tegenkomen, konden deze heren
gebieden verkopen of schenken, of nieuwe gebieden verwerven via
overdracht, huwelijk én verovering. Tegenwoordig kun je je nauwelijks
voorstellen dat de Verenigde Staten ooit onder de soevereiniteit zouden
vallen van een niet-Engelssprekende Portugese president, louter omdat
hij met de dochter van een voormalige Amerikaanse president trouwde.
Toch was zoiets in middeleeuws Europa heel gewoon. Macht werd via
erfopvolging overgedragen. Steden en landen wisselden van heersers op
dezelfde manier als antiek van eigenaar wisselt. Vaak waren de heersers
niet van oorsprong uit de gebieden waar hun bezittingen lagen. Soms
spraken ze de lokale taal niet of deden dat met een zwaar, gebrekkig
accent. Maar voor de persoonlijke verplichtingen maakte het nauwelijks
uit of een Spanjaard koning van Athene was of een Oostenrijker koning
van Spanje.

\subsection{Corporatieve
soevereiniteit}\label{corporatieve-soevereiniteit}

Ook religieuze corporaties zoals de Tempeliers, de ridders van Sint-Jan
en de Teutoonse ridders oefenden soevereiniteit uit.\footnote{Ibid.,
  p.~83.} Deze hybride organisaties hebben geen moderne tegenhangers.
Zij combineerden religieuze, sociale, juridische en financiële
activiteiten met bestuurlijk gezag over lokale gebieden. Hoewel zij
territoriale jurisdictie uitoefenden, functioneerden zij totaal anders
dan de hedendaagse overheden, aangezien nationaliteit geen rol speelde
bij het aantrekken van volgelingen of in hun bestuursstructuur. Leden en
functionarissen van deze ordes kwamen uit alle delen van het
christelijke Europa -- het `Christendom' zoals men dat destijds noemde.

Men vond het niet noodzakelijk dat de heersers uit de lokale bevolking
afkomstig waren. In het gefragmenteerde bestuursmodel van de
middeleeuwen hing het verkrijgen van steun niet af van een nationale
identiteit of een plicht aan de staat, zoals tegenwoordig vaak het geval
is, maar van persoonlijke loyaliteit en traditionele banden die als
kwestie van persoonlijke eer hoog in het vaandel stonden. Iedereen kon
deze eden afleggen, ongeacht zijn afkomst, mits men op basis van zijn
sociale positie als waardig werd geacht.

\subsection{De eed}\label{de-eed}

Ridderlijke eden verbonden mensen met elkaar en werden afgelegd op grond
van persoonlijke eer. Zoals Huizinga schreef: `Door een eed af te
leggen, legden mensen zichzelf een zekere ontzegging op als aansporing
tot het verrichten van de handelingen waartoe zij zich hadden
verplicht.'\footnote{Ibid., p.~88--89.} Men hechtte zoveel belang aan
het nakomen van eden dat mensen vaak hun leven riskeerden of zware
consequenties ondervonden om te voorkomen dat zij de eed verbraken. Vaak
verplichtte de eed de betrokkenen tot specifieke handelingen uit eer,
handelingen die u en de meeste lezers van dit boek waarschijnlijk als
belachelijk zullen ervaren.

Zo zwoeren bijvoorbeeld de Ridders van de Ster dat zij zich nooit meer
dan vier acres van het slagveld zouden verwijderen, wat er spoedig toe
leidde dat meer dan negentig van hen het leven lieten. Het verbod op
zelfs een tactische terugtrekking mag dan als militair irrationeel
klinken, maar in de ridderlijke eden was dat een vast onderdeel van hun
geloften. Kort voor de Slag bij Agincourt beval de Engelse koning dat
ridders tijdens hun patrouilles hun harnas aflegden, omdat zij, als zij
dat niet deden, hun eer schonden door zich terug te trekken uit de
vijandelijke linies. Toevallig raakte de koning de weg kwijt en kwam hij
langs het dorp dat dienstdeed als nachtverblijf voor de voorhoede van
zijn leger. Doordat hij in zijn harnas bleef, verbood zijn ridderlijke
eer hem, nadat hij zijn vergissing besefte, nog van koers te wijzigen en
terug te keren naar het dorp. Hij bracht de nacht door in een kwetsbare,
blootgestelde positie.

Hoe absurd dit voorbeeld ook lijkt, koning Hendrik had waarschijnlijk
terecht ingeschat dat hij meer risico liep door zijn eer te schenden en
zich terug te trekken -- waarmee hij een demoraliserend signaal naar
zijn hele leger zou afgeven -- dan door achter vijandelijke linies te
overnachten.

De middeleeuwse geschiedenis barst van voorbeelden van invloedrijke
personen die geloften hielden die ons tegenwoordig belachelijk zouden
overkomen. Vaak leken de handelingen op zichzelf geen direct nut te
hebben, maar dienden ze als een levendige demonstratie van hoe
belangrijk de betrokkenen hun geloften vonden. Tot deze veelvoorkomende
beloften behoorden onder meer het sluiten van één oog, het uitsluitend
eten en drinken in staande positie en het doelbewust laten ontstaan van
een zelfopgelegde kreupelheid door deel te nemen aan een
eenpersoonskettingploeg. Het dragen van pijnlijke voetijzers was
eveneens wijdverbreid. Als je tegenwoordig iemand op straat zou zien
worstelen met een zwaar beenijzer, zou je hoogstwaarschijnlijk denken
dat hij krankzinnig is in plaats van een man van grote deugd. In de
context van de ridderlijkheid gold het echter als een ereteken om zo'n
hulpmiddel vrijwillig te dragen. Er waren tal van vergelijkbare
gebruiken die ons vandaag de dag net zo absurd zouden voorkomen. Zoals
Huizinga beschrijft, legden velen de belofte af `niet op zaterdag in een
bed te slapen, niet op vrijdag dierlijk voedsel te nuttigen, enzovoort.'
De ene daad van ascese stapelde zich op de andere: een edelman beloofde
geen harnas te dragen, één dag per week geen wijn te drinken, niet in
een bed te slapen, niet aan tafel te zitten tijdens de maaltijden en een
penitentiehemd te dragen.

De vastentijd is bewaard gebleven als een veel gematigder vorm van dit
zelfopgelegde ongemak.

Veel enthousiastelingen voor geloften stichtten ordes die hun leden
bijzondere ontberingen oplegden als test van hun eer. Zo kleedde de Orde
van Clalois en Galoises zich in de zomer in bont en bontgevoerde
capuchons en stak zij een vuur in de haard, terwijl men in de winter
slechts een eenvoudig jasje zonder bont mocht dragen -- noch mantels,
noch hoeden, noch handschoenen -- en men bovendien enkel over zeer
lichte bedkleding beschikte. Zoals Huizinga opmerkt: `Het is niet
verwonderlijk dat heel wat leden aan de kou zijn gestorven.'

\begin{quote}
`Middeleeuwse zelfkastijding was een brute marteling waaraan mensen
zichzelf onderwierpen, in de hoop dat een oordelende en bestraffende God
van het straffen afzag, hun zonden vergaf en hen behoedde voor de
strengere straffen die anders op hen wachtten in deze wereld en het
hiernamaals.'\footnote{Norman Cohn, \emph{De zoektocht naar het
  millennium: revolutionaire millennialisten en mystieke anarchisten van
  de middeleeuwen}, herziene en uitgebreide editie (Oxford: \emph{Oxford
  University Press}, 1970), p.~127.} - NORMAN COHN
\end{quote}

\subsection{Zelfkastijding, toen en nu}\label{zelfkastijding-toen-en-nu}

Het maakte maar een korte stap uit om na het afleggen van een gelofte --
die al gevaar en ontbering inhield -- over te gaan op beproevingen,
pelgrimstochten, boetedoening, ongemakken en zelfs het opzettelijk
toebrengen van zelfverwonding. In de middeleeuwen vond men deze
handelingen bijzonder waardevol en prijzenswaardig, omdat ze de ernst
toonden waarmee men zijn geloften nakwam -- een denkwijze die zelfs
tegenwoordig nog weerklank vindt bij inwijdingsrituelen in
broederschappen en zusterschappen.

Het benauwende zomerweer, de ijskoude winter en het op blote voeten door
de sneeuw trekken tijdens pelgrimstochten leken relatief mild in
vergelijking met de `grimmige marteling' van zelfkastijding. Deze
specifieke vorm van boetedoening ontstond vrijwel gelijktijdig met het
opkomen van het feodalisme. Al in het begin van de elfde eeuw namen
kluizenaars in de kloostergemeenschappen van \emph{Camaldoli} en
\emph{Fonte Avellana} deze praktijk over.\footnote{Ibid.}

In plaats van alleen op blote voeten de kou te trotseren, organiseerden
de zelfkastijders uitgebreide processies waarin ze dag en nacht van de
ene stad naar de andere marcheerden. Telkens wanneer ze een stad
bereikten, verzamelden zij zich voor de kerk en sloegen zichzelf
urenlang.\footnote{Ibid., p.~128.}

Wij zijn ervan overtuigd dat mensen in de toekomst, wanneer zij
terugblikken op het tijdperk van de natiestaat, bepaalde activiteiten
die in de twintigste eeuw in naam van het burgerschap werden ondernomen,
net zo absurd zullen vinden als wij tegenwoordig de zelfkastijding
beschouwen. Vanuit het perspectief van de informatiesamenleving lijkt
het schouwspel van soldaten die de halve wereld doorkruisen om de dood
uit te dagen uit loyaliteit aan de natiestaat ronduit grotesk en
belachelijk. Het doet denken aan de bijzondere, overdreven rituelen uit
de riddertijd, zoals het rondlopen in beenijzers, een praktijk waaraan
anders weloverwogen mensen in het feodale tijdperk met trots deelnamen.

\subsection{Ridderlijkheid maakt plaats voor
burgerschap}\label{ridderlijkheid-maakt-plaats-voor-burgerschap}

Ridderlijkheid verdween en maakte ruimte voor burgerschap, zodra de
megapolitieke omstandigheden veranderden en de militaire functie van de
trouwdaad aan een heer achterhaald raakte. Het tijdperk van
buskruitwapens en industriële legers bracht totaal andere verhoudingen
tussen de strijders en hun bevelhebbers met zich mee. Burgerschap
ontstond in een periode waarin de opbrengsten van geweld hoog waren en
verder stegen, terwijl de staat over veruit grotere middelen beschikte
dan de maatschappelijke entiteiten die in de middeleeuwen vochten.
Dankzij haar overweldigende macht en rijkdom kon de natiestaat direct
overeenkomsten sluiten met de massa gewone soldaten in haar uniform.

Dergelijke overeenkomsten bleken voor de staat veel goedkoper en minder
problematisch dan het bijeenbrengen van troepen via onderhandelingen met
machtige heren en lokale grootheden, die telkens eisen afwezen die niet
in hun eigen belang waren -- iets wat geen enkele individuele burger in
de natiestaat kon.

Om redenen die we later uitgebreider behandelen, hing burgerschap in
belangrijke mate af van het feit dat geen enkel individu, noch een
kleine groep, op megapolitieke schaal zelfstandig militaire macht kon
uitoefenen. Naarmate de informatietechnologie de logica van het gevecht
verandert, zullen de mythen over burgerschap onvermijdelijk verouderen,
net zoals buskruit de middeleeuwse ridderlijkheid achterhaald deed
lijken.

\subsection{Hell's Angels te paard}\label{hells-angels-te-paard}

De aristocratie van bereden krijgers die eeuwenlang West-Europa
beheerste, sloeg totaal niet op als de nobele heren die later als
rolmodel zouden gelden voor hun nageslacht. Zij waren ruw en
gewelddadig. Je zou ze tegenwoordig het beste kunnen vergelijken met
middeleeuwse motorbendes. Hun omgangsregels en ridderlijke pretenties
waren er niet op gericht hun ware karakter te onthullen, maar dienden
vooral om hun excessieve gedrag in bedwang te houden.

\subsection{Perfectie als synoniem voor
uitputting}\label{perfectie-als-synoniem-voor-uitputting}

De introductie van effectieve buskruitwapens aan het einde van de
vijftiende eeuw bracht een enorme schok teweeg -- net op het moment dat
gewapende ridders hun kunst tot in de perfectie hadden verfijnd. Door
zorgvuldig fokken had men inmiddels een oorlogspaard van zestien handen
hoog ontwikkeld, een ros dat een ridder in volledige wapenuitrusting
comfortabel kon dragen. Zoals C. Northcote Parkinson scherp opmerkte,
wordt `perfectie' bereikt in systemen die op instorten staan.\footnote{C.
  Northcote Parkinson, \emph{Parkinson's law and other studies in
  administration} (Boston: Houghton Mifflin, 1957), p.~60, geciteerd in
  Tilly, p.~4.} Terwijl men het nieuwe oorlogspaard verfijnde,
introduceerden de nieuwe wapens een significante verandering: ze wisten
zowel paard als ridder van het slagveld weg te vegen. Deze
buskruitwapens lieten gewone mensen toe ze af te vuren; zij vereisten
weinig vaardigheid, maar bleken duur in aanschaf wanneer men ze in grote
aantallen wilde inzetten. Doordat hun verspreiding toenam, steeg het
belang van de handel gestaag ten opzichte van de landbouw, die tot dan
toe de kern vormde van de feodale economie.

\subsection{Oorlog op grotere schaal}\label{oorlog-op-grotere-schaal}

Hoe hebben buskruitwapens zo'n transformatie teweeggebracht?\\
Ten eerste vergrootten ze de schaal van de gevechten, waardoor het
voeren van oorlog al snel veel kostbaarder werd dan in de
middeleeuwen.\\
Vóór de buskruitrevolutie vochten oorlogen meestal met zo'n kleine
troepen, die uit een klein en arm grondgebied konden worden
opgeroepen.\\
Buskruit bood een nieuw voordeel bij grootschalige gevechten.\\
Alleen leiders die aanspraak konden maken op welvarende onderdanen
konden zich veroorloven effectieve strijdkrachten op te bouwen onder de
nieuwe omstandigheden.\\
Leiders die de groei van de handel omarmden -- meestal monarchen die
samenwerkten met stedelijke kooplieden -- ontdekten dat zij op het
slagveld een concurrentievoordeel behaalden.\\
In de woorden van van Creveld: `Dankzij, deels, de superieure financiële
middelen die tot hun beschikking stonden, konden zij meer kanonnen
aanschaffen dan wie dan ook en de tegenstand in stukken
blazen.'\footnote{van Creveld, op. cit., p.~50.}

Hoewel het nog eeuwen zou duren voordat buskruitwapens hun volledige
potentie zouden laten zien in de burgerlegeren van de Franse Revolutie,
wees de invoering van militaire uniformen tijdens de renaissance al
vroeg op de transformatie van de oorlogsvoering door buskruit.\\
Deze uniformen symboliseren treffend de nieuwe verhoudingen tussen de
krijger en de natiestaat, die samenhingen met de overgang van
ridderlijkheid naar burgerschap.\\
In wezen sloot de nieuwe natiestaat een uniforme overeenkomst met haar
burgers, in tegenstelling tot de uiteenlopende afspraken die een vorst
of paus binnen het feodale systeem met een lange keten vazallen
maakte.\\
In het oude systeem bekleedde ieder een unieke positie binnen een
gelaagde hiërarchie; ieders overeenkomst was net zo uniek als zijn
familiewapen en de kleurrijke vaandels die hij droeg.

\subsection{Het verlagen van de alternatieve kosten van
rijkdom}\label{het-verlagen-van-de-alternatieve-kosten-van-rijkdom}

Buskruitwapens hebben de samenleving op nog een andere, radicaal andere
manier veranderd.\\
Ze scheidden het uitoefenen van macht van fysieke kracht, waardoor de
alternatieve kosten van commerciële activiteiten daalden.\\
Rijke kooplieden hoefden niet langer te vertrouwen op hun eigen
behendigheid en kracht in hand-tot-handgevechten, noch op huurlingen met
twijfelachtige loyaliteit om zich te verdedigen.\\
Ze konden rekenen op de bescherming van de nieuwe, grotere legers van
invloedrijke vorsten.\\
Zoals William Playfair over de middeleeuwen opmerkte: ``Zolang fysieke
kracht de macht was waarmee men bij vijandigheid werd geconfronteerd,
\ldots{} {[}t{]}erwijl het toen onmogelijk was om langdurig zowel rijk
als machtig te zijn.''\footnote{Playfair, op. cit., p.~72.}\\
Toen het buskruit opkwam, werd het namelijk onmogelijk om macht uit te
oefenen zonder rijkdom.

\subsection{Status en statisch begrip}\label{status-en-statisch-begrip}

Net zoals de meeste mensen tegenwoordig niet voorbereid zijn op de
veranderende dynamiek van de informatiemaatschappij, bleven de
vooraanstaande denkers in de middeleeuwen achter bij het voorspellen en
doorgronden van de opkomst van de handel, die een cruciale rol speelde
in de vorming van de moderne tijd. Vijf eeuwen geleden zagen de mensen
hun snel veranderende samenleving als iets statisch. Zoals Huizinga
opmerkte: ``Zeer weinig eigendom is in de moderne zin liquide, terwijl
macht nog niet hoofdzakelijk met geld wordt geassocieerd; het is nog
eerder inherent aan de persoon en berust op een soort religieus ontzag
dat hij inboezemt; het komt tot uiting in praal en grootsheid, of in een
talrijke stoet van trouwe volgelingen. Feodaal of hiërarchisch denken
drukt het idee van grootsheid uit door zichtbare
tekenen\ldots.''\footnote{Huizinga, op. cit., p.~26.}

Omdat men in de late middeleeuwen vooral aan status dacht, zagen zij
niet in dat kooplieden een belangrijke bijdrage konden leveren aan het
functioneren van het rijk. Kooplieden behoorden vrijwel altijd tot de
laagste stand van de drie, onder de adel en de geestelijkheid.

Zelfs de meest vooruitziende denkers van die tijd erkenden niet dat
handel en ander ondernemerschap buiten de landbouw een wezenlijke bron
van rijkdom konden zijn. Voor hen was armoede een deugd. Ze maakten
letterlijk geen onderscheid tussen een vermogende bankier en een
bedelaar. Zoals Huizinga verwoordde: ``Er werd in de derde stand in
principe geen onderscheid gemaakt tussen rijke en arme burgers, noch
tussen stadsbewoners en plattelandsmensen.''\footnote{Ibid., p.~57.} In
hun opvatting deden beroep en rijkdom er niet toe; alleen de ridderlijke
status telde.

Deze onoplettendheid voor de economische dimensie van het leven
versterkten ook de geestelijken, de ideologische hoeders van de
middeleeuwse samenleving. Zij schatten het belang van handel zo laag in
dat een breed geprezen hervormingsprogramma uit de vijftiende eeuw
voorschreef dat alle niet-adellijke personen zich uitsluitend op
ambachtelijk werk of landbouw moesten richten. Handel kreeg daarbij
werkelijk geen enkele plek.\footnote{Ibid.}

\begin{quote}
`De datum 1492, die conventioneel wordt gebruikt om de middeleeuwen van
de moderne geschiedenis te scheiden, is net zozeer een geschikt
breukpunt als elk ander, want in het perspectief van de
wereldgeschiedenis symboliseert Columbus' reis het begin van een nieuwe
relatie tussen West-Europa en de rest van de wereld.'\footnote{Frederic
  C. Lane, \emph{Venetië: een maritieme republiek} (Baltimore:
  \emph{Johns Hopkins University Press}, 1973), p.~275.} - FREDERICA C.
LANE
\end{quote}

\section{De geboorte van het industriële
tijdperk}\label{de-geboorte-van-het-industriuxeble-tijdperk}

Veel van de meest scherpzinnige geesten in de vijftiende eeuw ontbeerden
een cruciale ontwikkeling in de geschiedenis, een verschijnsel dat zich
vlak voor hun ogen ontvouwde. De ondergang van het feodalisme markeerde
het begin van de grote moderne fase van westerse overheersing. Het was
een periode waarin men geweld steeds hoger beloonde en ondernemerschap
op grote schaal groeide. In de afgelopen tweeënhalf eeuwen heeft de
moderne economie de levensstandaarden van de landen die er het meest van
profiteerden ongekend verhoogd. Nieuwe technologieën -- van
buskruitwapens tot de drukpers -- verlegden de grenzen van het leven op
manieren die maar weinigen zich konden voorstellen.

Tegen het einde van de vijftiende eeuw openden ontdekkingsreizigers
zoals Columbus de toegangspoorten tot immense, onbekende continenten.
Voor het eerst in onherinnerde tijden bereikte de mens alle uithoeken
van de wereld. Galeons -- nieuwe, imposante improvisaties op mediterrane
galeien -- zeilden de wereld rond en legden routes vast die al snel
uitgroeiden tot belangrijke handelswegen en hoofdaders voor zowel
ziekteverspreiding als verovering. Conquistadores, die met hun
gloednieuwe bronzen kanonnen zowel op zee als aan wal te werk gingen,
openden nieuwe horizonten. Zij verwierven fortuinen in goud en
specerijen, legden de basis voor nieuwe winstgewassen -- van tabak tot
aardappelen -- en reserveerden uitgestrekte weidegronden voor hun vee.

\subsection{De eerste industriële
technologie}\label{de-eerste-industriuxeble-technologie}

Net zoals het kanon nieuwe economische mogelijkheden ontsloot, opende de
boekdrukkunst de deur naar een geheel nieuw intellectueel tijdperk. De
drukpers functioneerde als de eerste massaproductiemachine en betekende
daarmee het begin van het industrialisme. Hiermee onderschrijven we het
standpunt dat Adam Smith in \emph{`The Wealth of Nations'} naar voren
bracht, namelijk dat de industriële revolutie al gaande was lang voordat
hij schreef. Hoewel het systeem nog niet volgroeid was, lagen de
fundamenten van massaproductie en het fabrieksysteem al stevig
verankerd. Zijn beroemde voorbeeld van de speldfabrikanten illustreert
dit treffend: Smith legt uit dat men achttien afzonderlijke handelingen
toepast bij de productie van speldjes. Dankzij de gespecialiseerde
technologie en arbeidsverdeling maakte elke werknemer in één dag wel
4.800 keer zoveel speldjes als wanneer hij het in eigen kracht had
moeten doen.\footnote{Adam Smith, \emph{Een onderzoek naar de aard en
  oorzaken van de rijkdom van de naties} (Chicago: \emph{University of
  Chicago Press}, 1976), pp.~8--9.}

Smiths voorbeeld maakt duidelijk dat de industriële revolutie al eeuwen
eerder begon dan historici doorgaans aannemen. De meeste leerboeken
situeren het begin ervan in het midden van de achttiende eeuw -- een
redelijke datum voor de eerste fase waarin de levensstandaarden langzaam
verbeterden. In werkelijkheid zette de ingrijpende transitie van het
feodalisme naar het industrialisme al aan het einde van de vijftiende
eeuw door, met vrijwel onmiddellijke gevolgen voor de heersende
instituties, vooral merkbaar in de snel afnemende invloed van de
middeleeuwse kerk.

Historici die de industriële revolutie later in de tijd plaatsen, kijken
eigenlijk naar een ander aspect: de eerste opmars in de
levensstandaarden dankzij massaproductie met motorische aandrijving. Dit
zorgde ervoor dat de waarde van ongeschoolde arbeid toenam en dat de
prijzen van een breed scala aan consumptiegoederen daalden. Het feit dat
in verschillende landen de levensstandaarden op uiteenlopende momenten
fors begonnen te stijgen, wijst erop dat men meer meet dan enkel de
megapolitieke omslag. \emph{The Cambridge Economic History of Europe}
hanteert zelfs de term `industriële revoluties' in het meervoud en
koppelt deze expliciet aan de geleidelijke groei van nationale
inkomens.\footnote{Zie H. J. Habakkuk en M. Postan, red., \emph{De
  Cambridge economische geschiedenis van Europa}, vol.~6, \emph{De
  industriële revolutie en daarna: inkomens, bevolking en technologische
  verandering} (Cambridge: \emph{Cambridge University Press}, 1966).} In
Japan en Rusland kwam die inkomensstijging pas aan het einde van de
negentiende eeuw op gang. Ook in andere delen van Azië en in sommige
Afrikaanse regio's zagen we in de twintigste eeuw een substantiële
stijging van zowel de levensstandaarden als de nationale inkomens. In
een aantal Afrikaanse gebieden blijft een stabiele groei tot op heden
echter slechts een droom -- wat niet betekent dat die regio's niet deel
uitmaken van het moderne tijdperk.

\subsection{Daling van het inkomen in de
transitie}\label{daling-van-het-inkomen-in-de-transitie}

De groei van het inkomen gaat niet automatisch gepaard met de komst van
het industrialisme. De omschakeling naar een industriële samenleving
vormde een megapolitiek fenomeen dat niet direct zichtbaar werd in de
inkomenscijfers. Sterker nog, het reële inkomen van de meeste Europeanen
daalde tijdens de eerste twee eeuwen van het industriële tijdperk. Pas
na het begin van de achttiende eeuw begon het te stijgen, zodat het rond
1750 weer het niveau van 1250 bereikte. Wij stellen dat het industriële
tijdperk al aan het einde van de vijftiende eeuw inzet. De kenmerken van
de vroegmoderne technologie -- zoals chemisch aangedreven wapens en
drukpersen -- versnelden de val van het feodalisme.

\subsection{Verlaging van de kosten van
kennis}\label{verlaging-van-de-kosten-van-kennis}

Het massaal produceren van boeken ondermijnde in de middeleeuwen de
gevestigde instellingen, net zoals microtechnologie later op subversieve
wijze de moderne natiestaat zal beïnvloeden. De opkomst van de drukpers
verzwakte al snel het monopolie van de Kerk op het Woord van God en
creëerde tegelijkertijd een nieuwe markt voor ketterij. Ideeën die fel
tegen de gesloten feodale samenleving ingingen, verspreidden zich
razendsnel -- in de laatste tien jaar van de vijftiende eeuw publiceerde
men maar liefst 10 miljoen boeken. Omdat de Kerk de drukpers wilde
onderdrukken, kwamen de meeste nieuwe werken alleen in de delen van
Europa te voorschijn waar de macht van de gevestigde autoriteiten het
zwakst was. Dit vormt een treffende analogie met de hedendaagse pogingen
van de Amerikaanse overheid om encryptietechnologie te onderdrukken. De
Kerk kwam er al snel achter dat censuur de verspreiding van subversieve
technologie niet kon stoppen; integendeel, het leidde er slechts toe dat
men deze op een nog subversievere manier toepaste.

\subsection{Waardevermindering van de
kloosters}\label{waardevermindering-van-de-kloosters}

Veel toepassingen van de drukpers, die in eerste instantie onschuldig
leken, bleken door hun inhoud subversief te zijn. Alleen al het
verspreiden van kennis over de rijkdommen die onverschrokken avonturiers
en kooplieden konden vergaren, werkte als een krachtig middel om de
banden van feodale verplichtingen te verbreken. De aantrekkingskracht
van nieuwe markten, gecombineerd met de behoefte en het vermogen om op
grote schaal legers en vloten te financieren, schonk geld een waarde die
in de feodale tijden ontbrak. Deze nieuwe investeringskansen, versterkt
door krachtige wapens die de opbrengsten van geweld verhoogden, zorgden
ervoor dat de heer in het achterland en de koopman in de stad het steeds
kostbaarder vonden om hun kapitaal aan de Kerk te schenken. Daardoor
verloren de feodale instellingen hun stabiliteit en raakte de
onderliggende ideologie ondermijnd.

Een ander subversief gevolg van de boekdrukkunst was dat de kosten voor
het reproduceren van informatie drastisch daalden. Een belangrijke reden
waarom geletterdheid en economische groei in de middeleeuwen zo beperkt
bleven, lag in de hoge prijs van het handmatig kopiëren van
manuscripten. Zoals we zagen, behoorde het reproduceren van boeken en
manuscripten in benedictijnse kloosters na de val van Rome tot de
voornaamste productieve taken van de Kerk. Dat was een uiterst kostbare
onderneming. Een van de meest dramatische effecten van de boekdrukkunst
was dat de scriptoria -- waar monniken dag na dag manuscripten
kopieerden die vervolgens binnen enkele uren door de drukpers in
overvloed werden gereproduceerd -- in waarde daalden. De nieuwe
technologie maakte het benedictijnse scriptorium tot een overbodige en
dure methode voor het reproduceren van kennis, waardoor de religieuze
orden en de Kerk, die op de kopisten rekenden, economisch minder van
betekenis werden.

De massaproductie van boeken maakte een einde aan het monopolie van de
Kerk op de Schrift en andere informatievormen. Dankzij de bredere
beschikbaarheid van boeken daalden de kosten voor geletterdheid,
waardoor steeds meer denkers in staat waren hun eigen mening te geven
over belangrijke onderwerpen, met name op theologisch gebied. Zoals de
theologische historicus Euan Cameron opmerkte, legde `een reeks
publicatiemijlpalen' in de eerste twee decennia van de zestiende eeuw de
basis voor de toepassing van `moderne tekstkritiek op de
Schriften.'\footnote{Euan Cameron, \emph{De Europese reformatie}
  (Oxford: \emph{The Clarendon Press}, 1992), p.~68.} Hiermee
ondermijnde men het monopolie van de Kerk door corrupte interpretaties
van teksten -- die traditionele dogma's ondersteunden -- in twijfel te
trekken.\footnote{Ibid.} Deze nieuwe inzichten stimuleerden de opkomst
van protestantse sekten die hun eigen interpretaties van de Bijbel
wilden formuleren. Bovendien verlaagde de massaproductie van boeken de
drempel voor ketterij, waardoor ketters konden rekenen op een veel
groter lezerspubliek.

De publicatie droeg ook bij aan de ondergang van de middeleeuwse
wereldbeschouwing. De toegenomen beschikbaarheid en de lagere kosten van
informatie brachten een verschuiving teweeg: een wereldbeeld dat tot
stand kwam door symboliek maakte plaats voor een visie gebaseerd op
oorzakelijke verbanden. `Het wereldbeeld van de symboliek kenmerkt zich
door een feilloze orde, een architectonische structuur en een
hiërarchische ondergeschiktheid. Elke symbolische verbinding duidt op
een verschil in rang of heiligheid\ldots{} De walnoot staat voor
Christus; de zoete pit voor Zijn goddelijke natuur, de groene en zachte
buitenhuid vertegenwoordigt Zijn menselijkheid en de houten schaal
ertussen symboliseert het kruis. Op die wijze tillen alle dingen de
gedachten op naar het eeuwige\ldots{}'\footnote{Huizinga, op. cit.,
  p.~198.}

Een symbolische denkwijze versterkte niet alleen de hiërarchisch
gestructureerde samenleving, maar paste ook bij een analfabetisch
publiek. Ideeën die via symbolen op houtsneden werden overgebracht,
waren toegankelijk voor analfabetische mensen. Daarentegen stimuleerde
de opkomst van de boekdrukkunst in de moderne tijd het ontwikkelen van
oorzakelijke verbanden en het toepassen van de wetenschappelijke methode
voor een geletterde bevolking.

\section{Een parallel voor vandaag}\label{een-parallel-voor-vandaag}

Midden in de vijftiende eeuw leek de middeleeuwse samenleving stabiel en
zelfverzekerd in haar overtuigingen, maar een technologische revolutie
transformeerde deze in een oogwenk. De Kerk, haar voornaamste
instelling, zag haar monopolie onder vuur komen en verbrijzeld raken.
Een autoriteit die al eeuwenlang als onbetwist werd gezien, kwam plots
in twijfel te staan. Overtuigingen en loyaliteiten -- eens heiliger dan
de band tussen een burger en de natiestaat van vandaag -- werden binnen
enkele jaren heroverwogen en opgegeven, allemaal door een technologische
revolutie die in het laatste decennium van de vijftiende eeuw tot volle
bloei kwam.

Wij menen dat een ingrijpende verandering, vergelijkbaar met die van
vijfhonderd jaar geleden, opnieuw zal plaatsvinden. De
informatierrevolutie zal het machtsmonopolie van de natiestaat
tenietdoen, net zoals de buskruitrevolutie ooit het monopolistische
gezag van de Kerk beëindigde. Er valt een opvallende gelijkenis op
tussen het einde van de vijftiende eeuw, waarin het leven doordrenkt was
met georganiseerde religie, en nu, waarin de wereld verzadigd is met
politiek. Zowel de oude Kerk als de hedendaagse natiestaat laten zien
hoe instellingen tot een seniele extremiteit kunnen doorgroeien. Net als
de laatmiddeleeuwse Kerk kampt de natiestaat aan het einde van de
twintigste eeuw met torenhoge schulden en kan zij zichzelf steeds minder
bekostigen. Haar functioneren verliest snel zijn relevantie en werkt
zelfs contraproductief voor de welvaart van degenen die haar tot voor
kort nog trouw steunden.

\subsection{Verarmd, gretig en
extravagant}\label{verarmd-gretig-en-extravagant}

Net zoals de overheid tegenwoordig weinig waar voor haar geld levert,
deed de Kerk aan het einde van de vijftiende eeuw hetzelfde. De
kerkelijke historicus Euan Cameron merkte op: ``{[}E{]}en verarmd lokaal
priesterschap leek nauwelijks de prijs waard waarvoor er geld werd
gevraagd; veel van hetgeen werd geheven `verdween' effectief in
afgesloten kloosters of in de obscure domeinen van het hoger onderwijs
en de administratie.''\footnote{Cameron, op. cit., pp.~26--27.} Ondanks
de royale giften aan bepaalde sectoren van de Kerk slaagde de instelling
als geheel er toch in om tegelijkertijd verarmd, gretig en extravagant
over te komen. Het valt moeilijk te ontkennen dat er een duidelijke
parallel ligt met de overheid van de late twintigste eeuw.

In de late vijftiende eeuw breidden religieuze vieringen zich uit op een
wijze die doet denken aan de programma's in hedendaagse welvaartsstaten.
Niet alleen namen speciale zegeningen vrijwel onbeperkt toe -- vergezeld
van de toekenning aan heiligen en de tentoonstelling van hun botten --
maar elk jaar ontstonden er tevens meer kerken, nonnenkloosters,
abdijen, kloosters van broeders, biechtvaders (inwonende huispriesters),
predikantschappen, kathedraalkapitelen, gestifte gebedshuizen,
reliekvereringen, religieuze confraterniteiten, feesten en nieuwe
heilige dagen. De diensten duurden langer en zowel de gebeden als de
hymnen kregen een complexer karakter. Steeds weer introduceerde men
nieuwe bedelorders om aalmoezen te innen. Het resultaat was een
institutionele overbelasting die doet denken aan die in sterk
gepolitiseerde samenlevingen van tegenwoordig.

Religieuze festivals en feestdagen schoten als paddenstoelen uit de
grond. Ook het aantal religieuze diensten nam toe, met speciale
festiviteiten ter ere van de zeven smarten van Maria, van haar zusters
en van alle heiligen in de afstamming van Jezus.\footnote{Huizinga, op.
  cit., p.~149.} Voor de gelovigen werd het steeds kostbaarder en
zwaarder om aan hun religieuze verplichtingen te voldoen, vergelijkbaar
met de stijgende kosten die men tegenwoordig maakt om binnen de grenzen
van de wet te blijven.

\subsection{De onschuldigen betalen}\label{de-onschuldigen-betalen}

Toen, net als nu, droeg de productieve klasse een steeds zwaardere last
door inkomensherverdeling. De kosten stegen veel sneller dan de
machthebbers doorhadden doordat men kapitaal anders toepaste. Het
relatieve voordeel van landbezit tegenover geldkapitaal nam af. Toch
bleef men in de middeleeuwse geest denken in termen van een
statusgebonden samenleving, waarbij je sociale positie afhing van je
afkomst en niet van je vermogen om kapitaal effectief in te zetten. Men
hield nauwelijks rekening met de stijgende opportuniteitskosten van het
organiseren van overdreven religieuze plechtigheden. Vooral de
ambitieuze en hardwerkende boeren, burghers en vrije boeren -- die, in
tegenstelling tot de aristocratie, afhankelijk waren van een efficiënt
gebruik van hun kapitaal -- droegen een onevenredig deel van de kosten
voor het dekken van tafels tijdens eindeloze feesten en heilige dagen,
en voor het ondersteunen van een buitensporige kerkelijke bureaucratie.

\subsection{Contraproductieve
regelgeving}\label{contraproductieve-regelgeving}

Aan het einde van de vijftiende eeuw oefende de Kerk vrijwel alle
regulerende bevoegdheden uit, bevoegdheden die later door
overheidsinstanties werden overgenomen. De Kerk domineerde essentiële
rechtsgebieden: het vastleggen van akten, het registreren van
huwelijken, het afhandelen van testamenten, het verlenen van
handelslicenties, het toekennen van landtitels en het bepalen van de
voorwaarden voor het handelsverkeer. Vroeger werd het dagelijkse leven
bijna net zo nauwgezet geregeld door het canonieke recht als
tegenwoordig door de bureaucratie, en met min of meer hetzelfde doel
voor ogen. Net zoals hedendaagse politieke regelgeving vaak vol
verwarringen en tegenstrijdigheden zit, kampten de regels van het
canonieke recht vijfhonderd jaar geleden met soortgelijke problemen.
Deze regelgeving belemmerde het handelsverkeer zo sterk dat al snel
duidelijk werd dat de belangen van de regelgevers ver afstonden van het
bevorderen van de productiviteit.

Zo verbood men bijvoorbeeld gedurende een heel jaar op de dag van de
week waarop het meest recente 28 december viel, legale handel te
drijven. Valt 28 december op een dinsdag, dan mocht men op dinsdag geen
handel drijven, als verplichte uiting van vroomheid ter ere van de
Slachting der Onschuldigen. In jaren waarin 28 december op een andere
dag dan zondag viel, belemmerde dit verbod diverse vormen van handel,
waardoor transacties moesten worden uitgesteld of zelfs geheel werden
vermeden, met stijgende kosten als gevolg.

\subsection{Monopoly pricing}\label{monopoly-pricing}

Men voerde het kanonieke recht in om de monopolistische prijsstelling te
ondersteunen. Dankzij de verkoop van aluin -- gewonnen uit de
bezittingen in Tolfa, Italië -- boekte de Kerk flinke inkomsten. Toen
klanten in de textielindustrie kozen voor goedkoper aluin uit Turkije,
handhaafde het Vaticaan haar monopolistische prijzen door het goedkope
aluin als zondig te bestempelen. Men excommuniceerde handelaars die toch
voor het Turkse product kozen. Het beroemde verbod op het eten van vlees
op vrijdag ontstond in dezelfde geest. De Kerk was immers niet alleen de
grootste feodale landeigenaar, maar beschikte ook over waardevolle
visgronden. Kerkvaders stelde dat de vrome vis moesten eten, waarmee zij
-- niet toevallig -- in een periode waarin transport- en
hygiëneproblemen de visconsumptie bemoeilijkten, de vraag naar vis
veiligstelden.

Net zoals moderne natiestaten reguleerden ook de laatmiddeleeuwse Kerkse
autoriteiten bepaalde industrieën om eigenbelang te dienen en benutten
zij hun bevoegdheid om op andere manieren inkomsten te genereren.
Geestelijken stelden voorschriften en edicten op die vrijwel onmogelijk
in de praktijk te volgen waren. Zo hanteerden zij een ruim begrip van
incest, waardoor zelfs verre neven en nichten -- en mensen die
uitsluitend door huwelijk met elkaar verbonden waren -- een bijzondere
dispensatie nodig kregen om te mogen trouwen. Omdat deze bepaling
vrijwel iedereen in de kleine Europese dorpen vóór de moderne
reismogelijkheden trof, groeide de verkoop van vrijstellingen voor
incestueuze huwelijken uit tot een winstgevende inkomstenbron voor de
Kerk. Ook de seksuele omgang binnen het huwelijk kende strikte
kerkelijke regels. Echtgenoten mochten geen seks hebben op zondagen,
woensdagen en vrijdagen en ook niet gedurende de veertig dagen
voorafgaand aan Pasen en Kerstmis. Verder verbood de Kerk stellen om
drie dagen voor het ontvangen van de communie seks te hebben. Kortom,
gehuwde stellen mochten pas seksuele betrekkingen hebben nadat zij een
indulgentie hadden verkregen, die minstens 55 procent van het jaar
geldig was. In \emph{The Bishop's Brothels} stelt historicus E. J.
Burford dat deze schijnbaar absurde huwelijksvoorschriften de bloei van
de middeleeuwse prostitutie stimuleerden, waaruit de Kerk aanzienlijk
profiteerde.\footnote{Cameron, op. cit., pp.~26--27.} Burford meldt dat
de bisschop van Winchester jarenlang de voornaamste eigenaar was van de
bordelen aan Bankside in Londen, Southwark. Bovendien was het kerkelijke
winstbejag uit prostitutie geenszins een louter lokaal Engels
verschijnsel:

\begin{quote}
Paus Sixtus IV (ca. 1471), die naar verluidt syfilis opliep van een van
zijn vele minnaressen, was de eerste paus die prostituees licenties
verleende en een belasting op hun inkomsten instelde, waardoor de
pauselijke inkomsten aanzienlijk toenamen. De Romeinse Curia financierde
daarmee gedeeltelijk de bouw van Sint-Pieters, terwijl de verkoop van
licenties eveneens veel geld opleverde. Zijn opvolger, paus Leo~X, zou
naar verluidt ongeveer tweëntwintigduizend gouden ducats hebben verdiend
met de verkoop van licenties -- vier keer zoveel als hij met de verkoop
van aflaten in Duitsland binnenhaalde.\footnote{Ibid., p.~102.}
\end{quote}

Ook de beroemde celibatsregel voor priesters vormde een winstgevende
inkomstenbron voor de middeleeuwse Kerk. Volgens Burford introduceerde
de Kerk een zogenaamd `cullagium' -- een heffing die zij aan
concubinaire priesters oplegde. Het bleek zo lucratief dat bisschoppen
in Frankrijk en Duitsland deze heffing als standaard voor alle priesters
verplicht stelden, ondanks dat het Lateranenconcilie in 1215 dit
`schandelijke verkeer, waarbij prelaten geregeld toestemming tot zonde
verkochten,' veroordeelde.\footnote{Ibid., p.~103.} Dit was slechts één
van de vele winstgevende markten waarin licenties werden verhandeld om
het kanonieke recht en de kerkelijke regels te overtreden -- een handel
die voortkwam uit dezelfde logica die ook hebzuchtige politici ertoe
aanzet willekeurige reguleringsbevoegdheden op de handel te leggen.

\subsection{aflaatbrieven}\label{aflaatbrieven}

De mogelijkheid om naar eigen goeddunken regelgeving in te voeren houdt
ook in dat vrijstellingen verkocht mogen worden ter compensatie van de
schade die die regels met zich meebrengen. De kerk verkocht
vergunningen, oftewel `aflaatbrieven', die uiteenlopende privileges
verleenden -- van het kwijtschelden van kleine heffingen op de handel
tot de toestemming om tijdens de vastentijd zuivelproducten te nuttigen.
Deze `aflaatbrieven' werden niet alleen voor hoge prijzen aan de adel en
de welgestelde burgerij verkocht, maar verschenen ook als loterijprijzen
-- vergelijkbaar met de door de overheid georganiseerde loterijen van
tegenwoordig -- om zo ook het geld van de armen binnen te
halen.\footnote{Huizinga, op. cit., p.~151.} De handel in aflaatbrieven
groeide naarmate de uitgaven van de kerk haar inkomsten overtroffen.
Hieruit concludeerden velen dat de institutionele kerk haar macht
hoofdzakelijk inzette om inkomsten te genereren. Zoals een hedendaagse
criticus opmerkte: `{[}C{]}anoniek recht werd uitsluitend ingesteld met
het doel om een hoop geld te verdienen; wie christen wil zijn, moet zich
uitkopen.'\footnote{Cameron, op. cit., p.~31.}

\subsection{Bureaucratische
overbelasting}\label{bureaucratische-overbelasting}

Aan het einde van de vijftiende eeuw waren de kosten om een
geïnstitutionaliseerde religie draaiende te houden historisch hoog,
vergelijkbaar met de buitensporige uitgaven waartegenover moderne
overheden tegenwoordig lijken te strijden. Naarmate het dagelijks leven
meer doordrenkt raakte met religie, namen ook de kosten en de
bureaucratische lagen binnen de kerk toe. In de woorden van Cameron:
`Het was veel gemakkelijker om mensen te vinden om de enorm gegroeide
functies binnen de kerk aan het einde van de middeleeuwen in te vullen,
dan om geld te vinden om hen te betalen.'\footnote{Ibid., p.~24.} Net
zoals failliete overheden tegenwoordig op contraproductieve wijze
proberen hun inkomsten te verhogen, deed de kerk dat vijfhonderd jaar
geleden ook. Inderdaad, de geestelijken maakten gebruik van enkele van
die roofzuchtige trucs die moderne politici inmiddels hebben verfijnd.

Vijfhonderd jaar geleden verbruikte de middeleeuwse kerk, net als de
hedendaagse natiestaten, meer van de middelen van de samenleving dan
ooit tevoren -- en meer dan ooit nog zou doen. Destijds leek de kerk,
net als de huidige staat, niet goed te kunnen functioneren en zichzelf
draaiende te houden, zelfs niet bij recordinkomsten. Net zoals de staat
de laat-industriële economieën domineerde -- waarbij in sommige
West-Europese landen meer dan de helft van alle inkomsten werd verbruikt
-- domineerde de kerk de laat-feodale economie, waarbij zij systematisch
middelen wegzapte en de economische groei belemmerde.

\subsection{Deficituitgaven in de vijftiende
eeuw}\label{deficituitgaven-in-de-vijftiende-eeuw}

De kerk zette alle mogelijke middelen in om meer geld van haar
volgelingen af te trogelen, zodat haar opgeblazen bureaucratie
gefinancierd kon worden. Regio's die rechtstreeks onder het gezag van de
kerk stonden, moesten steeds hogere belastingen betalen. In provincies
en koninkrijken waar de kerk geen eigen belastingbevoegdheid had, legde
het Vaticaan de zogenoemde `annaten' op -- een betaling die door de
lokale vorst in plaats van directe kerkbelasting moest worden voldaan.

Net als de moderne staat plunderde de kerk bovendien haar eigen
schatkisten door geld dat oorspronkelijk voor specifieke doeleinden
bestemd was, af te leiden om de vaste lasten te dekken. Kerkelijke
benefices en geldzuchtige religieuze ambten werden openlijk verhandeld,
evenals de opbrengsten uit de tienden. In feite fungeerden de
vorderingen op de tienden als het kerkelijke equivalent van obligaties
die moderne regeringen gebruiken om hun chronische tekorten te
financieren.

Hoewel de kerk optrad als ideologisch voorvechter van het feodalisme en
een felle criticus was van handel en kapitalisme, maakte zij -- net als
de hedendaagse natiestaat -- gebruik van iedere denkbare
marketingtechniek om haar inkomsten te maximaliseren. De kerk dreef een
bloeiende handel in sacramentelen, zoals gewijde kaarsen, op Palmzondag
gezegende palmtakken, kruiden die tijdens het feest van de
tenhemelopneming werden gezegend en in het bijzonder diverse soorten
heilig water.\footnote{Ibid., p.~15.}

Net zoals politici van tegenwoordig kiezers bedreigen met ingeperkte
afvalophaling en andere onaanvaardbare maatregelen als zij weigeren
hogere belastingen te betalen, waren religieuze autoriteiten in de
vijftiende eeuw evenzeer geneigd de erediensten stil te leggen en zo
gemeenten te chanteren tot het betalen van willekeurige boetes. Vaak
legden zij boetes op voor kleine overtredingen, begaan door enkelen die
zelfs geen lid hoefden te zijn van de betreffende gemeenschap. Zo sloot
in 1436 bisschop Jacques Du Chatelier -- `een zeer opzichtig,
hebzuchtige man' -- de Kerk der Onschuldigen in Parijs gedurende
tweeëntwintig dagen, waarbij hij alle religieuze diensten staakte
terwijl hij wachtte op een onvoorstelbaar hoge boete die door twee
bedelaars betaald moest worden. De mannen veroorzaakten in de kerk een
ruzie en lieten enkele druppels bloed vallen, waarvan de bisschop stelde
dat deze de kerk gedeconsecreerd hadden. Vervolgens verbood hij het
gebruik van de kerk voor huwelijken, begrafenissen en de reguliere
sacramenten totdat de boete volledig was voldaan.\footnote{Huizinga, op.
  cit., p.~27.}

\begin{quote}
De Italiaanse beheerders (om de paus in een goed humeur te brengen)\\
betaalden in één jaar twintigduizend ducatten.\\
Bovendien geven zij een priester (om zijn honorarium te verhogen)\\
de winst van een hoer, of twee of drie\ldots{}\\
Ik denk dat hij wel goddelijk moet zijn\\
dat hij met de beheerders zo'n affiniteit heeft.\footnote{Burford, op.
  cit., p.~103.}
\end{quote}

-- vijftiende-eeuwse Engelse ballade

\subsection{Haat jegens kerkelijke
leiders}\label{haat-jegens-kerkelijke-leiders}

Het is dan ook niet verrassend dat in de late vijftiende eeuw zowel de
hogere als de lagere geestelijkheid het mikpunt was van algemene afkeer,
vergelijkbaar met de hedendaagse verachting voor bureaucraten en
politici in sterk gepolitiseerde samenlevingen. Zoals Johan Huizinga het
verwoordde: `Haat is het juiste woord in deze context, want haat was er
-- latent, maar algemeen en hardnekkig. Het volk raakte nooit moe van
het horen aan de kaak stellen van de ondeugden van de
geestelijkheid.'\footnote{Huizinga, op. cit., p.~173.} Men was er immers
van overtuigd dat de kerk `hebzuchtig en extravagant' was, want dat
bleek daadwerkelijk zo te zijn. `De wereldlijkheid van de hogere
geestelijkheid en de verslechtering van de lagere geestelijkheid' liet
men immers niet aan zich voorbijgaan.\footnote{Ibid.} Van de
parochiepriester tot aan de paus doordrenkte corruptie de gehele
geestelijkheid, zoals vaak het geval is bij het personeel van een
dominante instelling.

Vijfhonderd jaar geleden liet paus Alexander~VI integriteit op zo'n
manier gelden dat zelfs figuren als Giuho Andreotti en Bill~Clinton als
toonbeelden zouden kunnen opstaan. Alexander~VI stond bekend om zijn
uitbundige feesten. Als kardinaal in Siena organiseerde hij een beroemde
orgie waartoe uitsluitend `de mooiste jonge vrouwen van Siena' werden
uitgenodigd, terwijl hun `echtgenoten, vaders en broers' werden
uitgesloten.\footnote{William Manchester, \emph{A world lit only by
  fire: the medieval mind and the renaissance} (Boston: Little, Brown,
  1992), pp.~75--76.} De orgie in Siena kreeg alom bekendheid, maar
verbleekte in vergelijking met de feesten die Alexander na zijn
aantreden als paus organiseerde. Wellicht was het zogenaamde
Kastanjeballet het meest opzienbarend, waarbij de `vijftig mooiste
hoeren' uit Rome meededen aan een copulatie-wedstrijd met de kerkvaders
en andere invloedrijke Romeinen. Zoals William Manchester het
beschrijft: `Dienstmannen hielden de telling bij van de orgasmes van
elke man, want de paus had een groot respect voor viriliteit\ldots{}
Nadat iedereen uitgeput was, deelde Zijn Heiligheid prijzen uit --
mantels, laarzen, petten en fijne zijden tunieken. De winnaars, schreef
de diarist, waren degenen die het grootste aantal malen liefde bedreven
met die courtisanen.'\footnote{Ibid., p.~79.}

Alexander verwekte minstens zeven, mogelijk acht, buitenechtelijke
kinderen. Een van zijn vermeende zonen, Giovanni, kreeg de bijnaam
`Infans Romanus' omdat hij werd verwekt uit de buitenechtelijke dochter
van Alexander, Lucrezia Borgia, toen zij achttien was. In een geheime
pauselijke bul erkende Alexander dat hij Giovanni als vader had. Als hij
niet als vader werd aangemerkt, trad hij in elk geval op als grootvader
aan beide zijden.

De paus raakte verstrikt in een drielinkig incestueus schandaal met
Lucrezia, die tevens de minnares was van Juan, hertog van Gandia -- de
oudste buitenechtelijke zoon van Alexander -- en van kardinaal Cesare
Borgia, een andere buitenechtelijke zoon. Cesare werd `de prins van de
kerk' genoemd en vormde de inspiratiebron voor Niccolò Machiavelli's
\emph{`The Prince'}. Cesare bleek een moordenaar te zijn, net als de
paus, die berucht was vanwege het beraamen van verschillende moorden.
Eén van deze moorden zou gepleegd zijn uit jaloezie op Juan, wiens
levenloze lichaam op 15 juni 1497 uit de Tiber werd
teruggevonden.\footnote{Ibid., pp.~82--84.}

Het leiderschap van de laatmiddeleeuwse kerk bleek net zo corrupt als
dat van de hedendaagse natiestaat.

\begin{quote}
`Vandaag ben ik twee keer vader geworden, Gods zegen
daarbij.'\footnote{Huizinga, op. cit., p.~154.} - Rodolph Acricola, toen
hij hoorde dat zijn minnares op de dag van zijn verkiezing tot abt een
zoon had gebaard.
\end{quote}

\section{Hypocrisie}\label{hypocrisie}

Verborgen achter een dunne laag vroomheid onthulde de laatmiddeleeuwse
samenleving haar opvallend blasfemische, onbeschaamde en losbandige
aard. Kerken dienden niet alleen als ontmoetingsplekken voor jonge
mannen en vrouwen, maar fungeerden ook als verzamelplaatsen voor
prostituees en handelaren in obsceen beeldmateriaal. Historici melden
dat `de onbeschaamde omgang met de dagelijkse religieuze praktijk
vrijwel geen grenzen kende.' Ingeschakelde koorzangers, die zongen voor
de zielen der doden, vervingen tijdens de mis regelmatig heilige woorden
door profane termen. Wachtrijen en processies, die destijds een veel
grotere rol speelden dan tegenwoordig, werden vaak overschaduwd door
grove humor, spot en uitbundig drankgebruik. Denis de Kartusiaan, een
vooraanstaand theoloog in het laatmiddeleeuwse Europa, verwoordde het
zo.\footnote{Ibid., p.~155.}

Hoewel kille moralisten zo'n verslag misschien als geklaag afdoen, komt
het beeld overeen met talloze andere getuigenissen. Het schunnige en het
heilige gingen in het middeleeuwse leven vaak hand in hand.
Pelgrimstochten ontspoorden regelmatig in rellen en uitbundige
losbandigheid, waaraan prominente hervormers tevergeefs probeerden in te
grijpen. Ook lokale religieuze processies boden vaak de gelegenheid voor
menigten om te vandaliseren, te plunderen en volop te genieten van
dronken capriolen. Zelfs tijdens de mis was de sfeer zelden sober; in de
kerk nuttigde men enorme hoeveelheden wijn, vooral op feestnachten.
Verslagen van de Raad van Straatsburg geven aan dat degenen die `in
gebed wachtten' op de nacht van Sint-Adolphus maar liefst 1.100 liter
wijn dronken, zoals de raad ter ere van de heilige beschikbaar stelde.

Jean Gerson, een toonaangevend theoloog uit de vijftiende eeuw,
verklaarde dat `de heiligste feesten, zelfs de kerstnacht' in
losbandigheid werden doorgebracht -- met kaartspelen, vloeken en
godslasteringen. Wanneer men gewone mensen berispte voor deze
misstappen, haalden zij aan dat ook de adel en de geestelijkheid zich op
dezelfde wijze ongestraft gedroegen.\footnote{Ibid.}

\subsection{Vroomheid en mededogen}\label{vroomheid-en-mededogen}

De vroomheid waarmee de georganiseerde religie in de late middeleeuwen
de samenleving doordrenkte, vervulde dezelfde functie als het
`mededogen' dat men tegenwoordig aanhaalt om politieke overheersing te
rechtvaardigen. De verkoop van aflaten, bedoeld om een hunkering naar
een vroomheid zonder moraal te bevredigen, lijkt op buitensporige
uitgaven aan sociale voorzieningen die slechts de schijn van mededogen
-- zonder oprechte liefdadigheid -- wegnemen. Het maakte nauwelijks uit
of deze maatregelen daadwerkelijk het morele karakter verbeterden of
zielen redden, net zoals het weinig uitmaakt of een welzijnsprogramma de
levens van de beoogde mensen daadwerkelijk verbetert. Zowel `vroomheid'
als `mededogen' klonken bijna als bijgelovige kreten.

In een tijd waarin men oorzakelijke verbanden nauwelijks inzag,
doordrongen de rituelen en sacramenten van de Kerk elke levensfase.
``\ldots{} Een reis, een taak, een bezoek werden evenzeer vergezeld door
duizend formaliteiten: zegeningen, ceremonies, formules.''\footnote{Ibid.,
  p.~9.} Mensen die aan koorts leden, schreven gebeden op perkament en
rijgden deze als een ketting aan elkaar. Ondervoede meisjes lieten
lokken uit hun haar hangen voor het beeld van Sint Urban, in de hoop
verdere haaruitval te voorkomen. Boeren in Navarra marcheerden in
processie achter een beeld van Sint Petrus om tijdens droogtes om regen
te smeken.\footnote{Deze voorbeelden van religieuze rituelen komen uit
  Cameron (op. cit., pp.~10--11).} Mensen namen deze en andere
`ineffectieve technieken' gretig over om hun angst te bedaren wanneer er
geen werkzame methoden beschikbaar waren.\footnote{Keith Thomas,
  \emph{Religion and the Decline of Magic} (London: Penguin, 1971),
  p.~800, geciteerd in Cameron (op. cit., p.~10).}

\subsection{Twee fouten maken een
ritueel}\label{twee-fouten-maken-een-ritueel}

Mensen geloofden zo sterk in de bijzondere krachten die zij aan de
relikwieën van heiligen toeschreven, dat de dood van een opvallend vrome
persoon al snel een dolle wedloop ontketende om zijn lichaam op te
delen. Toen Thomas van Aquino in het klooster Fossanuova overleed,
onthoofden de monniken zijn lichaam en kookten het om de controle over
zijn beenderen te grijpen. Toen Sint Elisabeth van Hongarije in algemene
uitvaart lag, stroomde een menigte aanbidders toe en rukten zij stroken
uit de linnen doek die haar gezicht omhulde; ze sneed zelfs haar haar,
nagels en tepels af.\footnote{Huizinga (op. cit., p.~161).}

\subsection{Vroomheid zonder deugd}\label{vroomheid-zonder-deugd}

De middeleeuwse denkwijze zag heiligen en hun relikwieën als onmisbare
wapens van het geloof in een wereld waarin de winter kil, de nachten
duister en de strijd tegen ziekte hopelozer leek dan welke lezer van dit
boek zich kan voorstellen. In die tijd waren mensen er stellig van
overtuigd dat demonen echt bestonden, dat God actief ingreep in de
wereld en dat gebed, boetedoening en bedevaarten hemelse gunst
opleverden.

Alleen het feit dat mensen in God geloofden, dekte niet de intensiteit
van hun toewijding en evenmin de schijnbaar eenvoudige manier waarop hun
vroomheid met zonde verstrengeld raakte. Men had zo'n groot vertrouwen
in de effectiviteit van rituelen, ceremonies en sacramenten dat zij het
belang van deugdzaam handelen onvermijdelijk teniet deden. Voor elke
zonde of geestelijke tekortkoming kende men een remedie, een
boetedoening die alles kon wissen -- wat men later de `wiskunde van de
verlossing' noemde.\footnote{Cameron (op. cit., p.~19).} De religie
doordrong het leven zo volledig dat haar oprechtheid geleidelijk begon
te vervagen. Zoals Huizinga het verwoordde: «Religie, die doordringt in
alle relaties in het leven, betekent een constante vermenging van het
heilige en het profane. Heilige zaken worden zo alledaags dat men ze
niet langer diep kan voelen.»\footnote{Huizinga, op. cit., p.~148.} En
zo was het.

\section{De verkleining van de kerk}\label{de-verkleining-van-de-kerk}

Aan het einde van de vijftiende eeuw kwam de kerk niet alleen even
corrupt over als de hedendaagse natiestaat, maar werkte ze ook als een
zware rem op de economische groei. De kerk vergaarde op onproductieve
wijze enorme hoeveelheden kapitaal en legde lasten op die zowel de
productie als de handel binnen de samenleving beperkten. Deze lasten,
vergelijkbaar met wat tegenwoordig door de natiestaat wordt opgelegd,
waren talrijk. We weten wat er met de georganiseerde religie gebeurde na
de buskruitrevolutie: die ontwikkelingen leidden tot krachtige prikkels
om religieuze instellingen te verkleinen en hun kosten te drukken. Toen
de traditionele kerk weigerde hiertoe over te gaan, grepen protestantse
sekten de kans om te concurreren. Daarbij gebruikten zij bijna elk
denkbaar middel om de kosten van een vroom bestaan te verlagen:

\begin{itemize}
\tightlist
\item
  Zij bouwden sobere nieuwe kerken en verwijderden soms de altaren uit
  oudere kerken, zodat kapitaal voor andere doeleinden vrijkwam.
\item
  Zij herformuleerden de christelijke leer op een wijze die de kosten
  deed dalen, doordat zij het geloof als sleutel tot verlossing vóór
  goede werken plaatsten.
\item
  Zij ontwikkelden een nieuwe, beknopte liturgie, schrapten of beperkten
  feestdagen en schafden talrijke sacramenten af.
\item
  Zij sloten kloosters en nonnenkloosters en stopten met het verstrekken
  van aalmoezen aan bedelaarsordes. Armoede veranderde daarmee van een
  apostolische deugd tot een ongewenst en vaak verwijtbaar sociaal
  probleem.\footnote{Voor meer details over de scherpe verschillen
    tussen vijftiende-eeuwse en zestiende-eeuwse perspectieven op
    armoede, zie Robert Jutte, \emph{Poverty and Deviance in Early
    Modern Europe} (Cambridge: \emph{Cambridge University Press}, 1994),
    pp.~15--17.}
\end{itemize}

Om te begrijpen hoe de afname van de invloed van de Kerk de
productiviteit bevrijdde, moeten we bekijken op welke wijze de Kerk de
groei blokkeerde voordat haar monopolie werd doorbroken. Net zoals de
moderne natiestaat dat vandaag de dag doet, drukte de Kerk aan het einde
van de vijftiende eeuw een enorme last op met buitensporige kosten.

\begin{enumerate}
\def\labelenumi{\arabic{enumi}.}
\item
  De Kerk legde directe lasten op -- zoals tienden, belastingen en
  heffingen -- die een opgeblazen kerkelijke bureaucratie voedden. Ook
  de protestantse kerken, die de middeleeuwse `Heilige Moederkerk'
  opvolgden, incasseerden tienden, maar in stedelijke gebieden bleken
  deze doorgaans niet inningbaar te zijn. Uiteindelijk zorgde het
  wegvallen van het kerkelijke monopolie voor dalende marginale
  belastingtarieven in regio's met een bloeiende handel.
\item
  Religieuze leerstellingen maakten sparen lastig. De aartsvijand van de
  middeleeuwse Kerk bleek de `gierigaard' te zijn -- iemand die zijn
  goud ophoopte ten koste van zijn ziel. Gelovigen moesten immers
  bijdragen aan het financieren van `goede daden', wat vaak kostbare
  giften inhield. Volgens de doctrine van `satisfacties waren zij die
  bezorgd waren om hun zaligheid verplicht om missen of 'gildenkapellen'
  in te wijden, zodat ze het vagevuur konden ontlopen. Luther bestookte
  dit onderwerp direct in de achtste en dertiende van zijn
  vijfennegentig stellingen. Hij schreef: `de stervenden zullen al hun
  schulden afbetalen door hun dood.'\footnote{Cameron, op. cit., p.~127.}
  Kortom, het kapitaal van de protestantse gelovige kwam beschikbaar om
  aan zijn erfgenamen over te dragen. Volgens de protestantse opvatting
  hoefden gelovigen geen gildenkapellen in te wijden voor het
  herhaaldelijk houden van missen -- meestal voor dertig jaar en, voor
  de zeer rijken, soms zelfs voor in het eeuwige.
\item
  De ideologie van de middeleeuwse Kerk leidde ertoe dat kapitaal werd
  afgeleid naar de aanschaf van relikwieën. Talrijke reliekenculten
  ontvingen flinke sommen geld om fysieke voorwerpen te bemachtigen die
  met Christus of diverse heiligen werden geassocieerd. De welgestelden
  stelden zelfs persoonlijke reliekencollecties samen. Kiesheer
  Frederick van Saksen verzamelde bijvoorbeeld negentienduizend
  relikwieën, waarvan hij er een deel tijdens een pelgrimstocht naar
  Jeruzalem in 1493 bemachtigde. Zijn collectie bevatte wat hij
  beschouwde als `het lichaam van een heilige onschuldige, de melk van
  Maria en stro uit de stal van de Kribbe.'\footnote{Ibid., p.~148.}
  Naar alle waarschijnlijkheid leverde de investering in deze relikwieën
  weinig rendement op. De verschuiving naar een nadruk op geloof en het
  begrip van de uitverkorenen deed afbreuk aan het belang van het
  vergaren van uiterlijke tekenen van het christelijk leven als talisman
  en stimuleerde kapitaal om te zoeken naar meer productieve
  investeringskanalen die wel het rendement opleverden waarop de vorst
  kon rekenen.
\item
  De opkomst van protestantse denominaties verbrak het economische
  monopolie van de middeleeuwse Kerk en verzwakte de regulering
  aanzienlijk. We weten dat men het kanonieke recht regelmatig aanpaste
  om de monopolistische en commerciële belangen van de Kerk te
  beschermen. Doordat de nieuwe denominaties minder economische belangen
  hoefden te behartigen, zorgde hun opvatting over de religieuze leer
  doorgaans voor een veel vrijer systeem, met minder belemmeringen voor
  de handel.
\item
  De protestantse revolutie schrapte talrijke rituelen en gebruiken van
  de middeleeuwse Kerk, die de gelovigen enorm veel tijd kostten. Aan
  het einde van de vijftiende eeuw waren rituelen, sacramenten en
  heilige dagen zo uitgebreid dat zij vrijwel het hele jaar in beslag
  werden genomen. Deze ceremonieel overbelaste planning vloeide voort
  uit de hardnekkige overtuiging van de Kerk dat `men zovaak als men
  wilde gebeds- of aanbiddingshandelingen kon vermenigvuldigen en er
  voordelen uit kon halen.'\footnote{Ibid., p.~11.} En vermenigvuldigen
  deden ze dan ook onvermoeibaar.
\end{enumerate}

Lange, uitgebreide diensten, de verplichting om herhalende boetedoening
in de vorm van gebeden te verrichten en de steeds talrijker wordende
festiviteiten ter ere van heiligen -- waar men geen arbeid mocht
verrichten -- belemmerden de productiviteit. De talloze voorschriften en
ceremonies bepaalden hoe men de dagen en seizoenen indeelde, waardoor de
tijd voor productieve werkzaamheden aanzienlijk werd ingeperkt. Dit had
wellicht weinig effect op het ritme in de middeleeuwse landbouw, waar
ruim 90 procent van de bevolking werkte, aangezien elk seizoen vele
perioden kende waarin dagelijkse veldarbeid niet noodzakelijk bleek. De
opbrengst van gewassen schommelde in middeleeuwse omstandigheden
vermoedelijk vooral door weersinvloeden en onvoorspelbare
plaaguitbarstingen, eerder dan dat marginale extra arbeid bovenop het
minimum, zoals bepaald door de kerkelijke kalender, enige invloed had.

Het verlies aan productiviteit werd in andere sectoren veel meer
voelbaar dan in de landbouw. De door de Kerk opgelegde tijdsdruk paste
immers veel minder bij ambachtelijk werk, productie, transport, handel
of andere ondernemingen waarin de bestede arbeidstijd direct invloed
heeft op productiviteit en winstgevendheid.

Het is dan ook geen toeval dat de ingrijpende transitie aan het einde
van de vijftiende eeuw plaatsvond in een periode waarin de grondprijzen
stegen en de reële lonen voor boeren daalden. De toenemende
bevolkingsdruk drukt de opbrengst van de gemeenschappelijke gronden --
vaak gelegen aan rivieren en beken, waar boeren hun vee lieten grazen en
deels ook rekenden op visvangst en brandhout -- op de proef. De dalende
levensstandaard dwong boeren steeds vaker alternatieve inkomstenbronnen
te zoeken, waardoor een groter deel van de landelijke bevolking zich tot
kleinschalige marktgerichte productie wendde, met name in textiel, in
een proces dat bekendstaat als `putting-out' of
`proto-industrialisatie.' De door de Kerk opgelegde ceremoniële
tijdsdruk belemmerde de ambitieuzere boeren om hun landbouwinkomen aan
te vullen met ambachtelijk werk en verhinderde een eventuele
verschuiving van arbeid naar nieuwe economische sectoren.

Een van de meest opvallende bijdragen van protestantse sekten aan de
productiviteit betrof het afschaffen van veertig feestdagen. Hiermee
bespaarde men niet alleen op de aanzienlijke kosten die gepaard gingen
met het organiseren van festiviteiten -- denk aan het royaal bevoorraden
van dorpskantines met eten en drinken -- maar kwam er ook een flinke
hoeveelheid waardevolle tijd vrij. Impliciet bespaarde iedere gelovige
die stopte met het vieren van deze veertig afgeschaft feestdagen
jaarlijks zo'n driehonderd arbeidsuren of meer, wat de basis legde voor
een merkbare toename van de productie door tijd vrij te maken die anders
verloren zou blijven gaan.

\begin{enumerate}
\def\labelenumi{\arabic{enumi}.}
\setcounter{enumi}{5}
\tightlist
\item
  De doorbraak in het kerkelijke monopolie heeft enorme hoeveelheden
  activa vrijgemaakt die, wanneer zij onder kerkelijk beheer stonden,
  slechts lage rendementen opleverden -- een situatie die duidelijke
  parallellen vertoont met staatsbezittingen aan het eind van de
  twintigste eeuw. De kerk was veruit de grootste feodale grondeigenaar.
  Haar greep op het land evenaalde die van de staat in de sterk
  gepolitiseerde samenlevingen van tegenwoordig -- in sommige Europese
  landen, zoals Bohemen, bezat zij meer dan vijftig procent van het
  totale grondbezit. Volgens canoniek recht mocht een eigendom, zodra
  het in handen van de kerk kwam, niet vervreemd worden. Dientengevolge
  groeide het bezit van kerkelijk land gestaag, omdat de kerk
  voortdurend meer testamentaire giften van gelovigen ontving ter
  financiering van diverse sociale welzijnsdiensten, kapellen en overige
  activiteiten.
\end{enumerate}

Hoewel het lastig is om de relatieve productiviteit van kerkelijke
bezittingen precies vast te stellen, was deze aan het einde van de
middeleeuwen vermoedelijk aanzienlijk lager dan in de vroege periode. In
de veertiende eeuw leidde de verschuiving naar marktgerichte productie
in plaats van zelfvoorzienende landbouw ertoe dat de meeste wereldlijke
heren hun onoplettende hoofdmannen verruilden voor professionele
beheerders, zodat zij de opbrengsten van hun bezittingen maximaal konden
benutten. Hun prikkels zorgden er vermoedelijk voor dat zij de
opbrengsten van kerkelijke eigendommen al snel overtroffen --
eigendommen die in principe niet bedoeld waren voor particuliere winst.
Natuurlijk beheerden sommige wereldse prins-bisschoppen hun domeinen op
een wijze nauwelijks te onderscheiden van die van de wereldlijke heren;
daarentegen kampte de productiviteit van andere kerkelijke bezittingen
ongetwijfeld met de gebreken van onverschillig bestuur door een grote,
verspreid opererende organisatie, met nadelen vergelijkbaar met die van
hedendaags staats- en gemeenschappelijk eigendom. Bovendien herverdeelde
de inbeslagname van kloosters onmiskenbaar middelen die, na de opkomst
van de boekdrukkunst, niet langer nodig waren voor het kopiëren van
boeken en manuscripten.

\begin{enumerate}
\def\labelenumi{\arabic{enumi}.}
\setcounter{enumi}{6}
\item
  Zoals we in \emph{The Great Reckoning} uiteenzetten, reageerden enkele
  protestantse sekten direct op de kruitrevolutie door hun doctrines
  zodanig aan te passen dat de handel werd gestimuleerd. Zo schafften
  zij bijvoorbeeld het verbod op `woekerrente' af, oftewel maakten zij
  het verlenen van leningen tegen rente mogelijk. De ideologische
  weerstand van de middeleeuwse kerk tegen het kapitalisme belemmerde de
  groei. De kern van de kerkelijke leer richtte zich namelijk op het in
  stand houden van het feodale systeem -- een systeem waarin de kerk als
  grootste feodale grondeigenaar een aanzienlijk belang had. Bewust of
  onbewust verhulde de kerk haar eigen economische belangen door deze
  als religieuze deugden uit te beelden, terwijl zij zich verzette tegen
  de ontwikkeling van de industrie en de opkomst van een onafhankelijke
  commerciële rijkdom die het feodale systeem onherroepelijk zou
  ondermijnen. Zo waren de verboden op `avance' met name bedoeld voor
  commerciële transacties in plaats van voor feodale heffingen en werden
  ze nooit toegepast op de verkoop van aflaten. Ook de beruchte pogingen
  van de kerk om een `billijke prijs' voor handelswaren vast te leggen,
  drukten de economische opbrengsten van producten en diensten waar de
  kerk zelf geen producent van was. Het verbod op `woekerrente'
  illustreerde treffend de afkeer van de kerk tegenover commerciële
  vernieuwing. Want bankieren en kredietverlening waren namelijk
  onontbeerlijk voor de ontwikkeling van grootschalige commerciële
  bedrijven, en door de toegankelijkheid van krediet te beperken,
  belemmerde de kerk de economische groei.
\item
  De nieuwe kerkgemeenschappen legden op subtiele wijze de nadruk op de
  Bijbel als bron, wat bijdroeg aan het afbreken van zowel de
  denkpatronen als de ideologie van de middeleeuwse kerk -- beide
  obstakels die groei in de weg stonden. De culturele conditionering van
  de late middeleeuwen spoorde mensen aan de wereld te bekijken in
  termen van symbolische overeenkomsten in plaats van oorzaak en gevolg.
  Dit maakte logisch redeneren moeizaam en weerhield hen ervan het leven
  op een zakelijke manier te benaderen, want symbolisch denken vertaalt
  zich immers niet eenvoudig naar marktwaarden. `De drie standen
  vertegenwoordigen de eigenschappen van de Maagd. De zeven kiesmannen
  van het Rijk staan voor de deugden; de vijf steden van Artois en
  Henegouwen, die in 1477 trouw bleven aan het huis van Bourgondië, zijn
  de vijf wijze maagden. . . . Op dezelfde wijze betekenen schoenen zorg
  en ijver, kousen volharding en de kousband vastberadenheid,
  enzovoorts.'\footnote{Huizinga, op. cit.} Zoals dit voorbeeld,
  geciteerd door de vooraanstaande middeleeuwse historicus Johan
  Huizinga, impliceert, werd het denken gedomineerd door dogma's, rigide
  symbolen en allegorieën die elk aspect van het leven in een
  hiërarchisch systeem plaatsten. Zo beschouwden mensen de dagelijkse
  facetten van het leven niet als gevolg van oorzaak en gevolg, maar als
  vaste symbolen en allegorieën. Vaak, wanneer deugden en ondeugden
  werden gepersonifieerd, stond elk object voor iets, wat op zijn beurt
  weer voor iets anders symboliseerde -- een opstelling die de
  oorzaak-gevolgrelatie eerder belemmerde dan verduidelijkte. Om het nog
  verwarrender te maken, koppelde men relaties vaak willekeurig in
  numerieke systemen. Hierbij speelde het getal zeven een bijzondere
  rol: er waren de zeven deugden, de zeven hoofdzonden, de zeven
  supplicaties van het Onzevader, de zeven gaven van de Heilige Geest,
  de zeven momenten van de passie, de zeven zaligsprekingen en de zeven
  sacramenten, `gerepresenteerd door de zeven dieren en gevolgd door de
  zeven ziekten.'\footnote{Ibid., p.~199.}
\end{enumerate}

\subsection{Vijftiende-eeuwse
journalistiek}\label{vijftiende-eeuwse-journalistiek}

Een nieuwsbericht uit de vijftiende eeuw -- als er al een geschreven zou
zijn -- beantwoordt de klassieke vragen over het rapporteren van feiten
slechts indirect, namelijk via allegorische personificatie. Bekijk
bijvoorbeeld dit verslag uit een privédagboek over de Bourgondische
moorden in het vijftiende-eeuwse Parijs:

\begin{quote}
Daarop kwam de godin van tweedracht op, die woonde in de toren van de
kwade raad, en wekte Woede, de gestoorde vrouw, en Hebzucht, en Razernij
en Wraak, en zij grepen allerlei wapens en verdreven met grote schande
Rede, Gerechtigheid, de Herinnering aan God en Matigheid. Toen raakte
Waanzin hen tot razernij, en Moord en Slachting doodden, hakten neer,
stelden ter dood en massaleerden alles wat zij in de gevangenissen
aantroffen. \ldots{} en Hebzucht stak haar rokken op in haar ceintuur
met Roofzucht, haar dochter, en Diefstal, haar zoon. \ldots{} Vervolgens
gingen de voornoemde mensen, geleid door hun godinnen, te weten Woede,
Hebzucht en Wraak, die hen door alle openbare gevangenissen van Parijs
leidden, enzovoort.\footnote{Ibid., p.~203.}
\end{quote}

De verschuiving weg van het middeleeuwse denken stimuleerde een
denkwijze in moderne termen, waarin oorzaak en gevolg centraal stonden
in plaats van symbolische verbanden en allegorische personificatie.

Men hoeft niet te bewijzen dat de leer en denkwijze van de
laatmiddeleeuwse Kerk eigenlijk niet oprecht waren; zij sloegen immers
perfect aan op de behoeften van het agrarische feodalisme, een tijd
waarin er nauwelijks ruimte was voor handel en om nog maar te zwijgen
van industriële ontwikkeling. Het ging er juist om dat de Kerk als
dominante instelling morele, culturele en juridische kaders vormgaf die
naadloos aansloten op de imperatieven van het feodalisme. Daardoor
bleken deze kaders ongeschikt voor een industriële samenleving, net
zoals de morele, culturele en juridische beperkingen van de moderne
natiestaat de handel in het informatietijdperk niet ten goede komen. Wij
zijn ervan overtuigd dat de staat -- net als de Kerk -- een radicale
transformatie zal ondergaan om het nieuwe potentieel te benutten.

De protestantse leer stelde dat men het paradijs uitsluitend door geloof
kon bereiken en dat men daarvoor geen gebeden voor de doden nodig had.
Hiermee plaatste men de kwestie in een theologisch kader. Tegelijkertijd
sloot deze benadering perfect aan bij de economische realiteiten van een
nieuw tijdperk. Ze bood een duidelijke en kostenefficiënte weg naar
verlossing, juist op het moment dat de opportuniteitskosten voor extra
investeringen in de opgeblazen kerkelijke bureaucratie plotseling
stegen. Mensen waren bereid hun geld aan de Kerk af te staan als er geen
andere keuze was. Maar zodra zij de kans kregen hun kapitaal
honderdvoudig te vermeerderen door een specerijenreis naar het Oosten te
bekostigen of een rendement van 40 procent per jaar te behalen door een
bataljon voor de koning te ondersteunen, zochten zij -- terecht -- de
genade van God, omdat dit naadloos aansloot bij hun eigen belangen.

Veel kooplieden en andere gewone mensen vergaarden al snel aanzienlijk
meer rijkdom dan hun voorouders onder het feodalisme hadden bereikt. De
scherpe toename van de levensstandaard bij kooplieden en kleinschalige
producenten stuitte op felle afkeur bij degenen wier inkomen en
levensstijl stevig verankerd waren in het feodale systeem. De
verzwakking van het kerkelijke monopolie en de groeiende macht van de
vermogenden leidden tot een drastische afname van inkomensherverdeling.
Boeren en stedelijke armen, die niet direct konden profiteren van het
nieuwe systeem, voelden een bittere jaloezie jegens degenen die dat wel
deden. Huizinga omschreef de heersende gevoelens -- wat een interessante
parallel vormt met de informatierévolutie -- als: `Haat jegens rijke
mensen, en in het bijzonder de nieuwe rijken -- die toen zeer talrijk
waren -- is algemeen.'\footnote{Ibid., p.~27.}

Een even opvallende parallel kwam tot uiting in de enorme stijging van
de criminaliteit. Wanneer oude ordes instortten, ontstond vrijwel altijd
een golf van misdrijven -- zo zag men ook dat tijdens de feodale
revolutie, zoals we in het voorgaande hoofdstuk bespraken. Aan het einde
van de middeleeuwen steeg de criminaliteit fors naarmate de traditionele
systemen van sociale controle uiteenvielen. Huizinga verwoordde het als
volgt: `{[}C{]}riminaliteit werd als een bedreiging voor orde en
samenleving beschouwd.'\footnote{Ibid., p.~22.} Ook in de toekomst kan
dit net zo zorgwekkend blijken.

De moderne wereld ontstond te midden van de verwarring over nieuwe
technologieën, baanbrekende ideeën en de stank van kruit. Buskruitwapens
en de vooruitgang in scheepvaart verzwakten de militaire fundamenten van
het feodale systeem, terwijl opkomende communicatietechnologieën diens
ideologie ondermijnden. Dankzij de nieuwe druktechniek kwamen de
corruptie binnen de Kerk steeds meer aan het licht; zowel de hiërarchie
als het gewone volk kwamen hierdoor in een negatief daglicht te staan,
temidden van een samenleving die op een paradoxale wijze religie
centraal stelde. Deze paradox vertoont duidelijke parallellen met de
hedendaagse desillusie jegens politici en bureaucraten in een
maatschappij waarin de politiek de boventoon voert.

Het einde van de vijftiende eeuw kenmerkte zich door ontgoocheling,
verwarring, pessimisme en wanhoop -- een periode die opvallend doet
denken aan onze eigen tijd.

De inhoud van het te vertalen boek is als volgt:

\bookmarksetup{startatroot}

\chapter{Het leven en de gezondheid van de
natiestaat}\label{het-leven-en-de-gezondheid-van-de-natiestaat}

\begin{quote}
`Het allerbelangrijkste is dat je in oorlog pas succesvol bent als je
over voldoende geld beschikt om in alle benodigdheden te voorzien.'
{[}\^{}163{]} -- Robert de Balsac, 1502
\end{quote}

\section{Het puin van de
geschiedenis}\label{het-puin-van-de-geschiedenis}

Op 9 en 10 november 1989 zonden televisiezenders beelden naar de hele
wereld, waarop uitbundige Oost-Berliners de Berlijnse Muur met
sloophamers afbraken. Onder de menigte pakten veelbelovende ondernemers
stukken van de muur op, die later als souvenir-papiergewichten aan
kapitalisten in alle windstreken werden verhandeld. In de daaropvolgende
jaren bloeide de handel in deze relikwieën op. Zelfs nu we dit
schrijven, kom je nog incidenteel advertenties tegen in kleine
tijdschriften, waarin stukjes oud Oost-Duits beton worden aangeboden
voor prijzen die normaal gesproken voor hoogwaardig zilvererts gelden.
Wij zijn ervan overtuigd dat wie de souvenir-papiergewichten van de
Berlijnse Muur in bezit heeft, ze niet snel verkoopt. Zij bezitten
immers een aandenken aan iets dat veel groter is dan de val van het
communisme. Wij denken dat de Berlijnse Muur het meest betekenisvolle
historische puin is geworden sinds de muren van San Giovanni in februari
1495, die bijna vijf eeuwen geleden tot puin werden
gesloopt.{[}\^{}164{]}

Het platmaken van San Giovanni door de Franse koning Karel~VIII
markeerde de eerste uitbarsting van de buskruitrevolutie. Dit betekende
het einde van de feodale periode en het begin van het industrialisme,
zoals we eerder al bespraken. De val van de Berlijnse Muur vormt een
andere historische breuklijn: de overgang van het industriële tijdperk
naar het nieuwe informatietijdperk. Nooit eerder heeft efficiëntie zo
krachtig de macht overwonnen. Toen de muren van San Giovanni vielen,
werd overduidelijk dat de economische opbrengsten van geweld sterk waren
toegenomen. De val van de Berlijnse Muur suggereert echter dat deze
opbrengsten nu afnemen. Weinigen lijken zich hiervan bewust te zijn,
maar het zal dramatische gevolgen hebben.

Om redenen die we in dit hoofdstuk bespreken, kan de Berlijnse Muur veel
meer symbool staan voor het tijdperk van de industriële natiestaat dan
wel de menigte die die avond in Berlijn aanwezig was of de miljoenen
toeschouwers van verre afstand. De Berlijnse Muur werd met een totaal
ander doel gebouwd dan de muren van San Giovanni -- zij moest voorkomen
dat mensen van binnenuit vluchtten, in plaats van indringende roofdieren
van buitenaf tegen te houden. Dit feit wijst al duidelijk op de
groeiende macht van de staat van de vijftiende tot de twintigste eeuw.
En dat op meer dan één manier.

Eeuwenlang maakte de natiestaat alle externe muren overbodig. In de
gebieden waar zij voor het eerst voet aan de grond zette, oefende de
staat een monopolie uit over dwang. Daardoor ontstonden intern
vreedzamere omstandigheden en ontstond er militair gezien een
indrukwekkender geheel dan wat eerdere soevereine staten ooit hadden
gekend. De staat gebruikte de middelen die zij verkreeg door de
ontwapening van de bevolking om kleinschalige plunderaars te
verpletteren. De natiestaat werd zo het meest succesvolle instrument in
de geschiedenis om hulpbronnen te bemachtigen. Haar succes berustte op
een superieur vermogen om de rijkdom van haar burgers af te persen.

\emph{MTV} is meer dan alleen een aanbieder van muziekvideo's en
fungeert tevens als promotioneel instrument voor de muziekindustrie. Het
markeert het eerste echte wereldwijde netwerk en zendt in bijna elk land
een uniforme stroom programma's uit. Hierdoor zorgt \emph{MTV} ervoor
dat kinderen, jongvolwassenen en andere kijkers een gedeelde mondiale
realiteit ervaren. Recent onderzoek wijst uit dat jongeren over de hele
planeet niet alleen overeenstemming vinden over popiconen en
muzieksmaken, maar ook vergelijkbare verwachtingen hebben ten opzichte
van hun carrière, dezelfde waarden delen over wat in het leven
betekenisvol is en waar men bang voor moet zijn, en beseffen dat
politiek minder belangrijk is dan hun vermogen om hun toekomst vorm te
geven.{[}\^{}165{]} -- Jim Taylor and Watts Wacker, The 500-Year Delta:
What Happens After What Comes Next

\subsection{`Hou ervan of verlaat het' (tenzij je rijk
bent)}\label{hou-ervan-of-verlaat-het-tenzij-je-rijk-bent}

Voordat de overgang van de natiestaat naar de nieuwe soevereiniteiten
van het informatietijdperk is afgerond, gaan veel inwoners van de
grootste en meest invloedrijke westerse landen -- net als hun
tegenhangers in Oost-Berlijn in 1989 -- plannen maken om hun land te
verlaten. Voor generaties die volwassen werden vóór de Tweede
Wereldoorlog of in de vroege jaren van de Koude Oorlog betekende
emigreren vaak een traumatische ervaring. Voor de nieuwe generaties, die
een mondiaal referentiekader hanteren, voelt het vertrek uit het
geboorteland niet als een ondenkbare optie, in tegenstelling tot ouderen
die diep in de ideologie van de natiestaat verankerd zitten.

Jim Taylor en Watts Wacker delen de fascinerende resultaten van een
grootschalige enquête onder 20.000 middelbare scholieren uit de
middenklasse op vijf continenten. Uit een onderzoek, uitgevoerd in het
schooljaar 1995--96 door \emph{Brainwaves Group}, een
consumentenonderzoeksbureau uit New York, bleek dat negen op de tien
leerlingen instemden met de uitspraak: `Het is aan mij om te bepalen wat
ik uit het leven wil halen.' Nog opvallender: bijna de helft van de
tieners verklaarde dat zij verwachtten hun geboorteland te verlaten om
hun doelen na te streven.\footnote{Ibid., p.~39.}

Misschien omdat hij nauw aansluit bij de opvattingen van de
\emph{MTV}-generatie, probeerde Bill Clinton -- als de eerste
presidentskandidaat die campagne voerde op \emph{MTV} -- het voor
Amerikanen moeilijker te maken om hun geboorteland te verlaten om hun
ambities na te streven. In 1995, ongeveer tegelijk met het moment waarop
de middelbare scholieren hun wens tot onafhankelijkheid kenbaar maakten,
stelde de Amerikaanse president de invoering van een exitbelasting voor,
een soort `Berlijnse muur voor kapitaal', die rijke Amerikanen zou
verplichten een aanzienlijk losgeld te betalen om zelfs een deel van hun
vermogen mee naar het buitenland te nemen.

Clintons losgeld doet niet alleen denken aan het beleid van de
voormalige Oost-Duitse staat, die haar burgers als bezittingen
beschouwde; het roept ook beelden op van steeds strenger wordende
maatregelen om de fiscale positie van het vervalende Romeinse Rijk te
versterken. Het volgende fragment uit \emph{The Cambridge Ancient
History} illustreert dit verhaal:

\begin{quote}
Zo begon de staat met een meedogenloze poging om de bevolking tot op de
laatste druppel uit te persen. Doordat de economische middelen schaars
waren, slaagden de machtigen er met grof geweld in het grootste deel
voor zichzelf toe te eigenen -- een aanpak die nauw aansloot bij de
oorsprong van de machthebbers en bij soldaten die gewend waren aan
plundering. De wet werd zonder pardon ingezet tegen de bevolking.
Soldaten fungeerden als deurwaarders of zwierven als geheime politie
door het land. Vooral de vermogende klasse leed het zwaar; men wist hun
bezit relatief eenvoudig in handen te krijgen en in noodsituaties werd
van hen het meeste en het snelst afgetapt.\footnote{The Cambridge
  Ancient History, op. cit., pp.~263--264.}
\end{quote}

Wanneer falende systemen daartoe in staat zijn, leggen ze vaak zware
sancties op degenen die proberen te ontsnappen. We citeren opnieuw
\emph{The Cambridge Ancient History}: ``Als de vermogende klasse haar
geld verstopte, of twee derde van haar bezit opofferde om een
magistratuur te ontvluchten, of zelfs haar gehele eigendom opgaf om
verlost te worden van de erfpacht, en de niet-vermogende klasse
vluchtte, reageerde de staat door de druk te verhogen.''

Dit is iets wat we niet mogen vergeten bij het plannen van de toekomst.
Staatsvallen uit het verleden verliepen zelden op een beschaafde en
ordelijke wijze. In hoofdstuk 2 bespraken we al de onprettige
handelswijzen van de Romeinse belastinginners. Het enorme aantal agri
deserti -- oftewel verlaten boerderijen -- in West-Europa na de val van
het Romeinse Rijk toonde slechts een deel van een groter probleem. In
Gallië en de aangrenzende gebieden, waaronder het huidige Luxemburg en
Duitsland, waren de buitensporige heffingen relatief mild. In Egypte, de
vruchtbaarste regio van Rome waar irrigatie de landbouwproductiviteit
stimuleerde, zorgde de caos die landeigenaren ondervonden voor nog
grotere problemen. Voor vrijwel iedereen met bezit stond dan ook de
vraag centraal of men moest proberen te ontsnappen -- wat in het Latijn
als het ultimum refugium werd aangeduid. Bronnen geven aan dat men aan
een Egyptisch orakel vrijwel standaard drie vragen voorlegde: ``Zal ik
een bedelaar worden?'', ``Zal ik vluchten?'' en ``Zal mijn vlucht worden
gestopt?''\footnote{Cook et al., op. cit., p.~268.}

Clintons voorstel beantwoordt die vragen bevestigend. Het betreft immers
een vroege vorm van een ontsnappingsdrempel die naar verwachting
zwaarder wordt naarmate de fiscale middelen van de natiestaat krimpen.
Natuurlijk is de eerste Amerikaanse variant van zo'n vertrekbarrière
veel minder ingrijpend dan het beton en de prikkeldraad van Erich
Honecker. Daarbij speelt prijssensitiviteit een grotere rol, waardoor de
belastinglast uitsluitend op `miljardairs' met een belastbare erfenis
van meer dan \$600.000 komt te rusten. Desalniettemin worden de
argumenten geteld die sterk lijken op die van Honecker, die ooit het
beroemdste openbare werkproject van de inmiddels verdwenen Duitse
Democratische Republiek verdedigde. Honecker betoogde dat de Oost-Duitse
staat fors had geïnvesteerd in potentiële vluchtelingen en dat het
toelaten van deze mensen een economisch nadeel voor de staat zou
veroorzaken -- de staat rekende immers op hun inzet in Oost-Duitsland.

Als men aanneemt dat mensen als activa van de staat worden gezien of
zouden moeten worden beschouwd, dan blijkt Honeckers muur logisch. Een
muurloos Berlijn bood de communisten een ontsnappingsmogelijkheid, net
zoals het ontwijken van de Amerikaanse belastingjurisdictie voor
Clintons \emph{IRS} een terugweg vormde. Clintons redeneringen over
vluchttende miljardairs -- afgezien van de gebruikelijke minachting van
politici voor de betrouwbaarheid van de cijfers -- lijken sterk op die
van Honecker, maar vallen net wat tegen wanneer je bedenkt dat de
Amerikaanse overheid feitelijk geen grote economische investering heeft
gedaan in rijke burgers die eventueel zouden willen vluchten. Het draait
niet om het feit dat zij op staatskosten onderwijs genoten en daarna
stilletjes de advocatuur ingingen, want de overgrote meerderheid van
degenen op wie de vertrekbelasting zou worden geheven, heeft hun fortuin
eigenhandig vergaard -- en niet dankzij de staat.

Nu het bovenste 1 procent van de belastingbetalers inmiddels 28,7
procent van de totale inkomstenbelasting in de Verenigde Staten betaalt,
is het ten volle haalbaar dat rijke burgers de eventuele daadwerkelijke
investering van de staat in hun opleiding en economische voorspoed
daadwerkelijk terugbetalen. Integendeel, wie de meeste kosten draagt,
betaalt ruimschoots meer dan zij aan voordelen ontvangen. Met een
gemiddelde jaarlijkse belasting van meer dan \$125.000 valt het top
1-procent in de VS veel meer uit dan zij zich realiseren. Stel dat zij
een rendement van 10 procent behalen op die extra betaalde belasting
over een periode van veertig jaar; dan zorgt elke jaarlijkse
overschrijding van \$5.000 voor een daling van hun netto vermogen met
\$2,2 miljoen. Bij een rendement van 20 procent betekent elke extra
\$5.000 belastingdat het netto vermogen met maar liefst \$44 miljoen
afneemt.

Naarmate we richting het nieuwe millennium gaan, wordt onder de huidige
megapolitieke voorwaarden van het informatietijdperk steeds duidelijker
dat de natiestaat -- zoals we die geërfd hebben uit het industriële
tijdperk -- eigenlijk een uitbuitende entiteit is. Elk jaar komt zij
minder als zegen voor de welvaart over en steeds meer als een hindernis
waartegen mensen willen zich verzetten. Een verzet dat radeloze
regeringen maar al te onwillig zullen toestaan. De stabiliteit en zelfs
het voortbestaan van de westerse verzorgingsstaten berust op hun
vermogen om een aanzienlijk deel van de wereldwijde productie af te
tappen en dit her te verdelen onder een specifieke groep kiezers binnen
de OESO-landen. Dit betekent dat de meest productieve burgers in de
huidige welvarende landen belast moeten worden tegen buitensporige
tarieven -- honderden of zelfs duizenden malen hoger dan de werkelijke
kosten van de diensten die de overheid daarvoor levert.

\section{The life and death of the
nationstate}\label{the-life-and-death-of-the-nationstate}

De val van de Berlijnse Muur was meer dan slechts een zichtbaar symbool
van het einde van het communisme. Het betekende een nederlaag voor het
mondiale systeem van natiestaten en een overwinning voor efficiëntie en
marktwerking. De machtsbasis die eens de geschiedenis bepaalde, is
ingrijpend verschoven. Wij zijn ervan overtuigd dat de val van de
Berlijnse Muur in 1989 het hoogtepunt betekende van het tijdperk van de
natiestaat -- een merkwaardige tweehonderdjarige periode die begon met
de Franse Revolutie. Staten bestaan al meer dan zesduizend jaar, maar
vóór de negentiende eeuw vertegenwoordigden zij slechts een klein deel
van de wereldwijde soevereine entiteiten. Hun opkomst begon en eindigde
met revoluties. De ingrijpende gebeurtenissen van 1789 stuurden Europa
de richting op van werkelijk nationale regeringen, terwijl de
gebeurtenissen van 1989 niet alleen het communisme deden instorten, maar
ook de voorrang van marktmechanismen op geconcentreerde macht
bevestigden. Deze twee revoluties, precies tweehonderd jaar van elkaar
verwijderd, definiëren het tijdperk waarin de natiestaat de dominante
kracht was binnen het systeem van grote mogendheden, die op hun beurt de
wereld beheersten door staatsstructuren te verspreiden of op te leggen,
zelfs aan de meest afgelegen tribale gemeenschappen.

De overwinning van de staat als het belangrijkste middel om geweld te
organiseren in de wereld kwam niet voort uit ideologie, maar werd
gedreven door de harde logica van geweld. Zoals wij het graag
verwoorden: het betrof een megapolitieke gebeurtenis, gevormd door de
onderliggende hefboomwerking van geweld -- niet door de idealen van
denkers, politici of het manoeuvreren van generaals -- op precies de
wijze waarop Archimedes had gedroomd de wereld in beweging te brengen.

Hoewel staten gedurende de afgelopen tweehonderd jaar van de moderne
periode de norm vormden, waren zij in het bredere historisch perspectief
zeldzaam. Ze konden immers alleen bestaan onder uitzonderlijke
megapolitieke omstandigheden. Vóór de moderne tijd waren de meeste
staten zogenoemde `oriëntaalse despotieën': agrarische samenlevingen in
woestijngebieden die afhangen van de controle over irrigatiesystemen om
te overleven. Zelfs het Romeinse Rijk, dat Egypte en Noord-Afrika
beheerste, functioneerde in feite als een hydraulische samenleving --
maar niet in de mate die nodig was om lang te overleven. Rome, net als
de meeste premoderne staten, miste uiteindelijk het vermogen om het
geweldmonopolie af te dwingen dat ervoor zorgt dat mensen niet
verhongeren. Zo kon de Romeinse staat buiten Afrika het water voor de
gewassen niet afsluiten voor degenen die ongehoorzaam waren en daardoor
de toegang tot het irrigatiesysteem ontzogen. Dergelijke hydraulische
systemen boden een veel grotere hefboomwerking voor geweld dan welke
andere megapolitieke opstelling in de oude economie dan ook. Degene die
in deze samenlevingen het beheer over het water had, kon inkomsten
vergaren ter hoogte van een percentage dat bijna even hoog is als het
aandeel van de totale productie dat moderne natiestaten innen.

\subsection{Grootte boven
efficiëntie}\label{grootte-boven-efficiuxebntie}

Buskruit maakte het voor staten eenvoudiger om zich uit te breiden
voorbij de grenzen van rijstvelden en droge rivierdalen. De aard van
buskruitwapens en de opbouw van de industriële economie brachten flinke
schaalvoordelen in de oorlogsvoering met zich mee. Dit leidde tot hoge
én steeds verder toenemende uitkomsten van geweld. Zoals historicus
Charles Tilly het verwoordde: `{[}S{]}taten die de grootste
dwangmiddelen hadden, wonnen de oorlogen; efficiëntie (de verhouding
tussen output en input) kwam op de tweede plaats na effectiviteit
(totale output).'{[}ˆ170{]} Aangezien regeringen zich vrijwel altijd op
grote schaal organiseerden, hadden zelfs de weinige kleine
soevereiniteiten -- zoals Monaco of Andorra -- de erkenning van de
grotere staten nodig om hun onafhankelijkheid te waarborgen. Alleen
grote regeringen met een steeds grotere controle over middelen konden op
het slagveld concurreren.

\subsection{De grote onbeantwoorde
vraag}\label{de-grote-onbeantwoorde-vraag}

Dit zet ons voor een van de grote onopgeloste raadsels uit de moderne
geschiedenis: waarom plaatste de Koude Oorlog -- die volgde op het
tijdperk van grootmachten -- uiteindelijk communistische dictaturen
tegenover verzorgingsdemocratieën? Het vraagstuk is zo weinig onderzocht
dat velen het aannamen toen Francis Fukuyama, analist bij het ministerie
van Buitenlandse Zaken, na de val van de Berlijnse Muur `het einde van
de geschiedenis' uitriep. Het enthousiaste publiek nam het te veel voor
lief. Blijkbaar stelden noch de auteur, noch velen anderen de
fundamentele vraag: welke gemeenschappelijke kenmerken van
staatssocialisme en verzorgingsdemocratieën zorgden ervoor dat zij
uiteindelijk elkaars ultieme rivalen in de strijd om wereldmacht werden?

Dit is een belangrijk vraagstuk, want in de afgelopen vijf eeuwen kwamen
talloze vormen van soevereiniteit op en verdwenen ze weer, variërend van
absolute monarchieën en tribale enclaves tot prins-bisdommen, directe
pauselijke heerschappij, sultanaten, stadsstaten en anabaptistische
kolonies. Tegenwoordig verbaast het de meeste mens te horen dat een
ziekenhuisbeheermaatschappij -- gewapend met eigen strijdkrachten --
ooit een land kon besturen. En toch gebeurde er iets dergelijks.
Driehonderd jaar na 1228 regeerden de teutoonse ridders van
\emph{St.~Mary's Hospital} in Jeruzalem -- later verenigd met de ridders
van het Zwaard van Livonië -- over Oost-Pruisen en diverse gebieden in
Oost-Europa, zoals delen van Litouwen en Polen. Toen brak de
buskruitrevolutie aan. Binnen enkele decennia verdreven de teutoonse
ridders als soevereine heersers uit al hun gebieden, en verloor hun
grootmeester zijn militaire betekenis, nauwelijks meer dan die van een
schaakkampioen. Waarom? Waarom raakten zoveel andere systemen van
soevereiniteit tot onbeduidendheid teruggedrongen, terwijl aan het einde
van het industriële tijdperk de grote strijd om wereldmacht
massademocratieën van de verzorgingsstaat tegenover staatssocialistische
systemen plaatste?

\subsection{Onbelemmerde controle}\label{onbelemmerde-controle}

Als onze theorie over megapolitiek klopt, wordt het antwoord simpel. Het
doet denken aan de vraag waarom sumoworstelaars vaak zwaar gebouwd zijn.
Een slanke sumoworstelaar, hoe indrukwekkend ook zijn
kracht-tot-gewichtverhouding, kan simpelweg niet op tegen een massieve
tegenstander. Zoals Tilly aangeeft, draait het hier niet om efficiëntie
-- de verhouding tussen input en output -- maar om effectiviteit,
oftewel de totale productie. In een wereld waarin geweld steeds vaker de
norm blijkt te zijn, domineerden de systemen die vijf eeuwen lang aan de
macht waren, omdat zij de beste toegang boden tot de middelen die nodig
waren om op grote schaal oorlog te voeren.

Hoe werkte dit?

Bij het communisme is het antwoord overduidelijk. De
staatsfunctiehebbers beheersten vrijwel alles. Als je tijdens de Koude
Oorlog in de Sovjet-Unie woonde, hadden de KGB-agenten zelfs jouw
tandenborstel kunnen afnemen als dat in hun belang was -- ze hadden
zelfs jouw tanden kunnen wegnemen. Betrouwbare schattingen, die extra
gewicht kregen na de opening van de voormalige Sovjetarchieven in 1992,
geven aan dat de geheime politie en andere agenten van de voormalige
Sovjetstaat in 74 jaar bewind zo'n 50 miljoen levens hebben weggenomen.
Het staatssocialistische systeem wist alle middelen binnen zijn grenzen
te mobiliseren voor het leger, zonder dat bijna iemand zich ertegen
verzette.

Bij de westerse democratieën is het verhaal minder eenduidig, mede
doordat we democratie vaak als tegenovergestelde van het communisme
zien. Als we terugkijken op de industriële revolutie, leken beide
systemen inderdaad enorme tegenpolen te zijn. Maar vanuit het
perspectief van het informatietijdperk blijken de twee systemen meer
gemeen te hebben dan je zou vermoeden. Beide gaven de overheid namelijk
de mogelijkheid om zonder belemmering over de hulpbronnen te beschikken.
Het verschil is dat in de democratische verzorgingsstaten de overheid
nog meer middelen in handen kreeg dan in de staatssocialistische
systemen.

Dit is een duidelijk voorbeeld van een bijzonder verschijnsel, waarbij
minder soms meer blijkt te zijn. Het staatssocialistische systeem
baseerde zich op de opvatting dat de staat alles in bezit had.
Daarentegen stelden de democratische verzorgingsstaten minder ambitieuze
eisen en creëerden ze daardoor effectievere prikkels om de productie te
verhogen. In plaats van vanaf het begin alles te claimen, gaven de
westerse regeringen burgers de kans om privébezit te verwerven en zo
vermogen op te bouwen. Daarna onttrokken de westerse staten een
aanzienlijk deel van dat rijkdom via belastingen. Hoge vastgoed-,
inkomsten- en erfbelastingen voorzagen de democratische verzorgingsstaat
van enorme financiële middelen vergeleken met wat beschikbaar was in de
staatssocialistische systemen.

\subsection{Inefficiëntie, waar het ertoe
deed}\label{inefficiuxebntie-waar-het-ertoe-deed}

Vergeleken met het communisme werkte de welvaartsstaat inderdaad veel
efficiënter. Echter, als je deze vergelijkt met andere systemen voor het
vergaren van rijkdom -- zoals een echte laissez‑faire‑enclave als
\emph{Hongkong} -- bleek de welvaartsstaat op dat gebied minder
efficiënt te opereren. Opnieuw werd duidelijk: minder kan meer
betekenen. Juist deze ogenschijnlijke inefficiëntie stelde de
welvaartsstaat tijdens de megapolitieke periode van het industriële
tijdperk in staat zich te onderscheiden.

Als je eenmaal begrijpt waarom dit zo is, kom je dichter bij het
doorgronden van wat de val van de Berlijnse Muur en het verdwijnen van
het communisme werkelijk inhouden. In plaats van aan te tonen dat de
democratische welvaartsstaat een triomfantelijk systeem vormt, zoals
velen beweren, lijkt het eerder alsof je constateert dat je
tweelingbroer op natuurlijke wijze aan ouderdom is gestorven. Dezelfde
megapolitieke revolutie die het communisme omverwierp, zal
hoogstwaarschijnlijk ook de democratische welvaartsstaten, zoals we die
in de twintigste eeuw kenden, ondermijnen en ten val brengen.

\section{Wie controleert de
overheid?}\label{wie-controleert-de-overheid}

De kern van deze onconventionele conclusie schuilt in het inzicht waar
de macht over de democratische staat werkelijk rust. Dit vraagstuk
blijkt veel ingewikkelder dan het op het eerste gezicht lijkt.
Tegenwoordig benadert men de vraag wie de overheid bestuurt doorgaans
als een zuiver politieke aangelegenheid. Hoewel hierover al talloze
antwoorden zijn gegeven, wordt er vrijwel uitsluitend gewezen naar de
politieke partij, groep of factie die op een bepaald moment de touwtjes
in handen heeft bij een specifieke staat. Je hebt vast wel gehoord van
regeringen die in handen zijn van kapitalisten, maar ook van overheden
die door arbeiders worden geleid. Misschien ken je ook die verhalen over
staten waar katholieken of islamitische fundamentalisten de controle
hebben, of waar stam- en etnische groepen de machtspositie bezitten --
denk bijvoorbeeld aan regeringen die door Hutu's of blanken worden
bestuurd. Daarnaast hoor je vaak dat beroepsgroepen -- zoals advocaten
of bankiers -- het roer in handen hebben, en dat landelijke belangen, de
politieke machines van grote steden of de inwoners van voorsteden een
sturende rol spelen. Uiteraard ken je ook oppassingen waarbij politieke
partijen -- of het nu om democraten, conservatieven, christendemocraten,
liberalen, radicalen, republikeinen of socialisten gaat -- de macht
uitoefenen.

Maar je hebt waarschijnlijk nog weinig gehoord over het idee dat een
overheid door haar klanten wordt aangestuurd. De economische historicus
Frederic Lane legde de basis voor een alternatieve kijk op de
machtverhoudingen binnen de overheid, zoals hij in enkele duidelijke
essays over de economische gevolgen van geweld uiteenzette. Door de
overheid te beschouwen als een economische entiteit die bescherming
levert, besloot Lane de controle niet in politieke, maar in economische
termen te analyseren. Volgens deze benadering bestaan er drie
fundamentele alternatieven voor wie de macht in de overheid uitoefent --
elk met een eigen prikkelsysteem: eigenaren, werknemers en klanten.

\subsection{Eigenaren}\label{eigenaren}

In zeldzame gevallen, zelfs in de huidige tijd, leidt één eigenaar een
regering -- meestal een erfelijke heerser die feitelijk het land bezit.
Zo beschouwt de sultan van Brunei de regering van Brunei deels als een
privébezit. In de middeleeuwen kwam dit vaker voor, toen heren hun
leenbezit als particuliere onderneming zagen om hun inkomsten te
optimaliseren.

Lane verwoordde de prikkels voor `de eigenaren van de
productieproducerende onderneming' als volgt:

Een streven naar winstmaximalisatie zou hem er, ondanks het aanhouden
van vaste prijzen, toe brengen zijn kosten te drukken. Hij zou -- net
als Hendrik~VII van Engeland of Lodewijk~XI van Frankrijk -- goedkope
trucs inzetten, of liever nog, zo goedkope middelen als mogelijk, om
zijn legitimiteit te bevestigen, de binnenlandse orde te handhaven en de
aandacht van naburige vorsten af te leiden, zodat zijn eigen militaire
uitgaven laag blijven. Hij behaalde een overschot door
kostenbesparingen, door hogere heffingen die dankzij de strengheid van
zijn monopolie mogelijk waren, of door een combinatie van beide.
\footnote{Lane, `Gevolgen van georganiseerd geweld', op. cit., p.~406.}

Overheden die door eigenaren worden bestuurd, hebben sterke prikkels om
de kosten voor het bieden van bescherming of het monopoliseren van
geweld in een bepaald gebied te drukken. Maar zolang hun heerschappij
onbetwist blijft, vinden zij weinig reden om de prijs (belasting) die
zij aan hun klanten vragen te verlagen tot onder het niveau dat de
opbrengsten maximaliseert. Hoe hoger de prijs die een monopolist kan
vragen en hoe lager zijn daadwerkelijke kosten, hoe groter de winst
uitvalt. Het ideale fiscale beleid voor een door eigenaren beheerde
regering leidt dan ook tot een fors overschot. Wanneer overheden erin
slagen hun inkomsten hoog te houden en tegelijkertijd hun uitgaven te
beperken, heeft dat een grote invloed op het gebruik van middelen.
Arbeid en andere waardevolle productiefactoren, die anders verspild
zouden worden aan het leveren van onnodig dure bescherming, komen dan
beschikbaar voor investeringen en andere doeleinden. Hoe meer winst de
vorst boekt door kosten te drukken, hoe meer middelen er vrijkomen.
Worden die middelen voor investeringen ingezet, stimuleren ze de
economische groei. Zelfs wanneer ze worden aangewend voor
pronkconsumptie, dragen ze bij aan het creëren en in stand houden van
nieuwe markten, markten die anders niet zouden ontstaan als de middelen
aan inefficiënte `bescherming' waren verspild.

\subsection{Werknemers}\label{werknemers}

De prikkels die spelen binnen overheden die door werknemers worden
geleid, kun je eenvoudig in kaart brengen. Vergelijkbare motieven gelden
ook voor andere organisaties die door werknemers worden bestuurd.
Allereerst steunen deze organisaties beleid dat de werkgelegenheid
bevordert en verzetten zij zich tegen maatregelen die banen schrappen.
Zoals Lane verwoordde: `Als werknemers als geheel de macht bezaten,
hadden zij weinig interesse om de verhaalde bedragen voor bescherming te
beperken en waren zij niet geneigd dat grote deel van de kosten --
vertegenwoordigd door de loonkosten, oftewel hun eigen salarissen -- te
verlagen. Het vergroten van de omvang paste immers beter in hun aard.'
\footnote{Ibid.} Een door werknemers geleide overheid heeft zelden de
prikkel om zowel de overheidsuitgaven als de tarieven voor haar klanten
te verlagen. Maar wanneer omstandigheden een sterke oppositie
veroorzaken tegen hogere belastingen, zullen zulke overheden eerder
toestaan dat hun inkomsten onder hun uitgaven zakken dan dat zij hun
uitgaven beperken. Met andere woorden, hun prikkelsystemen wijzen erop
dat zij mogelijk structurele tekorten veroorzaken, iets wat bij door
eigenaren bestuurde overheden minder voorkomt.

\subsection{Klanten}\label{klanten}

Zijn er voorbeelden van overheden die door hun klanten worden bestuurd?
Zeker. Lane liet zich inspireren door de manier waarop middeleeuwse
koopmansrepublieken, zoals Venetië, de overheid in economische termen
aanstuurden. Daar hield een groep groothandelaren -- die bescherming
zocht -- in feite eeuwenlang de touwtjes in handen. Zij waren oprechte
klanten van de beschermingsdienst die de overheid bood, geen eigenaars.
Ze betaalden voor de dienst en waren er niet op uit winst te maken door
hun controle over het geweldmonopolie; wanneer enkelen dat toch
probeerden, hielden de overige klanten hen er lange tijd van af. Nog
andere voorbeelden van overheden die door hun klanten worden bestuurd
vind je in democratieën en republieken met een beperkt kiesrecht, zoals
de oude democratieën of de Amerikaanse republiek in haar beginjaren,
waarin alleen mensen mochten stemmen die een bijdrage leverden aan de
overheid -- zo'n tien procent van de bevolking.

Overheden die op deze manier worden aangestuurd, hebben net als
ondernemingen in handen van eigenaren prikkels om hun bedrijfskosten tot
een minimum terug te dringen.

In tegenstelling tot overheden die door eigenaren of werknemers worden
beheerd, hebben regeringen die daadwerkelijk door hun klanten worden
gecontroleerd een extra stimulans om de geheven prijzen laag te houden.
Wanneer klanten het woord voeren, opereert de overheid mager en
onopvallend, met lage bedrijfskosten, beperkt personeel en minimale
belastingen. Een door haar klanten gestuurde overheid bepaalt de
belastingsniveaus niet om maximale overheidsinkomsten te realiseren,
maar juist om het bedrag dat de klanten kunnen behouden te
optimaliseren.

Net als ondernemingen in concurrerende markten wordt zelfs een
monopolistische instelling, die door haar klanten wordt aangestuurd,
gedwongen efficiënt te werken. Zij kan namelijk geen prijs -- in de vorm
van belastingen -- vragen die de kosten met meer dan een minimale marge
overschrijdt.

\section{De rol van democratie: kiezers als werknemers en
klanten}\label{de-rol-van-democratie-kiezers-als-werknemers-en-klanten}

Lane bekijkt de democratie op de traditionele manier en gaat ervan uit
dat zij ervoor zorgt dat ondernemingen die geweld inzetten en produceren
steeds meer onder de controle van hun klanten komen.\footnote{173}
Natuurlijk lijkt dat de politiek correcte conclusie, maar is het wel zo?
Wij twijfelen daar sterk aan. Kijk eens goed hoe moderne democratieën nu
werkelijk functioneren.

Allereerst vertonen zij nauwelijks de kenmerken van concurrerende
markten, waarin de klanten ondubbelzinnig de voorwaarden van de handel
bepalen. Zo besteedt een democratische overheid doorgaans maar een
schamele fractie van haar totale uitgaven aan bescherming -- een taak
die centraal had moeten staan. In de Verenigde Staten gaat slechts 3,5
procent van de totale overheidsuitgaven naar politie, rechtbanken en
gevangenissen. Tel daar de militaire uitgaven bij op en het aandeel dat
aan bescherming wordt besteed blijft nog maar ongeveer 10 procent van de
inkomsten bedragen. Een ander duidelijk signaal dat de massademocratie
niet door haar klanten wordt aangestuurd, blijkt uit de hedendaagse
politieke cultuur, een erfenis uit het industriële tijdperk. Men zou het
volstrekt onacceptabel vinden als cruciaal overheidsbeleid daadwerkelijk
werd beïnvloed door de belangen van degenen die de rekeningen betalen.
Stel je eens voor dat een Amerikaanse president of een Britse premier
zou voorstellen dat de burgers, die samen het grootste deel van de
belastingen betalen, mogen bepalen welke overheidsprogramma's
gehandhaafd blijven en welke groepen medewerkers moeten worden
ontslagen. Dat zou haaks staan op de verwachtingen over hoe de overheid
hoort te functioneren, terwijl het niet opvalt als ambtenaren bepalen
wiens belastingen verhoogd worden.

Maar bedenk eens: als klanten werkelijk de macht bezaten, zou het
schandalig zijn als zij niet kregen waar ze recht op hebben. Stel dat je
een winkel binnengaat om meubels te kopen en de verkopers nemen je geld
aan, maar negeren vervolgens jouw wensen en raadplegen in plaats daarvan
anderen over hoe je je geld het beste kunt besteden. Dat zou je terecht
verontwaardigen. Je zou het zeker niet normaal vinden als de
winkelmedewerkers zouden beweren dat je de meubels eigenlijk niet
verdient en dat zij ze liever aan iemand anders zouden ineenslaten, die
zij meer waard achten. Dat zo'n situatie bijna dergelijks speelt in de
omgang met de overheid, toont aan hoe weinig invloed haar `klanten'
daadwerkelijk hebben.

Hoe je het ook bekijkt, de kosten van democratische regeringen lopen
volledig uit de hand, in schril contrast met de omstandigheden in
concurrerende markten, waar klanten de aanbieders dwingen efficiënt te
werken. De meeste democratieën kampen met structurele
begrotingstekorten. Dit is kenmerkend voor een fiscaal beleid dat
voortkomt uit de controle van de werknemers. Overheden blijken bijzonder
terughoudend als het gaat om het snijden in hun operationele kosten.
Wereldwijd horen we vrijwel altijd de klacht dat eenmaal ingevoerde
politieke programma's met mankementen nauwelijks tot inperking komen.
Ambtenaren ontslaan blijkt vrijwel onmogelijk. Sterker nog, een van de
belangrijkste voordelen van de privatisering van voorheen staatsbezette
functies is dat particuliere controle het veel eenvoudiger maakt om
overbodige banen te schrappen. Van Groot-Brittannië tot Argentinië laten
nieuwe particuliere managers vaak 50 tot 95 procent van de voorheen door
de staat tewerkgestelde werknemers gaan.

Denk ook eens na over de basis waarop de overheid de fiscale
uitgangspunten voor haar beschermingsdiensten vastlegt. Je vindt
nauwelijks aanwijzingen dat de marktwerking invloed uitoefent op de
belastingtarieven waarmee deze diensten worden geprijsd. Zelfs de
incidentele debatten over belastingverlagingen, die de normale politieke
discussie de afgelopen jaren hebben verstoord, benadrukken hoe ver
democratische regeringen doorgaans verwijderd zijn van de belangen van
hun klanten. Voorstanders van lagere belastingen betogen dan ook soms
dat overheidsinkomsten juist zouden stijgen, omdat eerdere hoge tarieven
de economische activiteit ontmoedigden.

De afweging die zij wilden benadrukken ging niet over de concurrentie
tussen rechtsgebieden, maar over iets veel opvallenders. Zij stelden
niet dat de tarieven in de Verenigde Staten of Duitsland niet hoger
zouden mogen zijn dan 15 procent, louter omdat Hongkong een tarief van
15 procent hanteerde. Integendeel. Belastingdebatten gingen er doorgaans
vanuit dat belastingplichtigen niet simpelweg een rechtsgebied boven een
ander konden verkiezen, maar dat zij moesten kiezen tussen ondernemen
tegen naderend straftarief en even pauze nemen. Men stelde dat
ondernemende mensen, wanneer zij geconfronteerd worden met roofzuchtige
belastingen, hun werk zouden verwaarlozen en in plaats daarvan zouden
gaan golfen als hun belastingdruk niet wordt verlaagd.

Dat zo'n argument überhaupt wordt ingebracht, laat zien in hoeverre de
kosten van beschermingsmaatregelen, die door democratische
welvaartsstaten worden opgelegd, afwijken van de werking van een echte
vrije markt. De wijze waarop progressieve inkomstenbelastingen in de
twintigste eeuw in iedere democratische welvaartsstaat ingevoerd werden,
verschilt radicaal van de prijsmechanismen die klanten verkiezen. Dit
wordt overduidelijk als je de belastingen die geheven worden om een
monopolistische beschermingsvoorziening te ondersteunen afzet tegen de
telefonietarieven, waarbij tot voor kort op de meeste plaatsen sprake
was van een monopolie. Klanten zouden uit het dak gaan als een
telefoonmaatschappij de kosten voor gesprekken op dezelfde wijze in
rekening bracht als bij de inkomstenbelasting. Stel je voor dat je na
een gesprek naar Londen een rekening van \$50.000 ontvangt, simpelweg
omdat je tijdens dat gesprek een deal van \$125.000 had gesloten. Noch
jij, noch een andere klant in haar volle verstand zou dat betalen. Maar
dat is precies de logica achter de heffing van inkomstenbelastingen in
elke democratische welvaartsstaat.

Als je goed nadenkt over hoe industriële democratieën daadwerkelijk
functioneerden, kun je ze het beste zien als regeringen die door hun
eigen werknemers worden aangestuurd. Wanneer je massademocratie
beschouwt als een regering die door haar werknemers wordt gecontroleerd,
verklaar je daarmee ook waarom het zo ingewikkeld is om overheidsbeleid
aan te passen. In veel opzichten opereert de overheid immers in het
voordeel van haar werknemers. Overheidscholen in de meeste democratische
landen lijken structureel te falen en missen vaak een duurzame
oplossing. Als de klanten echt het stuur in handen hadden, zouden zij
veel sneller nieuwe beleidsrichtingen bepalen. De mensen die de
democratische overheid financieren, leggen zelden voorwaarden aan
overheidsuitgaven. In plaats daarvan functioneert de overheid als een
coöperatie die enerzijds buiten particulier eigendom opereert en
anderzijds als een natuurlijk monopolie optreedt. De prijzen staan
vrijwel los van de daadwerkelijke kosten, terwijl de kwaliteit van de
dienstverlening meestal lager ligt dan in de private sector. Klachten
van klanten worden zelden effectief opgelost. Kortom, massademocratie
zorgt ervoor dat de overheid in feite wordt beheerst door haar eigen
`werknemers.'

Maar wacht even. Je zou kunnen beweren dat er in de meeste
rechtsgebieden veel meer kiezers zijn dan werknemers die daadwerkelijk
bij de overheid in dienst zijn. Hoe komt het dan dat deze ambtenaren
toch zo'n dominante positie innemen? De verzorgingsstaat is immers
ontstaan om dat dilemma op te lossen. Omdat anders niet genoeg mensen
zouden deelnemen om een werkende meerderheid te vormen, plaatste men
steeds meer kiezers op de overheidsloonlijst, zodat zij aanspraak konden
maken op allerlei soorten uitkeringen. In feite veranderden zij in
pseudo-overheidsmedewerkers, waardoor zij het dagelijkse gedoe van naar
het werk gaan konden vermijden. Dit was een gevolg van de megapolitieke
logica uit het industriële tijdperk.

Wanneer de omvang van de dwangkracht zwaarder weegt dan een efficiënte
inzet van middelen -- zoals dat tot ongeveer 1989 vaak het geval was --
is het vrijwel onmogelijk dat overheden daadwerkelijk door hun burgers
worden gecontroleerd. Zoals het voorbeeld van de laat vergane
Sovjet-Unie zo treffend liet zien, kon een staat tot nog enkele jaren
geleden enorme macht uitoefenen, ook al verspilde zij op grote schaal
middelen. Wanneer de opbrengsten van geweld hoog en alsmaar stijgen,
speelt omvang een belangrijkere rol dan efficiëntie. Grotere entiteiten
hebben de neiging om de overhand te krijgen op de kleinere, en overheden
die effectiever militair kunnen mobiliseren -- zelfs als dat veel
verspilling met zich meebrengt -- overtreffen doorgaans degenen die hun
middelen efficiënter inzetten.

Bedenk eens wat dit betekent. Als omvang belangrijker wordt geacht dan
efficiëntie, houdt dat in dat overheden die onder controle staan van hun
klanten niet alleen niet kunnen zegevieren, maar vaak zelfs niet kunnen
overleven. In zulke omstandigheden presteren de militaire eenheden die
de meeste middelen inzetten het best. Overheden die daadwerkelijk door
hun klanten -- de betaalders -- worden aangestuurd, krijgen vrijwel
nooit de vrijheid om alles in beslag te nemen en middelen te verspillen.

Klanten willen immers lagere prijzen betalen voor alle producten en
diensten -- inclusief bescherming. Als de westerse democratieën tijdens
de Koude Oorlog daadwerkelijk door hun klanten werden beheerst, dan
waren zij militair gezien zwakkere concurrenten, omdat de toestroom van
middelen naar de overheid vrijwel zeker beperkt zou zijn gebleven.
Onthoud: waar klanten de touwtjes in handen hebben, worden zowel prijzen
als kosten streng gecontroleerd. Maar dat gebeurde nauwelijks. Tijdens
de Koude Oorlog kwamen de verzorgingsstaten duidelijk als winnaars uit
de uitgavenstrijd. Commentatoren beweren wel eens dat een van hun
succesfactoren lag in het vermogen om de Sovjet-Unie in faillissement te
drijven.

Dit feit laat precies zien hoe de inefficiënties van de democratie
ervoor zorgden dat zij in een tijdperk van groeiende opbrengsten uit
geweld, in de megapolitieke arena, de overhand kreeg. Enorme militaire
uitgaven, met al hun verspilling, duiden op een duidelijk suboptimale
inzet van kapitaal dat niet ten goede komt aan particuliere winst.
Eerder suggereerden we dat, hoewel verzorgingsstaten economisch
efficiënter waren dan statssocialistische systemen, zij ver onderdoen in
het creëren van welvaart in vergelijking met laissez-faire enclaves
zoals Hongkong. Ironisch genoeg maakte juist deze inefficiëntie van de
democratische verzorgingsstaat, in tegenstelling tot een onbelemmerd
vrijemarktsysteem, haar succes in de megapolitieke omstandigheden van
het industrialisme mogelijk.

Hoe droeg de door de democratie aangewakkerde inefficiëntie bij aan haar
succes in een tijdperk vol geweld? De sleutel tot het ontrafelen van
deze schijnbare paradox ligt in het inzien van twee punten:

\begin{enumerate}
\def\labelenumi{\arabic{enumi}.}
\item
  Het succes van een soevereine staat in de moderne tijd zit niet in het
  scheppen van welvaart, maar in het opbouwen van een militaire macht
  die met overweldigend geweld elk ander land kan overwinnen. Natuurlijk
  is er geld voor nodig, maar geld zelf wint geen veldslag. De uitdaging
  ligt niet in het ontwikkelen van een uiterst efficiënte economie of
  het realiseren van de snelste groei, maar in het creëren van een
  systeem dat extra middelen kan aantrekken en deze direct naar het
  leger kan leiden. Van nature leveren militaire uitgaven immers weinig
  tot geen financieel rendement op.
\item
  De eenvoudigste manier om instemming te krijgen om geld te steken in
  activiteiten die nauwelijks of geen directe financiële opbrengst
  opleveren -- zoals belastingbetalingen -- is door die goedkeuring te
  vragen aan iemand anders dan degene van wie je het geld wilt
  gebruiken. Een reden dat de Nederlanders Manhattan konden kopen voor
  kralen ter waarde van drieëntwintig dollar, was dat de Indianen aan
  wie zij hun aanbod deden niet de rechtmatige eigenaren waren. `Getting
  to Yes', zoals marketeers zeggen, verloopt onder zulke omstandigheden
  veel eenvoudiger. Stel u voor dat wij als auteurs van dit boek van u
  zouden eisen om, in plaats van de normale prijs, 40 procent van uw
  jaarlijkse inkomen te betalen voor een exemplaar. Dan zouden we al
  sneller instemming krijgen als we die vraag aan een ander zouden
  richten. Sterker nog, we zouden veel overtuigender overkomen als we
  konden rekenen op de steun van meerdere mensen die u zelfs niet kent.
  We zouden bijvoorbeeld een ad-hoc stemming kunnen organiseren -- zoals
  H. L. Mencken het, met minder overdrijving dan hij wellicht had
  bedoeld, omschreef als `een geavanceerde veiling van gestolen
  goederen' -- en om het voorbeeld nog realistischer te maken, zouden we
  ermee instemmen om een deel van het van u geïnde geld te delen met
  deze anonieme toeschouwers in ruil voor hun steun.
\end{enumerate}

Dat is de taak die de moderne democratische verzorgingsstaat zich heeft
toegeëigend. In het industriële tijdperk kende dit systeem geen gelijke,
omdat het op de punten die er toe deden zowel efficiënt als inefficiënt
opereerde. Het combineerde de voordelen van privébezit en de prikkels om
welvaart te creëren met een mechanisme dat in principe onbeperkte
toegang tot die welvaart bood. De democratie hield de zakken van de
welvaartscheppers open. Op militair vlak boekte men tijdens de
hoogtijdagen van die periode succes, omdat het burgers bemoeilijk werd
om de door de overheid geheven belastingen of andere
financieringsmethoden voor defensie-uitgaven -- zoals inflatie --
effectief te beperken.

\subsection{Waarom klanten niet konden
domineren}\label{waarom-klanten-niet-konden-domineren}

Degenen die in de moderne periode voor `bescherming' betaalden, stonden
niet in de positie om -- zelfs niet collectief -- de staat van de
benodigde middelen te beroven, omdat dat hen eenvoudigweg het risico had
opgeleverd om door andere, mogelijk vijandige staten overweldigd te
worden. Deze afweging speelde een duidelijke rol tijdens de Koude
Oorlog. De klanten -- oftewel belastingbetalers -- die een onevenredig
groot deel van de overheidskosten in de vooraanstaande westerse
industriestaten droegen, konden hoge belastingen geenszins weigeren. Dit
zou hebben betekend dat zij zich volledig hadden blootgesteld aan
confiscatie door de Sovjet-Unie of een andere agressieve groep die in
staat was tot organisatie van geweld.

\subsection{Industrialisme en
democratie}\label{industrialisme-en-democratie}

Op de lange termijn kan massademocratie wel een anakronisme blijken te
zijn dat na het industriële tijdperk snel zal verdwijnen.
Massademocratie en de natiestaat ontstonden gelijktijdig met de Franse
Revolutie aan het einde van de achttiende eeuw, waarschijnlijk als
gevolg van de sterke stijging van het reële inkomen. Rond 1750 stegen de
inkomens in West-Europa fors, mede door mildere weersomstandigheden.
Deze periode viel samen met een golf van technologische vernieuwing,
waardoor traditionele ambachtelijke beroepen plaatsmaakten voor nieuwe
machines die ongeschoolde arbeiders -- zelfs vrouwen en kinderen --
eenvoudig konden bedienen. De toepassing van deze nieuwe industriële
apparatuur verhief de lonen van ongeschoolde arbeiders en zorgde zo voor
een egalere inkomensverdeling.

Misschien lag de doorslaggevende aanleiding voor een revolutie niet in
het vaak gehoorde idee dat mensen in opstand komen zodra hun
levensomstandigheden verbeteren. Belangrijker is dat, zodra de inkomens
een zekere drempel bereikten, de vroegmoderne staat in staat werd om
particuliere tussenpersonen en machtige magnaten -- waarmee zij voorheen
financiële middelen veiligstelden -- te omzeilen. De staat voerde direct
bestuur in, ging rechtstreeks in contact met individuele burgers, legde
hen steeds hogere belastingen op en eiste ongunstig vergoede militaire
dienst in ruil voor diverse voordelen.\footnote{Tilly, op. cit.,
  pp.~96--126.}

Doordat de opkomende middenklasse al snel over voldoende geld beschikte,
hoefden heersers niet langer, zoals vroeger, te onderhandelen met
invloedrijke landeigenaren of grote kooplieden. Zoals historicus
Charles~Tilly opmerkte, konden deze voorkomen dat een sterke staat
ontstond die hun bezittingen in beslag nam en hun handel
belemmerde.\footnote{Ibid., p.~130.} Het ligt voor de hand waarom
overheden succesvoller middelen konden innen wanneer zij met miljoenen
burgers afzonderlijk in gesprek gingen, in plaats van met een relatief
handvol heren, hertogen, graven, bisschoppen, contracthuurlingen, vrije
steden en andere semi-soevereine entiteiten waarmee heersers van
Europese staten tot het midden van de achttiende eeuw gedwongen waren te
onderhandelen.

Doordat de reële inkomens stegen, konden overheden een strategie voeren
die hen meer middelen opleverde. Talrijke kleine belastingbedragen
leverden uiteindelijk meer op dan de grotere sommen die door enkele
machtige personen werden betaald. Bovendien werkte het veel eenvoudiger
met de velen, terwijl de enkelen doorgaans weigerden hun geld af te
staan en bovendien veel effectiever weerstand boden.

Tot slot beschikte de doorsnee boer, kleine koopman of arbeider over
vrijwel verwaarloosbare middelen in vergelijking met de staat zelf. De
gemiddelde particulier in West-Europa had op de vooravond van de Franse
Revolutie simpelweg geen kans om met de staat over een verlaging van
zijn belastingtarief te onderhandelen of effectief bezwaar te maken
tegen overheidsplannen en -beleid die zijn belangen bedreigden. Maar dat
was precies de aanpak van de machtige particuliere magnaten, die al
eeuwenlang effectief verzet boden en met de heersers onderhandelden,
zodat de staatsinmenging in hun rijkdom beperkt bleef.

``De oorlogsvoering versnelde de overgang van indirecte naar directe
heerschappij. Bijna elke staat die in oorlog is, ontdekt al snel dat zij
de strijd niet kan financieren met haar opgebouwde reserves en lopende
inkomsten. De overgrote meerderheid van oorlogvoerende staten leent in
ruime mate, verhoogt de belastingen en onttrekt de middelen van de
strijd -- inclusief soldaten -- van burgers die hun middelen anders
zouden hebben ingezet.''\footnote{Ibid., p.~110.} - Charles Tilly

Het voorbeeld van Polen midden in de achttiende eeuw illustreert dit
perfect. In 1760 telde het Poolse nationale leger slechts achttien
duizend soldaten -- een magere strijdmacht vergeleken met de legers van
de heersers in naburige Oostenrijk, Pruisen en Rusland, waarvan de
kleinste al een staand leger van 100.000 soldaten beheerste. Sterker
nog, het Poolse nationale leger bleek in 1760 zelfs kleiner dan andere
bewapende eenheden binnen Polen, want de gecombineerde strijdkrachten
van de Poolse adel bedroegen dertig duizend soldaten.\footnote{Dit
  voorbeeld is te vinden in Ibid., p.~139.}

Had de Poolse koning rechtstreeks contact kunnen leggen met miljoenen
individuele Polen om hen direct te belasten, in plaats van te moeten
vertrouwen op de middelen die via de bijdragen van invloedrijke Poolse
magnaten werden vergaard, dan is er weinig twijfel dat de centrale
Poolse regering veel meer inkomsten had kunnen genereren en daarmee een
groter leger had kunnen bekostigen.

Terwijl gewone burgers, die niet in staat zijn om als massa op te
treden, nauwelijks iets kunnen bereiken, bleken de centrale autoriteiten
overal onweerstaanbaar machtig. In 1760 had de Poolse koning echter niet
de optie om zijn burgers direct te belasten. Hij moest werken via de
heren, rijke kooplieden en andere vooraanstaanden -- een kleine, hechte
groep die gezamenlijk optrad om te voorkomen dat de koning zonder hun
instemming hun middelen zou ontnemen. Aangezien de Poolse adel veel meer
troepen tot haar beschikking had dan hij, kon de koning hierop niet
aandringen.

Hieruit bleek dat het militaire nadeel, voortkomend uit het niet kunnen
omzeilen van de invloed van de rijke en machtige partijen bij het
vergaren van middelen, doorslaggevend was in het Tijdperk van Geweld.
Binnen enkele jaren hield Polen op te bestaan als onafhankelijk land.
Het land werd overmeesterd door invallen vanuit Oostenrijk, Pruisen en
Rusland -- drie landen waarvan elk leger vele malen groter was dan de
bescheiden Poolse strijdmacht. In elk van deze staten vonden de heersers
manieren om de weerstand van rijke kooplieden en de adel tegen de
onteigening van hun middelen te omzeilen.

\subsection{Na de Franse Revolutie}\label{na-de-franse-revolutie}

De Franse Revolutie veroorzaakte een sterke uitbreiding van de
legermachten en bewees daarmee de slagkracht van de democratische
strategie in een periode waarin het geweld uit de hand liep. Vanaf die
tijd sloten regeringen overeen dat zij burgers een ongekende
betrokkenheid in hun dagelijks leven boden, in ruil voor hun inzet in
oorlogen als vervanging van huurlingen. Tegelijkertijd werden zij
geconfronteerd met een steeds zwaardere belastingdruk vanwege hun
groeiende inkomens.

Zoals Tilly zei,

\begin{quote}
De reikwijdte van de staat strekte zich ver uit voorbij haar militaire
kern en burgers begonnen aanspraak te maken op een breed scala aan
diensten, zoals bescherming, rechtspraak, productie en distributie.
Naarmate de nationale parlementen hun bevoegdheden aanzienlijk
uitbreidden -- veel verder dan slechts het goedkeuren van belastingen --
werden zij al snel het middelpunt van eisen van goed georganiseerde
groepen, wiens belangen door de staat werden beïnvloed of hadden kunnen
worden beïnvloed. Directe heerschappij en een massale nationale politiek
ontwikkelden zich onlosmakelijk met elkaar en versterkten elkaar ten
zeerste.\footnote{Ibid., p.~115.}
\end{quote}

Dezelfde redenering die in de achttiende eeuw gold, bleef van kracht tot
1989, het jaar waarin de Berlijnse Muur viel. Terwijl het industriële
tijdperk vorderde, bleven de lonen voor ongeschoolde arbeid stijgen,
waardoor massademocratie uitgroeide tot een nog effectievere manier om
middelen te vergaren. Hierdoor nam de macht van de overheid geleidelijk
toe: in het gemiddelde industriële land stegen haar totale aanspraken op
het jaarlijkse inkomen over de twintigste eeuw met ongeveer 0,5 procent.

In het industriële tijdperk vóór 1989 bleek de democratie, militair
gezien, de meest doeltreffende bestuursvorm te zijn, juist omdat het
nauwelijks mogelijk was om de staatsinmenging in economische middelen
effectief te beperken. Door het royale toekennen van sociale
zekerheidsuitkeringen fungeerden de meeste kiezers als ambtenaren. Dit
werd een dominant politiek kenmerk in alle vooraanstaande industriële
landen, omdat kiezers -- als klanten van beschermingsdiensten -- in een
zwakke positie verkeerden om de overheid in de gaten te houden. Daarbij
stonden zij niet alleen oog in oog met de agressieve dreiging van
communistische systemen, die door staatscontrole enorme middelen voor
militaire doeleinden konden mobiliseren, maar werd een daadwerkelijke
controle door belastingbetalers ook door praktische beperkingen
onhaalbaar.

Miljoenen gewone burgers slagen er simpelweg niet in om effectief samen
hun belangen te verdedigen. De hoge kosten van samenwerking en het
geringe individuele rendement van een gezamenlijke actie zorgen ervoor
dat grote groepen minder succesvol hun bezittingen kunnen beschermen
tegen de staat dan kleinere, beter gemotiveerde groepen.

Als alle andere factoren gelijk zijn, mag je verwachten dat in een
massademocratie de overheid een hoger aandeel van de totale middelen
opeist dan in een oligarchie of in een systeem met gefragmenteerde
soevereiniteit -- een systeem waarin rijkemagnaten militaire macht
uitoefenden en eigen legers onderhielden, zoals in het vroegmoderne
Europa vóór de achttiende eeuw.

Een cruciale, maar zelden onderzochte, reden voor de opmars van de
democratie in de westerse wereld is het relatieve belang van
onderhandelingskosten in een periode waarin de opbrengsten van geweld
toenamen. Het bleek namelijk altijd duurder te zijn om middelen weg te
nemen van de brede massa dan van een kleine elite.

Een relatief kleine elite van welgestelden vormt een hechter en
effectiever geheel dan een grote massa burgers. Zij hebben sterkere
prikkels om samen te werken en beschermen op die manier vrijwel
onvermijdelijk beter hun belangen. Zelfs als de meeste leden besluiten
niet gezamenlijk op te treden, kunnen enkelen met aanzienlijke rijkdom
vaak al genoeg middelen inzetten om de klus te klaren.\footnote{Zie
  Mancur Olson, \emph{The Logic of Collective Action} (Cambridge:
  Harvard University Press, 1965).}

Door democratische besluitvorming toe te passen, kon de natiestaat haar
macht veel effectiever over miljoenen mensen uitoefenen. Die massa kon
niet eenvoudigweg gezamenlijk haar belangen behartigen, terwijl een
kleine groep veel gemakkelijker de organisatorische hindernissen rond
het verdedigen van geconcentreerde belangen overwon. Bovendien bood de
democratie als systeem voor besluitvorming het overtuigende voordeel dat
zij een legitieme regel vastlegde, waardoor de staat de middelen van de
welgestelden kon inzetten zonder telkens hun expliciete instemming te
hoeven vragen. Kortom, als besluitvormingssysteem sloot de democratie
perfect aan op de megapolitieke omstandigheden van het industriële
tijdperk en versterkte zij de natiestaat door bestuurders in staat te
stellen de militaire macht te centraliseren in een tijd waarin de omvang
van de ingezette kracht zwaarder woog dan de efficiëntie van de
mobilisatie.

De Franse Revolutie liet onmiskenbaar zien hoe men de omvang van de
militaire macht op het slagveld kon vergroten. Vervolgens bleven andere
concurrerende natiestaten nauwelijks alternatieven over en namen zij een
vergelijkbare organisatiestructuur over, waarbij hun legitimiteit
uiteindelijk gebaseerd was op democratische besluitvorming. Samengevat
slaagde de democratische natiestaat de afgelopen twee eeuwen dankzij de
volgende onderliggende factoren:

\begin{enumerate}
\def\labelenumi{\arabic{enumi}.}
\item
  De opbrengsten uit geweld namen toe, waardoor de omvang van het geweld
  belangrijker werd dan een efficiënte bestuurswijze.
\item
  De inkomens stegen zo ver boven het bestaansminimum dat de staat grote
  middelen kon innen, zonder dat zij hoefde te onderhandelen met
  invloedrijke magnaten die tegenstand konden bieden.
\item
  De democratie paste zo goed bij de werking van vrije markten dat zij
  de voortdurende toename van welvaart mogelijk maakte.
\item
  De democratie stelde ambtenaren in staat de overheid in eigen hand te
  houden, waardoor het moeilijk werd de uitgaven -- ook op militair
  gebied -- in te perken.
\item
  Als principe voor besluitvorming vormde de democratie een krachtig
  tegengif tegen het gezamenlijke optreden van de rijken, zodat de staat
  hen naar believen kon belasten en op andere wijze hun bezittingen kon
  beschermen tegen invasies.
\end{enumerate}

De democratie bleek de militair zegevierende strategie te zijn, omdat
zij het mogelijk maakte om meer middelen bij de staat te concentreren.
In vergelijking met andere vormen van soevereiniteit -- zoals feodale
heffingen, het goddelijk recht van koningen, collectieve religieuze
plichten of vrijwillige bijdragen van de rijken -- was de massale
democratie militair gezien de krachtigste, omdat zij de meest zekere
methode bood om in een industriële economie middelen samen te brengen.

\begin{quote}
`De natie, als cultureel gedefinieerde gemeenschap, belichaamt de
hoogste symbolische waarde van de moderniteit; zij heeft een
quasi-heilig karakter dat alleen door de religie geëvenaard wordt.
Immers, dit quasi-heilige karakter vindt zijn oorsprong in de religie.
In de praktijk is de natie inmiddels ofwel de moderne, seculiere
vervanger van de religie geworden, ofwel haar machtigste bondgenoot. In
de moderne tijd worden de door de natie opgewekte gemeenschappelijke
sentimenten hoog gewaardeerd en nagestreefd als basis voor
groepsloyaliteit\ldots{} Gezien haar overweldigende macht zou het
geenszins verrassend moeten zijn dat de moderne staat vaak de
begunstigde is.'\footnote{Josep R. Llobera, \emph{The God of Modernity:
  The Development of Nationalism in Western Europe} (Oxford: Berg
  Publishers, 1994), pp.~ix--x.} - JOSEPH R. LLOBERA
\end{quote}

\subsection{Nationalisme}\label{nationalisme}

Hetzelfde geldt voor het nationalisme, dat logisch voortkwam uit de
massale democratie. Staten die het nationalisme wisten te benutten,
ontdekten dat ze tegen lagere kosten grotere legers konden mobiliseren.
Nationalisme bleek een uitvinding waarmee een staat haar militaire
slagkracht kon vergroten. Net als de politiek zelf is nationalisme
grotendeels een moderne ontwikkeling. Zoals socioloog Joseph Llobera in
zijn rijk gedocumenteerde boek over de opkomst van het nationalisme
aantoont, blijkt de natie een denkbeeldige gemeenschap te zijn die
vooral ontstond om de staatsmacht te mobiliseren tijdens de Franse
Revolutie. Zoals hij het verwoordt, `In de moderne zin van het begrip
heeft nationaal bewustzijn pas bestaan sinds de Franse Revolutie, sinds
de tijd dat in 1789 de Constituerende Vergadering het volk van Frankrijk
gelijkstelde aan de Franse natie.'\footnote{Ibid., p.~xiii.}

Nationalisme maakte het eenvoudiger macht te mobiliseren en grote
aantallen mensen te beheersen. Natiestaten ontstonden doordat men de
gemeenschappelijke kenmerken van mensen -- in het bijzonder de gesproken
taal -- sterk benadrukte. Hierdoor kon men rechtstreeks besturen zonder
tussenpersonen, wat de taken van de bureaucratie vereenvoudigde.
Officiële uitvaardigingen die slechts in één taal werden opgesteld,
cirkelden veel sneller en met minder verwarring dan documenten die in
een bonte verzameling talen vertaald moesten worden. Zo daalden de
kosten van het besturen van uitgestrekte gebieden. Vóór de opkomst van
het nationalisme moest de vroegmoderne staat rekenen op de hulp van
heren, hertogen, graven, bisschoppen, vrije steden en andere
corporatieve en etnische tussenpersonen -- van belastingverzamelaars tot
militaire contracthandelaars en huurlingen -- om inkomsten te innen,
troepen op te richten en andere overheidsfuncties uit te voeren.

Daarnaast verlaagde nationalisme aanzienlijk de kosten voor het
mobiliseren van militair personeel door het collectieve gevoel in dienst
van de staat aan te wakkeren. Het aanjagen van een gemeenschapsgevoel
leverde zo'n groot voordeel op dat de meeste staten -- zelfs de
zogenaamd internationalistische \emph{Sovjet-Unie} -- het nationalisme
als aanvullende ideologie omarmden.

Vanuit een breder perspectief vormt nationalisme net zozeer een anomalie
als de staat zelf. Zoals historicus William McNeill aantoont, waren
polyetnische soevereiniteiten vroeger de norm.\footnote{Zie William
  McNeill, \emph{Polyethnicity and National Unity in World History}
  (Toronto: University of Toronto Press, 1986).}

Volgens McNeill zelf: `Het idee dat een regering rechtmatig slechts over
burgers van één etnische groep moet heersen, begon zich tegen het einde
van de middeleeuwen in West-Europa te ontwikkelen.'\footnote{Ibid.,
  p.~7.} De Pruisische Bond (\emph{Preussicher Bund}), opgericht in 1440
als verzet tegen de heerschappij van de Teutoonse Orde, vormt een vroeg
voorbeeld van nationalistische organisatie. Eerder werden enkele
kenmerken van die orde aangehaald als illustratief voor een
soevereiniteit die sterk afwijkt van de natiestaat. De Teutoonse Orde
functioneerde als een charterbedrijf waarvan vrijwel geen lid een
inheemse Pruisenaar was. Haar hoofdkantoor verschoven in de loop der
tijd van Bremen en Lübeck naar Jeruzalem, daarna naar Acre, vervolgens
naar Venetië en uiteindelijk naar Marienberg aan de Wisła. Op een
gegeven moment beheerste zij zelfs het district Burzenland in
Transsylvanië. Het is dan ook niet verrassend dat een soevereiniteit die
zo radicaal afweek van wat men als staat beschouwde, het doelwit werd
van een van de eerste pogingen om het nationale gevoel als machtfactor
te mobiliseren. Ter illustratie van hoe verschillend het vroege
nationalisme was ten opzichte van latere vormen, dienden de Duitstalige
edelen van de Pruisische Bond een petitie in bij de Poolse koning om
Pruisen onder Poolse heerschappij te brengen -- mede omdat de Poolse
vorst, zelfs in die tijd, relatief zwak bleek en niet zo streng regeerde
als de Teutoonse Orde.

Het vroege nationalisme deed zich voor net voorafgaand aan de
buskruitrevolutie, bleef zich ontwikkelen naarmate de vroegmoderne staat
vorm kreeg en kreeg een enorme impuls tijdens de Franse Revolutie. Wij
zijn ervan overtuigd dat nationalisme als krachtig idee inmiddels zijn
hoogtepunt heeft bereikt en langzaam afneemt. Waarschijnlijk bereikte
het zijn toppunt tijdens Woodrow Wilsons poging om aan elke etnische
groep in Europa, aan het einde van de Eerste Wereldoorlog, een eigen
staat toe te kennen. Tegenwoordig zien we een reactionaire variant, die
vooral oplaait in gebieden met dalende inkomens en verslechterende
vooruitzichten -- zoals in Servië.

Zoals we later zullen toelichten, verwachten wij dat nationalisme een
belangrijk verzamelpunt wordt voor mensen die heimwee hebben naar een
samenleving waarin dwang een rol speelde, nu de verzorgingsstaat instort
in de westerse democratieën. Je hebt nog niets gezien. Voor de meeste
mensen in het Westen leek de nasleep van de val van het communisme
relatief onschadelijk: men zag een daling in de militaire uitgaven, een
scherpe ineenstorting van de aluminiumprijzen en een nieuwe bron van
hockeytalent voor de \emph{NHL}. Dat was het positieve nieuws waarover
de meesten van ons, die in de twintigste eeuw volwassen werden -- zeker
als we hockeyfans zijn -- konden juichen. Het grootste deel van het
minder prettige nieuws moet echter nog komen.

Naarmate het industriële tijdperk tot een einde komt, verdwijnen ook de
megapolitieke voorwaarden waaraan de democratie ooit voldeed. Daarom
betwijfelen we of de massademocratie en de verzorgingsstaat lang stand
zullen houden in de nieuwe megapolitieke realiteit van het
informatietijdperk.

\begin{quote}
`Het congres was geen tempel van democratie, het was een markt voor het
ruilen van wetten.' - Alberto Fujimori, president van Peru
\end{quote}

Inderdaad, toekomstige historici zouden kunnen concluderen dat we al
getuige waren van de eerste postmoderne staatsgreep -- de opvallende
vergrendeling van het congres in Peru in 1993. Deze gebeurtenis kreeg
nauwelijks aandacht in de toonaangevende industriële democratieën, maar
kan uiteindelijk meer betekenen dan conventionele analisten suggereren.
Degenen die erover nagedacht hebben, zien het vaak als een machtsgreep
zoals die deprimerend bekend is in de Latijns-Amerikaanse geschiedenis.
Wij beschouwen het echter als mogelijk de eerste stap richting de
delegitimatie van een bestuursvorm waarvan de directe megapolitieke
bestaansreden met de komst van het informatietijdperk begon te vervagen.
Fujimori's sluiting van het congres wijst op de uiteindelijke devaluatie
van politieke beloften. Eenzelfde lot kan andere wetgevende organen
treffen wanneer hun krediet is opgeraakt.

De technologische verschuiving die het industrialisme ondermijnt, heeft
veel landen in zijn greep die worden bestuurd door regeringen die niet
(meer) functioneren -- of het nu gaat om slecht functionerende
overheden. Wetgevende lichamen lijken steeds disfunctioneler te worden;
ze maken wetten die vijftig jaar geleden misschien simpelweg dom leken,
maar vandaag de dag gevaarlijk blijken. Dit kwam in Peru op
spectaculaire wijze aan het licht, toen de interne soevereiniteit van de
staat in 1993 bijna instortte.

\begin{quote}
`Aanvallen, ontvoeringen, verkrachtingen en moorden gingen gepaard met
steeds agressievere rijgewoonten en onveilige straten. De politie
verloor geleidelijk de controle over de situatie en enkele van haar
leden raakten betrokken bij schandalen en werden doorgewinterde
criminelen\ldots{} Mensen raakten er geleidelijk aan gewend buiten de
wet te leven. Diefstal, illegale inbeslagname en bedrijfsovernames
werden aan de orde van de dag.'\footnote{Hernando de Soto, \emph{The
  Other Path} (New York: Harper \& Row, 1989).} - Hernando de Soto
\end{quote}

\subsection{Peru in ruïnes}\label{peru-in-ruuxefnes}

In zekere zin was Peru in 1993 geen moderne natiestaat meer. Het had nog
een vlag en een leger, maar de meeste instituties lagen in puin. Zelfs
de gevangenissen waren overgenomen door de gedetineerden. Deze
desintegratie kent meerdere oorzaken, maar de meeste verklaringen van
deskundigen missen de kern. Peru viel namelijk al vroeg ten prooi aan de
technologische veranderingen die gesloten economieën disfunctioneel
maken en de centrale autoriteit ondermijnen. Bovendien verergeren deze
megapolitieke spanningen wanneer besluitvormingsinstanties -- zoals het
Peruaanse congres -- door perverse prikkels worden gedwongen de
problemen eigenlijk te verergeren in plaats van op te lossen.

De representatieve democratie in Peru was te vergelijken met een paar
geladen dobbelstenen. Als middel om de staat uit te breiden was zij
ongeëvenaard. Maar toen nieuwe omstandigheden om een machtssverandering
vroegen, zorgden de vooringenomen kenmerken die de democratie in de oude
megapolitieke context eens zo effectief hadden gemaakt ervoor dat het
systeem steeds disfunctioneler werd. De wetten die het congres aannam,
verzwakten razendsnel elke basis voor waarde en respect voor de wet.
Zoals de Soto in \emph{The Other Path} verwoordt: `Kleine
belangengroepen ruziën met elkaar, veroorzaken faillissementen en
beïnvloeden ambtenaren. Overheden kennen privileges toe. De wet wordt
ingezet om veel meer te geven en ontnemen dan de moraal toelaat.'
\footnote{Ibid.} Een congres als dat in Peru, volledig beheerst door
belangengroepen, heeft dezelfde morele status als een bende vuilkopers
die gestolen goederen veilen. Dit maakte de vrije markt ontoegankelijk,
waardoor de wet haar geloofwaardigheid verloor. Zoals de Soto schrijft
over de periode voor Fujimori:

\begin{quote}
Een volledige ommekeer van doelen en middelen keerde het leven in de
Peruaanse samenleving compleet op zijn kop, tot op het punt dat
handelingen -- hoe officieel strafbaar ook -- niet langer door het
collectieve bewustzijn werden veroordeeld. Smokkelen is daar een
treffend voorbeeld van. Iedereen -- van de aristocratische dame tot de
nederigste man -- verwerft gesmokkelde goederen. Niemand heeft er
scrupules over; integendeel, men ziet het als een uitdaging voor de
individuele vindingrijkheid of als wraak op de staat. Deze infiltratie
van geweld en criminaliteit in het dagelijks leven gaat gepaard met
toenemende armoede en ontbering. In het algemeen is het reële gemiddelde
inkomen van de Peruanen de afgelopen tien jaar gestaag gedaald en
bevindt het zich nu op het niveau van twintig jaar geleden. Overal hopen
bergen afval zich op. Dag en nacht omsingelen legioenen bedelaars,
autowassers en aaseters de voorbijgangers op zoek naar geld. Geestelijk
zieken zwermen naakt door de straten en stinken van urine. Kinderen,
alleenstaande moeders en gehandicapten bedelen op elke straathoek om
aalmoezen. Het traditionele centralisme van onze samenleving heeft
duidelijk niet voldaan aan de vele behoeften van een land in
transitie.\footnote{Ibid., p.~6.}
\end{quote}

De Soto omschreef het loslaten van de groteske juridische economie ten
gunste van de zwarte markt -- een ontwikkeling die al in gang was gezet
voordat Fujimori het congres opsloot -- als `een onzichtbare revolutie.'

Wij staan positief tegenover de voordelen van de vrije markt, maar zijn
minder enthousiast over een samenleving waarin de wet net zo vervallen
is als het geld. De wereld die de Soto in Peru schetste vóór 1993, was
een `Clockwork Orange'-wereld waarin overmatig gecentraliseerde en
dysfunctionele overheidsinstellingen de burgermaatschappij letterlijk
teniet deden.

Dit was immers wat Fujimori wilde veranderen. Hij verlaagde de inflatie
drastisch door de geldpersen stil te leggen. Ook slaagde hij erin
vijftigduizend overheidsmedewerkers te ontslaan en diverse subsidies in
te perken. Hij begon met het in evenwicht brengen van de begroting. Zijn
hervormingsprogramma omvatte uitgebreide plannen om vrije markten tot
stand te brengen en de industrie te privatiseren. Maar, net als in de
voormalige Sovjet-Unie, had hij in 1993 nog niet de meeste cruciale
elementen van zijn hervorming gerealiseerd -- waaronder de eerste ronde
van grootschalige privatiseringen van staatsbanken, mijnbouwbedrijven en
nutsbedrijven. In plaats van deze noodzakelijke voorstellen door te
voeren, trachtte het congres van Peru -- net als het Russische congres
dat Yeltsins hervormingen in Moskou tegenging -- een stap achteruit te
zetten. Hun plan luidde: subsidies herstellen vanuit een lege schatkist,
de loonlijst opblazen en alle gevestigde belangen beschermen, met name
de bureaucratie -- precies wat je zou verwachten van een regering die
door haar eigen werknemers wordt beheerst.

Fujimori beweerde dat het congres van Peru aarzelend en corrupt was --
een uitspraak waar vrijwel iedereen het mee eens was. Hij voegde eraan
toe dat de aarzelende en corrupte werkwijze in het congres iedere poging
om de ineenstortende economie van Peru te hervormen of de gewelddadige
aanvallen van narco-terroristen en nihilistische guerrilla's van
\emph{het Sendero Luminoso (Shining Path)} tegen te gaan, dwarsboorde.

\subsection{De 70-procentoplossing}\label{de-70-procentoplossing}

Fujimori sloot dus het congres, een actie die hem net zo autoritair zou
laten lijken als vele eerdere Latijns-Amerikaanse leiders. Wij waren er
destijds van overtuigd dat hij een fundamenteel obstakel voor
hervormingen terecht had aangekaart. De uitbundige officiële lofzangen
op het Peruaanse congres, lovend aangehaald door Amerikaanse
opiniemakers en ambtenaren van \emph{het Ministerie van Buitenlandse
Zaken}, vonden geen weerklank bij het Peruaanse volk. Terwijl
Noord-Amerikanen het congres prezen als het toonbeeld van vrijheid en
beschaving, vierde het Peruaanse volk juist de maatregel. De
populariteit van president Fujimori steeg tot boven de 70 procent toen
hij het congres naar huis stuurde, en kort daarop herkozen de kiezers
hem met een verpletterende meerderheid voor een tweede termijn. Veel
burgers zagen hun wetgevende macht eerder als een belemmering voor hun
welzijn dan als een afspiegeling van hun rechten. In 1994 groeide de
reële economie van Peru met 12,9 procent -- het hoogste cijfer
wereldwijd.

\subsection{Deflatie van politieke
beloften}\label{deflatie-van-politieke-beloften}

Wij beschouwden de onrust in Peru niet als een terugkeer naar oude
dictatorschappen, maar als het begin van een bredere transitiecrisis. Je
kunt ervan uitgaan dat in veel landen wanbestuurscrises opduiken zodra
politieke beloften leeglopen en regeringen hun krediet kwijtraken.
Uiteindelijk moeten nieuwe institutionele vormen ontstaan die de
vrijheid weten te waarborgen onder de nieuwe technologische voorwaarden,
terwijl ze tegelijkertijd de gemeenschappelijke belangen van alle
burgers weerspiegelen en levend houden.

Enkel weinigen hebben stilgestaan bij de onverenigbaarheid tussen
bepaalde instituties van industrieel bestuur en de megapolitiek van de
postindustriële samenleving. Of men deze tegenstellingen nu expliciet
erkent of niet, de gevolgen ervan worden steeds duidelijker naarmate
zich wereldwijd meer voorbeelden van politiek falen aandienen. De
regeringsinstellingen die in de moderne tijd zijn ontstaan,
weerspiegelen immers de megapolitieke omstandigheden van eeuwen geleden.
Het informatietijdperk vraagt daarom om nieuwe representatiemechanismen
om chronische disfunctie en zelfs maatschappelijk verval te voorkomen.

Toen in 1989 de Berlijnse Muur viel, betekende dat niet alleen een einde
aan de Koude Oorlog, maar liet het ook de eerste sporen zien van een
stille aardbeving in de fundamenten van de mondiale machtsstructuur. Het
luidde een einde in aan een lange periode waarin geweld steeds meer
vruchten afwierp. De val van het communisme -- door ons in 1987
voorspeld in \emph{Blood in the Streets} en zelfs eerder aangekondigd in
onze maandelijkse nieuwsbrief \emph{Strategic Investment} -- was veel
meer dan louter een afwijzing van een ideologie. Het vormde een tastbaar
signaal van de meest ingrijpende ontwikkeling in de geschiedenis van
geweld in de afgelopen vijf eeuwen. Als onze analyse klopt, past de
organisatiestructuur van de samenleving zich onvermijdelijk aan om de
groeiende diseconomieën van schaal in geweldgebruik te weerspiegelen. De
grenzen waarbinnen de toekomst vorm krijgt, zijn hertekend.

De inhoud van het te vertalen boek is:

\bookmarksetup{startatroot}

\chapter{De megapolitiek van het
informatietijdperk}\label{de-megapolitiek-van-het-informatietijdperk}

\begin{quote}
`\ldots het is gecomputeriseerde informatie, niet arbeidskracht of
massaproductie, die steeds meer de Amerikaanse economie aandrijft en die
oorlogen zal winnen in een wereld die is verbonden met 500 tv-kanalen.
De gecomputeriseerde informatie bestaat in cyberspace -- de nieuwe
dimensie die is gecreëerd door de eindeloze reproductie van
computernetwerken, satellieten, modems, databases en het publieke
internet'\footnote{Neil Munro, `The Pentagon's new nightmare: an
  electronic Pearl Harbor', \emph{Washington Post}, 16 juli 1995, p.~C3.}:
- NEIL MUNRO
\end{quote}

Op 30 december 1936 grepen autowerkers, die op zoek waren naar hogere
lonen, met geweld twee hoofdvestigingen van \emph{General Motors} in
Flint, Michigan in beslag. Ze legden de machines stil, schakelden de
lopende banden uit en gedroegen zich alsof ze thuis waren. De
werknemers, die normaal gesproken de fabriek draaiende hielden, hielden
een zitstaking aan -- een industriële confrontatie die weken zou duren.
Het drama werd gekenmerkt door gewelddadige rellen en wisselende
steunbetuigingen van de politie, de militie van Michigan en politici op
alle overheidsniveaus. Omdat de vakbond nauwelijks succes boekte met het
afdwingen van haar eisen, greep ze op 1 februari 1937 opnieuw tot actie.

Vervolgens bezetten vakbondsactivisten met geweld de
\emph{Chevrolet}-fabriek van \emph{GM} in Flint. Door de voornaamste
fabrieken van \emph{General Motors} in beslag te nemen en plat te
leggen, hielden de arbeiders de productie van het bedrijf effectief
stil. In de tien dagen na de inbeslagname van de derde fabriek
produceerde \emph{GM} in de Verenigde Staten slechts 153 auto's.

We kijken nog eens terug op een nieuwsflits van zestig jaar geleden om
de revolutionaire veranderingen in de megapolitieke omstandigheden die
zich tegenwoordig ontvouwen in een helderder perspectief te plaatsen.
Sommigen van jullie hebben zelf de bezettingsstaking bij \emph{GM}
meegemaakt. Toch zijn wij ervan overtuigd dat bezettingsstakingen in het
informatietijdperk net zo anachronistisch zullen blijken als het beeld
van slaven die door de woestijn ploeteren terwijl ze enorme stenen
verslepen om piramides voor de farao's te bouwen. Terwijl vakbonden en
hun intimidatietactieken tijdens de industriële periode zo ingeburgerd
raakten dat ze vanzelfsprekend deel uitmaakten van het sociale
landschap, leunden ze op specifieke megapolitieke omstandigheden die nu
in rap tempo verdwijnen. Op de \emph{Information Superhighway} kom je
géén \emph{Chevrolets} tegen en gaat de \emph{UAW} ook niet in staking.

Het fortuin van overheden zal ten onder gaan, net als dat van hun
tegenhangers, de vakbonden. De door de overheid vastgelegde dwang, die
in de twintigste eeuw een cruciale rol speelde, is niet langer houdbaar.
Technologie zorgt voor een ingrijpende ommekeer in de logica van
afpersing en bescherming.

\begin{quote}
'' \ldots{} er is geen eigendom, geen heerschappij, geen duidelijk
onderscheid tussen `de mijne en de jouwe'; maar slechts datgene dat
iedere man kan bemachtigen, en zolang hij het kan
behouden.''\footnote{Thomas Hobbes, Leviathan, hoofdstuk 13, `De
  natuurlijke toestand van de mens betreffende hun gelukzaligheid en
  ellende'.}: -THOMAS HOBBES
\end{quote}

\subsection{Afpersing en bescherming}\label{afpersing-en-bescherming}

Door de eeuwen heen stak men met geweld een dolk in het hart van de
economie. Thomas Schelling merkte scherp op: ``De macht om te kwetsen --
om dingen te vernietigen die iemand dierbaar zijn, om pijn en verdriet
toe te brengen -- is een soort onderhandelingsmacht, niet eenvoudig te
hanteren, maar vaak gebruikt. In de onderwereld vormt het de basis voor
chantage, afpersing en ontvoering, in de commerciële wereld voor
boycots, stakingen en lock-outs\ldots{} Het is vaak de basis voor
discipline, zowel civiel als militair; en goden gebruiken het om
discipline af te dwingen.''\footnote{Thomas Schelling, Arms and
  Influence (New Haven: Yale University Press, 1966).}

Hoe goed een overheid belastingen kan innen, hangt af van dezelfde
kwetsbaarheden die particuliere afpersing en chantage kenmerken.
Alhoewel we er zelden zo naar kijken, geeft de verhouding tussen
bezittingen die met dwang worden gecontroleerd en besteed -- of dat nu
door criminelen of door de overheid gebeurt -- een globale indicatie van
de megapolitieke balans tussen afpersing en bescherming. Als
technologische ontwikkelingen het beschermen van bezittingen
bemoeilijken, zou criminaliteit op de loer liggen en zouden vakbonden
actiever worden. In dat geval vraagt bescherming via de overheid immers
een meerprijs, wat leidt tot hoge belastingen. Daarentegen verschuift de
balans dankzij technologie naar bescherming wanneer belastingen laag
blijven en de lonen op de werkvloer via marktwerking worden vastgesteld
in plaats van door politieke inmenging of dwang.

Aan het einde van het derde kwartaal van de twintigste eeuw bereikte de
technologische discrepantie tussen afpersing en bescherming een extreem
niveau. In een aantal geavanceerde westerse landen namen regeringen meer
dan de helft van de middelen in beslag. Voor een groot deel van de
bevolking werden de inkomens ofwel op fiat ingesteld, ofwel bepaald
onder druk -- bijvoorbeeld door stakingen en andere dreigementen met
geweld. Zowel de verzorgingsstaat als de vakbonden waren het directe
gevolg van technologische ontwikkelingen, waarbij macht in de twintigste
eeuw boven efficiëntie triomfeerde en de opbrengsten daarvan werden
gedeeld. Ze hadden niet kunnen ontstaan zonder de militaire en civiele
technologieën die in het industriële tijdperk het gebruik van geweld
aantrekkelijker maakten.

Het vermogen om activa op te bouwen ging altijd gepaard met een
inherente kwetsbaarheid voor afpersing. Hoe meer activa je creëert of
bezit, hoe hoger de tol die je uiteindelijk betaalt. Je betaalt dan
ofwel aan iedereen die de macht bezit om met geweld af te persen, ofwel
voor een militaire kracht die elke poging tot afpersing met brute kracht
verijdelt.

\begin{quote}
`Geweld zal niet langer in uw land te horen zijn, noch verwoesting
binnen uw grenzen\ldots{}' - ISAIAH 60:1
\end{quote}

\subsection{De wiskunde van
bescherming}\label{de-wiskunde-van-bescherming}

Nu temperen we de dolk van geweld snel. Door informatietechnologie
verandert de balans tussen bescherming en afpersing ingrijpend: we
beveiligen onze activa veel eenvoudiger, terwijl afpersers steeds minder
kans maken. Dankzij de technologie van het informatietijdperk kunnen we
activa ontwikkelen die buiten bereik blijven van allerlei vormen van
dwang. Deze nieuwe scheve verhouding tussen bescherming en afpersing
vloeit voort uit een fundamentele wiskundige waarheid: vermenigvuldigen
is eenvoudiger dan delen. Hoewel deze waarheid vanzelfsprekend is,
ontdekten we de verstrekkende gevolgen ervan pas met de komst van
microprocessors. Supercomputers voerden in het afgelopen decennium
miljarden malen meer berekeningen uit dan in de gehele voorgaande
geschiedenis. Dankzij deze sprong in rekenkracht konden we voor het
eerst enkele universele kenmerken van complexiteit doorgronden.
Computers laten zien dat je complexe systemen pas echt begrijpt als je
ze van onderaf opbouwt. Het vermenigvuldigen van priemgetallen verloopt
eenvoudig, maar het ontrafelen van een product van grote priemgetallen
blijkt bijna onmogelijk. Kevin Kelly, hoofdredacteur van \emph{Wired},
verwoordt het als volgt: `Het vermenigvuldigen van enkele priemgetallen
tot een groter getal gaat eenvoudig -- elk basisonderwijskind kan dat --
maar de supercomputers wereldwijd raken overbelast wanneer ze proberen
een product te ontwarren tot zijn eenvoudige priemgetallen.'\footnote{Kevin
  Kelly, Out of Control: The New Biology of Machines, Social Systems,
  and the Economic World (Reading, Mass.: Addison-Wesley, 1995),
  pp.~45--46.}

\subsection{De logica van complexe
systemen}\label{de-logica-van-complexe-systemen}

De cybereconomie vormt zich onvermijdelijk met deze diepgaande
wiskundige waarheid als basis. Dit zie je al terug in de robuuste
encryptie-algoritmen. Later in dit hoofdstuk onderzoeken we hoe deze
algoritmen een nieuw, beveiligd domein van cyberhandel mogelijk maken,
waarin geweld sterk wordt beperkt. De balans tussen afpersing en
bescherming kantelt drastisch in het voordeel van bescherming. Dit
stimuleert de opkomst van een economie die steunt op spontane, adaptieve
mechanismen in plaats van op bewuste besluitvorming en bureaucratische
middelenallocatie. Het nieuwe systeem, waarin bescherming vooropstaat,
verschilt sterk van wat voortkwam uit de overheersende dwang in de
industriële periode.

\subsection{Command-and-control systemen zijn
primitief}\label{command-and-control-systemen-zijn-primitief}

In \emph{The Great Reckoning} schreven we dat computers ons in staat
stellen de verborgen complexiteit te onthullen die aan talloze systemen
ten grondslag ligt.\footnote{Zie hoofdstuk 8 van \emph{The Great
  Reckoning}, `Lineaire verwachtingen in een niet-lineaire wereld: hoe
  de telescoop ons leerde rekenen; hoe de computer ons kan helpen te
  zien.'} Dankzij geavanceerde rekenkracht begrijpen we de dynamiek van
complexe systemen beter én benutten we deze complexiteiten productief.
In zekere zin is dit geen keuze, maar een noodzaak als de economie
verder wil groeien dan de starre, centraal gestuurde ontwikkelingsfase.
Dergelijke systemen, die berusten op lineaire relaties, zijn in wezen
primitief. Wanneer de overheid middelen in beslag neemt, trekt ze die
onherroepelijk weg uit waardevolle, complexe toepassingen en zet ze ze
in voor eenvoudige, primitieve projecten. Dit proces kent dezelfde
beperking als de wiskundige asymmetrie die het ontrafelen van het
product van grote priemgetallen bijna onmogelijk maakt. De manier waarop
we de buit verdelen, blijft altijd primitief.

\subsection{Alles wordt complexer}\label{alles-wordt-complexer}

Waar je in het universum ook kijkt, zie je systemen die in de loop van
hun evolutie steeds ingewikkelder worden. Dit valt op in de astrofysica,
maar zelfs in iets eenvoudigs als een plas water. Laat regenwater op een
lager gelegen plek zijn gang gaan en je zult zien dat het geleidelijk
complexer wordt. Geavanceerde systemen werken als adaptieve netwerken
zonder centrale sturing. Elk complex systeem in de natuur -- waarvan de
markteconomie de meest duidelijke sociale uiting is -- vertrouwt op
verspreide vermogens. Systemen die het beste presteren in een breed
scala aan omstandigheden ontlenen hun veerkracht aan een spontaan
ontstaan orde, die ruimte biedt voor nieuwe mogelijkheden. Zo vormt het
leven zelf een complex systeem; miljarden mogelijke gencombinaties
leiden tot één menselijk individu -- een ordening die elke bureaucratie
de kop zou uitslaan.

Vijfentwintig jaar geleden bleef dit slechts een vermoeden, maar
tegenwoordig kunnen we het aantonen. Nu computers ons steeds dichterbij
brengen in het begrijpen van de wiskunde achter kunstmatig leven,
krijgen we ook meer inzicht in de wiskunde van het echte leven, oftewel
biologische complexiteit. Dankzij informatietechnologie ontsluiten we de
geheimen van complexiteit en transformeren we economieën naar nog
verfijndere vormen. Het internet en het world wide web vertonen al
kenmerken van een levend systeem, zoals Kevin Kelly suggereert in
\emph{Out of Control: The New Biology of Machines, Social Systems, and
the Economic World.}\footnote{Ibid., pp.~2--4.}

Volgens hem fungeert de natuur als `een ideeënfabriek. Overal in de
jungle, zelfs in elke mierenhoop, schuilen levendige, postindustriële
paradigma's\ldots{} Een grootschalige overgang van biologische principes
naar machines zou ons met ontzag vervullen. Zodra het natuurlijke en het
door de mens gemaakte volledig samensmelten, zullen onze creaties leren,
zich aanpassen, zichzelf herstellen en evolueren -- een kracht waarvan
we nauwelijks durfden te dromen.'\footnote{Ibid., p.~4.}

De gevolgen van die `grootschalige overdracht van biologie naar
machines' zullen ingrijpend zijn. Sociale systemen hebben altijd de
neiging gehad om de kenmerken van de dominante technologie te
weerspiegelen -- waarin Marx gelijk had. Grote fabrieken maakten plaats
voor een tijdperk van een sterke overheid, terwijl microprocessing leidt
tot de verkleining van organisaties. Als onze analyse klopt, creëert de
technologie van het informatietijdperk uiteindelijk een economie die
beter profiteert van de voordelen van complexiteit.

Toch worden de wereldwijde implicaties van zo'n verandering nauwelijks
begrepen. Zelfs degenen die het wiskundige belang inzien, presenteren
hun inzichten vaak op een verouderde manier. Het is moeilijk te bevatten
en te verankeren dat technologische ontwikkelingen in de komende jaren
de meeste politieke vormen en concepten van de moderne wereld achter
zich zullen laten. Zoals de overleden natuurkundige Heinz Pagels in zijn
vooruitziende boek \emph{The Dreams of Reason} schreef: `Ik ben ervan
overtuigd dat de naties en volkeren die de nieuwe wetenschap van
complexiteit beheersen, de economische, culturele en politieke
supermachten van de volgende eeuw zullen worden.'\footnote{Heinz Pagels,
  \emph{the dreams of reason} (New York: Bantam Books, 1989), geciteerd
  in Roger Lewin, \emph{complexity: life at the edge of chaos} (New
  York: Macmillan, 1992), p.~10.}

Het is een indrukwekkende voorspelling, maar wij denken dat deze
uitspraak uiteindelijk nog preciezer zal blijken dan Pagels ooit durfde
te beweren. Samenlevingen die zich herstructureren tot complexere,
adaptieve netwerken zullen zeker bloeien. Echter, de grootste winnaars
worden dan niet naties of `politieke supermachten', maar de soevereine
individuen van het nieuwe millennium.

Zoals Pagels' voorspelling tegenwoordig luidt, doet ze denken aan een
sjamaan uit een jachtgroep vijfhonderd generaties geleden die, terwijl
zijn mannen rondom het kampvuur zaten, zei: `Ik ben er vast van
overtuigd dat de eerste jachtgroep die de nieuwe wetenschap van
geïrrigeerde landbouw onder de knie krijgt, meer vrije tijd zal hebben
om verhalen te vertellen dan zelfs die mannen aan het meer die de enorme
vis vangen.' Hoe terecht hij ook was in zijn waardering van
complexiteit, Pagels miste het meest fundamentele gegeven: wanneer de
logica van geweld verandert, verandert de samenleving.

\section{De logica van geweld}\label{de-logica-van-geweld}

Om te begrijpen hoe en waarom, moeten we ons richten op diverse aspecten
van de megapolitiek die je zelden tegenkomt. Deze vraagstukken
onderzocht historicus Frederic C. Lane, wiens werk over geweld en de
economische betekenis van oorlog elders in dit boek aan bod komt. Toen
Lane midden in deze eeuw schreef, was een informatiesamenleving nog
ondenkbaar. Onder die omstandigheden kon hij er rotsvast van overtuigd
zijn dat de strijd om geweld wereldwijd in zijn definitieve fase beland
was met de opkomst van de natiestaat. In zijn werken geeft hij niet aan
dat hij microverwerking had voorzien of dat hij geloofde dat het
technologisch haalbaar was om activa in cyberspace te creëren -- een
rijk zonder fysieke aanwezigheid. Lane ging niet in op de implicaties
van de mogelijkheid dat grote handelsvolumes vrijwel immuun zouden
kunnen worden gemaakt voor de invloed van geweld.

Hoewel Lane de huidige technologische revoluties niet had voorzien,
blijken zijn inzichten in de verschillende stadia van de monopolisatie
van geweld uit het verleden zo scherp dat ze duidelijk toepasbaar zijn
op de opkomende informatierevolutie. In zijn studie van de gewelddadige
middeleeuwse wereld richtte hij de aandacht op vraagstukken die
conventionele economen en historici vaak over het hoofd zien. Hij
realiseerde zich dat de manier waarop geweld wordt georganiseerd en
gecontroleerd een cruciale rol speelt bij de toewijzing van schaarse
middelen. Lane erkende bovendien dat, hoewel de productie van geweld
doorgaans niet als onderdeel van de economische output wordt gezien, de
beheersing ervan essentieel is voor de economie. De voornaamste taak van
de overheid bestaat immers uit het bieden van bescherming tegen geweld.
Zoals hij het verwoordde,

`Elke economische onderneming heeft bescherming nodig en betaalt
daarvoor; bescherming tegen de vernietiging of de gewapende inbeslagname
van haar kapitaal en tegen de gewelddadige verstoring van haar arbeid.
In sterk georganiseerde samenlevingen behoort het leveren van deze
bescherming tot de taken van een speciale vereniging of onderneming die
men 'overheid' noemt. Inderdaad, een van de meest onderscheidende
kenmerken van overheden is hun poging om wet en orde te handhaven door
zelf kracht in te zetten en op diverse wijzen het gebruik van geweld
door anderen te beheersen.'\footnote{Frederic C. Lane, `the economic
  meaning of war and protection' in \emph{Venice and history: the
  collected papers of Frederic C. Lane} (Baltimore: The Johns Hopkins
  Press, 1966), pp.~383--384.}

Dat is een punt dat zo fundamenteel blijkt te zijn dat het óf in
leerboeken thuishoort, óf een vast onderdeel vormt van de burgerlijke
discussie die zogenaamd de koers van de politiek bepaalt. Maar men kan
het niet zomaar negeren als men de zich ontvouwende informatierevolutie
wil doorgronden. Immers, de bescherming van leven en eigendom is een
essentiële behoefte waarmee elke samenleving ooit te maken heeft gehad.
Het afweren van gewelddadige agressie vormt het centrale dilemma in de
geschiedenis en kent geen eenvoudige oplossing, al bestaan er meerdere
manieren om bescherming te bieden.

\subsection{Het einde van een
tijdperk}\label{het-einde-van-een-tijdperk}

Terwijl we dit schrijven, beginnen de megapolitieke gevolgen van het
informatietijdperk net voelbaar te worden. De economische verschuivingen
van de afgelopen decennia hebben ons doen afwenden van de industriemacht
naar een tijdperk waarin informatie en berekeningen centraal staan, van
machinale kracht naar microverwerking, van fabrieken naar werkstations,
van massaproductie naar kleine teams, of zelfs naar zelfstandigen.
Naarmate ondernemingen kleiner worden, neemt de kans op sabotage en
chantage op de werkvloer af. Vakbonden vinden het veel lastiger om
kleinere bedrijven te organiseren.

Dankzij microtechnologie kunnen ondernemingen kleiner en mobieler
opereren. Velen leveren diensten of producten die nauwelijks natuurlijke
hulpbronnen vereisen. In principe kunnen deze ondernemingen vrijwel
overal op aarde werken, omdat ze niet gebonden zijn aan een specifieke
locatie, zoals een mijn of een haven. Daardoor lopen ze op den duur veel
minder kans om door vakbonden of politici onder druk te worden gezet.
Een oude Chinese volkswijsheid luidt: `Van alle zesendertig manieren om
uit de problemen te komen, is de beste manier: vertrek.'\footnote{Shi
  Mai'an and Lao Guanzhong, \emph{outlaws of the marsh}, vertaald door
  Sidney Shapiro (Bloomington: Indiana University Press, 1981), p.~12.}

In het informatietijdperk zal die oosterse wijsheid haar vruchten
afwerpen. Wanneer het bedrijfsleven onaangenaam wordt door buitensporige
eisen op één locatie, wordt verhuizen een stuk eenvoudiger. Inderdaad,
zoals we straks bespreken, kunnen ondernemers in het informatietijdperk
virtuele ondernemingen opzetten, waarbij hun vestigingsplaats volledig
afhankelijk is van de spotmarkt. Een plotselinge toename van
afpersingspogingen, of het nu door overheden of anderen komt, kan er
namelijk toe leiden dat de activiteiten en activa van een virtuele
onderneming bliksemsnel uit de betreffende jurisdictie verdwijnen.

De toenemende integratie van microtechnologie in industriële processen
zorgt ervoor dat zelfs bedrijven die nog steeds massaproductie met grote
schaalvoordelen aanbieden, niet langer zo kwetsbaar zijn voor
gewelddadig misbruik als voorheen. Een illustratief voorbeeld is de
ineenstorting van een langdurige staking door de kleine vakbond van
autowerkers bij Caterpillar, die eind 1995 na bijna twee jaar werd
beëindigd. In tegenstelling tot de lopende bandsystemen uit de jaren
dertig werkt de huidige Caterpillar-fabriek met een aanzienlijk hoger
aantal geschoolde medewerkers. Door de druk van buitenlandse
concurrentie besteedde Caterpillar een groot deel van het ongeschoolde
werk uit, sloot inefficiënte fabrieken en investeerde bijna 2 miljard
dollar in de computerisering van machinegereedschappen en de installatie
van assemblagerobots. Zelfs de staking stimuleerde het invoeren van
arbeidsbesparende efficiëntieverbeteringen. Het bedrijf stelt nu dat het
twee duizend werknemers minder nodig heeft dan toen de staking
begon.\footnote{George E Will, `Vaarwel aan de welvaartsstaten,'
  \emph{Washington Post}, 17 december 1995, p.~C7.}

De megapolitieke organisatie van het productieproces is ingrijpender
veranderd dan velen zich realiseren. Deze transformatie is nog niet
volledig zichtbaar, mede doordat er altijd een vertraging optreedt
tussen een revolutionaire verschuiving in megapolitieke omstandigheden
en de bijbehorende institutionele aanpassingen. Bovendien zorgt de
razendsnelle ontwikkeling van microprocessortechnologie ervoor dat
binnenkort producten op de markt komen waarvan we de megapolitieke
gevolgen al kunnen voorspellen, nog voordat ze daadwerkelijk bestaan.
Die innovaties zullen een geheel nieuwe wereld inluiden.

\section{Uitbuiting van de kapitalisten door de
arbeiders}\label{uitbuiting-van-de-kapitalisten-door-de-arbeiders}

De eigenschappen van de technologie gedurende het grootste deel van de
twintigste eeuw maakten het voor eigenaren en managers buitengewoon
lastig om krachtig in te grijpen zodra arbeiders met geweld een fabriek
innamen -- bijvoorbeeld via een sit-downstaking. Zoals historicus Robert
S. McElvaine opmerkte, ``maakte een sit-downstaking het voor werkgevers
moeilijk om de staking te doorbreken zonder hetzelfde met hun eigen
apparatuur te moeten doen.''\footnote{Robert S. MeElvaine, De grote
  depressie: Amerika, 1929--1941 (New York: \emph{Times Books}, 1984),
  p.~292.} In feite hielden de arbeiders het kapitaal van de eigenaren
als losgeld vast. Om redenen die we hieronder nader toelichten bleek dat
grotere industriële bedrijven veel toegankelijker waren voor vakbonden
dan kleinere ondernemingen. In 1937 werd General Motors beschouwd als
wellicht de toonaangevende industriële onderneming ter wereld. De
fabrieken behoorden tot de grootste en meest kostbare concentraties van
machines die ooit samengebracht werden, en telden duizenden werknemers
in dienst. Elk uur en iedere dag dat de GM-installaties tot stilstand
kwamen, kostte dat het bedrijf een aardig fortuin. Een staking die
wekenlang voortduurde -- zoals in de winter van 1936--37 -- leidde tot
snel oplopende verliezen.

\subsection{Het trotseren van vraag en
aanbod}\label{het-trotseren-van-vraag-en-aanbod}

Na de inbeslagname van haar derde fabriek en door het onvermogen om
auto's te produceren, gaf GM zich al snel over aan de vakbond. Dit was
beslist geen economische beslissing gebaseerd op vraag en aanbod --
integendeel. Toen General Motors uiteindelijk aan de eisen van de
vakbond voldeed, waren er in de Verenigde Staten maar liefst negen
miljoen mensen werkloos, goed voor 14 procent van de beroepsbevolking.
De meeste werklozen hadden met plezier een baan bij GM aangenomen; zij
beschikten over de noodzakelijke vaardigheden voor assemblagelijnwerk,
al suggereerden moderne rapporten vaak het tegendeel.

Een subtiele etiquette verhulde een directe analyse van de
arbeidsverhoudingen in de industriële periode. Een van de voorwendselen
was het idee dat fabrieksbanen, met name in het midden van de twintigste
eeuw, specialistisch werk zouden zijn. Maar dat bleek volstrekt onjuist.
De meeste fabrieksbanen hadden door vrijwel iedereen vervuld kunnen
worden, zolang men maar stipt op tijd opdook. Voor deze banen had je
vrijwel geen opleiding nodig -- zelfs lezen en schrijven waren niet
vereist. Nog in de jaren tachtig bleek een aanzienlijk deel van het
personeel van General Motors analfabeet, rekenkundig onkundig of beide
te zijn. Tot in de jaren negentig kreeg de gemiddelde
assemblagemedewerker bij GM slechts één dag oriëntatie voordat hij aan
de slag ging. Een baan die je in één dag onder de knie krijgt, is
geenszins specialistisch werk.

Toch dwongen de GM-fabrieksarbeiders in 1937 hun werkgevers tot een
loonsverhoging, zelfs toen zowel ongeschoolde als geschoolde arbeiders
wanhopig op zoek waren naar werk. Hun succes had veel meer te maken met
de dynamiek van geweld dan met de conventionele marktwerking.

In maart 1937 -- de maand na de afwikkeling van de confrontatie bij GM
-- vonden in de Verenigde Staten nog eens 17 sit-downstakingen plaats,
waarvan de meeste succesvol waren. Vergelijkbare gebeurtenissen deden
zich in elk geïndustrialiseerd land voor. Arbeiders namen simpelweg de
controle over de fabrieken en eisten vervolgens losgeld van de
eigenaren. Deze tactiek was opvallend eenvoudig en bleek in de meeste
gevallen zowel winstgevend als plezierig voor de betrokkenen. Één
sit-downstaker schreef: `Ik heb de tijd van mijn leven, iets nieuws,
iets anders, veel eten en muziek.'\footnote{Ibid., p.~293.}

De sit-downstaking bij GM in 1936--37 en de andere gewelddadige
bezettingen van fabrieken in die periode illustreerden een verschijnsel
dat wij in \emph{Blood in the Streets} omschreven als `de uitbuiting van
de kapitalisten door de arbeiders.' Dit was niet de visie die Pete
Seeger in zijn droevige liederen muzikaal verwoordde. Tenzij je ervan
droomt om folkmuzikant te worden in een arbeiderswijk, moet je je niet
laten misleiden door de populaire interpretatie, maar juist de
onderliggende realiteit doorgronden. Waar je in de geschiedenis ook
kijkt, vind je altijd een laag van rationalisaties en schijnpremissen
die de ware megapolitieke fundamenten van systematische afpersing
verhullen. Als je deze rationalisaties als de waarheid beschouwt, raak
je waarschijnlijk niet door wat er werkelijk speelt.

\section{Het ontcijferen van de logica van
afpersing}\label{het-ontcijferen-van-de-logica-van-afpersing}

Om de megapolitieke gevolgen van de huidige verschuiving naar het
informatietijdperk te doorgronden, moet je de holle retoriek doorprikken
en je richten op de échte logica achter geweld in de samenleving. Het is
net alsof je de lagen van een overrijpe ui verwijdert: het kan je aan
het huilen maken, maar je mag je ogen niet sluiten. We beginnen met een
analyse van afpersing op de werkvloer en breiden deze vervolgens uit
naar bredere vraagstukken, zoals het creëren en beschermen van activa en
de aard van de moderne overheid. Veel meer dan de meeste mensen zich
realiseren, waren de welvaart van de overheid en die van vakbonden
direct verbonden met de hefboomwerking die afpersing mogelijk maakte. In
de negentiende eeuw lag die hefboomwerking aanzienlijk lager dan in de
twintigste eeuw, en in het volgende millennium zal zij bijna volledig
verdwijnen.

De gehele werking van overheidsinstellingen en de aard van macht hebben
een ware transformatie ondergaan door de opkomst van microprocessoren.
Dit lijkt in eerste instantie misschien overdreven, maar bekijk het
goed: in de twintigste eeuw groeide de welvaart van regeringen parallel
aan die van vakbonden. Vroeger gebruikten de meeste regeringen beduidend
minder middelen dan de militante verzorgingsstaten waaraan we nu gewend
zijn. Ook waren vakbonden in het verleden vaak kleine, onbeduidende
spelers in het economische landschap. Het vermogen van arbeiders om hun
werkgevers te dwingen om meer dan marktconforme lonen te betalen, hing
immers samen met de megapolitieke voorwaarden waardoor regeringen 40
procent of meer van de economische output als belastingen konden innen.

\subsection{Afpersing op de werkvloer vóór de twintigste
eeuw}\label{afpersing-op-de-werkvloer-vuxf3uxf3r-de-twintigste-eeuw}

De opkomst én ondergang van de afpersing door vakbonden van kapitalisten
verklaar je gemakkelijk door te kijken naar de veranderende
megapolitieke omstandigheden binnen het productieproces. In 1776, toen
Adam Smith \emph{The Wealth of Nations} publiceerde, waren de condities
voor afpersing op de werkvloer zo ongunstig dat het vormen van
`combinaties' van arbeiders om hun loon te verhogen zelden houdbaar
bleek. De meeste productiebedrijven waren klein en door families gerund,
en grootschalige industriële activiteiten stonden nog in de
kinderschoenen. Hoewel geweld niet uitgesloten werd, bood het weinig
hefboomwerking. In de tijd van Smith en tot ver in de negentiende eeuw
beschouwden men in Groot-Brittannië, de Verenigde Staten en andere
common-law-landen vakbonden als illegale combinaties. Adam Smith
omschreef de stakingen als volgt: `Hun gebruikelijke voorwendselen zijn
soms de hoge prijzen van levensmiddelen, soms de enorme winst die hun
meester behaalt met hun werk\ldots{} Zij grijpen altijd terug op luid
tumult, en soms zelfs op schokkend geweld en de meest gruwelijke
beledigingen.'\footnote{Smith, op. cit., p.~75.} Toch behaalden
arbeiders `zeer zelden enig voordeel uit die tumultueuze combinaties,'
behalve `de bestraffing of ondergang van de aanvoerders.'\footnote{Ibid.,
  p.~76.}

Tijdens de negentiende eeuw namen de schaalvoordelen in de industrie en
de omvang van ondernemingen toe. Toch bleven de meeste mensen als boeren
of kleine zelfstandigen werken, en mislukten pogingen tot
vakbondsorganisatie -- zoals Adam Smith beschreef -- vaak in het
niets.\footnote{Ibid.} Pas toen ondernemingen groter werden, veranderde
de juridische en politieke positie van vakbonden. De eerste vakbonden
die daadwerkelijk slaagden, waren ambachtsvakbonden van hoogopgeleide
werknemers, die zich doorgaans zonder grof geweld organiseerden. Zij
namen vaak genoegen met loonstijgingen die in lijn waren met de
marginale kosten om hen te vervangen. Voor ongeschoolde arbeiders liep
dit anders. Zij benutten juist de verschuiving naar grotere
ondernemingen door zich te richten op sectoren die bijzonder vatbaar
waren voor dwang -- ofwel omdat ze op grote schaal opereerden, of omdat
de aard van hun activiteiten eigenaren blootstelde aan fysieke sabotage.
Dit patroon bleek zowel in Newcastle als in Argentinië.\footnote{Een van
  de eerste vakbonden in Argentinië die zich daadwerkelijk
  organiseerden, was de spoorwegvakbond, opgericht in 1887. Zie Carmelo
  Mesa-Lago, \emph{Social Security in Latin America: Pressure Groups,
  Stratification, and Inequality} (Pittsburgh: University of Pittsburgh
  Press, 1978), p.~161.}

Een vroeg voorbeeld van gewelddadig arbeidersactivisme in de Verenigde
Staten is de aanval op het Chesapeake and Ohio‑kanaal in 1834. In
tegenstelling tot de meeste ondernemingen uit de vroege negentiende eeuw
bood het C\&O‑kanaal geen besloten en makkelijk bewaakte operatie.
Volgens het oorspronkelijke plan zou het kanaal 342 mijl lang worden en
een hoogteverschil van 3.000 voet overbruggen tussen de laaggelegen
Potomac en de hooggelegen Ohio. Het uitgraven van zo'n greppel bleek een
enorme klus, die men nooit volledig afrondde. Toch werd er met een grote
ploeg arbeiders aan de slag gegaan en werd al snel duidelijk dat men het
kanaal eenvoudig kon lamleggen. Doordat men niet regelmatig onderhoud
pleegde, konden muskratten het systeem saboteren door onder het sleeppad
te graven. Ook tijdens de werking leidde onzorgvuldig gebruik -- evenals
overstromingen door hevige regenval en aanvaringen met niet gesleepte
boten -- regelmatig tot verwoesting van de sluizen en kanalen. Voor de
stakers bleek het een eenvoudige opgave om de waterweg te blokkeren door
gezonken boten of ander puin in te zetten. Aan het begin van 1834
zorgden rellen tussen rivaliserende groepen Ierse arbeiders op het
C\&O‑kanaal ervoor dat men dit potentiële nadeel wilde omzetten in een
voordeel en het kanaal wilde overnemen. De poging mislukte echter en
eiste vijf levens, nadat president Andrew Jackson federale troepen van
Ft. McHenry inzette om de arbeiders uiteen te drijven.

Ook mijnen en spoorwegen vormden in Amerika al vroeg aantrekkelijke
doelwitten voor vakbondsactivisten. Net als het C\&O‑kanaal bleken zij
erg vatbaar voor sabotage. Saboteurs zorgden er bijvoorbeeld voor dat
zij de schachten van mijnen onderlieten te beschermen, waardoor deze
onderliepen of de ingangen werden afgesloten. Het simpele feit dat men
de muilezels -- waarmee de ertscarrières uit de ondergrondse mijnen
werden getrokken -- uitschakelde, plaatste de eigenaren in een lastige
en onprettige positie. Daarnaast kon men de verspreidliggende
spoorwegbeddingen nauwelijks in de gaten houden. Handlangers van
vakbonden vielen daarom vaak eenvoudig mijnen en spoorwegen aan, met als
gevolg aanzienlijke economische schade. Tijdens de opkomst van
effectieve vakbonden kwamen dergelijke aanvallen regelmatig voor. Deze
acties liepen gewoonlijk op tot een hoogtepunt in perioden waarin
deflatie zorgde voor een stijging van de reële lonen. Wanneer eigenaren
de nominale lonen probeerden aan te passen, leidde dat vaak tot
protesten die uitmondden in geweld. Dit soort voorvallen deed zich
wijdverbreid voor tijdens de depressie die volgde op de Paniek van 1873.

In december 1874 ontvouwde zich een openlijke oorlog in de
anthracietkolenvelden van oostelijk Pennsylvania. Vakbonden
organiseerden een gewelddadige stakingstroep, vermomd als het geheime
genootschap `Ancient Order of Hibernians'. Ook wel de `Molly Maguires'
genoemd, naar een Ierse revolutionair, kreeg deze groep de reputatie de
kolenvelden te terroriseren en mijnwerkers te weerhouden van werk. Men
schreef haar leden zowel sabotage en vernieling van eigendom als
regelrechte moorden en moordaanslagen toe.

Ook onder het personeel van de spoorwegen escaleerde het geweld
herhaaldelijk. In juli 1877 voerden betrokkenen ernstige acties uit die
gericht waren op het vernietigen van de eigendommen van zowel de
Pennsylvania‑ als de Baltimore \& Ohio‑spoorwegen. Werknemers wisselden
van rol, rukten de rails eruit, sloten wagenparken af, haalden
locomotieven buiten gebruik, pleegden sabotage, plunderden treinen en
voerden nog tal van andere acties uit. In Pittsburgh staken aanvallers
de draaihuizen van de Pennsylvania Railroad in brand, waarbij honderden
mensen in het gebouw aanwezig waren. Aan deze incidenten verloren
tientallen mensen hun leven; aanvallers staken tweeduizend spoorwagens
in brand en plunderden ze, en verwoestten daarbij een werkplaats, een
graanelevator en 125 locomotieven. Federale troepen schakelden in om de
orde te herstellen.

Hoewel men deze vroege stakingen vaak interpreteerde als daadkracht van
socialistische en vakbondsactivisten, kwam de publieke steun nauwelijks
op gang. Ondanks de inherente kwetsbaarheid van sectoren zoals mijnen en
spoorwegen stonden de overkoepelende politieke omstandigheden arbeiders
nog in de weg om kapitalisten systematisch af te persen. De
ondernemingen waren simpelweg te klein om een dergelijke systematiek
mogelijk te maken. Omdat kwetsbare sectoren slechts een gering deel van
de bevolking bedienden, konden de voordelen van de dwangmaatregelen
tegen werkgevers zich niet breed verspreiden. Zonder brede steun bleven
de stakingen dan ook onhoudbaar, zeker omdat eigenaren op
overheidsbescherming konden rekenen. Hoewel vakbonden soms via
intimidatie lokale functionarissen weerhielden van het uitvoeren van
gerechtelijke bevelen, slaagden zij zelden. De autoriteiten sloegen
zelfs de meest gewelddadige stakingen meestal binnen enkele dagen of
weken met behulp van militaire middelen neer.

\subsection{Chantage eenvoudig
gemaakt}\label{chantage-eenvoudig-gemaakt}

Voor het informatietijdperk valt er een les te trekken uit het feit dat
pogingen van vakbonden om lonen boven het evenwichtsloon te realiseren,
zelden slagden wanneer ondernemingen nog klein waren. Zelfs in sectoren
die duidelijk vatbaar waren voor sabotage -- zoals kanalen, spoorwegen,
trams en mijnen -- bleek het lastig de controle over te nemen. Dit kwam
niet doordat vakbonden bang waren voor geweld; integendeel, zij grepen
royaal tot geweld, soms zelfs tegen prominente personen. Zo liet een in
de Amerikaanse arbeidersbeweging gevierd voorbeeld -- bekend als `wraak
van de mijnwerkers' -- zien dat een huurmoordenaar, ingehuurd door de
vakbond, met een bom gouverneur Frank Steunenberg van Idaho doodsloeg,
nadat hij zich had verzet tegen een poging van mijnwerkers om
eigendommen in Coeur d'Alene te blokkeren.\footnote{Sloan, op. cit.,
  p.~202. Zie ook S.~S. Boynton, `Miners' vengeance', \emph{`Overland
  Monthly'}, vol.~22 (1893), pp.~303--307} Toch zorgden moorden en
dreigementen er meestal niet voor dat vakbonden de erkenning kregen die
zij nastreefden, nog vóór de opkomst van grootschalige fabrieken en
massaproductiebedrijven in de twintigste eeuw.

Om te doorgronden waarom de omstandigheden voor vakbonden in de
twintigste eeuw drastisch veranderden, kijken we naar de eigenschappen
van de productietechnologie. De snelle opkomst van fabrieksbanen voor
arbeiders in de vroege twintigste eeuw bracht ingrijpende veranderingen
teweeg. Daardoor werden ondernemingen die vooroplopen in de economie
extra kwetsbaar voor afpersing. Sterker nog, de fysieke kenmerken van de
industriële technologie leken de arbeiders bijna aan te moedigen om de
kapitalisten met dwang te ontpersen. Overweeg het volgende:

\begin{enumerate}
\def\labelenumi{\arabic{enumi}.}
\item
  \emph{De meeste industriële producten bevatten een hoog gehalte aan
  natuurlijke hulpbronnen.}\\
  Daardoor concentreerde de productie zich op slechts enkele locaties,
  vergelijkbaar met mijnen die enkel op ertslagen voorkomen. Fabrieken
  die strategisch lagen nabij vervoersknooppunten en gemakkelijk toegang
  boden tot leveranciers van onderdelen en grondstoffen, behaalden
  aanzienlijke operationele voordelen. Hierdoor kregen organisaties
  zoals overheden en vakbonden een gemakkelijke kans om een deel van die
  voordelen voor zichzelf te benutten.
\item
  \emph{Het toenemen van schaalvoordelen leidde tot reusachtige
  ondernemingen.}\\
  In de beginjaren van de negentiende eeuw waren fabrieken nog relatief
  klein. Toen in de twintigste eeuw de lopende band werd geïntroduceerd,
  namen de schaalvoordelen snel toe, waardoor zowel de omvang als de
  kosten van vooraanstaande productievoorzieningen al snel stegen. Dit
  maakte deze ondernemingen op diverse vlakken extra kwetsbaar als
  doelwit. Bovendien gaan aanzienlijke schaalvoordelen vaak gepaard met
  lange productcycli. Lange productcycli zorgen voor stabielere markten
  en lokken daardoor roofzuchtige acties tegen ondernemingen uit,
  aangezien zij langetermijnwinst suggereren.
\item
  \emph{Het aantal concurrenten in de toonaangevende industrieën daalde
  fors. In de industriële periode was het niet ongebruikelijk dat
  slechts een handvol bedrijven streden om markten ter waarde van
  miljarden dollars}. Dit maakte hen kwetsbaar voor afpersing door
  vakbonden. Het valt immers veel gemakkelijker om vijf bedrijven onder
  druk te zetten dan vijfduizend. De concentratie in de industrie
  vergemakkelijkte dit proces: ondernemingen die gedwongen werden lonen
  boven het marktconforme niveau te betalen, ondervonden nauwelijks
  felle concurrentie van bedrijven die niet met zulke hoge loonkosten
  werden belast. Daardoor konden vakbonden een aanzienlijk deel van de
  winst afpersen zonder de bedrijven direct in faillissement te storten.
  Immers, als werkgevers routinematig failliet gingen bij het betalen
  van bovengemiddelde lonen, bleef er nauwelijks winst over.
\item
  \emph{De kapitaaleisen voor vrije investeringen stegen in verhouding
  tot de omvang van de onderneming}. Dit vergrootte niet alleen de
  kwetsbaarheid van het kapitaal en deed de kosten voor het sluiten van
  fabrieken stijgen, maar maakte het ook vrijwel onmogelijk dat een
  moderne fabriek in handen bleef van één individu of familie -- tenzij
  via een erfenis van iemand die het bedrijf kleinschalig had opgestart.
  Om de enorme uitgaven voor machines en het exploiteren van een grote
  fabriek te dekken, moest het vermogen van honderden of duizenden
  personen via de kapitaalmarkten worden gebundeld. Hierdoor kregen de
  gefragmenteerde en vrijwel anonieme eigenaren minder mogelijkheden om
  hun eigendomsrechten te verdedigen. Zij waren genoodzaakt hun
  vertrouwen te stellen in professionele managers, die doorgaans slechts
  een minimaal belang in de uitstaande aandelen van het bedrijf bezaten.
  Dit vertrouwen op ondergeschikte managers verzwakte het verzet van
  ondernemingen tegen afpersing, want de managers hadden weinig prikkels
  om met leven en gezondheid het eigendom van de onderneming te
  beschermen -- hun inzet benaderde zelden de felle strijdlust die men
  vaak zag bij eigenaren van slijterijen en andere kleine ondernemingen
  wanneer hun eigendom onder druk kwam te staan.
\item
  \emph{Grotere bedrijfsgrootte hield ook in dat een groter deel van de
  beroepsbevolking in een steeds kleiner aantal ondernemingen werkte dan
  ooit tevoren}. Soms bood één bedrijf werk aan tienduizenden
  werknemers. In militaire termen werden de eigenaren en managers
  verpletterd door het aantal medewerkers op de lagere niveaus;
  verhoudingen van dertig tegen één of slechter waren gangbaar. Dit
  nadeel verergerde met de groei van de onderneming, omdat massa's
  werknemers gemakkelijker anoniem geweld konden gebruiken. Onder deze
  omstandigheden onderhielden arbeiders doorgaans weinig directe
  contacten met de fabriekeigenaren, wat het voor hen onmiskenbaar
  eenvoudiger maakte het belang van de eigendomsrechten terzijde te
  schuiven.
\item
  \emph{Gecentraliseerde werkgelegenheid binnen een beperkt aantal
  ondernemingen vormde een wijdverspreid sociaal fenomeen}. Dit
  versterkte de megapolitieke voordelen die vakbonden genoten,
  vergeleken met de negentiende eeuw in Amerika, toen de meeste mensen
  zelfstandig werkten of in kleine bedrijven actief waren. In 1940 had
  slechts 6 procent van de Amerikaanse beroepsbevolking een baan in de
  industrie.\footnote{Benjamin Schwartz, `American inequality: its
    history and scary future', \emph{`New York Times'}, 19 december
    1995, p.~A25} Hierdoor verspreidde de steun voor loonstijgingen door
  middel van afpersing zich onder velen die er mogelijk baat bij zagen.
  Dit illustreerde een onderzoek uit 1938--1939 naar de opvattingen van
  1.700 inwoners van Akron, Ohio, over ondernemingsbezit. De enquête
  toonde aan dat 68 procent van de CIO Rubber Workers weinig tot geen
  sympathie koesterde voor het idee van ondernemingsbezit, terwijl
  slechts één procent sterke steun betuigde aan de rechten van
  ondernemingsbezit.\footnote{MeElvaine, op. cit., p.~293} Daarentegen
  kwam geen enkele zakenman -- zelfs niet een kleine ondernemer -- in
  aanmerking voor de categorie `sterke afwijzing van ondernemingsbezit',
  want 94 procent werd beoordeeld met extreem hoge steun voor
  eigendomsrechten.\footnote{Ibid.}
\item
  \emph{Lopende bandtechnologie werkte per definitie sequentieel}. Omdat
  het volledige productieproces afhankelijk was van de continue beweging
  en assemblage van onderdelen in een vaste volgorde, ontstonden extra
  kwetsbaarheden voor verstoringen. In feite functioneerde de lopende
  band als een spoorweg binnen de fabriek: als het spoor geblokkeerd
  raakte of een enkel onderdeel ontbrak, kwam het hele productieproces
  abrupt tot stilstand.
\item
  \emph{Lopende bandtechnologie standaardiseerde het werk}. Daardoor
  daalde de variabiliteit in de uitkomsten, zelfs als mensen met
  uiteenlopende vaardigheden met dezelfde gereedschappen werkten.
  Sterker nog, een belangrijk doel bij het ontwerpen van fabrieken was
  het creëren van een systeem waarbij zowel een genie als een mindere
  werknemer op de lopende band hetzelfde product afleverden. Wat men
  `domme' machines noemt, werd zodanig ontworpen dat ze slechts één type
  output konden voortbrengen. Daardoor hoefde zelfs de koper van een
  Cadillac niet na te vragen wie de lijnarbeiders waren die zijn
  voertuig in elkaar zetten. Alle producten werden bedoeld identiek te
  zijn, ongeacht de verschillen in vaardigheid en intelligentie van de
  arbeiders.
\end{enumerate}

Het feit dat ongeschoolde arbeiders op de lopende band precies hetzelfde
product konden maken als meer bekwame personen, droeg bij aan de
egalitaire opvatting dat ieders economische bijdrage gelijk is.
Ondernemerskwaliteiten en mentale inspanning leken minder belangrijk, en
de magie van moderne productie lag schijnbaar in de machines zelf.
Hoewel niet iedereen deze machines daadwerkelijk had kunnen ontwerpen,
leken ze intellectueel toegankelijk voor bijna iedereen. Dit versterkte
het idee dat fabriekseigenaren ongeschoolde arbeid `uitbuitten', zodat
zij zelf buiten de vergelijking bleven zonder dat een ander benadeeld
werd. `We hebben geleerd dat we de fabriek kunnen overnemen,' verwoordde
een streiker van \emph{GM}. `We wisten al hoe we ze moesten runnen. Als
\emph{General Motors} niet oppast, leggen we twee en twee bij
elkaar.'\footnote{Ibid.}

Deze eigenschappen van de industriële technologie zorgden er unaniem
voor dat vakbonden werden opgericht om de kwetsbaarheid voor afpersing
te benutten en dat er grotere overheden ontstonden die profiteerden van
de hoge belastingen op grootschalige industriële faciliteiten. Dit
gebeurde niet incidenteel, maar overal waar grootschalige industrie voet
aan de grond kreeg. Steeds opnieuw ontstonden vakbonden die geweld
inzetten om lonen boven het marktconforme niveau af te dwingen. Zij
konden dit realiseren omdat industriële fabrieken doorgaans duur,
opvallend, immobiel en kostbaar zijn; ze bleven nauwelijks verborgen en
verplaatsen kon niet. Iedere keer als ze buiten gebruik raakten, zorgde
dat ervoor dat hun torenhoge kosten niet meer afgeschreven konden
worden.

Al deze factoren maakten hen tot makkelijke prooi voor dwingende
afpersing -- een feit dat in de geschiedenis van de vakbonden veel
duidelijker naar voren komt dan de heersende ideologie van de twintigste
eeuw doet vermoeden. De bekende econoom Henry Simons verwoordde in 1944
de kwestie als volgt:

`Arbeidsorganisatie zonder grote machten van dwang en intimidatie is een
onwerkelijke abstractie. Vakbonden beschikken nu over zulke machten; dat
hadden ze altijd en zullen ze altijd hebben, zolang zij in de huidige
vorm blijven bestaan. Wanneer de macht klein of niet stevig verankerd
is, moet zij openlijk en uitgebreid worden uitgeoefend; wanneer zij
groot en onbevochten is, lijkt zij op de macht van de overheid: vol
vertrouwen in bezit, met respect behandeld en zelden opvallend
tentoongesteld.'\footnote{Henry C. Simons, `Enkele overpeinzingen over
  syndicalisme', \emph{Journal of Political Economy}, maart 1944, p.~22.}

Hoe nauwkeurig Simons' analyse ook is, hij sloeg wel op één essentieel
punt de plank mis.

Hij ging ervan uit dat vakbonden altijd over wat hij omschreef als
`grote machten van dwang en intimidatie' zouden beschikken. In
werkelijkheid verdwijnen vakbonden -- niet alleen in de Verenigde Staten
en Groot-Brittannië, maar ook in andere volwassen industriële
samenlevingen. Simons zag over het hoofd dat de verschuiving naar een
informatiesamenleving de megapolitieke omstandigheden ingrijpend heeft
veranderd, waardoor de veiligheid van eigendom sterk toeneemt -- iets
wat zelfs veel vakbondsorganisatoren niet lijken te begrijpen. Bovendien
ondermijnt microtechnologie op subtiele wijze de afpersing waarop de
verzorgingsstaat leunt, doordat het in het commerciële domein geheel
andere prikkels genereert dan in het industriële tijdperk.

\begin{enumerate}
\def\labelenumi{\arabic{enumi}.}
\item
  \emph{Informatietechnologie verbruikt nauwelijks natuurlijke
  hulpbronnen}. Het biedt nauwelijks voordelen op het gebied van
  locatie. De meeste informatietechnologie is namelijk zeer
  mobiliseerbaar: omdat het niet gebonden is aan een vaste plek, stroomt
  het gemakkelijk ideeën, mensen en kapitaal door de markt. Zo kon
  General Motors haar drie assemblagelijnen in Flint, Michigan, niet
  simpelweg inpakken en mee verplaatsen -- iets wat een softwarebedrijf
  wel kan. Ondernemers downloaden hun algoritmen op draagbare computers
  en vliegen met de volgende vlucht naar een nieuwe bestemming.
  Dergelijke bedrijven hebben dan ook een extra stimulans om hoge
  belastingen of vakbondseisen voor buitensporige lonen te omzeilen.
  Kleinere ondernemingen hebben doorgaans te maken met veel meer
  concurrentie. Als tientallen of zelfs honderden rivalen je klanten
  proberen af te snoepen, kun je het je niet veroorloven politici of
  werknemers meer te betalen dan zij daadwerkelijk waard zijn. Als je
  dit in je eentje zou trachten, lopen je kosten hoger op dan die van je
  concurrenten en ben je snel failliet. Het ontbreken van
  noemenswaardige operationele voordelen op een specifieke locatie
  betekent immers dat dwangorganisaties -- zoals overheden en vakbonden
  -- minder invloed hebben om deze voordelen in hun voordeel te
  benutten.
\item
  \emph{Informatietechnologie verlaagt de bedrijfsschaal}. Hierdoor
  worden bedrijven kleiner, wat weer betekent dat er meer concurrentie
  ontstaat. Deze toegenomen concurrentie verkleint de kans op afpersing,
  omdat het aantal doelwitten dat je fysiek kunt controleren om lonen of
  belastingtarieven kunstmatig boven het competitieve niveau te drijven,
  afneemt. De sterke daling in de gemiddelde omvang van ondernemingen --
  mede dankzij informatietechnologie -- heeft al geleid tot minder
  mensen in ondergeschikte functies. In de Verenigde Staten schatten
  diverse rapporten dat in 1996 maar liefst 30 miljoen mensen werkzaam
  waren in hun eigen bedrijf. Het is dan ook vanzelfsprekend dat die 30
  miljoen niet gezamenlijk in staking zullen gaan tegen hun eigen
  belangen. Nog minder aannemelijk is dat de extra miljoenen werknemers
  in kleine bedrijven, waar slechts een handje collega's werken, hun
  werkgevers zouden dwingen om lonen boven het marktniveau te betalen.
  In dit informatietijdperk beschikken werknemers die via afpersing hun
  lonen willen verhogen niet over die overweldigende aantallen die in
  traditionele fabrieken een militair effect hadden. Hoe minder mensen
  in een bedrijf werken, hoe minder mogelijkheden er ontstaan voor
  collectieve, anonieme intimidatie. Alleen al daardoor vormen
  tienduizend werknemers, verspreid over vijfhonderd bedrijven, een veel
  kleinere bedreiging voor het eigendom dan wanneer tienduizend
  werknemers in één bedrijf samenkomen -- ook als de verhouding tussen
  werknemers en eigenaars of management identiek zou zijn.
\item
  \emph{De afname in de omvang van ondernemingen betekent tevens dat
  pogingen om bovenmarktlonen af te dwingen, minder snel op brede
  maatschappelijke steun kunnen rekenen, in tegenstelling tot wat in het
  industriële tijdperk vaak gebeurde.} Vakbonden die er op uit zijn
  werkgevers af te persen, bevinden zich al sneller in de positie van de
  kanaalarbeiders, spoorwegarbeiders en mijnwerkers uit de negentiende
  eeuw. Zelfs als nog enkele bedrijven met grootschalige economische
  voordelen als relieken uit het industriële tijdperk blijven bestaan,
  spelen zij hun rol in een milieu waarin kleine ondernemingen de norm
  zijn. De dominantie van deze kleine bedrijven en kleinschalige
  producenten duidt op een breder maatschappelijk draagvlak voor
  eigendomsrechten -- ook als de roep om inkomensherverdeling
  onveranderd blijft.
\item
  \emph{Informatietechnologie verlaagt de kapitaalkosten, wat bovendien
  de concurrentie stimuleert door ondernemerschap aan te moedigen en
  meer mensen de mogelijkheid te geven zelfstandig te werken.} Lagere
  kapitaaleisen verlagen niet alleen de toetredingsdrempels, ze drukken
  ook de `uittredingsdrempels' omlaag. Met andere woorden: ondernemingen
  hebben vermoedelijk een kleinere verhouding tussen hun activa en
  inkomsten, waardoor zij minder goed in staat zijn verliezen op te
  vangen. Bovendien kloppen zij minder vaak bij banken aan voor
  leningen, omdat ondernemingen in het informatietijdperk doorgaans over
  minder fysieke activa beschikken die als onderpand kunnen dienen.
\item
  \emph{Informatietechnologie verkort de productcyclus.} Daardoor
  verouderen producten sneller, en zijn de extra winsten die via het
  afpersen van bovenmarktlonen worden behaald vaak van korte duur. In
  sterk concurrerende markten kunnen te hoge lonen er namelijk direct
  toe leiden dat banen verdwijnen en bedrijven failliet gaan. Het
  najagen van tijdelijk hogere lonen, zelfs als dat ten koste gaat van
  de zekerheid van je baan, is vergelijkbaar met het in brand steken van
  je meubels om je huis slechts een paar graden warmer te maken.
\item
  \emph{Informatietechnologie werkt niet op een lineaire wijze, maar
  gelijktijdig en verspreid.} In tegenstelling tot een lopende band
  ondersteunt informatietechnologie meerdere processen tegelijk. Ze
  verspreidt activiteiten over netwerken, waardoor redundantie en
  vervanging tussen werkplekken mogelijk worden -- werkplekken die in de
  duizenden of zelfs miljoenen kunnen lopen en overal ter wereld
  gevestigd zijn. Hierdoor kunnen mensen samenwerken zonder elkaar ooit
  fysiek te ontmoeten. Naarmate virtual reality en videoconferenties
  geavanceerder worden, versnelt de trend naar verspreide functies en
  telewerken. Dit vormt het informatiele tijdperk-equivalent van
  `putting out', een systeem dat ooit de macht van de middeleeuwse
  gilden doorbrak. Het feit dat steeds minder mensen samen in rokerige
  fabrieken werken, ontnemen de arbeiders niet alleen een belangrijk
  hefboommiddel bij het afdwingen van hogere lonen van kapitalisten,
  maar maakt het ook steeds moeilijker om het onderscheid te zien tussen
  afperspraktijken en racketeering -- de vorm van afpersing die op de
  werkvloer ooit als acceptabel werd beschouwd. Tot nu toe mochten
  alleen mensen die gezamenlijk in een gedeelde werkomgeving werkten,
  geweld inzetten om hun inkomsten te verhogen. Maar wanneer de
  `werkplek' niet langer een centrale ontmoetingsplek vormt en de meeste
  werkzaamheden worden uitbesteed aan onderaannemers en telewerkers,
  vervaagt het onderscheid tussen een afperskartel en pogingen om geld
  af te persen van cliënten of `werkgevers.'
\end{enumerate}

Is een telewerker die extra contant geld eist onder de dreiging de
bedrijfscomputers met een virus te besmetten te worden bestempeld als
een stakende werknemer, of als een internetafperser? Of hij nu het een
of het ander is, uiteindelijk doet het nauwelijks uit. De reactie van de
getroffen bedrijven zal naar alle waarschijnlijkheid ongeveer hetzelfde
zijn. Technische maatregelen tegen informatiesabotage -- zoals
verbeterde encryptie en sterkere netwerkbeveiliging, die externe hackers
af moeten weren -- zouden tevens voorkomen dat een ontevreden werknemer
of onderaannemer schade veroorzaakt bij partijen waarmee hij regelmatig
of sporadisch zaken doet. Natuurlijk kun je stellen dat een werknemer of
thuiswerker altijd naar kantoor kan gaan om daar een traditionele
staking te organiseren. Maar zelfs dat blijkt in het informatietijdperk
minder eenvoudig dan het lijkt. De mogelijkheid van
informatietechnologie om lokale grenzen te overschrijden en economische
activiteiten te verspreiden, houdt in dat werknemers en werkgevers voor
het eerst niet eens binnen hetzelfde rechtsgebied hoeven te opereren.
Hierbij gaat het niet om het verschil tussen de stadsdelen Mayfair en
Peckham, maar om het contrast tussen werkgevers op Bermuda en
thuiswerkers in New Delhi.

Bovendien, als mensen uit India geïntrigeerd raken door verhalen over de
grote GM-stakingen van 1936--37 en besluiten naar Bermuda te reizen om
te picketten, vinden zij bij aankomst wellicht helemaal geen fysiek
kantoor. \emph{Chiat/Day}, een groot reclamebureau, is immers al
begonnen met het ontmantelen van zijn hoofdkantoor. Werknemers -- of
beter gezegd, onderaannemers -- houden via doorschakeling van oproepen
en via het internet contact. Wanneer het tijd is om talentteams samen te
stellen voor klantprojecten, huren zij vergaderruimtes in hotels; zodra
een project is afgerond, checken zij weer uit. Doordat microprocessing
het productieproces loskoppelt van de starre assemblagelijn, neemt de
invloed van dwingende instituten als vakbonden en overheden aanzienlijk
af. Als de assemblagelijn te vergelijken zou zijn met een spoorweg
binnen de muren van een fabriek die makkelijk door een bezettingsstaking
veroverd kan worden, dan is cyberspace een grenzeloos domein zonder
fysieke aanwezigheid -- het kan immers niet met geweld worden bezet of
gegijzeld voor losgeld. Werknemers die geweld als hefboom inzetten om
een hoger loon te bedingen, komen in het informatietijdperk veel minder
ver dan tijdens de bezettingsstaking bij General Motors in 1936--37.

\begin{enumerate}
\def\labelenumi{\arabic{enumi}.}
\setcounter{enumi}{6}
\tightlist
\item
  \emph{Microprocessing individualiseert werk, industriële technologie
  standaardiseerde werk}
\end{enumerate}

Wie dezelfde instrumenten gebruikt, behaalt altijd hetzelfde resultaat.
Microtechnologie vervangt steeds vaker `domme' machines door
intelligente systemen die op verrassend uiteenlopende manieren output
leveren. Die toegenomen variabiliteit in wat mensen met dezelfde
hulpmiddelen produceren, heeft ingrijpende gevolgen -- vele daarvan
bespreken we in de komende hoofdstukken. Een belangrijk punt is dat
wanneer de output varieert, ook de inkomens uiteenlopen. In domeinen
waarin vaardigheden sterk verschillen, creëert doorgaans een klein
aantal mensen het merendeel van de waarde. Dit kenmerk zien we vaak in
sterk concurrerende markten. Een duidelijk voorbeeld komt uit de sport:
wereldwijd spelen miljoenen jongeren allerlei varianten van voetbal,
maar liefst 99 procent van het geld dat wordt besteed aan het bekijken
van wedstrijden gaat naar de optredens van een handvol sterren. Ook in
de filmwereld zijn talloze aspirant-acteurs en -actrices actief, maar
slechts een beperkt aantal doorbreekt als ster. Evenzo worden jaarlijks
tienduizenden boeken uitgegeven, terwijl de meeste royalty's naar een
klein aantal bestverkopende auteurs gaan die hun lezers écht weten te
vermaken. Helaas behoren wij niet tot die gelukkigen.

De enorme variatie in wat mensen met dezelfde middelen produceren,
bemoeilijkt bovendien het afdwingen van een uniforme verdeling van de
opbrengsten. Wanneer een klein deel van de deelnemers het grootste
gedeelte van de waarde genereert, is het vrijwel onmogelijk dat zij
beter afkomen door een afgedwongen, gemiddelde inkomstenverdeling. Zo
kan één softwareprogrammeur een algoritme ontwikkelen voor het aansturen
van een robot dat miljoenen waard is, terwijl een ander met dezelfde
apparatuur een programma schrijft dat niets oplevert. De veel
productievere programmeur koppelt zijn inkomen evenmin aan dat van zijn
collega, net zoals Tom Clancy er niet voor zou kiezen om zijn
boekroyalty's met die van anderen te middelen.

Al in de vroege dagen van de informatierevolutie werd duidelijker dan in
1975 dat vaardigheden en mentale capaciteiten cruciaal zijn voor de
economische productie. Daarmee valt de vroegere rechtvaardiging van
arbeiders om kapitalisten af te persen -- zoals in de industriële
periode gangbaar -- tegen de grond. De gedachte dat ongeschoolde arbeid
daadwerkelijk de waarde voortbrengt waardoor onevenredig veel in de
zakken van kapitalisten en ondernemers belandt, behoort inmiddels tot
het verleden. In de informatietechnologie is zo'n verhaal simpelweg
onhoudbaar. Wanneer een programmeur aan de slag gaat, bestaat er een
directe koppeling tussen zijn vaardigheden en zijn product; daar is geen
twijfel over mogelijk. Het is vanzelfsprekend dat een analfabeet of
halfgeleide niet in staat is een computer te programmeren, en daarom kan
hij niets claimen van de waarde die in programma's wordt verkocht die
door anderen zijn geschreven. Daarom hoor je tegenwoordig vooral dat
werknemers -- met name schoonmakers -- klagen over uitbuiting.

Informationstechnologie laat zien dat het probleem van mensen met
beperkte vaardigheden niet zozeer is dat hun productieve krachten
oneerlijk worden uitgebuit, maar dat zij vrezen geen werkelijke
economische bijdrage te kunnen leveren. Zoals Kevin Kelly in \emph{Out
Of Control} suggereert, kan het `upstart'-autobedrijf van het
informatietijdperk wel eens het geesteskind zijn van `een dozijn
mensen'. Deze ondernemers kunnen bijna al hun onderdelen uitbesteden én
toch auto's produceren die perfect op maat zijn gemaakt en volledig
aansluiten bij de wensen van hun kopers -- veel preciezer dan alles wat
we tot nu toe in Detroit of Tokyo hebben gezien: ``Auto's, die ieder
klantgericht op maat worden gebouwd, worden besteld via een netwerk van
klanten en verzonden zodra ze gereed zijn. Mallen voor autocarrosserieën
worden razendsnel gevormd met computersaangestuurde lasers en gevoed met
ontwerpen die voortkomen uit klantfeedback en gerichte marketing. Een
flexibele robotlijn assembleert de auto's, terwijl het repareren en
verbeteren van robots wordt uitbesteed aan een gespecialiseerd
robotbedrijf.'' \footnote{Kelly, op. cit., pp.~191--192.}

\subsection{Gereedschappen met een
stem}\label{gereedschappen-met-een-stem}

Steeds meer ongeschoold werk nemen geautomatiseerde machines, robots en
computationele systemen over, zoals digitale assistenten. Toen
Aristoteles slaven omschreef als `gereedschappen met een stem', bedoelde
hij eigenlijk mensen. In de nabije toekomst zullen zulke `gereedschappen
met een stem', vergelijkbaar met de sprekende geesten uit fabels, kunnen
praten, instructies opvolgen en zelfs complexe opdrachten uitvoeren. De
rap stijgende rekenkracht heeft al geleid tot de eerste toepassingen van
spraakherkenning, zoals handsfree telefoons en computers die op basis
van mondelinge instructies wiskundige berekeningen maken. Al eind 1996
introduceerde men computers die spraak naar tekst omzetten. Naarmate de
mogelijkheden voor patroonherkenning verbeteren, zullen computers
gekoppeld aan spraaksynthesizers via netwerken talloze taken overnemen
die voorheen werden uitgevoerd door telefonisten, secretaresses,
reisagenten, administratieve assistenten, schaakkampioenen,
schadebehandelaars, componisten, obligatiehandelaren,
cyberoorlogspecialisten, wapenanalisten of zelfs straatwijze flirters
die 900-nummers beantwoorden.

Michael Mauldin van de \emph{Carnegie-Mellon Universiteit} heeft een
kunstmatige persoonlijkheid gecreëerd, genaamd Julia, die bijna iedereen
voor de gek houdt tijdens haar online gesprekken. Volgens persberichten
is Julia een geestige dame die haar leven leidt in een online
rollenspel. Ze is slim, grappig en houdt van flirten. Bovendien lijkt ze
een hockeyexpert te zijn en verzint ze in een oogwenk de perfecte
sarcastische opmerking. Toch is Julia geen echte dame; ze is een bot,
een kunstmatige intelligentie die uitsluitend in de digitale wereld
bestaat. \footnote{Gayle N. Hanson, `Een meeslepend verslag van ``het
  leven'' in de postmodernistische cyberspace', \emph{Washington Times},
  24 december 1995, p.~B7.} De indrukwekkende vooruitgang in het
programmeren van kunstmatige intelligentie en digitale assistenten doet
weinig twijfel bestaan over de komst van nog vele praktische
toepassingen. Dit zal ongetwijfeld ingrijpende geopolitieke gevolgen
hebben.

\subsection{Het individu als ensemble}\label{het-individu-als-ensemble}

De ontwikkeling van hulpmiddelen met stemfunctionaliteit voor
uiteenlopende toepassingen biedt de mogelijkheid dat een individu
tegelijk aan meerdere activiteiten deelneemt. Een persoon blijft niet
langer één en dezelfde, maar fungeert potentieel als een verzameling van
tientallen of zelfs duizenden parallelle activiteiten, uitgevoerd door
intelligente agenten. Dit vergroot niet alleen het productieve vermogen
van de meest getalenteerde mensen aanzienlijk, maar maakt het
individuele, soevereine optreden op militair gebied ook veel
formidabeler dan ooit tevoren.

Niet alleen kan een individu zijn activiteiten exponentieel uitbreiden
door een praktisch onbeperkt aantal intelligente agenten in te
schakelen, maar hij of zij kan zelfs na de dood blijven handelen. Voor
het eerst komt iemand in staat uitgebreide taken uit te voeren, ook als
hij biologisch gestorven is. Een vijand in oorlog of een crimineel kan
door iemand simpelweg te vermoorden de vergeldingsmogelijkheden niet
langer volledig uitschakelen. Dit behoort tot de meest revolutionaire
vernieuwingen in de logica van geweld in de hele geschiedenis.

\subsection{Inzichten voor het
informatie-tijdperk}\label{inzichten-voor-het-informatie-tijdperk}

De ingrijpendste veranderingen in ons leven betreffen variabelen die
bijna niemand opmerkt. Met andere woorden, we gaan er van uit dat
bepaalde grootheden al eeuwen -- zo niet al honderden generaties --
vrijwel onveranderd blijven. In het grootste deel van de geschiedenis --
zo niet gedurende het grootste deel van het menselijk bestaan -- bleef
de balans tussen bescherming en afpersing binnen een beperkt bereik,
waarbij afpersing altijd de overhand kreeg. Dit lijkt nu echter te
veranderen. Informatietechnologie vormt de basis voor een fundamentele
verschuiving in de factoren die bepalen wat het kost en wat men wint bij
het gebruik van geweld. Dat intelligente agenten beschikbaar komen om
onderzoek te doen naar en eventueel vergelding te nemen op degenen die
geweld inzetten, is slechts een tipje van de sluier van dit nieuwe
perspectief op bescherming. Vijfentwintig jaar geleden zou een uitspraak
als `Als je me doodt, veeg ik het geld van je bankrekeningen weg en
doneer ik het aan liefdadigheidsinstellingen in Nepal' louter de tirade
van een gestoorde hebben doen klinken. Sinds de eeuwwisseling is het
echter mogelijk dat zo'n bewering werkelijkheid wordt. Of dit
daadwerkelijk een reële dreiging vormt, hangt af van factoren als tijd
en plaats. Zelfs als de rekeningen van een potentiële misdadiger
ondoordringbaar blijken, kan een leger van intelligente agenten
ongetwijfeld op andere manieren aanzienlijke schade aanrichten als
vergelding voor een misdrijf. Denk er eens goed over na.

\subsection{Nieuwe alternatieven in
bescherming}\label{nieuwe-alternatieven-in-bescherming}

Dit is slechts één van de talloze methoden om de
beschermingsmogelijkheden, die de informatietechnologie biedt, te
verbeteren. Veel van deze oplossingen ondermijnen de bijna
monopolistische positie op het gebied van bewaking en afpersing die
overheden de afgelopen twee eeuwen met zekerheid hebben genoten. Ook nog
voor de komst van moderne technologische snufjes bestonden er
alternatieven voor bescherming, die niet volledig in handen waren van de
overheid.

Wie zich bedreigd voelt, kan er eenvoudig voor kiezen om weg te
vluchten. Vroeger, toen de wereld nog jong was en de mogelijkheden
grenzeloos leken, koos men vaak voor die uitweg. Als mensen bang zijn
iets te verliezen door diefstal of vandalisme, sluiten zij verzekeringen
af om deze risico's op te vangen. Zelfs vloeken en betoveringen -- al
zijn het minder krachtige middelen -- hebben al menig leven gered en
diefstallen voorkomen, vooral in samenlevingen waar rovers vatbaar zijn
voor bijgeloof. Bovendien kun je waardevolle bezittingen beschermen door
ze te verstoppen. Je kunt ze begraven, met sloten beveiligen, achter
hoge muren plaatsen of uitrusten met sirenes en elektronische
bewakingssystemen. Toch blijkt het verbergen van personen en eigendommen
in de praktijk vaak niet haalbaar.

Van alle beschermingsmethoden die door de geschiedenis heen zijn
toegepast, domineert er één: het beantwoorden van geweld met geweld. Zo
roep je een macht op die iedereen overweldigt die jou probeert aan te
vallen of je eigendom te beroven. Maar waar vind je zo'n dienst, en hoe
overtuig je iemand om zijn leven en ledematen op het spel te zetten om
je te verdedigen tegen schurken die met geweld op zoek zijn naar een
kans om jou te treffen? Vaak bieden naaste familieleden uitkomst. Ook
clan- en stamgroepen traden op als informele politie, die bij aanvallen
met bloedige vergeldingen reageerden. Huurlingen en particuliere
bewakers werden soms ingezet om aanvallen af te weren, maar zij bleken
niet altijd even effectief als gewenst. De nieuwe, slimme actoren van
het informatietijdperk -- al beperken hun activiteiten zich grotendeels
tot de cyberwereld -- bieden een fris alternatief. Hun loyaliteit staat
buiten kijf, in scherp contrast met de vaak twijfelachtige trouw van
huurlingen, particuliere bewakers en zelfs verre familieleden.

\subsection{De paradoxen van de macht}\label{de-paradoxen-van-de-macht}

Geweld inzetten om jezelf te verdedigen tegen geweld zit vol
tegenstrijdigheden. Onder de huidige omstandigheden bezit iedere groep
of instantie die je inschakelt om je leven en bezittingen te beschermen,
ook de capaciteit om zelf aanvallen uit te voeren. Dit brengt een
probleem met zich mee waar geen simpele oplossing voor bestaat. Bij
economische diensten zorgt concurrentie er meestal voor dat aanbieders
de wensen van hun klanten respecteren. Maar als het op geweld aankomt,
leidt directe concurrentie vaak tot perverse, escalerende situaties. In
het verleden resulteerde dit doorgaans in een toename van geweld.
Wanneer twee potentiële beschermingsorganisaties hun strijdkrachten
tegen elkaar inzetten, lijkt dat eerder op een burgeroorlog dan op
bewaking. Als je bescherming zoekt tegen geweld, verwacht je dat de
dreiging afneemt, en niet juist escaleert, en dat niemand zijn klanten
afpersen of beroven.

``\ldots tijdens de tijd dat mensen leven zonder een gemeenschappelijke
macht die hen allen in vrees houdt, bevinden zij zich in die toestand
die men oorlog noemt: zo'n oorlog waarin elke mens tegen elke mens
strijdt en waarin men geen andere zekerheid geniet dan wat hun eigen
kracht en vindingrijkheid hen verschaft'' -- THOMAS HOBBES

\subsection{Monopolie en anarchie}\label{monopolie-en-anarchie}

Anarchie -- oftewel `de oorlog van allen tegen allen', zoals Hobbes het
omschreef -- bleek zelden een bevredigende situatie te bieden. Lokale
rivaliteit in het gebruik van geweld dwong mensen gewoonlijk tot hogere
kosten voor bescherming, waardoor er minder overbleef voor andere
levensgenietingen. Soms suggereerden vrijdenkende marktvoorstanders dat
marktmechanismen op zich voldoende zouden zijn om eigendomsrechten te
handhaven en het leven te beschermen, zonder dat een soevereine macht
daarvoor nodig was.\footnote{Een beknopte inleiding op het academische
  onderzoek naar anarchie vindt u in Gordon Tullock (red.),
  \emph{Explorations in the Theory of Anarchy} (Blacksburg, Va.:
  Virginia Polytechnic Institute and State University, 1972). Zie ook
  Murray N. Rothbard, \emph{Power and Market: Government and the
  Economy} (Menlo Park, Calif., 1970) en Robert Nozick, \emph{Anarchy,
  State and Utopia} (New York: Basic Books, 1974).} Hoewel sommige
analyses elegant waren, tonen de feiten aan dat het vrije marktmodel
voor politie- en justitiediensten niet standhield onder de megapolitieke
omstandigheden van het industrialisme. Enkel primitieve samenlevingen,
waarin gedrag strikt gestandaardiseerd was en bevolkingen klein en
homogeen bleven, slaagden erin te overleven zonder overheden die met
geweld een lokaal monopolie op bescherming uitoefenden.

Anarchistische samenlevingen die verder gingen dan de basis van de
jagers- en verzamelaarsstam komen zelden voor en dateren vrijwel altijd
uit vervlogen tijden. Zij behoorden tot de eenvoudigste economische
systemen van geïsoleerde regenwatergemeenschappen -- denk aan de Kafirs
in het pre-islamitische Afghanistan, enkele Ierse stammen in de donkere
middeleeuwen, Indiase groepen in Brazilië, Venezuela en Paraguay, en
andere inheemse volkeren verspreid over de wereld. Hun manieren om
zonder overheid tot bescherming te komen, zijn enkel bekend bij kenners
van uiterste gevallen. Wilt u er meer over weten, dan verwijzen wij in
onze aantekeningen naar diverse boeken met aanvullende
details.\footnote{Zie Pierre Clastres, \emph{Society Against the State:
  The Leader as Servant and the Humane Uses of Power Among the Indians
  of the Americas} (New York: Urizen Books, 1977) en Jones, op. cit.}
Primitieve groepen functioneerden zonder een aparte organisatie voor
geweld, simpelweg omdat zij kleine, gesloten en geïsoleerde
samenlevingen waren. Ze konden rekenen op hechte familiebanden om zich
op beperkte schaal te verdedigen tegen de meeste gewelddadige
bedreigingen, die als sporadische incidenten in hun omgeving optraden.
Wanneer ze met grotere bedreigingen werden geconfronteerd, georganiseerd
door staten, raakten zij al snel overweldigd en kwamen zij onder de
heerschappij te staan van externe groepen met een geweldsmonopolie. Dit
patroon herhaalde zich keer op keer. Zodra samenlevingen op een schaal
groter dan stammen en familienetwerken ontstonden -- vooral op
knooppunten waar handelsroutes verschillende volkeren samenbrachten --
duikten er altijd gewelddadige specialisten op om elk overschot dat
vreedzamere mensen voortbrachten, uit te plunderen. Toen technologische
ontwikkelingen de effectiviteit van geweld verhoogden, stortten
samenlevingen die niet waren ingericht op grootschalige oorlogsvoering
onvermijdelijk in.

\begin{quote}
``Welke vorsten verleenden eigenlijk de politiedienst? Wie fungeerden
als afpersers of zelfs als plunderaars? Een plunderaar kon feitelijk de
rol van politiechef op zich nemen zodra hij zijn `buit' systematisch
beheerste, deze afstemde op de draagkracht, zijn domein verdedigde tegen
andere plunderaars en zijn territoriale monopolie lang genoeg in stand
hield om door gewoonte als legitiem te worden beschouwd.''\footnote{Lane,
  \emph{economische gevolgen van georganiseerd geweld}, op. cit.,
  p.~403.} -- FREDERIC C. LANE
\end{quote}

Overheid als verkoper van bescherming

Zoals we al op diverse punten hebben aangegeven, bestaat de primaire
economische taak van de overheid -- gezien vanuit het oogpunt van de
belastingbetalers -- eruit om bescherming te bieden voor leven en
eigendom. Toch treedt de overheid vaak op als een georganiseerde
criminele organisatie, die binnen haar invloedssfeer middelen van mensen
incasseert als tribuut of plundering. Ze levert niet alleen
beveiligingsdiensten, maar functioneert ook als een beschermingsracket.
Hoewel de overheid bescherming biedt tegen gewelddadige aanvallen van
anderen, rekent ze -- net als een beschermingsracket -- kosten voor het
afwenden van schade die zij anders zelf zou veroorzaken. Het eerste is
een economische dienst, terwijl het tweede neerkomt op afpersing. In de
praktijk valt het vaak moeilijk om deze twee vormen van `bescherming'
van elkaar te onderscheiden. Zoals Charles Tilly opmerkte, kun je
overheden vrijwel het beste zien als `onze grootste voorbeelden van
georganiseerde misdaad.'\footnote{Charles Tilly, \emph{oorlogsvoering en
  staatsvorming als georganiseerde misdaad}, in Peter B. Evans, Dietrich
  Rueschemeyer en Theda Skoepol, \emph{de staat weer betrekken}
  (Cambridge: \emph{Cambridge University Press}, 1985), p.~169.}

Zelfs de best functionerende overheid combineert doorgaans economische
beschermingsdiensten met vormen van afpersing. Historisch gezien konden
beide activiteiten optimaal worden ingezet wanneer de overheid binnen
haar werkgebied vrijwel een monopolie op dwang wist te vestigen. Als een
enkele gewapende groep de controle over geweld behield, bood die vaak
een veel kwalitatief betere beschermingsdienst dan de meerdere
concurrerende partijen in dezelfde regio konden leveren.

Een natuurlijk monopolie op land

Door lokaal een monopolie op dwang te realiseren, kon de overheid haar
potentiële klanten niet alleen effectiever beschermen tegen gewelddadige
aanvallen, maar verlaagde zij tevens haar operationele kosten. Zoals
Lane het verwoordde: `De industrie die geweld toepast en beheerst, was
een natuurlijk monopolie, althans op land. Binnen territoriale grenzen
kon de aangeboden dienst veel goedkoper worden geleverd door een
monopolie.'\footnote{Ibid.} Een dergelijk `monopolie op het gebruik van
geweld binnen één aaneengesloten gebied' stelde een
beschermingsorganisatie in staat haar product te verbeteren en de kosten
te drukken.\footnote{Lane, \emph{economische gevolgen van georganiseerd
  geweld}, op. cit., p.~402.} Zo kon een dergelijke bestuursorganisatie
voor lagere kosten meer bescherming bieden, omdat zij niet voortdurend
militaire operaties hoefde te voeren om concurrerende groepen af te
weren die beschermingsgelden van haar klanten wilden innen.

Het vooruitzicht dat informatietechnologie zal bijdragen aan een
versoepeling van de opvatting dat soevereiniteit gebaseerd moet zijn op
een territoriaal monopolie, heeft al de aandacht getrokken van politieke
theoretici. Dit onderwerp staat centraal in \emph{Beyond Sovereignty:
Territory and Political Economy in the Twenty-First Century} van David
J. Elkins. Elkins onderschrijft onze these dat regeringen die een
monopolie op soevereiniteit nastreven, net zoals religieuze monopolieën
in de jaren na 1500, gedoemd zijn hun tussenpersonen te verliezen. Hij
schrijft: ``Vroeger gingen we ervan uit dat religies hun eigen
grondgebied of `turf' moesten hebben. Toen naties de universele religies
vervingen als soevereine rechters over leven en dood, maakten de
`compactheid' en `afgebakendheid' van religie plaats voor de nu zo
vertrouwde vermenging van gelovigen binnen één gebied. In plaats daarvan
weigeren wij de vermenging van naties -- of evenmin die van provincies
-- te tolereren, al geloof ik dat deze veronderstelling in een proces
van ontbinding verkeert.''\footnote{David J. Elkins, \emph{voorbij
  soevereiniteit: territorium en politieke economie in de
  eenentwintigste eeuw}. Toronto: \emph{University of Toronto Press},
  1995, pp.~13--14.}

Hij vervolgt, in lijn met onze opvatting, dat men territoriale
monopolieën op soevereiniteit zonder anarchie kan ontmantelen, zoals
blijkt uit de wijze waarop in federale systemen -- bijvoorbeeld in
Canada -- soevereiniteit wordt verdeeld tussen nationale en provinciale
overheden, en uit de condominiumregering -- met gedeelde Franse en
Britse soevereiniteit -- die een groot deel van deze eeuw enkele
Pacifische eilanden kenmerkte. Hoewel men zelden met geweld territoriale
monopolieën op soevereiniteit afbreekt, lossen we ze via overeenstemming
op.

Volgens Elkins -- en daar zijn wij het mee eens -- ``is de territoriale
natie een bundel of mand waarin andere facetten van ons leven passen.
Het is vergelijkbaar met het economische concept van een `mand' vol
goederen: je kunt de onderdelen niet gemakkelijk afzonderlijk
verkrijgen, maar moet ze collectief verwerven. In een restaurant kun je
`à la carte' bestellen; maar wat onze identiteit betreft, moeten we
nemen wat naties als bundel aanbieden, wat neerkomt op `table d'hote.'
\ldots{} Een regering \emph{à la carte} zal voor burgers in de
eenentwintigste eeuw volkomen natuurlijk lijken.''\footnote{Ibid., p.~29}

Geen enkele ontwikkeling draagt zo dramatisch bij aan de ontkoppeling
van soevereiniteit en de opkomst van een regering \emph{à la carte} als
de groei van een cybereconomie die de fysieke grenzen volledig
overstijgt.

\begin{quote}
``Nu frequenties stijgen en golflengten afnemen, verbetert de digitale
prestatie exponentieel. De bandbreedte stijgt, het energieverbruik
slinkt, de antennegrootte neemt af, interferentie stort in en de
foutpercentages kelderen.'' -- George Gilder
\end{quote}

\section{De wet van de telecosm schrapt de wetten van
naties}\label{de-wet-van-de-telecosm-schrapt-de-wetten-van-naties}

We zijn niet de enigen die beseffen dat bandbreedte -- de draagkracht
van communicatiemedia -- bedoeld is om het territoriale staatsstelsel te
overtreffen. Jim Taylor en Watts Wacker, auteurs van \emph{The 500-Year
Delta: What Happens After What Comes Next}, formuleren hun standpunt
niet op exact dezelfde wijze als wij, maar ze geven wel toe dat
`toegang' globalisme veroorzaakt en dat globalisme politieke systemen
verstoort, waardoor grenzen irrelevant worden. Naarmate grenzen
vervagen, wordt het concept van belastingheffing -- dat als steunpilaar
voor overheden fungeert -- steeds fragieler. Ook het idee van aanspraak
-- de opvatting dat je, omdat je op een bepaalde plaats geboren bent,
recht hebt op de economische voordelen die daaraan verbonden zijn --
valt uiteen. Zodra dat gebeurt, verdwijnen eveneens de voordelen van het
natie-zijn. Terwijl dit allemaal gaande is, worden de idealen die aan
het natie-zijn ten grondslag liggen -- patriotisme, democratie, het
staatsbestel, de smeltkroes, eenwording en verantwoord burgerschap, wat
het ook moge zijn in welke natie men leeft -- tot de vuilnisbelt van de
geschiedenis gedegradeerd.\footnote{Jim Taylor en Watts Wacker,
  \emph{The 500-year delta: what happens after what comes next.} New
  York: \emph{HarperCollins}, 1997, p.~40.}

Ook zonder het expliciet te benoemen voelen zij dat de geschiedenis
richting de bevrijding van het soevereine individu beweegt. Zoals ze
stellen: ``Aan de horizon wacht een veel zuiverdere vorm van
individualisme dan de democratie zoals wij die nu kennen.''\footnote{Ibid.,
  p.~67} Hoe dat zal gebeuren? Taylor en Wacker wijzen op een krachtige
dynamiek.

Het simpele feit is dat het brede gevoel van patriottisme -- de liefde
voor de eigen natie en het plichtsbesef jegens die natie -- niet langer
van toegevoegde waarde blijkt te zijn. Burgers die floreren in de
mondiale samenleving baseren hun keuzes niet meer op nationale
identiteit, maar op hun politieke, sociale en economische zelfbeeld. Ze
laten zich niet leiden door de vraag ``Wie zijn mensen zoals ik?'', maar
door wat voor hen persoonlijk relevant is. Ze benutten hun vrijheid ten
volle om te weten, te reizen, te ondernemen en zichzelf te zijn. Naties
en bedrijven die hier niet in meegaan en vasthouden aan achterhaalde
praktijken, gevoed door nostalgie, zullen verzwakken.\footnote{Ibid.,
  pp.~41--42}

Het vervagen van fysieke grenzen -- zoals blijkt uit de jaarlijkse
verdrievoudiging van de bandbreedte en de exponentiële groei van het
internet en het World Wide Web -- zal het verdwijnen van de traditionele
bureaucratische lagen van de overheid versnellen. Een voortdurende
verdrievoudiging van de bandbreedte tot 2012 zou sinds 1993, toen George
Gilder voor het eerst suggereerde dat bandbreedte zelfs sneller groeide
dan de capaciteit van microprocessors, neerkomen op een miljardvoudige
toename. Als dit werkelijkheid wordt -- wat wij verwachten naar
aanleiding van recente doorbraken in de geïntegreerde optica -- zal de
overvloed aan communicatiemogelijkheden leiden tot een spectaculaire
toename van cyberhandel. Dankzij golfdelingsmultiplexing kan een enkele
glasvezeldraad, zo dun als een mensenhaar, een capaciteit hebben van één
biljoen bits per seconde.\footnote{George Gilder, \emph{Fiber keeps its
  promise: get ready. Bandwidth will triple each year for the next 25,
  creating trillions in new wealth.} \emph{Forbes ASAP}, 7 april 1997}
Met andere woorden, één glasvezelkabel kan vijfentwintig keer meer bits
vervoeren dan de totale capaciteit van alle communicatienetwerken ter
wereld samen. De uitbreidingsmogelijkheden zijn verbluffend. Door deze
ongekende communicatiewaarde zal er aanzienlijk meer geld aan
communicatie worden uitgegeven, omdat die diensten zo goedkoop zijn.
Tegelijkertijd raken traditionele media, zoals vaste telefonie en
televisie, snel achterhaald. Het World Wide Web voorziet elke computer
van een veel rijker signaalaanbod dan consumenten tegenwoordig ervaren
via netwerktelevisie. Naarmate de bandbreedterevolutie zich verder
ontvouwt, trekt zij steeds meer mensen aan in de grensloze virtuele
wereld van online gemeenschappen en cyberhandel -- een wereld met
voldoende grafische rijkdom om te transformeren in de `metaverse', de
alternatieve cyberspace-realiteit zoals de sciencefictionschrijver Neal
Stephenson zich voorstelde. Stephenson's `metaverse' beschrijft een
virtuele gemeenschap met eigen wetten, vorsten en schurken.\footnote{Zie
  Neal Stephenson, Snow Crash. New York: \emph{Bantam Books}, 1993} Als
steeds meer economische activiteiten naar cyberspace verhuizen, neemt de
waarde van staatsmonopoliekracht binnen fysieke grenzen af, wat staten
prikkelt om hun soevereiniteit te verhuren en te fragmenteren.

Net zoals natiestaten tegenwoordig worden gestimuleerd om vrije havens,
vrijhandelszones en zona franca's te huisvesten, krijgen ze ook prikkels
om hun soevereiniteit te verhuren. We hebben al uitvoerig de
onderhandelingen belicht tussen de al 900 jaar bestaande \emph{Suvereine
Militaire Hospitaller Orde van Sint-Jan van Jeruzalem, van Rhodos en van
Malta} -- beter bekend als de Ridders van Malta -- en de Republiek Malta
over de teruggave van de soevereiniteit over Fort St.~Angelo. We gaan
ervan uit dat deze onderhandelingen succesvol worden afgerond; andere
voorbeelden zullen in hun kielzog volgen. Sommige natiestaten geven hun
soevereiniteit over kleine enclaves en afgelegen gebieden af aan geheel
nieuwe affiniteitsgroepen en virtuele gemeenschappen. Het is bovendien
niet onwaarschijnlijk dat commerciële partijen, zoals
beveiligingsbureaus en hotelketens, gaan bieden om soevereiniteit over
kleine stukken grond over te nemen. In de toekomst zouden Wackenhut,
Pinkerton en Argenbright hybride, omheinde pensioenresidenties en
belastingvrije zones in aantrekkelijke klimaten wereldwijd kunnen
aanbieden. Religieuze entiteiten -- vergelijkbaar met de Ridders van
Malta maar voor elke denkbare denominatie -- gaan op hun eigen wijze
proberen om de hemel te belichamen in afgelegen hoekjes van de aarde.
Zelfs vermogende individuen en families zullen eigen percelen bezitten
waarop zij een beperkte soevereiniteit uitoefenen, hun eigen postzegels
en paspoorten uitgeven en een eigen website beheren.

\section{Monopolie en plundering}\label{monopolie-en-plundering}

Merk op dat de prikkels om soevereiniteit te delen of te verhuren tegen
een vergoeding totaal verschillen van die waarmee heersers in vroegere
tijden werden geconfronteerd, toen zij met militaire concurrentie binnen
hun lokale dwangmonopolie te maken hadden. Het verhuren van
soevereiniteit destabiliseert niet meer dan het vestigen van een
vrijhandelszone. Integendeel, de concurrentie tussen militaire machten
-- gevoerd door strijdende krijgsheren en guerrillabendes -- bepaalt
namelijk direct of een potentiële regering sterke prikkels heeft om haar
bevolking te beschermen of juist uit te plunderen. Wanneer strijdende
groepen in een fragiele balans ringen en manoeuvreren, neemt de
verleiding voor roofzuchtig geweld toe; plundering wordt
aantrekkelijker. Doordat de machtspositie minder stabiel is en het
lokale dwangmonopolie vervaagt, verkorten de tijdshorizonten van degenen
die bereid zijn geweld te gebruiken. De `koning van de berg' kan zo op
een onzekere basis staan dat hij niet lang genoeg overleeft om te
profiteren van de aanzienlijke opbrengsten die volgen op het beteugelen
van geweld. Daarmee lijkt er weinig in de weg te zitten voor overheden
om hun macht te misbruiken en de samenleving te terroriseren en uit te
plunderen.

De logica van geweld illustreert dat hoe meer concurrerende gewapende
groepen actief zijn in een gebied, des te groter de kans is dat zij
roofzuchtig geweld aanwenden. Zonder één allesbepalende macht die het
willekeurige geweld in toom houdt, grijpt dat geweldspatroon om zich
heen en gaan veel voordelen van economische en sociale samenwerking in
rook op.

De schade die ontstaat wanneer geweld de vrije loop krijgt in een staat
van anarchie, zie je duidelijk aan het lot van China onder de
krijgsheren in de jaren 1920. Dat verhaal vertelden we in \emph{The
Great Reckoning}. In gebieden waar geen absolute macht aanwezig was om
hen in toom te houden, richtten de concurrerende krijgsheren enorme
schade aan. Vergelijkbare verhalen, levendig in beeld gebracht door
\emph{CNN}-nieuwsploegen die de straten van Mogadishu in Somalië
trotseren, illustreren hetzelfde punt. Daar veroorzaakten de
strijdkrachten van de krijgsheren in Somalië -- beter bekend als de
`technicals' -- een staat van anarchie in dat trieste land, totdat de
Verenigde Staten met een grootschalige militaire interventie ingrepen om
hen te stoppen. Toen de overweldigende aanwezigheid van Amerikaanse
troepen werd teruggetrokken, haalden de `technicals' hun wapens weer
tevoorschijn en hervatten ze de anarchie.

Een verslag in de \emph{Washington Post} merkte op:

\begin{quote}
{[}P{]}ickuptrucks, uitgerust met luchtafweergeschut, ploegen weer door
stoffige, met puin bezaaide straten. Ook de zelfverzekerde jongemannen
in T-shirts, met Kalashnikov-geweren over hun schouders, zijn
teruggekeerd en persen geld af van passerende auto's en bussen bij
geïmproviseerde wegblokkades. Een buurt die door milities wordt
beheerst, is zo zwaar bewapend dat de lokale bevolking haar
`Bosnië-Herzegovina' noemt. Reizen door de straten van deze stad doet
sterk denken aan de dagen van 1992, toen chaotische gevechten tussen
rivaliserende milities in Somalië tot complete anarchie leidden en een
hongersnood uitbrak, waardoor een door de VS geleide militaire
interventie noodzakelijk werd. Om Mogadishu doorkruisen huren reizigers
inmiddels een voertuig vol zwaarbewapende bandieten in, in de hoop op
bescherming voor honderd dollar per dag, inclusief een
lunchpauze.\footnote{Keith B. Richburg, `Twee jaar na de landing van de
  VS in Somalië is het weer chaotisch', \emph{Washington Post}, 4
  december 1994, p.~A1.}
\end{quote}

De voorbeelden uit Somalië, Rwanda en andere landen -- die u binnenkort
op televisie zult zien -- bewijzen in alle duidelijkheid dat
gewelddadige strijd om territorium niet dezelfde directe economische
opbrengsten genereert als andere vormen van concurrentie. Integendeel:
de rondtrekkende bandieten en plunderaars, die in anarchie opereren,
ervaren zelfs niet de zwakke prikkels die nodig zijn om productieve
activiteiten te beschermen, prikkels die soms zelfs de harde hand van
dictatoren temperen wanneer hun bewind veilig is.

\begin{quote}
``De samenleving van wat wij het moderne tijdperk noemen, wordt --
vooral in het Westen -- gekenmerkt door een zekere mate van
monopolievorming. Het vrije gebruik van militaire wapens wordt
individuen ontzegd en voorbehouden aan een centrale autoriteit, en
evenzo wordt de heffing op het eigendom of inkomen van burgers
geconcentreerd in de handen van een centrale sociale autoriteit. De
financiële middelen die daardoor in deze centrale autoriteit stromen,
waarborgen haar monopolie op militaire kracht, wat op zijn beurt het
belastingmonopolie in stand houdt. Geen van beide domineert de ander; ze
vormen twee kanten van hetzelfde monopolie. Verdwijnt de ene, dan volgt
de andere automatisch; het monopoliesysteem kan soms aan de ene zijde
sterker doorzakken dan aan de andere.''\footnote{Geciteerd in Tilly,
  Coercion, Capital and European States, op. cit., p.~85.}
\end{quote}

\begin{itemize}
\tightlist
\item
  Norbert Elias
\end{itemize}

\section{De evolutie van bescherming}\label{de-evolutie-van-bescherming}

Lane presenteerde een betoog dat wij voor onze doeleinden onjuist hebben
overgenomen, waarin hij schetst hoe het informatietijdperk zich zou
kunnen ontvouwen. Hij stelde dat we de geschiedenis van de westerse
economieën sinds de donkere middeleeuwen kunnen bekijken als vier fasen
waarin concurrentie en monopolie in de organisatie van geweld centraal
stonden. Hoewel Lane grotendeels niets zegt over de megapolitieke
factoren die volgens ons bepalend zijn voor de schaal waarop overheden
opereren, sluit zijn verkenning van de economie van geweld naadloos aan
bij het betoog dat we uiteenzetten in \emph{Blood in the Streets} en
\textbf{The Great Reckoning} -- en elders in dit volume.

We hebben al enkele van de megapolitieke factoren geanalyseerd die een
rol speelden in de evolutie van de westerse samenleving na de val van
Rome. Ook Lane richtte zich op deze periode, met een nadruk op de
economische consequenties van de strijd om een monopolie op geweld te
vestigen. Hij onderscheidde vier belangrijke fasen in het functioneren
van economieën in de afgelopen duizend jaar, waarbij elke fase een ander
stadium in de organisatie van geweld weerspiegelde.\footnote{Merk op dat
  Lane's vier fasen van concurrentie en monopolie in het gebruik van
  geweld afwijken van de vier fasen in de organisatie van het economisch
  leven die wij onderscheiden, namelijk: jagen en verzamelen, landbouw,
  industrialisme en het informatietijdperk.}

\subsection{Uit de donkere
middeleeuwen}\label{uit-de-donkere-middeleeuwen}

De eerste fase kenmerkt zich door `anarchie en plundering', zoals
tijdens de feodale revolutie van duizend jaar geleden. Lane geeft geen
specifieke data voor de door hem onderscheiden perioden, maar eenvoudige
berekeningen maken duidelijk waar zijn eerste periode begint. Zijn
beschrijving van de fase `anarchie en plundering' lijkt te passen bij de
omstandigheden in de overgang van de donkere middeleeuwen, toen het
gebruik van geweld `zeer competitief, zelfs op land' was.\footnote{Lane,
  `Economische gevolgen van georganiseerd geweld', op. cit., p.~411.}
Hij licht niet toe waarom, maar wanneer geweld `zeer competitief' is,
wijst dat doorgaans op aanzienlijke belemmeringen voor het uitoefenen
van macht over afstand. In militaire termen domineert de verdediging
immers altijd de aanval.

Zoals we in hoofdstuk 3 hebben toegelicht, viel deze fase van `anarchie
en plundering' samen met een daling van de landbouwproductiviteit als
gevolg van ongunstige klimaatveranderingen. Omdat de technologie
destijds nauwelijks effectieve schaalvoordelen leverde waarmee een
monopolie op geweld kon worden gevestigd, woedde de concurrentie tussen
potentiële machthebbers hevig. Hierdoor kwam de economische activiteit
ernstig in het gedrang.

De zwakte van de economie verergerde het probleem om een stabiele orde
tot stand te brengen. Het vestigen van een lokaal geweldsmonopolie ging
gepaard met hoge militaire kosten in verhouding tot de beperkte
economische opbrengsten. Niet in staat een effectief monopolie te
handhaven over een economisch levensvatbaar gebied, terroriseerden en
plunderden de bewapende ridders te paard terwijl zij nauwelijks echte
`bescherming' boden aan hun klanten.

\subsection{Feodalisme}\label{feodalisme}

``De tweede fase begint zodra kleine regionale of provinciale heersers
een geweldsmonopolie oprichten. De landbouwproductie neemt toe en de
recent opgerichte geweldsmonopolisten innen het grootste deel van het
overschot.''\footnote{Ibid.} -- Toch valt het overschot in deze periode
-- die we met de vroege middeleeuwen associëren -- relatief gering uit.
Het ontbreken van schaalvoordelen in de organisatie van geweld remt de
economische groei en zorgt ervoor dat de militaire kosten voor het
handhaven van lokale monopolies hoog blijven. Ondertussen stijgt het
bedrag dat kleine soevereiniteiten voor bescherming kunnen vragen, omdat
de economische activiteit toeneemt naarmate de anarchie afneemt.

In een latere fase lokken veel tributennemers klanten met speciale
aanbiedingen voor landbouw- en handelsbedrijven. Zij bieden tegen lage
prijzen bescherming aan hen die nieuw land in gebruik nemen en verzorgen
daarnaast speciale politiediensten ter bevordering van de handel, zoals
de graven van Champagne voor kooplieden die hun markten bezoeken. Met
andere woorden, zodra lokale oorlogsheersers voldoende controle over hun
grondgebied kregen om geloofwaardig te onderhandelen, deden zij precies
wat kooplieden doen als zij hun marktaandeel willen vergroten: zij boden
kortingen op hun diensten om klanten te winnen. Later investeerden de
oorlogsheersers de extra inkomsten uit de groeiende economische
activiteit om hun grip op grotere gebieden te versterken. Zodra zij hun
controle stevig verankerd hadden, plukten zij de vruchten van hun
monopoliepositie: hun militaire kosten voor ordehandhaving daalden,
waardoor zij de prijzen voor hun diensten konden verhogen zonder dat dit
de aantrekkingskracht voor hun klanten verminderde.

In deze complexe periode van de westerse geschiedenis houden degenen die
geweld inzetten -- de middeleeuwse heren en vorsten -- vrijwel al het
overschot boven het bestaansniveau voor zichzelf. Er zijn maar weinig
kooplieden, en de meest succesvolle onder hen slagen erin de
belastingen, heffingen en andere kosten die men betaalt voor
`beschermingsdiensten' te vermijden of te verlagen.

\subsection{De vroegmoderne periode}\label{de-vroegmoderne-periode}

We komen in de derde fase wanneer kooplieden en landeigenaren -- die
niet tevens als geweldspecialisten optreden -- een groter deel van het
economische overschot ontvangen dan leenheren en monarchen. Zoals het
citaat luidt:\\
`meer van het overschot van de economie ontvangen dan de leenheren en
monarchen \ldots{} in deze derde fase ontvangen ondernemingen die
gespecialiseerd zijn in het gebruik van geweld minder van het overschot
dan ondernemingen die bescherming via de overheid inkopen.'\\
Omdat succesvolle kooplieden hun winsten eerder herinvesteren dan
consumeren, zorgde hun verhoogde winst in die periode voor een
zelfversterkende groei.

\subsection{Het tijdperk van de
fabrieken}\label{het-tijdperk-van-de-fabrieken}

Lane ziet de overgang van de derde naar de vierde fase als nauw verweven
met de opkomst van technologische en industriële innovaties, die
belangrijker bleken voor het realiseren van winsten dan het verlagen van
de kosten voor bescherming. Hiermee lijkt hij te verwijzen naar de
periode vanaf 1750. Vanaf dat moment kreeg technologie een steeds
dominantere rol in de welvaart van regio's. Als voorbeeld: zelfs
gebieden zonder een overheidsapparaat -- zoals in sommige delen van
Nieuw-Zeeland vóór 1840 -- werden niet automatisch welvarend, alleen
omdat zij geen belastingen hoefden te betalen. In die tijd waren
innovaties in de industriële technologie bepalender voor
winstmaximalisatie dan enige besparing op beschermingskosten, zelfs
wanneer die kosten tot nul gereduceerd konden worden. Naarmate de
overheid groeide, stelden de krediet- en financieringssystemen --
oorspronkelijk ingevoerd om geld op te halen voor militaire operaties --
zich ter beschikking voor de financiering van bedrijfsactiviteiten op
grotere schaal.

Hoewel Lane dat niet expliciet benoemt, verminderde de concentratie van
technologische voordelen in bepaalde gebieden de concurrentie tussen
rechtsgebieden. Hierdoor konden overheden -- oftewel ondernemingen die
gespecialiseerd zijn in het gebruik van geweld -- hogere prijzen vragen.
Wanneer er grote technologische verschillen bestaan tussen
rechtsgebieden, zoals in het industriële tijdperk, boeken ondernemers in
de gebieden met de beste technologie doorgaans hogere winsten, ook al
betalen zij mogelijk hogere belastingen en andere kosten aan hun
overheid.

\subsection{Plunderen met een
glimlach}\label{plunderen-met-een-glimlach}

Regeringen in het industriële tijdperk genoten van een royaal monopolie
op uitbuiting. De werkelijke kosten voor de bescherming van leven en
ledematen bleken verwaarloosbaar in vergelijking met de hoge belastingen
die zij hanteerden. Toch opereerden zij in een sfeer waarin de
concurrentie zo pervers was dat zij in feite veel meer ruimte lieten
voor plundering dan voor bescherming -- een feit dat vrijwel onopgemerkt
bleef. Het was een werkelijk zeldzaam moment in de geschiedenis.

De nadelen van anarchie binnen de megapolitieke context van het
industrialisme maakten het praktisch onmogelijk om binnen één
grondgebied concurrerende beschermingsdiensten te leveren. Onder die
omstandigheden kon men veiligheid alleen garanderen door over een
aanzienlijk groter geweldsarsenaal te beschikken. Daarom had men
nauwelijks te winnen door te proberen een scherp onderscheid te maken
tussen het deel van de belastingen dat, volgens Lane, `als betaling voor
de geleverde dienst' moest worden aangemerkt en het deel dat men als
plundering bestempelde.\footnote{Ibid., p.~403.} Het onderscheid was al
reëel genoeg. Aangezien men echter toch verplicht was belastingen te
voldoen, leverde een uitgebreide analyse hiervan nauwelijks meer op dan
het stillen van een morbide nieuwsgierigheid. Zoals Lane opmerkte:
ongeacht welk deel van de belastingen als plundering bleek te worden
aangemerkt, betaalde men immers de prijzen `om ernstigere verliezen te
voorkomen.'\footnote{Ibid., p.~404.}

\subsection{De stijging van inkomens onder het
industrialisme}\label{de-stijging-van-inkomens-onder-het-industrialisme}

Een van de redenen dat dit dilemma gedurende de afgelopen twee eeuwen
onder natiestaten draaglijk bleef, lag in de drastische stijging van
inkomens, vooral in de gebieden waar de industriële ontwikkeling het
meest volop was. Leidinggevenden van de \emph{OESO}-regeringen wisten
vrijwel elk jaar een groter aandeel van de inkomsten te innen. Toch ging
de toegenomen plundering gepaard met een aanzienlijk hogere welvaart en
een sterkere ongelijkheid in rijkdom ten opzichte van de rest van de
wereld. Onder zulke omstandigheden waren bezwaren tegen
belastingverhogingen marginaal en niet krachtig genoeg om de natuurlijke
gang van zaken te doorbreken. Inderdaad, om redenen die in eerdere
hoofdstukken aan bod kwamen, hing het militaire voortbestaan van een
industrieel natiestaat grotendeels af van het feit dat men geen
effectieve grenzen kende aan de aanspraken op de hulpbronnen van haar
burgers.

In iedere industriële staat ontwikkelde het beleid zich min of meer in
dezelfde richting. Op het hoogtepunt van het industrialisme na de Tweede
Wereldoorlog bereikte het marginale belastingtarief op inkomen wel 90
procent of meer. Dit vormde een veel agressievere aanspraak op het recht
van de staat om middelen te onttrekken dan zelfs de oosterse despoten
uit de vroege hydraulische beschavingen doorgaans vorderden. Toch kende
de industriële vorm van plundering haar eigen logica. Een groot deel
hiervan werd bepaald door de aard van de industriële technologie in de
eerste helft van de twintigste eeuw, zoals we eerder beschreven.

Deze technologie maakte het vrijwel onvermijdelijk dat de staat een
groot deel van het inkomen in beslag nam en herverdeelde, waarbij een
aanzienlijk deel van de belastingdruk op een klein segment van de
kapitalisten viel. De meeste industriële processen hingen sterk af van
natuurlijke hulpbronnen en waren daardoor nauw gekoppeld aan de locaties
waar die middelen beschikbaar waren. Het verplaatsen van een
staalfabriek, een mijn of een haven ging gepaard met torenhoge kosten,
waardoor het in veel gevallen praktisch onmogelijk bleek. Dergelijke
faciliteiten vormden derhalve vaste doelwitten waarop men gemakkelijk
belastingen kon heffen. De afgelopen eeuw stegen de belastingen op
eigendom, ondernemingen en op het onttrekken van natuurlijke hulpbronnen
aanzienlijk. Ook de inkomstenbelastingen namen toe, aanvankelijk voor de
kapitalisten en later zelfs voor de arbeiders. De opkomst van
grootschalige industriële werkgelegenheid maakte het praktisch
uitvoerbaar om een breed gedragen inkomstenbelasting in te voeren. Men
hield de lonen al bij de bron in, terwijl de belastingautoriteiten de
inning coördineerden met de boekhoudafdelingen van industriële
bedrijven. Vandaag de dag vinden we dit vanzelfsprekend, maar het innen
van de inkomstenbelasting bij de fabriekspoort bleek een stuk
eenvoudiger dan door het hele platteland te trekken om van miljoenen
onafhankelijke ambachtslieden en boeren een deel van de winst te
onttrekken.

Kortom, de industriële technologie zorgde er doorgaans voor dat de
belastinginning routinematig, voorspelbaarder en minder persoonlijk
ingrijpend verliep dan in veel eerdere perioden. Niettemin nam deze
wijze van belastingheffing een hoger percentage van de middelen van de
samenleving in beslag dan welke vorm van soevereiniteit daarvoor ooit
had gedaan.

\subsection{Wat wordt beschermd?}\label{wat-wordt-beschermd}

Het feit dat samenlevingen rijker werden, terwijl het aandeel van het
inkomen dat door belastingen werd opgeslokt aanzienlijk toenam, roept de
vraag op welke bescherming overheden aan industriële economieën boden.
Wat beschermden zij? Ons antwoord: in de eerste plaats industriële
installaties met hoge kapitaalkosten en een grote kwetsbaarheid voor
aanvallen. Grootschalige industriële bedrijven hadden zich immers niet
kunnen handhaven in een wanordelijke omgeving waar het geweld van de
concurrentie sterker heerste, ook al leidde die concurrentie ertoe dat
de overheid uiteindelijk een kleiner deel van de productie afpakte.

Dit is de reden waarom kapitaalintensieve operaties economisch vaak niet
renderen in Amerikaanse sloppenwijken en in samenlevingen in de Derde
Wereld waar ad hocgeweld wijdverspreid is. De industriële samenleving
kon zich ontwikkelen doordat men erin slaagde een zekere orde tot stand
te brengen en die te handhaven. Ondernemingen ondervonden regelmatige,
voorspelbare afpersingspraktijken in plaats van willekeurig geweld.

Zelfs tijdens de bloeitijd van het industrialisme overdrijven velen
wanneer ze spreken van een overheid die een `monopolie op kracht' bezit.
Regeringen proberen allemaal zo'n monopolie te behouden, maar werknemers
in industriële bedrijven kwamen er vaak achter zelf in staat te zijn tot
geweld tegen hun bazen. Zolang gewone burgers toegang hebben tot wapens
of een ongereguleerde menigte fysiek sterk genoeg blijft om bijvoorbeeld
een bus om te duwen of stenen naar de politie te gooien, grijpen degenen
die de overheid beheersen de macht niet volledig. Zij beschikken slechts
over een overheersende vorm van geweld, waardoor het voor de meeste
mensen economisch niet rendabel is om met hen te concurreren.

\begin{quote}
Een op het net gebaseerde overheid kan alleen functioneren als de
geregeerden instemmen. Iedere netregering moet haar burgers daarom reële
voordelen bieden om hun steun te behouden. Die voordelen hoeven niet
uitsluitend uit persoonlijke goederen of diensten te bestaan, maar
kunnen ook de bredere meerwaarde van een reguleringsregime inhouden:
bijvoorbeeld een schone en transparante markt met vaste regels en
consequenties, of een toezichthoudende gemeenschap waar kinderen
vertrouwen kunnen op de mensen om hen heen en waar de privacy van
individuen beschermd is. \footnote{Esther Dyson, \emph{Release 2.1: een
  ontwerp voor het leven in het digitale tijdperk}. New York:
  \emph{Broadway Books}, 1998, p.~131}: - Esther Dyson
\end{quote}

\subsection{Het informatietijdperk}\label{het-informatietijdperk}

Het informatietijdperk luidt de start in van een vijfde fase in de
evolutie van de strijd om geweld in het Westen. Lane had deze fase niet
voorzien. Deze fase draait om concurrentie in cyberspace, een domein
waar geen enkele `geweldsprekende onderneming' erin slaagt een monopolie
te vestigen, omdat het geen fysiek grondgebied beslaat.

Hoewel Lanes' betoog de gangbare naoorlogse overtuigingen over de
onvermijdelijkheid van de natiestaat weerspiegelt, onderkende hij een
cruciaal punt voor het begrijpen van de toekomst -- wellicht zelfs
belangrijker dan veertig of vijftig jaar geleden leek. Hij wees erop dat
regeringen op de open zee nooit een stabiel, dwangmatig monopolie hebben
weten te vestigen. Denk er eens over na: geen enkele overheidswet heeft
daar ooit exclusieve toepassing gevonden. Dit inzicht is essentieel om
te begrijpen hoe de organisatie van geweld en bescherming zich zal
ontwikkelen nu de economie richting cyberspace migreert -- een domein
zonder fysiek bestaan. Om dezelfde redenen die Lane aanhaalde toen hij
opmerkte dat geen enkele regering geweld op zee heeft kunnen
monopoleren, lijkt het des te onwaarschijnlijker dat een overheid
succesvol een oneindig domein zonder fysieke grenzen kan beheersen.

\section{Concurrentie zonder
anarchie}\label{concurrentie-zonder-anarchie}

Vroeger, wanneer de omstandigheden het voor één enkele macht die geweld
gebruikt onmogelijk maakten een monopolie te vestigen, resulteerde dat
in anarchie en plundering. Het informatietijdperk heeft echter de
technologische voorwaarden voor het organiseren van geweld ingrijpend
veranderd. In tegenstelling tot vroeger -- toen het onvermogen om een
bepaald gebied te beveiligen leidde tot hogere militaire kosten en
lagere economische opbrengsten -- zorgt het feit dat regeringen
cyberspace niet kunnen beheersen juist voor lagere militaire kosten en
hogere economische opbrengsten. Dat komt doordat informatietechnologie
een geheel nieuwe dimensie aan bescherming toevoegt. Voor het eerst in
de geschiedenis stelt deze technologie ons in staat activa te creëren en
te beschermen die volledig buiten het territoriale geweldmonopolie van
enige regering vallen.

\begin{quote}
`Landen waarin de eenheden van politieke macht en bestuur meervoudig
zijn en die geen centrale, stabiele en onaangevochten toezichthoudende
bron van rechtsmacht en macht hebben, moeten hun eigen werkbare
oplossingen bedenken voor het omgaan met de problemen die door
dergelijke grenzen worden opgeworpen.'\footnote{Rees Davies,
  \emph{Frontier arrangements in fragmented societies: Ireland and
  Wales}, in Robert Bartlett en Angus MacKay (red.), \emph{Medieval
  frontier societies} (Oxford: \emph{Oxford University Press}, 1992),
  p.~80.}: - REES DAVIES
\end{quote}

\subsection{De analogie met de grens}\label{de-analogie-met-de-grens}

Cyberspace functioneert in zekere zin als een technologisch beveiligd
grensgebied, vergelijkbaar met de middeleeuwse grensstreken. In vroegere
tijden, toen de macht van edelen en vorsten zwak was en verschillende
machten elkaar aan de grenzen uitdaagden, ontstond een bestuursvorm die
aan concurrentieel gezag deed denken. Wanneer we kijken hoe deze
grensgebieden werkten, ontdekken we mogelijk hoe `grenswetten' of iets
dergelijks in cyberspace vorm kunnen krijgen.

Andorra blijft bestaan als een als het ware gefossiliseerd grensgebied
tussen Frankrijk en Spanje, een relikwie uit megapolitieke
omstandigheden waarin geen van beide koninkrijken de ander kon
overheersen in dat gure, bijna ontoegankelijke gebied van 190 vierkante
mijl in de Pyreneeën. In 1278 sloten de lokale Franse en Spaanse feodale
heren -- de Franse graaf van Foix en de Spaanse bisschop van Urgell --
een overeenkomst waarin zij de suzerainiteit over Andorra onderling
verdeelden. Ze benoemden ieder een viquier die, met een minimaal
overheidsgezag, de kleine Andorraanse militia -- tegenwoordig een
politie-eenheid -- aanstuurde. De positie van de graaf verviel al lang;
de Franse overheid neemt zijn taken nu over vanuit Parijs en heft onder
meer de helft van de jaarlijkse tribuut die Andorra verschuldigd is --
een bedrag dat lager ligt dan de huur van een goedkoop appartement voor
één maand. De bisschop van Urgell ontvangt nog steeds zijn deel van de
tribuut, net zoals zijn middeleeuwse voorgangers dat deden.

Uit het gesplitste tribuut blijkt dat Andorra twee bronnen van
toezichthoudend gezag en macht kende. Traditiegetrouw diende men bezwaar
tegen Andorraanse civiele rechtszaken in bij het Episcopaals College van
Urgell of bij het Hof van Cassatie in Parijs.

Andorra's ambigue status leidde ertoe dat er nauwelijks wetten werden
vastgesteld. Meer dan zevenhonderd jaar kende het land praktisch geen
overheidsgezag en hanteerde het geen belastingen. Tegenwoordig groeit
het imago van Andorra als belastingparadijs, maar tot nog een generatie
geleden stond het bekend als arm. Waar de streek eens was dichtbegroeid
met bossen, leidden houtkappraktijken door inwoners die zich in de barre
winters wilden warmhouden er gaandeweg toe dat de bomen verdwenen. Elk
jaar sluit sneeuw het gebied van november tot en met april volledig af.
Zelfs in de zomer is het zo kil dat landbouwgewassen uitsluitend op de
zuidelijke hellingen te telen zijn. Als deze beschrijving Andorra
onaantrekkelijk doet lijken, heeft u eigenlijk het geheim van haar
succes ontdekt: het overleefde als feodale enclave in het tijdperk van
de natiestaat doordat het zowel afgezonderd als extreem arm bleef.

Vroeger kende Europa talloze middeleeuwse grensgebieden -- zogenaamde
`march-regio's' -- waarin soevereiniteiten in elkaar overvloeiden. Deze
vaak gewelddadige streken hielden decennialang, soms zelfs eeuwenlang
stand en waren overwegend arm. Zoals eerder aangetoond ontstonden er
grensgebieden tussen gebieden onder Keltische en Engelse heerschappij in
Ierland; tussen Wales en Engeland; tussen Schotland en Engeland; tussen
Italië en Frankrijk; tussen Frankrijk en Spanje; tussen Duitsland en de
Slavische grensstreken van Centraal-Europa; en tussen de christelijke
koninkrijken van Spanje en het islamitische koninkrijk Granada. Net als
Andorra ontwikkelden deze grensgebieden eigen institutionele en
juridische systemen, die we waarschijnlijk in het komende millennium
opnieuw zullen aantreffen.

Doordat beide concurrerende autoriteiten zwak waren, riepen sommige
overheden zelfs hun onderdanen op om zich vrijwillig in grensgebieden te
vestigen, zodat zij hun invloed konden uitbreiden. De
belastingvrijstelling in die streken trok de onderdanen vrijwel vanzelf,
waardoor zij zich in de grenszone gingen vestigen.

Omdat de concurrentie op zeer geringe marges berustte, zou een
belastingheffing door één autoriteit in een grensgebied niet alleen het
voortbestaan van haar volgelingen bemoeilijken, maar bovendien anderen
prikkelen om zich bij de tegenpartij aan te sluiten. Inwoners in
grensgebieden konden daardoor doorgaans zelf kiezen welke wetten zij
wilden naleven. Die keuze was ingegeven door de zwakke positie van de
rivaliserende autoriteiten en had niets met ideologische standpunten te
maken.

Toch ontstonden er praktische knelpunten die men moest oplossen. Binnen
het feodale systeem kwamen landeigenaren die aan beide kanten van een
symbolische grens land bezaten, vaak in een ernstig belangenconflict
terecht. Een heer met bezittingen in zowel Schotland als Engeland liep
bijvoorbeeld in oorlogstijd het risico voor beide koninkrijken
dienstplichtig te worden. Om deze schijnbare tegenstrijdigheid op te
lossen, kon vrijwel iedereen in de feodale hiërarchie middels een
juridische procedure -- de `avowal' genoemd -- vaststellen aan welk
rechtsstelsel hij zich zou onderwerpen.

Informatietechnologie zal gelijke kansen creëren voor een onafhankelijke
keuze bij het opzetten van economische activiteiten, al gaan daar
belangrijke verschillen mee gepaard. Ten eerste zal cyberspace, in
tegenstelling tot de middeleeuwse grensgemeenschappen, op termijn
waarschijnlijk het welvarendste economische domein worden. Het vormt dan
eerder een expansieve dan een samentrekkende grens. Mensen uit de
kernregio's van de middeleeuwse samenleving verhuisden zelden naar
grensgebieden zonder krachtige stimulansen -- vaak gekoppeld aan
religieuze verplichtingen -- omdat die streken doorgaans door geweld en
armoede werden gekenmerkt. Zij fungeerden dan ook niet als magneet om
middelen uit de macht van de autoriteiten weg te trekken. Cyberspace zal
die aantrekkingskracht echter wel bezitten.

Ten tweede vormt de nieuwe grens geen duopolie dat uitnodigt tot
collusie tussen twee autoriteiten om compromissen te sluiten over hun
grensclaims. In de middeleeuwen bleken zulke compromisvoorstellen om
twee redenen weinig effectief. Enerzijds splijtten scherpe culturele
kloven de rivaliserende machten, en anderzijds beschikten zij vanwege
hun geringe lokale militaire macht niet over de fysieke capaciteit om
een onderhandelde oplossing af te dwingen. In het tijdperk van de
natiestaat, waarin nationale autoriteiten hun militaire macht effectief
konden inzetten om besluiten op te leggen, verdwenen de meeste
onduidelijke grenslijnen en werd duidelijke grensafbakening de norm.
Deze oplossing is stabiel wanneer twee machtige duopolisten
geconfronteerd worden met de noodzaak hun gezag over aaneengesloten
gebieden te verdelen. In de cybereconomie daarentegen concurreren immers
niet twee, maar honderden autoriteiten wereldwijd om economische
transacties te faciliteren. Territoriale staten zullen daardoor
nauwelijks in staat zijn een effectief kartel te vormen dat de
belastingtarieven op peil houdt. Net zoals collusie in markten met
honderden concurrenten niet werkt om monopolistische prijzen te
realiseren, geldt dit principe hier eveneens.

Als bewijs kun je denken aan de zet van de Seychellen -- een klein land
in de Indische Oceaan -- om een nieuwe investeringswet in te voeren die
door Amerikaanse regeringsfunctionarissen als een `verwelkom
criminalen'-wet wordt bestempeld. Volgens deze wet geniet iedereen die
10 miljoen dollar investeert in de Seychellen niet alleen van
bescherming tegen uitlevering, maar ontvangt hij ook een diplomatiek
paspoort. In tegenstelling tot wat de Amerikaanse regering beweert, zijn
de begunstigden geen drugshandelaren -- die doorgaans al door
invloedrijkere regeringen worden beschermd -- maar zelfstandige
ondernemers die als politiek incorrect worden beschouwd. De eerste
potentiële begunstigde is een blanke Zuid-Afrikaan die rijk werd door de
economische sancties tegen het voormalige apartheidsregime te omzeilen.
Nu loopt hij het risico economisch getroffen te worden door de nieuwe
Zuid-Afrikaanse regering en is hij bereid een bedrag aan de Seychellen
te betalen in ruil voor bescherming.\footnote{Zie Thomas W. Lippman,
  \emph{Seychelles offers investors safe haven for \$10 million},
  \emph{Washington Post}, 31 december 1995, p.~A27.}

Dit voorbeeld laat zien waarom overheden die proberen een kartel voor
bescherming op te zetten, uiteindelijk gedoemd zijn te mislukken --
ongeacht wat individuele gevallen opleveren.

Waar middeleeuwse grenzen de concurrentie tussen slechts twee
autoriteiten beperkten, bestaat de grens in de cyberhandel uit honderden
rechtsgebieden, waarvan het aantal naar verwachting snel zal oplopen tot
duizenden.

In het tijdperk van virtuele ondernemingen kiezen individuen er bewust
voor hun inkomstengenererende activiteiten te vestigen in een
rechtsgebied dat de beste service biedt tegen de laagste kosten. Met
andere woorden, soevereiniteit wordt gecommercialiseerd.

Terwijl middeleeuwse grensgemeenschappen vaak arm waren en met geweld te
kampen hadden, geldt dat voor cyberspace niet.

De concurrentie waarmee overheden op het gebied van
informatietechnologie worden geconfronteerd, draait niet om militaire
macht, maar om de kwaliteit en prijs van een economische dienst --
namelijk echte bescherming. Kortom, overheden moeten hun klanten bieden
waar zij naar verlangen.

\subsection{De verminderde effectiviteit van
geweld}\label{de-verminderde-effectiviteit-van-geweld}

Dit betekent uiteraard niet dat overheden het gebruik van geweld
volledig opgeven. Integendeel, we stellen dat geweld een groot deel van
zijn hefboomeffect verliest.

Overheden zouden er bijvoorbeeld voor kunnen kiezen om hun inzet van
geweld op lokaal niveau op te voeren in een poging de wereldwijd dalende
impact te compenseren. Wat ze ook ondernemen, ze slagen er in geen geval
in om cyberspace met geweld te verzadigen zoals ze ooit de fysieke
terreinen, die zij met geweld monopoleerden, volledig beheersten.

Hoeveel pogingen overheden ook doen om voet aan de grond te krijgen in
cyberspace, zij zullen in dat domein nooit machtiger worden dan iemand
anders.

Opmerkelijk genoeg versnellen staten die `informatieoorlogen' voeren --
met als doel de dominantie over, of het blokkeren van, de toegang tot
cyberspace -- waarschijnlijk eerder hun eigen ondergang. Grote systemen
raken immers al snel onhoudbaar door het wegvallen van schaalvoordelen
en de stijgende kosten om verspreide sociale groepen bijeen te houden.
Ironisch genoeg kunnen dergelijke informatieoorlogen een grotere schok
veroorzaken voor de fragiele structuren uit het industriële tijdperk dan
voor de opkomende informatie-economie zelf.

Zolang de cruciale informatietechnologie operationeel blijft, kunnen
cybercommerce en de strijd in de informatieoorlog naast elkaar
plaatsvinden op een manier die in een territoriale oorlog simpelweg niet
mogelijk is. Je kunt je toch nauwelijks voorstellen dat miljoenen
commerciële transacties ooit aan de frontlijn van een oorlog in de
twintigste eeuw plaatsvonden. Digitale conflicten putten de
veelzijdigheid van cyberspace niet uit, en aangezien virtuele realiteit
geen fysieke substantie kent, loopt niemand het risico geraakt te worden
door explosies van virtuele granaatfragmenten.

\subsection{Kwetsbaarheid van grootschalige
systemen}\label{kwetsbaarheid-van-grootschalige-systemen}

In een informatieoorlog vormen de gevaren met name een bedreiging voor
grootschalige industriële systemen die draaien op centraal bevel en
controle. Militaire autoriteiten in de Verenigde Staten en andere
toonaangevende landen maken zich zorgen over en bereiden zich voor op
daden van informatiesabotage die grote systemen kunnen uitschakelen.
Cyberaanvallen kunnen bijvoorbeeld een telefoonwisselstation stilleggen,
het luchtverkeersbeheer ontregelen of een pompsysteem dat de
watertoevoer naar een stad regelt saboteren. Een geprogrammeerd virus
kan zelfs conventionele of nucleaire stroomopwekkers doen uitvallen,
waardoor delen van het elektriciteitsnet tot stilstand komen. Zogenaamde
logic bombs zijn in staat om enorme hoeveelheden informatie te
vernietigen, waaronder de uiterst gevoelige gegevens in de centrale
controlesystemen die de kwetsbare, grote industriële installaties
aansturen. Tenzij men daadwerkelijk alle informatietechnologie massaal
vernietigt -- wat de wereldeconomie letterlijk tot stilstand zou brengen
-- blijven cybercommerce en virtuele realiteit buiten het bereik van
overheden om onderdrukt of gemonopoliseerd te worden.

Zelfs een belangrijk nadeel van informatietechnologie -- namelijk de
vermeende kwetsbaarheid van informatiestoragesystemen voor bederf en
vernietiging -- lost de recente archiveringstechnologie grotendeels op.
Een innovatief systeem, High-Density Read-Only Memory (HD-ROM), benut
een ionmolen, vergelijkbaar met die in computerondersteunde productie,
om in een vacuüm archieven te creëren. De opslagcapaciteit bereikt maar
liefst 25.000 megabytes per vierkante inch. Waar oudere systemen snel
achteruitgingen door bederf en schokken, verzekert HD-ROM dat de
opgeslagen data de volledige levensduur beschikbaar blijft. Eén van de
ontwikkelaars van HD-ROM, Bruce Lamartine, verklaart: `Het is vrijwel
ondoordringbaar voor de verwoestende invloeden van tijd, thermische en
mechanische schokken, alsmede elektromagnetische velden die andere
opslagmedia beschadigen.'\footnote{Zie \emph{ROM of ages}, \emph{Wired},
  januari 1996, p.~52.} Zelfs als nucleaire terroristen een explosie in
gang zetten, betekent dat niet per se dat vitale informatie -- zoals de
codes voor digitaal geld, waarop het soepele functioneren van een
cybereconomie steunt -- verstoord of vernietigd wordt.

\begin{quote}
``Moderne legers zijn zo afhankelijk van informatie dat het mogelijk is
hen te verblinden en doof te maken, zodat men de overwinning kan behalen
zonder op de conventionele wijze te vechten.''\footnote{Geciteerd in
  James Adams, \emph{Dawn of the cyber soldiers}, \emph{The Sunday
  Times} (London), 15 oktober 1995, pp.~3--5.}: - COL. ALAN CAMPEN,
U.S.A.F (Ret.)
\end{quote}

\section{Supermachten van virtuele
oorlogsvoering}\label{supermachten-van-virtuele-oorlogsvoering}

Naarmate informatie een steeds centralere rol speelt in de
oorlogsvoering, verliezen de traditionele opvattingen over de natiestaat
hun megapolitieke betekenis. Doordat cyberspace geen fysieke
aanwezigheid kent, spelen de grootheden die we in de echte wereld
hanteren daar geen doorslaggevende rol. Het maakt niet uit hoeveel
programmeurs meewerken aan de code; uiteindelijk telt alleen of het
programma werkt.

In cyberspace kan een soeverein individu evenveel invloed uitoefenen als
een natiestaat, compleet met een vertegenwoordiging bij de Verenigde
Naties, een eigen vlag en zelfs een leger op eigen grondgebied. Op
economisch vlak realiseren sommige individuen jaarlijks investeerbare
inkomsten in de honderden miljoenen, wat de discretionaire
bestedingsruimte van enkele failliete natiestaten ruimschoots
overschrijdt. Bovendien kunnen, als het gaat om virtuele oorlogsvoering
via informatie-manipulatie, bepaalde individuen net zo'n prominente --
zo niet een nog grotere -- rol spelen dan veel wereldstaten. Een bizarre
genie die samenwerkt met digitale dienaren kan in theorie tijdens een
cyberoorlog dezelfde slagkracht hebben als een natiestaat. Bill Gates
zou dat zeker kunnen.

In dat opzicht betekent het tijdperk van het soevereine individu meer
dan slechts een loze kreet. Een hacker of een klein team wiskundigen --
om nog maar te zwijgen van een bedrijf als \emph{Microsoft} of vrijwel
elk computersoftwarebedrijf -- kan in principe alles voor elkaar krijgen
wat de \emph{Cyber War Task Force} van het \emph{Pentagon} voor handen
heeft. Honderden bedrijven in Silicon Valley en elders beschikken al
over een grotere capaciteit om een cyberoorlog te voeren dan 90 procent
van de huidige natiestaten.

Wie denkt dat regeringen het leven op aarde blijven beheersen, loopt
achter de feiten aan nu er aan alle kanten alternatieve vormen van
bescherming opduiken. Veel waarschijnlijker is dat natiestaten zich
opnieuw moeten organiseren om hun kwetsbaarheid voor computervirussen,
logische bommen, geïnfecteerde draden en achterdeurprogramma's te
verkleinen -- bedreigingen die zelfs door de Amerikaanse \emph{National
Security Agency} of een tienerhacker in de gaten gehouden kunnen worden.

Volgens de megapolitieke logica van cyberspace moeten de centrale
commando‑ en controlesystemen, die de wereldwijde infrastructuur
domineren, plaatsmaken voor multicentrische beveiligingsmodellen met
gedistribueerde capaciteiten. Zo verklein je de kans dat een
computervirus de systemen overneemt of verlamt. Nieuwe software, bekend
als agorische open systemen, vervangt de traditionele commando‑ en
controlesoftware uit de Industriële Tijd. Die oude software verdeelde de
computercapaciteit volgens starre prioriteiten, vergelijkbaar met hoe de
centrale planners van het \emph{Gosplan} in de voormalige Sovjet-Unie
goederen toewijdden aan goederentreinen volgens vaste regels.

Nieuwe systemen draaien op algoritmen die marktwerking imiteren. Zo
wijzen zij middelen efficiënter toe via een intern biedproces dat de
competitieve mechanismen van de hersenen nabootst. In het nieuwe
millennium decentraliseren we cruciale commando‑ en controlefuncties in
plaats van ze te concentreren bij reusachtige computermonopolies.

Er is weinig dat de veerkracht van gedistribueerde netwerken zo treffend
illustreert ten opzichte van commandocentrums als het voorbeeld dat
Digital Equipment presenteerde in het Palo Alto-onderzoekscentrum. Een
ingenieur opende de deur van een kast waar het interne computernetwerk
van het bedrijf was gehuisvest. Volgens Kevin Kelly liet hij met
theatrale flair een kabel los, waarna het netwerk de breuk onmiddelijk
opving zonder enige aarzeling.

Het informatietijdperk bevordert niet alleen een concurrerende, maar
toch ordelijke cyberspace; het leidt onvermijdelijk tot een
herinrichting van belangrijke systemen die nog geworteld zijn in het
industrialisme. Zo'n herconfiguratie is cruciaal om deze systemen minder
kwetsbaar te maken voor kattenkwaad, ongeacht door wie of van waaruit
het komt. Net zoals de industriële tijd onvermijdelijk leidde tot de
herstructurering van instellingen die hun oorsprong vinden in de
middeleeuwen -- denk aan scholen en universiteiten -- zullen de
overgebleven structuren uit de industriële periode waarschijnlijk
verkleind voortbestaan op een wijze die de logica van microtechnologie
weerspiegelt.

Om zich te weren tegen digitale bandieten op de informatieweg, zullen we
massaal moeten inzetten op encryptie-algoritmen met publieke en
privésleutels. Daarmee kan iedere pc-gebruiker berichten beveiligen op
een manier die ver vooruitloopt op de lanceercodes die het Pentagon een
generatie geleden verzegelde. Dergelijke krachtige, onbreekbare
versleuteling wordt essentieel om financiële transacties te beschermen
tegen hackers en dieven.

Bovendien komt daar nog een andere reden bij. Zodra private financiële
instellingen en centrale banken inzien dat de Amerikaanse regering -- en
waarschijnlijk niet de enige -- de huidige banksystemen en software kan
binnendringen om een heel land failliet te laten gaan of bijna alle
bankrekeningen leeg te vegen, zullen zij overstappen op waterdichte
encryptie-algoritmen. Er bestaat simpelweg geen technologische
rechtvaardiging waarom iemand of een heel land zijn financiële
deposito's en transacties zou moeten toevertrouwen aan de grillen van de
Amerikaanse National Security Agency, opvolgers van de KGB of enige
vergelijkbare organisatie, ongeacht of zij legaal of illegaal opereert.

Encryptie-algoritmen die zelfs voor overheden ondoordringbaar zijn,
behoren niet tot dagdromen. Ze zijn al beschikbaar als shareware op het
internet. Zodra lage-orbit satellietsystemen volledig operationeel zijn,
kunnen mensen met geavanceerde personal computers en antennes zo compact
als die van draagbare telefoons wereldwijd communiceren, zonder dat ze
het conventionele telefoonsysteem hoeven te gebruiken. Daarnaast zal het
voor iedere overheid even onmogelijk zijn om cyberspace te monopoliseren
-- een rijk zonder fysieke aanwezigheid -- als dat middeleeuwse ridders
in de industriële periode transacties konden beheersen terwijl ze op een
zwaar rijdier reden.

\subsection{Bescherming door
onopvallendheid}\label{bescherming-door-onopvallendheid}

In informatietechnologische samenlevingen worden enorme middelen zo
ingericht dat ze buiten het bereik van roofzuchtige criminelen blijven.
Naarmate cyberspace een steeds belangrijker toneel wordt voor financiële
transacties en andere vormen van handel, verlenen de gebruikte middelen
vrijwel een natuurlijke immuniteit tegen veelvoorkomende afpersing en
diefstal. Daardoor krijgen plunderaars veel minder de kans om een even
groot deel van de middelen te bemachtigen als vroeger het geval was en
gedurende een groot deel van de twintigste eeuw gebeurde.

Uiteindelijk wordt de overheidsbescherming van een aanzienlijk deel van
de wereldwijde rijkdom overbodig. Geen enkele overheid kan immers een
banksaldo in cyberspace beter beveiligen dan u zelf. Doordat de rol van
de overheid afneemt, zal haar relatieve prijs vanzelf dalen -- en daar
spelen nog andere factoren een rol.

Nu in het nieuwe millennium een groot en groeiend deel van de financiële
transacties in cyberspace plaatsvindt, krijgen individuen de
mogelijkheid hun zaken onder te brengen in diverse rechtsgebieden. Dit
leidt tot felle concurrentie over de tarieven van overheidsdiensten --
oftewel de belastingen -- die op een niet-monopolistische basis worden
vastgesteld. Dit is werkelijk revolutionair. Zoals George Melloan in
\emph{The Wall Street Journal} betoogde, is de verzorgingsstaat de enige
instelling die de druk van mondiale concurrentie tot nu toe met succes
heeft weerstaan. `Een studie van onderzoekers van de \emph{Wharton
School} en de \emph{Australian National University} onderzocht de
krachten die invloed uitoefenen op inkomensoverdrachten. Geoffrey
Garrett en Deborah Mitchell concludeerden dat er vrijwel geen bewijs
bestaat dat een toegenomen marktintegratie neerwaartse druk zet op de
fundamentele welzijnsprogramma's.' Integendeel, schrijven zij:
`Overheden hebben steevast gereageerd op een toenemende integratie in
internationale markten door de inkomensoverdrachten te verhogen.'

De opkomst van de cybereconomie zal de verzorgingsstaat eindelijk
blootleggen aan échte concurrentie. Het verandert de aard van
soevereiniteiten en transformeert economieën, omdat de balans tussen
bescherming en afpersing steeds meer in het voordeel van bescherming
kantelt dan ooit tevoren.

\bookmarksetup{startatroot}

\chapter{Plaatsgebondenheid
overstijgen}\label{plaatsgebondenheid-overstijgen}

\begin{quote}
`Waar het werkelijk om draait, is controle. Het internet verspreidt zich
zo breed dat geen enkele regering het moeiteloos kan beheersen. Door een
naadloze, wereldwijde economische zone te creëren die losstaat van
nationale soevereiniteit en niet gereguleerd kan worden, stelt het
internet het idee van een natiestaat ter discussie.'\footnote{Perry
  Barlow, `Thinking Locally, Acting Globally', \emph{Time}, 15 januari
  1996, p.~57.} - John Perry Barlow
\end{quote}

De informatiesnelweg is uitgegroeid tot een van de bekendste metaforen
uit de beginjaren van het digitale tijdperk. Zij springt in het oog,
niet alleen vanwege haar alomtegenwoordigheid, maar ook door het
misverstand dat zij schetst over de cybereconomie. Een snelweg fungeert
immers als een industriële variant van een voetpad, een netwerk dat
mensen en goederen fysiek verplaatst. De informatie-economie laat zich
echter niet vergelijken met een snelweg, spoorlijn of pijpleiding. In
tegenstelling tot de Trans‑Canada Highway, die zware vrachtwagens van
Alberta naar New Brunswick vervoert, verplaatst de informatiesnelweg
geen informatie van punt naar punt. Wat men `informatiesnelweg' noemt,
is dus niet louter een verbindingsweg, maar tevens de bestemming zelf.

Cyberspace overstijgt de beperkingen van fysieke locaties. Dit houdt in
dat iedereen direct en op elke plek gegevens deelt. De opkomende
informatie-economie rust op netwerken waarin miljoenen gebruikers van
talloze computers voortdurend met elkaar in contact komen. De kern ervan
schuilt in de nieuwe mogelijkheden die deze verbindingen opleveren.
Zoals John Perry Barlow opmerkte: `Wat het net ons biedt, is de belofte
van een nieuwe sociale ruimte, globaal en anti-soeverein, waarin
iedereen, overal, zonder angst kan uiten wat hij of zij gelooft. In deze
nieuwe media zien we alvast een voorbode van de intellectuele en
economische vrijheid die alle autoritaire machten op aarde zou kunnen
ontmantelen.'\footnote{Ibid.}

Cyberspace doet denken aan het denkbeeldige rijk van Homers goden; het
is een domein dat losstaat van onze vertrouwde, aardse wereld vol
boerderijen en fabrieken. Toch raken de gevolgen concreet: op een veel
grotere schaal dan velen zich nu realiseren, werkt het directe delen van
informatie als een oplosmiddel dat grote instellingen doet instorten.
Het verandert niet alleen de logica van geweld, zoals we eerder betoogd
hebben, maar past ook de informatie‑ en transactiekosten -- de
sleutelfactoren voor de organisatie van bedrijven en het functioneren
van de economie -- ingrijpend aan. Wij verwachten dat microverwerking de
wereldwijde economische organisatie transformeert.

\begin{quote}
`Het is tegenwoordig, in veel grotere mate dan ooit in de geschiedenis
van de wereld, mogelijk voor een bedrijf zich overal te vestigen, overal
middelen te benutten om een product te vervaardigen dat overal verkocht
kan worden.' - MILTON FRIEDMAN
\end{quote}

\section{De tirannie van de plaats}\label{de-tirannie-van-de-plaats}

Het feit dat het afnemende industriële tijdperk de
informatiemaatschappij probeert te doorgronden door haar te beschouwen
als een enorm openbaar werkproject, bewijst hoe stevig onze ideeën nog
verankerd liggen in oude denkkaders. Het doet denken aan hoe boeren aan
het eind van de achttiende eeuw een fabriek omschreven als `een
boerderij met een dak.' Toch is de metafoor van de `snelweg' nog
onthullender. Ze maakt bovendien duidelijk hoezeer we nog gevangen
zitten in de tirannie van de plaats. Zelfs als technologie ons in staat
stelt de beperkingen van een specifieke locatie te doorbreken, krijgt
het middel van onze bevrijding een naam die impliceert dat het slechts
een route is van de ene plaats naar de andere. Net zoals zalmen,
gedreven door hun drang naar thuis, blijft ons denken diep beïnvloed
door ideeën over localiteit.

Doorheen de geschiedenis waren economieën altijd verbonden met een
lokaal geografisch gebied. De meeste mensen die vóór de twintigste eeuw
leefden, brachten hun dagen door als ware huisgevangenen en waagden zich
zelden verder dan een paar dagen lopen van de plek waar ze geboren
waren. Een reis over een ruime afstand vereiste vaak de inzet van
generaties. Alleen af en toe leidden crises, oorlogen, pestuitbraken of
ongunstige klimaatveranderingen tot een grootschalige migratie. Iets
spectaculairs en dringends was nodig om mensen uit een ellendig dorp weg
te lokken; niets minder kon hen motiveren om hun spullen in te pakken en
op zoek te gaan naar een beter bestaan.

Tot voor kort genoten de weinigen die buiten hun vertrouwde omgeving
kansen zochten vaak grote bekendheid. Denk maar: Marco Polo staat nog
altijd bekend om zijn reizen over het Euraziatische continent naar het
hof van de Grote Khan. Hij was in zijn tijd de ware uitzondering. Andere
reisverslagen uit de premoderne periode zijn nauwelijks overgeleverd.
Een van de meest gelezen werken is \emph{Mandeville's Travels},
geschreven in het Frans in 1357, dat opvalt omdat het vermoedelijk door
iemand is samengesteld die Europa nooit heeft verlaten. Mandeville
beschrijft op verleidelijke en vaak fantasierijke wijze het leven in de
wereld, en suggereert daarbij dat veel Ethiopiërs slechts één voet
zouden hebben: ``{[}D{]}e voet is zo groot dat hij het hele lichaam
overschaduwt tegen de zon wanneer zij gaan liggen en
rusten.''\footnote{M. C. Seymour (red.), Mandeville's Travels (Oxford:
  \emph{Oxford University Press}, 1968), p.~122.}

Pas toen de moderne tijd ingeluid werd met ontdekkingsreizen aan het
einde van de vijftiende eeuw, ontstonden er structurele contacten tussen
de continenten. Onverschrokken kapiteins als Christopher Columbus en
Vasco da Gama, die op zoek waren naar de specerijenhandel, waren zo
buitengewoon dat zij gedurende vrijwel alle vijf eeuwen in de
herinnering van elk geletterd huishouden zijn gebleven.

Vanaf de opkomst van de landbouw tot in de recente generaties werd het
leven bepaald door stilstand. Tegenwoordig is dat bijna vervlogen,
vooral in de Europese vestigingskolonies in de `Nieuwe Wereld', waar
mobiliteit vanzelfsprekend is en iedereen het leven bekijkt vanuit het
perspectief van een immigrant. Een centraal thema in het
Noord-Amerikaanse basisonderwijs is dat de kolonisten uit Europa op zoek
gingen naar vrijheid en kansen -- en dat klopt. Wat echter zelden aan
bod komt, is hoe terughoudend de meesten waren om op reis te gaan,
ondanks de armoede waarmee ze thuis geconfronteerd werden. De enkelen
die wel migreerden, ondergingen onvoorstelbare ontberingen om zich
ergens te vestigen -- iets wat wij ons tegenwoordig nauwelijks kunnen
voorstellen. Alleen de meest ondernemende of wanhopige armen durfden de
reis te maken. In het midden van de zeventiende eeuw gingen de
gevangenen uit Bridewell, het beruchte correctiehuis van Londen, in
opstand om te laten zien hoe onwillig zij waren om naar Virginia te
vertrekken.\footnote{R. C. Johnson, `The Transportation of Vagrant
  Children from London to Virginia, 1618--1622', in H. S. Reinmuth
  (red.), \emph{Early Stuart Studies} (Minneapolis: \emph{University of
  Minnesota Press}, 1970), pp.~43--44, geciteerd in Jutte, op. cit.,
  p.~168.} In 1720 laaiden de rellen in de straten van Parijs op, toen
zwervers, dieven en moordenaars -- bestemd voor deportatie naar
Louisiana -- vrij werden gelaten.

\subsection{Smalle horizonnen}\label{smalle-horizonnen}

Fysieke beperkingen in communicatie en transport, vaak versterkt door
een beperkte taalvaardigheid, beperkten menselijk handelen tot het
lokale gebied. Al in het begin van de twintigste eeuw spraken Chinese
dorpen, die op slechts vijf mijl van elkaar lagen, dialecten die elkaar
totaal niet verstonden, zelfs niet langs de kust. De lokale organisatie
van vrijwel alle economieën legde zo de basis voor kleine markten en
gemiste kansen. Door de beperkte concurrentie bleven de kosten van
productiefactoren hoog en was de toegang tot specialistische
vaardigheden minimaal. Kleine boeren in grote delen van de wereld
raakten door hun uiterst lage inkomens -- nauwelijks boven de
bestaansgrens -- en het ontbreken van toegang tot extern kapitaal of
efficiënte verzekeringsmarkten, verstrikt in armoede. We belichten
enkele van de problemen waarmee boeren in het beperkte dorpsleven werden
geconfronteerd. Zelfs nu, terwijl we dit schrijven, worstelt minstens
een miljard mensen -- voornamelijk in Azië en Afrika -- om rond te komen
op minder dan een dollar per dag.

\section{`All politics is local'}\label{all-politics-is-local}

Veel meer dan men zich vaak realiseert, heeft de beperkte mobiliteit van
mensen en hun eigendommen onze kijk op de wereld gevormd. Zelfs degenen
die volhouden dat de aarde inmiddels klein is, denken nog steeds in
verouderde termen over industriële politiek. Dit illustreert een slogan
die in de jaren tachtig populair was onder milieubewuste mensen: `Denk
globaal maar handel lokaal.' De slogan weerspiegelt immers een politiek
die altijd draaide om lokale machtsvoordelen.

Lokale denkwijzen werden altijd bepaald door de megapolitiek van
vroegere samenlevingen. Alle topografische kenmerken die als obstakels
of als stimulans voor machtsuitoefening fungeerden, dragen een
intrinsiek lokale lading. Elke rivier, elke berg en elk eiland heeft
zijn eigen betekenis op lokaal niveau. Ook het klimaat kent lokale
variaties: temperatuur, neerslag en groeicondities veranderen naarmate
je een berg beklimt of oversteekt. Elk micro-organisme beweegt zich
binnen een bepaalde omgeving en niet zomaar overal.

Het verbaast dus niet dat de tirannie van de plaats ons denken over de
organisatie en werking van de samenleving doordringt. De machtsvoordelen
die sommige groepen behaalden met een lokaal geweldmonopolie, hebben
altijd hun oorsprong ergens in de lokale context en vervagen aan de
megapolitieke randen waar men grenzen trekt. Daarom heeft een
wereldregering nooit bestaan.

Hoewel men zelden expliciet benoemt hoe belangrijk de locatie is voor
het uitoefenen van macht, merkten enkele voorstanders van een gedwongen
herverdeling van de vruchten van menselijke activiteit al in de jaren
dertig dat de invloed van de plaats afnam. Zij zagen in modern vervoer
een duidelijke scheiding tussen de sociale ruimten van de hoogbetaalde
en de minderverdienenden. John Dos Passos verwoordde deze zorg treffend
in \emph{The Big Money}: `De zwerver zit aan de rand van de snelweg,
gebroken, hongerig. Boven hem vliegt een transcontinentaal vliegtuig vol
hoogbetaalde leidinggevenden. De hogere klasse heeft zich in de lucht
begeven, de lagere klasse op de weg: er is geen band meer tussen hen,
het zijn twee naties.'\footnote{John Dos Passos, \emph{The Big Money}
  (New York: \emph{Harcourt, Brace \& Co.}, 1936).} Hiermee geeft hij
aan dat verbeterd vervoer de hefboomwerking van afpersing verminderde,
simpelweg doordat succesvolle mensen een veel grotere keuze kregen qua
vestigingsplaats. De zwerver langs de weg kon immers nooit eisen dat hij
profiteerde van degenen die hoog in de lucht reisden. De tendensen die
Dos Passos zestig jaar geleden opmerkte, zijn sindsdien alleen maar
sterker geworden.

\subsection{Massatransport}\label{massatransport}

In 1995 staken dagelijks, ergens ter wereld, ongeveer een miljoen mensen
de grenzen over -- een opmerkelijke verandering vergeleken met vroeger.
In de twintigste eeuw was reizen zo zeldzaam dat grenzen vooral als
randgebieden werden gezien en nauwelijks een belemmering vormden voor
doorgang; paspoorten waren nog onbekend. De opkomst van oceaanstomers,
treinen en andere verbeterde vervoermiddelen leidde tot een
spectaculaire toename van verplaatsingen. Tegelijkertijd voerden staten
steeds strengere regels in, mede dankzij de verbeterde transport- en
communicatiemiddelen die reizen voor burgers zowel goedkoper als
eenvoudiger maakten. Films en, vooral, televisie speelden daarnaast een
belangrijke rol in het verbreden van horizonten en het stimuleren van
reizen en immigratie. Desondanks blijven de fundamentele beginselen van
de sociale en economische organisatie verankerd in de lokale context.

\begin{quote}
`We moeten voorkomen dat we ons lef verliezen, waarvoor de geschiedenis
meedogenloos straft. We moeten de moed hebben om alle technische
extrapolaties tot hun logische conclusie door te voeren.'\footnote{Clarke,
  op. cit., p.~29.}: - Arthur C. Clarke
\end{quote}

\section{De fout van minimale
verwachtingen}\label{de-fout-van-minimale-verwachtingen}

De beperkte verbeelding blijft nog altijd zo strak begrensd dat een
aantal experts, die in 1995 het internet onder de loep namen,
concludeerden dat het weinig commercieel potentieel bezit en vrijwel
geen betekenis heeft, behalve als medium voor chat en als uitlaatklep
voor pornografie. De vele sceptici over het economische belang van
cyberspace vormen de kolonelblimps van het informatietijdperk. Hun
zelfgenoegzaamheid evenaart die van het Britse establishment, dat in de
jaren dertig te maken kreeg met de achteruitgang van het rijk. Telkens
wanneer elites zich bedreigd voelen, reageren zij allereerst door de
dreiging te ontkennen. Dit blijkt uit hun verwachting dat het internet
nooit meer dan een bijzaak zal blijven, een visie die soms zelfs door
autoriteiten wordt gedeeld die beter hadden moeten weten. We verwezen
eerder naar het werk van David Kline en Daniel Burstein, \emph{Road
Warriors: Dreams and Nightmares Along the Information Highway}. Hun
afwijzing van het economische potentieel van het net levert extra bewijs
dat technische onderlegdheid niet synoniem staat met het doorgronden van
technologische gevolgen.\footnote{Geciteerd in Kline en Burstein, op.
  cit., p.~05.}

Zelfs technisch onderlegde waarnemers slaagden het in het verleden vaak
niet om de implicaties van nieuwe technologieën te doorgronden. In 1878
kwam een Brits parlementair comité, dat de vooruitzichten voor Thomas
Edison's gloeilamp moest beoordelen, tot de conclusie dat Edison's
ideeën `goed genoeg zijn voor onze trans-Atlantische vrienden, \ldots{}
maar de aandacht van praktische of wetenschappelijke mannen niet waard'
waren.\footnote{Clarke, op. cit., p.~20.}

Nog net voordat de gebroeders Wright aantoonden dat vliegtuigen konden
vliegen, liet de vooraanstaande Amerikaanse astronoom Simon Newcomb op
gezaghebbende wijze zien waarom vliegen met een voertuig dat zwaarder is
dan lucht onmogelijk is. Hij stelde: `Het is aantoonbaar dat geen enkele
combinatie van bekende materialen, bekende machinebouwtechnieken en
bekende vormen van kracht samen een praktische machine oplevert waarmee
mensen grote afstanden door de lucht kunnen reizen, zo overtuigend als
elk natuurkundig feit kan worden bewezen.'\footnote{Ibid.}

Gezien deze traditie van misverstanden is het nauwelijks verrassend dat
velen er lang over doen om de belangrijkste implicaties van de nieuwe
informatietechnologie te doorgronden, namelijk dat deze de tirannie van
de plaats overstijgt. De nieuwe technologie opent voor het eerst een
oneindig, locatieonafhankelijk domein voor economische activiteiten. Het
biedt de mogelijkheid om de grenzen van de cybereconomie te verkennen en
zo `wereldwijd te denken en te handelen'. Dit hoofdstuk legt uit waarom.

\section{Voorbij de
plaatsgebondenheid}\label{voorbij-de-plaatsgebondenheid}

De verwerking en het gebruik van informatie verdringen razendsnel
fysieke producten als voornaamste bron van winst. Dit heeft
verstrekkende gevolgen. Informatietechnologie koppelt de mogelijkheid om
inkomen te genereren niet langer aan één specifieke woonplaats. Doordat
steeds meer waarde ontstaat door ideeën en kennis aan producten en
diensten toe te voegen, valt een kleiner aandeel van die toegevoegde
waarde binnen lokale rechtsgebieden. Ideeën ontstaan overal en kunnen
met de snelheid van het licht over de hele wereld verspreid worden. Dit
impliceert onvermijdelijk dat de informatie-economie radicaal verschilt
van de economie van het industriële tijdperk.

We geven de critici toe dat een opsomming van taken die je in 1996 via
het internet kon uitvoeren, op het eerste gezicht alledaags lijkt.
Tenslotte is er niets revolutionairs aan het lezen van een artikel over
tuinieren op het internet of aan het op afstand aanschaffen van een kist
wijn. Het potentieel van de cybereconomie laat zich echter niet louter
beoordelen op basis van haar vroege beginfase, net zoals men in 1900
niet kon inschatten welke transformatieve impact de auto op de
samenleving zou hebben. Wij verwachten dat de cybereconomie zich in
verschillende fasen zal ontwikkelen.

\begin{enumerate}
\def\labelenumi{\arabic{enumi}.}
\item
  De basale toepassingen in het informatietijdperk zien het internet als
  een simpel informatiemedium dat de gangbare transacties uit de
  industriële periode ondersteunt. Momenteel fungeert het internet
  eigenlijk louter als een exotisch bezorgingssysteem voor catalogi.
  Virtual Vineyards, een van de eerste cyberhandelaren, verkoopt gewoon
  wijn via een webpagina op het World Wide Web. Zulke transacties
  ondermijnen de bestaande instituties nog niet, omdat ze in industriële
  valuta plaatsvinden en binnen bekende rechtsgebieden opereren.
  Hierdoor blijft de megapolitieke impact beperkt.
\item
  In een tussentijdse fase van internetcommercie benut men
  informatietechnologie op manieren die in de industriële tijd
  onmogelijk waren, bijvoorbeeld voor boekhouding op afstand of voor
  medische diagnostiek. Hieronder worden nog enkele voorbeelden gegeven
  van deze nieuwe toepassingen van geavanceerde rekenkracht. Ook in dit
  tweede stadium blijft men binnen het oude institutionele kader werken:
  handelaren gebruiken nationale valuta en vallen onder de jurisdictie
  van natiestaten. Zij zetten het internet uitsluitend in om inkomsten
  te genereren en beheren er hun winsten nog niet mee; deze winsten
  blijven bovendien aan belastingheffing onderworpen.
\item
  Een verder gevorderd stadium markeert de overgang naar echte
  cybercommercie. Transacties vinden niet alleen via het internet
  plaats, maar vallen ook buiten de jurisdictie van natiestaten. Men
  verricht betalingen in cybervaluta, boekt winsten in cyberbanken en
  investeert via cybermakelaardijen. Veel transacties blijven bovendien
  onbelast. In dit stadium krijgt cybercommercie al aanzienlijke
  megapolitieke gevolgen. De nieuwe logica van het internet
  transformeert de traditionele machtspositie van overheden over
  economische domeinen. Overheden verliezen hun extraterritoriale
  regelgevende macht, jurisdicties decentraliseren en bedrijven
  reorganiseren zich, terwijl ook de aard van werk en tewerkstelling
  ingrijpend verandert. Deze schets van de fasen in de
  informatierevolutie biedt slechts een oppervlakkige indruk van wat
  mogelijk de meest ingrijpende economische transformatie ooit wordt.
\end{enumerate}

\section{De globalisering van handel}\label{de-globalisering-van-handel}

In het informatietijdperk zullen technologische ontwikkelingen de meeste
traditionele jurisdictievoordelen snel tenietdoen. Tegelijkertijd
ontstaan er nieuwe vormen van voordeel. Lagere communicatiekosten hebben
de noodzaak om fysiek aanwezig te zijn voor zakendoen al sterk
verminderd. In 1946 kon een investeerder in Londen via een makelaar in
New York een order plaatsen, maar alleen de grootste en meest
overtuigende transacties rechtvaardigden dat: een drie minuten durend
telefoongesprek tussen New York en Londen kostte toen \$650.
Tegenwoordig betaal je daar slechts \$0,91 voor. In een halve eeuw is de
prijs van een intercontinentaal telefoongesprek met meer dan 99 procent
gedaald.

\subsection{Convergente communicatie}\label{convergente-communicatie}

Binnenkort merk je nauwelijks verschil tussen intercontinentale chat en
een lokaal telefoongesprek. Ook vervagen de verschillen tussen je
telefoon, computer en televisie steeds meer; je onderscheidt ze immers
vooral op basis van ergonomie in plaats van functionaliteit. Met je
persoonlijke computer voer je spraakgesprekken via het internet,
gebruikmakend van de ingebouwde microfoon en luidsprekers, en bekijk je
films. Daarnaast kun je met je televisie communiceren en grote
hoeveelheden data uitwisselen via netwerken die door de
televisie-entertainmentmedia worden aangeboden. Zodra het onderscheid
tussen de verschillende vormen van communicatie in het industriële
tijdperk verdwijnt en de kosten kelderen, reken je steeds vaker per
gebruiksduur in plaats van per bestemming van je berichten. Al met al
betaal je straks voor gesprekken en datatransmissies wereldwijd
nauwelijks meer dan wat je in 1985 voor een lokaal telefoongesprek
betaalde.

\subsection{Internet onbedraad}\label{internet-onbedraad}

Innovatieve satellieten in een lage baan en andere doorbraken op het
gebied van draadloze technologie zenden signalen rechtstreeks naar een
beeper in je zak, een draagbare computer of een werkstation -- helemaal
zonder aansluiting op een lokaal telefonienet of kabel-tv-systeem.
Kortom, het internet werkt straks volledig draadloos. De eerste stappen
in deze richting verlopen ongetwijfeld stroef: de datadoorvoersnelheid
in vroege draadloze media is nog relatief laag en het opvangen van
zwakke signalen van mobiele, op batterijen werkende abonneetoestellen
brengt zijn uitdagingen met zich mee. Niettemin pakken we deze
technische problemen aan en lossen we ze op.

\subsection{Zakendoen zonder grenzen}\label{zakendoen-zonder-grenzen}

De voortdurende toename van rekencapaciteit leidt tot geavanceerdere
compressietechnieken, wat de gegevensstroom versnelt. Door bestaande
algoritmen voor encryptie met publieke en private sleutels op grote
schaal toe te passen, kunnen aanbieders -- zoals satellietsystemen -- de
factureringsfunctie naadloos in hun dienst integreren en zo kosten
besparen. Tegelijkertijd krijgen leveranciers de mogelijkheid om
rekeningen die op pc's zijn geladen, direct te belasten, net zoals
\emph{France Telecom} de `smartcards' in de telefoonhokjes in Parijs
debiteert.

\subsection{De telefoon wordt een
bank}\label{de-telefoon-wordt-een-bank}

Het verschil is dat je binnenkort door diverse transacties tegoed
opbouwt en dat je je telefooncel altijd bij je hebt. Je pc fungeert als
het filiaal van je bank en als wereldwijde betalingsfacilitator,
vergelijkbaar met dat Parijse loket waar je anonieme telefoonkaarten
aanschaft. Bovendien, net zoals de 150 betaaltelefoons met smartcards --
die voor inbrekers vrijwel nutteloos zijn wanneer ze met een koevoet
opengebroken worden -- kan jouw computer alleen gehackt worden door
iemand die geavanceerde computercode weet te kraken of manipuleren,
waardoor veel ruige inbrekers buiten spel blijven.

Met de juiste encryptie kun je er zeker van zijn dat niemand jouw
computerdata ontcijfert of misbruikt.

Tegen het millennium kun je vrijwel overal zaken doen -- zelfs ten
noorden van Antarctica. Je bevindt je in omgevingen waar bedrade of
digitale mobiele telefonie beschikbaar is, waar interactieve
kabeltelevisiesystemen draaien of waar een satelliet of ander draadloos
transmissiesysteem actief is. Je kunt communiceren, data uitwisselen en
dankzij virtual reality moeiteloos grenzen en barrières oversteken.
Traditionele telefoonnummers, die met netnummers de locatie van de
beller aangeven, maken waarschijnlijk plaats voor universele
toegangsnummers waarmee je wereldwijd contact legt met de gewenste
partij.

\subsection{Chinees begrijpen}\label{chinees-begrijpen}

U zult niet enkel kunnen praten en faxen. Na verloop van tijd verkort u
het jarenlange leertraject, zodat u in het Chinees kunt converseren met
een voorman in een fabriek in Shanghai. Het maakt weinig uit dat u zijn
taal of dialect niet spreekt. Hoewel hij in het Chinees communiceert,
vangt u zijn woorden op als `transcending locality i 8' -- ongeveer
vertaald naar het Engels. Hij hoort immers uw gesprek in het Chinees.
Spoedig vergroot uw vermogen tot onmiddellijke vertaling de
concurrentiekracht in regio's waar taal- en uitdrukkingsbarrières
voorheen een struikelblok vormden. Op dat moment maakt het nauwelijks of
helemaal niet uit dat de Chinese regering misschien bezwaar heeft tegen
het plaatsen van de oproep.

\subsection{Gepersonaliseerde media}\label{gepersonaliseerde-media}

Naarmate de wereld steeds meer samenkomt, krijgt u meer mogelijkheden
dan ooit om uw eigen positie vorm te geven. Ook de informatie die u via
de media binnenkrijgt, kiest u zelf. De traditionele massamedia maakt
plaats voor gepersonaliseerde media. Bent u een fanatieke schaker of een
fervent kattenliefhebber? Dan kunt u uw avondnieuws zo inrichten dat het
uitsluitend nieuws bevat over uw favoriete onderwerpen. U hoeft niet
langer afhankelijk te zijn van Dan Rather of van \emph{BBC} voor uw
nieuwsvoorziening. U selecteert zelf het nieuws dat volledig is
afgestemd op uw wensen.

\subsection{Van massa naar op maat gemaakte
productie}\label{van-massa-naar-op-maat-gemaakte-productie}

Als het nieuws traag verloopt, kun je een virtuele catalogus raadplegen
op het \emph{World Wide Web}. Als je een broek ziet die je bijna bevalt,
kun je bij je bestelling de breedte van de pijpen aanpassen. De broek
wordt dan op maat gesneden en door robots in Maleisië nauwkeurig
afgestemd op je lichaam, op basis van foto's die je via je computer
scant en over het internet verstuurt.

\subsection{Cyberbroking}\label{cyberbroking}

Je kunt cybergeld inzetten om te investeren en voor diensten en
producten te betalen. Als je in een rechtsgebied woont zoals de
Verenigde Staten, waar investeringsmogelijkheden streng gereguleerd
zijn, kies je er bewust voor je activiteiten onder te brengen in een
omgeving die volop vrijheid biedt in investeringskeuzes. Of je nu in
Cleveland of in Belo Horizonte woont, je kunt je investeringszaken
regelen in Bermuda, op de Kaaimaneilanden, in Rio de Janeiro of in
Buenos Aires. Waar je ook bent, digitale middelen krijgen steeds meer
gewicht naarmate de cybereconomie floreert. Je zet expertssystemen in om
je investeringen te selecteren en schakelt cyberaccountants en
-boekhouders in om de voortgang van je portefeuille in realtime te
volgen.

\subsection{Virtuele cultuur}\label{virtuele-cultuur}

Wanneer je even niet bezig bent met winst- en verliescijfers, kun je een
virtueel bezoek brengen aan het Louvre. Voordat je op pad gaat, moet je
mogelijk een royalty betalen ter waarde van een derde van een cent aan
Bill Gates of aan iemand met een vergelijkbare vooruitziende blik die de
rechten op virtuele realiteit voor museumbezoeken heeft verworven.
Terwijl je je afvraagt of de Mona Lisa ooit problemen met haar tanden
had, downloadt je computer ondertussen S.~I.~Hsiung's vertaling van De
romance van de westerse kamer. Op het moment dat jij dat wilt, leest je
persoonlijke communicatiesysteem de tekst voor, als een bard uit weleer.
Dankzij multitaskingprogramma's kun je meerdere functies gelijktijdig
uitvoeren.

\subsection{Het uitkiezen van rechtsgebieden op het
internet}\label{het-uitkiezen-van-rechtsgebieden-op-het-internet}

Als je geïnspireerd bent door je dosis klassiekers, kun je een virtuele
onderneming opzetten om indrukwekkende uitvoeringen van bekende
literaire werken te presenteren via een driedimensionale
netvliesweergave. In plaats van dat de beelden in de lucht worden
geprojecteerd, verschijnen ze rechtstreeks op de netvliezen van de
kijkers met laagenergetische lasers die vijftigduizend keer per seconde
pulseren. Deze technologie, die door \emph{MicroVision} uit Seattle,
Washington al in ontwikkeling is, stelt veel mensen die wettelijk blind
zijn in staat om te zien. Voordat je aan het project begint, geef je je
digitale assistent de opdracht om de huidige contractaanbiedingen voor
de bescherming van productiefaciliteiten in Maleisië, China, Peru,
Brazilië en Tsjechië in kaart te brengen. Zodra je een locatie kiest,
regelt \emph{St.~George's Trust Company} dat je bedrijf binnen een uur
wordt geregistreerd in de Bahama's. Je instructies zorgen ervoor dat
alle liquide activa van je bedrijf op een cyberrekening worden
ondergebracht bij een cyberbank die tevens actief is in Newfoundland, de
Kaaimaneilanden, Uruguay, Argentinië en Liechtenstein. Als een van deze
rechtsgebieden besluit de operationele bevoegdheid in te trekken of de
activa van de depositohouders in beslag te nemen, worden de activa
automatisch met de lichtsnelheid naar een ander rechtsgebied
overgeheveld.

\section{Kwalitatieve vooruitgang}\label{kwalitatieve-vooruitgang}

Veel transacties die je binnenkort in cyberspace kunt uitvoeren, waren
in het industriële tijdperk ondenkbaar -- en dat komt niet alleen
doordat ze een taalbarrière overschreden. Het inzetten van digitale
assistenten om onvertaalde artikelen uit Hongaarse wetenschappelijke
tijdschriften te verzamelen, onderscheidt zich qua kwaliteit van een
gesprek met een bibliothecaris. Het deelnemen aan een Oxford-tutorial op
een afstand van vijfduizend mijl is niet te vergelijken met het volgen
van diezelfde tutorial terwijl je binnen zes mijl van Carfax slaapt. En
roulette spelen in het \emph{Hotel de Paris} in Monte Carlo biedt een
totaal andere ervaring wanneer je dit via \emph{virtual reality} vanuit
een feest in Punte del Este, Uruguay beleeft.

\subsection{Een cyberbezoek aan de
cyberdokter}\label{een-cyberbezoek-aan-de-cyberdokter}

Binnen korte tijd -- misschien wel sneller dan veel deskundigen
verwachten -- verschuift het economische landschap naar de
cybereconomie. Hierbij combineer je technologieën op vernieuwende wijze
om de beperkingen van locatiegebondenheid en de achterhaalde instituties
van de industriële economie te doorbreken.

Al snel, wanneer je buikpijn krijgt, raadpleeg je een digitale dokter --
een expertsysteem met encyclopedisch inzicht in symptomen, kwalen en
tegengif. Dit systeem doorzoekt, in versleutelde vorm, je medische
geschiedenis en vraagt of je pijn ervaart na of vóór de maaltijd. Of de
pijn scherp of dof, constant of sporadisch is, de digitale dokter stelt
alle vragen die een arts zou stellen. Hij kan daarbij vaststellen dat je
te veel of juist te weinig wijn drinkt en je eventueel doorverwijzen
naar een cyberspecialist. Heb je een operatie nodig, dan verricht een
cyberschirurg in Bermuda de ingreep op afstand met behulp van
gespecialiseerde apparatuur die microsneden maakt.

\subsection{Leven-en-dood
informatieverwerking}\label{leven-en-dood-informatieverwerking}

Dit klinkt wellicht als sciencefiction, maar veel onderdelen van de
cyberchirurgie zijn al beschikbaar. Tegen de tijd dat je dit boek leest,
werken de andere toepassingen waarschijnlijk ook.

General Electric heeft in vijftien ziekenhuizen wereldwijd een nieuwe
machine voor magnetische resonantietherapie (MRT) geïntroduceerd. Men
verwacht dat de machine een driejarig traject op het gebied van
onderzoek en ontwikkeling doorloopt, waarna hij waarschijnlijk snel
wijdverspreid wordt en de standaard vormt voor tal van operaties. Dit is
een uitstekend voorbeeld van hoe technologie onze samenleving
transformeert.

De meesten van ons kennen MRI‑machines, waarmee artsen via magnetische
resonantietechnieken beelden van zacht weefsel verkrijgen voor
diagnostische doeleinden. Deze machines leveren betere beelden dan
röntgenfoto's of echografie en zijn onmisbaar geworden in de moderne
diagnostiek, vooral bij kanker. Toch kampen ze momenteel met twee
belangrijke beperkingen: de buis biedt geen directe toegang tot de
patiënt en bovendien ontbreekt het hen aan kracht.

\subsection{Cyberchirurgie}\label{cyberchirurgie}

\emph{General Electric} heeft de magnetische resonantiemachines zo
aangepast dat ze zowel voor diagnostiek als voor behandelingen ingezet
kunnen worden. Ze hebben de kracht van de machines met een factor vijf
verhoogd en de buis in twee helften verdeeld, waardoor de patiënt niet
langer volledig wordt omsloten, maar tussen twee donutvormige
compartimenten komt te liggen. In plaats van eerst een beeld vast te
leggen waarop later de operatie wordt gebaseerd, ziet de chirurg direct
wat hij doet tijdens de ingreep. Het systeem koppelt de MRT aan
microchirurgische technieken die minder invasief zijn. De chirurg hoeft
geen grote sneden met een scalpel te maken, maar zet kleine incisies met
sonderingsinstrumenten, waarbij hij in real‑time observeert wat deze
onthullen. Hij voert de operatie uit op basis van het beeld, in plaats
van er direct in te kijken. Bovendien kan hij de sonderingsinstrumenten
in principe op afstand bedienen. Zo kunnen tumoren met uiterste precisie
worden vernietigd, bijvoorbeeld met behulp van laserapparatuur of via
cryogene warmte‑ en vriesbehandelingen.

Hiermee worden operaties mogelijk die tot nu toe onmogelijk leken,
vooral in de neurochirurgie, waar tumoren zich vaak zeer dicht bij
vitale hersengebieden bevinden. Ook maakt deze techniek herhaalde
operaties haalbaar, wanneer het trauma van de traditionele ingreep niet
herhaald kan worden zonder onaanvaardbare schade. Sommige onderzoekers
zijn van mening dat het mes voor zachte weefselchirurgie tegen 2010 tot
een verouderd relikwie zal behoren. Als dat standhoudt, verminderen
zowel de angst als de naschokken die bij traditionele operaties horen.
Uiteraard is dit uitstekend nieuws voor de patiënt. Terwijl operaties
tegenwoordig uren duren en gevolgd worden door dagen of weken
ziekenhuisopname, kan de ingreep in slechts een half uur worden voltooid
en is een opname mogelijk overbodig. In feite kunnen de chirurg en de
patiënt zelfs nooit in dezelfde ruimte aanwezig zijn. Maar wat betekent
dit voor ziekenhuizen en chirurgen?

\subsection{Minder microsurgen die meer operaties
uitvoeren}\label{minder-microsurgen-die-meer-operaties-uitvoeren}

Er staat een revolutie in de chirurgie op het punt van uitbreken.
Tijdens hun opleiding lukt een derde van de jonge chirurgen niet om de
vaardigheden voor microscopische chirurgie eigen te maken, een derde
beheerst de techniek enigszins en één derde blinkt uit. Vergelijkbare
verhoudingen zien we ook bij omscholingscursussen voor ervaren
chirurgen. Hierdoor zullen minder chirurgen in staat zijn om een groter
aantal operaties in een kortere tijd uit te voeren. Zorgverzekeraars en
patiënten zullen ongetwijfeld aandringen op prestatiestatistieken per
chirurg, die flink kunnen verschillen. Patiënten kiezen bij
levensbedreigende aandoeningen vooral voor de chirurgen met de beste
resultaten. In sommige gevallen opereren de beste chirurgen zelfs op
afstand; zij voeren de gehele ingreep uit vanuit een ander rechtsgebied
waar de belastingen lager zijn en waar rechtbanken buitensporige claims
wegens medische fouten niet honoreren.

\subsection{Digitale juristen}\label{digitale-juristen}

Voordat een ervaren chirurg akkoord gaat met een operatie, schakelt hij
of zij waarschijnlijk een digitale jurist in om onmiddellijk een
contract op te stellen. Dit contract specificeert en beperkt de
aansprakelijkheid op basis van de grootte en kenmerken van de tumor,
zoals die zichtbaar zijn in de beelden van de magnetische
resonantiemachine.

Digitale juristen werken als systemen voor informatieopvraging die
automatisch de juiste contractbepalingen selecteren. Ze maken daarbij
gebruik van kunstmatige intelligentietechnologieën, zoals neurale
netwerken, om particuliere contracten op maat te maken die voldoen aan
transnationale juridische eisen.

Deelnemers aan belangrijke en waardevolle transacties zoeken niet alleen
geschikte zakenpartners, maar kiezen ook een passend vestigingsadres
voor hun transacties.

\subsection{Spoedconsultatie}\label{spoedconsultatie}

Om het voorbeeld van cybersurgery voort te zetten, zet de technologie
van het informatietijdperk de topchirurgen extra in de schijnwerpers,
zoals in vrijwel elk ander vakgebied. Al zolang er messen bestaan, zijn
patiënten bereid een meerprijs te betalen. Beperkingen in informatie en
de moeite om in een noodsituatie een lokale chirurg te vinden, maken de
chirurgiemarkt onvolmaakt. In het informatietijdperk wordt die markt
aanzienlijk efficiënter.

Een patiënt die binnen 24 uur -- of zelfs binnen 45 minuten -- een
operatie nodig heeft, kan digitale assistenten inschakelen om wereldwijd
de tien beste chirurgen te vinden die op afstand opereren. Deze
assistenten beoordelen hun slagingspercentages in vergelijkbare gevallen
en vragen via geschikte digitale vertegenwoordigers offertes op voor het
specifieke geval. Dit alles kan binnen enkele ogenblikken geregeld
worden.

Hierdoor verovert de top tien procent op de wereldmarkt voor chirurgie
een aanzienlijk groter marktaandeel. De magnetische resonantiemachine,
gecombineerd met microchirurgische technieken, drijft bovendien de
meerprijs voor hun diensten verder op. Chirurgen met minder bekwaamheid
richten zich op de overgebleven lokale markten.

Dit leven-en-doodvoorbeeld illustreert enkele revolutionaire gevolgen
van het loslaten van de beperkingen van plaatsgebondenheid in de
economie. Sommigen zouden kunnen aanvoeren dat de magnetische
resonantiemachine van \emph{General Electric} niet bedoeld is voor op
afstand gebruik. Misschien wel, maar dat raakt de kern van de zaak niet.
Het toestel -- of een vergelijkbaar apparaat -- zal die mogelijkheid
spoedig bieden. Wanneer chirurgen operaties uitvoeren via beeldschermen
in plaats van de patiënt direct te zien, wordt de locatie van de chirurg
en zijn scherm steeds minder relevant.

Steeds meer diensten richten zich op het feit dat informatietechnologie
mensen wereldwijd verbindt, zelfs op een zo delicate plek als in de
chirurgie. In activiteiten waarvoor minder precieze apparatuur nodig is
en de kans op mislukking kleiner is, zal de cybereconomie nog sneller
floreren.

`Het financiële beleid van de verzorgingsstaat zorgt ervoor dat
vermogende burgers geen enkele mogelijkheid hebben om zichzelf te
beschermen.' - Alan Greenspan

\section{Devaluatie van dwang}\label{devaluatie-van-dwang}

In vrijwel elk concurrentieveld -- inclusief het merendeel van 's
werelds investeringsactiviteiten ter waarde van meerdere triljoenen
dollars -- drijft een bijna onstuitbare kracht de verschuiving van
transacties naar de cyberruimte. Die kracht vloeit voort uit de drang om
roofzuchtige belastingen te ontlopen, zoals de last van inflatie voor
iedereen die zijn vermogen in een nationale munteenheid aanhoudt.

\subsection{Ontsnappen aan het
beschermingsracket}\label{ontsnappen-aan-het-beschermingsracket}

Je hoeft niet lang stil te staan bij de megapolitiek van dit
informatietijdperk om te beseffen dat de roofzuchtige belastingen en de
inflatie, die de machtigste geïndustrialiseerde landen routinematig aan
hun burgers opleggen, op de nieuwe grens van de cyberruimte volstrekt
onconcurrerend zullen blijken. Kort na de millenniumwisseling zullen
mensen die momenteel inkomstenbelasting betalen er uiteindelijk voor
kiezen om 50 procent af te dragen. Zoals Frederic C. Lane opmerkte,
leert de geschiedenis ons dat `aan de grenzen en op de hoge zeeën, waar
niemand een blijvend monopolie had op het gebruik van geweld, handelaren
de betaling van heffingen vermeden die zo hoog waren dat bescherming op
andere, goedkopere wijze verkregen kon worden.' \footnote{Lane,
  `Economische gevolgen van georganiseerd geweld', op. cit., p.~404.}

De cybereconomie biedt precies zo'n alternatief. Geen enkele overheid
slaagt erin dit te monopoliseren. Bovendien bieden de bijbehorende
informatietechnologieën een veel goedkopere en effectievere bescherming
van financiële activa dan de meeste overheden ooit hebben gekund.

\subsection{De zwarte magie van samengestelde
rente}\label{de-zwarte-magie-van-samengestelde-rente}

Onthoud: als je elk jaar \$5.000 betaalt gedurende veertig jaar, slijt
dat je nettovermogen met \$2,2 miljoen -- uitgaande van een rendement
van slechts 10 procent op je kapitaal. Bij een rendement van 20 procent
loopt het samengestelde verlies op tot ongeveer \$44 miljoen. Voor
hoogverdieners in landen met hoge belastingen zijn de totale verliezen
door roofzuchtige belastingheffing over een heel leven ronduit
adembenemend. De meesten raken uiteindelijk met een verlies geëindigd
dat groter is dan alles wat ze ooit bezaten.

Dit klinkt misschien ongelooflijk, maar de wiskunde staat als een paal
boven water. Je kunt dit zelf gemakkelijk nagaan met een simpele
rekenmachine. De top 1 procent van de belastingbetalers in de Verenigde
Staten betaalt gemiddeld meer dan \$125.000 aan federale
inkomstenbelasting per jaar. Voor een fractie daarvan -- namelijk
\$45.000 per jaar -- kom je in aanmerking voor een particuliere
belastingregeling in Zwitserland, waar je profiteert van de orde en
veiligheid die men beschouwt als het meest complete politie- en
rechtssysteem ter wereld. Vanuit dit perspectief kun je de extra
\$80.000 aan jaarlijkse inkomstenbelasting boven dat royale niveau als
een ware tribuut of zelfs als plundering bestempelen. \$45.000 vormt
immers een aanzienlijke bijdrage aan het handhaven van orde en
veiligheid, aangezien politiebescherming als een publiek goed geldt. In
theorie leveren publieke goederen extra gebruikers op zonder bijkomende
marginale kosten. De Zwitsers zijn dan ook tevreden dat je een
overeengekomen vaste belasting van \$45.000 (oftewel 50.000 Zwitserse
frank) per jaar betaalt, want per aangemelde miljonair boeken zij
daarmee jaarlijks \$45.000.

Als je de Zwitserse regeling vergelijkt, lijdt een belegger die
gemiddeld 20 procent rendement behaalt en federale inkomstenbelasting
betaalt volgens Amerikaanse tarieven, over een hele levensloop een
verlies van ongeveer \$705 miljoen. Houd er wel rekening mee dat dit
uitgaat van een jaarlijkse belasting van \$45.000. Vergelijk dat eens
met een belastingparadijs als Bermuda, waar vrijwel geen
inkomstenbelasting geldt. Daar zou het levenslange verlies bij
Amerikaanse tarieven oplopen tot ongeveer \$1,1 miljard.

Je zou kunnen stellen dat een jaarlijks rendement van 20 procent
buitengewoon hoog is -- daar heb je zeker een punt. Maar dankzij de
indrukwekkende groei in Azië gedurende de afgelopen decennia hebben veel
beleggers wereldwijd dat rendement -- of zelfs meer -- weten te behalen.
Sinds 1950 ligt het samengestelde rendement op vastgoedbeleggingen in
Hongkong op meer dan 20 procent per jaar. Zelfs in economieën die niet
bekendstaan om hun sterke groei, boden zich vaak makkelijke kansen op
hoge winsten. In de afgelopen drie decennia had je met deposito's in
Amerikaanse dollars bij Paraguayaanse banken een reëel gemiddeld
rendement van meer dan 30 procent per jaar kunnen boeken. Op sommige
plekken kun je eenvoudig hoge investeringsopbrengsten realiseren, maar
ook ervaren beleggers kunnen in goede jaren winsten van 20 procent of
meer behalen, ook al evenaren zij niet altijd de prestaties van George
Soros of Warren Buffett.

Het ligt voor de hand: hoe hoger het rendement op je kapitaal, hoe
groter de opportuniteitskosten die ontstaan door buitensporige
inkomsten- en vermogenswinstbelastingen. Toch komt de constatering dat
het verlies enorm is -- soms zelfs groter dan de totale rijkdom die je
ooit had kunnen opbouwen -- niet door het behalen van uitzonderlijk hoge
rendementen. Sommige Amerikaanse beleggingsfondsen boeken al langer dan
een halve eeuw een gemiddeld jaarlijks rendement van meer dan 10
procent. Als dit het beste resultaat is dat je kunt behalen en je
behoort tot de top 1 procent van de Amerikaanse verdieners, daalt je
nettovermogen met meer dan \$33 miljoen, enkel door de
inkomstenbelasting die je betaalt over je inkomen boven de \$45.000 per
jaar. Vergeleken met een jurisdictie zonder inkomstenbelasting bedraagt
dat verlies zelfs \$55 miljoen.

\$55 in plaats van \$55 miljoen

Als de winstmaximaliserende aannames van economen kloppen -- wat wij in
het algemeen veronderstellen -- zullen de meeste mensen proberen \$55
miljoen te redden als dat mogelijk was. Dat is onze voorspelling.

Wanneer de zwarte magie van samengestelde rente meer doordringt in de
hoofden van succesvolle mensen in landen met hoge belastingen, gaan zij
serieus op zoek naar gunstigere rechtsgebieden -- net zoals zij
tegenwoordig auto's kopen of verzekeringspremies vergelijken.

Als je nog twijfelt, spreek dan gewoon wat mensen op straat in New York
of Toronto aan en vraag of zij voor \$55 miljoen bereid zijn naar
Bermuda te verhuizen. Het antwoord spreekt voor zich.

Dit dilemma doet me denken aan de keuze waar Mark Twain ooit mee
worstelde: of hij de nacht doorbracht met Lillian Russell, helemaal
naakt, of met generaal Grant in vol ornaat uniform. Hij aarzelde geen
moment.

Bewoners van gevestigde verzorgingsstaten -- vooral in de Verenigde
Staten -- reageren misschien wat langzamer, simpelweg omdat ze nog niet
doorhebben welke keuze op hen wacht. Maar na verloop van tijd zal dat
veranderen.

Jij -- of wie er ook naar een beter leven streeft -- zult al gauw merken
hoe verleidelijk het is om de verliezen door roofzuchtige
belastingheffing te beperken.

Je hoeft alleen maar je transacties in de digitale wereld vast te
leggen. Natuurlijk is dat in veel rechtsgebieden illegaal, maar oude
wetten houden nauwelijks stand tegen nieuwe technologie.

In de jaren tachtig verbood de Verenigde Staten het versturen van faxen.
Het Amerikaanse postkantoor beschouwde faxen als eersteklas post -- een
terrein waarop het al eeuwenlang een monopolie uitoefende. De
autoriteiten gaven dan ook een bevelschrift uit waarin zij eisten dat
alle faxtransmissies via het dichtstbijzijnde postkantoor moesten
verlopen, zodat ze als gewone post werden afgeleverd.

Na miljarden verzonden faxen is het inmiddels onduidelijk of iemand ooit
echt aan die regel heeft voldaan -- en als dat wel het geval was, dan
duurde die naleving nauwelijks lang. De voordelen van opereren in de
opkomende cybereconomie wegen immers ruimschoots zwaarder dan het
incidenteel omzeilen van het postkantoor bij het versturen van een fax.

De brede toepassing van public-key en private-key encryptietechnologieën
maakt binnenkort talloze economische activiteiten op elke gewenste
locatie mogelijk. Zoals James Bennet, technologie-redacteur bij
\emph{Strategic Investment}, opmerkte:

\begin{quote}
De handhaving van wetten en met name belastingwetgeving is sterk
afhankelijk geworden van het toezicht op communicatie en transacties.
Zodra de volgende logische stappen zijn gezet en offshorebanken de
dienst aanbieden van communicatie via hard RSA-versleutelde e-mail met
rekeningnummers afgeleid van public-key-systemen, zullen financiële
transacties bijna onmogelijk te monitoren zijn, zowel bij de bank als in
de communicatie. Zelfs als de belastingautoriteiten een mol in de
offshorebank zouden plaatsen, of in de bankadministratie zouden
inbreken, zouden zij niet in staat zijn de rekeninghouders te
identificeren.\footnote{James Bennet, `De informatierevolutie en de
  ondergang van de inkomstenbelasting', \emph{Strategic Investment},
  november 1994, pp.~11--12.}
\end{quote}

Op een ongekend niveau kunnen mensen binnenkort zelf bepalen waar zij
hun economische activiteiten vestigen en hoeveel inkomstenbelasting zij
willen betalen. In het informatietijdperk hoeven veel transacties
helemaal niet binnen de grenzen van een territoriale soevereiniteit
plaats te vinden. Degenen die dat wel moeten, zullen steeds vaker
terechtkomen in regio's als Bermuda, de Kaaimaneilanden, Uruguay of in
vergelijkbare rechtsgebieden waar geen inkomstenbelasting en andere
kostbare transactiekosten op de handel worden geheven.

\subsection{Van monopolie naar
concurrentie}\label{van-monopolie-naar-concurrentie}

Overheden hebben er inmiddels aan gewend geraakt om zogenoemde
`beschermingsdiensten' op te leggen die -- om in de woorden van Frederic
C. Lane te zeggen -- `van slechte kwaliteit en belachelijk overprijsd'
zijn.\footnote{Lane, `Economische gevolgen van georganiseerd geweld',
  op. cit., p.~404.} Deze praktijk, waarbij men veel meer rekent dan de
daadwerkelijke waarde van overheidsdiensten, is ontstaan tijdens
eeuwenlange monopolievorming. Overheden hanteerden meedogenloos
belastingen voor iedereen die schijnbaar in staat was te betalen, omdat
zij feitelijk een (bijna-)monopolie op dwang bezaten. Deze
monopolietraditie zal op ingrijpende wijze botsen met de nieuwe
megapolitieke kansen binnen de cyberhandel.

Encryptie maakt het nu eenvoudig om transacties in de cybersfeer te
beveiligen. De kosten van een effectief encryptiesoftwareprogramma,
zoals \emph{PGP}, liggen namelijk lager dan de commissie die een
full-service makelaar rekent bij een transactie van honderd aandelen.
Toch zorgt deze technologie ervoor dat vrijwel elke transactie jarenlang
onzichtbaar en onaantastbaar blijft voor zowel overheden als dieven. De
informatietechnologie van vandaag beveiligt cyberactiva tegen
verwaarloosbaar lage kosten. Voor slechts \$55 -- in plaats van \$55
miljoen -- genieten deelnemers aan de cybereconomie van een betere
bescherming van hun activa dan ooit tijdens het industriële tijdperk of
enig ander moment in de geschiedenis. Gebruiksvriendelijke
encryptie-algoritmen en de mogelijkheid om transacties via verschillende
vestigingslocaties wereldwijd te verrichten, bieden een effectieve
afscherming tegen de grootste bedreiging: de natiestaten zelf.

Dit betekent echter niet dat territoriale overheden volledig te slim af
zijn. Zij kunnen nog steeds kwetsbaarheden benutten om hoge belastingen
op te leggen of zelfs rijke individuen voor losgeld te houden. Ook
blijven zij in staat verbruiksbelastingen af te dwingen. Maar de
bescherming -- de voornaamste dienst die overheden leveren -- komt
binnenkort bijna op een concurrerende markt beschikbaar. Hierdoor gaat
er minder van de kosten die productieve burgers voor bescherming
betalen, beschikbaar komen om door politieke autoriteiten te worden
opgeëist en herverdeeld. Technologische innovaties zorgen ervoor dat een
groot en groeiend deel van de wereldwijde rijkdom buiten het bereik van
overheden valt. Dit vermindert de handelsrisico's en verlaagt, om in de
woorden van historicus Janet Abu-Lughod te zeggen, `de verhouding van
alle kosten' die anders naar transittarieven, tribuut of eenvoudige
afpersing zouden gaan.\footnote{Abu-Lughod, op. cit., p.~177.}

In de loop van de geschiedenis komen we zelden overheden tegen die
werkelijk door concurrentie worden beperkt. In de weinige gevallen
waarin dat min of meer aan de orde was, waren overheden zwak en waren de
technologische mogelijkheden tussen jurisdicties vrijwel identiek.
Volgens Lane blijkt dat de belangrijkste factor voor winstgevendheid in
zulke situaties de verschillen in beschermingskosten zijn die
ondernemers dragen. De middeleeuwse koopman die twintig tolgelden
betaalde om zijn handelswaar naar de markt te brengen, kon niet op tegen
een collega die slechts vier tolgelden rees om dezelfde goederen bij de
klant af te leveren. Vergelijkbare omstandigheden zullen onvermijdelijk
weer opduiken in het informatietijdperk. Ook in de toekomst zal winst
niet zozeer bepaald worden door technologische voorsprong, maar door het
vermogen om de kosten voor bescherming te minimaliseren.

Deze nieuwe economische dynamiek druist in tegen de wens van een
overheid -- een overblijfsel uit het industriële tijdperk -- om voor
haar beschermingsdiensten monopolistische tarieven vast te stellen. Of
we het nu willen of niet, het oude systeem zal in de concurrerende
omgeving van het informatietijdperk niet standhouden. Elke overheid die
koppig volhardt haar burgers te belasten met zware heffingen, die haar
concurrenten niet hoeven te betalen, zorgt er immers alleen maar voor
dat winst en welvaart zich naar andere oorden verplaatsen. Daarom zullen
verzorgingsstaten die op de lange termijn niet slagen in het verlagen
van de belastingdruk, uiteindelijk zichzelf tot correctie dwingen.
Overheden die te veel heffen, maken het wonen in hun grondgebied
eenvoudigweg tot een financiële last die zelfs tot faillissement kan
leiden.

\begin{quote}
`\ldots{} want zoals de koning, op basis van zijn prerogatief en naar
eigen goeddunken, geld kan slaan van welk materiaal en in welke vorm hij
ook maar wil en daarvoor de norm kan vaststellen, zo kan hij ook het
geld in substantie en afdruk wijzigen, de waarde ervan verhogen of
verlagen, of het geheel afwijzen en annuleren'\footnote{Geciteerd in
  Henry Mark Holzer, `Het geldmonopolie van de overheden' (New York:
  Books in Focus, 1981), p.~4.}: - Uit een Engels rechtsbesluit, 1604
\end{quote}

\section{De dood van seigniorage}\label{de-dood-van-seigniorage}

Overheden verliezen niet alleen hun bevoegdheid om talrijke bronnen van
inkomsten en kapitaal te belasten, ze zijn ook voorbestemd hun
dwangbevoegdheid over het geldsysteem te kwijtraken. In het verleden
gingen grootschalige politieke transities gepaard met ingrijpende
veranderingen in de aard van geld.

\begin{itemize}
\item
  De invoering van munten luidde een vijfhonderdjarige expansiecyclus in
  de oude economie in, die culmineerde met de geboorte van Christus en
  leidde tot de laagste rentetarieven van vóór de moderne tijd.
\item
  De intrede van de donkere eeuwen viel samen met de bijna volledige
  sluiting van de munthallen. Hoewel het Romeinse geld bleef circuleren,
  nam zowel de geldvoorraad als de handel af, waardoor een vicieuze
  neerwaartse spiraal ontstond.
\item
  De feodale revolutie ging gepaard met de herinvoering van geld,
  munten, wisselbrieven en andere instrumenten om commerciële
  transacties af te handelen. Vooral de toegenomen zilverwinning in
  nieuwe mijnen in Rammelsberg, Duitsland, leidde tot een ruimere
  circulatie van munten, wat de handel aanzienlijk versoepelde.
\item
  De grootste revolutie op het gebied van geld vóór het
  informatietijdperk viel samen met de opkomst van het industrialisme.
  Tijdens de buskruitrevolutie verstevigde de vroegmoderne staat haar
  macht. Naarmate de controle groeide, nam de staat het beheer over geld
  in eigen hand en maakte zij volop gebruik van de drukpers -- het
  eerste massaproductie-instrument -- om papieren geld in grote
  hoeveelheden te produceren.
\end{itemize}

Papiergeld vormt een duidelijk industrieel product. Vóór de
boekdrukkunst was het praktisch onmogelijk om kwitanties of certificaten
te dupliceren, die later tot papiergeld werden herleid. Monniken in de
scriptoria zouden immers niet hun tijd hebben verspild aan het opmaken
van vijftigpondbiljetten. Bovendien versterkte papiergeld de macht van
de staat, niet alleen doordat overheden winst haalden uit de
waardevermindering van de valuta, maar ook doordat het hen een hefboom
bood om te bepalen wie vermogen kon vergaren. Zoals Abu-Lughod het
verwoordde: `wanneer door de staat gesteund papiergeld de goedgekeurde
valuta wordt, worden de kansen om kapitaal te vergaren in verzet tegen
of onafhankelijk van de staatsmachine ernstig beperkt.' \footnote{Abu-Lughod,
  op. cit., p.~332.}

\section{Cybercash}\label{cybercash}

De opkomst van het informatietijdperk ontketent een nieuwe revolutie in
de aard van geld. Nu de handel in cyberspace in een stroomversnelling
raakt, ontstaat onvermijdelijk een digitale valuta: cybergeld. Deze
nieuwe munt herdefinieert machtsverhoudingen en beperkt de mogelijkheden
van natiestaten om te bepalen wie als soeverein individu wordt
aangemerkt. Wat hierbij doorslaggevend is, is dat informatietechnologie
de bezitters van rijkdom beschermt tegen de onteigenende werking van
inflatie. Binnenkort betaal je vrijwel direct met cybercash voor elke
transactie die je via het `World Wide Web' uitvoert.

Deze digitale valuta staat op het punt een cruciale rol te gaan spelen
in de cyberhandel. Hij bestaat uit complexe, versleutelde reeksen van
priemgetallen met honderden cijfers. Dankzij zijn unieke, anonieme en
verifieerbare eigenschappen maakt hij ook megatransacties mogelijk.
Bovendien kun je hem tot op de kleinste fractie opdelen. Met slechts één
toetsaanslag handel je op een grensoverschrijdende groothandelsmarkt,
ter waarde van meerdere triljoenen dollars.

\subsection{Handelen zonder dollars}\label{handelen-zonder-dollars}

Het nieuwe cybergeld zal zich onvermijdelijk onttrekken aan nationale
beperkingen. Zodra mensen grensoverschrijdend handelen in een virtuele
wereld, weigeren ze de ouderwetse praktijk waarmee regeringen de waarde
van hun geld kunstmatig doen dalen door inflatie. Waarom zouden ze dat
nog accepteren? De controle over geld verschuift van de machtscentra
naar de wereldwijde marktplaats. Iedereen met toegang tot cyberspace --
of je nu een individu bent of een onderneming -- kan eenvoudig
overstappen naar elke valuta waarvan de waarde dreigt te kelderen. In
tegenstelling tot nu hoef je transacties niet langer in wettig
betaalmiddel te verrichten. Sterker nog, wanneer transacties wereldwijd
plaatsvinden, rekent ten minste één partij in elke transactie in een
valuta die voor die partij niet als wettig betaalmiddel geldt.

\subsection{Verminderde nadelen van
ruilhandel}\label{verminderde-nadelen-van-ruilhandel}

In de cybereconomie kun je in elk gewenst medium handelen. Zoals de
inmiddels overleden Nobelprijswinnende econoom \emph{E. A. Hayek}
betoogde, bestaat er `geen duidelijk onderscheid tussen geld en
niet-geld.' Hij schreef: ``Hoewel we er doorgaans van uitgaan dat er een
scherpe scheidslijn is tussen wat geld is en wat niet, en de wet vaak
probeert dit onderscheid te handhaven, bestaat er -- wat betreft de
causale effecten van monetaire gebeurtenissen -- geen dergelijk
duidelijk verschil. In werkelijkheid vinden we eerder een continuüm
waarin objecten met uiteenlopende niveaus van liquiditeit, of met
waarden die onafhankelijk van elkaar kunnen schommelen, geleidelijk in
elkaar overvloeien naarmate ze als geld functioneren.'' \footnote{Friedrich
  A. von Hayek, de denationalisatie van geld (London: \emph{Institute of
  Economic Affairs}, 1976), p.~47.}: Digitale valuta op wereldwijde
computernetwerken maken elk object op Hayeks continuüm van liquiditeit
extra liquide -- met uitzondering van overheidspapier.

Een direct gevolg hiervan is dat ruilhandel praktischer wordt. Steeds
meer goederen en diensten komen via specifieke ruilaanbiedingen
beschikbaar in ruil voor andere goederen en diensten. Deze potentiële
transacties verschijnen wereldwijd op internet, waardoor hun liquiditeit
aanzienlijk toeneemt.

Een van de grootste nadelen van ruilhandel is dat het lastig is precies
de juiste match te vinden: iemand met een specifieke behoefte koppelen
aan iemand die exact kan leveren wat daar nodig is. De traditionele
ruilhandel moest vaak de ontmoedigende uitdaging overwinnen om twee
partijen in een lokale markt perfect op elkaar af te stemmen. Contant
geld heeft de beperkingen van ruilhandel ver achter zich gelaten en
biedt bij de meeste transacties duidelijke voordelen. Dankzij de enorme
toename in rekenkracht en de globalisering van de handel in cyberspace
worden tegelijkertijd de nadelen van ruilhandel fors verminderd. Wanneer
je wereldwijd kunt zoeken, neemt de kans dramatisch toe dat je iemand
vindt met precies de wederzijdse wensen die jij hebt, in plaats van je
te beperken tot je lokale omgeving.

\subsection{Niet vatbaar voor
vervalsing}\label{niet-vatbaar-voor-vervalsing}

Papiergeld blijft ongetwijfeld in omloop als restvorm van ruilmiddel
voor de armen en voor mensen die minder vertrouwd zijn met digitale
technologie, maar geld voor transacties met hoge waarde zal
geprivatiseerd worden. Cybergeld wordt niet langer uitsluitend in
nationale valuta genoteerd, zoals het papiergeld uit het industriële
tijdperk. Vermoedelijk wordt het uitgedrukt in termen van gram of ounce
goud -- zo fijn deelbaar als dat metaal zelf -- of in termen van andere
tastbare opslagmiddelen van waarde. Zelfs als er verschillende
prijsmaatstaven worden gehanteerd of als sommige transacties nog steeds
in nationale valuta worden genoteerd, zal cybergeld de consument
aanzienlijk beter bedienen dan het ooit door de staat beheerde geld
deed. De snel groeiende rekenkracht dringt de moeilijkheden bij het
afstemmen van prijzen op diverse ruilmiddelen terug tot een
verwaarloosbaar niveau. Bij iedere transactie worden immers versleutelde
reeksen priemgetallen met honderden cijfers overgedragen. In
tegenstelling tot de papiergeldbewijzen die regeringen tijdens het
tijdperk van de goudstandaard uitgaven en die naar believen gekopieerd
konden worden, wordt de nieuwe digitale goudstandaard -- of de
ruilequivalenten daarvan -- vrijwel onmogelijk te vervalsen, simpelweg
omdat het product van priemgetallen met honderden cijfers praktisch
onontcijferbaar is. Elk ontvangstbewijs is daarbij verifieerbaar uniek.

De benamingen van traditionele valuta -- zoals `pond' en `peso' -- maken
duidelijk dat zij oorspronkelijk als maatstaven voor specifieke
hoeveelheden edele metalen ontstaan zijn. Het pond sterling bestond ooit
als sterlingzilver van een pond. Papiergeld in het Westen ontstond als
ontvangstbewijzen die aantoonden dat edele metalen in magazijnen of
kluizen werden bewaard. Regeringen die deze bewijzen uitgaven, ontdekten
al snel dat ze veel meer konden drukken dan zij daadwerkelijk met hun
ruwe goudreserves konden inwisselen. Ieder certificaathouder kon aan
zijn bewijs niets afleiden over de werkelijke voorraad edele metalen,
want afgezien van de serienummers zagen alle bewijzen er identiek uit.
Dit kwam zowel vervalsers, politici als bankiers in de smaak, omdat zij
zo konden profiteren van het kunstmatig opblazen van de geldvoorraad.

Op die manier wordt cybergeld -- zowel officieel als informeel --
vrijwel ontegensprekelijk vervalsingsbestendig. De controleerbaarheid
van digitale ontvangstbewijzen voorkomt namelijk de klassieke truc om
rijkdom via inflatie onteigend over te dragen. Het nieuwe digitale geld
van het informatietijdperk geeft de controle over het ruilmiddel terug
aan de eigenaren van rijkdom, die hun vermogen willen behouden in plaats
van het aan natiestaten te verliezen.

\subsection{De transactiekosten van `vrij'
geld}\label{de-transactiekosten-van-vrij-geld}

Dit nieuwe cybergeld bevrijdt u aanzienlijk van de macht van de staat.
Eerder wezen we op het sombere trackrecord van natiestaten wereldwijd
bij het in stand houden van de waarde van hun valuta in de afgelopen
vijftig jaar. Sinds de Tweede Wereldoorlog heeft geen enkele valuta een
kleinere waardevermindering door inflatie doorgemaakt dan de Duitse
mark. Toch verloor de mark tussen 1 januari 1949 en eind juni 1995 maar
liefst 71 procent van haar waarde. De wereldreserverende valuta in die
periode, de Amerikaanse dollar, daalde zelfs met 84 procent in
waarde.\footnote{Zie hoofdstuk 1, noot 6.}

Merk op dat er geen inherente reden bestaat dat valuta in waarde dalen
of dat de nominale kosten van levensonderhoud jaarlijks stijgen.
Integendeel: het behoud van de koopkracht van spaargeld vormt een
technisch eenvoudige opgave. Dit blijkt al als je naar de
langetermijnwaarde van goud kijkt.

Tussen 1 januari 1949 en eind juni 1995 verloren de beste nationale
valuta bijna driekwart van hun waarde, terwijl de koopkracht van goud
daadwerkelijk toenam. Volgens professor Roy W. Jastrom in zijn boek
\emph{The Golden Constant} behield goud, met slechts minimale
schommelingen, zijn koopkracht, voor zover betrouwbare prijsgegevens
beschikbaar zijn -- in het geval van Engeland gaan de gegevens zelfs tot
1560.

Ook behielden nationale valuta die aan goud gekoppeld waren hun
koopkracht in perioden waarin militaire urgenties niet doorslaggevend
waren. Zo steeg de waarde van het Britse pond sterling in de relatief
vreedzame negentiende eeuw in plaats van te dalen, ook al was de
koppeling met goud slechts beperkt.

De nieuwe geopolitieke omstandigheden van het informatietijdperk maken
een sterke schakel mogelijk -- een schakel die voor het eerst wordt
versterkt door de aanzienlijk verbeterde informatie- en
rekencapaciteiten die consumenten nu tot hun beschikking hebben.

\begin{quote}
`De dreiging van het snelle verlies van hun volledige onderneming als
zij er niet in slaagden de verwachtingen waar te maken (en hoe elke
overheidsorganisatie ongetwijfeld misbruik zou maken van de kans om met
grondstofprijzen te spelen!) zou een veel sterkere waarborg bieden dan
wat er bedacht zou kunnen worden ter bestrijding van een
staatsmonopolie'\footnote{Hayek, op. cit., p.~40.}: - FRIEDRICH A. VON
HAYEK
\end{quote}

\subsection{Privatisering van geld}\label{privatisering-van-geld}

In 1976 betoogde Friedrich von Hayek dat het gebruik van concurrerende,
particuliere valuta inflatie zou uitroeien. Omdat er binnen een
rechtsgebied geen wettelijke verplichtingen bestaan die de acceptatie
van een inflatoire valuta afdwingen, dwingt volgens Hayek de vrije
marktdruk particuliere valuta-uitgevers om de waarde van hun
ruilmiddelen te behouden. Elke uitgever die daarin faalt, verliest al
snel zijn klanten. De opkomst van versleutelde digitale contanten brengt
Hayeks logica op indringende wijze weer tot leven.

De theorie van `vrij bankieren' is meer dan slechts een hypothetische
academische gedachte. In Schotland circuleerden vanaf het begin van de
achttiende eeuw tot 1844 particuliere valuta, omdat het land destijds
geen centrale bank kende en bijna geen regels of toetredingsbeperkingen
tot het bankwezen hanteerde. Particuliere banken namen deposito's in en
gaven hun eigen valuta uit, ondersteund door goudstaven. Zoals professor
Lawrence White heeft gedocumenteerd, werkte dit systeem uitstekend: het
bleek stabieler te zijn en kende minder inflatie dan het strenger
gereguleerde en politisierde bank- en geldsysteem dat in Engeland in
dezelfde periode gold. Michael Prowse van de \emph{Financial Times} vat
de ervaring met vrij bankieren in Schotland als volgt samen: `Er was
weinig fraude. Er waren geen aanwijzingen voor overmatige uitgifte van
biljetten. Banken hielden doorgaans noch te veel, noch te weinig
reserves aan. Bankruns waren zeldzaam en niet besmettelijk. De vrije
banken genoten het respect van de burgers en boden een solide basis voor
economische groei die in de meeste perioden Engeland ver overtrof.'

Wat in de achttiende en negentiende eeuw goed functioneerde, zal met de
technologie van de eenentwintigste eeuw nog beter werken. Binnenkort
kunt u handelen in digitale valuta uitgegeven door particuliere
bedrijven, op dezelfde wijze als \emph{American Express} reischeques
verstaat als ontvangstbewijzen voor contant geld. Een instelling met een
reputatie die geen overheid kan evenaren -- bijvoorbeeld een
toonaangevend mijnbedrijf of de \emph{Swiss Bank Corporation} -- kan
versleutelde ontvangstbewijzen creëren voor hoeveelheden goud of zelfs
voor unieke goudbaren, waarbij moleculaire handtekeningen als
identificatie dienen en hologrammen er optioneel aan worden toegevoegd.
Handelaren verhandelen deze ontvangstbewijzen als geld, waarbij
vervalsing en inflatie vrijwel geen kans krijgen.

Het nieuwe digitale goud lost vele praktische problemen op die vroeger
het directe gebruik van goud als geld in de weg stonden. Het handelen in
grote hoeveelheden goud wordt dan niet langer onhandig, omslachtig of
riskant. Digitale ontvangstbewijzen wegen niets en bestaan uitsluitend
als uitgebreide computercode. Bovendien splitsen ze zich moeiteloos op
in zeer kleine eenheden, zodat zelfs aankopen van geringe waarde betaald
kunnen worden. Een fysiek stukje goud, zo klein dat het net voldoende is
voor een chiclet, raakt al snel zoek of raakt verward met een ander
stukje dat net voor twee chiclets toereikend is. Voor een computer maakt
het echter geen verschil of hij een kleine eekhoorn of een neushoorn als
maatstaf hanteert.

Het vermogen van digitaal geld om micropayments te verrichten stimuleert
de opkomst van geheel nieuwe bedrijfsvormen, waarbij ondernemingen zich
specialiseren in de distributie van informatie met geringe waarde.
Verkopers ontvangen dan directe incasso-royalty's, die de voorheen
ontmoedigend hoge transactiekosten overwint. Als de kosten voor
facturatie de waarde van een transactie overstijgen, vindt die
nauwelijks plaats. Met cybergeld worden extreem goedkope en
gelijktijdige factureringen mogelijk, waarbij rekeningen automatisch op
basis van gebruik worden afgeschreven. Zo zou u bijvoorbeeld een royalty
ter waarde van een derde van een cent kunnen betalen aan Bill Gates of
aan degene die de virtual-realityrechten bezit voor een rondleiding door
het Louvre. Vermenigvuldig dit op duizend manieren. Virtual reality
opent vrijwel onbeperkte licentiekansen, die desalniettemin slechts
microroyaltybetalingen vergen. Op een dag beleeft u de derde wedstrijd
van de World Series van 1969 opnieuw én betaalt u microroyalty's aan de
spelers wiens beelden uw virtual reality extra realistisch maken.

\section{Inflatie uitroeien}\label{inflatie-uitroeien}

Hoeveel mogelijkheden er ook zijn, de meest ingrijpende consequentie van
het nieuwe digitale geld wordt ongetwijfeld het einde van de inflatie en
het terugdringen van de financiële hefboomwerking. De economische impact
is enorm. Zoals we in \emph{Blood in the Streets} en \emph{The Great
Reckoning} hebben aangetoond, hing de opkomst van inflatie in de
twintigste eeuw nauw samen met de machtsverhoudingen in de wereld. De
toenemende opbrengsten uit geweld zorgden voor sterk hogere militaire
uitgaven, wat vervolgens leidde tot nog agressievere pogingen om rijkdom
te veroveren. Overheden ontdekten hoe ze in feite jaarlijks een
vermogensbelasting konden innen van iedereen die tegoeden in hun
nationale munteenheid bezat. Je kunt deze jaarlijkse belasting ook zien
als een transactiekost voor het gemak waarmee gebruikers hun vermogen in
een praktische vorm kunnen bewaren, precies zoals de valuta-uitgevers
dat aanbieden.\footnote{Tijdens de industriële periode trok inflatie
  mensen extra aan, omdat prijzen en lonen naar beneden toe niet
  flexibel waren. Een milde inflatie stimuleerde de productie doordat de
  reële lonen en prijzen daalden.}

Het lijkt misschien vreemd om inflatie als een transactiekost te
beschouwen, maar bedenk eens wat het houdt in. Tijdens het industriële
tijdperk raakten we er zo aan gewend dat valuta gratis als dienst werd
geleverd, dat we vergaten dat de overheden -- de uitgevende instanties
van dollars, pesos, ponden en franken -- ons wel degelijk lieten
betalen, zij het via inflatie. De hoogte van deze inflatoire
transactiekost schommelde de afgelopen vijftig jaar van een dieptepunt
van 2,7 procent per jaar voor de Duitse mark tot percentages die
gevaarlijk dicht bij de 100 procent kwamen. Tussen 1960 en 1991 wist
inflatie, onder leiding van president Menem en de hervorming van het
valutaboard in Argentinië, zeventien nullen uit opeenvolgende
muntsystemen te wissen. Als in 1960 alle welvaart ter wereld in
Argentijnse pesos was omgezet en opgeborgen, was het in 1991 de moeite
niet meer waard om die rijkdom op te graven.

Het Argentijnse voorbeeld wijst ons een belangrijke weg voor het komende
millennium. Valuta zal niet langer kunstmatig worden opgeblazen, omdat
staten daar niet meer mee weg kunnen komen -- net zoals Argentinië dat
niet meer kon. Tijdens de industriële periode, toen prijzen en lonen
nauwelijks naar beneden konden, bleek inflatie bovendien aantrekkelijk:
een gematigde inflatie stimuleerde de productie doordat reële lonen
daalden, terwijl prijsschade ontstond door in andere landen
geïmporteerde kredietkrimp. Privategeld zal door de werking van de
concurrentie niet vatbaar zijn voor inflatie.

Het verdwijnen van inflatie betekent dat de verborgen winsten wegvallen
die voorheen werden behaald door partijen die op een monopolistische
wijze valuta uitgaven. Als deze winsten uit de gelduitgifte zouden
verdwijnen, ontstaat de noodzaak voor een nieuwe betaalmethode om de
valuta-uitgevers direct te compenseren. Het gebruik van het nieuwe
monetaire systeem gaat daarom waarschijnlijk gepaard met een expliciete
transactiekost, mogelijk rond de 1 procent per jaar. Dat is een geringe
prijs vergeleken met de jaarlijkse inflatoire last -- variërend van 2,7
tot wel 99 procent -- die staten opleggen, temeer daar algemene prijzen
naar verwachting dalen naarmate monopolies afbrokkelen en de mondiale
concurrentie toeneemt.

\subsection{Afnemende hefboomwerking}\label{afnemende-hefboomwerking}

De opkomst van digitaal geld drukt de inflatie voorgoed de kop in en
vermindert tegelijkertijd de hefboomwerking in wereldwijde banksystemen.
Mensen kunnen immers de toezichthouders omzeilen en hun geld direct via
internet overmaken -- een ongekende uitkomst van de geglobaliseerde
markten. Dit valt buiten het bereik van elke overheid om te reguleren.
Zodra overheden hun valuta niet meer kunnen devalueren door geld bij te
drukken of spaarders te misleiden via onbeperkte kredietverstrekking in
captive banksystemen, verliezen zij een aanzienlijk deel van hun
indirecte macht om middelen in handen te krijgen.

\subsection{Hogere rentetarieven}\label{hogere-rentetarieven}

Voor de meeste westerse regeringen ontstaat hierdoor een duidelijk
dilemma. Zij krijgen te maken met scherpe dalingen in de
belastinginkomsten en met het vrijwel verdwijnen van de hefboomwerking
binnen het monetaire systeem. Tegelijkertijd blijven zij gebonden aan
ongedekte verplichtingen en opgeblazen verwachtingen voor sociale
uitgaven, geërfd uit het industriële tijdperk. De verwachting is dat
deze situatie leidt tot een intense fiscale crisis met talrijke
onaangename sociale neveneffecten, een onderwerp dat in latere
hoofdstukken verder aan bod komt. Economisch gezien zal deze
transitiekrisis waarschijnlijk resulteren in een eenmalige piek van de
reële rentetarieven. Debiteuren komen extra onder druk te staan zodra
zij hun langlopende verplichtingen -- aangegaan volgens het oude systeem
-- moeten aflossen en concessiekredieten opdrogen.

\subsection{Veranderd door
concurrentie}\label{veranderd-door-concurrentie}

Overheden die te maken hebben met serieuze concurrentie voor hun
valutamonopolies zullen waarschijnlijk proberen de cybervaluta, die
tegen vergoeding worden aangeboden, te onderwaarderen. Ze pakken het
kredietbeleid aan en bieden spaarders hogere reële opbrengsten op hun
contante saldi in de nationale valuta. Sommige overheden zouden zelfs
kunnen overwegen goud te remonetariseren als extra wapen tegen de
opkomst van particuliere valuta. Zij redeneren dat een los
gecontroleerde goudstandaard uit de negentiende eeuw tot hogere
zegnaturiewinsten leidt dan wanneer men toestaat dat de nationale valuta
volledig wordt verdrongen door commercieel cybergeld. Niet alle
overheden reageren echter op dezelfde wijze. In regio's waar
computergebruik en internettoegang beperkt zijn, kiezen beleidsmakers in
een vroege fase van de cybereconomie mogelijk voor ouderwetse
hyperinflatie. Dit beleid stelt hen niet in staat de contante saldi van
de rijken over te nemen, maar onttrekt juist middelen van mensen met
weinig vermogen of beperkte toegang tot de cybereconomie. Overheden die
voor zulke tactieken kiezen, kunnen desondanks op internationaal niveau
in cybergeld lenen.

Andere overheden spelen juist in op de mogelijkheden die de
informatie-economie biedt en faciliteren lokale transacties in
cybergeld. Rechtsgebieden die als eersten de geldigheid van digitale
handtekeningen erkennen en de lokale gerechtelijke handhaving van
terugvordering bij wanbetaling van cyberschulden waarborgen, zullen
waarschijnlijk profiteren van een sterke toename in langlopende
kapitaalleningen. In gebieden waar rechtbanken strenge straffen opleggen
of schuldenaren toestaan in gebreke te blijven zonder rechtsmiddelen,
komt er echter geen cybergeld beschikbaar voor langlopende kredieten.

\subsection{Yield gap}\label{yield-gap}

De samenloop van kredietcrisissen, competitieve aanpassingen door
nationale muntautoriteiten en vroege overgangsbelemmeringen bij het
verstrekken van kredieten in cybergeld zorgt in de beginfase van de
informatie-economie voor een opbrengstkloof. Naar verwachting biedt
cybergeld lagere rentetarieven dan de nationale valuta en brengt het
waarschijnlijk ook expliciete transactiekosten met zich mee. Een
verbeterde bescherming tegen verliezen door buitensporige belastingen en
inflatie zorgt ervoor dat deze nadelen worden gecompenseerd. Aangezien
cybergeld vermoedelijk aan goud gekoppeld wordt, profiteert het tevens
van een stijgende goudprijs. De prijs van goud zal naar verwachting
sterk stijgen ten opzichte van andere grondstoffen, ongeacht welke
alternatieve overheidsmaatregel de doorslag krijgt. Waarom? De reële
prijs van goud stijgt vrijwel altijd tijdens een deflatie, want deflatie
duidt immers op een tekort aan liquiditeit. Goud blijft uiteindelijk de
ultieme vorm van liquiditeit.

\subsection{De deflatie van het industriële
tijdperk}\label{de-deflatie-van-het-industriuxeble-tijdperk}

Hogere reële rentevoeten dwingen overal tot het liquideren van dure,
onproductieve activiteiten en drukken de consumptie tijdelijk omlaag. We
hebben de logica van de kredietcyclus en de daaropvolgende afwikkeling
ervan behandeld in \emph{Blood in the Streets} en \emph{The Great
Reckoning}, dus herhalen we die argumenten hier niet. Het volstaat te
stellen dat de deflatoire periode enige tijd kan aanhouden, waarbij de
dure industriële economieën in Noord-Amerika en West-Europa harder onder
de nadelige gevolgen lijden dan de goedkope economieën in Azië en
Latijns-Amerika.

\subsection{Lagere rentes op lange
termijn}\label{lagere-rentes-op-lange-termijn}

Hoewel de opkomst van de cybereconomie aanvankelijk tot hogere
rentetarieven leidt, keert het effect op de lange termijn precies om. Na
belasting stijgen de rendementen voor spaarders fors, zodra middelen
ontsnappen aan de greep van overheden. Dramatische verbeteringen in de
efficiëntie van het hulpbronnengebruik en de bevrijding van kapitaal om
wereldwijd de hoogste rendementen te behalen, compenseren naar
verwachting de aanvankelijk verloren productie tijdens de
transitiecrisis snel.

\subsection{Investeerderscontrole over
kapitaal}\label{investeerderscontrole-over-kapitaal}

Conventionele denkers beweren dat het instorten van de
inkomensherverdeling in de leidende natiestaten de wereld in een
economische catastrofe stort. Geloof dat niet. We betwisten niet dat een
transitiecrisis waarschijnlijk is, maar te denken dat de staat de
economie verbetert door middel van een massale herverdeling van
middelen, is anachronistisch. Dat geloofspunt doet denken aan de
bijgelovigheden aan het einde van de Middeleeuwen, toen men ervan
overtuigd was dat vasten en lijfstraffen gunstig waren voor een
gemeenschap. Vergeet niet dat overheden op grote schaal middelen
verspillen, en middelenverspilling maakt je arm. Dramatisch efficiënter
hulpbronnengebruik ontstaat wanneer de inkomsten, die historisch door
overheden zijn opgeslokt, in handen komen van mensen met werkelijk
talent.

Honderdduizenden -- later zelfs miljoenen -- onafhankelijke individuen
nemen het beheer van tientallen miljarden, uiteindelijk zelfs honderden
miljarden dollars, op zich. Deze nieuwe beheerders van de wereldwijde
rijkdom blijken naar alle waarschijnlijkheid veel vaardiger te zijn in
het optimaal inzetten van middelen en investeringen dan politici. Voor
het eerst in de geschiedenis maken de geopolitieke omstandigheden het
mogelijk dat de beste investeerders en ondernemers -- niet degenen die
zijn gespecialiseerd in geweld -- de uiteindelijke controle over het
kapitaal krijgen. Het is redelijk te verwachten dat de opbrengsten van
deze gedecentraliseerde, door de markt gestuurde investeringen twee- tot
driedubbel zo hoog worden als de schamele rendementen van politiek
gedreven budgettoewijzingen uit het tijdperk van de natiestaat.

In de laatste decennia van de twintigste eeuw vond men in elk land wel
voorbeelden van overheidsinvesteringen die in wezen negatief bleken te
zijn. In de herziene versie van \emph{The Great Reckoning} uit november
1992 haalden wij officiële Russische statistieken aan, waaruit bleek dat
de hele Russische economie ``slechts \$30 miljard waard was, minder dan
een derde van de waarde van haar grondstofinvoer. Impliciet zou de
output van de Russische economie meer dan verdrievoudigen als de
binnenlandse productie- en diensteneconomie volledig zou worden
stilgelegd. In plaats van waarde toe te voegen, onttrekt zij die.''
\footnote{Davidson and Rees-Mogg, op. cit., p.~203.}

Hoewel Rusland na de ineenstorting van het communisme een extreem geval
vertegenwoordigt, toont het bewijs aan dat minder staatscontrole over
middelen de economische efficiëntie bevordert. De groeicijfers die
\emph{The Economist} aanhaalt, duiden erop dat economische vrijheid nauw
samenhangt met economische groei -- de vrijste landen leveren immers de
snelste groeipercentages. Ook de cybereconomie van het
informatietijdperk belooft vrijer te opereren dan welk ander commercieel
domein in de geschiedenis dan ook. Daarmee is het aannemelijk dat de
cybereconomie zich spoedig ontpopt tot de voornaamste economie van het
nieuwe millennium. Haar succes trekt deelnemers uit alle hoeken van de
wereld aan, net zoals het brede gebruik van faxapparaten het
telecopiëren aantrekkelijker maakte voor degenen die er voorheen geen
gebruik van maakten. Belangrijker nog: doordat er geen roofzuchtig
geweld heerst, kan de cybereconomie groeien met aanzienlijk hogere
samengestelde groeipercentages dan de conventionele economieën die door
natiestaten worden gedomineerd.

Dit vormt wellicht het kernpunt bij het inschatten van de economische
impact van de waarschijnlijke ineenstorting van het monopolistische
vermogen van de overheid om belastingen te innen en inflatie te creëren.
Als we de overgangsproblemen -- die zelfs decennialang kunnen voortduren
-- even buiten beschouwing laten, zien de langetermijnvooruitzichten
voor de wereldeconomie er bijzonder rooskleurig uit. Telkens als
omstandigheden mensen in staat stellen om de kosten van bescherming te
verlagen en het `tribut' aan degenen die het georganiseerde geweld
beheersen te minimaliseren, boekt de economie doorgaans dramatische
groei. Zoals Lane zei: ``Ik wil graag suggereren dat de belangrijkste
enkele factor in de meeste groeiperioden -- als er nu één factor het
zwaarst meegewogen heeft -- de vermindering is van het aandeel middelen
dat aan oorlog en politie wordt besteed.'' \footnote{Lane,
  \emph{economic consequences of organized violence}, op. cit., p.~413.}

We kunnen enorme efficiëntiewinsten realiseren door minder middelen te
besteden aan roofzucht en het profiteren van buit. Indien de prijs voor
bescherming competitief zou worden vastgesteld -- zodat lokale
monopolies op basis van prijs en kwaliteit met elkaar concurreren -- zou
dat kunnen leiden tot fors lagere belastingtarieven en minder
verspilling van middelen en energie in politieke activiteiten, die
voorheen enorme rendementen opleverden.

Zouden kiezers bereid zijn afstand te doen van de politieke `meevallers'
waaraan zij gewend zijn geraakt? Die kwestie behandelen we elders
uitgebreid. Eén eenvoudig antwoord zou echter kunnen zijn dat we
waarschijnlijk geen keuze hebben. Tegen regen of droogte protesteert
vandaag de dag immers niemand, hoe economisch nadelig of onaangenaam
deze omstandigheden ook mogen zijn. Niemand, hoe crimineel ook, gijzelt
een arme door met de dood te dreigen en exorbitante betalingen op te
eisen. Als politici niet in staat zijn middelen te vinden om te
herverdelen, reageert het publiek op een rationele manier en laat het de
politiek links liggen -- net zoals men in de middeleeuwen stopte met het
organiseren van boetemarsen.

\setsubtitle{De revolutie van het verdienvermogen in een wereld zonder banen}

\bookmarksetup{startatroot}

\chapter{Het einde van egalitaire
economie}\label{het-einde-van-egalitaire-economie}

\begin{quote}
`God laat zich niet spotten: want wat een mens zaait, zal hij ook
oogsten.' - Galaten 6:7
\end{quote}

Ingrijpende verschuivingen in de dominante vormen van productie en
defensie hertekenen de samenlevingsstructuur en de verdeling van rijkdom
en macht tussen diverse groepen. Het informatietijdperk omvat meer dan
het toenemende gebruik van krachtige computers; het luidt een revolutie
in levensstijlen, instituties en de toewijzing van middelen in. Doordat
de invloed van verborgen geweld bij het beheersen van middelen sterk
afneemt, ontstaat een nieuwe vorm van rijkdom zonder de dwangmatige
inmenging van de overheid, zoals dat in de twintigste eeuw het geval
was.

Omdat locatie in een informatiesamenleving steeds minder bepalend wordt,
spelen organisaties die binnen vaste geografische grenzen opereren in de
toekomst een kleinere rol dan partijen die internationaal actief zijn.
Ook politici, vakbonden, gereguleerde beroepen, lobbyisten en
overheidsinstanties verliezen aan belang. Door het afnemende gewicht van
door overheden opgelegde subsidies en handelsbeperkingen, verspillen we
minder middelen aan het stimuleren of juist tegenwerken van
lobbyactiviteiten.

Degenen die met dwang en regionale voordelen de inkomens herverdelden,
verliezen vrijwel zeker een aanzienlijk deel van hun macht. Daarmee
verschuift de controle over de middelen. De door individuen gegenereerde
rijkdom, die de natiestaat tot nu toe in beslag nam, blijft voortaan in
de handen van degenen die er werkelijk recht op hebben. Steeds meer
rijkdom vindt uiteindelijk zijn weg naar de meest bekwame ondernemers en
durfkapitalisten wereldwijd.

Globalisering en andere kenmerken van de informatie-economie zorgen
ervoor dat de inkomens van de meest getalenteerde vakmensen in elk
domein stijgen. Omdat de extra waarde die voortkomt uit uitmuntende
prestaties zo immens is, zal de verdeling van het verdienvermogen in de
wereldeconomie grotendeels de vorm aannemen die we tegenwoordig al
waarnemen in prestatiegerichte sectoren zoals atletiek en opera.

\section{Een omvang die verder gaat dan Pareto's
wet}\label{een-omvang-die-verder-gaat-dan-paretos-wet}

Pareto's wet houdt in dat 80 procent van het voordeel afkomstig is van
20 procent van de betrokkenen. Hoewel dit in grote lijnen klopt, valt op
dat 1 procent van de bevolking van de Verenigde Staten maar liefst 28,7
procent van de inkomstenbelasting betaalt. Dit wijst erop dat
samenlevingen in het informatietijdperk te maken krijgen met een nog
schevere verdeling van inkomens en vaardigheden dan Pareto aan het einde
van de vorige eeuw constateerde.

Mensen zijn inmiddels goed gewend geraakt aan opvallende verschillen in
rijkdom. Reeds in 1828 werd aangenomen dat 4 procent van de inwoners van
New York 62 procent van de totale rijkdom bezat. In 1845 beschikte de
top 4 procent over zo'n 81 procent van zowel de zakelijke als de
niet-zakelijke rijkdom in New York City. Ook bezat de top 10 procent in
1860 ongeveer 40 procent van de rijkdom in de gehele Verenigde Staten.
Gegevens uit 1890 suggereren dat de rijkste 12 procent destijds zo'n 86
procent van de nationale rijkdom in handen had.\footnote{Benjamin
  Schwarz, `American Inequality: Its History and Scary Future',
  \emph{New York Times}, 19 december 1995, p.~A25.}

De cijfers uit 1890 sluiten min of meer aan bij Pareto's theorie, al
wijken ze enigszins af van de beroemde 80/20-verhouding. Aan het einde
van de negentiende eeuw werd Amerika immers overspoeld door een enorme
instroom van hulpeloze immigranten, die een vrijwel verwaarloosbaar
aandeel in de totale rijkdom hadden. Hun komst maakte de
vermogensverdeling automatisch schever. Dit illustreert treffend dat
elke wezenlijke toename van kansen vrijwel altijd -- zij het tijdelijk
-- leidt tot een grotere ongelijkheid. In 1890 vertegenwoordigden
immigranten ongeveer 15 procent van de totale bevolking van de VS, maar
in enkele noordoostelijke staten -- waar het grootste deel van de
inkomsten en rijkdom werd gegenereerd -- steeg dat aandeel tot meer dan
40 procent.\footnote{Adna Ferrin Weber, `The Growth of Cities in the
  Nineteenth Century' (New York: \emph{Macmillan}, 1899; herdrukt door
  \emph{Cornell University Press}, 1963), p.~249.} Als men de toestroom
meerekent, paste het Amerika van eind negentiende eeuw vrijwel net zo
goed in Pareto's formule als het Zwitserland, waar hij destijds woonde.

Het informatietijdperk heeft de rijkdomsverdeling inmiddels al
ingrijpend veranderd, met name in de Verenigde Staten, en vormt een
belangrijke oorzaak van de hedendaagse politieke onvrede, waar we in het
volgende hoofdstuk dieper op ingaan. Economisch succes in dit tijdperk
vraagt om een behoorlijk hoog niveau van geletterdheid en
rekenvaardigheid. Een grootschalige enquête van het Amerikaanse
ministerie van Onderwijs, \emph{`Adult Literacy in America'}, toonde aan
dat maar liefst 90 miljoen Amerikanen ouder dan vijftien schrijnend
onbekwaam zijn. Of, om het kleurrijker te verwoorden zoals de
Amerikaanse expat Bill Bryson het formuleert: `Ze zijn zo dom als
varkenskwallen.'\footnote{Bill Bryson, `The Lost Continent' (New York:
  \emph{Harper Perennial}, 1989), p.~72.} Concreet bleek dat deze 90
miljoen volwassenen niet in staat waren om een brief te schrijven, een
busrooster te begrijpen of simpele rekensommen uit te voeren -- zelfs
niet met behulp van een rekenmachine. Degenen die al moeite hebben met
een gewoon busrooster kunnen de informatiesnelweg waarschijnlijk ook
niet goed volgen. Hieruit broeit een boze onderklasse bij de Amerikanen
die zich niet hebben voorbereid op de digitale informatiewereld. Aan de
top bevindt zich een kleine groep -- wellicht zo'n 5 procent -- van
hoogopgeleide informatiewerkers en kapitaalbezitters, die te vergelijken
vallen met de landeigen aristocratie uit de feodale tijd. Het
fundamentele verschil is dat de elite van het informatietijdperk
specialisten in productie vormt in plaats van in geweld.

\subsection{De megapolitiek van
innovatie}\label{de-megapolitiek-van-innovatie}

De meeste sociologen in de twintigste eeuw gingen ervan uit -- zonder
dat daar een goede reden voor was -- dat technologische vooruitgang
automatisch tot egalitairder samenlevingen leidt. Voor ongeveer 1750
liep dat echter anders. Vanaf dat moment bood de nieuwe industriële
technologie ongeschoolden werk en groeide de schaal van ondernemingen.
De fabriekstechnologie zorgde er niet alleen voor dat de reële inkomsten
van de armen stegen zonder extra inspanning, maar versterkte ook de
politieke instellingen, zodat zij effectiever inkomen konden herverdelen
en onrust konden bedwingen. Op de lange termijn bestaat er geen
fundamentele reden om te veronderstellen dat technologie de verschillen
in talent en motivatie altijd verkleint in plaats van juist te
benadrukken. Sommige technologieën bleken relatief egalitair omdat ze de
inzet van vele onafhankelijke arbeiders met ongeveer gelijke
vaardigheden vereisten, terwijl andere technologieën de macht of rijkdom
in handen legden van enkelen, waardoor de overgrote meerderheid
nauwelijks meer dan horigen bleef. Zowel historische ontwikkelingen als
innovatieve technologieën hebben naties op uiteenlopende wijze gevormd.
Het fabriekstijdperk bracht een bepaalde maatschappelijke vorm voort,
terwijl het informatietijdperk een ander model neerzet -- minder
gewelddadig, maar juist meer elitair en minder egalitair dan het systeem
dat het verdringt.

\section{Ammon's raap}\label{ammons-raap}

Aan het einde van de negentiende eeuw namen diverse economen het
initiatief om de mathematische economie te ontwikkelen, met als meest
prominente vertegenwoordiger in Engeland \emph{William Stanley Jevons}.
Een van de eersten die de kansrekening toepaste op een belangrijke
sociale kwestie was de Duitse econoom Otto Ammon. Zijn werk werd in 1899
voor het eerst in het Engels vertaald door \emph{Carlos C. Closson} in
een artikel in het \emph{Journal of Political Economy}. Het artikel
droeg de titel `Some Social Applications of the Doctrine of
Probability.'\footnote{Dit artikel is herdrukt in vol.~4 van de
  verzameling `Early Mathematical Economists' van Adrian Darnell (6
  delen, London: \emph{Pickering \& Chatto}, 1991).} Je zou kunnen
denken dat dit artikel puur een antiquarische curiositeit is, maar in
werkelijkheid behandelt het een economisch vraagstuk dat vandaag de dag
opnieuw in de belangstelling staat -- en dat op nog steeds stimulerende
wijze.

Otto Ammon richtte zich op de spreiding van bekwaamheid in de
samenleving en op de relatie daarvan met de verdeling van inkomen en
sociale status. Hij baseerde zijn analyse op de waarschijnlijke
uitkomsten van de totaalscores bij het werpen van vier dobbelstenen, elk
met zes zijden. Van de 1.296 mogelijke worpen komen sommige sommen
frequenter voor dan andere.

Een som van 24 punten komt slechts één keer voor.

Een som van 23 punten komt vier keer voor.

Een som van 22 punten komt tien keer voor.

Een som van 21 punten komt twintig keer voor.

Een som van 20 punten komt vijfendertig keer voor.

Je behaalt een totaal van 19 punten in 56 gevallen.

Je behaalt een totaal van 18 punten in 80 gevallen.

Je behaalt een totaal van 17 punten in 104 gevallen.

Je behaalt een totaal van 16 punten in 125 gevallen.

Je behaalt een totaal van 15 punten in 140 gevallen.

De som van 14 punten komt 146 keer voor.

De som van 13 punten komt 140 keer voor.

De som van 12 punten komt 125 keer voor.

De som van 11 punten komt 104 keer voor.

De som van 10 punten komt 80 keer voor.

De som van 9 punten komt 56 keer voor.

De som van 8 punten komt 35 keer voor.

De som van 7 punten komt 20 keer voor.

De som van 6 punten komt 10 keer voor.

De som van 5 punten komt 4 keer voor.

De som van vier punten komt slechts éénmal voor.*

Het valt direct op dat zowel extreem hoge als extreem lage scores vrij
zeldzaam zijn. Hoewel er twee mogelijke totalen bestaan, komen de vier
hoogste daarvan slechts 35 keer voor. De reeks van zeven middelste
scores verwacht je ongeveer 884 keer tegen te komen, want meer dan
tweederde van alle worpen levert een uitkomst in dat centrale bereik op.
Dit verklaart de karakteristieke concentratie in het midden van de
kansverdeling.

Otto Ammon stelde dat de willekeurige spreiding bij dobbelsteenworpen
ook tot uiting komt in de verdeling van menselijke vermogens. Hij
schreef hierover nog vóór de opkomst van intelligentietesten en IQ's en
bouwde voort op de eerdere inzichten van Francis Galton. Ammon ging er
niet vanuit dat maatschappelijk nut of succes in het leven louter
afhangt van intelligentie. Hij maakte onderscheid tussen drie groepen
mentale eigenschappen die grotendeels bepalen welke positie een man in
het leven zal innemen. Deze waren:

\begin{enumerate}
\def\labelenumi{\arabic{enumi}.}
\item
  Intellectuele eigenschappen: daarmee bedoelde hij alle kenmerken die
  tot de rationele vermogens van de mens behoren, zoals snel begrip, een
  scherp geheugen, goed beoordelingsvermogen, vindingrijkheid en alle
  andere kwaliteiten in dat domein.
\item
  Morele eigenschappen: zoals zelfbeheersing, wilskracht, ijver,
  doorzettingsvermogen, matigheid, respect voor gezinsverplichtingen en
  eerlijkheid.
\item
  Economische eigenschappen, zoals zakelijk inzicht, organiserend
  talent, technische bekwaamheid, voorzichtigheid, een slimme
  berekening, vooruitziendheid en zuinigheid, enzovoort.
\end{enumerate}

Aan deze mentale kwaliteiten voegde hij het volgende toe:

\begin{enumerate}
\def\labelenumi{\arabic{enumi}.}
\setcounter{enumi}{3}
\tightlist
\item
  Lichamelijke eigenschappen, zoals arbeidscapaciteit,
  uithoudingsvermogen, het vermogen om inspanningen en alle vormen van
  opwinding te weerstaan, vitaliteit en een goede gezondheid, enzovoort.
\end{enumerate}

Volgens Otto Ammon lijkt de verdeling van deze kwaliteiten --
intelligentie, karakter, talent en lichamelijke vermogens -- op de
uitkomsten van dobbelsteenworpen. Hij ging nog verder en betoogde dat er
in werkelijkheid veel meer variabelen meespelen dan enkel vier, en dat
deze zich in meer dan zes gradaties onderscheiden. Als men acht
dobbelstenen gooit in plaats van vier, ontstaan er maar liefst 1.679.616
mogelijke uitkomsten, waarbij de hoogste score, achtenveertig, naar
verwachting slechts éénmalig optreedt. Iemand die op alle factoren die
de levenspositie bepalen uitzonderlijk hoog scoort, komt dan veel minder
vaak voor dan de kans op vier zessen doet vermoeden; mogelijk is die
zeldzaamheid te vergelijken met het werpen van acht zessen.

Ammon merkt echter op dat een combinatie van hoge en lage scores op deze
menselijke kwaliteiten kan resulteren in `personen met onevenwichtige,
onharmonieuze gaven, die, ondanks enkele briljante eigenschappen, de
beproevingen van het leven niet met succes doorstaan.'

\begin{quote}
`Als een eenzame bergtop, of liever de spits van een kathedraal, rijzen
de mannen van hoog talent en genie boven de brede massa van
middelmatigheid\ldots{} Het aantal hoogbegaarden is immers zo gering dat
het ondenkbaar lijkt dat door de onvolledigheid van sociale instellingen
`vele' van zulke mensen in de lagere klassen blijven hangen.' -- OTTO
AMMON
\end{quote}

\subsection{Eigenschappen en inkomens}\label{eigenschappen-en-inkomens}

Ammon richt zich vervolgens op de inkomensverdeling. Natuurlijk waren de
statistieken uit de jaren 1890 minder nauwkeurig dan tegenwoordig, maar
de Duitse bureaucratie was al goed georganiseerd. Otto Ammon ontdekte in
Saksen, Pruisen, Baden en andere Duitse staten inkomenscurves die hij
vergeleek met zowel de door hem geconstateerde verdeling van menselijke
capaciteiten als met de kansen van dobbelstenen. Hij vond soortgelijke
gegevens in \emph{Charles Booth's Life and Labour of the People of
London} (1892). Inderdaad, de sociale indeling volgens Booth komt
precies overeen met wat men op basis van Ammons
waarschijnlijkheidstheorie zou verwachten. In Londen constateerde Booth
dat 25 procent als arm of slechter werd aangemerkt, 51,5 procent `het
goed had' en 15 procent als welvarend of beter werd gezien; wanneer men
de twee onderste categorieën combineert, blijkt dat slechts 9,5 procent
betreft. Vóór de verzorgingsstaten van de twintigste eeuw noemde men de
minstwelvarenden vaak de `ondergedoken tiende.' Samen vormden de twee
hoogste categorieën volgens Booth 7 procent.

Hieruit trok Otto Ammon enkele interessante conclusies. Hij stelde dat
de capaciteiten van mensen -- in ruime zin gezien -- hun plaats in de
samenleving en hun inkomen bepalen. Hij was ervan overtuigd dat hoge
capaciteiten er vanzelf toe leiden dat men qua inkomen en sociale status
omhoog klimt. `Zoals een eenzame bergtop -- of liever de torenspits van
een kathedraal -- steken mannen met uitzonderlijk talent en genie boven
de massa van middelmatigheid uit\ldots{}' Daarnaast meende hij dat de
ware vorm van de zogenaamde sociale piramide te vergelijken valt met een
tamelijk platte ui of rap -- een rap met een smalle steel aan de
bovenkant en een beperkte wortel aan de onderkant. Deze metafoor van de
sociale rap verkiest hij boven de traditionele piramide, omdat de rap --
net als de moderne industriële samenleving -- haar massa in het midden
vindt, terwijl de piramide juist onderaan concentreert.

\subsection{De vorm van de rap}\label{de-vorm-van-de-rap}

Moderne industriële samenlevingen lijken in feite op rapen: aan de top
bevindt zich een kleine elite van welgestelden en hooggekwalificeerden,
gevolgd door een ruime middenklasse, met onderaan een minderheid van de
armen. In hedendaags Londen -- zo niet in Washington -- telt men zeker
meer miljonairs dan daklozen.

Al dit is intrigerend, maar wat Ammons werk direct relevant maakt, is de
ingrijpende langetermijnverandering in de financiële en politieke
verhoudingen tussen de top en de middenklasse. De vaardigheden die ooit
in het fabriekstijdperk nodig waren -- een tijdperk dat nu ten einde
komt -- verschillen radicaal van die welke het informatietijdperk
vandaag de dag vraagt. De meeste mensen beheersen de competenties die
nodig waren om de machines uit het midden van de twintigste eeuw te
bedienen, maar die functies vervallen inmiddels doordat slimme,
zelfsturende machines hun werk overnemen. Een aanzienlijk deel van laag-
en middengekwalificeerde banen is al verdwenen. Als wij gelijk hebben,
duidt dit op het verdwijnen van de meeste banen en op een ingrijpende
herstructurering van arbeid op de spotmarkt.

\begin{quote}
`Toch is het een feit, officieel maar stilletjes erkend, dat de meeste
werkloze jongeren totaal geen kwalificaties hebben\ldots{}'\footnote{Clive
  Jenkins en Barrie Sherman, The Collapse of Work (London: Methuen,
  1979), p.~103.} - Clive Jenkins en Barrie Sherman
\end{quote}

\section{Minder mensen zullen meer werk
verrichten}\label{minder-mensen-zullen-meer-werk-verrichten}

Neem als uitgangspunt de eenvoudige voorstelling van menselijke
bekwaamheid met vier dobbelstenen en stel dat mensen in het
fabriektijdperk een score van 4 x 2 of hoger konden halen. Dat zou
betekenen dat meer dan 95 procent van de bevolking boven wat Charles
Booth `de laagste grens van positieve maatschappelijke bruikbaarheid'
noemde scoorde. In de jaren veertig en vijftig stelde men overigens een
norm van 3 procent voor volledige werkgelegenheid vast. Stel nu dat in
het informatietijdperk de benodigde score stijgt naar 4 x 3, waardoor
het vereiste minimum toeneemt van 8 naar 12. Dan komt bijna 24 procent
van de bevolking onder deze grens van `maatschappelijke bruikbaarheid'.

Hetzelfde geldt aan de bovenkant van de schaal. Vroeger volstond een
hoge bekwaamheidsgraad van 4 x 4 in het fabriektijdperk; stel dat die in
het informatietijdperk oploopt naar 4 x 5. In dat geval daalt het
aandeel mensen dat in aanmerking komt voor topfuncties -- oftewel de
best betaalde banen -- van 34 procent naar slechts 5 procent.

Deze cijfers zijn louter hypothetisch. We weten natuurlijk niet hoe de
vereisten voor vaardigheden zich precies zullen verschuiven -- of al
hebben verschoven -- maar er is onmiskenbaar sprake van een toename.
Door de steile aard van de vaardighedencurve kan al een bescheiden
verhoging van de minimale vereiste ertoe leiden dat veel mensen buiten
een zinvolle economische rol vallen. Ook een kleine verhoging van de
hogere vaardigheidseis doet het aantal mensen dat in aanmerking komt
voor hogere functies drastisch dalen. Er is sprake van een verschuiving,
maar hoe groot deze zal worden, blijft voorlopig onduidelijk.

Er ontbreekt zeker niet aan sociaal en politiek bewijs dat deze
verschuiving in alle geavanceerde industriële samenlevingen plaatsvindt,
dat het tempo toeneemt en dat de beweging al omvangrijk is. De
beloningen voor zeldzame vaardigheden zijn gestegen en blijven groeien.
Conventionele denkers hebben dit met tegenzin opgemerkt. Neem
bijvoorbeeld \emph{The Winner-Take-All Society} van Robert H. Frank en
Philip J. Cook.\footnote{Robert H. Frank en Philip J. Cook, The
  Winner-Take-All Society (New York: The Free Press, 1995).} Dit boek
documenteert de groeiende tendens dat de meest getalenteerde deelnemers
in vele vakgebieden in de Verenigde Staten extreem hoge inkomens
vergaren. Tegelijkertijd nemen de kansen voor mensen met gemiddelde
vaardigheden af; een aanzienlijk aantal mensen met weinig vaardigheden
valt nu buiten de inkomensbandbreedte die een comfortabel bestaan
mogelijk maakt, al vinden zij vaak nog wel een plek in kleinschalige
dienstverlening.

Als het informatietijdperk hogere vaardigheden eist, zowel aan de top
als aan de onderkant, zullen vrijwel alle mensen -- met uitzondering van
de bovenste 5 procent -- relatief benadeeld uitpakken, terwijl de top 5
procent enorm profiteert. Zij vergaren een groter deel van het inkomen
en behouden tevens een groter aandeel van wat zij verdienen. Bovendien
verrichten zij een aanzienlijk groter deel van het werk wereldwijd dan
ooit tevoren. Velen ontwikkelen zich tot soevereine individuen. In het
informatietijdperk zal de inkomensverdeling waarschijnlijk meer op die
van 1750 lijken dan op die van 1950.

Samenlevingen waarin men indoctrineert dat inkomensgelijkheid en hoge
consumptie voor mensen met weinig of gemiddelde vaardigheden de norm
dienen te zijn, zullen te maken krijgen met demotivatie en onzekerheid.
Naarmate meer landen informatietechnologie in hun economie integreren,
zal de opkomst van een min of meer niet-inzetbare onderklasse -- zoals
al evident in Noord-Amerika -- steeds duidelijker worden. Precies dat is
wat er gebeurt. Dit zal leiden tot een reactie met een nationalistische,
antitechnologische inslag, zoals we in het volgende hoofdstuk
toelichten.

De fabrieksleeftijd zou wel eens een unieke periode kunnen blijken te
zijn waarin eenvoudige machines een zeer winstgevende niche voor
ongeschoolde mensen openden. Nu machines voor zichzelf kunnen zorgen,
profiteren de bovenste 5 procent van Otto Ammon's knol ten volle van de
geschenken van het informatietijdperk. Het informatietijdperk bood al
veel voordelen voor de bovenste 10 procent, de zogenoemde cognitieve
elite. Maar het allerbeste komt naar de bovenste 10 procent van die
bovenste 10 procent, de cognitieve dubbele top. Tijdens de feodale
periode had één hoogopgeleide krijgsheer (of ridder) op paard honderd
halfgeschoolde boeren ter ondersteuning nodig. De soevereine individuen
in de informatie-economie worden geen krijgsheren, maar meesters in
gespecialiseerde vaardigheden, zoals ondernemerschap en investeringen.
Toch lijkt de oude feodale verhouding van honderd tegen één weer op te
duiken. Wat de gevolgen ook zullen zijn, samenlevingen in de
eenentwintigste eeuw worden waarschijnlijk ongelijker dan die waarin we
in de twintigste eeuw leefden.

\section{De meeste mensen zullen profiteren van de dood van de
politiek}\label{de-meeste-mensen-zullen-profiteren-van-de-dood-van-de-politiek}

Het is onwaarschijnlijk dat de egalitaire economie -- en daarmee de
naties die haar ondersteunen -- zonder crisis zal verdwijnen. Hoewel een
crisis van nature kort duurt, verwachten we desalniettemin dat het
trauma dat gepaard gaat met het verdwijnen van naties nog jarenlang zal
voortduren. Terwijl we dit trauma, waarvan we de omvang later uitgebreid
behandelen, niet uit het oog verliezen, mogen we niet vergeten dat de
overgang naar de informatie-economie in veel delen van de wereld tot een
enorme productieboost leidt en voor iedereen hogere inkomens oplevert.
Inderdaad, in regio's die nooit ten volle hebben kunnen profiteren van
de voordelen van het industrialisme maar nu de vrije markt omarmen,
zullen de inkomens in alle lagen van de bevolking stijgen -- of ze
stijgen al.

De vermindering van economische dwang geeft producenten de mogelijkheid
om activa te behouden die voorheen in beslag werden genomen en
herverdeeld. Vroeger hield herverdeling er vaak toe dat deze activa naar
minder waardevolle toepassingen werden verplaatst, wat de productiviteit
van het kapitaal verlaagde. Politici namen op onevenredige wijze
vermogen weg van degenen die uitblonken in het investeren en
herverdeelden dit vervolgens ten gunste van mensen die minder bedreven
waren. In de meeste gevallen stroomde het herverdeelde inkomen in
economische activiteiten van een lagere orde. Het losmaken van middelen
uit systematische dwang zal per rechtsgebied sterk verschillen. Deze
`bevriezing' van middelen dreigt welvaartsstaten tot faillissement te
brengen en vergroot de nadelen van schaal die grote overheden en hun
gesubsidieerde instellingen ondermijnen. Aan de andere kant zal de
verschuiving naar de cybereconomie de economische nadelen beperken die
mensen ervaren in regio's waar soevereine staten traditioneel moeite
hebben met grootschalige organisatie.

\begin{quote}
`Als de wereld functioneert als één grote markt, zal iedere werknemer
concurreren met iedere persoon, overal ter wereld, die in staat is
hetzelfde werk te verrichten. Er zijn er veel en velen van hen hebben
honger.'\footnote{Clay Chandler, `Buchanans succes schrikt het
  bedrijfsleven af', \emph{Washington Post}, 22 februari 1996, p.~D12.}
- ANDREW S. GROVE, PRESIDENT, INTEL CORP.
\end{quote}

\section{Verschuivende
locatievoordelen}\label{verschuivende-locatievoordelen}

Omdat er geen gestage stijging van de opbrengsten uit geweld meer zal
zijn, biedt het niets om onder een regering te leven die daarvan
profiteert. Regeringen die ooit als bondgenoten in de vermogensopbouw
werden gezien, vormen nu juist de tegenstanders ervan. Hoge belastingen,
forse reguleringslasten en ambitieuze verplichtingen tot
inkomensherverdeling maken de gebieden onder hun controle
onaantrekkelijk voor ondernemerschap.

Degenen die in rechtsgebieden wonen die tijdens de industriële periode
arm of onderontwikkeld bleven, winnen het meest wanneer economieën
loskomen van de beperkingen van de geografie. Dit staat in schril
contrast met wat u wellicht hoort. De belangrijkste controverse rond de
opkomst van de informatie-economie en het soevereine individu richt zich
op de vermeende nadelige gevolgen voor `billijkheid' als gevolg van het
verdwijnen van de politiek. Ongetwijfeld zal de mondiale
informatie-economie een fatale klap toebrengen aan grootschalige
inkomensherverdeling. In het industriële tijdperk profiteerden vooral
inwoners van welvarende rechtsgebieden, waar het consumptieniveau tot
wel twintigmaal hoger lag dan het wereldgemiddelde. Inkomensherverdeling
heeft binnen de OESO-landen slechts merkbare effecten gehad op de
inkomens van laaggeschoolde personen.

De meest opvallende inkomensverschillen vallen tussen rechtsgebieden op.
Inkomensherverdeling heeft nauwelijks bijgedragen deze kloof te
verkleinen. Sterker nog, wij menen dat buitenlandse hulp en
internationale ontwikkelingsprogramma's een perverse uitwerking hebben:
door incompetente regeringen te subsidiëren, verlagen zij de reële
inkomens van armen in arme landen. Dit onderwerp behandelen we
uitvoeriger in onze analyse van de impact van de informatierevolutie op
de moraal.

Een eeuw van stijgende inkomensongelijkheid

In de industriële periode bepaalde het rechtsgebied waarin iemand woonde
grotendeels zijn levenslange inkomen. In tegenstelling tot de gangbare
indruk in de welvarende economieën van tegenwoordig, nam de
inkomensongelijkheid in die tijd sterk toe. Volgens een schatting van de
\emph{Wereldbank} lag het gemiddelde inkomen per inwoner in de rijkste
landen in 1870 elf maal hoger dan in de armste, en in 1985 was dit
evenredig gestegen tot tweeënvijftig maal. Hoewel de wereldwijde
ongelijkheid drastisch toenam, merkten de inwoners van de rijke
industriële landen dat nauwelijks op: de inkomensverschillen namen
vooral toe tussen rechtsgebieden en niet binnen dezelfde rechtsgebieden.

Zoals we eerder al bespraken, zorgde de aard van de industriële
technologie ervoor dat in rechtsgebieden waar capabele regeringen op
grote schaal de macht uitoefenden, de inkomenskloof afnam. Naarmate de
effecten van geweld toenamen -- zoals tijdens het industriële tijdperk
gebeurde -- namen de eigen medewerkers vaak de controle over grote
regeringen over. Daardoor bleek het praktisch onmogelijk om grenzen te
stellen aan de middelenaanspraken van deze regeringen. Hun onbelemmerde
controle over middelen leverde een aanzienlijk militair voordeel op,
zolang de omvang van hun macht zwaarder woog dan de efficiëntie waarmee
ze deze benutten.

Niet toevallig versnelde de interne controle van regeringen door hun
eigen medewerkers de inkomensherverdeling sterk. Bijna iedere
samenleving kent wel een vorm van inkomensherverdeling, al is dit soms
slechts tijdelijk in uitzonderlijke omstandigheden. Als je de
geschiedenis van de ondersteuning aan armen nader bekijkt, valt op dat
`welfare'-uitkeringen doorgaans genereuzer uitvallen wanneer armoede
minimaal is. De herverdeling van inkomens wordt eerder beperkt wanneer
de inkomens van grote groepen dalen. In de welvarende industriële
samenlevingen van de tweede helft van de twintigste eeuw waren de
omstandigheden vrijwel ideaal voor inkomensherverdeling. Dat resulteerde
in aanzienlijk hogere beloningen voor laaggeschoolde arbeid in deze
bevoorrechte rechtsgebieden, waardoor zelfs mensen die helemaal niet
werkten een hoog consumptieniveau konden handhaven.

\subsection{De paradox van industriële
rijkdom}\label{de-paradox-van-industriuxeble-rijkdom}

De ironie is dat juist in deze rechtsgebieden meer mensen rijk werden.
Zodra je de dynamiek van de megapolitiek -- zoals in eerdere
hoofdstukken besproken -- doorziet, verklaart deze schijnbare paradox
zich volstrekt. Voor belangrijke sectoren binnen de industriële economie
was het essentieel om op grote schaal orde te handhaven om optimaal te
kunnen functioneren. Daardoor werden zij extra kwetsbaar voor afpersing
door vakbonden en overheden die vastbesloten waren hun invloedssfeer
zoveel mogelijk uit te breiden. Toch belemmerde de brede
inkomensherverdeling het functioneren van de industriële economie niet
volledig. Wie het geluk had geboren te worden in West-Europa, in de
voormalige Britse koloniën of in Japan tijdens de hoogtijdagen van de
industrialisatie, vergaarde doorgaans aanzienlijk meer rijkdom dan
iemand met vergelijkbare vaardigheden in Zuid-Amerika, Oost-Europa, de
late Sovjet-Unie, Afrika of op het vasteland van Azië. Bovendien zal de
positieve invloed van \emph{informatietechnologie} helpen tal van
ontwikkelingsdrempels te overwinnen -- drempels die gedurende een groot
deel van de moderne periode de meerderheid van de wereldbevolking ervan
weerhielden te profiteren van de voordelen van vrije markten.

\begin{quote}
`De inheemse kenmerken van arme landen zijn opvallend ongunstig voor een
effectieve grootschalige organisatie, in het bijzonder voor
grootschalige organisaties die -- zoals overheden -- over een groot
geografisch gebied moeten opereren.'\footnote{Zie Mancur Olson,
  `Diseconomies of scale and development,' \emph{Cato Journal}, vol.~7,
  nr. 1 (lente/zomer 1987).} - MANCUR OLSON
\end{quote}

\section{Schaalnadelen en vertraagde
groei}\label{schaalnadelen-en-vertraagde-groei}

Zoals Mancur Olson aantoont, lag de achterstand in de twintigste eeuw
niet per se aan een tekort aan kapitaal of aan gespecialiseerde
vaardigheden. In zijn essay `Schaalnadelen en ontwikkeling' -- dat in
1987 verscheen, twee jaar voor de val van de Berlijnse Muur -- schreef
Olson: `Als kapitaal in arme landen werkelijk schaars was, zou de
marginale productiviteit en daarmee de winstgevendheid bij het gebruik
ervan hoger moeten liggen dan in de welvarende landen. Bovendien
weerlegden de lage groeipercentages in landen die toch voldoende
buitenlandse hulp ontvingen en de geringe productiviteit van zelfs
enkele moderne fabrieken in arme landen de verklaring `schaarste aan
kapitaal' als oorzaak van onderontwikkeling.'\footnote{Ibid.} Dat is
logisch: als het tekort aan kapitaal of vaardigheden de voornaamste
tekortkoming was, zouden de opbrengsten in arme regio's hoger uitvallen
dan in ontwikkelde landen. Zowel gekwalificeerd personeel als kapitaal
zou massaal naar die gebieden zijn getrokken zodat de opbrengsten zich
zouden normaliseren. In werkelijkheid gebeurde vaak precies het
tegenovergestelde. Er stroomden aanzienlijk veel opgeleide mensen weg
uit achterstandsgebieden en enkelen die erin slaagden kapitaal te
vergaren, exporteerden dat haastig naar Zwitserland en andere
ontwikkelde landen.

Beter bestuur liet zich immers niet importeren

Olson betoogt -- en daar zijn wij het mee eens -- dat het reële obstakel
voor ontwikkeling in achterblijvende landen het bestuur was: de enige
productiefactor die je niet eenvoudig kunt lenen of importeren. Dit
probleem verergerde naarmate de twintigste eeuw vorderde. Rond 1900
exporteerden Groot-Brittannië, Frankrijk en enkele andere Europese
landen bekwaam bestuur naar regio's waar inheemse machten niet in staat
waren op grote schaal effectief te functioneren. Maar ingrijpende
geopolitieke veranderingen in de twintigste eeuw deden de kosten van
deze activiteit stijgen en de opbrengsten dalen. Kolonialisme -- of
imperialisme, zoals men het liever niet noemde -- bleek geen
winstgevende onderneming meer te zijn. Technologische ontwikkelingen
maakten het duurder om macht vanuit het centrum naar de periferie over
te brengen, terwijl tegelijk de militaire kosten voor effectieve
weerstand afnamen. Hierdoor trokken de imperialistische machten zich
terug, of bleven zij slechts in kleine enclaves aanwezig, zoals op
Bermuda of de Kaaimaneilanden.

\begin{quote}
`Als de postkoloniale natiestaat een belemmering voor vooruitgang was
geworden, zoals tegen het einde van de jaren tachtig steeds meer critici
in Afrika leken te beamen, lag de voornaamste reden vrijwel
onbetwistbaar elders. De staat bevrijdde en beschermde haar burgers
niet, ongeacht wat haar propaganda beweerde; integendeel, haar algehele
werking was beperkend en uitbuitend, of functioneerde simpelweg in geen
enkele maatschappelijke zin.'\footnote{Basil Davidson, \emph{The Black
  Mans' Burden: Africa and the Curse of the Nation State} (New York:
  Times Books, 1992), p.~290.} -- BASIL DAVIDSON
\end{quote}

De inheemse regeringen die de koloniale overheersing vervingen in landen
waar Europeanen zich niet hadden gevestigd, rekruteerden hun leiders en
bestuurders uit bevolkingsgroepen met weinig ervaring of deskundigheid
in het aansturen van grootschalige ondernemingen. Vaak werd de
infrastructuur die zij overnamen van de vertrekkende koloniale machten
-- vooral in Afrika -- rap geplunderd, vernietigd of in verval
achtergelaten. Telefoonlijnen werden door schavelaars neergehaald en tot
armbanden gesmeed, wegen kregen geen onderhoud meer en spoorwegen
raakten nutteloos doordat de spoorbeddingen uit elkaar vielen en de
locomotieven stukgingen. In Zaïre verdween de uitgebreide
transportinfrastructuur, die ooit door de Belgen was aangelegd, tegen
1990 vrijwel geheel. Slechts enkele piepende rivierschepen bleven nog in
gebruik, waarvan één door de dictator als een soort drijvend paleis werd
ingezet.

Onbetrouwbare communicatie en transport tonen de onkunde van
achterblijvende natiestaten bij het handhaven van de orde. Zij houden
kunstmatig hoge prijzen aan en beperken zo de kansen voor het merendeel
van de wereldbevolking. Olson verwoordt dit als volgt:

\begin{quote}
Ten eerste dwingen gebrekkige transport- en communicatiesystemen
ondernemingen vrijwel uitsluitend te vertrouwen op lokale
productiefactoren. Als een onderneming groeit, moet zij ver reiken om de
benodigde productiefactoren te vinden -- en hoe minder betrouwbaar die
systemen zijn, hoe sneller de kosten om de productie uit te breiden
zullen stijgen. Ten tweede -- en nog belangrijker -- bemoeilijken zulke
gebrekkige systemen de effectieve coördinatie van grootschalige
ondernemingen aanzienlijk.\footnote{Olson, op. cit.}
\end{quote}

\subsection{Het verlichten van de last van slecht
bestuur}\label{het-verlichten-van-de-last-van-slecht-bestuur}

De ambitieuze armen wereldwijd profiteren in het bijzonder nu
informatietechnologie het genereren van inkomen loskoppelt van de
woonplaats. Nieuwe technologieën -- denk aan de digitale mobiele
telefoon -- zorgen ervoor dat communicatie gewoon doorgaat, zonder dat
de lokale politie iedere telefoonpaal binnen haar rechtsgebied hoeft te
bewaken tegen koperdieven. Nu draadloze fax- en internetverbindingen
beschikbaar komen, maakt het bovendien weinig uit als noodlijdende arme
postmedewerkers de post in beslag nemen, zelfs als dat enkel is om
postzegels te stelen.

Effectieve communicatie maakt in veel gevallen zelfs het fysiek
verplaatsen van goederen en diensten overbodig. Verbeterde
communicatiemiddelen gecombineerd met een enorme toename in rekenkracht
zorgen ervoor dat het coördineren van complexe activiteiten niet alleen
goedkoper en efficiënter gaat, maar dat de voordelen van
schaalvergroting afnemen en grote organisaties overbodig worden. Deze
ontwikkelingen verlichten de druk waaronder mensen in achtergestelde
landen lijden door te leven onder incompetente overheden. Door de
informatierevolutie wordt het minder essentieel dat overheden capabel
functioneren, waardoor mensen in traditioneel arme landen makkelijker de
obstakels kunnen overwinnen die hun overheden tot nu toe in de weg
stonden van economische groei.

\subsection{Gelijke kansen in het
informatietijdperk}\label{gelijke-kansen-in-het-informatietijdperk}

In het informatietijdperk transformeert technologie de traditionele
locatievoordelen razendsnel. Mensen met vergelijkbare vaardigheden gaan
naar verwachting evenveel verdienen, ongeacht in welk rechtsgebied zij
wonen -- een ontwikkeling die al in gang is gezet. Omdat instanties die
dwang en lokale voordelen inzetten voor inkomensherverdeling snel
terrein verliezen, neemt de inkomensongelijkheid binnen rechtsgebieden
toe. Wereldwijde concurrentie zorgt bovendien dat de verdiensten van de
meest getalenteerde professionals, waar ze ook wonen, stijgen --
vergelijkbaar met wat we zien in de professionele atletiek. Superieure
prestaties op de wereldmarkt leveren namelijk een enorme meerwaarde op.

Terwijl het publieke debat in de OESO-landen vaak de groeiende
ongelijkheid onder de loep neemt, krijgen mensen wereldwijd veel meer
gelijke kansen. Succes wordt niet langer bepaald door een verblijfplaats
in een rechtsgebied dat optimaal functioneert. Aangeboren talent en de
wil om dit verder te ontwikkelen, worden steeds eerlijker beoordeeld. De
voorgaande jurisdictievoordelen, die in het industriële tijdperk de
kloof tussen rijke en arme economieën vergrootten, zullen ingrijpend
veranderen.

\subsection{Hogere rendementen in arme
gebieden}\label{hogere-rendementen-in-arme-gebieden}

De belemmeringen die regeringen in armere regio's opwerpen voor het
functioneren van vrije markten, verminderen sterk naarmate de
cybereconomie op gang komt. Daardoor leveren kapitaal en vaardigheden --
die schaars zijn -- in veel van de huidige arme regio's hogere
rendementen op, precies zoals ontwikkelingsdeskundigen in de jaren
vijftig voorspelden. Bovendien worden kapitaal en expertise veel
makkelijker ingevoerd. Opkomende economieën hoeven zich niet meer zo
sterk te laten binden aan lokale productiefactoren als in het
industriële tijdperk. Hun toenemend vermogen om op afstand kapitaal en
kennis aan te trekken, stimuleert hogere groeicijfers, ongeacht of
overheidsinstanties efficiënter worden of beter eigendomsrechten
beschermen. Door hun beperkte controle over de cybersfeer kunnen zwakke
regeringen immers minder voorkomen dat mensen in hun rechtsgebied
genieten van economische vrijheid.

\subsection{Positieve versterking}\label{positieve-versterking}

In de nieuwe cybereconomie zorgt de bijna volledige draagbaarheid van
informatietechnologie ervoor dat men niet langer kan hamsteren op de in
het industriële tijdperk vergaarde jurisdictievoordelen. De toenemende
concurrentie tussen steeds meer jurisdicties draait om nieuwe vormen van
lokaal voordeel. Soevereiniteit wordt commercieel benut in plaats van op
roofzuchtige wijze ingezet. Door de druk van de concurrentie worden
overheden gedwongen beleid te voeren dat zich richt op de mensen in hun
klantenkring die het meeste bijdragen aan het economische welzijn, en
niet op degenen die weinig of zelfs negatief bijdragen.

Dit betekent een ingrijpende verschuiving ten opzichte van de gangbare
praktijk in de twintigste eeuw. Volgens de ideologie van de natiestaat
kon en moest men het leven positief inrichten door ongewenste uitkomsten
te compenseren met subsidies en gewenste uitkomsten te bestraffen met
zware belastingen. Armoede werd als ongewenst ervaren, waardoor men
subsidies aan de armen verstrekte. Rijkdom werd juist nagestreefd, wat
leidde tot hoge, bestraffende belastingen voor de rijken om het leven
`eerlijker' te maken.

Omdat men deze beleidsaanpak baseerde op een onbetwistbaar megapolitiek
fundament, trok men nauwelijks in overweging welke perverse gevolgen het
subsidiëren van disfunctie met zich meebracht. Daarbij hield men
nauwelijks rekening met de inzet, het harde werken en de vindingrijkheid
die nodig waren om rijkdom te vergaren, rijkdom die vervolgens werd
herverdeeld. De uitkomsten werden gemeten in termen van aanspraken. De
politieke visie van de twintigste eeuw veronderstelde dat uitkomsten
`eerlijk' zijn als ze gelijk zijn.

\subsection{Het nieuwe paradigma}\label{het-nieuwe-paradigma}

De nieuwe megapolitieke omstandigheden in de eenentwintigste eeuw laten
toe dat we marktexperimenten inzetten om resultaten te sturen op
terreinen die voorheen door de politiek werden beheerst. Het
marktparadigma gaat ervan uit dat je betere uitkomsten behaalt door
gewenst gedrag te belonen en ongewenst gedrag te bestraffen. Arm zijn
wordt als onwenselijk beschouwd, terwijl rijkdom juist nagestreefd
wordt. Hierdoor moeten prikkels de creatie van rijkdom stimuleren en
mensen aansporen te betalen voor de middelen die zij gebruiken. Het
leven voelt `eerlijker' als mensen een groter deel van hun inkomen mogen
behouden.

Deze visie zal in het nieuwe millennium vaker aan de orde komen dan in
de eeuw die nu ten einde loopt. Bovendien zal ze overtuigender blijken
dan ooit, omdat ze rust op grootschalige politieke grondslagen. Kapitaal
in het informatietijdperk verplaatst zich met de minuut steeds sneller.
De kans op een hoog inkomen hangt niet langer samen met het wonen op
specifieke locaties, zoals vroeger toen de meeste rijkdom voortkwam uit
de exploitatie van natuurlijke hulpbronnen. Met elke dag die
voorbijgaat, wordt het voor mensen die gebruikmaken van uiterst
draagbare informatietechnologie eenvoudiger om activa te creëren die
veel minder vatbaar zijn voor de invloed van geweld dan welke vorm van
rijkdom dan ooit.

Willekeurige politieke regelgeving die kosten met zich meebrengt zonder
evenredige marktvoordelen op te leveren, blijkt al snel onhoudbaar te
zijn. Sterke concurrentiekrachten zorgen ervoor dat de prijzen van
goederen, diensten, arbeid en kapitaal wereldwijd op elkaar worden
afgestemd. Overheden krijgen minder speelruimte om willekeurige
beleidsmaatregelen op te leggen dan ze voorheen gewend waren. Elke
overheid die voor een bepaalde activiteit strengere regels probeert op
te leggen dan andere soevereine staten, jaagt die activiteit
vanzelfsprekend naar andere jurisdicties. In sommige gevallen kan het
wegjagen van bepaalde activiteiten juist een marktvoordeel opleveren,
waardoor die jurisdicties populairder en welvarender worden. In dit
opzicht kun je sommige regels vergelijken met de huisregels die
eigenaren van een hotelketen hanteren. Als zij bijvoorbeeld verbieden
dat gasten blootsvoets lopen of in de lobby roken, verliezen zij
ongetwijfeld een deel van hun clientèle. Toch betekent het mislopen van
deze klanten niet per se dat de jurisdictie beroofd wordt van haar
klanten of dat haar totale inkomsten dalen. Goed beschaafde niet-rokers
kunnen hierdoor juist meer betalen. Ook maatregelen die het duur of
zelfs onmogelijk maken om een renderingfabriek in een bepaalde
jurisdictie te exploiteren, kunnen ervoor zorgen dat de verwerking naar
andere regio's verhuist, zonder dat de hele jurisdictie van inkomsten
wordt beroofd.

Deze voorbeelden laten zien dat regelgeving in sommige gevallen een
positieve marktwaarde kan creëren in plaats van een negatieve, zeker in
een wereld waarin het aantal jurisdicties snel groeit. Voorschriften die
hoge standaarden handhaven op het gebied van volksgezondheid, schone
lucht en schoon water, worden in veel regio's zeer gewaardeerd. Ook
andere, soms meer bijzondere, regelgevingen en convenanten -- zoals die
door vastgoedontwikkelaars of hotels die zich op specifieke
marktsegmenten richten -- zullen veel aanzien genieten.

\subsection{Geen douanehuis in
cyberspace}\label{geen-douanehuis-in-cyberspace}

Wij verwachten dat de commercialisering van soevereiniteit al snel zal
leiden tot de decentralisatie van talrijke grote territoriale
entiteiten. Het feit dat informatietechnologie niet gebonden is aan
grenscontroles -- controles die nog steeds de handel in vervaardigde
goederen en landbouwproducten kunnen beperken -- heeft belangrijke
gevolgen. Dit houdt in dat protectionisme op den duur minder effectief
wordt, nu de informatiestroom de fysieke producten verdrijft die voor
het scheppen van welvaart nodig waren. Ook worden kleinere regio's
steeds minder afhankelijk van omvangrijke politieke jurisdicties om
toegang tot markten te garanderen waar zij inkomsten kunnen genereren.

Informatietechnologie confronteert mensen in voorheen beschermde
dienstensectoren met buitenlandse concurrentie. Twintig jaar geleden
moest een bedrijf in Toronto, wanneer het een boekhouder wilde inhuren,
er nauwlettend op toezien dat deze persoon zich fysiek in Toronto of in
een nabijgelegen gemeente binnen redelijke pendeltijd bevond. In het
informatietijdperk kan een boekhouder in Boedapest of in Bangalore,
India, de klus klaren en alle benodigde gegevens in versleutelde vorm
downloaden via het internet. Dankzij directe satellietverbindingen ligt
elk deel van de wereld slechts een ogenblik verwijderd, zodat
communicatie via modem en fax vrijwel direct verloopt. Een onderneming
die op zoek is naar aandelenanalisten kan in India er er zevenentwintig
aannemen voor de prijs van één analist op Wall Street. Omdat
informatietechnologie volgens de wet van Moore elke achttien maanden met
een orde van grootte verbetert, komen steeds meer werknemers in de
dienstensector in aanraking met prijsconcurrentie die politici feitelijk
niet kunnen tegenhouden. Deze concurrentie treft uiteindelijk ook de
geleerde beroepen even hard als boekhouders. Digitale advocaten en
cyberdokters zullen in de informatie-economie in rap tempo opkomen.

\subsection{Natiestaten op de
doodsklok}\label{natiestaten-op-de-doodsklok}

Nu de economische voordelen, die voorheen binnen natiestaten werden
geconcentreerd, verdwijnen, zullen natiestaten uiteindelijk instorten
onder hun zware lasten. Dat alle natiestaten op de doodsklok staan,
betekent echter niet dat zij allemaal gelijktijdig zullen vergaan. De
druk tot decentralisatie wordt vooral voelbaar in grote politieke
eenheden waar de inkomens van de meerderheid stagneren of dalen.
Regeringsgebieden in Latijns-Amerika en Azië, waar het inkomen per hoofd
van de bevolking snel toeneemt, kunnen generaties lang overeind blijven
-- of totdat het levenslange inkomensperspectief daar gelijk is aan dat
in de voormalige, welvarende industriële landen. Op dat moment zijn er
geen simpele kostenbesparingen meer te behalen, waardoor het groeibeleid
aanzienlijk uitdagender wordt.

Wij vermoeden tevens dat natiestaten met één overheersende metropool
langer samenhangend blijven dan landen met meerdere grote steden, omdat
die juist meerdere centra van belangen met bijbehorende achterlanden met
zich meebrengen.

Een andere drijfveer voor decentralisatie vormt de enorme schuldenlast
van de centrale overheid. De drie welvarende industriële landen met de
hoogste relatieve schulden -- Canada, België en Italië -- zijn geen
toeval; het betreft natiestaten waar sterke separatistische bewegingen
actief zijn. Alle drie kampen met aanhoudende begrotingstekorten en
hebben een nationale schuld die meer dan 100 procent van het BBP
bedraagt. Naarmate de schuldenlast toeneemt, groeit ook de
aantrekkingskracht van separatistische bewegingen.

In Italië heeft de Lega Nord zich ontwikkeld tot een dynamische en
populaire regionale politieke beweging. Haar programma rust op een
simpele rekenkundige observatie: Noord-Italië -- of Padonia -- zou
rijker zijn dan Zwitserland als grote delen van zijn inkomen niet werden
afgetapt om Rome en het armere zuiden te subsidiëren. De Lega Nord stelt
dan ook een voor de hand liggende oplossing voor: afscheiding van Italië
om zo de schrijnende effecten van samengestelde rente te ontlopen.

Ook in België -- waar de nationale schuld meer dan 130 procent van het
BBP betreft -- bewegen Vlamingen en Walen zich als een vijandig koppel
richting afscheiding. Een groeiende minderheid onder de Vlamingen stelt
dat zij de Walen oneerlijk subsidiëren en dat zij hun economische
positie kunnen verbeteren door België in tweeën te splitsen.

In Canada speelt de situatie anders. Frans Canada, de voornaamste regio
die zich nu inzet voor afscheiding, werd historisch gesubsidieerd door
het Engelstalige deel van het land. Nu de federale schuld en het
begrotingstekort blijven stijgen, wordt in Quebec steeds duidelijker dat
deze vorm van inkomensherverdeling zal afnemen. De Bloc Québécois
overweegt daarom een optie die een decennium geleden nog ondenkbaar was:
het netto-inkomen verhogen door de Canadese federale belasting af te
schaffen. Separatistische leiders stellen bovendien dat Quebec zich van
Canada zou moeten losmaken zonder evenredig de federale schuld over te
nemen.

Engelstalige Canadezen verzetten zich tegen dit standpunt en tonen hun
afkeer, omdat zij zich terdege bewust zijn van de enorme geldstromen die
in de loop der jaren naar Quebec zijn overgeheveld. Desalniettemin
blijft de aantrekkingskracht van het Parti Québécois sterk en lijkt het
slechts een kwestie van tijd voordat een afscheidingsreferendum Canada
doet uiteenvallen. Een vergelijkbaar lot wacht ook andere natiestaten
wanneer hun financiële situatie verslechtert.

Een andere factor die slecht nieuws belooft voor het langetermijnbestaan
van Canada, is dat het een dunbevolkt land is met een uitgestrekte
infrastructuur uit de industriële tijd, die nog onderhouden moet worden.
De overgang naar het informatietijdperk zorgt er onvermijdelijk voor dat
fysieke infrastructuur in waarde daalt. Nu thuiswerkers
fabrieksarbeiders en kantoormedewerkers vervangen, verliest de aanleg en
goede instandhouding van snelwegen en andere belangrijke transportroutes
aan relevantie. Terwijl fiscale crises van alle kanten toeslaan, keren
steeds meer groepen in het Canadese leven zich terug tot een exclusief
achttiende-eeuwse visie op de financiering van publieke goederen, zoals
Adam Smith bepleitte in \emph{The Wealth of Nations}:

\begin{quote}
Indien de straten van Londen verlicht en betegeld zouden worden ten
laste van de (nationale) schatkist, is er dan enige kans dat zij zo goed
verlicht en betegeld zouden zijn als nu, of zelfs tegen zo'n geringe
kost? Bovendien, in plaats van dat de kosten worden opgehaald via een
lokale belasting op de inwoners van iedere specifieke straat, parochie
of wijk in Londen, zouden deze in dat geval uit de algemene
staatsinkomsten worden bekostigd en bijgevolg via een belasting op alle
inwoners van het koninkrijk worden opgehaald, van wie het grootste deel
geen enkel voordeel haalt uit de verlichting en betegeling van de
Londense straten.\footnote{Adam Smith, \emph{The Wealth of Nations},
  p.~724. Dit punt werd gesuggereerd door een argument van Edwin G. West
  in \emph{Adam Smith and Modern Economics} (Aldershot, Engeland: Edward
  Elgar Publishing, 1990), pp.~88--89.}
\end{quote}

Vervang `Londen' door `Toronto' en je bevindt je in een vergelijking die
het denken van velen in Alberta en British Columbia ingrijpend zal
beïnvloeden. De logica achter devolutie werkt immers aanstekelijk.

When Canada breaks apart, this will lead to a marked increase in
secessionist activity in the Pacific Northwest of the United States.
Residents of Alaska, Washington, Oregon, Idaho, and Montana would find
themselves at a distinct disadvantage in competition with Alberta and
British Columbia as independent sovereignties.

Als Canada uit elkaar valt, zorgt dat onvermijdelijk voor een
aanzienlijke toename van afscheidingsbewegingen in het pacifisch
noordwesten van de Verenigde Staten. Inwoners van Alaska, Washington,
Oregon, Idaho en Montana komen als onafhankelijke staten onmiskenbaar in
het nadeel te staan ten opzichte van Alberta en British Columbia.

\section{Na de natiestaat}\label{na-de-natiestaat}

Na de natiestaat

In plaats van natiestaten zie je aanvankelijk kleinere rechtsgebieden op
provinciaal niveau en uiteindelijk nog kleinere soevereiniteiten --
enclaves van diverse aard die doen denken aan middeleeuwse stadstaten,
omgeven door hun achterland. Hoewel dit vreemd kan overkomen op mensen
die van jongs af aan geleerd hebben dat politiek allesbepalend is,
kiezen de leiders van deze nieuwe ministaten er in veel gevallen voor om
hun beleid te laten vormgeven door ondernemerschap en strategische
positionering, in plaats van door politieke conflicten. Deze
gefragmenteerde soevereiniteiten bedienen een breed scala aan smaken,
net zoals hotels en restaurants dat doen, door in hun openbare ruimten
specifieke regels vast te stellen die naadloos aansluiten bij de wensen
van de marktsegmenten waaruit zij hun klanten halen. Dat betekent echter
niet dat er geen bijzondere uitdagingen ontstaan bij het organiseren van
bescherming op nomadische basis; die pakken we in het volgende hoofdstuk
aan.

\begin{quote}
``Stadslucht brengt vrijheid.'' - MIDDELEEUWSE GEWOONHEID
\end{quote}

\subsection{Non citizens of the Pale}\label{non-citizens-of-the-pale}

Niet-burgers van het Bleke

Ondanks deze moeilijkheden weet de menselijke vindingrijkheid meestal
toch een manier te vinden om instellingen op te richten die winstgevende
kansen benutten, zelfs als de vraag afkomstig is van mensen die
nauwelijks kunnen betalen. Als de potentiële klanten tot de rijkste ter
wereld behoren, zou die drang nog sterker moeten zijn. Exit -- oftewel
`stemmen met je voeten' -- blijft altijd een optie wanneer achterhaalde
producten, organisaties of zelfs regeringen hun aantrekkingskracht
verliezen en er weinig vooruitzicht is op onmiddellijke verbetering.

Neem bijvoorbeeld de opkomst van middeleeuwse steden, waar lijfeigenen
die aan de feodale onderdrukking ontsnapten, een toevlucht zochten. Deze
steden kunnen een vergelijkbare rol vervullen als nieuwe rechtsgebieden
die de uittocht vanuit natiestaten opvangen. Het opnemen van
vreemdelingen die aan een heer ontsnapten en als `burgers van het Pale'
werden erkend, tartte de geldende conventies van het feodale recht en de
bisschoppelijke autoriteit. Toch bleek deze alternatieve aanpak voor
degenen die er gebruik van maakten over het algemeen succesvol en
leverde zij een belangrijke bijdrage aan het afnemen van de macht van
het feodalisme. Zoals de middeleeuwse historicus Fritz Rorig het
verwoordde: een lijfeigenen van een seculiere heer zou `na een jaar en
een dag een vrije burger van de stad' worden.\footnote{Fritz Rorig,
  \emph{The Medieval Town} (Berkeley: University of California Press,
  1967), p.~28.} Het is dan ook aannemelijk dat er nieuwe institutionele
toevluchtsoorden komen, gebaseerd op `nieuwe juridische principes', die
staatsburgers fiscale bescherming bieden -- vergelijkbaar met de
schuilplaats die middeleeuwse steden boden aan onderdanen die in de
schaduw van hun muren leefden.

De econoom Albert O. Hirschman, die de theoretische nuances van `stemmen
met je voeten' onderzocht in \emph{Exit, Voice, and Loyalty} (eerste
publicatie 1969), voorzag dat technologische vooruitgang de kans
vergroot om staatsuitstroom als strategie te hanteren wanneer staten in
verval raken. Hij stelde: `Pas wanneer landen door de vooruitgang in
communicatie en algehele modernisering steeds meer op elkaar gaan
lijken, zal het gevaar van vroegtijdige en buitensporige uittredingen op
de proppen komen \ldots{}'\footnote{Albert O. Hirschman, `Exit, Voice,
  and Loyalty' (Cambridge: \emph{Harvard University Press}, 1969),
  p.~81.} Precies dat zien we nu gebeuren. Informatietechnologie
verkleint in rap tempo de verschillen tussen rechtsgebieden, waardoor
uittreding een veel aantrekkelijkere keuze wordt. Volgens Hirschmans
terminologie worden `vroegtijdige en buitensporige uittredingen'
beoordeeld vanuit het oogpunt van wat optimaal is voor de staat die
wordt verlaten. Zeker meenden de heren in middeleeuws Europa dat hun
lijfeigenen al te vroeg en in grote getale richting steden uittreden,
waar zij vrijheden genoten.

Als we terugkeren naar ons eerdere voorbeeld, is het niet zo vergezocht
te denken dat er ministaten zullen ontstaan die onderdak bieden aan
ballingen die vluchten voor stervende natiestaten. Deze soevereine
eenheden zullen met elkaar concurreren op het terrein van ballingschap.
Sommigen, mogelijk aan de westkust van Noord-Amerika, spelen in op
mensen die niet roken en geen enkele tolerantie hebben voor passief
roken. Duidelijk zal een dergelijk regime bij rokers niet in de smaak
vallen, want wetten die hun gewoonte verbieden, komen voor hen over als
een willekeurig opgelegde maatregel.

In het industriële tijdperk van de massale politiek werden dergelijke
meningsverschillen beslecht via politieke campagnes, die er uiteindelijk
toe leidden dat de ene of de andere groep zich neerlegde aan de wensen
van de machtigere partij. Toch is het volstrekt overbodig dat geschillen
over wederzijds exclusieve keuzes op een wijze worden opgelost die
inhoudt dat de voorkeuren van grote groepen mensen ondergeschikt worden
gemaakt.

Sommige mensen kiezen bewust voor foie gras, anderen voor hotdogs en
weer anderen voor tofu. Normaal gesproken maken zij zich geen zorgen
over elkaars dieetvoorkeuren, omdat hun culinaire keuzes onafhankelijk
van elkaar zijn -- niemand verplicht iedereen hetzelfde te eten.
Megapolitieke omstandigheden dwongen het echter in het industriële
tijdperk om collectieve en zelfs particuliere goederen, die door
overheden werden geleverd, breed te gebruiken. Waarom? Omdat er enorme
economische voordelen te behalen waren bij grootschalige samenwerking.
Het was daarom onpraktisch om uitgestrekte rechtsgebieden op te splitsen
in enclaves waar iedereen zijn eigen gang kon gaan, zelfs niet in
belangrijke zaken. De door Adam Smith bepleitte exclusieve benadering
van het leveren van publieke goederen werkt veel eenvoudiger wanneer het
aantal rechtsgebieden met een factor tien of zelfs honderd toeneemt. In
het informatietijdperk zullen steeds meer soevereiniteiten opgaan in
kleine enclaves in plaats van in continentale rijken. Sommige van deze
enclaves zullen bestaan uit Noord-Amerikaanse inheemse stammen die, net
zoals zij nu al aanspraak maken op het recht om gokcasino's te
exploiteren of te vissen buiten de geldende beperkingen, ook aanspraak
maken op belastingheffing over hun reservaten en grondgebied.

Doordat informatietechnologie veel nadelen wegneemt van de
decentralisatie van handelsgebieden, gaan de nieuwe soevereiniteiten
waarschijnlijk meer opereren volgens de principes van clubs of
affiniteitsgroepen dan volgens de uitgangspunten van territoriale
natiestaten. Net zoals het niet noodzakelijk is dat elke potentiële
klant dezelfde smaak in kleding heeft of dezelfde televisieprogramma's
kijkt, is het minder van belang dat iedereen overeenstemming bereikt
over de affiniteitspunten die de bestuursstijl van gefragmenteerde
soevereiniteiten bepalen.

De brede diversiteit aan smaakvoorkeuren leidt tot een grote variatie in
vormen van gefragmenteerde soevereiniteit, vergelijkbaar met de steeds
grotere keuzemogelijkheden in kledingstijlen of televisieprogramma's.
Sommige microstaten slaan zelfs de handen ineen -- vergelijkbaar met
hotelketens binnen franchises -- of bundelen hun krachten om voordelen
te behalen op het gebied van politietaken en overige overheidsdiensten.
Wie schone straten waardeert en het irritant vindt als er kauwgom onder
de tafel ligt, zal Singapore aantrekkelijk vinden. Fans van \emph{Beavis
en Butthead} daarentegen niet. Wie van een bruisend nachtleven houdt,
kiest waarschijnlijk voor Macao, Panama of een vergelijkbare locatie.
Wie zich in een bepaald rechtsgebied niet thuisvoelt, kan in andere
gebieden rekenen op een warm onthaal. Hoewel Salt~Lake~City mogelijk
rookvrij is, zal de nieuwe stadstaat in Havana -- eventueel hernoemd tot
Monte Cristo -- hoogstwaarschijnlijk gehuld zijn in een dikke
sigarenrookmist.

\begin{quote}
`Het betekent dat alle monopolies, hiërarchieën, piramides en
machtsnetwerken van de industriële samenleving uiteen zullen vallen door
de voortdurende druk om intelligentie naar de randen van alle netwerken
te verspreiden. Bovenal zal de wet van Moore de essentiële concentratie,
de fundamentele fysieke samenklontering van macht in Amerika van
vandaag, omverwerpen: de grote stad -- dat grote geheel van industriële
steden dat nu steunt op ondersteuningssystemen -- zo'n 360 miljard aan
directe subsidies van de rest van ons ieder jaar. Grote steden zijn
overgebleven bagage uit het industriële tijdperk.'\footnote{Tom Peters
  en George Gilder, `City vs.~country: debat van Tom Peters en George
  Gilder over de impact van technologie op locatie', \emph{Forbes},
  februari 1995.} - GEORGE GILDER
\end{quote}

Een merkwaardige ironie rondom de heropleving van micro-soevereiniteiten
of `stadstaten' is dat dit mogelijk samenvalt met het leegtrekken van
vele steden. De grote stad is immers grotendeels een bijproduct van het
industrialisme in het Westen. Ze ontstond doordat het fabrieksysteem
schaalvoordelen bood bij de productie van goederen die sterk afhankelijk
waren van natuurlijke grondstoffen.

Toen de negentiende eeuw begon, werden steden met meer dan 100.000
inwoners als enorm gezien en, buiten Azië waar de bevolkingscijfers vaak
ter discussie stonden, waren er geen steden met boven een miljoen
inwoners. In 1800 was Philadelphia de grootste stad van de Verenigde
Staten met 69.403 inwoners, terwijl New York slechts 60.489 telde en
Baltimore -- de op twee na grootste stad -- 26.114 inwoners
kende.\footnote{Weber, op. cit., p.~21.} De steden die later als grote
Europese metropolen zouden doorgroeien, hadden oorspronkelijk populaties
die volgens de maatstaven van de twintigste eeuw betrekkelijk klein
waren. Londen, met 864.845 inwoners, werd waarschijnlijk als 's werelds
grootste stad beschouwd, terwijl Parijs met 547.756 inwoners de enige
andere Europese stad was die in 1801 meer dan een half miljoen bewoners
huisvestte.\footnote{Ibid., p.~46 voor Londen, p.~73 voor Parijs.}
Lissabon telde 350.000 inwoners,\footnote{Ibid., p.~120.} Wenen
252.000,\footnote{Ibid., p.~95.} en Berlijn had in 1819 amper iets meer
dan 200.000 bewoners.\footnote{Ibid., p.~84.} Madrid herbergde 156.670
inwoners, Brussel kende in 1802 66.297 en Boedapest slechts
61.000.\footnote{Ibid., p.~119.}\footnote{Ibid., p.~101.}

Men gaat er vaak van uit dat de groei van grote steden simpelweg een
direct gevolg is van bevolkingsgroei, maar dat hoeft niet te kloppen.
Stel je voor dat iedereen op aarde in Texas zou wonen, waarbij ieder
gezin in een vrijstaand huis met tuin leeft; toch zou Texas nog
ruimschoots voldoende ruimte bieden. Zoals Adna Weber in zijn klassieke
studie \emph{The Growth of Cities in the Nineteenth Century} betoogde,
verklaart bevolkingsgroei op zich niet waarom mensen liever in
stedelijke centra wonen in plaats van verspreid over het platteland. In
1890 had Bengalen ongeveer dezelfde bevolkingsdichtheid als Engeland,
maar slechts 4,8 procent van Bengalen's bevolking leefde in steden,
tegenover 61,7 procent in Engeland.\footnote{Ibid., p.~5.}

Vroeger beschermden stadswallen steden door rovers en de lagere klassen
buiten te houden. De opkomst van industriële werkgelegenheid in de
negentiende en twintigste eeuw leidde tot de bloei van grote steden.
Tegenwoordig blijkt dat grote steden bijzonder kwetsbaar worden nu het
industrialisme afneemt. Een treffend voorbeeld is Detroit, dat in de
midden twintigste eeuw uitgroeide tot een toonaangevende industriële
stad waar een aanzienlijk deel van de wereldwijde productie samenkwam.
Nu is Detroit slechts een uitgehold omhulsel, geteisterd door
criminaliteit en wanorde. In het centrum van Detroit laten talrijke
straten zien dat branden en sloopacties vervallen gebouwen hebben doen
instorten, waardoor men het idee krijgt dat bomwerpers de stad tijdens
de Tweede Wereldoorlog herhaaldelijk hebben getroffen.

Detroit laat ons zien dat veel industriële steden hun levensvatbaarheid
kwijtraken. Nu informatie en ideeën waardevoller worden dan het
vervaardigen van goederen uit natuurlijke hulpbronnen, raken deze steden
uit balans. Vaak is een grote stad simpelweg te omvangrijk geworden om
in eigen beheer te blijven functioneren. Om een metropool draaiende te
houden, moeten talloze ondersteunende systemen op grotere schaal
efficiënt werken; de concentratie van miljoenen mensen vergroot immers
de kans op criminaliteit, sabotage en willekeurig geweld. Tijdens het
industriële tijdperk weegden de schaalvoordelen in de productie de
kosten voor het handhaven van de orde vaak ruimschoots op.

In het informatietijdperk overleven alleen steden die hun
onderhoudskosten weten te compenseren door een hoge levenskwaliteit te
bieden, want inwoners buiten de stad hoeven deze metropolen dan niet
langer financieel te ondersteunen. Een goede graadmeter voor de
levensvatbaarheid van een stad is of de mensen in het centrum
welvarender zijn dan degenen in de randwijken. Buenos Aires, Londen en
Parijs blijven aantrekkelijke plekken om te wonen en zaken te doen, lang
nadat het laatste goede restaurant in South Bend, Louisville en
Philadelphia zijn deuren heeft gesloten.

\subsection{Country states}\label{country-states}

Sommige stadsstaten blijken uiteindelijk slechts enclaves te zijn,
zonder omliggende steden. Misschien kun je ze beter bestempelen als
dorpsstaten of landstaten.

Ook waarderen mensen natuurlijke hulpbronnen op uiteenlopende manieren.
Als je overal kunt ondernemen, kies je er mogelijk voor je te vestigen
op een aangename plek waar je diep kunt ademen zonder blootgesteld te
worden aan schadelijke, kankerverwekkende vervuiling. Dankzij
communicatietechnologieën die taalbarrières terugdringen, wordt het
steeds gemakkelijker om vrijwel overal te wonen, mits de omgeving
aantrekkelijk is. Dunbevolkte gebieden met een gematigd klimaat en een
overvloed aan vruchtbaar land per inwoner -- zoals Nieuw-Zeeland en
Argentinië -- hebben bovendien een comparatief voordeel doordat zij hoge
standaarden in de volksgezondheid hanteren en lage productiekosten voor
voedsel en hernieuwbare producten realiseren. Naarmate de
levensstandaarden van miljarden mensen in Oost-Azië en Latijns-Amerika
stijgen, neemt de vraag naar dergelijke producten toe.

\subsection{Het
inequivalentietheorema}\label{het-inequivalentietheorema}

Veel veronderstellingen over menselijk gedrag, die economen hanteren,
zijn gebaseerd op de overheersende invloed van de plaats. Een treffend
voorbeeld is Ricardo's `Equivalence Theorem', dat stelt dat burgers in
een land met forse begrotingstekorten hun verwachtingen aanpassen, omdat
zij ervan uitgaan in de toekomst hogere belastingen te moeten betalen om
de staatsschuld af te lossen. Zo blijkt er een equivalentie te bestaan
tussen het financieren van overheidsuitgaven via belastingheffing en via
schulden. Tenminste, in het begin van de negentiende eeuw hield die
equivalentie stand, tijdens Ricardo's tijd. In het informatietijdperk
reageert de rationele mens daarentegen niet door zijn spaarratio te
verhogen op de vooruitzichten van hogere belastingen; hij verhuist of
voert zijn transacties in een ander rechtsgebied uit. Net zoals
producenten op zoek gaan naar leveranciers met de laagste kosten, gaan
zij des te gemotiveerder op zoek naar alternatieve aanbieders van
bescherming. De voordelen daarvan wegen ruimschoots op tegen de
besparingen die behaald kunnen worden door over te stappen op een nieuwe
leverancier van plastic buizen. Het te verwachten resultaat is dat
zelfstandige burgers en andere rationeel denkenden uiteindelijk vluchten
uit rechtsgebieden met hoge niet-gefinancierde verplichtingen.

Overheden die minimale verplichtingen opleggen en lage lasten rekenen,
trekken steeds meer bedrijven aan en vormen daardoor de
voorkeurslocaties voor het creëren van welvaart in het
informatietijdperk. Dit schept aantrekkelijkere vooruitzichten voor
ondernemerschap in regio's waar de schuldenlast laag is en de
overheidsstructuren al hervormd zijn, bijvoorbeeld in Nieuw-Zeeland,
Argentinië, Chili, Peru, Singapore en andere delen van Azië en
Latijns-Amerika. Bovendien bieden deze gebieden betere uitgangspunten
voor zakendoen dan de dure, niet-gereformeerde economieën in
Noord-Amerika en West-Europa.

\subsection{De afname van lokale
prijsafwijkingen}\label{de-afname-van-lokale-prijsafwijkingen}

Zeer lage informatiekosten doen de meeste lokale prijsvoordelen
verdwijnen. Kopers doorzoeken immers talloze verkooppunten om de laagste
prijzen voor verhandelbare goederen te vinden en schakelen daarbij
externe diensten in om over landsgrenzen heen te winkelen. Hierdoor valt
het voor mensen veel eenvoudiger om de eigenschappen van complexe
producten, zoals verzekeringen, met elkaar te vergelijken. Daarbij
omzeilt dit de handelsbeperkingen die lokale vergunningprocedures met
zich meebrengen. Naar verwachting zorgen extra informatie en
intensievere concurrentie ervoor dat lokale prijsafwijkingen verdwijnen,
wat resulteert in lagere winstmarges.

\section{Nieuwe organisatorische
imperatieven}\label{nieuwe-organisatorische-imperatieven}

De cybereconomie verschilt radicaal van de industriële economie in de
wijze waarop de betrokkenen met elkaar omgaan. Dankzij
informatietechnologie verdwijnen veel langetermijnvoordelen die
bedrijven traditioneel haalden uit hoge transactiekosten en
informatiekosten. Het informatietijdperk wordt het tijdperk van de
`virtuele onderneming'.

Veel analisten, experts op het gebied van informatietechnologie,
realiseerden zich niet dat deze technologie de logica van economische
organisatie ingrijpend zou veranderen. De nieuwste technologie
doorbreekt immers niet alleen grenzen en barrières, maar zet ook de
interne rekenkosten volledig op scherp.

Zelfs de weinige bedrijven die dankzij innovaties in informatie- en
communicatietechnologie niet direct te maken krijgen met toegenomen
buitenlandse concurrentie, krijgen nieuwe organisatorische uitdagingen.
Doordat informatie- en transactiekosten razendsnel dalen, krimpen de
schaalvoordelen fors, waardoor de prikkels die in de industriële periode
leidden tot grote, langdurige ondernemingen en loopbaangerichte
werkgelegenheid, verdwijnen.

\subsection{Waarom bedrijven?}\label{waarom-bedrijven}

Klassieke economen als Adam Smith besteedden nauwelijks aandacht aan de
vraag welke omvang bedrijven idealiter zouden moeten hebben. Ze gingen
niet in op wat de optimale grootte bepaalt, waarom ondernemingen de
vormen aannemen die zij aannemen, of zelfs waarom ze überhaupt bestaan.
Waarom kiezen ondernemers ervoor medewerkers in dienst te nemen in
plaats van elke benodigde taak via een veiling aan onafhankelijke
aannemers uit te besteden? De Nobelprijswinnaar Ronald Coase stippelde
een nieuwe koers in de economie uit door enkele van deze fundamentele
vragen te adresseren. Zijn bevindingen tonen duidelijk de revolutionaire
impact van informatietechnologie op de structuur van bedrijven. Coase
betoogde dat ondernemingen een efficiënte oplossing bieden om het gebrek
aan informatie en de hoge transactiekosten te overbruggen.\footnote{Zie
  Ronald Coase, \emph{The Nature of the Firm}, herdrukt in Louis
  Putterman en Randall S. Kroszner (red.), \emph{The Economic Nature of
  the Firm: A Reader}, 2e editie (Cambridge: \emph{Cambridge University
  Press}, 1996), pp.~89--104.}

\subsection{Informatie- en
transactiekosten}\label{informatie--en-transactiekosten}

Om te begrijpen waarom, stel je eens voor met welke obstakels je te
maken krijgt als je een assemblagelijn uit het industriële tijdperk in
gang wilt zetten zonder dat één onderneming de activiteiten coördineert.
In principe had men een auto kunnen bouwen zonder dat één centrale
organisatie de productie overzag. Econoom Oliver Williamson -- samen met
Coase -- werd een pionier in de ontwikkeling van de bedrijfstheorie.
Williamson onderscheidde zes verschillende werk- en controlemethoden. Zo
kent hij de `ondernemende modus', waarbij een specialist iedere werkplek
bezit en bestuurt.\footnote{Geciteerd door West, op. cit., p.~58; zie
  ook Oliver E. Williamson, \emph{The Organization of Work: A
  Comparative Institutional Assessment}, \emph{Journal of Economic
  Behaviour and Organisation}, vol.~1, nr. 1.} Een andere methode noemt
hij de `gefedereerde werkstations', waarbij elke werknemer een
tussentijds product van de ene naar de volgende fase
doorgeeft.\footnote{Geciteerd door West, op. cit., p.~59; zie eveneens
  Williamson, op. cit.} Er is geen technische reden waarom de duizenden
werknemers niet vervangen hadden kunnen worden door een groep
zelfstandige aannemers die elk een werkruimte op de fabrieksvloer huren,
meedingen naar onderdelen en aanbieden om bijvoorbeeld de as te monteren
of de spatborden op het chassis te lassen. Toch kom je nergens een
automobielfabriek uit die tijd tegen die geheel georganiseerd en gerund
wordt door onafhankelijke aannemers.

\subsection{Coördinatieproblemen}\label{couxf6rdinatieproblemen}

Het runnen van een industriële faciliteit zonder de coördinatievoordelen
van één centrale organisatie zou vrijwel alle schaalvoordelen doen
vervagen die bij grootschalige exploitatie mogelijk zijn. Het
coördineren van een wirwar aan kleine bedrijven brengt enorme
transactionele problemen met zich mee, waardoor de assemblagelijn
praktisch onbruikbaar wordt. Om het systeem überhaupt werkend te houden,
moet je voortdurend onderhandelen met de diverse aannemers. In plaats
van je op de productie te concentreren, zouden zij hun tijd moeten
besteden aan het bepalen van de prijzen voor onderdelen en het
vastleggen van de voorwaarden voor hun voortdurend wisselende
samenwerkingen. Zelfs het louter toezicht houden op de productie bleek
een lastige opgave.

\subsection{De bevoegdheid om op te
treden}\label{de-bevoegdheid-om-op-te-treden}

Als zo'n netwerk van onafhankelijke organisaties al moeite heeft om een
auto in elkaar te zetten, dan zou het ontwikkelen en herontwerpen van
modellen een regelrechte nachtmerrie vormen. Stel je eens voor met welke
problemen een ontwerper wordt geconfronteerd wanneer hij de honderden
zelfstandige aannemers moet overtuigen van de noodzakelijke wijzigingen
om een nieuw model te introduceren. In de praktijk vereist dat vrijwel
unanieme instemming. Wie ook maar terughoudend is of bezwaar maakt tegen
een enkele aanpassing in de productspecificaties, kan de
modelverbetering effectief blokkeren of de introductiekosten doen
stijgen, waardoor de voordelen van grootschalige exploitatie nog verder
in gevaar komen.

\subsection{Onnodige onderhandelingen}\label{onnodige-onderhandelingen}

Een assemblagelijn die door onafhankelijke aannemers wordt ingehuurd (of
apart eigendom is) kent talloze kwetsbaarheden die je kunt vermijden
door alles onder één dak te organiseren. In ondernemingen waar duizenden
mensen samenwerken om één product te realiseren, komen de dood, ziekte
of financiële problemen van individuele aannemers té vaak voor. Hoewel
de veilingmarkt deze aannemers zeker had kunnen vervangen, ging elke
vervanging gepaard met complexe onderhandelingen, zoals de overname van
de vorige exploitant door zijn vervanger. Daarbij moest men ook
afspraken maken over het overnemen van de huurovereenkomst voor de
fabriek en mogelijk een nieuw leasecontract afsluiten voor de lasmachine
of de pers die de achterlichtkappen stanst. Al deze regelingen maakten
de zaken bijzonder ingewikkeld.

\subsection{Incentive Traps}\label{incentive-traps}

Een ander cruciaal probleem op een lopende band met onafhankelijke
opdrachtnemers in het Industriële Tijdperk was dat de kapitaaleisen per
opdrachtnemer sterk konden verschillen. Zo kostte de kunststof mal voor
een dashboardschakelaar relatief weinig, terwijl de apparatuur om een
motorblok te gieten of plaatwerk voor een spatbord uit te stansen
miljoenen kon bedragen. Door het hoge grondstofverbruik en de
opeenvolgende aard van assemblagelijnproductie werden de problemen door
hoge kapitaalkosten onvermijdelijk, zoals in het vorige hoofdstuk is
aangetoond. Opdrachtnemers met kapitaalintensieve taken waren immers
afhankelijk van de medewerking van anderen om hun investeringen terug te
verdienen. Hun vermogen om kapitaal aan te trekken en winstgevend te
opereren hing af van de samenwerking met tal van andere partijen,
waarvan de kapitaalkosten ver beneden lagen -- medewerking die in veel
gevallen uitbleef.

Kleine opdrachtnemers hadden bovendien er een grote prikkel om de
grotere partijen te exploiteren. Diegenen die minder geld nodig hadden
voor hun specifieke taak konden er op vertrouwen om op cruciale momenten
niet mee te werken. Net als stakende arbeiders konden zij de
assemblagelijn op een of andere voorwendsel stilleggen, waardoor zij
zelf weinig kosten opliepen maar partijen met hoge investeringen flink
benadeelden. Het productieproces werd daardoor voortdurend onderworpen
aan spelletjes, waarbij kleinschalige opdrachtnemers met hun tactieken
de grote partijen onder druk zetten door de productie te verstoren. Deze
manoeuvres om extra betalingsregelingen af te dwingen, verlaagden de
efficiëntie van het hele systeem.

\subsection{The firm solution}\label{the-firm-solution}

Kortom, veel schaalvoordelen die in het industriële tijdperk behaald
hadden kunnen worden door op grote schaal een lopende band in te zetten,
zouden weer teniet zijn gegaan als de productie verspreid was over
talloze individuele opdrachtnemers. Een groot, enkel bedrijf bleek een
efficiënte manier om deze nadelen te overbruggen, ondanks de eigen
beperkingen. Grootschalige ondernemingen waren van nature
bureaucratisch, maar in die periode waren bureaucratie en hiërarchie
precies wat nodig was. Administratieve teams én managers hielden de
productie nauwlettend in de gaten en coördineerden de processen, waarbij
talloze middenmanagers bevelen doorspeelden en informatie weer naar
boven brachten. De bedrijfsbureaucratie voerde tevens de boekhoudkundige
controles uit en minimaliseerde het principaal-agentprobleem, dat
ontstaat wanneer werknemers niet altijd in het belang van het bedrijf
handelen. Het bijhouden van een geavanceerde administratie in die tijd
vergde de inzet van vele mensen. Het opzetten van zo'n bureaucratische
organisatie was kostbaar en bleef geld kosten, ongeacht of de productie
draaide of stil lag. Omdat deze bestuurders over cruciale kennis
beschikten die essentieel was voor de bedrijfsvoering, betaalde men hen
doorgaans boven de marktconforme tarieven.

\subsection{Organisatorische speling}\label{organisatorische-speling}

Het grote aantal professionele managers en bestuurders had ook als
keerzijde dat zij het bedrijf soms voor eigen gewin `kapen' in plaats
van de aandeelhouders te dienen. In het industriële tijdperk kwam je
regelmatig bedrijven tegen die royaal uitgaven deden aan
kantoormeubilair, clublidmaatschappen en andere extraatjes waar het
management van profiteerde, terwijl deze kosten weinig rechtstreeks
rendement opleverden voor de investeerders. Bij complexe ondernemingen
bleek het van buitenaf vaak onmogelijk om vast te stellen welke
overheaduitgaven essentieel waren en welke louter extraatjes voor de
werknemers vormden. Daarbij was het lastig te voorkomen dat een
aanzienlijk deel van de bedrijfsarbeiders hun taken verwaarloosde. Omdat
het technologisch gezien moeilijk was de prestaties te monitoren, werd
een omvangrijk middenkader onmisbaar, maar dit maakte het tegelijkertijd
ingewikkeld om datzelfde kader effectief te controleren. Al deze
omstandigheden leidden tot wat men `organisatorische speling' noemde --
een term die in 1963 werd geïntroduceerd door Richard Cyert en James
March in \emph{A Behavioral Theory of the Firm}.\footnote{Richard Cyert
  en James March, \emph{A Behavioral Theory of the Firm} (Englewood
  Cliffs, N.J.: \emph{Prentice-Hall}, 1983).} Uit gedegen onderzoek
bleek dat talrijke bedrijven hun potentiële prestaties fors niet
haalden.

\begin{quote}
'Of je nu resultaten behaalt of niet, het salaris blijft hetzelfde.

'Of je nu hard werkt of niet, het salaris blijft hetzelfde.

`Of je nu om je werk geeft of niet, het salaris blijft
hetzelfde.'\footnote{Chris Dray, `Civil servants lead lives of quiet
  collusion', \emph{Globe and Mail}, 2 februari 1996, p.~A14.} -- Chris
Dray
\end{quote}

\subsection{Dat is niet mijn taak}\label{dat-is-niet-mijn-taak}

Als organisatie die continuïteit nastreeft, kampte het grote industriële
bedrijf met het eerder genoemde nadeel: het was kwetsbaar voor afpersing
door vakbonden. Daarbij vertoonde het ook enkele bureaucratische
trekken, zoals je die vaak in overheidskantoren aantreft. Opdrachten
kwamen voortdurend van bovenaf. Men standaardiseerde taken en deelde ze
strikt op in compartimenten. Men definieerde deze taken vaak rigide,
waardoor er scherpe grenzen ontstonden tussen functiecategorieën,
vergelijkbaar met de scheidslijnen die de kartels in de geleerde
beroepen hanteerden. In het industriële tijdperk vond men het net zo
absurd als te verwachten dat een boekhouder een kapotte gloeilamp in
zijn bureaulamp verving, als dat men een advocaat inschakelde om de
griep te verhelpen. Van de werknemers verwachtte men dan ook geen
overtreding van de strikte functiedeling -- en in veel gevallen was het
zelfs niet toegestaan om die grenzen over te steken.

`Dat is niet mijn taak' was een veel gehoorde slogan die de
organisatorische traagheid van het industriële tijdperk benadrukte.
Iedereen kreeg een nauw omschreven taak, gebaseerd op gestandaardiseerde
werkinstructies, waaraan niet mocht worden afgeweken, hoe de
productiviteit ook verbeterd had kunnen worden. Medewerkers werden
aangenomen op grond van `kwalificaties' waarvan men dacht dat ze de
prestaties in hun specifieke functie voorspelden. Bijna iedereen kreeg
een vast salaris op basis van de functieclassificatie, waardoor de
beloning min of meer uniform bleef binnen de hele organisatie. Omdat men
in de administratieve hiërarchieën van het bedrijfsleven -- net als in
staatsbureaucratieën -- zelden de specifieke prestaties matigde, liep
het werk in een rustig tempo. Hoewel het bedrijf de schaalvoordelen van
massaproductie benutte, ging dat ten koste van andere inefficiënties.

\begin{quote}
In een markt doe je niets simpelweg omdat iemand het je opdraagt of
omdat het op pagina dertig van het strategisch plan staat. Een markt
kent immers geen taakgrenzen. Er komen geen bevelen, geen signalen van
boven die vertaald worden en er is niemand die het werk in porties
opsplitst. In een markt heb je klanten, en de relatie tussen een
leverancier en een klant is fundamenteel niet-organisatorisch, omdat
deze tussen twee onafhankelijke entiteiten plaatsvindt.\footnote{William
  Bridges, Jobsh ft: How to Prosper in a Workplace Without Jobs
  (Reading, Mass.: \emph{Addison-Wesley}, 1994), pp.~62, 64.} William
Bridges
\end{quote}

\subsection{Nieuwe imperatieven}\label{nieuwe-imperatieven}

De nieuwe megapolitieke omstandigheden in dit informatietijdperk
veranderen ingrijpend de logica achter bedrijfsorganisaties. Sommige
gevolgen liggen al voor de hand. Als informatietechnologie verder niets
anders doet dan informatie verwerken, rekenen, analyseren en daarbij de
kosten drastisch drukt, vermindert dat de behoefte om massaal
middenmanagers in dienst te nemen voor het toezicht op
productieprocessen. Dankzij geavanceerde rekenkracht vervangen
geautomatiseerde werktuigmachines in veel gevallen werknemers op
uurloon. Ook wanneer mensen het productieproces bedienen, verloopt het
toezicht en de coördinatie grotendeels automatisch. Apparatuur met
geïntegreerde microprocessors monitort de voortgang van de lopende band
veel efficiënter dan menselijke managers ooit konden. De nieuwe systemen
meten niet alleen de snelheid en nauwkeurigheid van de arbeid, maar
stellen tevens de administratie op en bestellen onderdelen automatisch
opnieuw zodra ze uit de voorraad raken. Nu kunnen zelfs de kleinste
ondernemingen geavanceerde financiële beheersystemen inzetten, die hun
geldstromen sneller en nauwkeuriger in de gaten houden dan de grootste
bedrijven via hun uitgebreide productiehiërarchieën tientallen jaren
geleden konden realiseren.

Doordat informatietechnologie een verspreide, niet-lineaire productie
van goederen mogelijk maakt met een lager gebruik van natuurlijke
hulpbronnen, verkleint dit drastisch de kwetsbaarheid voor manipulatie
en afpersing, zoals eerder al aangestipt. Maar dat is nog niet alles:
andere eigenschappen van informatietechnologie maken het tevens
aantrekkelijker om functies, die vroeger door werknemers werden vervuld,
uit te besteden. De kapitaalkosten dalen en de productcycli worden
korter. Ook onafhankelijke opdrachtnemers, waaronder eenmanszaken,
beschikken nu over veel geavanceerdere informatienetwerken. Binnenkort
kunnen zij rekenen op een breed scala aan digitale assistenten voor
uiteenlopende kantooractiviteiten, variërend van het beantwoorden van
telefoongesprekken tot secretariaatsdiensten. Deze digitale assistenten
nemen de taken van secretarissen, reclameagenten, reisagenten,
bankmedewerkers en ambtenaren over.

\subsection{Het verdwijnen van goede
banen}\label{het-verdwijnen-van-goede-banen}

Wanneer steeds meer mensen, die in staat zijn aanzienlijke economische
waarde te creëren, dat grootste deel voor zichzelf houden, verandert de
arbeidsmarktdynamiek fundamenteel. Steeds meer ondernemingen vervangen
het ondersteunende personeel -- dat vroeger een groot deel van de omzet
verteerde die uit de kernactiviteiten kwam -- door goedkope,
geautomatiseerde agenten en informatiesystemen. Dit betekent dat
organisaties hun dienstverlening van topkwaliteit beter kunnen
waarborgen door functies uit te besteden, in plaats van ze intern te
behouden, waar het vaak lastiger valt om medewerkers naar behoren te
belonen voor goed werk. Virtuele bedrijven schrappen de overbodige
organisatorische lagen door de traditionele organisatie als geheel
achterwege te laten.

`Goede banen' liggen in het verleden. Zoals Princeton-econoom Orly
Ashenfelter ooit verwoordde, betaal je bij een `goede baan' meer dan je
werkelijk waard bent.\footnote{Zie Al Ehrbar, `Re-engineering geeft
  bedrijven nieuwe efficiëntie, werknemers de ontslagbrief,' \emph{Wall
  Street Journal}, 22 juli 1992, p.~A14, geciteerd door Bridges, op.
  cit., p.~39.} In het industriële tijdperk waren er veel `goede banen'
omdat de informatie- en transactiekosten hoog waren. Ondernemingen
groeiden en haalden een groter aantal functies in huis om
schaalvoordelen te realiseren. Daarnaast stimuleerden belastingwetten de
uitbreiding en complexiteit van bedrijfsstructuren. Hoge belastingen in
de latere fase van het industriële tijdperk vergrootten kunstmatig de
voordelen om langdurige ondernemingen op te zetten en vaste werknemers
in dienst te nemen. In de meeste landen maakten belastingwetten en
regelgeving het opstarten en ontbinden van projectgebaseerde bedrijven
aanzienlijk duurder. Tevens dwongen deze regels ondernemers er vaak toe
zelfstandige contractanten in dienst te nemen. Bovendien veroorzaakten
juridische maatregelen een tijdelijke schaarste aan `goede banen',
doordat het duur en omslachtig werd om een werknemer te ontslaan,
ongeacht hoe gering zijn bijdrage aan de productiviteit was.

Het karakter van bedrijfsorganisaties in het industriële tijdperk zorgde
er onvermijdelijk voor dat de meest hooggekwalificeerde en getalenteerde
mensen -- die een onevenredig groot aandeel van de meerwaarde binnen een
organisatie genereerden -- relatief minder betaald kregen dan hun werk
daadwerkelijk waard was. In het informatietijdperk verandert dat.

Dankzij de microprocesrevolutie neemt de beschikbaarheid van informatie
sterk toe en dalen de transactiekosten. Hierdoor decentraliseert het
bedrijf. In plaats van te berusten op een permanente bureaucratie
organiseert men activiteiten per project, vergelijkbaar met de werkwijze
van filmmaatschappijen. De meeste voorheen interne functies worden
uitbesteed aan zelfstandige contractanten. Werknemers uit het
industriële tijdperk -- die `goede banen' hadden, weinig bijdroegen en
vertrouwden op collega's om hun tekortkomingen op te vangen -- komen al
snel in de positie om te bieden op contracten op de spotmarkt. Ook vele
loyale en hardwerkende medewerkers krijgen dit te verduren. `Goede
banen' behoren dan tot het verleden, want vaste tewerkstelling raakt
uiteindelijk achterhaald.

Bij de grootste Japanse bedrijven ging men ervan uit dat werknemers een
baan voor het leven hadden. Zelfs als zij niets productiefs te doen
hadden, bleven ze in dienst -- soms simpelweg door op te dagen en plaats
te nemen achter een kaal bureau in een hoek van de fabriek. Zelfs in
Japan schaalt men de opgeblazen populatie kantoormedewerkers nu terug.
De kop van een artikel in de \emph{International Herald-Tribune} luidde:
``Afscheid nemen is zo bitter: Japans baan-voor-het-leven-cultuur loopt
pijnlijk ten einde.''\footnote{Sheryl WuDunn, `Parting Is Such Sour
  Sorrow: Japan's Job-for-Life Culture Painfully Expires,'
  \emph{International Herald Tribune}, 13 juni 1996, p.~13.}

In de postindustriële periode draait een baan om de taken die je
verricht, niet om iets wat je simpelweg `hebt.' Vóór het industriële
tijdperk was vaste tewerkstelling vrijwel onbekend. Zoals William
Bridges het verwoordde: ``Vóór 1800 -- en in veel gevallen zelfs lang
daarna -- verwees \emph{job} altijd naar een specifieke taak of
onderneming, nooit naar een rol of positie binnen een organisatie.
\ldots{} Tussen 1700 en 1890 vindt de Oxford English Dictionary talloze
toepassingen van termen als \emph{job-coachman, job-doctor en
job-gardener} -- allen verwijzend naar mensen die op een eenmalige basis
werden aangenomen. Jobwork (een andere veelgebruikte term) betrof
incidenteel werk, niet reguliere tewerkstelling.''\footnote{Bridges, op.
  cit., pp.~31--32.} In het informatietijdperk keren de taken, die
vroeger binnen bedrijven werden georganiseerd om informatie- en
transactiekosten te verlagen, terug naar de spotmarkt. `Just in
time'-voorraadbeheer en outsourcing zijn dankzij informatietechnologie
mogelijk geworden. Beide ontwikkelingen vormen stappen richting het
verdwijnen van vaste banen. Grote ondernemingen zoals \emph{AT\&T}
hebben al alle vaste functiecategorieën afgeschaft; de posities zijn nu
tijdelijk en op voorwaardelijke basis. In de woorden van Bridges:
``Werkgelegenheid wordt weer tijdelijk en situationeel, en categorieën
verliezen hun grenzen.''\footnote{Ibid., p.~58.} In de nieuwe
cybereconomie werken zelfstandige contractanten over continenten heen op
afstand en organiseren zij zich op het informatietijdperk-equivalent van
de lopende band.

\subsection{Hollywood neemt het over}\label{hollywood-neemt-het-over}

De ideale bedrijfsorganisatie in de nieuwe informatie-economie doet
denken aan een filmproductiemaatschappij. Dit soort ondernemingen kan
heel verfijnd opereren en over budgetten van honderden miljoenen dollars
beschikken. Hoewel het vaak grote operaties betreft, hebben deze van
nature een tijdelijke opzet. Een filmmaatschappij die een film
produceert voor 100 miljoen dollar vormt zich bijvoorbeeld voor één jaar
en lost daarna weer op. De mensen die aan zo'n productie werken,
beschikken weliswaar over veel talent, maar zij verwachten niet dat
werken aan een project gelijkstaat aan het hebben van een vaste baan.
Zodra het project afgerond is, gaan de lichttechnici, cameramensen,
geluidsingenieurs en kledingspecialisten ieder hun eigen weg -- al kan
het natuurlijk voorkomen dat ze in een volgend project weer samenwerken.

Naarmate de schaalvoordelen afnemen en de kapitaaleisen voor tal van
informatie-intensieve activiteiten dalen, ontstaat er een sterke prikkel
voor bedrijven om zich te ontbinden. Bedrijfsactiviteiten worden dan
steeds meer ad hoc en tijdelijk, waardoor ondernemingen over het
algemeen korter bestaan. Virtuele bedrijven die talenten bundelen voor
specifieke opdrachten blijken efficiënter te zijn dan gevestigde
bedrijven. Zodra encryptie wijdverspreid raakt en door concurrentie de
druk op kapitaal wordt verlaagd, vervallen de kunstmatige
schaalvoordelen -- die het voortbestaan van `vaste' ondernemingen
ondersteunen. Dit gebeurt, of de belastingen nu snel of geleidelijk
worden verlaagd. Wordt er snel verlaagd, verdwijnen de extra kosten voor
projectmatig werken rap; gaat het traag, dan dragen bestaande
ondernemingen de zware last van de verouderd hoge belastingen, terwijl
nieuwe bedrijven als virtuele organisaties minder snel met deze kostbare
lasten worden geconfronteerd.

Hoewel specifieke vaardigheden en talenten in de informatie-economie
belangrijker worden dan ooit, vervagen de meeste kunstmatige
beroepsgrenzen. Geavanceerde technologieën voor het opslaan en opvragen
van informatie maken bedrijfsgeheimen en specialistische kennis in
vakgebieden als recht, geneeskunde en accountancy voor iedereen
toegankelijk. De economische waarde van pure memorisatie neemt af,
terwijl het vermogen om informatie te synthetiseren en creatief toe te
passen juist in belang groeit.

Verouderde regelgeving zal de volledige impact van deze veranderingen
vertragen, maar op de lange termijn neemt de invloed van overheden op de
regulering van de cybereconomie tot een verwaarloosbaar niveau af. Alle
kunstmatige reguleringen die professionele monopolen in stand houden en
daarbij de kosten verhogen zonder dat de markt de voordelen waardeert,
zal men uiteindelijk terzijde schuiven.

Er volgen nog andere gevolgen van de overgang naar een
informatie-economie:

\begin{itemize}
\tightlist
\item
  Overheden transformeren lokale regelgeving, die hoge kosten
  veroorzaakt, naar een marktconforme basis.
\item
  Rechtsgebieden wedijveren intens om economische activiteiten met hoge
  toegevoegde waarde aan te trekken -- activiteiten die in principe
  overal gesitueerd kunnen worden. Geen enkele locatie blinkt per se
  uit.
\item
  Zakelijke relaties bouwen steeds meer voort op `circles of trust'.
  Omdat encryptie stelen onopgemerkt mogelijk maakt, groeit de waarde
  van eerlijkheid als eigenschap van zakenpartners.
\item
  Gemakkelijke toegang tot informatie zorgt voor veranderingen in
  patent- en auteursrechtregelingen.
\item
  Bescherming verschuift steeds meer naar technologische in plaats van
  juridische middelen. Men sluit de lagere lagen uit, en de opkomst van
  afgesloten woonwijken lijkt onvermijdelijk. Het uitsluiten van
  onruststokers blijft een beproefde en efficiënte manier om crimineel
  geweld in tijden van zwakke centrale autoriteit te beperken.
\item
  Bulkgoederen dragen hoge belastingen en worden lokaal verzonden --
  zoals in de middeleeuwen -- terwijl luxe goederen licht belast worden
  en over grote afstanden worden verscheept.\footnote{Abu-Lughod, op.
    cit., p.~186.}
\item
  Particuliere bewakers, die actief zijn binnen koopmansverenigingen,
  nemen steeds vaker de politietaken over.
\item
  Particuliere ondernemingen kunnen tijdelijk een voordeel behalen ten
  opzichte van beursgenoteerde bedrijven, doordat zij makkelijker
  overheidskosten omzeilen.
\item
  `Banen voor het leven' verdwijnen, omdat werk zich steeds meer
  vormgeeft als een reeks losse taken of stukloonwerk in plaats van
  vaste functies binnen organisaties.
\item
  De staat verliest de controle over economische hulpbronnen aan mensen
  met superieure vaardigheden en intelligentie, omdat het eenvoudiger
  wordt om met kennis rijkdom te creëren.
\item
  Interactieve systemen voor informatieopvraging verdringen veel
  hoogopgeleide professionals.
\item
  Mensen met een lager cognitief vermogen ontwikkelen nieuwe
  overlevingsstrategieën. Zij investeren meer in vrijetijdsvaardigheden,
  sport en zelfs criminele activiteiten, en bieden diensten aan voor de
  groeiende groep soevereine individuen, nu de inkomensongelijkheid
  binnen rechtsgebieden toeneemt.
\end{itemize}

Politieke systemen die in een tijdperk van winstgevend geweld
ontstonden, moeten ingrijpend veranderen. Nu efficiëntie zwaarder weegt
dan de omvang van macht, worden kleine, efficiënte soevereine staten
steeds aantrekkelijker voor hun klanten, omdat zij meer bescherming
bieden tegen lagere kosten.

Net als in de middeleeuwen komen de nadelen van schaalvergroting in het
organiseren van geweld weer duidelijk naar voren. Dit illustreert het
groeiende aantal soevereiniteiten sinds de val van het communisme. Wij
verwachten dat het aantal soevereiniteiten snel zal toenemen zodra de
logica van het informatietijdperk in de praktijk bevestigd wordt.

Macht wordt opnieuw op kleine schaal uitgeoefend. Enclaves en provincies
kunnen zelfs aanzienlijke voordelen behalen ten opzichte van landen die
zich over meerdere continenten uitstrekken, mits zij concurrerende
voorwaarden bieden voor soevereiniteitsdiensten aan hun `klanten'. Dit
verschilt sterk van de moderne periode, waarin geen enkele entiteit kon
overleven zonder een militaire macht die een koninkrijk beheerst.
Vroeger regeerden degenen die het meest profiteerden van bescherming --
zoals de welvarende kooplieden in laatmiddeleeuwse stadstaten -- de
regering. Naar onze mening ontstaat hier weer ruimte voor een dergelijke
situatie. Het verlagen van roofzuchtige lasten en een efficiëntere inzet
van middelen zullen naar verwachting leiden tot een snelle groei in
gebieden waar de klanten daadwerkelijk de controle over de lokale
soevereiniteiten hebben.

Zoals we verderop zullen onderzoeken, behoort de vraag of deze
ontwikkelingen, ondanks de tegenstand van velen, doorgang moeten vinden
tot de belangrijkste controverses van het informatietijdperk.

\bookmarksetup{startatroot}

\chapter{Nationalisme, reactie en de nieuwe
ludieten}\label{nationalisme-reactie-en-de-nieuwe-ludieten}

\begin{quote}
`Nationalisme is uiteraard van nature absurd. Waarom zou het toeval --
of het lot om geboren te worden als Amerikaan, Albanees, Schot of
eilandbewoner van \emph{Fiji} -- loyaliteiten toewijzen die een individu
volledig domineren en een samenleving zo inrichten dat ze formeel met
anderen in conflict komt? Vroeger leefden mensen met lokale
loyaliteiten, gehecht aan een plaats, clan of stam, en hadden zij
verplichtingen jegens een heer of landeigenaar, wat leidde tot
dynastieke of territoriale oorlogen. De voornaamste loyaliteiten waren
echter gericht op religie, op God of op de godkoning, eventueel op een
keizer of op een beschaving als geheel. Er was geen natie. Men voelde
wel een band met de patria, het land van je voorouders, of kende
patriottisme, maar over nationalisme spreken vóór de moderne tijd is een
anachronisme.'\footnote{William Pfaff, \emph{The Wrath of Nations:
  Civilization and the Furies of Nationalism} (New York:
  \emph{Simon~\&~Schuster}, 1993), p.~17.} - WILLIAM PFAFF
\end{quote}

\url{http://www.ibm.com} Zeggen dat `de wereld kleiner wordt' is een
treffende beeldspraak, versterkt door autoriteiten zo gerenommeerd als
het reclamebureau van \emph{IBM}. Hun multiculturele reclamecampagnes
voor `\emph{Solutions for a small planet}' op het Internet herinneren
sportliefhebbers -- die dit zelf wellicht niet doorhebben -- eraan dat
de verhoudingen tussen individuen in wijd verspreide rechtsgebieden door
technologische ontwikkelingen ingrijpend zijn veranderd. We verwijzen
naar de vooraanstaande historicus William McNeill voor een waardevolle
voetnoot over de implicaties. Hij schrijft: ``De voortdurende
intensivering van communicatie en transport, in plaats van de nationale
consolidatie te bevorderen, begint in te werken in een tegenovergesteld
verlopend proces, daar de reikwijdte ervan de bestaande politieke en
etnische grenzen overstijgt.''\footnote{William H. McNeill,
  `Herbevestiging van de poly-etnische norm', in John Hutchinson en
  Anthony D. Smith (red.), \emph{Nationalism} (Oxford: \emph{Oxford
  University Press}, 1994), p.~300.} Nu de wereld steeds `kleiner wordt'
en de communicatie verbetert, zullen de willekeurige en inherent absurde
aanspraken van naties en van het nationalisme onvermijdelijk verzwakken.

\section{De grote transformatie}\label{de-grote-transformatie}

Het probleem met deze redelijke verwachting blijkt uit de gehele
geschiedenis: zij toont aan dat deze verwachtingen op geen enkele wijze
vervuld kunnen worden. De noodzakelijke overgang gaat gepaard met een
crisis. Het vraagt om een radicaal andere denkwijze en om een nieuwe
voorstelling van gemeenschap die verder reikt dan het nationalisme en de
natiestaat. Zoals Michael Billig benadrukt, zijn onze opvattingen over
de natiestaat en de vanzelfsprekendheid van nationaal behoren simpelweg
het resultaat van een specifieke historische periode.\footnote{Michael
  Billig, \emph{Banal Nationalism} (Londen: \emph{Sage Publications},
  1995), p.~16.} Die periode -- het Moderne Tijdperk -- behoort mogelijk
inmiddels tot het verleden. De dominante instituties -- de natiestaten
-- bestaan nog, maar zij staan op een wankele, geërodeerde basis.
Wanneer het onvermijdelijke zich voltrekt en natiestaten instorten,
verwachten we een felle reactie, met name in die welvarende landen waar
de `nationale economie' in de twintigste eeuw ongeschoolde arbeid goed
betaalde. Wij zijn ervan overtuigd dat de veranderingen in de
megapolitieke omstandigheden, dankzij de opkomst van
informatietechnologie, zullen resulteren in een ingrijpende
institutionele transformatie. De stelling in dit boek luidt dat de
geconcentreerde macht van de natiestaat gedoemd is te worden
geprivatiseerd en gecommercialiseerd. Net als bij andere ingrijpende
institutionele veranderingen zal de privatisering en commercialisering
van soevereiniteit een revolutie ontketenen in ons `gezond verstand', in
de wijze waarop we de wereld interpreteren. Dergelijke veranderingen
verlopen zelden geleidelijk of lineair.

Integendeel. Zoals we in \emph{The Great Reckoning} hebben aangetoond,
sluiten we dat in de praktijk vrijwel altijd uit. Wij verwachten dat het
informatietijdperk ingrijpende discontinuïteiten teweegbrengt -- dat wil
zeggen, scherpe breuken met de bestaande instituties en met het
bewustzijn van het verleden. Let op het volgende naarmate het proces
vordert:

\begin{enumerate}
\def\labelenumi{\arabic{enumi}.}
\item
  Wijzigingen in de economische organisatie, zoals in voorgaande
  hoofdstukken besproken, door de invloed van microverwerking.
\item
  Een tamelijk snelle daling in het belang van organisaties die zich
  beperken tot binnenlandse operaties. Overheden, vakbonden,
  gereglementeerde beroepsgroepen en lobbyisten spelen in het
  informatietijdperk een minder prominente rol dan in het industriële
  tijdperk. Doordat overheden hun gunsten en handelsbeperkingen niet
  meer zo effectief kunnen inzetten, daalt ook de verspilling aan
  lobbyactiviteiten.\footnote{Gordon Tullock, \emph{Rent-Seeking}
    (Aldershot, Engeland: \emph{Edward Elgar}, 1993).}
\item
  Een groeiend besef dat de natiestaat verouderd is, wat wijdverspreid
  leidt tot afscheidingsbewegingen in diverse delen van de wereld.
\item
  Een afname in status en invloed van traditionele elites, met een
  verminderd respect voor de symbolen en idealen waarop de natiestaat is
  gebaseerd.
\item
  Een heftige, soms zelfs gewelddadige nationalistische reactie die
  vooral voorkomt bij mensen die hun status, inkomen en macht verliezen
  wanneer hun `normale leven' verstoord wordt door politieke
  decentralisatie en de opkomst van nieuwe marktsystemen. Onder de
  kenmerkende uitingen van deze reactie vallen onder meer:

  \begin{enumerate}
  \def\labelenumii{\alph{enumii}.}
  \item
    een diepgeworteld wantrouwen en verzet tegen globalisering,
    vrijhandel, `buitenlands' eigendom en het binnendringen van lokale
    economieën;
  \item
    een uitgesproken vijandigheid tegenover immigratie, in het bijzonder
    als het gaat om groepen die duidelijk afwijken van de traditionele
    nationale samenstelling;
  \item
    een sterke afkeer van de informatie-elite, de rijken en
    hoogopgeleiden, gecombineerd met klachten over kapitaalvlucht en het
    verdwijnen van banen;
  \item
    extreme maatregelen die erop gericht zijn de afscheiding van
    individuen en regio's uit wankelende natiestaten te voorkomen, zoals
    het inzetten van oorlogen en het uitvoeren van `etnische
    zuiveringen', waarmee de nationale verbondenheid met de staat
    versterkt en de aanspraken van de staat op haar burgers en hun
    middelen wordt legitimeerd.
  \end{enumerate}
\item
  Omdat het duidelijk wordt dat informatietechnologieën soevereine
  individuen in staat stellen zich los te maken van staatsmacht,
  reageert men op het wegvallen van dwang met een neo-Luddistische
  aanval op zowel deze nieuwe technologieën als op hun gebruikers.
\item
  De nationalistisch-luddistische reactie verschilt per regio en
  bevolkingsgroep:

  \begin{enumerate}
  \def\labelenumii{\alph{enumii}.}
  \tightlist
  \item
    In snelgroeiende economieën, waar in het industriële tijdperk het
    inkomen per hoofd nog laag was en waar een verdere ontwikkeling van
    de markten leidde tot inkomensstijging op alle vaardigheidsniveaus,
    valt deze reactie minder intens op.
  \item
    Reactionaire gevoelens worden het sterkst ervaren in de huidige
    rijke landen, met name in gemeenschappen met een hoog percentage
    waardeloze en vaardigheidsarme mensen die voorheen hoge inkomens
    genoten.\footnote{Lawrence E. Harrison bespreekt in zijn boek
      \emph{Who Prospers? How Cultural Values Shape Economic and
      Political Success} (New York: Basic Books, 1992) uitvoerig de
      nauwe relatie tussen vaardigheden en waarden en het daaruit
      voortvloeiende economische succes.}
  \end{enumerate}
\end{enumerate}

\begin{enumerate}
\def\labelenumi{\alph{enumi}.}
\setcounter{enumi}{2}
\item
  Met uitzondering van de Unabomber trekken de neo-Luddieten vooral
  aanhangers uit het onderste tweederde van de verdiencapaciteit binnen
  de bevolking van de toonaangevende natiestaten.
\item
  De nationalistische en Luddistische respons zal zich het sterkst tonen
  niet bij de top, maar juist bij mensen met gemiddelde vaardigheden --
  `underachievers' met diploma's die volwassen werden in het industriële
  tijdperk en nu geconfronteerd worden met neerwaartse mobiliteit.
\end{enumerate}

\begin{enumerate}
\def\labelenumi{\arabic{enumi}.}
\setcounter{enumi}{7}
\item
  Naarmate nieuwe megapolitieke omstandigheden een vernieuwd
  identiteitsbesef oproepen, vergezeld van aanvullende ideologieën en
  morele waarden, verliezen de traditionele imperatieven van het
  nationalisme geleidelijk aan hun aantrekkingskracht.
\item
  De nationalistische reactie bereikt zijn hoogtepunt in de vroege
  decennia van het nieuwe millennium en neemt daarna af wanneer
  gefragmenteerde soevereiniteiten efficiënter blijken dan de
  gecentraliseerde macht van de natiestaat. Wij vermoeden dat het
  aangeboren vijandige gedrag van natiestaten tegenover alternatieve
  rechtsgebieden -- zoals geïllustreerd door de Russische invasie van
  Tsjetsjenië -- ertoe leidt dat naties en nationalistische
  fanatiekelingen de sympathie van de nieuwe generaties missen, die
  opgroeien onder de megapolitieke voorwaarden van het
  informatietijdperk.
\item
  Uiteindelijk stort de natiestaat ineen door een fiscale crisis.
  Systemische crises ontstaan doorgaans wanneer zwakke instituties
  kampen met stijgende kosten en dalende inkomsten -- een ontwikkeling
  die de toonaangevende natiestaten onvermijdelijk treft nu pensioenen
  en zorguitgaven begin eenentwintigste eeuw de hoogte in schieten.
  Terwijl wij dit schrijven worstelen zowel het Verenigd Koninkrijk als
  de Verenigde Staten met onverzekerde pensioenverplichtingen ter waarde
  van meerdere biljoenen dollars (per hoofd van de bevolking
  vergelijkbaar), die waarschijnlijk niet beteugeld kunnen worden. Ook
  andere leidende natiestaten kampen met vergelijkbare,
  faillissementsverwekkende lasten.
\end{enumerate}

\section{Parallels with the
Renaissance}\label{parallels-with-the-renaissance}

Eerder hebben we uiteen gezet waarom velen ervan overtuigd zijn dat de
ondergang van de `nanny state' gevolgen zal hebben die sterk lijken op
het verval van het institutionele monopolie van de Heilige Moederkerk
vijf eeuwen geleden. De Kerk oefende, net als de moderne natiestaat, al
eeuwenlang een onbetwiste dominantie uit. Op sommige punten zat de Kerk
zelfs dieper verankerd dan de staat dat vijfhonderd jaar later ooit zou
worden. Al jarenlang profileerde de Kerk zich als `de universele
autoriteit aan het hoofd van de christelijke samenleving,' aldus de
middeleeuwse intellectueel historicus John~B. Morrall.\footnote{John B.
  Morrall, \emph{Political Thought in Medieval Times} (New York:
  \emph{Harper}, 1958), p.~48.} Nog voor de technologische revolutie van
de jaren 1490 zouden nauwelijks Europeanen de aanspraak van de Kerk op
suprematie binnen het christendom in twijfel trekken, maar daarna hield
de traditionele rol van de Kerk nauwelijks langer stand.

\subsection{The privatization of
conscience}\label{the-privatization-of-conscience}

Aan het begin van de jaren 1520 verwierpen miljoenen Europeanen de
universele autoriteit van de Rooms-Katholieke Kerk -- een daad die als
ketterij werd beschouwd, waarvoor enkele decennia eerder foltering en de
doodstraf waren uitgerekend. Inderdaad, talloze middeleeuwse kathedralen
en kerken in Europa pronkten met leerzame gravures waarop ketters
afgebeeld stonden, als bewijsmateriaal dat demonen hun tong
uittrokken.\footnote{Voorbeeld: de gevel van de kathedraal in Angoulême
  (Frankrijk).}

Die martelingen maakten ongetwijfeld diepe indruk op de analfabete
parochianen, die de slachtoffers uitsluitend herkende aan het feit dat
zij als ketters bestraft werden. De beeldspraak liet geen twijfel
bestaan: ketters waren degenen van wie de tong werd verminkt. Hoe streng
de straf ook was, het was slechts een voorproefje van de ultieme
bestraffing voor ketterij -- de dood op de brandstapel. Tot
teleurstelling van de Kerk schrok deze waarschuwing echter niet genoeg
af. Met de opkomst van de drukpers nam het aanbod aan ketterse
argumenten zo sterk toe dat zelfs de dreiging van gruwelijke straffen
potentiële ketters niet langer afschrok. Inderdaad, geen enkele pionier
voor religieuze vrijheid in het vroegmoderne Europa betaalde de tol van
het laten afhakken van zijn tong. Anderen werden op de brandstapel
verbrand. De reactionaire agenten van de Inquisitie verbrandden
letterlijk mensen omdat zij uitspraken deden die wij als gewone
gewetensuitingen beschouwen. Al met al hebben de Reformatie en de
daaropvolgende reactie miljoenen levens geëist. Alleen al in de tweede
helft van de Dertigjarige Oorlog kwamen er op het slagveld 1.151.000
doden aan te pas.\footnote{Karen A. Rasler en William R. Thompson,
  \emph{Oorlog en staatsvorming: de vorming van de wereldmachten.
  Studies in internationale conflicten}, vol.~2 (Boston: Unwin Hyman,
  1989), p.~13.} Veel meer mensen stierven door honger, ziekte en door
de hand van de Inquisitie en andere autoriteiten. Niet al het geweld
kwam van katholieke machthebbers. De botten van meer dan duizend
vooraanstaande Engelse katholieken -- waarvan men dacht dat koning
Hendrik~VIII ze bruut had vermoord -- kwamen aan het licht in de Tower
of London. Sommigen, waaronder Sir Thomas More en bisschop St.~John
Fisher, werden openlijk geëxecuteerd omdat zij weigerden het oude geloof
op te geven.\footnote{Julian Large, `Bisschop stierf omdat hij
  standvastig bleef tegen Hendrik VIII', \emph{Daily Telegraph}, 16 juni
  1996, p.~2.} Aan de andere kant liet koning Hendrik~VIII's katholieke
dochter, koningin Maria -- die door syfilis, geërfd van haar vader,
waanzinnig was geworden -- in de laatste twee jaren van haar
regeerperiode maar liefst driehonderd protestantse ketters op de
brandstapel verbranden.

Men eiste een enorme tol van mensen met uiteenlopende overtuigingen toen
zij in de praktijk hun geloofsovertuigingen en het lang ontzegde recht
om zelf de door henzelf gesteunde kerk te kiezen omarmden. Vanuit ons
perspectief aan het einde van de twintigste eeuw vallen deze
persoonlijke uitingen ruim binnen het bereik van wat de vrijheid van
godsdienst en meningsuiting zou moeten beschermen. Maar aan het begin
van de zestiende eeuw bestonden noch de vrijheid van religie, noch de
vrijheid van meningsuiting. De autoriteiten hanteerden nog steeds een
vervagende middeleeuwse wereldbeschouwing. Zij zagen elke uiting van
individuele autonomie -- zeker wanneer die in opstand kwam tegen de
`plentitude potestatis' (volheid van macht) van de paus -- als
schandalig en subversief. Theologische historicus Euan Cameron merkte op
dat religieuze hervormers zoals Maarten~Luther overtuigingen omarmden
die `een bewuste en beslissende breuk betekenden met de institutionele
en geestelijke continuïteit van de oude Kerk.'\footnote{Cameron,
  \emph{op. cit.}, p.~97.}

\subsection{Ketterij en verraad}\label{ketterij-en-verraad}

In deze geest verwachten we `een weloverwogen en beslissende breuk' met
de institutionele en ideologische continuïteit van de natiestaat. Tegen
het einde van het eerste kwartaal van de volgende eeuw zullen miljoenen
rechtschapen burgers het seculiere equivalent van de ketterij uit de
zestiende eeuw hebben begaan -- een soort klein verraad. Zij trekken hun
loyaliteit aan de wankele natiestaat in om hun eigen soevereiniteit te
bevestigen, namelijk het recht om als burgers niet blindelings hun
bisschoppen of gebedshuizen te accepteren, maar zelf de vorm van bestuur
te kiezen. De privatisering van soevereiniteit gaat hand in hand met de
privatisering van het geweten van vijf eeuwen geleden. Beide vormen een
massale defectie van voormalig trouwe aanhangers van dominante
instellingen. Zoals Albert~O.~Hirschman, een expert op het gebied van
`reacties op verval in bedrijven, organisaties en staten', schreef, is
dit type exit zwaarwegend, omdat `exit vaak als crimineel wordt
bestempeld, aangezien het wordt gezien als desertie, defectie en
verraad.'\footnote{Hirschman, \emph{op. cit.}, p.~17.}

Soevereine individuen laten zich niet langer behandelen als louter
menselijke hulpbronnen van de staat. Miljoenen zullen de verplichtingen
van het burgerschap afwijzen om klanten te worden van de
overheidsdiensten. Zij zullen parallelle instellingen oprichten en
ondersteunen die vrijwel alle diensten die met burgerschap samenhangen
volledig op commerciële basis aanbieden. Gedurende het grootste deel van
de twintigste eeuw behandelde de staat werkenden als activa, net zoals
een melkveehouder zijn koeien inzet. Zij werden steeds harder uitgebuit
-- en nu krijgen de koeien vleugels.

\subsection{Defectie van het
burgerschap}\label{defectie-van-het-burgerschap}

Net zoals in de zestiende eeuw nieuwe megapolitieke ontwikkelingen het
kerkelijke monopolie ondermijnden, verwachten wij dat de megapolitiek
van het informatietijdperk uiteindelijk de voorwaarden voor het bestuur
in de eenentwintigste eeuw zal bepalen -- ongeacht hoe extreem haar
nieuwe voorschriften ook lijken voor degenen die de waarden van de
moderne politiek als de hunne beschouwen. De transformatie van de status
van `burger' naar die van `klant' betekent een breuk met het verleden,
zo ingrijpend als de overgang van ridderlijkheid naar burgerschap in de
vroegmoderne tijd. Wanneer de informatie-elite haar band met het
burgerschap verbreekt, levert dat een stimulans die vergelijkbaar is met
de reden waarom vijfhonderd jaar geleden miljoenen Europeanen hun
vertrouwen in de paus verloren.

Als de vergelijking met de Reformatie niet overtuigend overkomt, ligt
dat misschien deels aan het feit dat men tegenwoordig niet direct inziet
dat het afzweren van loyaliteit aan religieuze instituties ooit zo
ingrijpend was als het verraad dat in de twintigste eeuw streng werd
bestraft. Buiten enkele islamitische landen beschouwt men ketterij aan
het einde van de twintigste eeuw als een geestelijke overtreding die
iemands reputatie net zo min aantast als een bekeuring voor het rijden
met 45 mijl per uur in een 30-mijlzone. Sterker nog, het komt regelmatig
voor in Europa en Noord-Amerika dat geestelijken -- en zelfs bisschoppen
-- openlijk aangeven niet in God te geloven of cruciale fundamenten van
het geloof dat zij belijden af te wijzen. Tegenwoordig zal men ketterij
bijna alleen nog constateren wanneer zij neerkomt op flagrante
duivelaanbidding. In de meeste westerse landen zijn de religieuze
doctrines zo onsamenhangend en losjes vastgelegd dat weinigen nog de
theologische kernelementen kunnen aanwijzen die ooit het middelpunt
vormden van ketterijgeschillen. Dit weerspiegelt de algemene
verschuiving van de aandacht weg van religieuze kwesties.

Religieuze leiders hebben er in zekere mate toe bijgedragen dat men in
de late twintigste eeuw spirituele onderwerpen niet langer serieus nam.
Zij verlegden hun energie van de spirituele sferen naar het worden van
lobbyisten en agitatoren. Als losse individuen, aangetrokken door de
kracht van de macht, richten zij zich voornamelijk op het onder druk
zetten van politieke leiders om herverdelingsmaatregelen in te voeren
die essentieel blijken voor het nationalistische compromis. Denk
bijvoorbeeld aan de luidruchtige acties van de katholieke kerk in
Argentinië, die de regering van president Carlos Menem probeert te
dwingen economische hervormingen te laten varen ten gunste van
conventionele, inflatoire monetaire en Keynesiaanse fiscale maatregelen.
Vergelijkbare klachten hebben religieuze leiders ook geuit over pogingen
de opgeblazen begrotingen in Nieuw-Zeeland -- en in vele andere landen
-- te hervormen. Katholieke bisschoppen lobbyden fel tegen de hervorming
van de sociale zekerheid in de Verenigde Staten.

\subsection{Een fiscale inquisitie?}\label{een-fiscale-inquisitie}

Simpel gezegd richten hedendaagse religieuze leiders -- die hun moreel
gezag zien afnemen -- zich vooral op seculaire verlossing en op het
beïnvloeden van de staat via agitatie, in plaats van op spirituele
redding. In deze situatie kun je verwachten dat zij als medeplichtigen
meewerken aan de tegenreactie op de opkomende seculaire reformatie.
Wanneer de natiestaat wordt uitgedaagd en begint te wankelen, slaagt zij
er niet langer in de materiële beloftes na te komen die de kern van haar
populaire steun vormen. Het compromis dat tijdens de Franse Revolutie
werd gesloten, vervalt. De staat kan haar burgers niet langer goedkoop
of gratis onderwijs, laat staan medische zorg, werkloosheidsverzekering
en pensioen bieden als ruil voor toch al laag betaalde militaire dienst.
Hoewel de veranderende eisen van oorlogsvoering regeringen in staat
stellen zich en hun grondgebied te verdedigen zonder massale legers te
mobiliseren, nodigt het doorbreken van dit verouderde akkoord vrijwel
onvermijdelijk tot kritiek uit.

Inderdaad, zodra de nieuwe megapolitieke logica de overhand krijgt,
zullen de verliezers in de nieuwe informatie-economie de gevolgen als
buitengewoon onwelgevallig ervaren. Het is vrijwel zeker dat talloze
religieuze leiders, samen met de voornaamste begunstigden van
overheidsuitgaven, de leiding nemen in een nostalgische reactie die het
nationalisme opnieuw kracht bijzet. Zij zullen stellen dat geen enkele
Amerikaan, Fransman, Canadees of iemand van een andere nationaliteit
naar bed zou mogen gaan zolang hij honger heeft. Zelfs landen die
vooroplopen in hervormingen en onevenredig veel profiteren van
`marktvriendelijk globalisme', zoals Nieuw-Zeeland, zullen geteisterd
worden door reactionaire verliezers. Zij zullen proberen de
grensoverschrijdende beweging van kapitaal en mensen te belemmeren -- en
daarmee stoppen ze beslist niet.

Demagogen zoals Winston Peters, leider van de \emph{New Zealand First
Party}, hebben simpelweg geen zin om op originele wijze na te denken
over hoe de nieuwe wereld gaat functioneren. Maar na verloop van tijd
zullen Winston en zijn volgelingen de logica van de informatie-economie
doorgronden. Zij zullen trachten de verspreiding van computers,
robotica, telecommunicatie, encryptie en andere technologieën uit het
informatietijdperk -- die de verdringing van arbeiders in vrijwel elke
sector van de wereldeconomie bevorderen -- te blokkeren. Overal ontmoet
je politici die de vooruitzichten op langdurige welvaart actief willen
dwarsbomen, uitsluitend om te voorkomen dat individuen hun
onafhankelijkheid van de politiek gaan verkondigen.

\subsection{20/20-visie}\label{visie}

Tegen 2020 -- ofwel ongeveer vijf eeuwen nadat Martin Luther zijn 95
subversieve stellingen op de kerkdeur in Wittenberg had vastgenageld --
zal de opvatting over de kosten en baten van burgerschap een even
subversieve verheldering hebben doorgemaakt. De visie op de natiestaat
bij mensen met talent en vermogen -- de soevereine individuen van de
toekomst -- zal een politieke transformatie ondergaan die te vergelijken
is met laserchirurgie. Zij zullen 20/20 zien. In de twintigste eeuw --
evenals gedurende het hele moderne tijdperk -- zorgde het voortdurende
succes van geweld ervoor dat een grote overheid rendabel werd. De kracht
van geconcentreerde macht wist de loyaliteit van welgestelden en
ambitieuze individuen te winnen voor de \emph{OECD}-natiestaten, ondanks
de roofzuchtige belastingen op inkomen en kapitaal. Politici slaagden
erin om in elk \emph{OECD}-land marginale belastingtarieven op te leggen
die in het decennium direct na de Tweede Wereldoorlog de 90 procent
naderden of zelfs overschreden.

Zoals we al bespraken, hadden de rijken nauwelijks een andere keus dan
zich neer te leggen bij dergelijke heffingen. De omstandigheden dwongen
hen immers hun veiligheid toe te vertrouwen aan regeringen die
grootschalig geweld konden beheersen. Het deed nauwelijks ertoe --
behoudens wellicht Britse agenten die de kans kregen in Hongkong te gaan
werken -- dat \emph{OECD}-regeringen monopolistische belastingen
oplegden. Mensen met een hoog verdienvermogen, die tijdens het
Industriële Tijdperk toonaangevende kansen wilden benutten, hadden bijna
geen andere keuze dan in een economie met hoge belastingen te blijven
wonen. Dat betekende dat zij een belastingdruk moesten dragen die
onevenredig was ten opzichte van de geleverde diensten.

\subsection{De rekenkunde van de
politiek}\label{de-rekenkunde-van-de-politiek}

De Amerikaanse vicepresident John~J.~Calhoun uit de negentiende eeuw
beschreef op slimme wijze de `rekenkunde' van de moderne politiek. In
zijn formule verdeelde hij de volledige bevolking van de natiestaat in
twee groepen: enerzijds de belastingbetalers, die meer bijdragen aan
overheidsuitgaven dan zij ervan profiteren, en anderzijds de
belastingconsumenten, die meer overheidsvoordelen ontvangen dan zij
bijdragen. Met enkele opvallende uitzonderingen waren tegen het einde
van de twintigste eeuw de meeste ondernemers in \emph{OECD}-landen in
hoge mate netto belastingbetalers. In 1996 droeg bijvoorbeeld het top 1
procent van de Britse belastingplichtigen 17 procent bij aan de totale
inkomstenbelastingdruk. Zij betaalden 30 procent meer dan de onderste 50
procent van de verdieners, die slechts 13 procent van de
inkomstenbelasting voor hun rekening namen. In de Verenigde Staten droeg
de elite zelfs nog zwaarder bij, waarbij het top 1 procent in 1994 28
procent van de totale inkomstenbelastingontvangsten betaalde.\footnote{David
  Smith, `Wat Clarke van Reagan zou kunnen leren,' \emph{The Sunday
  Times} (Londen), 16 juni 1996, p.~6.} De rijken moesten niet alleen
betalen voor overheidsdiensten die -- zoals Frederic C. Lane ons
herinnert -- `van slechte kwaliteit en belachelijk overgeprijsd' waren,
maar hun betalingen stonden vaak totaal los van de geleverde
diensten.\footnote{Lane, `Economische gevolgen van georganiseerd
  geweld,' p.~404.} Voordelen waarvoor zij betaalden, kwamen vaak
volledig bij anderen terecht. Meestal waren de welgestelden er zelfs mee
tevreden dat zij minder gebruik maakten van overheidsdiensten, die
doorgaans van lage kwaliteit bleken te zijn. Overheidsinstanties in
bijna elk land stonden erom bekend buitengewoon inefficiënt te werken,
vooral omdat zij vaak werden bestuurd door medewerkers die geen
aansporing hadden om de productiviteit te verbeteren. Volgens vrijwel
alle maatstaven betaalden de grootste belastingbetalers tijdens het
industriële tijdperk vele malen meer voor overheidsdiensten dan deze in
een concurrerende markt waard zouden zijn.

Dat bleef nauwelijks onopgemerkt. Helaas leverde de erkenning dat
betalingen aan de overheid voor bescherming -- in Lanes woorden
`verspilling volgens ideale normen' -- zijn, halverwege de twintigste
eeuw zelden bruikbare inzichten op. Integendeel, men accepteerde het
eenvoudigweg als een onvermijdelijk gebrek, als `een van de
verschillende vormen van verspilling die inherent zijn aan de sociale
organisatie.'\footnote{Ibid.}

Het alternatief voor de ontevredenen was niet om bijvoorbeeld van
Groot-Brittannië naar Frankrijk te verhuizen of van de Verenigde Staten
naar Canada. Behalve in uitzonderlijke gevallen zou zo'n verhuizing
weinig verschil maken. Alle leidende natiestaten kampten immers met
hetzelfde nadeel: zij voerden min of meer confiscatoire
belastingregimes. Om aanzienlijk meer autonomie te verkrijgen, moest men
zich volledig losmaken van de kernlanden in Europa en Noord-Amerika en
op zoek gaan naar de periferie. In delen van Azië, Zuid-Amerika en op
diverse afgelegen eilanden lag de belastingdruk aanzienlijk lager. Toch
moest men doorgaans een prijs betalen voor het ontlopen van roofzuchtige
belastingheffing, namelijk een verlies van economische kansen en vaak
een daling van de levensstandaard. Zoals we hebben aangetoond, waren de
economische mogelijkheden beperkt en bleken de levensstandaarden
ondermaats in de meeste rechtsgebieden buiten de
kernindustrie‑natiestaten die zich schuldig maakten aan confiscatoire
belastingen.

Neem de communistische systemen als voorbeeld. Net als veel
Derde-Wereldregimes legden zij doorgaans geen hoge inkomstenbelastingen
op -- soms zelfs helemaal niet.\footnote{Cuba introduceerde pas in 1996
  een inkomstenbelasting als noodmaatregel, als reactie op de
  economische depressie die ontstond nadat subsidies wegvielen door de
  ineenstorting van het communisme in Europa.} Desondanks zochten over
de bijna drie kwart eeuw dat de Sovjet-Unie bestond vrijwel geen
ondernemer naar belastingontwijking. Hoewel de
inkomstenbelastingtarieven in de Sovjet-Unie niet hoog waren, boden zij
geen voordeel, omdat de Sovjets er een deugd van maakten dat zij
eigendomsrechten niet erkenden. Dit legde een nog zwaardere last op dan
de gewone belastingdruk. De communistische systemen maakten het bijna
onmogelijk om een bedrijf op te zetten en er serieus geld mee te
verdienen. In feite nam de communistische staat het inkomen al in beslag
vóórdat er belastingen werden geheven.

Bovendien, als iemand -- om welke eigenzinnige reden dan ook -- met een
vast inkomen had besloten om in Moskou of Havana te gaan wonen, zou hij
moeite hebben gehad om een fatsoenlijke levensstandaard te bekostigen.
Buiten de toegang tot goede sigaren, kaviaar, voortreffelijke orkesten
en ballet bood het leven in de voormalige communistische systemen
nauwelijks consumentelijk genot. De schaarse geneugten bleven vaak
ontoegankelijk of werden streng gerantsoeneerd op basis van politieke
invloed in plaats van via vrije uitwisseling. Hoewel critici van het
postmoderne leven het `belang van consumptie in de postmoderne ervaring'
benadrukken, heeft de gestegen standaard van goederen en diensten die
sinds de val van het communisme wereldwijd beschikbaar is, er
ongetwijfeld toe bijgedragen dat de concurrentie tussen rechtsgebieden
levendiger werd, waardoor de banden met natie en plaats
verzwakten.\footnote{Ni. Featherstone, \emph{Consumer Culture and
  Postmodernism} (Londen: Sage, 1991), en J. F. Sherry, `Postmodern
  alternatief: de interpretatieve wending in consumentonderzoek,' in T.
  Robertson en H. Kassarjian (red.), \emph{Handbook of Consumer
  Research} (Englewood Cliffs, N.J.: Prentice-Hall, 1991), besproken in
  Billig, \emph{op. cit.}}

Onder het oude regime waren de keuzevrijheden voor consumenten zo
beperkt dat zelfs Castro moeite had kunnen hebben om een pakje
fatsoenlijke flosdraad te bemachtigen als hij cohiba-resten uit zijn
tanden wilde verwijderen. Tot voor kort mochten zelfs de rijken in vele
delen van de wereld niet genieten van de levensstandaard waaraan de
middenklasse in West-Europa of Noord-Amerika gewend was. Geconfronteerd
met deze benarde situatie kozen de meeste getalenteerde mensen tijdens
het industriële tijdperk voor de nationalistische koers. Ze bleven ter
plaatse en betaalden schandalig hoge belastingen voor de twijfelachtige
bescherming die hun natiestaat bood, een staat die het gebruik van
geweld als monopolie hanteerde binnen het grondgebied waarin zij geboren
werden.

\begin{quote}
`Het paradijs is nu gesloten en vergrendeld, buitengesloten door
engelen, dus moeten we nu verder, de wereld rondreizen en kijken of er
op de een of andere manier een achterdeur is.' HEINRICH VON KLEIST
\end{quote}

De val van het communisme heeft het `ijzeren gordijn' weggenomen dat het
reizen ooit belemmerde en de globalisering van de handel effectief
blokkeerde, waardoor de wereld kunstmatig `groot' bleef. Door de komst
van straalvliegtuigen en informatietechnologieën die het communisme
ondermijnden, is de wedijver om de welvarende reisdollars toegenomen. De
stoet bankiers die zelfs via de meest afgelegen provincies stroomden,
zorgde voor een enorme impuls aan zowel de woonstandaarden als de
culinaire scene wereldwijd. Hiermee bedoelen we niet de opmars van
\emph{McDonald's}-hamburgers en \emph{Kentucky Fried
Chicken}-franchises, zelfs niet in voorheen onherbergzame steden als
Moskou en Boekarest. Wat minder in het oog valt, maar des te
belangrijker is, is de opkomst van vooraanstaande hotelketens en
kwaliteitsrestaurants waar men in alle rust kan genieten van grand
cru-clarets in plaats van wodka met cola. Dankzij deze transformatie kan
iedereen die het zich kan veroorloven, bijna overal op de planeet
genieten van een hoge materiële levensstandaard. Het is tegenwoordig
immers zeldzaam een land aan te treffen waar niet ten minste één
eersteklas hotel en een restaurant is te vinden dat zelfs een
Michelin-inspecteur zou bekoren.

Zoals Hirschman al een kwart eeuw geleden voorzag, heeft technologische
vooruitgang de aantrekkingskracht van `exit' als oplossing voor
ontoereikende dienstverlening en prijsstelling aanzienlijk versterkt.
Hij schreef: `Loyaliteit aan je land is daarentegen iets waarvan we ook
zonder kunnen. \ldots{} Pas wanneer landen door de vooruitgang in
communicatie en algehele modernisering steeds meer op elkaar gaan
lijken, zal het gevaar van voortijdige en buitensporige exits ontstaan,
waarbij de 'brain drain' een actueel voorbeeld is.' Merk daarbij op,
zoals we in hoofdstuk 8 opmerkten, dat Hirschmans norm voor `voortijdige
en buitensporige exits' wordt benaderd vanuit het perspectief van de
ontdoening van de natiestaat en niet vanuit het oogpunt van het individu
dat op zoek is naar een beter bestaan.

Toch is zijn conclusie onmiskenbaar: hoe meer landen op elkaar gaan
lijken, des te aantrekkelijker defectie en exit worden. Dat het
tegenwoordig overal eenvoudiger is om goed te leven, maakt het
aantrekkelijk om je te vestigen op plekken waar de kosten het laagst
liggen. Nog belangrijker dan dat je vrijwel overal comfortabel kunt
wonen, is dat je tegenwoordig overal een hoog inkomen kunt verdienen. Je
hoeft niet langer in een dure jurisdictie te wonen om voldoende vermogen
op te bouwen om, zoals Lord Keynes adviseerde, `wijs, aangenaam en goed'
te leven. Zoals we eerder betoogden, verandert microtechnologie de
fundamentele megapolitieke basis waarop de natiestaat rust. In het
informatietijdperk ontstaat een nieuwe cybereconomie die geheel buiten
het bereik valt van elke overheid om te monopoliseren. Voor het eerst
stelt technologie individuen in staat rijkdom op te bouwen in een domein
dat zich niet zomaar laat onderwerpen aan de eisen van systematische
dwang.

De nieuwe samenleving -- en daarmee de bijbehorende cultuur -- krijgt
vorm door twee krachten. Enerzijds spelen machines een bepalende rol
doordat zij taken uitvoeren waar mensen niet meer tegenop kunnen, zoals
automatisering die steeds meer laaggeschoolde werkzaamheden overbodig
maakt. Anderzijds biedt informatietechnologie krachtige mogelijkheden
aan mensen met het talent om er optimaal van te profiteren. In zo'n
samenleving ontstaan grotere spanningen tussen een kleine elite -- die
je eventueel `informatiesterren' of `informatiaristocraten' kunt noemen
-- en een groeiende achterstandsklasse, de `informatiearmen'. Een
onderscheid tussen beide groepen is dat de informatiearmen vaak
vastzitten aan een specifieke locatie of weinig baat hebben bij een
verhuizing, terwijl informatiesterren, zoals elders al toegelicht,
buitengewoon mobiel zijn en overal kunnen verdienen waar zij zich
aangetrokken voelen -- vergelijkbaar met populaire romanschrijvers.
Honderd jaar geleden verdiende Robert Louis Stevenson zijn kost op een
eiland in de Stille Oceaan; tegenwoordig doet de informatiearistocratie
dat net zo goed.

\subsection{Marktconcurrentie tussen
rechtsgebieden}\label{marktconcurrentie-tussen-rechtsgebieden}

Doordat informatietechnologie de beperkingen van een vaste locatie
overschrijdt, stelt zij automatisch alle rechtsgebieden bloot aan reële
wereldwijde concurrentie op basis van kwaliteit en prijs. Overheden die
als lokale monopolen opereren, komen uiteindelijk de harde concurrentie
van de markt te lijf, afhankelijk van hoe goed zij hun klanten bedienen.
Zo wordt al snel duidelijk dat de oude logica -- waarbij in het
industriële tijdperk hoge kostenregimes voordelig waren -- inmiddels op
zijn kop staat. Vooraanstaande natiestaten, met hun roofzuchtige,
herverdelende belastingstelsels en strenge regelgeving, gelden niet
langer als aantrekkelijke rechtsgebieden. Onpartijdig gezien leveren zij
immers bescherming van mindere kwaliteit en beperkte economische kansen
tegen monopolistische prijzen. In de komende jaren kunnen zij sociaal
zelfs minder ontvankelijk en gewelddadiger uitpakken dan regio's in Azië
en Latijns-Amerika, waar de inkomenshistorie traditioneel ongelijker is.
De vooraanstaande verzorgingsstaten zullen hun meest getalenteerde
burgers verliezen doordat zij massaal het land verlaten.

\subsection{Het ``extranationale'' tijdperk dat voor ons
ligt}\label{het-extranationale-tijdperk-dat-voor-ons-ligt}

Naarmate het tijdperk van het `soevereine individu' vorm krijgt, zullen
steeds meer bekwame mensen stoppen met zichzelf te definiëren als
onderdeel van een natie -- als `brits', `amerikaans' of `canadees'. In
het nieuwe millennium ontdekken we een transnationaal of extranationaal
wereldbeeld en een geheel nieuwe manier om iemands plek daarin te
bepalen. Deze nieuwe identiteit komt niet voort uit de systematische
dwang die in de twintigste eeuw leidde tot de universele invoering van
natiestaten en het statelijke systeem.

Het feit dat we wereldomspannende ontwikkelingen vaak als
`internationaal' bestempelen, laat zien hoezeer het nationalistische
paradigma onze kijk op de wereld heeft geïnternaliseerd. Na twee eeuwen
waarin men werd onderwezen in de geheimen van `internationale
betrekkingen' en `internationaal recht', vergeet men al snel dat het
begrip `internationaal' geen eeuwenoud Westers concept is. Jeremy
Bentham bedacht het woord in 1789 en introduceerde het in zijn boek
\emph{An Introduction to the Principles of Morals and Legislation}. Hij
schreef: `Het woord internationaal, dat moet worden erkend, is nieuw, al
is gehoopt dat het voldoende analoog en begrijpelijk zal
zijn.'\footnote{Jeremy Bentham, \emph{An Introduction to the Principles
  of Morals and Legislation}, geredigeerd door J. H. Burns en H. L. A.
  Hart (London: Methuen, 1982), p.~296, geciteerd door Billig, \emph{op.
  cit.}, p.~84.} Het begrip sloeg snel aan en ver ging al snel verder
dan wat Bentham oorspronkelijk bedoelde; `internationaal' werd een los
synoniem voor alles wat zich wereldwijd afspeelt.

Het internationale tijdperk ging in 1789 van start, het jaar van de
Franse Revolutie, en hield twee eeuwen stand tot 1989, toen in Europa de
opstand tegen het communisme losbarstte. Wij zijn ervan overtuigd dat
die tweede revolutie het internationale tijdperk daadwerkelijk
beëindigde -- en niet louter omdat het in diskrediet geraakte
communistische volkslied `The International' op de voorgrond trad. De
centrale planeconomie, waarin de staat het eigendom beheerste, vormde de
meest ambitieuze uiting van de natiestaat. De hechte band tussen
staatsmacht en nationalisme kwam duidelijk tot uiting in de taal.
`Nationaliseren' werd het meest geladen werkwoord van de moderne tijd,
wat inhield dat iets onder staatsbezit en -controle kwam. Demagogen in
bijna alle delen van de wereld gebruikten het woord moeiteloos, maar
tegenwoordig hoort het enkel nog bij het vocabulaire van vervlogen
tijden. Nationalisatie past niet meer in onze tijd, omdat ook de
traditionele staatsmacht tot het verleden behoort.

Aan het einde van het moderne tijdperk verzwakten technologische
innovatie en marktkrachten stelselmatig de gecentraliseerde macht van de
staat. Nu breekt een nieuwe fase aan, waarin de markt zijn triomf
voltooit. Individuele natiestaten verdwijnen geleidelijk en wij
verwachten zelfs dat de club der natiestaten -- de Verenigde Naties --
spoedig failliet gaat. We zouden ons dan ook niet verbazen als men de VN
kort na de millenniumwisseling ontbindt.

Als we `internationaal' als een aandeel beschouwen, is het nu tijd om te
verkopen. In het nieuwe millennium zal men het begrip waarschijnlijk
vervangen of op zijn minst terugbrengen naar de oorspronkelijke
betekenis, want de wereld laat zich niet langer regeren door een systeem
van onderling verbonden soevereine natiestaten. Relaties krijgen de
nieuwe `extranationale' vorm, mede door het groeiende belang van
microjurisdicties en soevereine individuen. Een conflict tussen een
enclave aan de kust van Labrador en een soeverein individu zal dan niet
als een `internationaal' geschil worden bestempeld, maar als
extranationaal.

In de tijd die komt, bepalen territoriale grenzen niet langer de
samenstelling van gemeenschappen en loyaliteiten. Mensen voelen zich
vooral verbonden door oprechte affiniteiten, gedeelde overtuigingen,
gemeenschappelijke belangen en zelfs gezamenlijke genetica, in plaats
van door de kunstmatige band die nationalisten zo nadrukkelijk
waarderen. Ook organiseren we onze bescherming op nieuwe manieren die
niet langer af te bakenen zijn met een sextant, een loodlijn of andere
instrumenten uit de vroegmoderne tijd die ooit territoriale grenzen
markeerden.

\section{Uitgevonden gemeenschappen en
tradities}\label{uitgevonden-gemeenschappen-en-tradities}

De kosmopolitische elite bestempelt in de komende eeuw het idee dat
mensen zichzelf van nature in een `uitgevonden' gemeenschap -- oftewel
een natie -- plaatsen als excentriek en onredelijk, net zoals men dit
gedurende het grootste deel van de menselijke geschiedenis al deed.
Zoals socioloog Anthony Giddens schreef, kent de natiestaat `geen
precedent in de geschiedenis.'\footnote{Anthony Giddens, \emph{Social
  Theory and Modern Sociology} (Cambridge: Polity Press, 1987), p.~166,
  geciteerd in Billig, \emph{op. cit.}} Michael Billig, een autoriteit
op het gebied van nationalisme, versterkte dat punt:

\begin{quote}
In vroegere tijden hechtten mensen niet dezelfde betekenis aan begrippen
als taal en dialect, laat staan aan die van grondgebied en
soevereiniteit -- begrippen die tegenwoordig zo vanzelfsprekend en
tastbaar lijken. Zulke ideeën liggen zo diep verankerd in het
hedendaagse gezond verstand dat men gemakkelijk vergeet dat ze feitelijk
verzonnen constructies zijn. De middeleeuwse schoenmakers in de
werkplaatsen van Montaillou of San Mateo zouden ons, na een
tussenperiode van 700 jaar, als beperkte, door bijgeloof doordrenkte
figuren zien. Zij zouden onze opvattingen over taal en natie als vreemd
mystiek beschouwen en zich verbazen dat zo'n mystiek een kwestie van
leven en dood kan zijn.\footnote{Billig, \emph{op. cit.}, p.~36.}
\end{quote}

\begin{quote}
Wij vermoeden dat ook denkers in een extranationale toekomst evenzeer
verbaasd zullen zijn. Zoals Benedict Anderson opmerkte, zijn naties
`ingebeelde gemeenschappen.'\footnote{Benedict Anderson, \emph{Imagined
  Communities} (London: Verso, 1983), geciteerd door Billig, \emph{op.
  cit.}, p.~10.} Dit wil niet zeggen dat wat ingebeeld is per se
onbeduidend is. Zoals Dr.~Johnson zei: zonder verbeelding zou een man
net zo graag `bij een kamermeisje liggen als bij een hertogin.' Toch
lijken naties voor degenen die in de twintigste eeuw opgroeiden zo'n
onvermijdelijke organisatie-eenheid te vormen dat het moeilijk te
bevatten is dat ze in feite louter ingebeeld zijn en niet natuurlijk. Om
te begrijpen hoe anders de toekomst kan zijn dan de wereld die ons
vertrouwd is, moeten we onderzoeken hoe het nationalisme eens aan het
`gezond verstand' van het Industriële Tijdperk werd opgelegd.
\end{quote}

\begin{quote}
Het valt gemakkelijk over het hoofd te zien in hoeverre de `nationale
gemeenschap' tot stand komt door een voortdurende inzet van verbeelding.
Er bestaan immers geen objectieve criteria om nauwkeurig vast te stellen
welke groep wel of geen `natie' behoort te zijn. Ook bestaan er, strikt
genomen, geen `natuurlijke grenzen', zoals de vooraanstaande historici
Owen Lattimore en C.~R. Whittaker hebben aangetoond. `Een belangrijke
keizerlijke grens is, aldus Lattimore, niet louter een lijn die
geografische gebieden en menselijke samenlevingen scheidt. Zij
vertegenwoordigt ook de optimale limiet van de groei van een bepaalde
samenleving.'\footnote{Owen Lattimore, \emph{Inner Asian Frontiers of
  China} (New York: Beacon Press, 1960), p.~60. Geciteerd door Ronald
  Findlay in `Towards a Model of Territorial Expansion and the Limits of
  Empire', opgenomen in Michelle R. Garfinkel en Stergios Skaperdas
  (red.), \emph{The Political Economy of Conflict and Appropriation}
  (Cambridge: Cambridge University Press, 1996), p.~54.} Of, zoals
econoom Ronald Findlay van \emph{Columbia University} het verwoordde:
`Voor zover men er in de economie überhaupt aandacht aan besteedt,
worden de grenzen van een bepaald economisch systeem of `land' doorgaans
als gegeven beschouwd, evenals de bevolking die binnen die grenzen
leeft. Toch is het overduidelijk dat, hoe gewijde deze grenzen ook
mochten zijn geworden in het internationale recht, zij op een gegeven
moment altijd betwist werden door rivaliserende claimanten en
uiteindelijk bepaald werden door het evenwicht tussen economische en
militaire macht van de betrokken partijen.'\footnote{Findlay, \emph{op.
  cit.}, p.~41.}
\end{quote}

\begin{quote}
Zelfs iemand die over alle gegevens beschikt over de helft van de
natiestaten wereldwijd en een indrukwekkende verzameling
satellietbeelden bezit, zou er niet in slagen te voorspellen waar de
grenzen van de overige staten liggen. Er bestaat bovendien geen
wetenschappelijke methode waarmee men op basis van biologische of
taalkundige criteria de leden van de ene nationaliteit van die van een
andere kan onderscheiden. Zelfs de meest geavanceerde autopsietechnieken
slagen er niet in om genetisch onderscheid te maken tussen de
overblijfselen van Amerikanen, Canadezen en Soedanese na een
vliegtuigongeluk. De grenslijnen tussen staten en nationaliteiten zijn
niet natuurlijk, zoals dat wel geldt voor de grenzen tussen soorten of
de fysieke verschillen tussen dierlijke rassen. Integendeel: zij vormen
artefacten van zowel vroegere als voortdurende pogingen tot
machtsprojectie.
\end{quote}

\begin{quote}
``Een taal is een dialect met een leger en een marine'' - MARIO PEI
\end{quote}

\section{talen als artefacten van
macht}\label{talen-als-artefacten-van-macht}

Opmerkelijk genoeg geldt voor talen ongeveer hetzelfde principe. Na
eeuwenlange dominantie door natiestaten lijkt het onbezonnen -- of zelfs
absurd -- te beweren dat `taal' geen objectieve basis biedt om volkeren
van elkaar te onderscheiden. Maar bekijk het eens goed: de geschiedenis
van de moderne talen toont duidelijk hoe bewust ze zijn vormgegeven om
de nationalistische identiteit te versterken. De westerse `talen', zoals
wij ze vandaag de dag spreken en begrijpen, hebben zich niet op
natuurlijke wijze ontwikkeld tot hun huidige vorm. Men kan ze bovendien
niet objectief onderscheiden van `dialecten'. Tegenwoordig kiest vrijwel
niemand ervoor een `dialect' te spreken; bijna iedereen verkiest dat
zijn moedertaal als authentiek wordt gezien -- als een volwaardige
`taal'.

\begin{quote}
`Laat niemand zeggen dat het woord in zulke momenten weinig nut heeft.
Woord en daad vormen samen één. De krachtige, energieke bevestiging die
harten geruststelt, schept daden -- dat wat gezegd wordt, wordt
gerealiseerd. De daad is hier de dienaar van het woord; hij volgt
onderdanig, zoals op de eerste dag van de wereld: Hij zei, en de wereld
was.' -MICHELET, augustus 1792
\end{quote}

\subsection{`woord en daad vormen samen
één'}\label{woord-en-daad-vormen-samen-uxe9uxe9n}

Alvorens de Franse Revolutie uitbrak, sprak men in Zuid-Frankrijk een
mengvorm van het Latijn, de la langue d'oc -- oftewel het Occitaans --
die meer weg had van de volkstaal in Catalonië dan van la langue d'oil,
de Parijse taal waaruit het moderne `Frans' is ontstaan. Toen de
`Verklaring van de Rechten van de Mens en de Burger' in Parijse stijl
werd gepubliceerd, begreep de meerderheid van de mensen die binnen de
huidige landsgrenzen van Frankrijk leefde de tekst
nauwelijks.\footnote{Billig, \emph{op. cit.}, p.~25.} De revolutionairen
stonden voor de uitdaging hun pamfletten en edicten te vertalen naar het
patois van talloze dorpen, die elkaar slechts vaag konden verstaan.

De inwoners van het gebied dat later Frankrijk zou worden, hanteerden
zeer uiteenlopende spreekwijzen. Zij kozen er bewust voor deze te
bundelen tot één officiële taal als beleidsoptie. Sinds Franciscus I in
1539 het Edict van Villers-Cotterets uitvaardigde en daarmee het
geschreven Frans als de officiële taal van de rechtbanken
instelde,\footnote{Anderson, \emph{op. cit.}, p.~93.} betekende dit nog
lang niet dat iedereen de taal kon begrijpen -- net zoals het `juridisch
Frans' in Engeland, dat na 1200 als officiële rechtbankstaal werd
gebruikt, niet breed werd verstaan. Beide vormen dienden immers louter
als administratieve volkstalen, niet als een gestandaardiseerde taal die
over het gehele grondgebied leefde.

De revolutionairen wilden meer: zij streefden naar een echte nationale
taal. Historicus Janis Langins merkt in \emph{The Social Histor,' of
Language} op: `een invloedrijke groep revolutionairen was ervan
overtuigd dat het succes van de Revolutie en de verlichting zou
bevorderen als men opzettelijk een standaard Frans oplegde in het
grondgebied van de Republiek.'\footnote{Janis Langins, `woorden en
  instellingen tijdens de Franse revolutie: het geval van
  ``revolutionair'' wetenschappelijk en technisch onderwijs', opgenomen
  in Peter Burke en Roy Porter, \emph{de sociale geschiedenis van de
  taal} (Cambridge: Cambridge University Press, 1987), p.~137.} Deze
bewuste inspanning leidde tot verhitte discussies over het gebruik van
specifieke woorden. Zo ontstond bijvoorbeeld het aansprekende
bijvoeglijk naamwoord `revolutionair', dat in 1789 voor het eerst door
Marabou werd ingezet. Na een periode van `enigszins wijd en
ongedifferentieerd gebruik', zoals Langins beschrijft, volgde tijdens de
Terreur een decennialange fase waarin het woord onderdrukt en vergeten
werd. Op 12 juni 1795 besloot de Conventie de taal -- evenals de
instellingen die onze voormalige tirannen {[}dat wil zeggen de verslagen
Robespierre-aanhangers{]} hadden opgericht -- te hervormen door het
woord `revolutionair' in officiële benamingen te vervangen.\footnote{Ibid.,
  pp.~140, 142.} Tot op de dag van vandaag blijkt deze traditie van
taalinterventie in de kieskeurige houding van de Franse autoriteiten
tegenover leenwoorden als `weekend', die vanuit het Engels hun weg naar
het Frans hebben gevonden.

Twee eeuwen geleden richtten de nationale taalplanners in Frankrijk zich
niet alleen tegen woorden uit het overkantse Engeland; zij kregen de
veel grotere taak om alle lokale spraakvarianten binnen het grondgebied
van de Republiek uit te bannen. Hun inspanning beperkte zich niet louter
tot het onderdrukken van la langue d'oc. Het `Frans' dat destijds aan de
Rivièra werd gesproken, deed meer denken aan het `Italiaans' dat in
oostelijke gebieden werd gehanteerd dan aan het Parijse Frans. Evenzo
had men de taal van de Elzas makkelijk kunnen beschouwen als een vorm
van Duits, dat zelf weer boordevol regionale varianten zat. In de
Pyreneeën hoorde men Baskisch, en net als het Breton dat langs de
noordwestkust van Frankrijk gesproken werd, had het Baskisch weinig weg
van de volkstaaldialecten van het Latijn waaruit het `Frans' is
voortgekomen. Ook in het noordoosten woonde een aanzienlijk aantal
Vlamingstaligen. `De Parijse spreekstijl', zoals Michael Billig ons
herinnert, verspreidde zich niet via spontane marktprocessen, maar werd
wettelijk en cultureel opgelegd als `Frans'.\footnote{Billig, \emph{op.
  cit.}, p.~27.}

Wat in Frankrijk gold, bleek ook elders geldig te zijn bij het ontstaan
van natiestaten. Talen verspreidden zich vaak via militaire campagnes en
werden door koloniale machten opgelegd. Zo werd bijvoorbeeld, na de
onafhankelijkheid, de kaart van Afrika getekend op basis van de gebieden
waar de administratieve talen van de Europese mogendheden domineerden.
Lokale dialecten kwamen zelden aan bod op scholen. Het verschil tussen
erkende `talen' -- die vaak dienen als basis voor de definitie van
`naties', zelfs bij natiestaten met willekeurige koloniale grenzen -- en
`dialecten' had in wezen een politieke lading.

Kortom, het opleggen van een `nationale taal' maakte deel uit van een
mondiaal proces om de staatsmacht te versterken. Door alle inwoners
binnen een grondgebied, waar de staat het geweld monopoleerde, te
verplichten of aan te moedigen de `moedertaal' te gebruiken, verwierven
de machthebbers aanzienlijke voordelen bij het uitoefenen van hun macht.

\subsection{De militaire dimensie van
taaluniformiteit}\label{de-militaire-dimensie-van-taaluniformiteit}

In een wereld waarin geweld steeds lonender werd, leverde het invoeren
van een nationale taal duidelijke militaire voordelen op. Een nationale
taal vormde vrijwel de noodzakelijke basis voor het versterken van de
centrale macht in natiestaten. Door hun burgers te stimuleren dezelfde
taal te spreken, konden centrale autoriteiten de militaire kracht van
lokale potentaten effectief ondermijnen. De standaardisering van de taal
na de Franse Revolutie maakte de inzet van nationale dienstlegers -- de
goedkoopste en meest efficiënte vorm van moderne militaire macht --
mogelijk. Een gemeenschappelijke taal stelde troepen uit alle delen van
de `natie' in staat om vlot met elkaar te communiceren. Dit bleek
essentieel voordat nationale dienstlegers de onafhankelijke bataljons,
die door machtige lokale potentaten in plaats van door centrale
autoriteiten werden gerekruteerd en aangestuurd, konden verdringen.

Voor de Franse Revolutie, zoals we in hoofdstuk 5 bespraken, werden
troepen opgevoerd en geleid door lokale potentaten, die al dan niet
reageerden op strijdoproepen uit Parijs of een andere hoofdstad. Hoe dan
ook, hun positie werd na grondige onderhandelingen vastgesteld. Zoals
Charles Tilly opmerkt: `het vermogen om steun te geven of te
onthouden\ldots{} grote onderhandelingsmacht' \footnote{Tilly,
  \emph{Coercion, Capital, and European States}, p.~22.}. Bovendien
vormden de onafhankelijke militaire eenheden een extra uitdaging voor
centrale autoriteiten, omdat zij zich konden verzetten tegen pogingen
van de overheid om binnenlandse middelen in beslag te nemen. Het is
duidelijk dat centrale autoriteiten -- of het nu de koning of de
revolutionaire conventie was -- voor een zware opgave stonden om
belastingen te innen of op andere manieren de middelen van lokale
potentaten, die met privélegers hun bezittingen konden verdedigen, te
bemachtigen.

Nationale legers vergrootten de macht van de centrale overheid
aanzienlijk, omdat zij haar in staat stelden haar wil over een
uitgestrekt grondgebied door te voeren. Het invoeren van een nationale
taal speelde daarbij een sleutelrol, aangezien het de totstandbrenging
van zulke legers mogelijk maakte. Voordat deze legers konden ontstaan en
daadwerkelijk effectief opereren, moesten hun leden elkaar immers
vloeiend verstaan.

Het bood een strategisch voordeel als iedereen binnen een rechtsgebied
bevelen en instructies begreep en bovendien via de hiërarchische keten
essentiële informatie kon terugkoppelen. De Franse revolutionairen
lieten vrijwel direct zien hoe waardevol dit was. Naast het runnen van
een soort taalschool organiseerden zij intensieve, een maand durende
cursussen, waarin -- volgens Langins -- `honderden studenten uit heel
Frankrijk werden opgeleid in de technieken van buskruit- en
kanonfabricage.'\footnote{Langins, \emph{op. cit.}, p.~143.}

Het militaire voordeel van de Franse aanpak bleek niet alleen uit hun
successen in de Napoleontische periode, maar ook uit de ervaringen van
regimes die tijdens oorlog geen baat hadden bij een gemeenschappelijke
taal. Zo droeg het aristocratische officierenkorps van de tsaar, dat
doorgaans in het Duits communiceerde (de andere hoftaal van de Romanovs
was Frans), er mede toe dat gewone soldaten en de burgerij de bevelen
niet begrepen -- een van de vele factoren die in de vroege dagen van de
Eerste Wereldoorlog leidden tot desastreuze nederlagen en een
demoraliserende stemming onder de Russische troepen.

Dit benadrukt een ander belangrijk militair voordeel van een
gemeenschappelijke taal: het verlaagd de drempel om oorlog te voeren.
Propaganda verliest immers zijn effect als de boodschap onbegrijpelijk
is. Ook op dit vlak twijfelden de Franse revolutionairen niet aan de
mogelijkheden. Volgens Langins lag hun overkoepelende gedachte in `de
wil der mensen'. Zij zorgden ervoor dat de volkse wil tot uiting kwam in
de taal van het volk. Vóór 1789 belemmerde het gebrek aan onderlinge
taalbegrip tussen burgers het gezamenlijk uiten van hun wil, wat de
uitoefening van nationale macht remde. Meertalige staten en rijken
ondervonden tijdens de industriële periode bovendien extra obstakels bij
de mobilisatie voor oorlog.

Op zij­niveau vervingen landen die intern meertalig waren, vaak
natiestaten die hun burgers beter konden enthousiasmeren voor de strijd
en efficiënter middelen konden mobiliseren. Dit blijkt uit
nationalistische consolidatie, zoals de vorming van Frankrijk en de
Fransen aan het eind van de achttiende eeuw, en uit voorbeelden van
nationalistische decentralisatie, zoals de ineenstorting van het
Oostenrijks-Hongaarse Rijk na de Eerste Wereldoorlog. De nieuwe
natiestaten die na het Habsburgse Rijk ontstonden -- Oostenrijk,
Hongarije, Tsjechoslowakije en Joegoslavië -- bestempelde Keynes als
`onvolledig en onvolwassen.' Toch overtuigden de argumenten voor
onafhankelijke natiestaten, gebaseerd op nationale identiteiten die voor
een deel door taal werden gedefinieerd, onder andere Woodrow Wilson en
andere geallieerde leiders bij de totstandkoming van het Verdrag van
Versailles.

De opdeling van Centraal-Europa na de Eerste Wereldoorlog laat zien hoe
taal als tweesnijdend zwaard werd ingezet bij staatsvorming. Toen het
geweld toenam, maakte een gemeenschappelijke taal de machtsuitoefening
eenvoudiger en verstevigde het de rechtsorde. Maar wanneer de prikkels
voor consolidatie verzwakten, veroorzaakten taalkundige minderheden vaak
dat meertalige staten uiteenvielen.

In het midden van de negentiende eeuw sloegen separatistische gevoelens
in de steden van het Oostenrijks-Hongaarse Rijk aan, nadat epidemieën de
Duitssprekende bevolking zwaar hadden getroffen. Praag was aan het begin
van de negentiende eeuw een Duitssprekende stad. Net als andere
stedelijke centra groeide Praag de loop van de eeuw snel, voornamelijk
door migratie: enorme aantallen landloze, Tsjechischsprekende boeren
trokken de stad binnen en werden geassimileerd. Aanvankelijk leerde men
vanuit noodzaak Duits, maar halverwege de eeuw, toen hongersnoden en
ziekte-uitbraken grote groepen Duitssprekende stadsbewoners wegvaagden,
werden zij opgevolgd door Tsjechischsprekenden. Plotseling woonden er
zoveel Tsjechischsprekenden in de stad dat de nieuwkomers niet langer
afhankelijk waren van de Duitse taal. Praag transformeerde tot een
overwegend Tsjechisch sprekende stad en werd een broedplaats voor
Tsjechisch nationalisme.

Tegenwoordig ontstaan separatistische bewegingen vaak rond taalkwesties
in meertalige landen. Dit is duidelijk in België en Canada, twee landen
die -- zoals eerder opgemerkt -- waarschijnlijk tot de eerste van de
OESO-lande behoren die in het nieuwe millennium uiteenvallen.

Weinige regeringen kunnen tippen aan de harde maatregelen die door de
\emph{Parti Québécois} in Québec worden opgelegd om taaleenheid af te
dwingen.\footnote{Rheal Seguin, `PQ staat op het punt de taalwetten aan
  te scherpen: in Quebec worden Engelse borden verboden', \emph{Globe
  and Mail}, 29 augustus 1996, p.~A1.} Wat nog verrassender is, is dat
taalklachten ook een rol speelden bij de opstart van de activiteiten van
de noordelijke separatisten in Italië, een land dat eveneens met
desintegratie kampt. Begin jaren tachtig verklaarde de Lombardische Liga
-- zoals ze destijds heette -- dat Lombardisch een zelfstandige taal
was, los van het Italiaans. Billig merkt op: `Als het programma van de
Liga succesvol was geweest in de vroege jaren tachtig en Lombardije zich
had afgescheiden van Italië en eigen staatsgrenzen had vastgesteld, had
men kunnen voorspellen dat Lombardisch steeds meer als verschillend van
het Italiaans zou worden erkend.'\footnote{Billig, \emph{op. cit.},
  p.~35.} Deze bewering is niet willekeurig, maar weerspiegelt wat er in
soortgelijke situaties is gebeurd. Zo gingen Noorse nationalisten, toen
Noorwegen in 1905 onafhankelijk werd, doelbewust te werk om kenmerken
van de `Noorse taal' te benoemen en te benadrukken die afweken van het
Deens en het Zweeds. Evenzo vervingen activisten die pleitten voor een
onafhankelijk Belarus de verkeersborden door `Witrussisch' te hanteren,
al slaagden zij er kennelijk niet in duidelijk te maken dat Witrussisch
een eigen taal is en geen dialect van het Russisch.

Nu de militaire drijfveren voor taaleenheid grotendeels tot het verleden
behoren, verwachten we dat nationale talen langzaam zullen wegkwijnen --
maar dan niet zonder weerstand. We mogen verwachten dat het bekende
gezegde `oorlog is de gezondheid van de staat' opnieuw zal worden
aangewend als middel tot herstel. Naarmate de relevantie van de
natiestaat afneemt, zullen demagogen en reactionairen oorlogen en
conflicten aanwakkeren langs etnische en tribale lijnen, zoals ooit in
het voormalige Joegoslavië en in talloze Afrikaanse landen, van Burundi
tot Somalië. Conflicten blijken nuttig, omdat zij een dekmantel bieden
voor hen die de trend naar de commercialisering van soevereiniteit
willen tegenhouden. Oorlogen vergemakkelijken tevens het handhaven van
strengere belastingregimes en het opleggen van zwaardere straffen voor
het ontlopen van de burgerlijke plichten en lasten. Daarbij versterken
zij de `wij-versus-hij'-mentaliteit binnen het nationalisme. Voor de
aanhangers van systematische dwang lijkt de commerciële soevereiniteit
-- waarbij individuen soevereiniteitsdiensten kunnen kiezen op basis van
prijs en kwaliteit -- even zondig als de opvatting dat mensen het recht
hebben de uitspraken van de paus te vetoën en tijdens de Reformatie hun
eigen weg naar redding te kiezen.

De gelijkenis valt op doordat zowel de nieuwe druktechnologie aan het
eind van de vijftiende eeuw als de informatietechnologie aan het eind
van de twintigste eeuw voorheen verborgen kennis op een bevrijdende
manier toegankelijk maakten voor individuen. Dankzij de boekdrukkunst
kwamen de Schrift en andere heilige teksten rechtstreeks beschikbaar
voor mensen, die tot dan toe afhankelijk waren geweest van priesters en
de kerkhierarchie om het Woord van God te interpreteren. De
informatietechnologie maakt het iedereen met een computerverbinding
mogelijk om informatie op te vragen over handel, investeringen en
actuele gebeurtenissen -- informatie die voorheen enkel voor personen
aan de top van overheids- en bedrijfsstructuren beschikbaar was.

\begin{quote}
`{[}T{]}e ontwikkeling van de drukkunst en de uitgeefkunst maakte het
nieuwe nationale bewustzijn mogelijk en stimuleerde de opkomst van
moderne natiestaten.'\footnote{Jack Weatherford, \emph{Savages and
  Civilization: Who Will Survive?} (New York: Fawcett Columbine, 1994),
  p.~143.} - Jack Weatherford
\end{quote}

\subsection{Rock and roll in
cyberspace}\label{rock-and-roll-in-cyberspace}

Vergeet niet dat de komst van het internet en het \emph{World Wide Web}
het nationalisme net zo hard zal ondermijnen als de introductie van
buskruit en de drukpers destijds deden. Wereldwijd verbonden computers
brengen het Latijn niet terug als universele taal, maar zorgen er wel
voor dat de handel verschuift van lokale dialecten -- zoals het Frans in
Quebec -- naar de nieuwe, wereldwijde taal van het internet en het
\emph{World Wide Web}: de taal van rock-'n-roll, Engels, zoals Otis
Redding en Tina Turner aan de wereld hebben laten zien.

Deze nieuwe media ondermijnen het nationalisme door grensoverschrijdende
verbondenheden te scheppen. Ze spreken brede doelgroepen aan, vooral op
die plaatsen waar hoogopgeleiden zich hebben verzameld. Zulke
niet-territoriale verbondenheden kunnen floreren en daarmee een basis
leggen voor een nieuw soort `patriottisme'. Of, om het anders te
verwoorden, ze vormen nieuwe `in-groepen' waarmee mensen zich kunnen
identificeren zonder daarbij hun economische rationaliteit op te
offeren. De geschiedenis van de joden in de afgelopen tweeduizend jaar
leert dat dit op de lange termijn mogelijk is, zelfs in vijandige lokale
omgevingen. Zoals William Pfaff aan het begin van dit hoofdstuk
opmerkte, is het historisch onjuist te veronderstellen dat loyaliteit
aan het land van de vaderen -- de patria -- per definitie trouw betekent
aan een instelling die op een natiestaat lijkt. Geoffrey Parker en
Lesley M. Smith geven in \emph{The General Crisis of the Seventeenth
Century} nog een duidelijker beeld: wat op het eerste gezicht
voorbeelden lijken van vroegmodern nationalisme, betreft in
werkelijkheid vaak patriotten die een veel nauwere patria verdedigen --
vaak tegen de binnendringing van een staat. Zij schrijven: `Al te vaak
blijkt een vermeende verbondenheid met een nationale gemeenschap, bij
nadere inspectie, niets dergelijks te zijn. De patria is minstens even
goed een stad of provincie als de gehele natie.'\footnote{Geoffrey
  Parker en Lesley M. Smith, \emph{The General Crisis of the Seventeenth
  Century} (London: Routledge \& Kegan Paul, 1985), p.~122.}

Zoals Jack Weatherford duidelijk toelicht in \emph{Savages and
Civilization} leidde de opkomst van de drukpers -- de eerste
massaproductietechnologie -- tot ingrijpende veranderingen bij het
ontstaan van de politiek, waarbij men steeds meer loyaliteit eiste aan
een bredere natiestaat. In 1500 waren in Europa drukpersen actief op 236
locaties, waarmee samen ongeveer 20 miljoen boeken werden
gedrukt.\footnote{Weatherford, \emph{op. cit.}, p.~144.} Gutenberg's
eerste gedrukte boek was een Latijnse Bijbeluitgave; daarop volgden
edities van andere populaire middeleeuwse teksten in het Latijn.
Weatherford laat zien dat de ontwikkeling van de drukkunst niet verliep
zoals men aanvankelijk verwachtte, namelijk dat de gemakkelijke
beschikbaarheid van teksten automatisch het gebruik van het Latijn -- en
zelfs het Grieks -- zou verspreiden. Integendeel, twee belangrijke
oorzaken verklaren waarom de drukpers het Latijngebruik niet bevorderde.

Ten eerste maakte de drukpers massaproductie mogelijk. Zoals Benedict
Anderson opmerkt: `{[}A{]}ls manuscriptkennis schaars was en als arcane
wijsheid gold, leefde gedrukte kennis van reproduceerbaarheid en
verspreiding.'\footnote{Anderson, \emph{op. cit.}, p.~90.} In 1500
spraken weinigen in Europa meerdere talen, waardoor de doelgroep voor
werken in het Latijn beperkt bleef. De overgrote meerderheid, die
slechts één taal beheerste, vormde juist een veel grotere markt van
potentiële lezers. Wat hierbij voor lezers geldt, geldt nog sterker voor
schrijvers: uitgevers hadden continu producten nodig om te verkopen.
Omdat er in de vijftiende- en zestiende eeuw maar weinig auteurs waren
die bevredigende nieuwe werken in het Latijn konden samenstellen, dwong
de marktvraag uitgevers ertoe over te stappen op publicaties in de
volkstaal. Zo droeg de drukkunst bij aan een taalkundige differentiatie
in Europa. Dit proces werd gestimuleerd door zowel de publicatie van
nieuwe werken die de identiteit van opkomende talen -- zoals het Spaans
en Italiaans -- bevestigden als door de introductie van karakteristieke
lettertypen, waaronder Romeins, Italic en de zware Gotische schriftstijl
die in Duitse uitgaven tot ver in de twintigste eeuw gangbaar bleef. De
opkomst van volkstaalpublicaties -- wat Anderson aanduidt als
`printkapitalisme' -- boekte immens succes. Opvallend daarbij was dat de
drukpers de ketterij de impuls gaf die we tegenwoordig associëren met de
denationalisering van het individu via het internet. Zo werd Luther `de
eerste bestverkopende auteur zoals we die kennen. Of om het anders te
zeggen, de eerste schrijver die zijn nieuwe boeken op basis van zijn
naam kon verkopen.'\footnote{Ibid., p.~91.} Opmerkelijk is dat Luther's
werken `niet minder dan een derde van alle Duitstalige boeken die tussen
1518 en 1525 werden verkocht' vertegenwoordigden.\footnote{Ibid.}

In veel opzichten zal de informatietechnologie van het digitale tijdperk
een deel van de megapolitieke impact van de vijftiende-eeuwse drukpers
tenietdoen bij het stimuleren en ondersteunen van de opkomst van
natiestaten. Het \emph{World Wide Web} schept een commerciële arena
waarin Engels dienstdoet als wereldtaal. Uiteindelijk zal
simultaan-vertalingssoftware deze ontwikkeling versterken, zodat bijna
iedereen effectief meertalig wordt -- wat bijdraagt aan de
denationalisering van taal en verbeelding. Net zoals de drukpers de
loyaliteit aan de dominante instelling van de middeleeuwen, de Heilige
Moederkerk, ondermijnde, verwachten wij dat de moderne
communicatietechnologie de autoriteit van de `nannystaat' zal aantasten.
Op den duur zal vrijwel elk gebied meertalig worden, lokale dialecten
zullen belangrijker worden en de samenhang van centrale propaganda
verzwakt, naarmate immigranten en sprekers van minderheidstalen de moed
vinden zich tegen assimilatie in te zetten.

\section{Militaire mystiek}\label{militaire-mystiek}

Naties vormen bij lange na geen objectieve gemeenschappen zoals
jager-verzamelaarsgroepen dat wel doen; in plaats daarvan ontstaan
naties uit een mystiek dat geworteld is in een inmiddels verdwenen
militair imperatief. Dat imperatief wilde iedere inwoner van een bepaald
grondgebied verbinden aan een identiteit die men als belangrijker
beschouwde dan het eigen leven. Zoals Kantorowicz opmerkte, is het geen
toeval dat de staat op een gegeven moment in de geschiedenis -- in
abstracte vorm of als corporatie -- verscheen als een corpusmystiek, en
dat de dood van dit nieuwe mystieke lichaam dezelfde betekenis leek te
hebben als de dood van een kruisvaarder voor de goddelijke
zaak!\footnote{E. H. Kantorowicz, geciteerd door Llobera, \emph{op.
  cit.}, p.~83.} In die zin kun je de natiestaat beschouwen als een
mystieke constructie. Maar zoals Billig aangeeft, betreft nationalisme
`een banale mystiek, zo alledaags dat het mystieke schijnbaar al lang is
verdampt.' Het bindt `ons' aan het vaderland -- die bijzondere plek die
meer is dan slechts een geografisch gebied. Het vaderland wordt
voorgesteld als iets huiselijks, vanzelfsprekend en, indien nodig, zelfs
opofferingswaardig. Vooral mannen koesteren hun eigen, met vreugde
doordrenkte herinneringen aan de mogelijkheden tot opoffering.

Nationalisten benadrukken steevast de denkbeeldige band tussen de natie
en het thuis. Volgens Billig zien zij de natie als een huiselijke
omgeving, knus binnen haar grenzen en beschermd tegen een bedreigende
buitenwereld. Wij -- als leden van het vaderland -- stellen onszelf
hierdoor al snel voor als een hechte familie. De clichés van het
nationalisme, onvermoeibaar en routinematig herhaald, omvatten talloze
alledaagse metaforen van verwantschap en identiteit. Ze verbinden de
natie met het persoonlijke gevoel van `inclusieve fitheid', een
krachtige drijfveer voor altruïsme en opoffering.

\begin{quote}
`Dat opofferingsaltruïsme daadwerkelijk bestaat bij sociale insecten,
andere niet-menselijke dieren en mensen, impliceert dat maximalisatie
van eigenbelang niet louter kan worden gedefinieerd in termen van de
wensen en behoeften van een individueel organisme. Inderdaad, de
aanwezigheid van altruïsme, met name jegens verwanten, heeft een
volledige heroverweging van de traditionele opvattingen over het
overleven van de best aangepaste in de biowetenschappen noodzakelijk
gemaakt. Dit heeft geleid tot een groeiende overtuiging dat natuurlijke
selectie uiteindelijk niet op het individu opereert.'\footnote{Shaw en
  Wong, \emph{op. cit.}, pp.~26--27.} - R. Paul Shaw en Yuwa Wong
\end{quote}

\section{Nationalisme en inclusieve
fitheid}\label{nationalisme-en-inclusieve-fitheid}

In dit boek richten we ons vooral op de objectieve ``megapolitieke''
factoren die de kosten en beloningen van menselijke keuzes beïnvloeden.
We gaan ervan uit dat mensen voortdurend op zoek zijn naar
aantrekkelijke beloningen en kosten proberen te vermijden -- een
fundamentele waarheid zoals Charles Darwin `de economie van de natuur'
noemde. Hoewel deze visie veel inzicht biedt, verklaart het najagen van
beloningen niet alles in het leven. Wel verheldert het twee van de drie
belangrijkste vormen van menselijke socialiteit, die Pierre Van Den
Berghe aanduidt als `wederkerigheid' en `dwang'. Met wederkerigheid
bedoelt hij samenwerking die voor beide partijen gunstig is. De meest
complexe en ingrijpende voorbeelden hiervan zie je bij marktinteracties:
handelen, kopen, verkopen, produceren en andere economische
activiteiten. Dwang betreft het inzetten van geweld voor eenzijdig
voordeel -- oftewel intra-specifiek parasitisme of predatie. In dit boek
en twee voorgaande werken hebben we aangetoond dat dwang een cruciaal
element van de menselijke samenleving vormt, vaak belangrijker dan men
doorgaans erkent. Dwang beïnvloedt de veiligheid van eigendom en beperkt
de mogelijkheden voor wederzijds voordelige samenwerking; het vormt
immers de basis van alle politiek. Het derde aspect binnen Van Den
Berghe's typologie van menselijke socialiteit is `verwantschapsselectie'
-- het coöperatieve gedrag dat dieren tegenover hun naaste verwanten
tonen. We bespreken verwantschapsselectie verderop uitgebreid, maar ook
dit fenomeen vormt een essentieel kenmerk van `de economie van de
natuur.'

Zoals Jack Hirshleifer verwoordde: ``De heropleving van de
darwinistische evolutietheorie van selectie, toegepast op problemen van
sociaal gedrag -- wat bekendstaat als sociobiologie -- heeft een
duidelijk economisch aspect.''

Wanneer we het volledige spectrum van het leven bekijken, streeft de
sociobiologie ernaar de universele wetten te achterhalen die de
uiteenlopende vormen van associatie tussen organismen verklaren. Waarom
zie je soms zowel geslacht als families, soms enkel geslacht, en soms
helemaal geen van beide? Waarom zwermen sommige dieren terwijl anderen
liever alleen leven? Binnen groepen ontstaan soms duidelijke
hiërarchieën, terwijl dat in andere groepen ontbreekt. Waarom delen
bepaalde soorten hun territoria, terwijl andere dat niet doen? Wat
motiveert het onbaatzuchtige gedrag van sociale insecten, en waarom komt
dat fenomeen zo zelden voor in de natuur? En wanneer worden hulpbronnen
vreedzaam toegekend en wanneer wordt er geweld ingezet? Deze vragen
formuleer je in vertrouwde economische termen. Sociobiologen onderzoeken
welke netto voordelen de waargenomen associatiepatronen bieden en welke
mechanismen ervoor zorgen dat deze sociale structuren in evenwicht
blijven. Misschien verklaart juist de opvatting over de continuïteit in
economisch gedrag tussen mens en andere levensvormen -- door een
criticus `genetisch kapitalisme' genoemd -- de vijandige houding van
sommige ideologen tegenover de sociobiologie\ldots{}

Wij integreren sociobiologie in onze analyse van het nationalisme omdat
dit perspectief inzicht geeft in die aspecten van de menselijke natuur
die systematische dwang rechtvaardigen. Wij delen de overtuiging van
natuurwetenschapper Cohn Tudge, auteur van \emph{The Time Before
History}, dat we -- voordat we de hedendaagse wereld doorgronden, laat
staan een toekomstvisie ontwikkelen -- allereerst de inleiding op de
geschiedenis moeten begrijpen. Dat houdt in dat we onszelf moeten
plaatsen in een veel bredere tijdsschaal. Tudge wijst ons erop dat
achter de alledaagse beroeringen van ons leven talrijke diepgewortelde
en krachtige krachten schuilgaan die uiteindelijk ons en al onze
medebewoners beïnvloeden\ldots{} Wij vermoeden dat in deze diepere,
krachtige krachten een genetisch bepaalde motivatie schuilgaat die het
nationalisme fundeert. Zoals Hirshleifer opmerkt, in navolging van Adam
Smith en R. H. Coase, `zijn menselijke verlangens uiteindelijk adaptieve
reacties die gevormd zijn door de biologische aard van de mens en zijn
positie op aarde.' Dit komt duidelijk naar voren in de overwegend
biologische verwijzingen binnen de meeste discussies over nationalisme.
Zelfs in de Verenigde Staten, een opvallend multietnische natie,
personifieert men de overheid in familiale termen als `Oom Sam.'

De biologische erfenis

Kortom, als we de voortdurende evolutie van de menselijke samenleving
willen begrijpen, moeten we rekening houden met de menselijke natuur, de
oorsprong van soorten en hun ontwikkeling door natuurlijke selectie. In
dit geval onderzoeken we de waarschijnlijke reactie van de mens op de
nieuwe omstandigheden die de informatietechnologie met zich meebrengt.
We richten ons vooral op de reactie op de opkomst van de cybereconomie
en de vele gevolgen daarvan, waaronder een economische ongelijkheid die
uitgesprokener is dan ooit tevoren. Sleutels tot een deel van die
verwachte reactie vind je in onze genetische erfenis.

Wanneer een nieuwe soort verschijnt, verwerpt zij niet al het DNA dat
zij in haar vorige vorm bezat, maar bouwt zij er juist op voort. Het
genetische verschil tussen mens en chimpansee betreft minder dan twee
procent van hun DNA -- we delen dus ruim 98 procent -- en een deel
daarvan herleiden we tot oerprimitieve organismen, diep verankerd in de
evolutionaire geschiedenis.

\section{Genetische inertie}\label{genetische-inertie}

Menselijke culturen bevatten ook universele elementen, waarvan sommige
daadwerkelijk van premenselijke voorouders zijn overgeërfd. Hoe we
voedsel zoeken, ons voortplanten, gezinnen stichten, met onbekenden
omgaan en ons verdedigen, vormt samen een complexe mix van instinct en
cultuur met zeer primitieve wortels. Tegelijkertijd kunnen deze
gedragingen zich op een moderne manier aanpassen, zoals kenmerkend is
voor de hedendaagse natiestaat. Als we culturen op deze wijze bekijken,
trekken we parallellen met de genetische evolutie. Drie belangrijke
verschillen springen hierbij eruit. Ten eerste worden culturen
overgedragen via de uitwisseling van informatie tussen mensen in plaats
van via de genetische keten van generatie op generatie. Ten tweede
kunnen culturen -- wellicht in mindere mate dan vaak gedacht --
aangepast worden door doelbewust en intelligent handelen. Ten derde
passen ze zich aan aan een snel veranderende omgeving van kosten en
beloningen, die veel sneller muteert dan genetische eigenschappen.
Fysiek lijken we nog sterk op onze voorouders van 30.000 jaar geleden,
maar cultureel hebben we een enorme sprong voorwaarts gemaakt.

\subsection{Evolutionaire modellen}\label{evolutionaire-modellen}

Er bestaan twee biologische modellen die de evolutie van soorten
proberen te verklaren. De heersende wetenschappelijke consensus volgt de
neo-darwinistische benadering: willekeurige genetische mutaties leiden
tot diverse fysieke varianten. De meeste van deze vormen bieden geen
overlevingsvoordeel -- zoals bijvoorbeeld te zien is bij de albino merel
-- en sterven daarom doorgaans uit. Slechts enkele varianten bevorderen
overleving en verspreiden zich binnen de soort. Hoewel deze theorie nog
gepaard gaat met veel moeilijkheden -- problemen die wetenschappers in
de komende eeuw wellicht oplossen -- vormt het idee dat willekeur en het
overleven van voordelige aanpassingen centraal staan wel de basis van de
huidige wetenschappelijke orthodoxie.

Een alternatief model is een variant op de theorie van de vroege
twintigste-eeuwse Franse filosoof Henri Bergson, die stelde dat de
natuur een niet-willekeurig, creatief doel nastreeft -- een intelligente
kracht die voortdurend op zoek is naar oplossingen. Dit idee vindt
weerklank in het werk van hedendaagse denkers zoals \emph{David Layzer}
en \emph{Stephen Jay Gould}, die benadrukken dat genetische variatie
niet louter willekeurig verloopt, maar duidelijke tendensen
vertoont.\footnote{Zie Stephen Jay Gould, `Evolutionaire biologie van
  beperkingen,' \emph{Daedalus}, lente 1980, en David Layzer, `Altruïsme
  en natuurlijke selectie,' \emph{Journal of Social and Biological
  Structures} (1978), geciteerd door Howard Margolis, \emph{Egoïsme,
  altruïsme en rationaliteit} (Chicago: \emph{University of Chicago
  Press}, 1984).} Dit betreft geen kreationisme in de strikte bijbelse
zin, maar het omzeilt wel veel van de problemen die het orthodoxe
darwinisme met zich meebrengt.

„De belangrijkste theoretische bijdrage van de sociobiologie ligt in de
uitbreiding van het `fitness'-begrip naar dat van `inclusieve fitness.'
Een dier kan immers zijn genen rechtstreeks doorgeven via zijn eigen
nageslacht of indirect via de voortplanting van verwanten met wie het
bepaalde proporties genen deelt. Daarom mag men verwachten dat dieren,
naarmate zij genetisch nauwer verwant zijn, zich coöperatiever opstellen
en zo elkaars overlevingskansen vergroten. Dit principe noemt men
verwantschapsselectie. Kortom, dieren vertonen nepotisme: zij geven de
voorkeur aan familieleden boven aan niet-verwanten en aan naaste
familieleden boven aan verre verwanten. Bij mensen kan dit bewust
gebeuren, maar vaker verloopt het onbewust.``\footnote{Van den Berghe,
  op. cit., p.~96.}

--- Pierre van den Berghe

\section{Genetisch beïnvloede motivationele
factoren}\label{genetisch-beuxefnvloede-motivationele-factoren}

De introductie van het concept `inclusieve fitness' door W. D. Hamilton
in 1963 in \emph{The Evolution of Altruistic Behavior} gaf een flinke
impuls aan het biologische perspectief op menselijk gedrag. Hamilton
merkte op dat mensen van nature vaak op eigenbelang gericht zijn, maar
dat zij ook wel eens altruïstische of zelfopofferende handelingen
verrichten die op het eerste gezicht weinig voordeel bieden. Hij
probeerde deze schijnbare tegenstrijdigheden te verklaren door te
stellen dat de fundamentele eenheid van maximalisatie niet het individu
zelf is, maar het gen. Volgens hem streeft elk organisme ernaar niet
alleen het eigen welzijn te bevorderen, maar ook wat hij `inclusieve
fitness' noemt. Daarmee bedoelde hij niet enkel het veiligstellen van de
eigen overleving in darwinistische zin, maar ook het vergroten van de
voortplantingskansen en overleving van naaste familieleden die dezelfde
genen delen.\footnote{Zie W. D. Hamilton, `The Evolution of Altruistic
  Behavior', \emph{American Naturalist}, 1963, pp.~346--54.} Hamiltons
these over `inclusieve fitness' werpt licht op vele anderszins
raadselachtige kenmerken van menselijke samenlevingen, waaronder
politieke fenomenen binnen natiestaten.

\subsection{Altruïsme: misbenaming of fossiele
verwantschapsselectie?}\label{altruuxefsme-misbenaming-of-fossiele-verwantschapsselectie}

Volgens Van Den Berghe richt `altruïsme' zich in de praktijk
hoofdzakelijk op familie -- en dan vooral op naaste familie -- waardoor
de term eigenlijk misleidend blijkt te zijn. Het vertegenwoordigt in
feite het ultieme genetische egoïsme en vormt een blinde poging om
inclusieve fitness te maximaliseren.\footnote{Van den Berghe, op. cit.,
  p.~96.} Dit betekent echter niet dat alle vormen van hulpvaardigheid
uitsluitend afhangen van nauwe genetische verwantschap, zoals Hamilton
en Van Den Berghe veronderstellen. De onzekerheden die voortkomen uit
het feit dat mensen zich seksueel voortplanten in plaats van asexueel te
klonen, zorgen er bijna voor dat een drang tot het vergroten van de
inclusieve fitness ook een behoorlijke mate van `altruïstisch' gedrag
stimuleert -- hetgeen soms zelfs allelen bevordert die niet met het
`egoïstische gen' geassocieerd worden. Zo kan het gebeuren dat mensen
hulp verlenen in de veronderstelling dat zij naaste familie
ondersteunen, terwijl dit in werkelijkheid niet altijd het geval is. De
vader die een opofferende daad verricht voor zijn nakomelingen is
mogelijk niet zijn biologische ouder, maar hij gedraagt zich er wel
naar.\footnote{Dezelfde redenering geldt uiteraard voor de zoon of
  dochter die zich opoffert voor degenen die hij als zijn broers en
  zussen beschouwt, ook al is dat in werkelijkheid niet zo.} Dit is niet
louter een onderwerp voor soapseries, maar raakt een fundamenteel
vraagstuk: het voortbestaan van `egoïstische genen' wordt waarschijnlijk
juist bevorderd als iedere vermeende ouder zich gedraagt alsof hij
daadwerkelijk de biologische ouder is, ook al bestaat de kans dat dit
niet zo is.

Wanneer we dit in de juiste context plaatsen, valt -- zoals Hirshleifer
opmerkt -- op dat veel van de ogenschijnlijke paradoxen rond `altruïsme'
eigenlijk semantische verwarringen zijn. Mensen raken daardoor de
competitieve context uit het oog waarin hulpverlening juist een
overlevingsvoordeel kan opleveren. Hirshleifer stelt: `Als een keuze
voor een altruïstische strategie levensvatbaar moet zijn in de
concurrentie met niet-altruïsme, dan moet altruïsme meer bijdragen aan
het eigen overleven dan niet-altruïsme, en kan het dus eigenlijk niet
echt altruïsme zijn.' We kunnen al deze verwarring voorkomen door de
term `altruïsme' achter ons te laten en in plaats daarvan de vraag te
stellen: wat bepaalt eigenlijk het objectieve verschijnsel dat wij
hulpverlening noemen?\footnote{Hirshleifer, op. cit., p.~179.}

Deze vraag is met name intrigerend bij hulpverlening binnen
familiebanden. Hamiltons fundamentele formulering van inclusieve fitness
bevatte een biologische kosten-batenanalyse, waarin een individu --
ofwel `het gen dat het hulpverleningsgedrag beheerst' -- de overleving
van een identieke kopie van zichzelf net zo waardeert als zijn eigen
overleving. Daarom hangt de bereidheid om hulp te verlenen -- of zelfs
offers te brengen -- af van de kans dat een ander individu dat identieke
gen draagt. Concreet dicteert een gen voor hulpverlening binnen
familiebanden dat een man (onder gelijke omstandigheden) zijn leven mag
opofferen als hij daarmee twee broers of zussen, vier halfbroers of
halfzussen, acht neven of nichten, enzovoort kan redden.\footnote{Ibid.}

\section{Kansproblemen bij inclusieve
fitness}\label{kansproblemen-bij-inclusieve-fitness}

Hoewel dit biologische concept in eerste instantie helder lijkt, onthult
een nadere beschouwing enkele complicaties. Zo impliceert het
bijvoorbeeld niet per se dat het feit dat broers, zussen en kinderen een
kans van 50 procent hebben om een identiek gen te erven, inhoudt dat dit
gen ook daadwerkelijk tot expressie komt. Elk individu draagt twee
kopieën van elk gen -- één van de vader en één van de moeder. Dit
betekent dat nakomelingen altijd slechts de helft van de genen van een
ouder erven. Bovendien bestaat er altijd het risico op mutatie tijdens
de voortplanting, wat -- hoe onwaarschijnlijk ook -- de zekerheid van de
genetische kosten-batenanalyse ondermijnt. Als we de metafoor `het gen
als optimalisator' serieus nemen, vormt het geval van een vader die niet
de biologische ouder blijkt te zijn slechts het duidelijkste voorbeeld
van een breder probleem. Als het immers daadwerkelijk gaat om de
optimalisatie van de overleving van het `egoïstische gen' door offers te
brengen voor naaste familie, geldt iedere vervanging van een identieke
kopie van dit gen door een ander allel als één van die complexe
mechanismen die Moeder Natuur hanteert.

Onzekere gevolgen

Altruïsme dat zich op verwanten richt, brengt dan ook zijn eigen
problemen met zich mee. Niet alleen speelt het
waarschijnlijkheidsprobleem rond het `egoïstische gen' een rol --
schijnverwanten van de gastheer dragen mogelijk niet exact dezelfde
kopieën -- maar bovendien is het lastig om, onder omstandigheden van
onzekerheid, vast te stellen of een bepaalde opofferingshandeling
primair voordelig is voor verwanten in plaats van voor anderen. (Een
opoffering die vooral anderen bevoordeelt, kan immers de inclusieve
fitness van het `egoïstische gen' schaden doordat de kans dat het in
volgende populaties voorkomt, afneemt.)

Overweeg eens een afschuwelijk voorbeeld, geïnspireerd door het nieuws
dat we tijdens het schrijven meemaakten. Stel je voor dat een ouder in
Dunblane, Schotland, op het laatste moment hoort dat een gewapende
maniak op weg is naar een lokale school, met als doel er schade aan toe
te richten. Door direct in te grijpen, kon die ouder de heroïsche --
maar wellicht gedoemde -- actie ondernemen om de maniac te confronteren
en zo mogelijk de schoolkinderen te redden.

Of misschien niet.

Zelfs een meedogenloze maniak, die vastbesloten is elk kind op aarde te
vermoorden, kan maar beperkte schade aanrichten voordat hij zonder
munitie komt of door anderen wordt overmeesterd. Als de ouder er echter
voor kiest niet in te grijpen, overleven zijn of haar kinderen
hoogstwaarschijnlijk toch, net als de meeste schoolkinderen. De schade
die door een moedige daad had kunnen worden voorkomen, zou anders
waarschijnlijk de kinderen van anderen treffen. Door zijn of haar leven
te riskeren om in essentie andermans kinderen te redden, verlaagt de
ouder feitelijk zijn `inclusieve fitness'. Daarmee berooft hij zijn
eigen nakomelingen van de volledige genetische bijdrage van beide
ouders, waardoor zij in de darwinistische strijd in een nadeligere
positie komen te staan.

Hoewel dit voorbeeld wat geforceerd lijkt, blijft het realistisch. Het
illustreert dat er talloze situaties bestaan waarin zowel grootschalige
als bescheiden hulpbetonen positieve effecten hebben. Vaak beperken de
directe begunstigden van zulke daden zich niet tot de naaste familie.
En, ironisch genoeg, zoals we hierna zullen zien, kan dit zelfs
bijdragen aan het overlevingsvoordeel dat mensen met minder kieskeurige
hulpgenen in staat stelde alle eeuwen van ontberingen tot nu toe door te
komen.

Altruïsme en genetische inertie

Als we aannemen dat de `selfish gene'-these een nauwkeurige verklaring
biedt voor wat menselijk gedrag motiveert, is het te simpel te
veronderstellen dat het daaruit voortvloeiende hulpzame of offerende
gedrag zich uitsluitend richt op echte verwanten. Beperkte kennis maakt
het in sommige situaties lastig om nauwkeurig vast te stellen wie tot de
familie behoort. Zelfs wanneer de verwantschap duidelijk is, kan de
aanwezigheid van een specifieke `selfish gene' binnen de groep slechts
op basis van waarschijnlijkheden worden vastgesteld. Tot voor kort kon
men echte genetische markers bij individuen niet onderscheiden, en we
zijn nog ver verwijderd van het praktisch kunnen vaststellen welke
naaste verwanten daadwerkelijk de `selfish gene' tot uiting brengen die
hun overleving bevordert. Daarbij speelt ook het probleem mee dat
voordelen strikt beperkt worden tot verwanten en niet worden doorgegeven
aan anderen.

Uit ervaring blijkt bovendien dat mensen hun `zorginstinct' soms richten
op niet-verwanten wanneer geschikte familieleden ontbreken. Een
duidelijk voorbeeld hiervan is het gedrag van ouders tegenover
geadopteerde kinderen, maar ook kinderloze personen tonen vergelijkbare
betrokkenheid bij hun huisdieren. Het is niet ongewoon dat zulke mensen
zichzelf ernstig verwonden of zelfs hun eigen leven riskeren om
bijvoorbeeld een kat te redden die in een boom vastzit. Elk jaar komen
bovendien aanzienlijke aantallen mensen om bij huishoudelijke ongevallen
die op de een of andere manier samenhangen met het gevaar voor
huisdieren. Wat geldt voor huisdieren geldt in nog grotere mate voor
adoptiekinderen. Het is zeker niet overdreven te stellen dat
adoptieouders hun kinderen vaak behandelen `alsof' het eigen bloed
betreft, waardoor het begrip `verwantschapsselectie' een andere lading
krijgt.

Dergelijke gevallen weerleggen de `selfish gene'-theorie niet zozeer als
sommigen beweren. Integendeel, we zien voorbeelden van mensen die zich
gedragen alsof zij offers brengen voor nauwe familiebanden om zo hun
eigen inclusieve fitness te vergroten -- wat we `genetische inertie'
noemen. Met andere woorden, hun gedrag weerspiegelt het feit, zoals
Howard Margolis opmerkt in \emph{Selfishness, Altruism and Rationality},
dat `de menselijke samenleving sneller verandert' dan onze genetische
samenstelling. Daarom gedragen mensen zich nog steeds `in wezen alsof ze
tot een kleine groep jagers-verzamelaars behoren.'\footnote{Margolis,
  op. cit., p.~32.}\\
\textgreater{} Het waren kleine, door inbroring gekenmerkte populaties
van enkele honderden individuen. \ldots{} Leden van de stam, hoewel zij
waren onderverdeeld in kleinere verwantengroepen, zagen zichzelf als één
volk, geïsoleerd van de buitenwereld en met elkaar verbonden door een
web van verwantschap en huwelijksbanden, waardoor de stam feitelijk een
superfamilie vormde. Een hoge mate van inbroring verzekerde dat de
meeste echtgenoten tevens verwanten waren.\footnote{Van Den Berghe, op.
  cit., p.~98.}

Samenvattend waren etnische groepen gedurende het hele menselijke
bestaan vóór de komst van de landbouw `inbroedende superfamilies.' Door
de hechte band tussen familie en ingroep kan er een genetisch bepaalde
neiging ontstaan om de eigen groep als verwanten te zien. Het is goed
voor te stellen dat dergelijk gedrag in het verleden overlevingsvoordeel
bood, toen elk lid van zo'n superfamilie nauw verwant was. Zoals
Margolis suggereert, is het aannemelijk dat binnen zulke kleine, hecht
verwante groepen alomvattend eigenbelang -- los van enige verwachting
van wederkerigheid of vergelding -- op zichzelf al voldoende motivatie
bood om bij te dragen aan het groepsbelang. We kunnen dan stellen dat
een zekere drang tot groepsgericht handelen voortleeft als een fossiel
van verwantenaltruïsme.\footnote{Margolis, op. cit., p.~32.} Met andere
woorden, doordat we het genetische profiel van jagers-verzamelaars in
ons dragen, weerspiegelt ons gedrag tegenover ingroepen de vorm van
`altruïsme' die nodig is om het overlevingssucces van groepen, gevormd
door inbroedende superfamilies, optimaal te ondersteunen.

Naar verluidt, zoals Margolis speculeert, heeft de neiging tot
groepsbelanggericht gedrag -- voortkomend uit wat sommigen `fossiel
verwantenaltruïsme' of genetische traagheid noemen -- bijgedragen aan
het overleven van Homo sapiens, terwijl andere humanoïde soorten
uitstierven.\footnote{Ibid.}

\subsection{Epigenese}\label{epigenese}

Wij beschouwen dit `alsof'-gedrag als een treffend voorbeeld van
epigenese, oftewel de aangeboren neiging van genetisch beïnvloede
motivatiefactoren om de voorkeur te geven aan de ene keuze boven de
andere. Met andere woorden, onze geest is geen onbeschreven blad, maar
functioneert als een harde schijf met vooraf ingestelde circuits die
bepaalde reacties sneller laten iften en aantrekkelijker maken dan
andere. Volgens deze stelling denken wij automatisch in termen van een
out‑groep -- die we associëren met vijandigheid of onverschilligheid --
en een in‑groep, waaraan wij doorgaans sterke gevoelens van
vriendelijkheid en loyaliteit koesteren, meestal gericht op
verwanten.\footnote{Shaw en Wong, op. cit., pp.~68-74.}

Deze epigenetische aanleg om binnen een groep te handelen alsof het om
naaste familieleden gaat, schept een kwetsbaarheid voor manipulatie.
Nationalisten hebben hier vaak op ingespeeld om de bereidheid tot
opoffering voor de staat aan te wakkeren. In dat opzicht valt het dus
niet te verwonderen dat nationalistische propaganda overal doordrenkt is
met de taal van verwantschap.

\begin{quote}
``Door de stem van haar schrikwekkende kanon roept het mooie Frankrijk
haar kinderen op. Soldaten om ons heen bewapenen zich. Vooruit, vooruit,
want het is onze moeder die roept.''\footnote{Geciteerd door Shaw en
  Wong, op. cit., p.~91.} - lied van Franse soldaten
\end{quote}

\subsection{vals verwantschap}\label{vals-verwantschap}

Overweeg de sterke neiging van politici om de staat te beschrijven met
termen die zij uit de verwantschap ontlenen. Zo noemt men de natie `ons
vaderland' of `ons moederland' en wordt over haar burgers gesproken als
over `wij', `familieleden' of onze `broers en zussen'.\footnote{Zie
  Billig, op. cit., p.~71.} Dat staten als Frankrijk, China en Egypte --
ondanks hun culturele verschillen -- zulke vergelijkingen hanteren, is
volgens ons geen toevalligheid in de retoriek, maar een duidelijk
voorbeeld van `epigenese', oftewel de invloed van genetisch bepaalde
motivatiefactoren die de mens van nature bepaalde voorkeuren laten
ontwikkelen.

Hoe werkt deze epigenese? Het identificatiemechanisme dat we inzetten om
de emotionele verbondenheid met de natiestaat te stimuleren, maakt
gebruik van middelen die in de oeroude tijden als herkenningspunten voor
verwantschap dienden -- `om de zorgen van het individu over zijn
inclusieve fitheid te koppelen' aan de belangen van de staat.\footnote{Shaw
  en Wong, op. cit., p.~106.} Shaw en Wong benoemen bijvoorbeeld vijf
middelen die moderne natiestaten aanwenden om hun bevolking tegen
buitenstaanders te mobiliseren:

\begin{enumerate}
\def\labelenumi{\arabic{enumi}.}
\tightlist
\item
  een gemeenschappelijke taal,
\item
  een gedeeld thuisland,
\item
  overeenkomende fenotypische kenmerken,
\item
  een gemeenschappelijke religieuze erfenis en
\item
  het geloof in een gezamenlijke afstamming.\footnote{Ibid.}
\end{enumerate}

Deze kenmerken hebben in primitieve tijden de etnische kern
waarschijnlijk onderscheiden. Veel van de aantrekkingskracht van het
nationalisme ontleent men aan de wijze waarop men deze
identificatiemiddelen heeft overgenomen en verpakt in de taal van
verwantschap -- zoals geïllustreerd door het eerder geciteerde Franse
soldatengezang. Zulke mobilisatie-instrumenten, die de staat aanduiden
als `het vaderland' of `het moederland', zijn wereldwijd gangbaar omdat
ze effectief blijken te zijn.

\subsection{Genetische boekhouding}\label{genetische-boekhouding}

Het imaginaire karakter van de vermeende verwantschapsbanden tussen
burgers en de staat blijkt uit het gebrek aan variatie dat je bij echte
familiebanden wel aantreft. Zelfs binnen grote families, waar iedereen
met elkaar vernocht is, verschilt de mate van verwantschap: ouders,
broers en zussen vormen de naaste kring, terwijl grootouders, neven en
nichten minder nauw verbonden zijn en zelfs verre, kussende neven en
nichten nauwelijks meer kans hebben dan vreemden om een bepaald gen met
elkaar te delen. Tegenwoordig delen echtgenoten doorgaans geen nauwe
verwantschap, in tegenstelling tot wat in de steentijd het geval was.
Men definieert echte verwantschap wiskundig via de
`verwantschapscoëfficiënt', een maat voor genetische overlap die
Hamilton berekende.\footnote{Zie Hamilton, op. cit., en W. D. Hamilton,
  \emph{The Genetical Evolution of Social Behavior, land!!}, Theoretical
  Biology vol.~7, pp.~1--16, 17--52.}

Daarentegen presenteert men de nationale `familie' als volledig en
flexibel samenvallend met de territoriale grenzen van de staat.
Nationaliteit verspreidt zich als een vloeistof gelijkmatig over iedere
spleet binnen die strikt afgebakende perken. Benedict Anderson stelt:
`In de moderne opvatting is staatssouvereiniteit volledig, vlak en
gelijkmatig van kracht over elke vierkante centimeter van een wettelijk
afgebakend grondgebied.'\footnote{Anderson, op. cit.} Daarbij komt dat
de coëfficiënt van de denkbeeldige verwantschap altijd één bedraagt
wanneer het opoffering voor de staat betreft.

Het samenvoegen van inclusieve fitheid met de natiestaat biedt
interessante inzichten in de neiging van mensen om de veranderingen van
het nieuwe millennium te omarmen of zich ertegen te verzetten. Zoals we
eerder toonden, berustten alle samenlevingsvormen vóór het
informatietijdperk op territoriale grenzen. Men organiseerde zich ofwel
rondom het thuisgebied van de etnische kern, of -- zoals bij de
natiestaat -- op basis van groepssolidariteit om gezamenlijk kracht te
mobiliseren voor de verdediging van een lokaal grondgebied tegen
buitenstaanders. Buiten het eigen directe grondgebied beschouwde men
vreemdelingen dan als vijanden, wat vanuit de aannames van kinselectie
in de oertijd volstrekt logisch was. Toen de mens in zijn huidige
genetische vorm opkwam, waren stamleden nauw verwant en behoorden zij
tot een etnische kern, een `inteeltsuperfamilie.'

Bovendien bood de logica van verwantschapsselectie het individu een
concrete economische reden om de voorspoed en overleving van directe
verwanten gelijk te stellen aan die van de hele stam of superfamilie.
Een lid van een jager-verzamelaarsstam was immers sterk afhankelijk van
het succes van de gehele stam voor zijn eigen welzijn. Er bestond geen
sprake van onafhankelijk bezit, noch had een individu of familie
daadwerkelijk de mogelijkheid om los van de stam te overleven en te
floreren. Hierdoor raakte het eigenbelang van het individu nauw verweven
met dat van de groep. Zoals Hirshleifer verwoordde: `Voor zover de leden
van een groep een gemeenschappelijk lot delen, wordt het elkaar helpen
gezien als een vorm van eigenhulp.'\footnote{Hirshleifer, op. cit.,
  p.~188.}

``Blijkbaar gaat de primitieve mens ervan uit -- en de Lovedu, die als
afspiegeling van honderden vergelijkbare volkeren kunnen worden
beschouwd -- dat een samenleving er altijd zo uitziet dat ieders positie
exact gelijk is.''

\subsection{Nieuwe omstandigheden, oude
genen}\label{nieuwe-omstandigheden-oude-genen}

Tegenwoordig stelt microtechnologie ons in staat om omstandigheden te
creëren die sterk afwijken van de omgevingen waartegen we genetisch
gezien gewend raakten in het Stenen Tijdperk. Tegelijkertijd zorgt
informatietechnologie voor economische ongelijkheid op een schaal die
onze voorouders in het ongerepte, egalitaire Stenen Tijdperk nooit
hebben ervaren. Bovendien schept informatietechnologie supraterritoriale
activa die het gewicht van de traditionele ingroep -- de natiestaat --
doen afnemen. Ironisch genoeg kunnen deze nieuwe cyberactiva juist meer
waard blijken te zijn, juist doordat ze ver van huis verblijven. Dit
effect neemt des te meer toe als er, als reactie op de groeiende
economische ongelijkheid in de welvarende industriële landen, een
afgunstige tegenreactie ontstaat. Activa die ver weg worden gehouden
stijgen vervolgens in waarde, omdat ze niet alleen minder vatbaar
blijken voor jaloezie, maar ook buiten het bereik blijven van diegenen
die je het meest kwaad willen doen -- namelijk je eigen natiestaat.

\subsection{Diseconomieën van de natuur en
nationalisme}\label{diseconomieuxebn-van-de-natuur-en-nationalisme}

Misschien toont het belang van epigenese bij het vormen van houdingen
wel dat men tot nu toe nauwelijks stil heeft gestaan bij de ironie van
ingroepidentificatie in relatie tot de moderne natiestaat. In de moderne
periode heeft de logica van geweld de oorspronkelijke drijfveer, die de
neiging veroorzaakte om de inclusieve fitheid met de eigen groep te
identificeren, doen verwateren. Waarom? Omdat door de identificatie van
de inclusieve fitheid van het individu met een nationale groep, in
plaats van in te zetten op het overleven en de voorspoed van naaste
familie, de waarde van iedere daad van opoffering voor die naasten
vrijwel onbeduidend is geworden. De typische moderne natiestaat is
simpelweg te groot om een statistisch significante
`verwantschapscoëfficiënt' tussen een individu en de overige burgers --
die aanspraak op hem maken -- te realiseren. Niet alleen is het aandeel
van naaste verwanten binnen de groep sterk afgenomen -- van bijna
volledige samenhang in het Stenen Tijdperk tot een vrijwel
verwaarloosbaar biologisch spoor in de twintigste eeuw --, ook is de
`verwantschapscoëfficiënt' tussen een individuele burger en de rest van
de mensheid doorgaans niet hoger dan wanneer men naar de gehele
wereldbevolking kijkt. Een ingroep met tientallen of zelfs honderden
miljoenen leden (of, bij de Chinezen, meer dan een miljard) wordt
zodanig gigantisch dat het effect van elke opoffering of elk voordeel
als druppel in de oceaan wordt verwaterd. Logisch gezien kan een moderne
nationalist zich dan -- in tegenstelling tot de jager-verzamelaar uit
het Stenen Tijdperk -- niet redelijkerwijs beroepen op het idee dat een
daad van opoffering of hulp aan zijn groep de overlevingskansen van zijn
familie op een betekenisvolle wijze zou vergroten.

Hoewel nationale economieën in de moderne tijd als de fundamentele
maatstaven voor het meten van welzijn fungeerden, vormde de last die
namens de natie -- de eigen ingroep -- werd opgelegd de grootste
belemmering voor het succes van het talentvolle individu -- en daarmee
ook voor dat van zijn naasten. Dit gold althans voor hen die zich
primair bezighielden met een op wederkerigheid gebaseerde, in plaats van
met een dwingend opgelegde, socialiteit -- als we de
indelingscategorieën van menselijk gedrag volgens Van Den Berghe
herzien.\footnote{Van Den Berghe, op. cit., p.~97.}

De logica van de natiestaat impliceert dat de hoogste prijs van
burgerschap opoffering -- zelfs de dood -- is. Zoals Jane Bethke
Elshtain opmerkte, indoctrineren natiestaten hun burgers meer tot
opoffering dan tot agressie: `De jongeman gaat de oorlog in niet zozeer
om te doden als wel om te sterven, om zijn specifieke lichaam op te
offeren voor dat van het grote lichaam, het politieke
lichaam.'\footnote{J. B. Elshtain, `Sovereignty, Identity, Sacrifice',
  in M. Ringrove en A. J. Lerner (red.), \emph{Reimaging the Nation}
  (Buckingham, Engeland: \emph{Open University Press}, 1993), zoals
  opgemerkt door Billig, op. cit.}

Ook van belastingbetalers wordt verwacht dat zij opofferingen brengen.
Het betalen van belastingen -- net als het dragen van wapens -- geldt
als een plicht en is geen ruiltransactie waarbij men geld inruilt voor
een product of een dienst van gelijke of hogere waarde. Dat wordt in de
volksmond erkend. Men spreekt van een `belastinglast', terwijl men nooit
spreekt van een `voedsellast' bij boodschappen, van een `autolast' bij
de aanschaf van een auto of van een `vakantielast' voor reizen;
commerciële aankopen vormen immers doorgaans een eerlijke ruil, anders
zouden kopers daar niet mee instemmen.

Op deze wijze laat het nationalisme zien hoe epigenese de logica van de
darwinistische `natuureconomie' kan omkeren. De natiestaat maakte
systematische, territoriaal gebaseerde roofzucht mogelijk. In
tegenstelling tot de uitdagingen waarmee jagers-verzamelaars in het
Stenen Tijdperk werden geconfronteerd, bleek aan het eind van de
twintigste eeuw de voornaamste parasiet én het voornaamste roofdier van
het individu niet de `buitenstaander', de buitenlandse vijand, maar
juist de vermeende belichaming van de eigen ingroep -- de lokale
natiestaat. Het grootste voordeel van de opkomst van activa die de
grenzen van territorium overstijgen in het informatietijdperk lag er
juist in dat deze buiten het bereik vielen van de systematische dwang
die de lokale natiestaat -- binnen wiens grondgebied het zogenaamd
soevereine individu leefde -- uitoefende.

Als onze visie klopt, maakt microtechnologie het technisch mogelijk voor
individuen grotendeels te ontsnappen aan de lasten van een ondergeschikt
burgerschap. Zij zullen als extranationale soevereinen over hun eigen
bestaan opereren -- geen onderdanen meer in de nieuwe `Virtual City' --
en hun loyaliteit vastleggen via contracten of particuliere
overeenkomsten, op een wijze die doet denken aan het premoderne Europa,
waar kooplieden commerciële verdragen en charters veiligstelden om zich
te beschermen tegen willekeurige inbeslagnames van eigendommen en om
vrijstelling van feodaal recht te verkrijgen.\footnote{Zie Abu-Lughod,
  op. cit., p.~90.}

Binnen de cybercultuur krijgen succesvolle personen vrijstelling van de
burgerschapsverplichtingen die voortkomen uit een toevallige geboorte.
Zij zullen zichzelf niet langer primair als Brits of Amerikaan zien,
maar als extranationale bewoners van de hele wereld, die toevallig in
één of meerdere lokale gemeenschappen wonen.

\section{De cybereconomie en ons genetisch
erfgoed}\label{de-cybereconomie-en-ons-genetisch-erfgoed}

Het addertje onder het gras is overigens dat zowel dit technologische
wonder als het economische fenomeen -- namelijk het ontsnappen aan de
tirannie van plaatsgebondenheid -- afhankelijk is van de bereidheid van
individuen om een aanzienlijk deel van hun rijkdom en toekomst in handen
van vreemden te leggen. Volgens een strikte genetische boekhouding
zouden deze vreemden immers niet per se genetisch minder nauw met ons
verwant zijn dan de meeste `mede-burgers', op wie we de afgelopen eeuwen
hebben vertrouwd.

De vraag is of de verontrustende gevolgen van de onderlinge
verbondenheid binnen een natiestaat negatieve of juist positieve
signalen geven voor de cybereconomie. Zullen de `achterblijvers', die
dreigen de voordelen van dwangmatige herverdeling te verliezen, de
ondergang van de natiestaat ervaren als een aanval op hun verwanten? De
eerste 25 jaar van dit nieuwe millennium zullen laten zien hoe het
afloopt. De emotionele reacties kunnen complex worden. Het feit dat 115
miljoen mensen in de twintigste eeuw hun leven gaven in de strijd voor
natiestaten, toont schrijnend de kracht van onze genen aan.\footnote{Charles
  Tilly, `Collective Violence in European Perspective,' in T. R. Gurr
  (red.), \emph{Violence in America}, vol.~2, Protest, Rebellion, Reform
  (Newbury Park, Calif.: Sage Publications, 1989), p.~93.} Het maakt
duidelijk dat velen het voortbestaan van hun natie als een kwestie van
leven en dood zagen. De vraag blijft of deze houding doorwerkt in een
nieuw tijdperk met andere megapolitieke prioriteiten.

Dat offers, gemotiveerd door genetische verbondenheid, voor de
natiestaat vaak ingingen tegen het evolutionaire principe van
verwantschapsselectie, bewijst bovendien dat mensen zich flexibel genoeg
aanpassen aan omstandigheden waarvoor we in de steentijd niet genetisch
geprogrammeerd waren. Zoals Tudge opmerkt bij het beschrijven van de
`extreme generaliteit' van de mens: `Wij zijn het dierlijke equivalent
van de Turingmachine: het universele apparaat dat op iedere taak kan
worden ingezet.'\footnote{Tudge, op. cit., p.~168.} Welke trend tijdens
de komende overgangscrisis de overhand zal krijgen? Waarschijnlijk
beide.

De commercialisering van soevereiniteit hangt ervan af of
honderdduizenden soevereine individuen en miljoenen anderen bereid zijn
hun activa in te zetten in de `First Bank of Nowhere' om zo immuniteit
tegen directe dwang te verkrijgen. Dergelijk vertrouwen kent in het
oeroude verleden geen duidelijk equivalent. In de steentijd beschikte
men over nauwelijks activa; de weinigen die er waren, werden opgepot
door een stam -- een hechte, maar vaak ook paranoïde groep die
buitenstaanders wantrouwde. Desondanks biedt de cybereconomie, ondanks
haar evolutionaire nieuwheid, de mens de kans om ons meest baanbrekende
genetische erfgoed tot uiting te brengen: de intelligentie die onze
buitengewone hersenen kenmerkt. Degenen die tot de informatie-elite
behoren, herkennen ongetwijfeld een goede zaak zodra ze die tegenkomen.

Daarnaast kan de creatie van activa die vrijwel immuun zijn voor
predatie een opwaartse impuls geven aan de `inclusieve fitness' van
soevereine individuen. Hoewel de economische logica achter deelname aan
de cybereconomie de fundamenten van de natiestaat ondermijnt, spreekt
deze redenering vooral tot mensen met een hoog vaardigheidsniveau.

Om ten volle te profiteren van de kansen om tussen verschillende
jurisdicties te kiezen, moeten mensen bereid zijn hun natiestaat achter
zich te laten en hun persoonlijke veiligheid toe te vertrouwen aan
beveiligingsdiensten die hoofdzakelijk door marktprikkels worden
gedreven, ook al bevinden deze zich ver van hun geboorte- en
opvoedplaats. Dit biedt een duidelijk voordeel voor iedereen die
meerdere talen beheerst en een kosmopolitische cultuur omarmt in plaats
van jingoïstische neigingen te tonen. Bovendien betekent het dat
iedereen die serieus te werk gaat om het bevrijdende potentieel van de
cybereconomie voor zichzelf en zijn gezin te benutten, er verstandig aan
doet om in jurisdicties buiten de plek waar hij zijn hoofdcarrière heeft
opgebouwd een nieuwe, gunstige positie voor zichzelf te veroveren. Voor
meer details verwijs ik je naar onze bespreking van strategieën voor het
bereiken van onafhankelijkheid in de bijlagen.

\subsection{Echte affiniteiten}\label{echte-affiniteiten}

Een vernieuwd, extranationaal wereldbeeld en een andere manier om je
plek in de wereld te bepalen, kunnen onze culturele gewoonten -- zo niet
zelfs onze aangeboren neigingen -- ingrijpend veranderen. De
extranationale invulling van identiteit die in het nieuwe millennium
opkomt, maakt de aanpassing aan een veranderde wereld wellicht
eenvoudiger dan je zou denken. In tegenstelling tot nationaliteit
ontstaan deze nieuwe identiteiten niet door de systematische dwang die
in de twintigste eeuw natiestaten en het natiestaatssysteem universeel
oplegde. In het nieuwe tijdperk binden gemeenschappen en loyaliteiten
zich niet langer aan vaste territoria. Je identiteit zal steeds meer
gebaseerd zijn op oprechte affiniteiten, gedeelde belangen of werkelijke
verwantschap, in plaats van op de gefabriceerde banden van burgerschap
die de traditionele politiek onvermoeibaar nastreeft. Bescherming komt
op geheel nieuwe wijze tot stand, ver verwijderd van het meetgereedschap
waarmee een landmeter territoriale grenzen afbakent. Activa komen steeds
vaker in de cyberspace terecht in plaats van op een vaste locatie, wat
een nieuwe vorm van concurrentie stimuleert om de `beschermingskosten'
-- oftewel de belastingen -- in de meeste territoriale jurisdicties
omlaag te brengen.

\begin{quote}
`Ambitieuze mensen beseffen dat een migratoire levensstijl de prijs is
die je betaalt om vooruit te komen.'\footnote{Christopher Lasch,
  \emph{The Revolt of the Elites and the Betrayal of Democracy} (New
  York: W W Norton \& Company, 1995), p.~S.} - Christopher Lasch
\end{quote}

\section{Ontsnappen aan de
natiestaat}\label{ontsnappen-aan-de-natiestaat}

Ondanks de sterke invloed die de natiestaat als `in-group' op onze
moderne verbeelding uitoefent, zullen getalenteerde mensen -- die
vandaag nog vol vertrouwen profiteren van aansluiting bij zo'n
buitengewoon kostbare, ingebeelde gemeenschap -- dat nut spoedig in
twijfel trekken.

Inderdaad, de aanhangers van de natiestaat klagen inmiddels al over de
groeiende onthechting van de cognitieve elite.

De overleden Christopher Lasch valt in zijn tirade \emph{De opstand van
de elites en het verraad van de democratie} degenen aan `wiens
levensonderhoud niet zozeer berust op het bezit van eigendom als op de
manipulatie van informatie.'\footnote{Ibid., p.~34.} Lasch betreurt het
extranationale karakter van de opkomende informatie-economie. Hij
schrijft:

\begin{quote}
De markten waarin de nieuwe elites actief zijn, hebben inmiddels een
internationale reikwijdte. Hun fortuin is nauw verweven met
ondernemingen die grensoverschrijdend opereren. Ze richten zich meer op
het vlot functioneren van het geheel dan op de afzonderlijke onderdelen
ervan. Hun loyaliteit -- als we die term in deze context niet als
anachronistisch beschouwen -- is internationaal in plaats van regionaal,
nationaal of lokaal. Ze hebben meer gemeen met hun tegenhangers in
Brussel of Hong Kong dan met de massa Amerikanen die nog niet deel
uitmaken van het wereldwijde communicatienetwerk.\footnote{Ibid.,
  pp.~34--35.}
\end{quote}

Hoewel Lasch allesbehalve een onpartijdige waarnemer was en duidelijk
wilde maken dat zijn beeld van de informatie-elite allesbehalve vleiend
moest zijn, baseerde hij zijn minachting voor degenen die zich
onthechten van de plaatsgebonden tirannie op observaties van
ontwikkelingen die in dit boek centraal staan.

Lees je de kritieken van Lasch, Mickey Kaus (\emph{The End of
Equality}), Michael Walzer (\emph{Spheres of Justice}) en Robert Reich
(\emph{The Work of Nations}), dan bevestigen zij -- zij het op vaak
ongelukkige wijze -- delen van onze analyse. Deze auteurs tonen weinig
begrip voor de talrijke consequenties van een verdiepte marktwerking, om
nog maar te spreken van de denationalisatie van soevereine individuen.

Lasch hekelt fel degenen met extranationale ambities -- `die het
lidmaatschap van de nieuwe aristocratie van hersenen begeren' -- omdat
zij nauwe banden onderhouden met de internationale markt voor snel
bewegend geld, glamour, mode en populaire cultuur. Hij vervolgt:

Of zij zichzelf eigenlijk als Amerikanen beschouwen, valt te
betwijfelen. Patriottisme scoort duidelijk laag op hun ranglijst van
deugden. Daarentegen past `multiculturalisme' perfect bij hen en roept
het het charmante beeld op van een wereldbazaar waar men volop kan
genieten van exotische keukens, opvallende kleding, bijzondere muziek en
authentieke tradities -- zonder dat er vragen gesteld worden of
verplichtingen ontstaan. De nieuwe elites voelen zich alleen op hun
gemak wanneer ze onderweg zijn naar een conferentie op hoog niveau, de
grootse opening van een nieuwe franchise, een internationaal
filmfestival of een nog onontdekt vakantieoord. Hun blik op de wereld is
in wezen toeristisch, een houding die weinig bijdraagt aan een oprechte
betrokkenheid bij de democratie.\footnote{Ibid., p.~6.}

Economisch nationalisme

In de kritiek op de `transienten' die de virtuele gemeenschappen van het
informatietijdperk kenmerken, schuilt de erkenning dat voor velen in de
elite de baten van vergankelijkheid de kosten ruimschoots overtreffen.
Critici als Lasch en Walzer betwisten niet dat een nuchtere
kosten-batenanalyse het burgerschap voor hooggekwalificeerden overbodig
maakt, noch stellen zij dat de leden van de informatie‑elite -- waarvan
zij de houdingen afkeuren -- ten onrechte inschatten wat in hun belang
is. Evenmin doen zij alsof tabellen voor samengestelde rente aantonen
dat het voortdurend investeren in een nationaal
socialezekerheidsprogramma -- om nog maar te zwijgen over de
inkomstenbelastingen -- beter rendeert dan particuliere investeringen.
Integendeel, zij doorgronden de rekenkunde en hebben de berekeningen
doorgerekend tot de voor de hand liggende conclusies. In plaats van de
subversieve logica van economische rationaliteit te erkennen, keren ze
zich ervan af en zien ze het als `verraad' dat de informatie‑elite de
plaatsgebonden tirannie overstijgt en `de onwetenden'
achterlaat.\footnote{Ibid., p.~21.}

Net als Pat Buchanan behoren de sociaaldemocraten tot de economische
nationalisten die de dominantie van markten boven de politiek verwerpen.
Zij bekritiseren `de nieuwe aristocratie van hersenen' omdat die geen
band heeft met een specifieke locatie en zich niet oprecht bekommert om
wat volgens hen in het belang is van de massa. Hoewel zij de
denationalisatie van het individu niet expliciet erkennen, verzetten zij
zich tegen de vroege signalen en uitingen daarvan -- wat Walzer aanduidt
als `het imperialisme van de markt' -- en tegen de neiging van geld om
`grensoverschrijdend' te circuleren voor de aankoop van zaken die, zoals
Lasch betoogt, `niet koopbaar zouden moeten zijn', bijvoorbeeld
vrijstelling van de militaire dienst. Merk op hoe men reactionair
terugvalt op de militaire eisen van de natiestaat, als een heilige zone
waar geld en markten geen toegang mogen krijgen.\footnote{Ibid., p.~21.}

Deze kritiek op de informatie‑elite schetst de contouren van een
populaire tegenreactie tegen de opkomst van onafhankelijke individuen in
het komende millennium. Naarmate er nieuwe, meer marktgerichte
beschermingsvormen beschikbaar komen, zal voor talloze bekwame personen
steeds duidelijker worden dat de vermeende voordelen van nationaliteit
vooral illusoir zijn. Dit leidt niet alleen tot een zorgvuldiger
afweging van de alternatieve kosten van burgerschap, maar opent ook de
weg naar nieuwe manieren om zogenaamd `politieke' en zelfs `economische'
vraagstukken te benaderen. Voor het eerst kan `een zelfstandige
ondernemer' zijn eigen beschermingskosten aanpassen door zich tussen
verschillende rechtsgebieden te bewegen, zonder te hoeven wachten op
besluiten via groepsbesluiten en -actie, zoals een oud dilemma van
Frederic C. Lane ooit verwoordde.\footnote{Lane, `De economische
  betekenis van oorlog', in Venetië en geschiedenis. De verzamelde
  papieren van Frederic C. Lane, p.~385.}

Naarmate de prijs voor bescherming steeds meer volgens het
substitutieprincipe wordt vastgesteld, komen de rekenregels van dwang
aan het licht en verergert dit de conflicten tussen de nieuwe
kosmopolitische elite van het informatietijdperk en de `informatiearmen'
-- de rest van de bevolking die grotendeels eentalig is en niet
uitblinkt in probleemoplossend vermogen of beschikt over wereldwijd
verhandelbare vaardigheden. Deze `verliezers' of `achterblijvers', zoals
Thomas L. Friedman ze omschrijft, zullen onvermijdelijk afhankelijk
blijven van het politieke leven binnen de bestaande
natiestaten.\footnote{Zie Thomas L. Friedman, `Laat de verliezers van de
  globalisering niet buiten beschouwing', \emph{International Herald
  Tribune}, 18 juli 1996, p.~8.}

\section{De meeste politieke agenda's zullen reactionair
zijn}\label{de-meeste-politieke-agendas-zullen-reactionair-zijn}

De meesten met een vurige politieke agenda -- of ze nu nationalistisch,
milieuactivistisch of socialistisch zijn -- zullen, zodra de
eenentwintigste eeuw aanbreekt, massaal opkomen voor de wankelende
natiestaat. In de loop der tijd wordt steeds duidelijker dat het
voortbestaan van de natiestaat en de nationalistische gevoelens
essentieel zijn om een domein voor politieke dwang te behouden. Zoals
Billig opmerkt, vormt nationalisme `de voorwaarde voor conventionele
(politieke) strategieën, wat de specifieke politiek ook moge
zijn.'\footnote{Billig, op. cit., p.~99.} Daarom zal het
nationalistische element in alle politieke programma's de komende jaren
als de buik van een gulzigaard opzwellen. Milieuactivisten gaan zich
bijvoorbeeld minder richten op de bescherming van `Moeder Aarde' en meer
op die van `het vaderland.' Om redenen die we later behandelen, zullen
natie en burgerschap bijzonder heilig worden voor degenen die veel
waarde hechten aan gelijkheid. Meer dan ze nu wellicht beseffen, zullen
zij het eens worden met Christopher Lasch, die op Hannah Arendt volgde
met de uitspraak: `Het is het burgerschap dat gelijkheid verleent, niet
gelijkheid die een recht op burgerschap schept.'\footnote{Lasch, op.
  cit., p.~88.}

De privatisering van soevereiniteit doet de nadruk op gelijkheid, zoals
die in het industriële tijdperk aanwezig was, afnemen doordat het de
band tussen de welvaartsscheppers enerzijds en natie en plaats
anderzijds doorsnijdt. Het burgerschap vervult daarmee niet langer de
functie van een mechanisme voor inkomensherverdeling op basis van
gelijke stemrechten binnen een afgebakend grondgebied. De gevolgen
zullen opnieuw een deuk slaan in het progressieve historische
wereldbeeld. In tegenstelling tot de verwachtingen van zogenaamd
vooruitstrevende denkers aan het begin van de twintigste eeuw bleef de
vrije markt door de decennia heen triomfantelijk overeind. De marxisten
voorspelden de ondergang van het kapitalisme -- wat echter nooit
gebeurde -- in de hoop dat dit zou leiden tot het vervagen van
natiestaten en het ontstaan van een universeel klassenbewustzijn onder
arbeiders. In werkelijkheid zal de staat in de schaduw treden, maar op
een geheel andere manier. Er gebeurt bijna het tegenovergestelde van wat
men verwachtte. De triomf van het kapitalisme leidt tot het ontstaan van
een nieuw, wereldwijd of extranationaal bewustzijn onder kapitalisten,
waarvan velen als zelfstandige, soevereine individuen gaan opereren. In
plaats van afhankelijk te zijn van de staat om arbeiders te
disciplineren, zoals de marxisten voorspelden, blijken de meest bekwame
en rijkste personen uiteindelijk netto verliezers te worden door de
acties van de natiestaat. Het is duidelijk dat zij het meeste winnen
door het nationalisme achter zich te laten, nu de markten de overhand
krijgen ten koste van dwang.

Misschien niet onmiddellijk, maar binnen een generatie zal bijna
iedereen uit de informatie-elite besluiten zijn inkomensgenererende
activiteiten te verplaatsen naar lagebelastings- of belastingvrije
jurisdicties. Naarmate het informatietijdperk de wereld transformeert,
illustreert dit onmiskenbaar de kracht van samengestelde rente. Binnen
enkele jaren -- om nog maar te zwijgen van decennia -- wordt algemeen
duidelijk dat vrijwel iedereen met talent een aanzienlijk hoger
nettovermogen kan opbouwen en een beter leven kan leiden door afstand te
nemen van hoogbelaste natiestaten. We hebben al gewezen op de torenhoge
kosten die de vooraanstaande natiestaten met zich meebrengen, maar
aangezien dit de kern vormt van een kwestie die vaak onbegrepen blijft,
is het de moeite waard om nog eens de alternatieve kosten van
nationaliteit te benadrukken.

\subsection{Alternatieve kosten}\label{alternatieve-kosten}

In plaats van te lijden onder het verlies of de beperking van
overheidsdiensten die momenteel met hoge belastingen worden
gefinancierd, zal de informatie-elite ongekende bloei ervaren. Door
simpelweg de overmatige belastingdruk die zij nu dragen achter zich te
laten, creëren zij een enorme marge om het materiële welzijn van hun
gezinnen te verbeteren.

Zoals eerder vermeld, verkleint elke \$5.000 die je jaarlijks aan
belasting betaalt, je levenslange nettowaarde met \$2,4 miljoen als je
een jaarlijks rendement van 10 procent behaalt op je investeringen.
Verdient je portefeuille echter 20 procent, dan zorgt een jaarlijkse
belasting van \$5.000 over veertig jaar voor een verlies van maar liefst
\$44 miljoen. Cumulatief betekent dit dat je met \$5.000 per jaar meer
dan een miljoen dollar per jaar misloopt. Bij deze werkwijze leidt een
jaarlijkse belasting van \$250.000 al snel tot een verlies van meer dan
\$50 miljoen per jaar -- ofwel \$2,2 miljard over een heel leven. Zelfs
incidenteel hogere inkomsten, bijvoorbeeld in de vroege fase van je
leven, zorgen voor een nog schokkender vermogensverlies door
roofzuchtige belastingheffing.

Onze auteurs hebben tot onze tevredenheid vastgesteld dat rendementen
van meer dan 20 procent mogelijk zijn. Zo behaalden onze collega's bij
\emph{Lines Overseas Management} in Bermuda, gedurende de periode waarin
wij dit boek schreven, driemaal hun investering -- een gemiddeld
jaarlijks rendement van 226 procent. Hun ervaring bevestigt wat de
berekeningen suggereren: voor veel hoogverdieners en kapitaalbezitters
betekent roofzuchtige belastingheffing een levenslange kostenpost die
neerkomt op een aanzienlijk fortuin.

Iemand met een hoge verdiencapaciteit die belasting betaalt volgens de
tarieven van Hongkong kan uiteindelijk duizend keer meer vermogen
opbouwen dan iemand met vergelijkbare bruto-prestaties die belastingen
afdraagt volgens Noord-Amerikaanse of Europese standaarden. Je kapitaal
herhaaldelijk blootstellen aan de druk van een hoogbelastingsregime is
als deelnemen aan een race waarin je bij elke stap wordt tegengehouden.
Zou je diezelfde race met de juiste bescherming en zonder belemmeringen
kunnen lopen, dan zou je uiteraard veel verder en sneller komen.

De onafhankelijke individuen van de toekomst zullen profiteren van die
`vergankelijke' neigingen waar \emph{Christopher Lasch} en andere
critici van de informatie-elite zo fel op wijzen. Zij gaan actief op
zoek naar de rechtsgebieden met de meest gunstige fiscale voorwaarden om
zich te vestigen. Hoewel dit in strijd lijkt met het nationalistische
denken, is het economisch volstrekt logisch. Zelfs een verschil in
nettowinst van 10 procent -- laat staan een tienvoudig verschil -- zal
winstmaximaliserende mensen ertoe aanzetten om hun levensstijl,
productiemethoden én woonplaats te wijzigen. De geschiedenis van de
westerse beschaving is immers een aaneenschakeling van voortdurende
veranderingen, waarin mensen en welvaart keer op keer migreerden op zoek
naar nieuwe kansen, aangewakkerd door grillige, grootschalige politieke
omstandigheden. Een duizendvoudig verschil in nettowinst zou wel eens de
krachtigste stimulans kunnen zijn die ooit rationele mensen in beweging
heeft gezet. Of anders gezegd: de meeste mensen -- in het bijzonder
degenen die \emph{Thomas L. Friedman} aanduidt als de `verliezers en
achterblijvers' -- zouden, als ze de keuze hadden, elke natiestaat
verruilen voor \$50 miljoen, om nog maar te zwijgen van de nog hogere
belastinglasten die staten opleggen, vooral voor de top 1 procent. De
opkomst van onafhankelijke individuen die actief op zoek gaan naar
gunstige rechtsgebieden behoort dan ook tot de meest zekere
voorspellingen.

\section{De commercialisering van
soevereiniteit}\label{de-commercialisering-van-soevereiniteit}

Als we het burgerschap tegen het einde van de twintigste eeuw in een
kosten-batenanalyse meewegen, bleek het een buitengewoon koopje te zijn.
Dat werd treffend geïllustreerd door een humoristisch bedoelde
parlementaire onderzoeksnota, getiteld `Is the Queen an Australian
Citizen?', opgesteld in augustus 1995 door Ian Ireland van de
\emph{Australian Parliamentary Research Service}.\footnote{Ian Ireland,
  `Is de koningin een Australische staatsburger?', \emph{Parliamentary
  Research Service}, Australië, nr. 6, 28 augustus 1995.} Ireland
bestudeert de Australische staatsburgerschapswet van 1948 en bespreekt
de vier manieren om Australisch burgerschap te verkrijgen. Deze methoden
komen overeen met die in andere toonaangevende landen, namelijk:

\begin{itemize}
\tightlist
\item
  burgerschap door geboorte
\item
  burgerschap door adoptie
\item
  burgerschap door afstamming
\item
  burgerschap door verlening
\end{itemize}

Op zichzelf valt dit niet op, maar het legt wel het verschil bloot
tussen soevereiniteit en burgerschap. Zoals Ireland stelt: `Volgens
traditionele juridische en politieke opvattingen is de vorst soeverein
en zijn de mensen onderdanen. Onderdanen zijn gebonden door trouw en
onderwerping aan de vorst.' Gezien het voor de hand liggende feit dat
koningin Elizabeth~II soeverein is, concludeert hij dat `er een argument
bestaat dat de koningin geen Australisch staatsburger is.'\footnote{Ibid.,
  p.~2.}

En dat klopt ook: dat is zij immers niet. De koningin hoeft zich geen
zorgen te maken over haar staatsburgerschap, want zij is soeverein -- de
soeverein over haar onderdanen. Net als enkele andere monarchen bezit
zij van geboorte soevereiniteit, een status die zij heeft geërfd volgens
een oud gebruik dat de moderne tijd ver vooruit is. Het concept van
monarchie is immers al eeuwenoud en gaat terug tot de vroegste
historische verslagen van het menselijk bestaan. Landen die hun
monarchie hebben behouden, danken hun grondwet vaak hun rijke
geschiedenis, wat nog steeds bijdraagt aan de vorm van hun samenleving
-- al is dat vooral op sociaal vlak, zo niet ook wat betreft de
politieke macht. Postmoderne burgers, die niet over de voordelen van de
koningin beschikken, zullen nieuwe juridische fundamenten moeten
bedenken waarop zij de feitelijke soevereiniteit kunnen baseren die de
informatietechnologie hen biedt.

Soevereine individuen moeten ook de corrosieve effecten van afgunst
onder ogen zien -- een hinderpaal die monarchen soms teisteren, maar
voor mensen die niet traditioneel vereerd worden en zelf hun
soevereiniteit claimen, nog sterker zal doorwerken. Zoals Helmut Schoeck
in zijn uitgebreide onderzoek \emph{Envy} opmerkte: `Als er slechts één
koning -- of één president van de Verenigde Staten, met andere woorden,
slechts één persoon met een bepaalde status -- is, kan hij vrijwel
ongestraft leven op een manier die, zelfs op een veel kleinere schaal,
in dezelfde samenleving tot grote verontwaardiging zou leiden indien
succesvolle leden van bredere professionele of sociale groepen zich op
die wijze zouden gedragen.'\footnote{Schoeck, op. cit., p.~265.}
Monarchen, als verpersoonlijking van de natie, genieten een zekere
immuniteit tegen afgunst -- een bescherming die niet geldt voor andere
soevereine individuen.

De `verliezers en achterblijvers' in de informatiesamenleving zullen
ongetwijfeld het succes van de winnaars benijden en met afgunst toezien,
zeker nu de verdere ontwikkeling van de markten er op wijst dat we
steeds meer een `winnaars nemen alles'-wereld tegemoetzien. Steeds vaker
hangt de beloning af van relatieve prestaties in plaats van absolute
prestaties, zoals dat in de industriële productie het geval was. Vroeger
kreeg een fabrieksarbeider loon op basis van het aantal gewerkte uren,
gemeten met een tijdklok, of volgens een vastgestelde outputnorm --
bijvoorbeeld het aantal geproduceerde stuks of gemonteerde
eenheden.\footnote{Voor een kritische kijk op vergoedingssystemen
  gebaseerd op relatieve prestaties zie Robert H.~Frank en Philip
  J.~Cook, \emph{The winner-take-all society}, pp.~24f.} Die
gestandaardiseerde beloning was mogelijk omdat de output voor iedereen
die met dezelfde gereedschappen werkte, vergelijkbaar was. Het creëren
van conceptuele rijkdom -- denk aan artistieke prestaties -- verschilt
echter enorm tussen mensen die met dezelfde middelen werken. In dat
opzicht lijkt de hele economie steeds meer op een operawereld, waarbij
de beste stemmen de hoogste beloningen krijgen en degenen die vals
zingen, hoe oprecht ook, doorgaans weinig verdienen. Naarmate meer
terreinen opengaan voor echte wereldwijde concurrentie, zal de beloning
voor gemiddelde prestaties onvermijdelijk dalen. Gemiddelde talenten
komen in overvloed voor, mede dankzij mensen die hun tijd kunnen
verhuren voor een fractie van de tarieven die in toonaangevende
industriële landen gebruikelijk zijn. De verliezers worden de
buitenvelders in de lagere competitie, met `slider speed bats', waarvan
de reflexen een halve seconde te traag zijn om een fastball uit de major
leagues te raken. In plaats van een miljoen dollar per jaar te verdienen
met homeruns, krijgen zij slechts \$25.000 binnen, zonder bijkomende
inkomsten uit beroemdheidssponsoring. Anderen zullen totaal mislukken.

\begin{quote}
`Zodra een land zich openstelt voor de wereldmarkt, ontstaan er winnaars
onder de burgers die beschikken over de juiste vaardigheden, terwijl
degenen zonder die talenten als verliezers of achterblijvers uit de bus
gaan. Vaak beweert een partij in staat te zijn de globalisering te
weerstaan of haar impact te verzachten. Denk hierbij aan Pat Buchanan in
Amerika, de communisten in Rusland en nu de \emph{Islamitische
Welzijnspartij} hier in Turkije. Wat er in Turkije gebeurt, is dan ook
veel complexer dan louter een fundamentalistische overname. Het is
precies het gevolg van een verdere globalisering die steeds meer
verliezers voortbrengt, terwijl een toenemende democratisering hen een
stem geeft -- en religieuze partijen deze samenloop van omstandigheden
effectief misbruiken om de macht te grijpen.'\footnote{Friedman,
  \emph{op. cit.}}
\end{quote}

--- THOMAS L. FRIEDMAN

Wie worden de verliezers in het informatietijdperk? In de meeste
gevallen blijken dit de belastingbetalers te zijn. Zij vergroten hun
vermogen doorgaans niet doordat verhuizen naar een andere jurisdictie
voor hen geen optie is. Een groot deel van hun inkomen is immers stevig
verankerd in de regels van hun nationale politieke systeem, in plaats
van tot stand te komen via marktwaarderingen. Daarom maakt het
afschaffen of drastisch verlagen van belastingen die negatief op hun
nettovermogen inwerken, hen niet per se beter af -- want een lagere
belastingdruk gaat altijd gepaard met minder overdrachtsuitkeringen. Zij
zullen inkomsten mislopen, omdat zij niet langer op politieke dwang
kunnen rekenen om geld af te tappen van mensen die productiever zijn.
Degenen zonder spaargeld, die afhankelijk zijn van de overheid voor hun
pensioen en zorg, zullen vrijwel zeker een terugval in hun
levensstandaard ondervinden.

Dit verlies in inkomen vertaalt zich in een waardevermindering van wat
de financieel schrijver Scott Burns `transcendentale' -- oftewel
politieke -- kapitaal noemt.\footnote{James Dale Davidson, \emph{The
  squeeze} (New York: Summit Books, 1980), pp.~38--55.} Dit
`transcendentale' of denkbeeldige kapitaal berust niet op economisch
eigendom van activa, maar op de feitelijke aanspraak op de
inkomensstroom die voortvloeit uit politieke regels en voorschriften. Zo
kan het verwachte inkomen uit overheidsuitkeringen worden omgezet in een
obligatie, waarvan de waarde wordt berekend aan de hand van de geldende
rentetarieven. Deze denkbeeldige obligatie, gefinancierd door de
ingebeelde gemeenschap, vormt het transcendente kapitaal. Door de `grote
transformatie', gericht op het verzwakken van de greep van politieke
autoriteiten op de kasstroom die nodig is om hun beloften waar te maken,
zal dit kapitaal plotseling in waarde dalen.

\begin{quote}
`Aan de grenzen en op de hoge zeeën, waar niemand blijvend een monopolie
op het gebruik van geweld had, konden handelaren de hoge afpersingen
ontwijken, waardoor alternatieve vormen van bescherming goedkoper werden
verkregen.'\footnote{Lane, `Economic consequences of organized
  violence', p.~404.}
\end{quote}

--- FREDERIC C. LANE

Het is niet moeilijk voor te stellen dat de informatie‑elite de kansen
zal benutten die de nieuwe cybereconomie biedt voor bevrijding en
persoonlijke soevereiniteit. Ook mag men verwachten dat de
achterblijvers steeds jingoïstischer en onaangenamer worden, naarmate de
impact van informatietechnologie in dit millennium toeneemt. Het blijft
lastig precies te voorspellen wanneer de reactie lelijk zal worden. Wij
vermoeden dat de wederzijdse beschuldigingen zullen intensiveren zodra
de westerse landen onmiskenbaar uiteenvallen, vergelijkbaar met wat
destijds bij de voormalige Sovjet‑Unie gebeurde.

Telkens als een natiestaat uiteenvalt, versnelt dat de decentralisatie
en vergroot het de autonomie van soevereine individuen. Wij verwachten
een aanzienlijke toename van soevereine entiteiten, waarbij tientallen
enclaves en rechtsgebieden die meer op stadstaten lijken, herrijzen uit
het puin van natiestaten. Deze nieuwe eenheden kiezen er vaak voor om
beschermingsdiensten tegen zeer concurrerende tarieven aan te bieden,
vaak met lage of zelfs geen belastingen op inkomen en kapitaal. Bijna
onvermijdelijk prijzen zij hun beschermingsdiensten aantrekkelijker dan
de vooraanstaande OESO‑natiestaten. Als men het louter vanuit
marktsegmentatie bekijkt, geldt dat het segment dat momenteel het
slechtst bediend wordt, op een efficiënte en goedkope wijze kan worden
aangeleverd. Iedereen die bereid is hoge belastingen te betalen in ruil
voor een ingewikkeld pakket aan staatsuitgaven, krijgt daarvoor volop de
gelegenheid. Daarom ligt de meest voordelige en winstgevende strategie
voor een nieuwe mini‑soevereiniteit vrijwel onvermijdelijk in een hoog
efficiënte, laaggeprijsde variant. Zo'n mini‑soevereiniteit kan met
grote moeite verwachten een uitgebreider dienstenpakket te bieden dan de
overgebleven natiestaten. Omdat niet alle natiestaten gelijktijdig
instorten, zal er in de beginfase van de transitie ruim voldoende keuze
zijn. Bovendien kan een sobere invulling van orde en veiligheid relatief
goedkoop worden gerealiseerd. Als sociale onrust en criminaliteit zich
in de traditionele kernindustrieën uitbreiden tot het niveau dat wij
voorspellen, zal een toereikende orde en veiligheid in een rechtsgebied
veel aantrekkelijker blijken dan een nationaal ruimteprogramma, een door
de staat gesponsord vrouwenmuseum of gesubsidieerde
omscholingsprogramma's voor ontslagen leidinggevenden.

\section{De denationalisatie van het
individu}\label{de-denationalisatie-van-het-individu}

Het burgerschap wordt minder aantrekkelijk en houdbaar naarmate nieuwe
instituties ontstaan die de keuze in de diensten, waarover de overheid
momenteel de touwtjes in handen heeft -- te beginnen met bescherming --
vergemakkelijken. Dit maakt het voor individuen praktisch om te stoppen
met zichzelf in nationale termen te identificeren. Toch verloopt de
demystificatie van het burgerschap traag. Je wordt voortdurend
geconfronteerd met een lawine van alledaagse boodschappen in je
dagelijkse routine, die erop gericht zijn je band met je lokale
natiestaat te versterken. Daardoor is het vrijwel onmogelijk om je
nationaliteit te vergeten. Voor veel mensen vormt nationaliteit immers
een essentieel identiteitssymbool. `Wij' leren de wereld in termen van
nationaliteit te zien. Het is ons land, `onze' sporters strijden op de
Olympische Spelen en wanneer zij winnen, wappert `onze' vlag tijdens de
ceremonie. Ons volkslied trekt tijdens de prijsuitreiking de aandacht
van juryleden en medekandidaten. Wij geloven dat de overwinning van ons
is, ook al blijft het onduidelijk wat onze bijdrage precies inhoudt,
afgezien van het feit dat we als burgers binnen hetzelfde grondgebied
verblijven.

\subsection{Van de eerste persoon meervoud naar het
enkelvoud}\label{van-de-eerste-persoon-meervoud-naar-het-enkelvoud}

Naarmate informatietechnologie steeds meer in de schijnwerpers komt te
staan, zal zij bijdragen aan het vormen van een mondiale blik en tevens
nieuwe manieren bieden voor onafhankelijke individuen om het verborgen
potentieel van deze technologie te benutten en zo te ontsnappen aan de
nationale lasten van belastingen.

Binnen enkele decennia zal narrowcasting het traditionele broadcasten
vervangen als methode om nieuws te vergaren. Dit heeft ingrijpende
gevolgen: het verandert de manier waarop miljoenen mensen zichzelf zien,
namelijk door een verschuiving van het collectieve `wij' naar het
individuele `ik'. Wanneer mensen zelf als hun eigen nieuwsredacteur
optreden en zelf bepalen welke onderwerpen en nieuwsberichten voor hen
belangrijk zijn, verkleint de kans dat zij zich laten indoctrineren in
de opofferingsplicht voor de natiestaat.

Ook door de privatisering van het onderwijs -- mede mogelijk gemaakt
door technologische ontwikkelingen -- zal een vergelijkbaar effect
optreden. In de middeleeuwen hield de Kerk het onderwijs stevig in eigen
hand, waarna in latere tijden de staat het overnam. Zoals Eric Hobsbawm
opmerkt: `staatsonderwijs transformeerde mensen in burgers van een
specifiek land: ``boeren in Fransen''\,'. In het informatietijdperk
krijgt onderwijs een private en individuele inslag en zal het niet
langer beladen zijn met de zware politieke bagage die het onderwijs
tijdens de industriële periode kenmerkte. Nationalisme zal niet meer in
elk hoekje van ons denken worden ingeprent.

De opkomst van het internet en het \emph{World Wide Web} zal bovendien
het belang van locatie in de handel doen afnemen. Hierdoor ontstaan er
individuele adressen die niet aan een specifiek territorium zijn
gebonden. Digitale satelliettelefoondiensten evolueren zó dat zij meer
bieden dan louter locatiegebonden vaste-lijnsystemen met een
gemeenschappelijke internationale belcode. Ieder individu krijgt zo een
uniek, wereldwijd telefoonadres -- vergelijkbaar met een internetadres
-- waarmee hij overal te bereiken is. Na verloop van tijd vallen
nationale postmonopolies in duigen, zodat wereldwijd opererende,
geprivatiseerde postbezorgsystemen mogelijk worden, zonder band met een
specifieke natiestaat.

Deze en andere schijnbaar kleinigheden zullen zowel de doorsnee
consument als de intellectuele elite bevrijden van de vastgeroeste
identificatie met de natiestaat. De ontmythologisering van het
burgerschap zal op spectaculaire wijze versnellen door de opkomst van
praktische alternatieven voor zakendoen binnen de door staten
gemonopoliseerde, afgebakende territoria. De fundamenten van de
cybereconomie -- denk aan cybergeld, cyberbankieren en een wereldwijde,
ongereguleerde cybermarkt voor effecten -- zullen vrijwel onvermijdelijk
op grote schaal vorm krijgen. Zodra dit gebeurt, neemt het vermogen van
hebzuchtige regeringen om de rijkdom van `burgers' te confisqueren,
aanzienlijk af.

Hoewel de hoofdmachten ongetwijfeld zullen proberen een kartel op te
leggen dat hoge belastingen en fiatgeld handhaaft -- door samen te
werken om encryptie te beperken en burgers te verhinderen hun
rechtsgebied te ontvluchten -- zullen zij uiteindelijk falen. De meest
ondernemende mensen op de planeet vinden immers hun weg naar economische
vrijheid. Het is onwaarschijnlijk dat de staat überhaupt in staat zal
zijn mensen op te sluiten op plekken waar zij als gijzelaars kunnen
dienen. Het ineffectieve karakter van pogingen om illegale immigranten
buiten te houden, bewijst dat natiestaten hun grenzen niet kunnen
afsluiten tegen het vertrek van succesvolle individuen. De rijken zullen
even daadkrachtig hun vertrek regelen als taxichauffeurs of obers dat
doen bij hun intrede.

Voor het eerst sinds de middeleeuwen, toen soevereiniteit nog
gefragmenteerd was, zullen grenzen niet scherp afgebakend zijn. Zoals
eerder aangetoond, ontstaat er geen centraal gebied waar de meeste
toekomstige financiële transacties plaatsvinden. In plaats van een reeks
verplichtingen op basis van geboorte te accepteren, zullen steeds meer
onafhankelijke individuen deze ambiguïteit benutten om aan hun
belastingplichten te ontsnappen en daarmee het traditionele burgerschap
achter zich te laten. Zij gaan als `klanten' privé-belastingverdragen
onderhandelen, vergelijkbaar met de regelingen die nu in Zwitserland
bestaan, zoals in hoofdstuk 8 is geanalyseerd. Een typisch
privé-belastingverdrag dat met de Franstalige Zwitserse kantons wordt
gesloten, stelt een individu of familie in staat om in het grondgebied
te verblijven in ruil voor een vaste jaarlijkse belasting van 50.000
Zwitserse frank (momenteel ongeveer \$45.000). Let op: dit betreft geen
belasting op basis van een vast percentage, maar een door de overheid
vastgesteld vast bedrag, onafhankelijk van het inkomen. Verdient u
jaarlijks 50.000 Zwitserse frank, zou zo'n overeenkomst resulteren in
een belastingtarief van 100 procent. Bij een inkomen van 500.000
Zwitserse frank bedraagt het tarief dan 10 procent, bij 5.000.000
slechts 1 procent en bij 50 miljoen slechts 0,1 procent. Als dit een
buitengewoon aantrekkelijke deal lijkt ten opzichte van een marginale
belasting van 58 procent in New York City, illustreert dat slechts hoe
uitbuitend en monopolistisch de overheidsdiensten tijdens de industriële
periode hun tarieven bepaalden.

In feite is 50.000 Zwitserse frank per jaar ruimschoots voldoende om de
noodzakelijke en nuttige overheidsdiensten te bekostigen. De Zwitsers
profiteren er ongetwijfeld van: iedere miljonair die zich hier vestigt,
moet voor dat voorrecht jaarlijks 50.000 Zwitserse frank betalen. In
veel gevallen bedragen de bijkomende kosten voor de overheid om een
extra miljonair binnen haar jurisdictie te ontvangen bijna niets,
waardoor de jaarlijkse opbrengst per transactie bijna op 50.000
Zwitserse frank uitkomt. Elke overheidsdienst die ver onder de
marktprijs wordt aangeboden, terwijl de goedkoopste aanbieder er toch
circa 100 procent winst mee behaalt, getuigt van monopolievorming en
extreme overprijsing. Opvallend is niet dat het belastingtarief als
percentage van het inkomen daalt, maar dat men in de twintigste eeuw
ooit als `eerlijk' beschouwde dat mensen radicaal verschillende bedragen
voor overheidsdiensten betaalden. Dat is des te opmerkelijker, want
degenen die het meest van overheidsdiensten profiteren, betalen het
minste, terwijl degenen die er nauwelijks gebruik van maken, het hoogste
bedrag moeten betalen. Al deze regelingen leveren iedere hoogverdienende
Amerikaan een domicilievoordeel op ten opzichte van de Verenigde Staten,
dat over een heel leven kan oplopen tot tientallen miljoenen. Tenzij de
Amerikaanse belastingen zodanig hervormd worden dat ze concurrerender
zijn met die van andere jurisdicties en niet langer op basis van
nationaliteit worden geheven, zullen verstandige burgers hun Amerikaans
burgerschap opgeven -- ondanks de obstakels van Clintons exitbelasting
-- om paspoorten te bemachtigen met minder zware verplichtingen.

Overheden in het industriële tijdperk schreven de prijs voor hun
diensten op basis van het succes van de belastingbetaler, in plaats van
op basis van de werkelijke kosten of de waarde van de geleverde
diensten. De verschuiving naar een commerciële prijszetting voor
overheidsdiensten zorgt voor een aantrekkelijkere bescherming tegen een
aanzienlijk lagere prijs dan die door conventionele natiestaten wordt
afgedwongen.

\subsection{Burgerschap gaat de weg van de
ridderlijkheid}\label{burgerschap-gaat-de-weg-van-de-ridderlijkheid}

Kortom, burgerschap zal een pad volgen dat vergelijkbaar is met dat van
de ridderlijkheid. Nu we de basis voor bescherming opnieuw inrichten,
veranderen vanzelf ook de rechtvaardigingen en motiverende ideologieën
die het systeem ondersteunen. Een half millennium geleden, aan het einde
van de middeleeuwen, reageerden mensen voorspelbaar: toen het verlenen
van bescherming in ruil voor persoonlijke diensten doorgaans niet meer
rendabel bleek, lieten zij de ridderlijkheid achter zich. Gezwoer en
persoonlijke trouw werden niet meer zo gewaardeerd als in de voorgaande
vijf eeuwen.

Informatietechnologie belooft het burgerschap op een vergelijkbaar
subversieve manier te transformeren. De natiestaat en de aanspraken van
het nationalisme verliezen geleidelijk hun mystieke allure, net zoals
vijf eeuwen geleden de aanspraken van de monopolistische kerk werden
ontmaskerd. Hoewel reactionairen vernieuwers proberen te demoniseren en
het nationalistische sentiment nieuw leven in te blazen, betwijfelen wij
of de politiek verouderde natiestaat nog in staat is voldoende
loyaliteit op te roepen om de door informatietechnologie veroorzaakte
competitieve druk te weerstaan. De meeste kritische individuen in een
wereld vol failliete overheden kiezen er eerder voor als klanten van
beschermingsdiensten goed behandeld te worden dan als burgers van
natiestaten geplunderd te worden.

De welvarende OECD-landen leggen een zware fiscale en regelgevende last
op aan iedereen die binnen hun grenzen onderneemt. Deze kosten waren
wellicht nog draaglijk toen de OECD-landen de enige rechtsgebieden waren
waar men zowel redelijk kon ondernemen als prettig kon wonen, maar die
tijden zijn voorbij. De extra heffingen die inwoners betalen omdat ze in
de rijkste natiestaten wonen, wegen niet meer op tegen de voordelen. Nu
de concurrentie tussen rechtsgebieden toeneemt, wordt dit steeds
onacceptabeler. Wie beschikt over de middelen en het kapitaal om de
uitdagingen van het informatietijdperk het hoofd te bieden, kan zich
overal vestigen en zaken doen. Als men kan kiezen uit meerdere
domicilies, blijven uiteindelijk alleen de meest patriotische of domme
mensen wonen in landen met hoge belastingen.

Om die reden moet men verwachten dat één of meer natiestaten stiekem
maatregelen gaan treffen om de aantrekkingskracht van een tijdelijk
verblijf te verminderen. Reizen kan bijvoorbeeld actief ontmoedigd
worden door biologische oorlogsvoering -- denk aan het opzettelijk laten
uitbreken van een dodelijke epidemie. Dit zou niet alleen de reislust
temperen, maar landen wereldwijd ook een argument geven om hun grenzen
te sluiten en de immigratie te beperken.

\subsection{Het nadeel van
nationaliteitsbelasting}\label{het-nadeel-van-nationaliteitsbelasting}

Tenzij er een baanbrekende en bijna wonderbaarlijke beleidswijziging
plaatsvindt, krijgt de succesvolle belegger of ondernemer in het
informatietijdperk een levenslange financiële last opgelegd -- die kan
oplopen tot tientallen miljoenen, honderden miljoenen of zelfs miljarden
dollars -- wanneer hij in landen wil wonen die een fiscaal beleid
hanteren als dat van de landen die in de twintigste eeuw de hoogste
levensstandaard hadden.

Zonder een radicale verandering valt deze last voor Amerikanen het
zwaarst uit. De Verenigde Staten behoren tot de weinige rechtsgebieden
ter wereld die belastingen heffen op basis van nationaliteit in plaats
van op basis van woonplaats. De overige twee landen die op deze manier
te werk gaan, zijn de Filippijnen -- een voormalige Amerikaanse kolonie
-- en Eritrea, waar tijdens de langdurige opstand tegen het Ethiopische
bewind één van de verbannen leiders in de ban raakte van de \emph{IRS}.
Tegenwoordig heft Eritrea een nationaliteitsbelasting van 3 procent.
Hoewel dit een schamele imitatie is van de Amerikaanse tarieven,
betekent zelfs die geringe last dat het Eritrese staatsburgerschap in
het informatietijdperk een zware prijs verlangt. De huidige regelgeving
zorgt er bovendien voor dat het Amerikaanse staatsburgerschap nog
zwaarder belast wordt. De \emph{IRS} is inmiddels uitgegroeid tot een
van de toonaangevende exportproducten van Amerika. Meer dan enig ander
land reikt de Verenigde Staten wereldwijd uit om inkomsten te innen van
hun staatsburgers.

Stel je voor dat een 747-jet, gevuld met één belegger uit ieder
rechtsgebied ter wereld, in een net onafhankelijk geworden land aan land
gaat en dat iedere belegger daarnaast \$1.000 op het spel zet in een
start-up binnen de nieuwe economie. In dat scenario betaalt een
Amerikaan op eventuele winsten aanzienlijk meer belasting dan iedere
andere belegger. De bijzondere, in feite strafbelastende heffing op
buitenlandse investeringen -- zoals blijkt uit de zogenoemde
PFIC-belasting -- in combinatie met de Amerikaanse
nationaliteitsbelasting kan ertoe leiden dat belastingverplichtingen
oplopen tot 200 procent of meer op langetermijnactiva die buiten de
Verenigde Staten worden aangehouden. Een succesvolle Amerikaan kan zijn
levenslange belastingdruk verlagen door staatsburger te worden van een
van meer dan 280 andere rechtsgebieden wereldwijd.

De Verenigde Staten beschikken over het meest roofzuchtige, de rijken
uitpersende belastingstelsel ter wereld. Zowel binnen als buiten de VS
worden Amerikanen meer als een numerieke last behandeld en minder als
gewaardeerde klanten vergeleken met staatsburgers uit andere landen.
Daardoor blijkt het Amerikaanse belastingregime niet alleen achterhaald,
maar ook minder geschikt voor succes in het informatietijdperk dan zelfs
de berucht hoogbelaste verzorgingsstaten in Scandinavië. Burgers in
Denemarken of Zweden ondervinden nauwelijks juridische belemmeringen bij
het realiseren van hun toenemende technologische autonomie als individu.
Wil men zijn eigen belastingtarief bepalen, dan kan men er bijvoorbeeld
voor kiezen om in Zwitserland belast te worden op basis van een
privéverdrag of naar Bermuda te verhuizen, zodat men helemaal geen
inkomstenbelasting betaalt. Een Zweed of Deen die bereid is hoge
belastingen te betalen omdat hij overtuigd is van de meerwaarde van de
Scandinavische verzorgingsstaat, maakt daar bewust voor zijn keuze. Hij
kan belasting betalen tegen elk tarief dat in een ander rechtsgebied
geldt -- of dat nu in de beschaafde of in de onbeschaafde wereld
gebruikelijk is. Om zijn belastingtarief aan te passen, hoeft hij enkel
te verhuizen. Technologie maakt zo'n beslissing met de dag makkelijker,
maar Amerikanen wordt die mogelijkheid ontzegd. Het bezit van een
Amerikaans paspoort staat immers op het punt een groot nadeel te worden
bij het benutten van de kansen op individuele autonomie die de
informatie-revolutie biedt. In de industriële periode werd het als een
fortuinlijke omstandigheid beschouwd om als Amerikaan geboren te worden
-- maar in de vroege fase van het informatietijdperk blijkt dat
inmiddels een last ter waarde van meerdere miljoenen dollars te worden.

Om te illustreren hoe zwaar deze last weegt, laten we een vergelijking
maken. Onder redelijke aannames betaalt een Nieuw‑Zeelander met dezelfde
pre-belaste inkomensprestaties als het gemiddelde van de top 1 procent
van de Amerikaanse belastingplichtigen zoveel minder belasting dat de
samengestelde besparingen hem op lange termijn rijker maken dan welke
Amerikaan dan ook. Aan het einde van zijn leven zou de Nieuw‑Zeelander
maar liefst \$73 miljoen extra overhouden om door te geven aan zijn
kinderen of kleinkinderen. En Nieuw‑Zeeland is niet eens officieel een
belastingparadijs. Meer dan veertig andere rechtsgebieden hebben immers
lagere belastingtarieven op inkomen en vermogen. Als deze redenering
klopt, zal het aantal landen met lage belastingen waarschijnlijk
toenemen in plaats van afnemen. Al deze jurispudenties bieden als
vestigingsplaats een voordeel ten opzichte van de Verenigde Staten dat
over een heel leven gezien tientallen, zo niet honderden miljoenen
dollars kan opleveren. Tenzij de Amerikaanse belastingen zodanig
hervormd worden dat zij concurrerender worden met die van andere landen
en niet langer op nationaliteitsbasis worden geheven, zullen verstandige
mensen uiteindelijk afstand doen van hun Amerikaans staatsburgerschap --
ondanks de obstakels die de exitbelasting van Clinton met zich
meebrengt.

De competitieve condities van het informatietijdperk maken het mogelijk
bijna overal hoge inkomens te verdienen. In feite zullen de
locatiegebonden monopolies die natiestaten inzetten om extreem hoge
belastingen af te dwingen door technologische vernieuwing teniet worden
gedaan. Ze brokkelen nu al af en naarmate ze verder verzwakken, zal de
druk van de concurrentie de meest ondernemende en capabele mensen
vrijwel onvermijdelijk dwingen landen met exorbitante belastingtarieven
achter zich te laten. Zoals voormalig hoofdredacteur van \emph{`The
Economist'}, Norman Macrae, het verwoordde: zulke landen `zullen
residueel bewoond worden, voornamelijk door dummies.'

\begin{quote}
`{[}B{]}ij het jaar 2012 zullen de geraamde uitgaven voor aanspraken en
rente op de nationale schuld alle door de federale overheid geïnde
belastinginkomsten opslokken. \ldots{} Er zal geen cent overblijven voor
onderwijs, kinderprogramma's, snelwegen, nationale verdediging of enig
ander discretionair programma.' - \emph{`BIPARTISAN U.S. COMMISSION ON
ENTITLEMENT AND TAX REFORM'}
\end{quote}

De uittocht van de rijken uit hoogontwikkelde verzorgingsstaten valt
precies samen met het demografisch ongelijke moment. Aan het begin van
de eenentwintigste eeuw krijgen de vergrijzende bevolkingsgroepen in
Europa en Noord-Amerika te maken met een schrale spaarsom, waardoor zij
hun medische kosten en levensstijl tijdens de pensioenjaren niet kunnen
bekostigen. Zo blijkt dat maar liefst 65 procent van de Amerikanen
helemaal geen spaargeld voor hun pensioen heeft -- helemaal niets. En
degene die wel sparen, sparen ver te weinig. De gemiddelde Amerikaan
bereikt de leeftijd van vijfenzestig, maar rekent ermee dat hij voor
meer dan \$200.000 aan medische rekeningen komt te zitten vóór zijn
overlijden en een nettovermogen heeft van minder dan \$75.000. Zelfs de
weinigen met privépensioenen zullen waarschijnlijk niet in comfortabele
omstandigheden leven, want het gemiddelde pensioen dekt slechts 20
procent van het inkomen dat men vóór pensionering verdiende. Daarbij
vertegenwoordigen de meeste bezittingen van de typische gepensioneerde
geen werkelijk vermogen, maar `transcendent kapitaal' -- de te
verwachten waarde van toekomstige overdrachtsuitkeringen. Mensen raken
er inmiddels aan gewend dat men op deze uitkeringen vertrouwt om de
kloof in hun eigen middelen op te vullen, terwijl de addert onder het
gras is dat deze uitkeringen waarschijnlijk niet gerealiseerd worden.
Pay‑as‑you‑go-systemen beschikken simpelweg niet over de benodigde
cashflow of middelen om aan deze verplichtingen te voldoen. Uit een
onderzoek van Neil Howe blijkt dat, zelfs als de Amerikaanse
voorbelastinginkomsten sneller zouden stijgen dan in de afgelopen
twintig jaar, de gemiddelde na‑belastinginkomsten tegen 2040 met 59
procent moeten dalen om de huidige niveaus van \emph{Social Security} en
overheidsgezondheidsprogramma's te bekostigen.

Je kunt dit probleem niet zomaar omzeilen. De verzorgingsstaat staat op
de rand van insolventie en haar financieringskwesties zijn zelfs
nijpender in Europa dan in Noord-Amerika. Italië vormt wellicht het
ergste voorbeeld, gevolgd door Zweden en andere Noordse
verzorgingsstaten die bekend staan om hun royale inkomensondersteuning.
\emph{Financial Times} schat dat, wanneer de contante waarde van de
Italiaanse staats pensioenen wordt meegerekend, de publieke schuld van
het land meer dan 200 procent van het BBP zal bedragen.

Een schuldenlast van dergelijk niveau is volgens de cijfers vrijwel
onhoudbaar. Een diepgravend onderzoek naar de commerciële schuldenlast
van bedrijven op de Toronto Stock Exchange, uitgevoerd enkele jaren
geleden, toonde aan dat nauwelijks iemand een schuldratio kan overleven
die al een kwart zo extreem is als die waarmee de toonaangevende
verzorgingsstaten vandaag worstelen.

Kortom, zij zijn blut. Wanneer men deze onvermijdelijke realiteit -- zij
het met tegenzin -- eindelijk onder ogen ziet, volgt er letterlijk een
afschrijving van biljoenen aan niet-gefinancierde aanspraken.

Zo werkt de logica van de cybereconomie. Een mogelijke belemmering is
simpelweg traagheid -- dat nestinstinct waardoor mensen huiverig zijn
hun vertrouwde stek op te geven. Als er andere obstakels zijn, zitten
die mogelijk diep in de menselijke aard verankerd. De economische logica
achter het inzetten van activa in cyberspace kan echter botsen met de
biologische basis, die zich uit in een diepgeworteld wantrouwen jegens
buitenstaanders. In elke cultuur tonen kinderen van nature een afkeer
van vreemden. Tegenstanders van de commercialisering van soevereiniteit
doen er alles aan om twijfel te zaaien over de nieuwe mondiale cultuur
van het informatiestijdperk en over het verval van de natiestaat dat
daarmee gepaard gaat. Een andere mogelijke belemmering, die voortkomt
uit epigenese -- oftewel genetisch beïnvloede motivatiefactoren -- is
het vooruitzicht dat de `verliezers en achterblijvers' zullen reageren
op ontwikkelingen die de natiestaat ondermijnen met de woede van
jager-verzamelaars die hun familie beschermen. In een omgeving waarin
gedesoriënteerde en vervreemde mensen meer ruimte krijgen om te
verstoren en te vernietigen, kan een terugslag tegen de
informatie-economie leiden tot gewelddadige en onaangename gevolgen.

\begin{quote}
`Historisch gezien is collectief geweld regelmatig voortgekomen uit de
centrale politieke processen van Westerse landen. Mensen die erop uit
zijn om de machtshefbomen te grijpen, te behouden of te herschikken,
hebben in hun strijd steeds gebruikgemaakt van collectief geweld. De
onderdrukten hebben in naam van gerechtigheid geslagen, de bevoorrechten
in naam van orde, en de daartussen degenen in naam van angst. Grote
verschuivingen in de machtsverhoudingen hebben doorgaans geleid -- en
waren vaak afhankelijk van -- uitzonderlijke momenten van collectief
geweld.'\footnote{Tilly, `collective violence in European perspective,'
  p.~62.}
\end{quote}

\begin{quote}
CHARLES TILLY
\end{quote}

\section{Geweld in perspectief}\label{geweld-in-perspectief}

Er bestaan minstens twee concurrerende theorieën over de oorzaken van
geweld in veranderende omstandigheden. Historicus Charles Tilly vat een
van deze ideeën als volgt samen: `{[}D{]}e prikkel voor collectief
geweld komt grotendeels voort uit de angsten die mensen ervaren wanneer
gevestigde instituties uiteenvallen. Als ellende of gevaar de angst
verergert, wordt de reactie des te gewelddadiger.' Tilly betoogt dat
geweld niet primair uit angst voortkomt, maar een weloverwogen poging is
om de autoriteiten te dwingen hun verantwoordelijkheden na te komen --
gedreven door een `gevoel van ontzegde rechtvaardigheid.' Volgens Tillys
interpretatie leiden `grote structurele veranderingen' vaak tot
collectief geweld met een politieke inslag. In plaats van een abrupte
breuk met het `normale' politieke leven te markeren, gaan gewelddadige
opstanden vaak gepaard met georganiseerde, vreedzame inspanningen van
dezelfde mensen om hun doelstellingen te bereiken; ze vullen deze
aanvullen en versterken ze. Ze maken immers deel uit van dezelfde
dynamiek als geweldloos verzet.\footnote{Ibid., p.~68.}

Ongeacht welke theorie over de oorzaken van geweld het meest klopt,
lijken de vooruitzichten op sociale vrede tijdens de Grote Transformatie
beperkt. De ineenstorting van de natiestaat vormt immers een opvallend
voorbeeld van een gevestigde orde die uiteenvalt. Hierdoor zullen de
angsten waarschijnlijk sterk toenemen, net als de politieke stimulans
voor geweld. Dit geldt vooral in vooraanstaande verzorgingsstaten, waar
de bevolking gewend is aan een zekere mate van inkomensgelijkheid. Omdat
de mensen die nu volwassen worden in de vroege fase van de
informatie-economie zijn opgegroeid in een industriële tijd waarin
politieke autoriteiten klachten met materiële voordelen konden
beslechten, mag men verwachten dat de `achterblijvers' blijven
aandringen op materiële voorzieningen. Waarschijnlijk vergt het een
langzame en pijnlijke kennismaking met de realiteit van de cybereconomie
voordat de bevolking van de OESO loskomt van de verwachting dat
grootschalige inkomensherverdeling afgedwongen kan worden. Hoe dan ook,
of geweld nu voortvloeit uit angst of het resultaat is van een
weloverwogen poging om te profiteren van systematische dwang, de
omstandigheden maken geweld bijna onvermijdelijk.

\subsection{Achterban van verliezers}\label{achterban-van-verliezers}

De ondergang van de gedwongen inkomensherverdeling zal ongetwijfeld
degenen van streek maken die gewend zijn te rekenen op de biljoenen die
via overdrachtprogramma's worden verdeeld. Dit betreft doorgaans `de
verliezers' of `de achterblijvers', mensen die niet over de vaardigheden
beschikken om op de wereldmarkt te concurreren. Net zoals de
gepensioneerden in de voormalige Sovjet-Unie, die de kern vormden van
Zuganovs communistische aanhang, zullen de teleurgestelde
gepensioneerden van de vervalende verzorgingsstaten een reactionaire
achterban vormen. Zij zullen fel protesteren tegen de privatisering van
natiestaatelijke soevereiniteit, waardoor de staat zijn vergunning om te
stelen verliest. Naarmate zij beseffen dat de regeringen die zij eens
beheersten hun controle over hulpbronnen en hun vermogen om
grootschalige inkomensoverdrachten af te dwingen kwijtraken, worden zij
net zo vastberaden als Franse ambtenaren in hun strijd tegen de
rekenkunde.

U herinnert zich vast nog de heftige reactie die volgde op de tamelijk
bescheiden voorstellen van premier Alain Juppe om de `demografisch
onhoudbare' pensioenregelingen voor staatsmedewerkers in te krimpen en
de bedrijfsvoering van het genationaliseerde spoorwegsysteem efficiënter
in te richten. Een treffend symbool van de absurditeit van l'État
Providence, zoals de Fransen hun sociale zekerheidsstaat noemen, blijkt
uit de regel die ingenieurs op de geavanceerde hogesnelheidstreinen
\emph{TGV} toestaat om al op vijftigjarige leeftijd met pensioen te gaan
-- net als hun voorgangers die op door kolengestookte locomotieven
ploeterden.\footnote{Dick Howard, `french toast: can politicians
  anywhere tangle with entitlements without getting burned?' \emph{The
  New Democrat}, juli/augustus 1996, p.~39f.} Een heftig verzet tegen de
bezuinigingen op zulke onhoudbare voordelen is in elk OESO-land reëel.
Zelfs in landen waar de bevolking minder boos reageert, kunt u erop
rekenen dat de waarschijnlijke verliezers er alles aan zullen doen om de
inperking van staatsdwang te voorkomen.

Dit alles zal leiden tot enkele onverwachte wendingen. In de Verenigde
Staten heeft het nativistische sentiment van oudsher duidelijke
racistische ondertonen -- een traditie die begon met de \emph{White
Caps} en de \emph{Ku Klux Klan} uit de negentiende eeuw. Toch profiteren
zwarte mensen als groep in de eerste plaats van inkomensoverdrachten,
positieve discriminatie en andere vruchten van politieke dwang.
Bovendien zijn zij onevenredig vaak vertegenwoordigd in het Amerikaanse
leger. Daarom zullen zij, samen met de blanke arbeidersklasse,
waarschijnlijk opstaan als enkele van de meest vurige aanhangers van het
Amerikaanse nationalisme.

Politici die inspelen op de onzekerheden van degenen wiens talenten
relatief ver achterblijven bij Ammon's rapen, zullen in vrijwel elk land
luidruchtig de kop op de voorgrond zetten. Van Slobodan Milosevic in
Servië tot Pat Buchanan in de Verenigde Staten, van Winston Peters in
Nieuw-Zeeland tot Necmettin Erbakan van de fundamentalistische
Islamitische Welvaartspartij in Turkije -- demagogen zullen fel
protesteren tegen de globalisering van markten, immigratie en de
vrijheid om te investeren.

Degenen die zichzelf zien als `slachtoffers van de wereldeconomie'
richten hun specifieke vijandigheid op zowel de rijken als de
immigranten. Volgens Andrew Heal zullen zij `de binnenkomst van
immigranten verachten, waarvan het voornaamste toelatingscriterium
schijnbaar hun rijkdom of juist het ontbreken daarvan is, wat hen
volgens de schijnbare logica tot een last voor de verzorgingsstaat
maakt.'

\subsection{Angst voor vrijheid}\label{angst-voor-vrijheid}

Het vooruitzicht dat de natiestaat aan het begin van het nieuwe
millennium verdwijnt, lijkt zo getimed dat het voor de meest kwetsbaren
in hun leven de maximale ontwrichting veroorzaakt. Dit leidt tot
wijdverspreide onvrede. Velen merken op dat mensen die zich
buitengesloten voelen door de gedachte aan een wereld zonder grenzen
vaak op een herkenbare manier reageren. Naarmate de brede, inclusieve
nationale groep uiteenvalt en de meer mobiele `informatie-elite' haar
zaken globaliseert, zoeken de achterblijvers houvast in het lidmaatschap
van etnische subgroepen, stammen, bendes of religieuze en taalkundige
minderheden. Dit is deels een praktische reactie op het wegvallen van
diensten -- zoals de handhaving van orde -- die de staat vroeger
verzorgde. Voor mensen met weinig verhandelbare middelen is het vaak
lastig om toegang te krijgen tot marktalternatieven wanneer de publieke
voorzieningen falen.

Wat ooit als publieke goederen werden gezien -- onderwijs, schoon water
en buurtpolitiezorg -- verandert nu in particuliere producten. Degenen
met voldoende middelen om kwalitatief hoogwaardige alternatieven aan te
schaffen, weten hier hun voordeel mee te doen. Voor wie op contant geld
wil vertrouwen, is het praktisch om op familie te rekenen of zich aan te
sluiten bij een etnisch gebaseerde hulpgroep, zoals de oude etnische
Chinese `Hokkien' in Zuidoost-Azië, of bij een religieuze gemeenschap.
In regio's waar dynamische, missionerende religies actief zijn, winnen
hun programma's mede aan populariteit doordat ze terugvallen op
traditionele methoden voor het leveren van sociale voorzieningen en
publieke diensten. Zo namen bijvoorbeeld door moslims geleide
waakhondgroepen een voortrekkersrol in de bestrijding van gewelddadige
bendes in Kaapstad, Zuid-Afrika.\footnote{Roger Matthews, `Zuid-Afrika
  roept troepen op voor oorlog tegen criminaliteit,' \emph{Financial
  Times}, 31 augustus/1 september 1996, p.~1.}

Hoe nuttig en effectief die etnische en religieuze hulporganisaties ook
blijken te zijn, er speelt meer mee in de reactionaire respons op het
afbrokkelen van de staat. Daarbij speelt ook een sterk psychologisch
element een belangrijke rol in de tegenreactie op globalisering.

Dit standpunt sluit nauw aan bij de psychologische verklaring voor de
aantrekkingskracht van het fascisme, zoals Erich Fromm beschreef in zijn
beroemde werk `vrees voor vrijheid', dat voor het eerst verscheen in
1941.\footnote{Erich Fromm, \emph{Angst voor vrijheid} (Londen:
  Routledge \& Kegan Paul, 1942).} Fromm betoogt dat de sociale
mobiliteit die het kapitalisme met zich meebracht, de stabiele
identiteiten in het traditionele dorpsleven verbrijzelde. De zoon van
een boer wist niet meer zeker dat hij onvermijdelijk het boerenambacht
zou overnemen of gedoemd was te ploeteren op de magere grond waarop zijn
vader werkte. Plotseling had hij keuzevrijheid: hij kon leraar,
handelaar of soldaat worden, geneeskunde studeren of de zee opgaan.
Zelfs als boer bleef hij niet langer gebonden aan de oude gewoonten; hij
kon naar de Verenigde Staten, Canada of Argentinië emigreren en ver van
het ouderlijk huis een nieuw bestaan opbouwen. Deze vrijheid om eigen
identiteit te creëren, bood het kapitalisme zeker kansen, maar maakte
het ook angstaanjagend voor hen die niet bereid waren die vrijheid
creatief te benutten. Zoals Billig opmerkte, hunkerden zij naar `de
zekerheid van een solide identiteit' en voelden zij zich aangetrokken
tot `de eenvoud van nationalistische en fascistische
propaganda.'\footnote{Billig, \emph{op. cit.}, p.~137.} Eveneens
omschrijft Billig de schemering van het industriële tijdperk als volgt:
`Er is een mondiale psychologie die de natie van bovenaf treft en
loyaliteiten doet verwelken door het vrije spel van identiteiten. En dan
is er de vurige psychologie van kaste of stam, die met een krachtige
intolerante inzet en emotionele fellerheid de kwetsbare onderbuik van de
staat raakt.'\footnote{Ibid., p.~135.}

Andrew Heal benadert dit fenomeen vanuit een ander perspectief. Hij
onderscheidt twee grote wereldwijde politieke en economische trends:
enerzijds groeit de wereldeconomie, en anderzijds nemen
nationalistische, etnische en regionalistische sentimenten toe -- of het
nu gaat om de Maori, Schotten, Welsh of anti-immigrantgroepen -- die,
zelfs wanneer hun regeringen hen naar nieuwe, grenzeloze horizonten
drijven, zichzelf juist nog sterker in de tegenovergestelde richting
oriënteren. Hoe men er ook naar kijkt -- of men deze ontwikkelingen
beschouwt als belangrijke trends of als psychologische thema's -- het
valt niet te ontkennen dat het reactionaire sentiment dat het
nationalisme omarmt en zich verzet tegen het verdwijnen van grenzen en
de verdere integratie van markten wereldwijd, steeds sterker wordt.

\section{multiculturalisme en
slachtofferschap}\label{multiculturalisme-en-slachtofferschap}

Toen de verzorgingsstaat in haar laatste fase kwam en haar vermogen om
loze beloftes waar te maken al wankeler werd, ging ze nieuwe mythen over
discriminatie koesteren. Overal in Noord-Amerika benoemde men diverse
groepen als officieel `onderdrukte' mensen. Men vertelde individuen die
tot deze zogenaamde slachtoffergroepen behoorden dat zij niet
verantwoordelijk waren voor de gebreken in hun eigen leven. Integendeel,
men wees `dode blanke mannen' van Europese afkomst aan als schuldigen en
stelde dat een onderdrukkende machtsstructuur de uitgesloten groepen
benadeelde. Als je zwart, vrouwelijk, homoseksueel, Latino, Franstalig
of gehandicapt was, betekende dat dat je recht had op compensatie voor
vroegere discriminatie en onderdrukking.

Volgens Lasch had het versterken van het slachtoffergesin als doel
naties te verzwakken, zodat de nieuwe, ongebonden informatielobby
gemakkelijker aan de verplichtingen van het burgerschap kon ontsnappen.
Wij geloven echter niet dat de nieuwe elite, vooral de mensen in de
massamedia, zo sluw is dat zij zo'n houding zouden aannemen. Het zou
bijna geruststellend klinken als dat wel zo was. Wij zien de groei van
slachtofferschap vooral als een poging om de sociale vrede te kopen,
niet alleen door -- zoals Lasch betoogt -- de instroom in de
meritocratie te vergroten, maar ook door de rechtvaardigingen voor
inkomensherverdeling te herstructureren. In Noord-Amerika bereikte
victimologie haar overdreven vorm, mede doordat de informatietechnologie
daar dieper doordrong. Wij vermoeden echter dat nieuwe mythen over
discriminatie, zij het in een ietwat verouderde variant, uiteindelijk in
alle geïndustrialiseerde samenlevingen zullen voorkomen. De
multietnische verzorgingsstaten in Noord-Amerika gaven sneller toe aan
de verleiding om de kosten van inkomensherverdeling op de privésector af
te schuiven. Ze slaagden erin om een gevoel van benadeeldheid en van
recht op compensatie wakker te maken door de gehele maatschappelijke
structuur -- en in het bijzonder blanke mannen -- de schuld te geven
voor de economische tekortkomingen binnen verschillende subculturen.

De megapolitiek van innovatie

Nog voordat informatietechnologie dreigde met de `creatieve
vernietiging' van de industriële economie, weerlegde zij al duidelijk
veel van de gekoesterde mythen van marxisten en socialisten. In een
eerder hoofdstuk onderzochten we de megapolitiek van innovatie. Wat we
daar benadrukten helpt om de maatschappelijke impact van de
informatierevolutie in perspectief te plaatsen. Hoewel technologie in de
afgelopen eeuwen de arbeidskansen aanzienlijk heeft vergroot, garandeert
dat geen blijvende economische regelmaat. Het is namelijk mogelijk dat
de opbrengsten zich concentreren in de handen van een welvarende
minderheid.

\section{Reële lonen dalen met 50
procent}\label{reuxeble-lonen-dalen-met-50-procent}

Precies dat gebeurde in de eerste twee eeuwen -- of zelfs langer -- van
de moderne periode. Rond de buskruitrevolutie, zo'n periode van ongeveer
1500 tot 1700, zakten de reële inkomens van de onderste 60 tot 80
procent in West-Europa met vijf decimale procenten of meer. In veel
streken bleef het reële inkomen dalen tot ongeveer 1750 en keerde het
pas in 1850 weer terug op het niveau van 1500.

In tegenstelling tot de ontwikkelingen van de afgelopen 250 jaar gingen
de inkomenswinsten in de eerste helft van de moderne periode -- een tijd
waarin de West-Europese economieën dramatisch groeiden -- enkel naar een
kleine minderheid. De hedendaagse doorbraak in informatietechnologie
verschilt fundamenteel van de industriële vernieuwingen van voorgaande
eeuwen. Moderne, arbeidsbesparende innovaties leiden er juist toe dat
werk sterk gespecificeerd raakt en schaalvoordelen afnemen -- precies
het tegenovergestelde van wat we sinds circa 1750 zagen.

Industriële vernieuwing bood doorgaans werkgelegenheid aan
laaggeschoolde arbeiders en vergrootte de schaalvoordelen voor
bedrijven. Hierdoor stegen de inkomens van de minder bedeelden, vaak
zonder dat zij er veel voor hoefden te doen, en versterkten deze
veranderingen tevens de macht van politieke systemen, waardoor zij beter
bestand waren tegen sociale onrust. Wie in de vroege fase van de
Industriële Revolutie door mechanisatie en automatisering zijn baan
verloor, betrof meestal vakbekwame ambachtslieden en gezellen -- niet de
ongeschoolde werknemers. Dit kwam duidelijk naar voren in de
textielindustrie, de eerste sector die op grote schaal mechanisatie en
krachtapparatuur toepaste. De opkomst van deze machines leidde tot een
hevige opstand van de Luddieten, die textielmachines vernietigden en
fabriekseigenaren vermoordden in een razernij aan het begin van de
negentiende eeuw. Aan de andere kant behoorden dagloners tot de
volgelingen van Captain Swing, de legendarische leider van een opstand
in 1830 in zuidoostelijk Engeland. Zij eisten dat lokale welgestelden
een belasting invoerden om hen geld of bier te verschaffen, dat de
loonsverhoging bij dagloners bij werkgevers werd afgedwongen en dat
nieuwe landbouwmachines -- met name dorsmachines -- werden vernietigd,
omdat deze de vraag naar landelijke dagarbeiders bij boeren deden
afnemen.

In tegenstelling tot het romantische gepraat van marxisten en anderen,
die de tegenstanders van arbeidsbesparende technologie als helden
opdroegen, bestond dit gezelschap uit onaangename en gewelddadige
figuren die zich uitsluitend uit eigenbelang verzette tegen de
introductie van technologie die wereldwijd de levensstandaard
verbeterde.

Terwijl de gewelddadige volgelingen van Ned Ludd en Captain Swing
maandenlang de openbare orde in Engeland verstoorden, was het
onvermijdelijk dat hun opstanden mislukten zodra een centrale autoriteit
hen de kop indrukte. Bovendien sloeg de arme, laaggeschoolde meerderheid
al snel de oproep af om machines te vernietigen, omdat diezelfde
machines niet alleen banen creëerden, maar ook de kosten van
basisgoederen zoals warme kleding en brood verlaagden en zo hun
levensstandaard verbeterden.

\subsection{Hogere inkomsten voor de
laaggeschoolden}\label{hogere-inkomsten-voor-de-laaggeschoolden}

Na verloop van tijd ontdekten de minderbedeelden dat industriële en
agrarische automatisering nieuwe inkomenskansen bood en hun kosten van
levensonderhoud verlaagde. Dankzij technische vernieuwingen kon ook
iemand zonder specialistische kennis goederen produceren van dezelfde
kwaliteit als die door hoogopgeleiden werden vervaardigd. Op een lopende
band zou zowel een genie als een idioot tot hetzelfde product komen en
evenveel verdienen.

In de afgelopen twee eeuwen heeft industriële automatisering de lonen
voor ongeschoolde arbeid drastisch doen stijgen, vooral in dat kleine
deel van de wereld waar als eerste de omstandigheden voor het
kapitalisme aanwezig waren. De grootschaligheid van geavanceerde
industriële ondernemingen beloonde ongeschoolde werknemers niet alleen
met ongekende lonen, maar zorgde ook voor een herverdeling van inkomens.

De welvaartsstaat ontstond als logische uitkomst van de technologische
ontwikkelingen in het industrialisme. Door hun enorme schaal en hoge
kapitaalkosten vormden de leidende industriële werkgevers de meest voor
de hand liggende doelwitten voor belastingen. Men ging ervan uit dat
deze ondernemingen een nauwkeurige administratie voerden en
looninhoudingen beheersten, waardoor de invoering van de
inkomstenbelasting technisch realiseerbaar werd -- iets wat in
voorgaande eeuwen, toen de economieën nog gedecentraliseerder waren,
niet mogelijk was. Uiteindelijk zorgde de door industriële innovatie
versnelde groei van schaalvoordelen ervoor dat overheden rijker werden
en daardoor vermoedelijk beter in staat waren de orde te handhaven.

\subsection{Het proces is omgekeerd}\label{het-proces-is-omgekeerd}

Naar onze mening gebeurt er tegenwoordig juist het tegenovergestelde.
Informatietechnologie vergroot de verdienkansen van geschoolden en
ondermijnt grootschalige instituties, waaronder de natiestaat.

Dit wijst op nog een ironie van ons informatietijdperk: de schijnbaar
tegenstrijdige en fundamenteel belemmerende houding van vrije
marktcritici ten aanzien van zowel de opkomst als het verdwijnen van
industriële banen. In de beginfase van het industrialisme waren deze
critici geraakt door wat zij zagen als het kwaad van fabrieksbanen, die
landloze boeren weg trokken uit `de wereld die we verloren hebben.' Zij
beschouwden de komst van fabrieksbanen als een ongekend kwaad en als
`uitbuiting' van de arbeidersklasse. Nu lijkt het echter zo dat het
echte probleem niet de komst, maar juist het verdwijnen van
fabrieksbanen is. Zelfs de achterkleinkinderen van degenen die ooit
mopperden over de introductie ervan, klagen nu over het tekort aan banen
die een hoog loon bieden voor laaggeschoold werk.

De rode draad in al deze klachten is een onverzettelijke weerstand tegen
technologische vernieuwing en marktveranderingen. In de beginfase van
het fabrieksysteem leidde die weerstand zelfs tot geweld. Dat zou
opnieuw kunnen gebeuren.

En dat niet omdat kapitalisten de arbeiders `uitbuiten'. De opkomst van
de computer als paradigma-technologie toonde immers de belachelijkheid
van die stelling. Voor de onoplettene zou het nog enigszins aannemelijk
kunnen lijken dat een nauwelijks geletterde automedewerker op de een of
andere manier `uitbuit' werd bij de assemblage van een auto door
eigenaren die de ondernemingen hadden bedacht en gefinancierd waarin
arbeiders in dienst waren. De cruciale rol van conceptueel kapitaal bij
de productie en marketing van fysieke producten kwam minder duidelijk
naar voren dan bij de output van het informatietijdperk, waarin mentaal
werk duidelijk de boventoon voerde. Hierdoor won de bewering dat
ondernemers op de een of andere wijze de waarde van informatieproducten
-- in werkelijkheid door werknemers gecreëerd -- toe-eigenden,
grotendeels aan betekenis in. Wanneer de waarde onmiskenbaar voortkwam
uit mentaal werk -- zoals bij de ontwikkeling van consumentensoftware --
wordt het ronduit belachelijk geacht te stellen dat dat werk niet door
de experts is verricht die het hebben bedacht. Sterker nog, de
duidelijke en toenemende verschuiving weg van ongeschoolde arbeid riep
juist de vraag op of ongeschoolde arbeiders nog wel een economische
bijdrage kunnen leveren.\footnote{William~Julius~Wilson, \emph{Wanneer
  werk verdwijnt: de wereld van de nieuwe stedelijke armen} (New York:
  \emph{Alfred A. Knopf}, 1996).}

Hierdoor verandert de basis waarop inkomens wordt herverdeeld: vroeger
ging men uit van uitbuiting -- wat inhield dat mensen met lage inkomens
toch productief konden zijn -- terwijl men nu uitgaat van discriminatie,
die dat niet veronderstelt. Men betoogde bovendien dat discriminatie
verklaarde waarom mensen met beperkte vaardigheden er niet in slaagden
om waardevollere capaciteiten te ontwikkelen.

Deze vorm van discriminatie diende eveneens als rechtvaardiging voor het
opleggen van suboptimale aanwervingscriteria en andere maatstaven die
zogenaamd `kansen' creëren -- of, om het preciezer te zeggen, voor het
herverdelen van inkomen ten gunste van achterblijvende groepen. In de
Verenigde Staten zorgden rasgebonden aanpassingen in prestatie- en
bekwaamheidstesten er bijvoorbeeld voor dat zwarte kandidaten hogere
scores behaalden dan witte en Aziatische sollicitanten, terwijl zij op
objectieve maten lager scoorden. Via deze en andere methoden
verplichtten overheden werkgevers om meer zwarte mensen en andere
officieel als `benadeeld' bestempelde groepen in dienst te nemen tegen
hoger loon; wie de regels negeerde, kreeg te maken met kostbare
rechtszaken en procedures met hoge strafschadevergoedingen.

Het aanwijzen van slachtoffers had nooit tot doel paranoïde
achtervolgingswanen te creëren binnen belangrijke subgroepen van de
industrie, noch om het verspreiden van contraproductieve waarden te
bevorderen. Men wilde de falende staat immers ontlasten van de fiscale
druk die samenhangt met inkomensherverdeling. Het ontstaan van
achtervolgingswanen bleek echter een ongelukkige bijwerking van dit
streven. Ironisch genoeg viel de opkomst van de bezorgdheid over
`discriminatie' samen met de beginfase van een technologische revolutie
die ervoor zal zorgen dat willekeurige discriminatie in de toekomst veel
minder een probleem vormt. Op het internet bekommert niemand zich erover
of de auteur van een nieuw softwareprogramma zwart, wit, man, vrouw,
homoseksueel of een vegetarische dwerg is.

Hoewel toekomstige vormen van discriminatie vermoedelijk minder
onderdrukkend zullen zijn, neemt dat de roep om `reparaties' -- om
diverse reële of ingebeelde onrechtvaardigheden te compenseren -- niet
per se weg. Iedere samenleving, ongeacht de objectieve omstandigheden,
levert immers altijd één of meer rechtvaardigingen voor
inkomensherverdeling op. Deze motiveringen lopen uiteen van subtiele tot
absurde argumenten, van het bijbelse gebod om uw naaste lief te hebben
als uzelf tot het aanroepen van zwarte magie. Hekserij, tovenarij en het
kwaad oog vormen de keerzijde van religieuze gevoelens, het spirituele
equivalent van de Belastingdienst of de \emph{IRS}. Als mensen niet uit
liefde de armen steunen, zorgen de armen er via angst zelf voor dat ze
in actie komen. Soms uit zich dat in regelrechte afpersing -- een mes
aan de keel, een pistool aan het hoofd --, terwijl op andere momenten de
dreiging zich in vermomde of vergezochte gedaante toont. Het valt
daarbij niet te merken dat in de vroegmoderne periode de meeste `heksen'
weduwen of ongetrouwde vrouwen met beperkte middelen waren. Zij
terroriseerden hun buren met vloeken, waardoor die buren er niet zelden
toe overgingen te betalen. Je kunt dan ook niet zomaar aannemen dat zij
puur bijgelovig waren, want de kwaadaardige intentie achter het kwaad
oog bleek reëel. Zelfs een arme vrouw kon vee loslaten of iemands huis
in brand steken. In dat opzicht waren de heksenprocessen van weleer niet
zo belachelijk als ze op het eerste gezicht lijken, hoe wreed de
straffen ook waren en hoeveel onschuldigen er ten prooi vielen aan de
waanbeelden die ontstonden door schimmelvergiftiging.

Wij voorzien dat afpersing weer oplaait, gedreven door de wens mee te
profiteren van de vruchten van prestaties, naarmate het
informatietijdperk zich verder ontwikkelt. Groepen die zich benadeeld
voelen door vroegere discriminatie laten hun schijnbaar waardevolle
status als slachtoffer niet zomaar los, simpelweg omdat hun aanspraken
op de samenleving minder onderbouwd of moeilijker af te dwingen blijken.
Zij blijven hun rechten handhaven totdat het bewijs in hun directe
omgeving geen enkele twijfel meer toelaat dat zij niet de beloningen
ontvangen die hen toekomen.

De toename van sociopathisch gedrag onder Afro-Amerikanen en
Afro-Canadezen bevestigt dit punt. Het suggereert dat er nauwelijks
balans bestaat tussen de zwarte woede en een realistische inschatting
van in hoeverre de problemen binnen de zwarte gemeenschap het gevolg
zijn van antisociaal gedrag. De woede is gegroeid, terwijl de manier van
leven steeds disfunctioneler wordt. Het aantal kinderen dat buiten het
huwelijk wordt geboren, is explosief toegenomen, het opleidingsniveau
daalt en een groeiend percentage jonge zwarten raakt betrokken bij
criminele activiteiten, waardoor er nu meer zwarte mannen in
penitentiaires zitten dan op universiteiten.

Deze perverse ontwikkelingen hadden mogelijk tijdelijk tot gevolg dat er
tijdens de schemering van het industrialisme extra middelen naar de
onderklassegemeenschappen stroomden, doordat de druk die op de
samenleving werd uitgeoefend, toenam. Dit effect blijkt echter slechts
van korte duur te zijn. Doordat de positieve werking van concurrentie --
die mensen die onderpresteren aanspoort zich aan productieve normen te
houden -- wegvalt, heeft de verzorgingsstaat bijgedragen aan de opkomst
van legioenen disfunctionele, paranoïde en slecht aangepaste mensen, het
sociale equivalent van een kruitvat. De ondergang van de natiestaat en
het verdwijnen van grootschalige inkomensherverdelingen leiden er
ongetwijfeld toe dat enkele van deze psychopatische zielen zich keren
tegen iedereen die er welvarender uitziet. Daarom mogen we ervan uitgaan
dat de sociale vrede onder druk komt te staan naarmate het
informatietijdperk zich verder ontvouwt, met name in Noord-Amerika en in
multietnische enclaves in West‑Europa.

\begin{quote}
``Wij zullen onze wapens nooit neerleggen {[}totdat{]} het Lagerhuis een
wet aanneemt om alle machinewerktuigen die schadelijk zijn voor de
gemeenschap neer te leggen, en die wet niet teniet te maken om
framebrekers op te hangen. Maar wij. Wij doen geen smeekbedes meer ---
dat schikt niet --- er moet worden gevochten.''\\
Ondertekend door de generaal van het leger der rechtzetters, Ned
Ludd-klerk, ``Rechtzetter-vooraltijd, amen'' \footnote{Tilly,
  `Collectief geweld', p.~78.}
\end{quote}

Neo-Luddiet

Gezien de antitechnologische opstanden in het begin van de negentiende
eeuw en de lange traditie van collectief geweld in zowel Europa als
Noord‑Amerika, valt het niet te verwonderen als we tegenwoordig een
neo-luddietische aanval op de informatietechnologie en haar gebruikers
zien. De Luddites, zoals eerder genoemd, waren textielarbeiders die zich
in West Yorkshire, Engeland, hadden verzameld en in 1811--1812 een
terroristische campagne voerden tegen geautomatiseerde snoeimachines en
tegen de fabriekseigenaren die deze in gebruik namen.\footnote{Robert~Reid,
  \emph{Land van verloren inhoud: de ludditenopstand 1812} (London:
  \emph{Penguin}, 1986), p.~44.} Met zwartgeblakste gezichten raasden de
Luddites door West Yorkshire, staken fabrieken in brand en vermoordden
fabriekseigenaren die de nieuwe technologie durfden te omarmen.

Het grootste deel van het geweld kwam van de zogenaamde `croppers',
hooggekwalificeerde ambachtslieden wier meesterlijke beheersing van
enorme scharen -- die tot wel vijftig pond wogen -- ooit een cruciaal
onderdeel was van de productie van wollen stoffen.

Het afwerkingsproces dat de croppers toepasten -- het opwekken van
stofpluis met distels en het snoeien van de stof met scharen -- was
volgens Robert Reid, de auteur van `Land of Lost Content: The Luddite
Revolt 1812', `te eenvoudig om niet gemachinaliseerd te
worden.'\footnote{Ibid., p.~45.}

Leonardo da Vinci had al een ontwerp voor zo'n mechanische snoeimachine
uitgetekend, maar zijn idee voor automatische snoei bleef eeuwen
onbenut. Uiteindelijk heruitvonden ingenieurs in 1787 een apparaat naar
het model van Leonardo's ontwerp, en brachten zij het in Engeland in
productie.

Reid merkt daarbij op: `Alle bouwstenen van de technologie waren al
geruime tijd bekend, waardoor het opmerkelijk is dat men deze niet
eerder had doorgevoerd\ldots{} De nieuwe machines uit de Industriële
Revolutie vroegen zo weinig kracht en vaardigheid dat veel vacatures
werden ingevuld door vrouwen en jonge kinderen, aanvankelijk tegen lage
lonen. Één van deze apparaten kon -- zelfs bediend door relatief
ongeschoolden -- in achttien uur verrichten wat een ervaren cropper met
handscharen in achtentachtig uur voor elkaar kreeg.'\footnote{Ibid.,
  p.~26.}

Opmerkelijk is dat werknemers die fel tegen mechanisatie opstandig
werden, zich selectief verzetten tegen technologieën die hun eigen banen
wegvaarden of de vraag naar vakbekwame arbeid deden afnemen.

Toen ondernemer William Cooke tapijtweefmachines invoerde in West
Yorkshire, brak er geenszins geweld uit. Men trachtte zijn molen niet in
brand te steken, zijn machines niet te vernielen en zeker niet hem --
laat staan zijn leven -- te ruïneren. Volgens Reid, die de
Luddite-opstanden uitvoerig beschrijft, riep Cooke's nieuwe technologie
geen verzet op, omdat niemand in het dal tot dan toe gespecialiseerd was
in tapijtproductie. Hij vervolgt: `Doordat Cooke een nieuw product
introduceerde en banen creëerde die niet op traditionele methoden waren
gebaseerd, kende zijn molen een ware bloeiperiode.'

Dit voorbeeld heeft grote implicaties voor de toekomst. Het duidt erop
dat vooruitziende ondernemers in het komende millennium ingrijpende,
arbeidsefficiënte automatisering zullen introduceren in regio's waar men
nog geen traditie had in de productie van het betreffende product of de
dienst.

Als de geschiedenis ons iets leert, dan zullen de meest radicale
terroristen in de eerste jaren van het nieuwe millennium waarschijnlijk
niet uit dakloze verarmden bestaan, maar uit werknemers die ooit tot de
middenklasse behoorden en aanzien genoten, maar nu hun baan kwijt zijn
geraakt. Dat bleek immers tijdens de Luddite-opstand van 1812, waar het
merendeel van de betrokken Luddites geen verarmd proletariaat vormde,
maar hoogopgeleide ambachtslieden waren die gewend waren een inkomen te
verdienen dat ten minste vijfmaal hoger lag dan dat van een gemiddelde
arbeider. De hedendaagse variant zou vermoedelijk bestaan uit verdrongen
fabrieksarbeiders. Helaas toont een blik op de demografie van de meeste
OESO-landen dat er op meerdere plaatsen potentiële knelpunten liggen
voor gewelddadige reacties.

Wereldwijd zullen staten proberen de opkomst van de cybereconomie én die
onafhankelijke individuen tegen te gaan die hiervan profiteren om
rijkdom te vergaren. Deze pogingen roepen een woedende nationalistische
reactie op, waarin onvermijdelijk ook een antitechnologische stroom
meespeelt -- vergelijkbaar met de Luddieten- en andere
antitechnologie-opstanden in Groot-Brittannië tijdens de Industriële
Revolutie. Dit scenario vereist nauwgezette overweging, want het zou wel
eens de sleutel kunnen vormen tot de toekomstige evolutie van bestuur in
het nieuwe millennium. Een van de grootste uitdagingen van de aanstaande
transformatie zal het handhaven van de orde zijn te midden van
escalerend geweld, of het voorkomen van een rampzalige uitbarsting.
Personen en ondernemingen die nauw verbonden zijn met het ontstaan van
het informatietijdperk -- waaronder de spelers in \emph{Silicon Valley}
en zelfs de leveranciers van de elektriciteit die de nieuwe technologie
aandrijft -- moeten extra waakzaam zijn voor freelance, neo-luddietisch
terrorisme.

Een krankzinnige zoals de Unabomber zal helaas waarschijnlijk brigades
van na-ijmpers aanwakkeren, nu de frustratie over dalende inkomens en de
toegenomen wrok tegen prestaties de kop opsteken. Wij vermoeden dat een
groot deel van het toekomstige geweld uit bomaanvallen zal bestaan.
Volgens de \emph{New York Times} is het binnenlands terrorisme in de
Verenigde Staten in de jaren negentig fors toegenomen. `Ze zijn de
afgelopen vijf jaar met meer dan vijftig procent gegroeid en in het
afgelopen decennium bijna verdrievoudigd. Het aantal criminele explosies
en pogingen steeg van 1.103 in 1985 tot 3.163 in 1994\ldots. In kleine
steden en voorstedelijke buurten, evenals onder stedelijke straatbendes,
zien we een toename van wat op een alledaagse bommenwerper lijkt.'
\footnote{Timothy Egan, `Terrorisme wordt nu huisgemaakt naarmate
  bomaanslagen in de V.S. toenemen,' \emph{New York Times}, 25 augustus
  1996, p.~1.}

\subsection{Defensie wordt een privaat
goed}\label{defensie-wordt-een-privaat-goed}

Hoe hoog de belastingen voor bescherming ook zijn, natiestaten zullen in
de komende jaren waarschijnlijk niet in staat zijn om deze op effectieve
wijze te leveren. De afnemende schaal van geweld -- mede beïnvloed door
de nieuwe informatietechnologie -- maakt het opbouwen van een massief
militair apparaat steeds minder zinvol. Dit betekent niet alleen dat de
oorlogvoering minder beslissend wordt en staten hun burgers minder goed
kunnen beschermen, maar ook dat de schijnbare extraterritoriale
hegemonie van de Verenigde Staten als 's werelds supermacht in de
volgende eeuw waarschijnlijk minder effectief zal blijken dan de
hegemonie van Groot-Brittannië in de negentiende eeuw. Tot het uitbreken
van de Eerste Wereldoorlog oefende het centrum tegen relatief lage
kosten effectief en doorslaggevend zijn macht uit op de periferie. In de
eenentwintigste eeuw nemen de bedreigingen van grootmachten voor het
leven en eigendom onvermijdelijk af, nu geweld steeds vaker in kleinere,
minder grootschalige vormen voorkomt. De dalende opbrengsten van geweld
wijzen erop dat natiestaten of rijken die grootschalig militaire macht
inzetten, mogelijk niet zullen overleven of zelfs niet zullen ontstaan
in het informatietijdperk.

Nu de overheidsuitgaven voor een adequate verdediging dalen, wordt het
steeds aannemelijker om beschermingsdiensten als private goederen te
beschouwen. Beveiligingsbedreigingen op kleinere schaal bestrijd je
immers steeds beter met commerciële beveiligingstroepen -- denk aan
muren, hekken en beveiligingsperimeters om onruststokers buiten de boot
te houden. Bovendien kan een welvarend individu of een onderneming zich
doorgaans zelf de nodige bescherming veroorloven tegen de meeste
bedreigingen die in het informatietijdperk kunnen ontstaan. Daarnaast
vergroot de afnemende schaal van militaire bedreigingen het risico op
anarchie of op interne gewelddadige conflicten binnen een enkel
grondgebied. Tegelijkertijd neemt de concurrentie tussen rechtsgebieden
op het gebied van bescherming onder marktconforme voorwaarden toe. Dit
houdt in dat rechtsgebieden steeds intensiever zullen concurreren op het
terrein van beschermingsdiensten, paspoort- en consulaire diensten en de
uitvoering van justitie.

Op de lange termijn kunnen soevereine individuen waarschijnlijk reizen
met niet-gouvernementele documenten, uitgegeven door private
agentschappen en affiniteitsgroepen in de vorm van kredietbrieven. Het
is niet vergezocht te denken dat zich een groep zal vormen, een soort
koopmansrepubliek in cyberspace à la middeleeuwse Hanze, die het sluiten
van private verdragen en contracten tussen rechtsgebieden
vergemakkelijkt en tegelijkertijd haar leden bescherming biedt. Stel je
een speciaal paspoort voor, uitgegeven door de Liga van Soevereine
Individuen, dat de drager erkent als iemand die onder de bescherming van
de liga valt.

Als zo'n document ooit tot stand komt, blijft het slechts een tijdelijk
aandenken aan de overgang van de natiestaat en het bureaucratische
tijdperk dat deze ooit in stand hield.

In vroegere tijden had men doorgaans geen paspoort nodig om de vaak vaag
afgebakende grenzen over te steken. In middeleeuwse grensgebieden
hanteerde men soms veiligheidsbrieven, maar de lokale autoriteiten gaven
deze immers af voor het gebied dat bezocht moest worden, in plaats van
vanuit de jurisdictie waar de reiziger vandaan kwam. Veel belangrijker
dan een paspoort waren introductie- en kredietbrieven, die een reiziger
hielpen onderdak te vinden en zakelijke deals te sluiten. Die tijd komt
weer, want binnenkort zullen prominente personen zonder enige documenten
reizen. Ze identificeren zich dan via waterdichte biometrische methoden,
zoals spraakherkenningssystemen of netvliesscanners die hen uniek
herkennen.

Wij verwachten dat de wereld ergens in de eerste helft van de volgende
eeuw daadwerkelijk de privatisering van soevereiniteit zal meemaken. Dit
zal ervoor zorgen dat het bereik van dwang tot het logische minimum
krimpt. Toch zullen de seculiere inquisiteurs en reactionairen van het
volgende millennium in verontwaardiging en met een gevoel van bedreiging
reageren wanneer de eens `heilige' kenmerken van nationaliteit
verhandelbaar worden en als een kwestie van kosten-batenberekening
kunnen worden gekocht en verkocht.

In dit boek stellen wij dat een natiestaat niet langer nodig is om een
informatieoorlog te voeren. Computerprogrammeurs kunnen zulke oorlogen
uitvechten door talloze `bots' of digitale bedienden in te zetten. Bill
Gates beschikt inmiddels over een veel grotere capaciteit dan de meeste
natiestaten om logische bommen in kwetsbare systemen wereldwijd te laten
ontploffen. In het tijdperk van de informatieoorlog vormt iedere
softwaremaatschappij -- of zelfs de \emph{Scientology}kerk -- een
geduchtere tegenstander dan de gecombineerde dreiging van de meeste
staten met een zetel in de Verenigde Naties.

Het machtverlies van natiestaten is een logische consequentie van de
opkomst van goedkope, geavanceerde rekenkracht.
Microprocessortechnologie vermindert niet alleen de baten van geweld,
maar zorgt er ook voor dat er voor het eerst een concurrerende markt
ontstaat voor beschermingsdiensten, waarvoor overheden in het
industriële tijdperk monopolistische prijzen rekenden.

In de nieuwe wereld waarin soevereiniteit commercieel wordt
georganiseerd, kiezen mensen hun rechtsgebieden, net zoals velen
tegenwoordig hun verzekeringsmaatschappijen of religies
selecteren.\footnote{Zie Stephen J. Duhner, \emph{Choosing My Religion},
  \emph{New York Times Magazine}, 31 maart 1996, p.~36 ff.}
Rechtsgebieden die er niet in slagen een passende mix van diensten te
leveren -- welke diensten dat ook mogen zijn -- zullen failliet gaan en
geliquideerd worden, zoals incompetente bedrijven of mislukte religieuze
gemeenschappen dat ook doen. Concurrentie zet lokale rechtsgebieden
ertoe aan hun vermogen om diensten efficiënt en economisch te leveren te
verbeteren. Op die manier levert de concurrentie tussen rechtsgebieden
bij het aanreiken van publieke goederen een effect op dat vergelijkbaar
is met wat we in andere levenssferen zien, want concurrentie verhoogt
doorgaans de klanttevredenheid.

\section{Concurrentie en anarchie}\label{concurrentie-en-anarchie}

Belangrijk is te beseffen dat de rivaliteit tussen de jurisdicties die
wij voor ogen hebben niet primair draait om organisaties die binnen
hetzelfde grondgebied met geweld opereren. Zoals eerder is opgemerkt,
hebben gewelddragende organisaties vaak de neiging om geweld verder de
dagelijkse sfeer in te laten sijpelen, waardoor de economische kansen
afnemen. Zoals Lane het verwoordde,

\begin{quote}
Bij het gebruik van geweld waren er uiteraard grote schaalvoordelen bij
het concurreren met rivaliserende gewelddragende ondernemingen of bij
het vestigen van een territoriaal monopolie. Dit gegeven vormt de basis
voor de economische analyse van een aspect van de overheid: de
gewelddragende, geweldbeheersende industrie was een natuurlijk
monopolie, althans op land. Binnen territoriale grenzen kon de geleverde
dienst veel goedkoper door een monopolie worden geproduceerd. Natuurlijk
zijn er momenten geweest waarop gewelddragende ondernemingen wedijverden
om betalingen voor bescherming af te dwingen in vrijwel hetzelfde
grondgebied, bijvoorbeeld tijdens de Dertigjarige Oorlog in Duitsland.
Maar zo'n situatie was nog oneconomischer dan concurrentie in dezelfde
gebieden tussen rivaliserende telefoonsystemen.\footnote{Lane,
  `Economische gevolgen van georganiseerd geweld', p.~402.}
\end{quote}

Lanes opmerking levert op twee vlakken waardevolle inzichten. Ten eerste
onderschrijven we zijn algemene conclusie dat soevereiniteiten de
neiging hebben territoriale monopolieën te vestigen, zodat zij
goedkopere en effectievere beschermingsdiensten kunnen leveren. Ten
tweede valt het op dat hij een inmiddels verouderde vergelijking maakt
met het telefoonmonopolie, terwijl we inmiddels weten dat
telefoonsystemen niet per se een monopolie hoeven te vormen. Deze
constatering vraagt om enige voorzichtigheid in de analyse.
Technologische veranderingen kunnen de algemene conclusie ondermijnen
dat anarchie binnen territoriale grenzen onhoudbaar is. Stel
bijvoorbeeld dat cyberactiva massaal groeien in een domein dat buiten
het bereik van dwang valt; dan zal de prijsbepaling van
beschermingsdiensten minder door de `vraag' worden bepaald en meer door
onderhandelingsprocessen op de markt.

We hebben het hier niet over alomtegenwoordige anarchie, maar over de
concurrentie tussen jurisdicties die elk een monopolie op geweld binnen
hun eigen grondgebied hebben. Zulke jurisdicties wedijveren er immers
hard voor hun `klanten' de meest waardevolle en kosteneffectieve
beschermingsdiensten te leveren. Toegegeven, in het informatietijdperk
zullen er ongetwijfeld meer onzekerheden ontstaan rondom de levering van
deze diensten en zal de particuliere voorziening van politie- en
defensiediensten uitgebreider worden dan we vroeger gewend waren. Maar
de concurrentie waar wij over spreken, verschilt wezenlijk van een
strijd waarbij meerdere beschermingsinstanties op hetzelfde grondgebied
vechten om verschillende klanten te bedienen -- want dat zou pure
anarchie betekenen.

Hoe het ook zij, de groei van soevereiniteiten -- waarbij individuen
steeds vaker als soevereinen optreden zodra zij over voldoende middelen
beschikken -- zorgt er onvermijdelijk voor dat de ruimte voor anarchie
zal toenemen. De relaties tussen soevereiniteiten verlopen namelijk
altijd anarchaïsch. Er bestaat, en heeft nooit bestaan, een
wereldregering die het gedrag van alle individuele soevereiniteiten
reguleert, of het nu gaat om ministaten, natiestaten of rijken. Zoals
Jack Hirshleifer opmerkt: ``{[}W{]}aarin samenlevingen, variërend van
primitieve stammen tot moderne natiestaten, intern door een vorm van wet
worden bestuurd, blijven hun externe betrekkingen met elkaar
voornamelijk anarchaïsch.''\footnote{Jack Hirshleifer, `Anarchie en de
  ineenstorting', in Michelle R. Garfinkel en Stergios Skaperdas, red.,
  \emph{De politieke economie van conflict en toe-eigening} (Cambridge:
  \emph{Cambridge University Press}, 1996), p.~15.} Met meer soevereine
entiteiten ontstaan er daarom ook onvermijdelijk relaties die zich over
meerdere rechtsgebieden uitstrekken en dus anarchaïsch zijn.

Belangrijk is op te merken dat anarchie -- met andere woorden, het
ontbreken van een overkoepelende macht om geschillen te beslechten --
niet gelijkstaat aan totale chaos of aan het gebrek aan structuur.
Hirshleifer stelt dat zelfs interstam- of internationale systemen
regelmatigheden en systematisch analyseerbare patronen
kennen.\footnote{Ibid., p.~15.} Met andere woorden: net zoals `chaos' in
de wiskunde kan wijzen op een complexe en zeer geordende structuur, is
ook anarchie niet per definitie vormeloos of wanordelijk.

Hirshleifer bestudeert diverse anarchaïsche situaties. Hij bespreekt
daarbij niet alleen de relaties tussen soevereiniteiten, maar ook
bendeoorlogvoering in het Chicago van de drooglegging en de
confrontaties tussen mijnwerkers en claimjumpers tijdens de
Californische goudkoorts. Let op: hoewel Californië in 1849 al deel
uitmaakte van de Verenigde Staten, werden de omstandigheden in de
goudvelden terecht als anarchistisch beschreven. Zoals Hirshleifer
opmerkt: ``{[}D{]}e officiële organen van de wet bleken
machteloos.''\footnote{Ibid., p.~34.} Hij betoogt dat de barre
omstandigheden in de bergkampen, gecombineerd met de effectieve
organisatie van mijnwerkers die zichzelf als rechtshandhavers hadden
aangesteld tegen claimjumpers, het voor bendes van buitenstaanders
vrijwel onmogelijk maakten om de goudmijnen te veroveren -- ondanks het
gebrek aan een degelijke wetshandhaving. Met andere woorden, onder
bepaalde omstandigheden kan waardevol eigendom zelfs in een
anarchistische situatie goed beschermd worden.

De vraag rijst of Hirshleifer's theoretische analyse van de dynamiek
binnen de spontane orde van de darwinistische `natuurlijke economie' ook
relevant is voor de economie in het informatietijdperk. Wij denken van
wel. Hoewel we geen alomtegenwoordige anarchie -- of goudveldachtige
omstandigheden -- overal verwachten, zien we zeker een toename van
anarchaïsche relaties in het wereldsysteem. In dat licht is
Hirshleifer's betoog over de condities waaronder `twee of meer
anarchaïsche deelnemers' in evenwicht hun leefbare aandeel van de
maatschappelijk beschikbare middelen weten te behouden bijzonder
suggestief.\footnote{Ibid., p.~17.} Vooral onderzoekt hij wanneer
anarchie overgaat in tirannie of dominantiehiërarchieën, wat gebeurt
wanneer anarchaïsche partijen ondergeschikt raken aan een overweldigende
autoriteit.

Deze kwesties blijken in het informatietijdperk vaak nog cruciaalere
kennisgebieden dan in het industriële tijdperk. Een belangrijke reden
dat de fijne nuances in de dynamiek van anarchie in voorgaande eeuwen
minder doorslaggevend leken, is dat het rendement op geweld in de
moderne tijd is toegenomen. Daardoor leidde het samenbrengen van steeds
grotere militaire machten -- zoals natiestaten dat in recente eeuwen
deden -- tot beslissende oorlogvoering. Bijna per definitie zorgen zulke
oorlogsvoering ervoor dat strijdende partijen in hun strijd om
hulpbronnen onder de heerschappij van een sterkere autoriteit komen te
staan, wat de anarchie tempert. Tegelijkertijd bevordert een afnemende
beslissendheid in gevechten -- mede dankzij de technologische
superioriteit van de verdediging -- de dynamisch stabiele positie van
anarchie. Hierom zorgt informatietechnologie, die de beslissendheid van
militaire acties verlaagt, ervoor dat de anarchie tussen ministaten
stabieler blijft en dat een grote overheid deze situatie minder
gemakkelijk kan overnemen. Bovendien betekent een afname van de
beslissendheid ook minder gevechten, wat een hoopvolle uitkomst biedt
voor de wereld van vandaag.

\subsection{Levensvatbaarheid}\label{levensvatbaarheid}

Een andere belangrijke voorwaarde voor het voortbestaan van anarchie is
levensvatbaarheid, oftewel de mate waarin mensen beschikken over een
voldoende inkomen. Degenen die niet genoeg verdienen om in hun
levensonderhoud te voorzien, zullen waarschijnlijk óf (1) enorme
inspanningen leveren om te strijden en zo de benodigde middelen voor hun
overleving te bemachtigen, óf (2) zich onderwerpen aan een andere
uitdager in ruil voor voedsel en levensonderhoud. Een dergelijk fenomeen
deed zich voor tijdens de opkomst van het feodalisme rond het jaar 1000.
We verwachten dat steeds meer mensen met een laag inkomen in westerse
landen -- die voorheen afhankelijk waren van overheidsuitkeringen --
zich als hofdienaren bij welvarende huishoudens zullen voegen. Toch
geeft het feit dat sommige uitdager in een anarchistische samenleving
een laag inkomen hebben op zichzelf nog geen duidelijk signaal over hoe
de situatie zich verder zal ontwikkelen. Zoals Hirshleifer stelt:
`{[}H{]}et loutere feit van een laag inkomen onder anarchie \ldots{}
geeft op zichzelf geen duidelijk signaal als toespraak op wat er hierna
zal gebeuren.'\footnote{Ibid., p.~37.}

\subsection{De aard van activa}\label{de-aard-van-activa}

Een andere interessante voorwaarde voor de duurzaamheid van anarchie is
dat de middelen voorspelbaar en verdedigbaar zijn. In zijn analyse merkt
Hirshleifer op: `{[}A{]}narchie is een sociale ordening waarin uitdager
strijden om duurzame middelen te veroveren en te verdedigen.'\footnote{Ibid.,
  p.~16.} Hij definieert `duurzame middelen' als landgebieden of
verplaatsbare kapitaalgoederen.\footnote{Ibid.} In het
informatietijdperk kunnen digitale middelen voorspelbaar blijken, maar
ze vallen niet in de categorie `duurzame middelen' die Hirshleifer
koppelt aan territorialiteit en anarchie. Immers, wanneer digitaal geld
vrijwel onmiddellijk over de hele wereld wordt verstuurd, is het
veroveren van het grondgebied waar een cyberbank opereert zinloos. Als
natiestaten soevereine individuen willen onderdrukken, moeten zij
tegelijkertijd zowel de bankparadijzen als de dataparadijzen van de
wereld veroveren. Zelfs dan zijn zij -- mits de versleutelde systemen
naar behoren zijn ontworpen -- slechts in staat om een beperkte
hoeveelheid digitaal geld te saboteren of te vernietigen; het
daadwerkelijk in bezit krijgen blijft buiten hun bereik.

De conclusie komt erop neer dat in het komende informatietijdperk de
meest voorspelbare en kwetsbare bezittingen van de rijken wellicht hun
eigen lichamen -- met andere woorden, hun levens -- zijn. Daarom
verwachten we de komende decennia terreur in luddietenstijl, waarbij een
deel mogelijk heimelijk door agents provocateurs in dienst van
natiestaten wordt aangewakkerd.

Op lange termijn betwijfelen we of toonaangevende natiestaten soevereine
individuen daadwerkelijk kunnen onderdrukken. Vooral staten in
kapitaalarme regio's zullen inzien dat zij meer baat hebben bij het
huisvesten van soevereine individuen dan door vast te houden aan
solidariteit met Noord-Atlantische natiestaten en het in ere houden van
het `internationale' systeem. Dat failliete, zwaar belaste
welvaartsstaten erop aandringen om `hun burgers' en `hun kapitaal'
binnen eigen landsgrenzen te houden, biedt elders namelijk geen
overtuigende stimulans voor de honderden gefragmenteerde
soevereiniteiten.

Dit stellen we, ook al bestaan er duizenden multinationale organisaties
die proberen het gedrag van diverse soevereiniteiten wereldwijd te
beïnvloeden. We twijfelen er nauwelijks aan dat sommige van deze
organisaties, zoals de Europese Unie en de Wereldbank, grote invloed
uitoefenen. Maar bedenk dat rechtsgebieden die soevereine individuen
verwelkomen, aanzienlijk profiteren van hun aanwezigheid. Zelfs de
koppige Verenigde Staten, die volgens de huidige trends gedwongen worden
hard op te treden om te voorkomen dat een cybereconomie buiten hun
controle ontstaat, zullen uiteindelijk niet toestaan dat mensen met
positieve banksaldi -- mensen die bewust geen Amerikanen willen zijn --
wereldwijd buitengesloten raken. Dit wordt des te waarschijnlijker nu
winkelen tegenwoordig een grote aantrekkingskracht op reizigers
uitoefent. Uiteindelijk nemen ook de Verenigde Staten -- zij het later
dan anderen -- of delen daarvan deel aan de commercialisering van
soevereiniteit als gevolg van toenemende concurrentiedruk.

\subsection{Vraag creëert aanbod}\label{vraag-creuxebert-aanbod}

Die druk zal in een vroeg stadium vooral voelbaar zijn in natiestaten
met zwakke balansen. Binnen de nieuwe `offshore'-centra zullen
fragmenten en enclaves van de huidige natiestaten opduiken --
bijvoorbeeld Canada en Italië -- die vrijwel zeker al ruim voor het
einde van het eerste kwartaal van de eenentwintigste eeuw uiteen vallen.
Het ontstaan van een mondiale markt voor hoogwaardige en
kostenefficiënte rechtsgebieden speelt daarbij een belangrijke rol. Net
als in de reguliere handel kunnen kleinschalige concurrenten leniger
opereren en daardoor beter concurreren. Een rechtsgebied met weinig
inwoners organiseert zich immers eenvoudiger zodat het efficiënt
functioneert.

De informatie-elite zoekt op contractbasis hoogwaardige bescherming
tegen een redelijke vergoeding. Hoewel die vergoeding ruim onder het
niveau blijft dat nodig is om een merkbaar voordeel over de gehele
bevolking van natiestaten -- zoals zij nu functioneren, met tientallen
tot honderden miljoenen burgers -- te verspreiden, is dit in
rechtsgebieden met tienduizenden of honderdduizenden inwoners zeker niet
onbeduidend. De belastinginkomsten en andere economische voordelen, die
voortvloeien uit de aanwezigheid van een klein aantal buitengewoon rijke
individuen, leveren per hoofd van de bevolking in rechtsgebieden met een
kleine populatie een veel groter voordeel op.

Aangezien het in de praktijk vrijwel onbelangrijk wordt waar men zijn
onderneming vestigt -- afgezien van het negatieve feit dat bepaalde
adressen hogere verplichtingen met zich meebrengen -- kunnen kleinere
rechtsgebieden gemakkelijker commercieel aantrekkelijke
beschermingsvoorwaarden vaststellen. Daarom genieten rechtsgebieden met
een kleine bevolking een duidelijk voordeel bij het opstellen van een
fiscaal beleid dat soevereine individuen aanspreekt.

Wij zijn ervan overtuigd dat het tijdperk van de natiestaat voorbij is,
maar dat betekent niet dat de aantrekkingskracht van nationalisme, als
beroep op menselijke emoties, meteen zal verdwijnen. Nationalisme speelt
als ideologie uitstekend in op universele emotionele behoeften. We
hebben allemaal wel eens ontzag ervaren, bijvoorbeeld bij het voor het
eerst aanschouwen van een enorme waterval of bij het binnenstappen in
een indrukwekkende kathedraal. Eveneens hebben we het gevoel van
saamhorigheid ervaren, zoals tijdens een familiekerstfeest of als lid
van een succesvol sportteam. Onze cultuur roept reacties op beide
krachtige emoties op. We laten ons inspireren door de historische
tradities van ons eigen land, dat weer deel uitmaakt van de bredere
menselijke cultuur. We vinden troost in het besef dat we tot een
culturele gemeenschap behoren, die ons zowel een gevoel van
verbondenheid als van identiteit geeft.

De invloed van deze culturele symbolen kan het meest ingrijpende
emotionele effect hebben. Amerikaanse associaties met de vlag, het
volkslied of het familiediner op Thanksgiving, en Engelse associaties
met de monarchie of met cricket, grijpen respectievelijk Amerikanen en
Engelsen stevig in de verbeelding -- een greep die door herhaling steeds
versterkt wordt en diep in het onderbewustzijn doordringt. Zulke
symbolen helpen ons te begrijpen wat voor soort mensen we zijn en doen
ons herinneren aan een nationale cultuur. Toen
anti-Vietnamoorlogdemonstranten de rest van de Verenigde Staten wilden
choqueren, verbrandden zij de vlag. Vervreemden Engelsen vallen de
monarchie aan en er gaan zelfs verhalen rond dat zij gaten in
cricketbanen graven.

Deze prikkelende symbolen lijken misschien oppervlakkig, maar ze zijn
zeker niet onbelangrijk. Het zijn associaties waardoor ons bloed sneller
gaat stromen. Ongeacht de veranderingen in megapolitieke omstandigheden
of de daaruit voortvloeiende aanpassingen in instituties, zullen zij
waarschijnlijk een blijvende rol spelen in de verbeelding van mensen die
-- net als wij -- in de twintigste eeuw volwassen werden.

De inhoud van het te vertalen boek is:

\bookmarksetup{startatroot}

\chapter{De schemering van de
democratie}\label{de-schemering-van-de-democratie}

\begin{quote}
`Democratische politieke systemen vormen in historische termen een
recent fenomeen. Ze bestonden kort in Griekenland en Rome en keken
minder dan 200 jaar geleden in de 18e eeuw weer terug. Een cyclus van
afwijzing lijkt nu opnieuw aan te breken.'\footnote{John Dunn,
  \emph{Western Political Theory in the Face of the Future} (Cambridge,
  Eng.: Cambridge University Press, 1979, p.~2).} - William Pfaff
\end{quote}

Het valt niet te ontkennen dat democratie als bestuursvorm in de loop
der tijd vrij zeldzaam en van korte duur is geweest. In de perioden --
zowel in de oudheid als in modernere tijden -- waarin democratie
dominant was, bepaalde het succes ervan in sterke mate de aanwezigheid
van grootschalige omstandigheden die zowel de militaire macht als het
draagvlak van de massa versterkten. Historicus Carroll Quigley
onderzocht deze bepalende kenmerken in \emph{Weapons Systems and
Political Stability}.\footnote{Carroll Quigley, \emph{Weapons Systems
  and Political Stability} (Washington, D.C.: University Press of
  America, 1983).}

Hiervoor behoren:

\begin{enumerate}
\def\labelenumi{\arabic{enumi}.}
\item
  \textbf{Betaalbare en breed beschikbare wapens}. Democratieën bloeien
  doorgaans wanneer het aanschaffen van doeltreffende wapens weinig
  kost.
\item
  \textbf{Wapens die effectief door amateurs ingezet kunnen worden}. De
  kans op democratie neemt toe wanneer iedereen zonder langdurige
  training effectieve wapens kan hanteren.
\item
  \textbf{Een militair voordeel voor een groot aantal infanteristen in
  de strijd}. Zoals Quigley opmerkt: `{[}P{]}erioden waarin de
  infanterie de overhand had, gingen gepaard met een bredere spreiding
  van politieke macht binnen de gemeenschap, waardoor de democratie
  betere kansen had om te zegevieren.'\footnote{Ibid., p.~56.}
\end{enumerate}

Dit vormt geenszins een volledige opsomming van alle voorwaarden
waaronder democratie kan bestaan. Als dat wel zo was, had de democratie
aan het einde van de twintigste eeuw niet als een triomfantelijk systeem
doorgebroken. In de schemering van het industriële tijdperk waren wapens
vermoedelijk duurder dan ooit, en veel van de meest krachtige wapens
vereisten ongetwijfeld specialisten om hun potentie daadwerkelijk te
benutten. Bovendien toont de Golfoorlog tussen de Verenigde Staten, hun
bondgenoten en Irak aan hoe kwetsbaar grote contingenten infanterie
zijn, zelfs wanneer zij zich in loopgraven en diep ingegraven
verdedigingswerken bevinden. Dus waarom lijkt de democratie juist te
floreren in deze omstandigheden, nu de twintigste eeuw ten einde loopt?

\section{Democratie, de broederlijke tweeling van het
communisme?}\label{democratie-de-broederlijke-tweeling-van-het-communisme}

In hoofdstuk 5 gaven we een paradoxale verklaring: de democratie bloeide
als broederlijke tweeling van het communisme juist omdat de staat
onbelemmerd de controle over de middelen kon uitoefenen. Deze conclusie
lijkt wellicht absurd volgens het `gezond verstand' van het industriële
tijdperk. We ontkennen niet dat in die samenleving democratische
systemen en het communisme bittere tegenpolen vormden. Maar als je beide
systemen vanuit een megapolitiek perspectief bekijkt -- zoals het
informatietijdperk dat waarschijnlijker benadert -- hebben ze meer
gemeen dan je zou vermoeden.

In tijden waarin wapens grotesk duur waren, fungeerde de democratie als
het besluitvormingsmechanisme dat de staatscontrole over de middelen tot
het uiterste dreef. Net als staatssocialistische systemen investeerden
democratische regeringen enorme bedragen om een massale militaire opbouw
te financieren. Het verschil was dat de democratische verzorgingsstaat
de staat in staat stelde om nog meer middelen in handen te krijgen dan
de staatssocialistische systemen -- die immers praktisch elk waardevol
bezit opeisten.

Als zuiver instrument om financiering bijeen te drijven, bleek de
democratische staat objectief superieur te zijn aan het staatssocialisme
in haar vermogen de staatskas te verrijken. Zoals we eerder toelichtten,
kon de democratie haar militairen aanzienlijk meer geld verschaffen
dankzij de aansluiting op particulier eigendom en de productiviteit van
het kapitalisme.

Het staatssocialistische systeem baseerde zich op de doctrine dat de
staat alles bezat. Daarentegen beperkte de democratische
verzorgingsstaat haar initiële aanspraken en deed zij alsof particulier
eigendom was toegestaan -- zij het in een voorwaardelijke vorm --
waardoor zij betere prikkels bood om de productie te stimuleren. In
plaats van van meet af aan alles verkeerd te regelen, gaven westerse
regeringen burgers de kans om eigendom te vergaren en vermogen op te
bouwen. Pas nadat dat vermogen was ontstaan, grepen de democratische
natiestaten in en belastten zij er een Large fractie mee.

Het woord `Large' schrijven we met een hoofdletter. In 1996 bedroeg het
levenslange federale belastingtarief in de Verenigde Staten bijvoorbeeld
73 cent per dollar. Voor ondernemers die hun inkomen via dividenden
ontvingen, lag het tarief op 83 cent per dollar, terwijl het voor mensen
die geld wilden nalaten of aan kleinkinderen wilden schenken, 93 cent
per dollar bedroeg. Wanneer men ook staats- en lokale belastingen
meerekent, confisqueert de democratische overheid op alle niveaus het
merendeel van elke verdiende dollar. Deze roofdierachtige
belastingtarieven maakten de democratische staat in feite tot een
partner die tussen drie-kwarten en negen-tenths van alle inkomsten
opeiste -- al is dat niet hetzelfde als staatssocialisme, het vertoont
er wel duidelijke gelijkenissen.

De democratische staat overleefde langer omdat zij flexibeler was en
enorme middelen bijeenbracht, vergeleken met wat beschikbaar was in
Moskou of Oost-Berlijn.

Inefficiëntie, waar het telde

Wij hebben de megapolitieke troef van democratie als
besluitvormingsmethode binnen een krachtige regering verwoord met de
slagzin `inefficiëntie, waar het telde.' Vergeleken met het communisme
werkte de verzorgingsstaat immers efficiënter. Maar wanneer je de
verzorgingsstaat afzet tegen een zuivere laissez‑faire-enclave als
Hongkong, blijkt zij in alle eenvoud inefficiënt te zijn. De
groeicijfers in Hongkong waren ronduit indrukwekkend, vooral omdat de
inwoners -- en niet de overheid -- in staat waren om 85 procent van de
vruchten van de razendsnelle groei voor zichzelf te houden.

Hongkong is natuurlijk geen democratie. Integendeel, het
vertegenwoordigt juist een mentaal model van het type rechtsgebied dat
in het informatietijdperk zal floreren. In het industriële tijdperk had
Hongkong immers geen behoefte aan democratie, omdat men werd behoed voor
de lastige verplichting om middelen te verzamelen voor de ondersteuning
van een formidabele militaire macht. Aangezien Hongkong van buitenaf
werd verdedigd, kon men zich een werkelijk vrije economie permitteren.

Juist dat vermogen om eigen middelen te vergaren plaatste de democratie
in de megapolitieke context van het industriële tijdperk boven andere
systemen. Massademocratie en industrialisme gingen als een
onafscheidelijk duo hand in hand. Zoals \emph{Alvin Toffler} opmerkte,
is massademocratie `de politieke uitdrukking van massaproductie,
massadistributie, massaconsumptie, massaal onderwijs, massamedia,
massavermaak en alles daartussenin.'\footnote{Geciteerd in Kelly,
  \emph{op. cit.}, p.~46.}

Nu informatietechnologie de massaproductie grotendeels vervangt, mogen
we verwachten dat het tijdperk van massademocratie ten einde komt. Het
fundamentele megapolitieke imperatief dat massademocratie in het
industriële tijdperk deed zegevieren, is verdwenen. Het is daarom
slechts een kwestie van tijd voordat massademocratie hetzelfde lot
ondergaat als haar broederlijke tweeling, het communisme.

Massademocratie onverenigbaar met het informatietijdperk

Bij nader inzien blijkt dat de technologie van het informatietijdperk op
zichzelf geen massatechnologie vormt. In militaire context -- zoals al
eerder aangegeven -- maakt deze technologie de ontwikkeling van `slimme
wapens' en een `informatieoorlog' mogelijk, waarbij zogenaamde `logische
bommen' centraal aangestuurde commando- en controlesystemen kunnen
uitschakelen. Informatietechnologie illustreert niet alleen de
verfijning van wapens die door specialisten worden bediend, maar
vermindert tevens de impact van offensieve operaties, waardoor de
positie van de verdediging relatief versterkt wordt. Dankzij
microtechnologie behaalt een individu een aanzienlijk grotere militaire
macht, waardoor het strategisch belang van massale infanterieformaties
afneemt. Volgens \emph{`Rand Corporation'} rapporteerde aan de minister
van Defensie: `Onderling verbonden netwerken kunnen worden aangevallen
en verstoord, niet alleen door staten, maar ook door niet-statelijke
actoren, waaronder verspreide groepen en zelfs individuen.'\footnote{Molander
  et al., \emph{Strategic Information Warfare}, op. cit., p.~xv.}
Bovendien suggereert dit dat een cyberoorlog de inherente nadelen van
sterk gecentraliseerde systemen teniet zal doen.

Volgens \emph{`Rand-experts'} maken informatiegebaseerde technieken
geografische afstanden onbelangrijk; doelwitten binnen de continentale
Verenigde Staten lopen net zo'n risico als agenten in operationele
gebieden.\footnote{Ibid., p.~xiv.} Waar men vroeger aannam dat het
veilig was om binnen de grenzen van grote supermachten als de Verenigde
Staten te verblijven, keert in het informatietijdperk de logica van
machtsconcentratie om. Zelfs een plaats als Peoria, hoewel ver van een
mogelijk militair front, biedt geen bescherming meer tegen
cyberaanvallen van bijna elke tegenstander. Wonen binnen de grenzen van
een supermacht maakt je immers tot een direct doelwit. In plaats van te
centraliseren, kunnen lokale gemeenschappen hun veiligheid beter
waarborgen door te decentraliseren. Door de opkomst van cyberoorlog
groeit de kwetsbaarheid van centraal georganiseerde commando- en
controlesystemen, terwijl verspreide systemen steeds concurrerender
worden.

De op gang gebrachte feedbackmechanismen kunnen het
decentralisatieproces verder versnellen. Volgens \emph{`Rand-experts'}
zal de overheid, om de kwetsbaarheid van de in de latere fases van de
natiestaat ontwikkelde commandostructuren voor cyberaanvallen te
verkleinen, gedwongen worden de inzet van nieuwe
softwareversleutelingstechnieken te intensiveren. Daarmee worden deze
systemen, die grotendeels tot de privésector behoren, veel minder
vatbaar voor sabotage. Tegelijkertijd versnelt dit de commerciële
verspreiding van sterke encryptie, waardoor zij zich beter kunnen
losmaken van staatscontrole. Ook dit stimuleert de decentralisatie en
bevordert de spreiding van middelen in cyberspace, zodat zij buiten het
bereik van de politiek blijven.

Op de lange termijn leidt dit tot het einde van massademocratie, met
name in haar overheersende vorm van representatief wanbestuur -- zowel
in een congres- als in een parlementair stelsel.

\section{De megapolitiek van
misvertegenwoordiging}\label{de-megapolitiek-van-misvertegenwoordiging}

Wanneer de megapolitieke omstandigheden drastisch veranderen -- zoals we
nu duidelijk zien -- past de organisatie van de overheid zich
onvermijdelijk aan. Sterker nog, de representatieve overheid
weerspiegelt al altijd de verdeling van ruwe macht, want men kiest de
vertegenwoordigers op basis van geografie in plaats van volgens andere
criteria.

Stel je voor: een wetgevend orgaan zou net zo democratisch zijn als
wanneer de leden willekeurig uit de bevolking werden gekozen.
Parlementswijken of kiesdistricten zouden bijvoorbeeld gebaseerd kunnen
worden op geboortedata of zelfs op een alfabetische indeling. Iedereen
die op 1 januari geboren is, zou via één kandidatenlijst stemmen,
terwijl degenen die op 2 januari geboren zijn, via een andere lijst
zouden kiezen. Of iedere persoon wiens naam begint met `Aa' tot `Af' zou
uit één kandidatenlijst kiezen, en wie met `Ag' begint, zou via een
andere lijst gaan stemmen. En zo verder.

Zo'n systeem bestaat tegenwoordig om verschillende redenen niet. Ten
eerste was dit in de achttiende eeuw technologisch simpelweg niet
uitvoerbaar. Nog belangrijker is dat kiesdistricten gebaseerd op
geboortedata of alphabetische indelingen de werkelijke verdeling van
ruwe macht -- zoals die tijdens verkiezingen tot uiting komt -- niet
hadden kunnen weerspiegelen of benaderen. Mensen die slechts dezelfde
geboortedatum of de eerste paar letters van hun naam delen, vormen
immers nauwelijks een samenhangende machtsbasis.

\subsection{Waarom wegen geografische indelingen
zwaarder?}\label{waarom-wegen-geografische-indelingen-zwaarder}

Stemming ontstond oorspronkelijk als een alternatief voor militaire
strijd -- en dat is nog steeds het geval, zij het op een minder directe
wijze. Dergelijke conflicten worden doorgaans langs geografische lijnen
georganiseerd, en minder vaak op basis van verwantschap of religie. Je
kunt zulke conflicten immers niet inrichten op basis van verjaardagen of
de eerste letters van namen. Ook kan men conflicten op basis van
beroepen niet effectief organiseren, tenzij die beroepsgroepen beperkt
blijven tot erfelijke gilden -- zoals de kasten in India -- of wanneer
ze zich lokaal concentreren, bijvoorbeeld onder boeren in Iowa.

Het uitgangspunt van de huidige representatieprincipes is dat zij
belangen vertegenwoordigen die geografisch verankerd zijn, in plaats van
langs een andere dimensie. Historisch hing militair succes nauw samen
met het beheersen van grondgebied; alle militaire dreigingen ontstonden
op lokaal niveau. Vertegenwoordigingssystemen bieden een alternatieve
arena om die macht tot uiting te brengen. Het bevoordelen van gevestigde
lokale belangen is dan ook een onvermijdelijk gevolg van deze formule.
Geografisch ingedeelde kiesdistricten zorgen ervoor dat
vertegenwoordigers specifieke groepen gaan bevoordelen, vaak ten koste
van de gemeenschappelijke belangen die alle inwoners delen.

\subsection{Nieuwe mogelijkheden in het
verschiet}\label{nieuwe-mogelijkheden-in-het-verschiet}

Zoals analyses door \emph{Public Choice}-economen aantonen, veroorzaken
schijnbaar kleine aanpassingen in de structuur van een verkiezing of in
de wijze waarop de stemming wordt geteld grote en voorspelbare effecten
op de uitslag.\footnote{Dennis C. Mueller, \emph{Public Choice}, vol.~2
  (Cambridge: \emph{Cambridge University Press}, 1989), pp.~43--226.}
Daarom moeten serieuze politicastudenten tegenwoordig ook grondige
kennis van grondwetten bezitten. Het is bovendien een van de redenen om
verder te kijken dan traditionele grondwetten en in plaats daarvan de
ultieme metagrondwet te overwegen, zoals gevormd door de overheersende
megapolitieke factoren in een bepaalde omgeving.

Technologische vernieuwingen hebben enkele van de fundamenten weggevaagd
die ervoor zorgden dat stemmen vroeger uitsluitend binnen geografische
kiesdistricten plaatsvond. In de achttiende en negentiende eeuw verliep
bijna alle communicatie lokaal. De meeste mensen leefden en stierfden
binnen enkele mijlen van hun geboorteplaats, en al hun
handelsactiviteiten en communicatie vonden in de directe omgeving
plaats. Tegenwoordig verloopt de communicatie wereldwijd vrijwel
onmiddellijk. Je kunt net zo eenvoudig zaken doen met iemand die
vijfduizend mijl ver weg woont als met je buurman. Steeds vaker
overstijgt de economie geografische grenzen, en is de samenleving veel
mobieler geworden.

Dit geldt ook voor rijkdom in het informatietijdperk. In tegenstelling
tot een staalfabriek, die politici als drukmiddel kunnen inzetten, kan
men een computerprogramma niet zomaar gijzelen via het lokale politieke
systeem. Een staalfabriek blijft hardnekkig op één plek staan, zelfs
wanneer wetgevers besluiten extra belastingen te heffen of strengere
regels in te voeren. Daarentegen laat een computerprogramma zich via een
modem bliksemsnel wereldwijd overdragen. De eigenaar kan simpelweg zijn
laptop inpakken en letterlijk wegvliegen. Dit ondermijnt tevens de
megapolitieke basis van geografisch bepaalde kiesdistricten.

Een aanzienlijk probleem dat volgens onze analyse alle representatieve
democratieën teistert, is dat geografische kiesdistricten van nature de
gevestigde belangen uit het industrieel tijdperk bevoordelen.
`Achterblijvers' vormen ideale kiezers doordat zij geografisch
geconcentreerd wonen en politiek behoeftig zijn. Ook de geschiedenis van
de industriële democratie bevestigt dit. Tijdens de hoogtijdagen van het
industriële tijdperk in de jaren dertig kregen `winnaars' uit nieuwe
industrieën structureel te weinig vertegenwoordiging in de wetgevende
debatten.\footnote{Michael A. Bernstein, \emph{De grote depressie:
  uitgesteld herstel en economische verandering in Amerika, 1929--1939}
  (Cambridge: \emph{Cambridge University Press}, 1987).} Politici kiezen
er van nature voor de gevestigde spelers te ondersteunen in plaats van
de nieuwe ondernemingen of hun potentiële klanten -- een kenmerk dat het
representatieve bestuur inherent typeert. Zoals \emph{The Rise and
Decline of Nations} betoogt, smeden langlevende industrieën doorgaans
effectievere `distributieve coalities' om te lobbyen en te strijden voor
de politieke buit.\footnote{Mancur Olson, \emph{De opkomst en ondergang
  van naties: economische groei, stagflatie en sociale rigiditeiten}
  (New Haven: \emph{Yale University Press}, 1982).}

Dit probleem neemt in de economie van het informatietijdperk enorm toe.
De creatieve krachten van de nieuwe economie verspreiden zich over de
hele wereld, waardoor zij vrijwel nooit een voldoende geconcentreerde
groep vormen die de aandacht van wetgevers weet te trekken, zoals
zalmvissers in Schotland of tarweboeren in Saskatchewan dat doen.
Sterker nog, veel van de dynamische talenten achter de nieuwe economie
slagen er waarschijnlijk niet in burger te worden binnen zelfs de
ruimste jurisdicties. Daardoor hebben zij nauwelijks stem in de
wetgevende besluitvorming van representatieve democratieën. Als treffend
voorbeeld noemen we de dubieuze pogingen van Amerikaanse wiskunde-PhD's
om buitenlandse wiskundigen de kans op een baan in de Verenigde Staten
te ontnemen.\footnote{Michael M. Phillips, `Wiskunde-ph.d.'s dragen bij
  aan de anti-vreemdelingengolf: geleerden die, geconfronteerd met hoge
  werkloosheid, pleiten voor immigratiebeperkingen,' \emph{Wall Street
  Journal}, 4 september 1996, p.~A2.} Hun xenofobische betogen aan het
Congres om werkgevers op basis van verdienste te blokkeren, zijn
inmiddels al lang achterhaald. De achterhaalde geografische
vertegenwoordiging, een erfenis uit het industrieel tijdperk, houdt geen
rekening met buitenlandse wiskundigen of andere cruciale
welvaartsbijdragers die geen kiesrecht hebben.

`Waarom geloven mensen in de legitimiteit van democratische instituties?
Het beantwoorden van die vraag is bijna net zo ingewikkeld als verklaren
waarom mensen in bepaalde religieuze dogma's geloven, aangezien het
niveau van begrip, scepsis en geloof binnen een samenleving en in de
loop der tijd sterk uiteenloopt.'\footnote{Juan J. Linz en Alfred Stepan
  (red.), \emph{de ineenstorting van democratische regimes} (Baltimore,
  Md.: \emph{The Johns Hopkins University Press}, 1978), p.~18.} JUAN J.
LINZ

Slechts weinigen hebben systematisch stilgestaan bij de impact van
technologische veranderingen die het industrialisme ondermijnen en de
inkomensverdeling radicaal herschikken. Het lijkt erop dat democratie
dan vooral een recept voor wettelijk parasitisme vormt, indien de
inkomens zo ver uiteenlopen als in de informatie-economie mogelijk is.
Bijna niemand heeft doorgrond dat sommige instituties van de industriële
overheid inherent onverenigbaar zijn met de megapolitieke realiteit van
een postindustriële samenleving. Of we deze tegenstellingen nu expliciet
erkennen of niet, hun gevolgen worden steeds duidelijker naarmate er
wereldwijd meer voorbeelden van politiek falen optreden. De
overheidsinstellingen die in onze moderne tijd zijn gevormd,
weerspiegelen de geopolitieke condities van eeuwen geleden en hebben de
omslag van een agrarische samenleving naar stedelijk industrialisme
doorstaan. Het informatietijdperk kan echter om nieuwe
representatiemechanismen vragen om chronische disfunctie -- of zelfs een
ineenstorting in Sovjetstijl -- te voorkomen.

U kunt verwachten dat in veel landen crisis van slecht bestuur zal
uitbreken, omdat politieke beloften leeg blijken te zijn en regeringen
hun krediet en institutionele steun verliezen. Uiteindelijk zullen
nieuwe institutionele vormen ontstaan die de vrijheid in deze nieuwe
technologische omstandigheden waarborgen en tegelijkertijd de
gemeenschappelijke belangen van individuen tot uiting brengen.

Dit alles wijst op het einde van de massademocratie zoals wij die in de
twintigste eeuw kenden. De vraag luidt: wat neemt haar plaats in? Als
het enige alternatief voor massademocratie een dictatuur zou zijn
waarbij het individu geen zeggenschap heeft over zijn eigen lot, dan
zouden sommigen in de verleiding kunnen komen zich bij de neo-Luddieten,
oftewel de `opstand tegen de toekomst', aan te sluiten.

\subsection{Nieuwe instellingen}\label{nieuwe-instellingen}

Gelukkig is dictatuur niet het enige alternatief voor massademocratie.
Informatietechnologie maakt het mogelijk eigen keuzes te maken. In
plaats van een collectieve beslissing binnen de beperkende context van
`massaproductie, massaconsumptie, massaal onderwijs, massamedia, massaal
vermaak en al het andere', faciliteert informatietechnologie een
werkelijk consumentgerichte keuze voor op maat gemaakte
soevereiniteitsdiensten. Dit wordt mogelijk doordat grootschalig
opereren niet langer noodzakelijk is. Wij zijn ervan overtuigd dat de
technologie van het informatietijdperk zal leiden tot nieuwe vormen van
bestuur -- net zoals de agrarische revolutie en later het industriële
tijdperk elk hun eigen specifieke vormen van sociale organisatie
voortbrachten.

Wat voor nieuwe instellingen zouden dat kunnen zijn? Laat daarbij alle
verkeerd benoemde politicologieteksten voor wat ze zijn. De nieuwe
bestuursvormen voor het informatietijdperk zullen de grenzen van
conventioneel denken overschrijden. De overgang naar zulke instellingen
is al in gang gezet. Het gaat om nauwelijks erkende improvisaties die
gericht zijn op het herstructureren van onderbenutte voordelen van
soevereiniteit. De natiestaten van de wereld, bezorgd over
afscheidingsbewegingen en de ingrijpende effecten van devolutie, hebben
zich verenigd tot het krachtigste grensbewaringskartel dat zij kunnen
vormen. Hoewel het aantal nieuwe staten in de jaren negentig wereldwijd
is toegenomen, gebeurde dit vooral in twee clusters dankzij de
ineenstorting van multietnische, communistische dictaturen in de
voormalige Sovjet-Unie en Joegoslavië. Het valt op dat andere
vooraanstaande natiestaten -- waaronder de Verenigde Staten --
strategieën hanteerden om de Sovjet-Unie zo lang mogelijk in stand te
houden. En maar weinig regeringen stonden positief tegenover de
ontbinding van Joegoslavië. Pas nadat afscheidingsbewegingen,
ondersteund door eigen militaire inspanningen, de controle overnamen,
werd de onafhankelijkheid van de voormalige Joegoslavische republieken
erkend. De wereldmachten schikten zich uiteindelijk bij het aanschouwen
van ongewapende of slecht bewapende separatisten, die massaal door hun
Servische onderdrukkers werden afgeslacht. Zelfs het verre China, een
machtige natiestaat zonder direct belang bij het behoud van het resten
van Joegoslavië, verzette zich fel tegen de pogingen tot zelfbeschikking
van de onderdrukte etnische Albanezen in Kosovo. Opmerkelijk genoeg zal
deze fixatie op het vastleggen van grenzen eerder leiden tot een
gefragmenteerde soevereiniteit dan de devolutie daadwerkelijk tegengaan.
De felle weerstand van kwetsbare natiestaten wereldwijd tegen open
afscheiding en politieke desintegratie maakt erkende soevereiniteit tot
een waardevolle vorm van transcendent kapitaal, dat staten vrijwillig
kunnen fragmenteren en onderverhuren.

Een voorbeeld van het vrijwillig fragmenteren van soevereiniteit,
waardoor in feite een particuliere, belastingvrije jurisdictie ontstaat,
is de Agulhas Bay Concession Free Zone, die vijftig vierkante kilometer
van het eiland Principe voor de kust van West-Afrika omvat. Hoewel dit
grondgebied binnen de grenzen van de Democratische Republiek São Tomé en
Principe valt, wordt de zone administratief geprivatiseerd. Het bestuur
wordt contractueel geregeld en verzorgd door WADCO, de \emph{West
African Development Corporation Ltd.}, een particulier bedrijf dat in
Zuid-Afrika is opgericht. In de zone is de voertaal Engels in plaats van
het officiële Portugees van São Tomé en Principe, en wordt de handel
gevoerd in de dollar -- het wereldgeld, de Amerikaanse dollar. De
veiligheid wordt niet gewaarborgd door de nationale politie van São Tomé
en Principe, maar door private politiediensten die door WADCO in dienst
zijn genomen. Op de handel binnen de zone is het commerciële recht van
São Tomé niet van toepassing en hebben de rechtbanken van São Tomé geen
rechtsmacht. Alle geschillen lossen de partijen op via transnationale
arbitrage conform de Parijse ICC-regels. Behalve een paar strikt
gecontroleerde en triviale uitzonderingen geldt de belastingwetgeving
van São Tomé niet en functioneren officiële monopolies daar eveneens
niet. Zo worden bijvoorbeeld telecommunicatiediensten binnen de zone
automatisch gedereguleerd. Onder voorbehoud van tijdige huurbetaling en
naleving van de overige concessievoorwaarden mag WADCO haar huurcontract
op deze particuliere, gefragmenteerde soevereiniteit automatisch en
telkens met vijftig jaar verlengen, gerekend vanaf de eerste
verlengingsdatum in 2047.

Wat WADCO in São Tomé en Principe heeft bereikt, kan door anderen in
uiteenlopende rechtsgebieden worden gekopieerd -- en dat zal ook
gebeuren. Een ware pionier van de ontwikkeling in de eenentwintigste
eeuw, Joaquin Aguirre, heeft in de Centrale Aguirre Portuaria in
oostelijk Bolivia een soortgelijke zone van particuliere soevereiniteit
opgericht. Aguirre -- multimiljonair, romanschrijver, uitvinder,
medeoprichter van de Verenigde Naties én voormalig senator van de
Boliviaanse Republiek -- is op vele vlakken een echte pionier. Halve
eeuw nadat hij een belangrijke rol speelde bij de oprichting van de
Verenigde Naties, belichaamt hij nu het archetype van het soevereine
individu in de eenentwintigste eeuw. Zijn Zona Franca, vrij van de
meeste Boliviaanse belastingen, heffingen en regelgevende beperkingen,
wijst de weg naar een nieuwe soort geprivatiseerde stadstaten die door
succesvolle individuen in het informatietijdperk steeds vaker
gerealiseerd zullen worden. Het bewijst bovendien onmiskenbaar dat het
bestaan van de massa's -- zo vaak besingevuld in lofzangen door de
voorstanders van een sterke overheid -- ingrijpend kan verbeteren
dankzij de economische ontwikkeling die vrijkomt via vrijhandelszones,
zoals die gestart door Señor Aguirre. Geleidelijk zal het aantal de
facto stadsstaten wereldwijd aanzienlijk toenemen. Inderdaad, als je als
individu voldoende financiële onafhankelijkheid bereikt, kun je de
ultieme mate van onafhankelijkheid verwerven, net zoals Joaquin Aguirre.
Mocht een gefragmenteerde, gecommercialiseerde soevereiniteit van een
ander je geen comfortabele thuisbasis bieden, dan kun je altijd je eigen
ministaat opzetten -- net zo onafhankelijk als een middeleeuws
hertogdom. In plaats van met demagogen en politieke opportunisten in de
strijd te gaan om te voorkomen dat je bezittingen worden weggeharkt en
verdeeld onder de talloze kravende handen van de massademocratie, kun je
je eigen particuliere domein van bestuur creëren.

De ingrijpende overgang van massademocratie naar het soevereine individu
als ultieme vorm van zelfbestuur hoeft niet gepaard te gaan met een
radicale verandering in de publieke opinie, noch met een wonderbaarlijke
stemming waardoor ontgoochelde kiezers massademocratie afschaffen. Zo'n
revolutie kan namelijk al onopgemerkt zijn ingezet door het verhuren van
soeverein grondgebied voor belastingvrije zones, `Zona Francas', en
vrije havens. Op termijn zal soevereiniteit steeds verder worden
gefragmenteerd, tot het punt waarop verdere opsplitsing niet meer genoeg
waarde oplevert om de transactiekosten van de devolutie te compenseren.
Gelet op de wet van Moore en Gilder's corollarium -- dat stelt dat de
bandbreedte elk jaar verdrievoudigt -- is er momenteel geen reden om te
verwachten dat de devolutietrend vroegtijdig zal stoppen. Integendeel,
wij voorzien dat de schijnbaar solide macht van de natiestaten die
vandaag de dag massademocratie belichamen, zal uiteenvallen in
tienduizenden fragmenten, in een systeem dat meer doet denken aan de
middeleeuwen dan aan het moderne industriële tijdperk.

Op termijn zullen zelfs natiestaten die nog restanten van
massademocratie in zich dragen, een ingrijpende beleidsverschuiving
doormaken om in lijn te komen met de nieuwe metaconstitutionele
realiteiten. Zoals William Keech, een vurig voorstander van democratie,
in \emph{Economic Politics: The Costs of Democracy} betoogt:\footnote{William
  Keech, \emph{economische politiek: de kosten van de democratie}
  (Cambridge: \emph{Cambridge University Press}, 1995), p.~221.} `Mensen
leren te willen wat ze zien dat ze kunnen krijgen, maar ze kunnen ook
van gedachten veranderen als ze merken dat ze niet houden van wat ze
wilden en wat ze kregen.' Met andere woorden, dat massademocratie met
haar conventionele instituties overal wordt geprezen nu de twintigste
eeuw ten einde loopt, zou wel eens een verkoopsignaal kunnen zijn. Dit
garandeert echter geenszins dat deze besluitvormingsregels de tand des
tijds zullen doorstaan, zelfs niet volgens hun eigen maatstaven. Bedenk
dat, als je buiten de politiek kijkt, er nauwelijks bewijs is dat
bestuurders, beheerders, coaches of andere professionele leiders via
democratische keuzes worden aangesteld. Integendeel, de meest
succesvolle leiders worden routinematig door eigenaars voorgedragen via
selectieprocessen, waarbij diegenen met het grootste belang een
onevenredig grote stem hebben. Als democratische selectie werkelijk een
universele methode zou zijn om capabele leiders te vinden, zou je
verwachten dat dit vrijwel uitsluitend in de politieke sfeer gebeurt.
Kortom, op basis van de huidige gegevens lijkt het aannemelijker dat de
levering van soevereine diensten wordt belemmerd door de overheersing
van democratische besluitvorming, dan dat bedrijven en
handelsorganisaties eronder lijden dat zij geleid worden door door
eigenaars aangestelde bestuurders in plaats van door een stemming.

Tegen de helft van de eenentwintigste eeuw zal de toename van
particuliere jurisdicties, gebaseerd op gefragmenteerde soevereiniteit,
onomstotelijk de voordelen van particuliere administratie bewijzen.
Kiezers zullen merken dat zij de lasten van massademocratie dragen.
Daarom zullen zij -- zoals professor Keech suggereert -- concluderen dat
de voordelen van werknemerscontrole over de overheid de kosten niet
compenseren en zetten zij de stap richting hervorming. Zelfs kieskringen
in Europa en Noord-Amerika, die nu schijnbaar fel tegen hervorming zijn,
zouden uiteindelijk kunnen stemmen om hun regio's meer open te stellen
voor particuliere bestuursvormen. Meerderheden zullen wellicht
vrijwillig -- ja, zelfs graag -- de politieke schijnvertoning opgeven
ten gunste van een particuliere bestuursvorm die er werkelijk op gericht
is de optimale voorwaarden te scheppen voor het sluiten en afdwingen van
contracten.

Zolang de overheid met al haar vertrouwde toebehoren nog bestaat, kan
zij op totaal nieuwe manieren worden geïnformeerd. Ergens, in een
bepaald rechtsgebied -- nog voordat het te laat is -- zal iemand het
potentieel van computertechnologie doorgronden, waardoor een werkelijk
representatieve overheid mogelijk wordt. We kunnen het probleem van
buitensporige campagne-uitgaven en de onmiskenbare irritatie over
voortdurende politieke campagnes binnen een oogwenk oplossen. In plaats
van verkozen te worden, selecteren we de vertegenwoordigers volledig
willekeurig via trekking, zodat hun talenten en zienswijzen nauw
aansluiten bij die van de algemene bevolking.

Dit is in feite een moderne variant van het oude Griekse systeem waarbij
men door loting functies vervulde. Zoals E. S. Staveley beschrijft in
zijn gezaghebbende geschiedenis \emph{Greek and Roman Voting and
Elections} selecteerde Athene talrijke ambtspersonen -- van magistraten
tot archons -- via trekking in plaats van via verkiezingen.\footnote{E.S.
  Staveley, \emph{Grieks en Romeins stemmen en verkiezingen} (Ithaca,
  N.Y.: \emph{Cornell University Press}, 1972), p.~62.} Ze deden dit op
een slimme manier, ondanks de mechanische beperkingen bij het
willekeurig genereren van kansen, door gebruik te maken van een
sorteermachine -- of, zoals de Atheners het noemden, de cleroterion.

Men gebruikte een rij zwart-witte bonen als willekeurige tellers om te
bepalen wie voor diverse ambten in aanmerking kwam en om de volgorde
vast te stellen waarin de tribale afdelingen als prytaneis in de Raad
aan de beurt waren.\footnote{Ibid., p.~65.} Het feit dat dit idee uit de
klassieke oudheid stamt, geeft het extra geloofwaardigheid. Het meest
aantrekkelijke aan dit systeem is echter dat het de nadelen van
zelfselectie in de politiek voorkomt -- statistisch gezien domineren
hierdoor minder advocaten en egomaniakken de publieke aangelegenheden.

Wetgevende lichamen kunnen zo worden gevormd uit ware
vertegenwoordigers. Doordat zij niet worden samengebracht door
machtsbejag en vrijwel nooit opnieuw per lot worden gekozen, hebben zij
de vrijheid om staatszaken te beheren en beleid te formuleren op basis
van een rationele analyse van de kwesties.

\subsection{Rechtstreekse commissie}\label{rechtstreekse-commissie}

Tegenwoordig hebben politici -- die er alles aan doen om stemmen te
winnen -- nauwelijks motivatie om problemen op een samenhangende wijze
te onderzoeken. Het is dan ook weinig verrassend dat hun staat van
dienst in het daadwerkelijk oplossen van problemen zo armzalig is
vergeleken met die van ondernemers, bedrijfsleiders en sportcoaches, die
op basis van prestaties beloond worden. Een beloningssysteem dat
wetgevers -- die willekeurig gekozen worden -- naar hun resultaten
beloont, zou nooit leiden tot het niveau van effectiviteit dat iemand
als Lee Kuan Yew kenmerkt. Toch is er alle reden te geloven dat hun
prestaties sterk zouden verbeteren als het salaris gekoppeld werd aan
een objectieve maatstaf, bijvoorbeeld de groei van het inkomen per hoofd
van de bevolking na belastingen. Beloon hen op basis van prestaties en
de kans dat zij goed presteren zal duizendmaal toenemen.

Het voordeel voor de samenleving van beleid dat het reële inkomen na
belastingen verhoogt, kan enorm zijn. Waarom zouden premiers en
presidenten niet zelfs een klein deel van die winst ontvangen? De
benodigde financiering voor dergelijke betalingen zou opgehaald kunnen
worden via een kleine, onopvallende belasting. Zo'n regeling zou de
samenleving bevrijden van de bedreiging waar zij momenteel mee te maken
heeft door ambitieuze politici met specialistisch politiek talent, zoals
Richard Nixon en Bill Clinton.

\begin{quote}
`Ze brachten hem goud, zilver en kleding; maar de ``Christus'' deelde al
deze zaken uit aan de armen. Wanneer er geschenken werden aangeboden,
wierpen hij en zijn vrouwelijke metgezel zich op de knieën en spraken
zij gebeden uit; maar zodra hij weer overeind kwam, beval hij de menigte
hem te aanbidden. Later organiseerde hij een gewapende bende, die hij
door het platteland leidde, reizigers op de loer liggend en beroofde hen
onderweg. Ook hier was zijn ambitie niet om rijk te worden, maar om
aanbeden te worden. Hij verdeelde de hele buit onder hen die niets
hadden -- waaronder, naar men aanneemt, ook zijn eigen volgelingen.'
\footnote{Norman Cohn, \emph{De zoektocht naar het millennium} (Oxford:
  \emph{Oxford University Press}, 1970), p.~41.} NORMAN COHN
\end{quote}

\subsection{Messiaanse
persoonlijkheden}\label{messiaanse-persoonlijkheden}

Er wordt te weinig stilgestaan bij het feit dat verkiezingspolitiek
chaotische, messiaanse persoonlijkheden aantrekt. Zulke figuren hebben
altijd bestaan en vormden vaak een serieuze bedreiging voor de sociale
orde, zelfs in agrarische samenlevingen lang voordat democratische
systemen ontstonden. Kijk je naar de loopbanen van Eudo de Stell, de
Bretonse Christus, Adelbert in de achtste eeuw, Eon in de elfde eeuw,
Tanchelm uit Antwerpen, Melchior Hoffman, Bernt Rothmann en
soortgenoten, valt één ding op: hoe duidelijker hun politieke talenten
leken, hoe groter de schade die zij aanrichtten. Omdat de staat nog niet
de handen had uitgeslagen in het organiseren van grootschalige,
systematische dwang, namen deze vroege protopolitici vaak zelf het
initiatief om te plunderen en te beroven. Zo vergaarden zij contant
geld, dat ze vervolgens onder hun volgelingen -- de armen -- verdeelden.

\subsection{Protopolitici in actie}\label{protopolitici-in-actie}

De verhalen over hun capriolen wekken de indruk van talenten die ver
achter hun tijd liggen, alsof je leest over mannen van zeven voet die
over een veld rennen, lang voordat basketbal in opkomst was.
Tegenwoordig verdienen deze torenhoge spelers dankzij het \emph{NBA}
miljoenen met dribbelen en dunkelen. Mocht basketbal verdwijnen, dan
zouden zij weer in de kieren van de samenleving verdwijnen --
hoogstwaarschijnlijk als circusacten en in bijvoorstellingen.

Toen formele politiek nog niet bestond, sloegen demagogen de weg in van
rondtrekkende prediking, de beste politieke invulling die de agrarische
wereld te bieden had. Ze richtten hun preken tot grote menigten en
beloofden, net als hedendaagse politici, op welbespraakte wijze een
beter leven aan hun volgelingen. Zowel vroeger als nu richten demagogen
zich primair op de armen. Norman Cohns grootse geschiedenis van
millenaristische bewegingen, \emph{The Pursuit of the Millennium},
beschrijft de loopbanen van talloze messiaanse leiders in een tijd nog
vóór de invoering van verkiezingen. Wat daarbij opvalt, is hoe sterk hun
persoonlijkheid overeenkomt met die van de charismatische politicus van
vandaag.

\begin{quote}
De leider bezit -- net als farao's en vele andere `goddelijke koningen'
-- alle eigenschappen van een ideale vader: hij straalt volmaakte
wijsheid en rechtvaardigheid uit en beschermt de zwakken. Tegelijkertijd
fungeert hij als de zoon die belast is met het transformeren van de
wereld, als de Messias die een nieuw hemelrijk en een herboren aarde zal
stichten en die trots kan verkondigen: `Zie, ik maak alle dingen nieuw!'
Zowel in de rol van vader als die van zoon is deze figuur kolossaal,
bovenmenselijk en almachtig. Men schrijft hem een overvloed aan
bovennatuurlijke krachten toe, waardoor hij wordt voorgesteld als licht
dat in stralen naar buiten breekt. Bovendien, door de goddelijke geest
die hem vervult, beschikt de eschatologische leider over unieke
wonderkrachten. Zijn legers zegevieren steevast en triomfantelijk, zijn
aanwezigheid maakt de aarde vruchtbaar en zijn heerschappij leidt tot
een tijdperk van harmonie zoals de oude, corrupte wereld nog nooit heeft
gekend.
\end{quote}

\begin{quote}
Deze voorstelling is uiteraard puur fantasie, omdat zij niets zegt over
de werkelijke aard en capaciteiten van enig menselijk wezen dat ooit
heeft geleefd of ooit zou kunnen leven. Toch kon men dit beeld op een
levend mens projecteren; er waren altijd mannen die er meer dan in
meegingen, die vurig verlangden om als onfeilbare, wonderbaarlijke
redders gezien te worden. Het geheim van hun opmars lag nooit in hun
afkomst of opleiding, maar altijd in hun persoonlijkheid. Ook in moderne
verslagen van deze messiaanse figuren die de armen vertegenwoordigden,
staat vaak hun welbespraaktheid, hun indrukwekkende uitstraling én hun
magnetische persoonlijkheid centraal. Men krijgt zo de indruk dat, zelfs
als sommigen zich bewust voordeden als bedriegers, de meesten zichzelf
werkelijk zagen als belichaamde goden. En die overtuiging werkte
moeiteloos over op menigten waarvan het diepste verlangen juist uitging
naar een eschatologische redder. \footnote{Ibid., pp.~84--85.}
\end{quote}

Hoewel deze passage op bewonderenswaardige wijze de vermeende
millenaristische redders beknopt schetst -- zij die de middeleeuwse
samenleving zo vaak in beroering brachten -- slaagt zij er niet in de
volledige reikwijdte van Cohns gezaghebbende overzicht weer te geven.
Wie het gehele werk leest, kan niet anders dan in de capriolen van deze
profeten de vertrouwde kenmerken van de moderne demagoog opmerken:
welbespraaktheid, `persoonlijk magnetisme', messiaanse pretenties en het
terugkerende verlangen om als vertegenwoordiger van de armen aanbeden te
worden.

Het grootste verschil tussen hoe de middeleeuwse samenleving met deze
bedriegers omging en hoe de democratie hen aan het einde van de
twintigste eeuw benaderde, bestaat eruit dat men hen in de middeleeuwen
vrijwel altijd ter dood veroordeelde, terwijl de hedendaagse
democratische politiek via een open kanaal legitieme kansen biedt om de
macht in de natiestaat te grijpen.

Als een systeem routinematig de controle over de grootste en meest
dodelijke ondernemingen wereldwijd overdraagt aan de winnaar van
populariteitswedstrijden tussen charismatische demagogen, zal het op den
duur onvermijdelijk de negatieve gevolgen ervan ervaren.

\subsection{Betaal leiders om goed werk te
leveren}\label{betaal-leiders-om-goed-werk-te-leveren}

Zoals hierboven al werd aangegeven, is het vrij eenvoudig om een betere
manier te vinden om talentvol leiderschap binnen een organisatie veilig
te stellen, namelijk door deze leiders daadwerkelijk aan te stellen.
Deze werkwijze wordt in concurrerende economieën op grote schaal en met
succes toegepast.

Een doordacht selectieproces, gecombineerd met een stimulerend
beloningssysteem voor positief leiderschap, zorgt ervoor dat bekwame
mensen de overheid gaan leiden. Het trekt bovendien nieuw talent aan dat
zich anders nauwelijks zou interesseren voor bestuurlijke vraagstukken.

De meest getalenteerde topfunctionarissen wereldwijd zouden
wanpresterende regeringen met enthousiasme overnemen als hun beloning
afhangt van de concrete resultaatsverbetering die zij voor de
samenleving realiseren.

Een leider die het reële inkomen in elk vooraanstaand Westers land fors
weet te verhogen, verdient naar recht veel meer dan Michael Eisner. In
een betere wereld is iedere succesvolle regeringsleider multimiljonair.

\subsection{Elektronische
plebiscieten}\label{elektronische-plebiscieten}

Een andere voor de hand liggende oplossing voor representatief
wanbestuur is het inzetten van elektronische plebiscieten, waarbij
burgers -- bijvoorbeeld een representatieve groep die via een
manipulatiebestendige loting wordt samengesteld -- rechtstreeks stemmen
op wetgevende voorstellen. Dankzij computertechnologie kunnen we op een
democratische wijze besluiten vaststellen via deze plebiscieten. Je kunt
ze bovendien eenvoudig combineren met een toewijzingssysteem, zodat per
specifiek vraagstuk slechts een beperkt aantal stemmers meedoet. Hoe dan
ook, voor potentiële kiezers is het veel eenvoudiger om politieke
vraagstukken te doorgronden dan om te proberen politici te ontcijferen
en hun standpunten over dezelfde kwesties te beoordelen, om nog maar te
spreken van het werkelijk weten wat die politici gaan doen zodra zij de
macht in handen hebben. Dit wordt alleen maar ingewikkelder nu politici
en hun medewerkers steeds bedrevener worden in het verpakken en
manipuleren van de beelden die ze aan het publiek presenteren.

\section{Gecommercialiseerde
soevereiniteit}\label{gecommercialiseerde-soevereiniteit}

Wij voorzien dat er een nieuw systeem opstaat dat de traditionele
politiek vervangt. Hoewel we in de eerder genoemde opties elk wel wat
voordelen zien, verwachten we niet dat hervormingen de politiek nieuw
leven inblazen. Integendeel, zij raakt verouderd en verliest op veel
vlakken haar relevantie. We bedoelen hiermee niet dat wij een dictatuur
voor ogen hebben, maar dat de overheid een ondernemende, commerciële
vorm aanneemt -- een commercialisering van soevereiniteit.

In tegenstelling tot een dictatuur of zelfs democratie sluit
gecommercialiseerde soevereiniteit de individuele keuzevrijheid niet
uit. Integendeel, zij geeft ieder mens meer ruimte om zijn ideeën te
uiten. Bovendien biedt dit systeem diegenen die er hun voordeel mee
weten te doen een praktischere omgeving voor besluitvorming en
zelfbepaling dan alle andere vormen van sociale organisatie die we tot
nu toe kenden.

\subsection{Op maat gemaakte overheid}\label{op-maat-gemaakte-overheid}

Om dit niet als een ouderwets ideaal te laten klinken, bedenkt u dat
microtechnologie objecten miniaturiseert en uit elkaar haalt. Dankzij
deze technologische vooruitgang kunnen we kiezen voor maatwerk in plaats
van massaproductie. U kunt vandaag een winkel binnenlopen en een
spijkerbroek kopen die wordt gesneden volgens een patroon dat perfect op
uw maten is afgestemd en halverwege de wereld genaaid wordt. Zodra
nieuwe instituties zich ontwikkelen om in te spelen op de megapolitieke
realiteiten van het informatietijdperk, krijgt u een bestuur dat net zo
nauw aansluit op uw persoonlijke wensen en voorkeuren als een perfect
zittende spijkerbroek.

Alvin Toffler, van alle mensen, heeft kritiek geleverd op het idee dat
informatietechnologie burgers in klanten kan veranderen. Volgens Toffler
geeft dit model een te beperkte kijk: `Of we het nu willen of niet, er
bestaat een wereld van religie en gevoel die niet zomaar te reduceren
valt tot contractuele relaties.'\footnote{Geciteerd in Kelly, \emph{op.
  cit.}, p.~46.} We hebben eerder betoogd dat `de wereld van
nationalistisch gevoel' niet eenvoudig te reduceren is tot louter
`contractuele relaties.' Dit betekent echter niet dat zo'n benadering
onhaalbaar of onwenselijk is. Een minder irrationeel enthousiasme voor
nationalisme kan immers miljoenen levens redden.

\subsection{Toetreding, uittreding en
inspraak}\label{toetreding-uittreding-en-inspraak}

Voor velen blijft de commercialisering van soevereiniteit een vreemd
concept -- kennelijk zelfs voor Alvin Toffler. Toch drukt het
economische perspectief op soevereiniteit zich dagelijks uit in het
leven aan het einde van de twintigste eeuw. In elke enigszins vrije
economie kunt u uw wensen direct kenbaar maken door diensten en
producten te kopen of juist af te zien van aankopen. Als u ontevreden
bent over de uitvoering van een product of de service van een aanbieder,
toont u uw ongenoegen direct via `exit' -- kortom, u verlegt uw
klandizie naar een andere leverancier.

In de voorgaande hoofdstukken hebben we onderzocht hoe technologische
vooruitgang het binnenkort mogelijk maakt om in cyberspace activa te
creëren die vrijwel immuun zijn voor roofzuchtige inmenging door
natiestaten. Dit leidt tot een \emph{de facto} metaconstitutionele
voorwaarde: overheden moeten u eerst daadwerkelijk bevredigende
dienstverlening leveren voordat u hun rekeningen betaalt. Waarom? Omdat
het betalen van de inkomstenbelasting in de praktijk vrijwel net zo
vrijwillig zal verlopen als men in de theorie aanneemt.

\subsection{Het vermijden van logge politieke
kanalen}\label{het-vermijden-van-logge-politieke-kanalen}

In feite zorgt de verwachte ontwikkeling van de informatietechnologie
ervoor dat klanten overheden daadwerkelijk gaan controleren. Als klant
beschik je al snel over honderden -- later zelfs duizenden --
mogelijkheden om direct je beschermingskosten te verlagen, bijvoorbeeld
door een particuliere belastingafspraak te sluiten met een natiestaat of
door je volledig af te keren van natiestaten en over te stappen op
opkomende mini‑soevereiniteiten. Deze contractuele `toetredings'- en
`exit'-opties weerspiegelen in economische termen jouw wensen als klant.
Stemmen met je voeten en met je geld levert als groot voordeel op dat je
exact de gewenste resultaten behaalt.

Hoe verhouden jouw `toetredings'- en `exit'-opties als klant zich tot de
politieke manier van je mening geven in een democratie? Mensen die
ontevreden zijn over een product of dienst -- vooral als het gaat om
iets dat door de overheid geleverd of streng gereguleerd wordt -- laten
hun stem vaak horen door brieven te sturen naar de president van de
Verenigde Staten of door een afspraak te maken met hun parlementslid of
een andere geschikte gekozen functionaris. Soms hebben zulke
liefdesbrieven effect, maar meestal niet. Als die pogingen aanvankelijk
niet slagen, organiseren mensen die verandering wensen een demonstratie,
adverteren ze een volledige pagina in een krant of proberen ze zelfs
zelf een politieke functie te bekleden.

Hoewel de politieke vorm een platform biedt voor welsprekende
verklaringen en retoriek, heb je er zelden direct voordeel van en
verbeter je je positie er niet mee. Wanneer je te maken krijgt met een
inferieur product of een overheidsdienst, blijf je immers de kosten
betalen totdat je het hele politieke systeem ervan weet te overtuigen in
te stemmen met jouw verzoek tot verandering.

In westerse landen -- en tegenwoordig wereldwijd -- betekent dit dat men
de steun van de meerderheid voor een democratisch systeem moet
waarborgen. Het betrekken van een meerderheid brengt enorme
transactiekosten met zich mee tussen jou en het bereiken van wat
hoogstwaarschijnlijk een relatief eenvoudig en rationeel doel is.

Milton Friedman besprak in \emph{Capitalism and Freedom} de verdiensten
van een economische in plaats van een politieke manier van uiten bij het
formuleren van zijn voorstel voor schoolvouchers.

Ouders zouden veel directer hun mening over scholen kenbaar kunnen maken
door hun kinderen van school te halen en ze naar een andere school over
te plaatsen, in tegenstelling tot nu, waar verhuizen vaak de enige
uitweg is. Momenteel biedt verhuizen immers de enige mogelijkheid om een
standpunt te laten gelden en moeten zij hun ongenoegen anderszins via
omslachtige politieke routes verwoorden.\footnote{Milton Friedman,
  \emph{Capitalism and Freedom} (Chicago: University of Chicago Press,
  1962), p.~91. Besproken door Hirschman, \emph{op. cit.}, pp.~16--17.}

Albert O. Hirschman, een voorvechter van politieke betrokkenheid, sprak
zijn kritiek uit op Friedman's pleidooi voor `exit' als directe manier
om ontevredenheid over een organisatie te uiten. Wie niet vertrouwd is
met economische theorieën, zou gemakkelijk naïef denken dat het
simpelweg kenbaar maken van zijn mening voldoende is.\footnote{Hirschman,
  \emph{op. cit.}, p.~17.}

Of het effectiever is om als consument via marktmechanismen --
bijvoorbeeld door wel of niet zijn steun te verlenen -- zijn mening te
geven, of via de ingewikkelde gang van zaken in de politiek, blijft een
complexe en omstreden kwestie. Meningen hierover lopen uiteen. Voor
mensen die hun politieke invloed vooral uitoefenen door voordelen op de
kosten van anderen af te dwingen, kan de overstap naar een economische
vorm van uitdrukken als een trieste plaatsvervanger overkomen,
vergeleken met het schrijven van een brief aan een politicus om meer
eisen te stellen.

Economische expressie en `wederkerige socialiteit'

Wie ernaar streeft medemensen te betrekken in een wederkerige, in plaats
van dwangmatige of parasitaire, socialiteit ontdekt in de economische
manier van uiten de mogelijkheid om met minder inspanning veel meer
voldoening te behalen. Wat professor Hirschfield ervan beweert, valt
gemakkelijk aan te tonen.

Elke reeks economische handelingen -- zoals toetredingen, lopende
contracten en uittredingen -- kan als een politieke `stem' worden
ingezet door simpelweg een grote groep mensen bij het
besluitvormingsproces te betrekken. Probeer het eens als experiment.
Voor zo'n experiment heb je slechts een paar honderd mensen nodig die
vinden dat er te weinig politiek in hun leven is. In plaats van hun
jaarlijkse beschikbare inkomen aan duizenden losse aankopen te besteden,
zetten ze deze overvloed aan economische besluiten om in een handvol
politieke beslissingen.

Aanvankelijk stemt iedereen ermee in om zijn beschikbare inkomen samen
te voegen en af te zien van individuele aankopen. In plaats van dat
iedereen duizenden dollars op talloze manieren persoonlijk uitgeeft,
krijgt iedere persoon één stem -- of wellicht meerdere stemmen,
afhankelijk van het aantal te vervullen functies. In plaats van direct
geld te verbrassen om op elk gewenst moment precies te krijgen wat men
wil, brengt men zijn stem in op de weinige momenten waarop verkiezingen
plaatsvinden. De gekozen vertegenwoordigers bepalen vervolgens hoe de
inmiddels enorme gezamenlijke geldpot wordt besteed.

Vervolgens neem je, samen met de anderen, deel aan de consumptie van
precies die goederen die door de besturende commissie namens de
meerderheid zijn goedgekeurd.

Lijkt dat al op een `omslachtige politieke uitlaatklep' voor expressie?
Wacht maar. Dit model bezit de potentie om net zo welsprekend en
overtuigend te zijn als de nationale politiek, en vangt tevens een
aanzienlijk deel van de opgebouwde frustratie op.

Stel bijvoorbeeld dat je dol bent op verse broccoli, maar de groep qua
smaakvoorkeuren gemiddeld verdeeld is -- dan zit je waarschijnlijk in de
problemen. De kans is groot dat enkelen, of zelfs de meerderheid, liever
een groter deel van het gezamenlijke voedingsbudget aan rood vlees
besteden dan aan verse verse groenten. Om te voorkomen dat de
kantinecommissie naar een magazijnwinkel trekt en het hele jaarlijkse
groentebudget verkwist op ingeblikte erwten en maïs, moet je wellicht
opstaan en je mening kenbaar maken. Je kunt de aandacht van de groep
vestigen op de voordelen van het consumeren van meer vitamines en
fytonutriënten -- zoals het sulforaphaan in broccoli -- tegenover de
verzadigde vetten en het cholesterol in rood vlees.

Hoe je dit of een ander punt ook maar duidelijk maakt, blijft in dit
geconstrueerde politieke model even een raadsel als voor de voorstanders
van welke politieke zaak of kandidaat dan ook. Je kunt een toespraak
houden, maar daarvoor moet een aanzienlijk deel van je doelgroep zich al
hebben verzameld en bereid zijn te luisteren. Je kunt ook flyers laten
drukken, mits zo'n `campagne-uitgave' volgens de huisregels van jouw
politieke spel is toegestaan. Of je schrijft brieven. Beide opties
hangen ervan af dat de ontvangers voldoende leesvaardigheid bezitten.

\begin{quote}
`Het schetst een beeld van een samenleving waarin de overgrote
meerderheid van de Amerikanen zich er niet van bewust is dat zij niet
beschikken over de vaardigheden die nodig zijn om in onze steeds
technologischer wordende samenleving en op de internationale marktplaats
in hun levensonderhoud te voorzien.'
\end{quote}

\begin{quote}
Richard Riley; VS-secretaris van onderwijs, in \emph{`Adult Literacy in
America'}
\end{quote}

\subsection{Negentig miljoen
Alzheimerpatiënten?}\label{negentig-miljoen-alzheimerpatiuxebnten}

Als de groep waarmee je dit modelpolitieke experiment uitvoert toevallig
uitsluitend uit Amerikanen bestaat, wordt het buitengewoon lastig om een
overtuigende boodschap over te brengen, zeker wanneer de groepsleden
representatief zijn voor het gehele Amerikaanse electoraat. Het idee dat
een onevenredig groot deel van de burgers van 's werelds machtigste
natiestaat onderpresteerders zou zijn, wordt somber bevestigd door het
meest grondige onderzoek naar de competentie van Amerikaanse
volwassenen. De studie \emph{`Adult Literacy in America'} laat zien dat
het vinden van een geletterd publiek voor elk politiek betoog verre van
eenvoudig is. Een aanzienlijk deel, misschien zelfs de meerderheid, van
Amerikanen van vijftien jaar en ouder beschikt niet over de
basisvaardigheden die nodig zijn om ideeën te evalueren en onderbouwde
oordelen te vormen. Volgens het Amerikaanse Ministerie van Onderwijs
kunnen 90 miljoen Amerikanen geen brief schrijven, een bustijdschema
doorgronden of zelfs simpele rekensommen uitvoeren. Dat is precies wat
je zou verwachten als 90 miljoen Amerikanen de diverse stadia van de
ziekte van Alzheimer doormaakten. Bij dertig miljoen bleek men zo
onkundig dat zij zelfs niet in staat waren vragen te beantwoorden.

Mocht je gezondheidsboodschap het tij niet hebben kunnen keren --
terwijl deze zich anders vanzelf op een bepaald niveau zou stabiliseren
-- dan kun je hulp inschakelen van dierenrechtenactivisten. Misschien
krijg je hen zover om je tegenstanders te laten picketten in de
kantinecommissie of opschudding te zaaien over het kwaad van het doden
van koeien bij de huizen van invloedrijke personen.

We zouden dit voorbeeld oneindig kunnen uitwerken -- veel verder dan het
geduld van rationele mensen toelaat. Het laat namelijk twee zaken zien:
(1) dat elke economische handeling -- of het nu een toetreding of een
uittreding betreft -- kan worden omgevormd tot een politieke stemming
wanneer men er een collectieve beslissing van maakt en (2) dat
dergelijke collectieve besluiten, ondanks de ruimte voor welsprekendheid
die zij bieden, juist buitengewoon traag en vaak onbuigzaam zijn.

Precies dit bevestigt de praktijk. Het is allesbehalve eenvoudig de
benodigde inzet te organiseren om de werking van een democratie
fundamenteel te veranderen. Om het nogmaals te benadrukken: dit zou wel
eens de reden kunnen zijn dat democratische verzorgingsstaten eeuwenlang
konden concurreren met alternatieve bestuursvormen en aan het einde van
het industriële tijdperk de overhand kregen. Democratie slaagde als
politiek systeem juist doordat haar werking het voor burgers bijzonder
lastig maakte om de staat effectief te controleren of haar aanspraken op
middelen te beperken.

Maar nu de onbeperkte inmenging van de staat in jouw zaken in het
informatietijdperk geen militair voordeel meer oplevert, vinden
vindingrijke mensen ongetwijfeld betere manieren om de weinige,
waardevolle diensten die overheden daadwerkelijk leveren, te verkrijgen.
Het is waarschijnlijk dat de daadwerkelijke macht verschuift naar
collectieve mechanismen die zichzelf uiteindelijk niet meer kunnen
bekostigen. Wij verwachten dat efficiëntie de overhand krijgt boven
geconcentreerde macht. Zoals Neil Munro beknopt verwoordde: ``{[}H{]}et
is geautomatiseerde informatie, niet mankracht of massaproductie, die de
Amerikaanse economie steeds meer aandrijft en die oorlogen zal winnen in
een wereld die is toegerust op 500 tv-zenders. De geautomatiseerde
informatie bestaat in cyberspace -- de nieuwe dimensie gecreëerd door de
eindeloze reproductie van computernetwerken, satellieten, modems,
databanken en het publieke internet.'' \footnote{Neil Munro, `De nieuwe
  nachtmerrie van het Pentagon: een elektronische Pearl Harbor,'
  \emph{Washington Post}, 16 juli 1995, p.~C3.}

Grote legers zullen in zo'n wereld nauwelijks nog een rol spelen.
Efficiëntie krijgt daarbij een steeds grotere betekenis. Dankzij
microtechnologie, die ons een geheel nieuwe dimensie van bescherming
biedt -- zoals we in hoofdstuk 6 en elders al bespraken -- krijgen
individuen voor het eerst in de geschiedenis de mogelijkheid om activa
te creëren en te beveiligen die volledig buiten het bereik vallen van
het territoriale geweldmonopolie van welke overheid dan ook. Deze activa
kunnen dan namelijk optimaal door individuen worden beheerd.

Het lijkt volkomen logisch dat jij en vele toekomstige soevereine mensen
met `je voeten stemmen' door je af te melden bij dominante natiestaten
en in plaats daarvan een persoonlijke beschermingsdienst af te nemen bij
een perifere natiestaat of een nieuwe mini-soevereiniteit, die slechts
een commercieel betaalbaar tarief hanteert in plaats van het merendeel
van je vermogen opeist. Kortom, je zou waarschijnlijk instemmen met een
aanbod van \$50 miljoen om naar Bermuda te verhuizen.

Eerst uitstappen, later contracteren.

De eerste impuls voor de commercialisering van soevereiniteit moet komen
van mensen die door te vertrekken hun economische voorkeur kenbaar
maken. In de Verenigde Staten blijkt deze optie het moeilijkst
realiseerbaar, terwijl ze daar tegelijkertijd het meest waardevol is. De
`Berlijnse Muur' voor kapitalisten, ingesteld door president Bill
Clinton en het Republikeinse Congres, staat haaks op de slogan `hou
ervan of verlaat het', die in de jaren zestig vol vertrouwen door
Amerikaanse nationalisten werd uitgedragen. Door strafbelastingen te
heffen op degenen die vertrekken, wil de exitbelasting loyaliteit
afdwingen. Niettemin kan deze wraakzuchtige wetgeving -- die doet denken
aan de straffen die in de laatste dagen van het Romeinse Rijk op
vluchttende landeigenaars werden toegepast -- onbedoeld de basis vormen
voor een rationeler beleid in het latere informatietijdperk.

Op een gegeven moment, zodra voldoende bekwame mensen vertrokken zijn en
grote fortuinen offshore hebben opgebouwd, wordt het voor de Amerikaanse
autoriteiten aantrekkelijk om burgers of groenekaarthouders toe te staan
zich uit toekomstige belastingverplichtingen uit te kopen door éénmalig
een exitbelasting te betalen, zonder daadwerkelijk te vertrekken. Met
andere woorden, de exitbelasting kan wel eens als model dienen voor een
eenmalige afkoop. Een overheid die een exitbelasting heft, profiteert
immers aanzienlijk als zij toestaat dat vertrekkers hun verblijf later
hervatten op basis van een privaat verdrag, zoals dat tegenwoordig in
Zwitserland en elders mogelijk is.

Dergelijke maatregelen van de Verenigde Staten of andere regeringen
vormen rationele, inkomensoptimaliserende initiatieven. Uiteindelijk
dwingt de concurrentie op het gebied van beschermingsdiensten de
belastingtarieven omlaag en passen de voorwaarden voor belastingheffing
zich aan meer beschaafde normen aan. In plaats van te rekenen op
wetgevende lichamen om acceptabele belastingregimes in te voeren, zullen
toekomstige soevereine individuen via privaat verdrag in staat zijn om
op maat gemaakte, aanvaardbare beleidsvoorwaarden te bedingen.

\section{Het beledigen van de ware
gelovigen}\label{het-beledigen-van-de-ware-gelovigen}

Natuurlijk beweren we niet dat veel van dit beleid in de smaak zal
vallen. De denationalisering van het individu en de daarmee gepaard
gaande commercialisering van soevereiniteit zullen de achterblijvende
ware gelovigen -- zij die nog vasthouden aan de clichés uit de politiek
van de twintigste eeuw -- diep beledigen. Net als de inmiddels overleden
Christopher Lasch zien zij het verval van de politiek als een bedreiging
voor het welzijn van de meerderheid. Volgens hen zou een heropleving van
de politiek uit het industriële tijdperk, met de nadruk op
inkomensherverdeling, een oplossing kunnen bieden voor de problemen die
velen ondervinden door de competitieve druk van de
informatietechnologie.

E. J. Dionne, Jr.~is een politiek verslaggever voor de \emph{Washington
Post}. Net als Lasch kijkt hij met nostalgie terug op de politiek.
Daarnaast belichaamt hij een sociaal-democratische, egalitaire impuls
die de komende decennia steeds luider zal klinken, nu de nieuwe
megapolitieke realiteiten van het informatietijdperk de overgebleven
instituties in de moderne wereld steeds krachtiger ondermijnen. Dionne
wijt de brede materiële vooruitgang in levensstandaarden, die in de
twintigste eeuw in rijke rechtsgebieden wijdverspreid was, vooral aan
democratische politiek en niet aan technologische of economische
ontwikkelingen. Hij stelt dat hoop voor de toekomst inhoudt dat het
politieke gezag zich ook moet uitstrekken tot de technologieën van het
informatietijdperk:

\begin{quote}
De grootste noodzaak in de Verenigde Staten en in de hele democratische
wereld is een hernieuwde inzet op democratische hervorming, de politieke
drijfveer die het industriële tijdperk zo succesvol maakte. De
technologieën van het informatietijdperk bouwen op zichzelf geen
succesvolle samenleving op, net zoals het industrialisme de wereld,
wanneer het aan zichzelf wordt overgelaten, niet beter heeft gemaakt.
\ldots{} Zelfs de meest baanbrekende technologische doorbraken en de
meest briljante toepassingen van het internet gaan ons niet behoeden
voor sociale ontwrichting, criminaliteit of onrecht. Alleen de politiek
-- de kunst om ons te organiseren -- kan überhaupt beginnen dergelijke
taken op zich te nemen.\footnote{E. J. Dionne, `Waarom rechts het mis
  is,' \emph{Utne Reader}, juni 1996, p.~32.}
\end{quote}

Dionne en anderen zoals hij snappen niet dat de omstandigheden die het
leven in de twintigste eeuw zo bevorderlijk maakten voor systematische
dwang, niet het resultaat waren van bewuste menselijke keuzes. De
uitdrukking `de kunst van hoe wij ons organiseren' zou in vroegere
tijden nauwelijks te begrijpen zijn geweest. Samenlevingen zijn
simpelweg te complex om als het resultaat van een doelbewuste inspanning
tot zelforganisatie te worden gezien. De moderne natiestaten zijn
spontaan ontstaan als een toevallig bijproduct van de industriële
technologie, die de opbrengsten van geweld vergrootte. Tegenwoordig
verlaagt informatietechnologie die opbrengsten. Dit maakt de politiek
anachronistisch en onherstelbaar, hoe oprecht men er ook voor pleit haar
te behouden voor het volgende millennium.

\begin{quote}
'Niet van vandaag, noch van gisteren hetzelfde

Zij leven door alle tijden; en waar zij vandaan kwamen

Dat weet niemand.' - SOPHOCLES, Antigone
\end{quote}

\section{`Ze maken ze niet zoals
vroeger'}\label{ze-maken-ze-niet-zoals-vroeger}

De vurige drang om `wetten te maken' -- een vanzelfsprekend onderdeel
van het gezond verstand in de politiek van de twintigste eeuw -- is
geenszins universeel in alle culturen. Wanneer deze gewoonte in de
toekomst verdwijnt, kan dat wijzen op een cyclus die door de eeuwen heen
op en neer heeft gewogen. Zo meenden de oude Grieken dat wetten niet
door mensen gemaakt konden worden. Zoals de filosoof Ernst Cassirer
opmerkt, geloofden de Grieken namelijk dat `de ongeschreven wetten, de
wetten van de rechtvaardigheid, geen begin in de tijd
hebben.'\footnote{Ernst Cassirer, \emph{The Myth of the State} (New
  Haven: Yale University Press, 1946), p.~81.} Net als andere
pre-politieke volkeren waren zij ervan overtuigd dat niemand de
natuurlijke, `geometrische' wetten van de rechtvaardigheid kon
verbeteren -- wetten die niet door een menselijke macht waren gecreëerd.

Zij geloofden niet in een `wetgever.' Zoals Cassirer het verwoordde:
`Het is door rationeel denken dat wij de normen voor moreel gedrag
vinden, en het is enkel de rede, en alleen de rede, die hen hun
autoriteit kan verlenen.' Elke poging om wetten via wetgeving aan de
samenleving op te leggen, lijkt in die zin even onzinnig als het
proberen de geometrie via wetgeving te veranderen.

\subsection{Wetgeving als
heiligschennis}\label{wetgeving-als-heiligschennis}

Om uiteenlopende redenen bestond gedurende een groot deel van de
middeleeuwen een duidelijke weerstand tegen het `wetten maken.' John~B.
Morrall schrijft: `{[}V{]}oor de Germanen bestond de wet al van
oudsher.' Het vormde een garantie voor de rechten van de individuele
stamleden.\footnote{John B. Morrall, \emph{Politiek denken in de
  middeleeuwen} (New York: Harper Torchbooks, 1962), p.~15.}\\
Koningen en raden

\begin{quote}
hadden vooralsnog niet de intentie om nieuwe wetten te maken. Zo'n
voornemen zou in die vroege middeleeuwse tijden niet alleen overbodig
zijn geweest, maar zelfs enigszins godslasterlijk, want zowel de wet als
het koningschap straalden een heiligheid uit. In plaats daarvan zagen
koning en raadslieden zichzelf louter als uitleggers of verduidelijkers
van de ware betekenis van het reeds bestaande en volledige
wetgevingssysteem. De Germaanse traditie drukte een idee op de
middeleeuwse geest dat men nooit vergat, zelfs als de praktijk er anders
uitzag: goede wetten werden herontdekt of herformuleerd, maar nooit
geheel opnieuw gemaakt.\footnote{Ibid., p.~16.}
\end{quote}

Na alle excessen van de regelgeving in de twintigste eeuw krijgt die
oude houding een zekere, ouderwetse charme. Het verlangen om de staat
haar dwingende macht voor privébelangen in te zetten -- met name voor de
herverdeling van inkomsten -- leek bijna vanzelfsprekend.

\subsection{Regrets}\label{regrets}

Het verbaast dan ook niet dat er verdrietige liederen klinken over de
laatste dagen van de politiek. Ze zijn volledig voorspelbaar, en dat
niet alleen omdat ze de blindheid van de meeste denkers voor de eisen
van megapolitiek blootleggen. Weinig politieke verslaggevers, zoals
Dionne, accepteren de schijnbare verdroging en ondergang van de
politiek, ook al brengt dat hen weer in de nabijheid van misdaadzaken.
Aan het einde van de middeleeuwen riepen mensen op tot een heropleving
van de ridderlijkheid. Denk bijvoorbeeld aan \emph{Il Libro del
Cortegiano} oftewel \emph{The Book of the Courtier}, dat graaf
Baldassare Castiglione in 1514 schreef en in 1528 in Venetië door Aldus
uitgaf.

Castiglione voelde een diep verlangen naar de heropleving van
ridderlijke deugden, maar zo'n sentiment kon in de zestiende eeuw het
verleden dat uitsterfde niet weer tot leven brengen. Evenmin zal dat in
de eenentwintigste eeuw gebeuren.

Zoals wij in onze uitleg van de theorie van megapolitiek probeerden over
te brengen, vormen technologische drijfveren -- en niet de publieke
opinie -- de voornaamste bronnen van verandering. Als onze theorie van
megapolitiek klopt, komt de vervanging van het feodale systeem en de
ridderlijkheid -- die steunden op persoonlijke eden en relaties -- door
het moderne tijdperk, met zijn burgerschapsconcept en staatgerichte
politiek, niet voort uit ideeën. De cruciale drijfveer zijn de
verschuivingen in kosten en baten die nieuwe technologie met zich
meebrengt.

Ridderlijkheid verdween niet omdat Castiglione of anderen er niet in
slaagden een onverschillige bevolking -- die bovendien geen invloed had
op deze zaak -- ervan te overtuigen dat eer en moraliteit in de politiek
overbodig waren.

Integendeel, in \emph{Courtier} bekritiseert Castiglione vorsten en het
gedrag dat zijn tijdgenoot Niccolo Machiavelli in zijn \emph{Il
Principe} (oftewel \emph{The Prince}) juist toejuichtte.

Maar wat gebeurde er daarna? Machiavelli bereikte uiteindelijk een veel
breder publiek met zijn boek, niet doordat zijn betoog in \emph{The
Prince} overtuigender was, maar omdat zijn adviezen beter aansloten bij
het grootschalige politieke landschap van het moderne tijdperk.

Zoals de invloedrijke twintigste-eeuwse filosoof Ernst Cassirer opmerkte
in zijn bespreking van `het morele probleem in Machiavelli',

\begin{quote}
Het boek beschrijft, met volledige onverschilligheid, de manieren en
middelen waarmee politieke macht verkregen en behouden dient te worden.
Over het juiste gebruik van deze macht zegt het geen woord. \ldots{}
Niemand had ooit getwijfeld dat het politieke leven, zoals het ervoor
stond, vol misdaden, verraderij en zware misdrijven zat. Maar geen
enkele denker vóór Machiavelli had ooit de kunst van deze misdaden
onderwezen. Deze daden werden wel verricht, maar zij werden niet
onderwezen. Het feit dat Machiavelli beloofde een leermeester te worden
in de kunst van sluwheid, verraderij en wreedheid, was ongehoord.
\end{quote}

Kortom, \emph{The Prince} presenteerde een radicaal recept waarmee een
aspirant-heerser zijn carrière ten koste van anderen kon bevorderen.
Machiavelli keurde gedragingen goed die perfect aansloten op de harde
werkelijkheid van een machtsgetrokken tijdperk. Tegelijkertijd bleek het
dubbelspel -- dat moderne politici als een slimme strategie waarderen --
ronduit schandalig en ondermijnde het de in voorgaande eeuwen opgebouwde
cultuurnormen van ridderlijkheid.

Zoals we eerder zagen, draaiden de deugden van de ridderlijkheid er
vooral om dat iedereen zich strikt aan zijn eden hield. Dat vereiste men
in een samenleving waar bescherming werd gegarandeerd in ruil voor
persoonlijke diensten. De afspraken waarop het feodale systeem was
gebouwd, zouden niet vanzelf herleven als mensen alleen hun eigen
belangen nastreefden. Daarom moesten de feodale verplichtingen -- de
ruggengraat van de ridderlijkheid -- altijd steunen op een sterk gevoel
van eer. In dat opzicht was Machiavelli's advies, dat een vorst niet zou
moeten aarzelen te liegen, bedriegen en stelen wanneer dat hem in zijn
voordeel werkte, ronduit subversief.

Toen de twintigste eeuw ten einde kwam, bleven Machiavelli's argumenten
in de belangstelling als middel om de moderne politiek en de
uiteenlopende misdaden en tirannieën van die tijd te doorgronden. In
tegenstelling daarmee raakte het werk van Castiglione vrijwel in de
vergetelheid. Over een jaar lezen waarschijnlijk slechts een handjevol
masterstudenten literatuur en enkele kenners van de
etiquettegeschiedenis \emph{Il Libro del Cortegiano} van kaft tot kaft.

Binnen enkele decennia raakt de megapolitiek van het informatietijdperk,
zoals vervat in \emph{De vorst}, duidelijk achterhaald. Het
onafhankelijke individu zal een nieuw succesrecept moeten hanteren --
eentje die sterk inzet op eer en integriteit bij het inzetten van
middelen buiten de greep van de staat. We verwachten dat E. J. Dionne,
Jr.~en de overige nog levende sociaaldemocraten zo'n advies met weinig
enthousiasme zullen oppikken.

\subsection{Door klanten vastgesteld
beleid}\label{door-klanten-vastgesteld-beleid}

Dit geldt vooral in de beginfase van de transitie, wanneer de meeste
rechtsgebieden nog worstelen met de verplichting om een beleid te
ontwikkelen dat de meerderheid van de bevolking weet te bekoren. Later,
wanneer de democratie verzwakt en de markt voor soevereiniteitsdiensten
zich verder ontwikkelt, begrijpen we beter welke marktomstandigheden het
`beleid' beperken.

Wat we tegenwoordig onder `politiek leiderschap' verstaan -- dat we
altijd binnen een natiestaat plaatsen -- wordt steeds meer ondernemend
in plaats van louter politiek. In zulke omstandigheden krimpt het
praktische keuzepalet voor het samenstellen van een `beleidsregime' voor
een rechtsgebied, vergelijkbaar met de beperkte opties waarover
ondernemers beschikken bij het ontwerpen van een eersteklas resorthotel
of een vergelijkbaar product of dienst, bepaald door wat mensen ervoor
willen betalen. Een resorthotel probeert bijvoorbeeld zelden te opereren
onder voorwaarden waarbij van de gasten wordt verlangd dat zij zwaar
lichamelijk werk verrichten om de faciliteiten te repareren en uit te
breiden. Zelfs een resorthotel dat eigendom is van of door zijn
werknemers wordt bestuurd -- vergelijkbaar met de typische moderne
democratie -- zal tevergeefs proberen klanten aan dergelijke eisen te
onderwerpen, vooral wanneer betere accommodaties beschikbaar komen. Als
klanten er de voorkeur aan geven te golfen in plaats van zwaar
lichamelijk werk te verrichten in de hete zon, biedt de markt op dat
vlak weinig ruimte om willekeurige alternatieven op te leggen. In
dergelijke situaties maken de huidige `politieke' kwesties plaats voor
ondernemende inzichten, terwijl rechtsgebieden actief zoeken naar
beleidsbundels die hun klanten aanspreken.

\subsection{Het verval van de
politiek}\label{het-verval-van-de-politiek}

Als dit eenmaal duidelijk is, past men de houdingen ingrijpend aan. In
rechtsgebieden waar bevoegdheden gedeeld worden, verwachten mensen niet
langer te moeten kiezen uit het aanbod aan wensvervullende beleidsopties
dat in de politieke debatten van de twintigste eeuw gangbaar was. Nu de
inkomens ongelijker verdeeld zijn dan in het industriële tijdperk,
stemmen rechtsgebieden steeds meer af op de behoeften van de klanten
wiens economische activiteiten de meeste waarde genereren en die de
grootste keuzevrijheid hebben bij hun bestedingen.

Onder zulke omstandigheden maakt het waarschijnlijk minder uit of
beleidsmaatregelen, die commercieel optimaal blijken voor een
rechtsgebied, ook aanslaan bij de `mediaankiezer' in een focusgroep.
Kortom, de commercialisering van soevereiniteit maakt het voor klanten
mogelijk om overheden effectiever te controleren. Hierdoor worden de
meningen van niet-klanten irrelevant -- of in ieder geval minder
belangrijk -- net zoals de opvattingen van \emph{`Big Mac'-eters} over
foie gras geen enkele invloed hebben op het succes van drie-sterren
Franse restaurants zoals L'Arpege in Parijs.

\section{`De verraad van de
democratie'}\label{de-verraad-van-de-democratie}

Net als de overleden Christopher Lasch klagen tegenstanders er niet
alleen op dat informatietechnologie banen vernietigt, maar stellen zij
ook dat deze technologie de democratie ondermijnt doordat mensen hun
middelen buiten het bereik van politieke dwang kunnen houden. Daarom
ervaren reactionairen in het nieuwe millennium de via
informatietechnologie mogelijk gemaakte financiële privacy als een
ernstige bedreiging. Zij zullen terugdeinzen bij de gedachte dat
inkomsten- en vermogensbelastingen daadwerkelijk op `vrijwillige
naleving' moeten berusten. Ook steunen zij nieuwe, zelfs ingrijpende
methoden om van iedereen die welvarend lijkt middelen af te rommelen,
zoals het invoeren van een `presumptieve belasting' en het letterlijk
gijzelen van vermogende personen.

\subsection{Gemeenschappelijk
eigendom}\label{gemeenschappelijk-eigendom}

Al tijdens het schrijven dringen er hints door over wat nog komen gaat.
De eerste aanwijzingen dat regeringen steeds minder in staat zijn
internationale markten te beheersen, stoten degenen tegen die ervan
uitgaan dat individuen van rechtswege als activa van de natiestaat
behoren. Zij eisen dan ook dat de staat haar middelen gebruikt om
burgers als waardevolle activa te beschouwen in plaats van als klanten.
Volgens de reactionairen moeten alle inkomens als opbrengsten van de
gemeenschap worden gezien, wat inhoudt dat deze volledig ter beschikking
van de staat zouden moeten staan.\footnote{Robert I Shapiro, `Volledig
  mis: nieuwe belastingregelingen kunnen niet tippen aan oude
  progressieve waarheden,' \emph{Washington Post}, 24 maart 1996, p.~C3,
  en Thomas L. Friedman, `Politiek in het tijdperk van NAFTA,' \emph{New
  York Times}, 7 april, p.~Eli.}

We hebben de argumenten van Lasch in \emph{Revolt of the Elites} en
\emph{Betrayal of Democracy} al besproken. Maar zijn tirade is niet de
enige stem ter ondersteuning van de natiestaat. De politicoloog van
\emph{Harvard University}, Michael Sandel, betoogt in \emph{Democracy in
Discontent} dat `democratie van vandaag niet mogelijk is zonder een
politiek die de mondiale economische krachten kan beheersen, want zonder
die controle maakt het niet uit op wie mensen stemmen, de bedrijven
zullen regeren.'\footnote{Geciteerd door Friedman, \emph{op. cit.}} Met
andere woorden moet de staat haar parasitaire macht over burgers
behouden om te waarborgen dat politieke uitkomsten kunnen afwijken van
marktuitkomsten.

Volgens ons is Sandels klaagzang, net als die van Lasch, slechts half
correct. We geven toe dat de democratie veel van haar waarde verliest
wanneer regeringen niet de macht hebben om individuen te verplichten
zich te gedragen zoals politici dat eisen. Dit is vanzelfsprekend.
Inderdaad, de democratie zoals wij die in de negentiende en twintigste
eeuw kenden, zal verdwijnen. Maar Sandel slaat de ware betekenis van de
triomf van de markten boven de dwang mis. Zijn pleidooi voor `corporate
rule', gekoppeld aan het verval van de natiestaat, komt opvallend
anachronistisch over.

Bedrijven zullen nauwelijks in staat zijn de markten van de nieuwe
wereldeconomie te beheersen. Inderdaad, zoals we al suggereerden, is het
allesbehalve zeker dat bedrijven zelfs in hun hedendaagse, moderne vorm
blijven bestaan. Integendeel, zij zullen vrijwel onvermijdelijk
transformeren in de megapolitieke revolutie die de komst van het
informatietijdperk markeert. Zoals we eerder bespraken, zal de opkomst
van microprocessering de `informatiekosten' doen veranderen, wat mede
bepaalt wat het `nexus van contracten' vormt dat bedrijven kenmerkt.
Volgens economen Michael C. Jensen en William H. Meckling vormen
bedrijven slechts één juridische structuur die `een nexus voor een
geheel van contractuele relaties tussen individuen' biedt.\footnote{Zie
  Louis Putterman en Randall S. Kroszner, `The Economic Nature of the
  Firm: A New Introduction,' in Louis Putterman en Randall S. Kroszner
  (red.), \emph{The Economic Nature of the Firm: A Reader} (Cambridge:
  Cambridge University Press, 1996), p.~17.}

Of een onderneming het in zijn geheel kan overleven -- om nog maar te
zwijgen over het besturen ervan als een domein van bureaucratische
sturing dat afgeschermd is van marktkrachten -- hangt waarschijnlijk af
van `de volledigheid van de marktkrachten en het vermogen van die
krachten om door te dringen in intrafirmale relaties', zoals economen
Louis Putterman en Randall S. Kroszncr opmerken.\footnote{Ibid.}

Zoals we eerder betoogden, twijfelen we eraan of bedrijven de toenemende
penetratie van marktkrachten in wat voorheen `intrafirmale relaties'
waren, kunnen doorstaan. Hiervoor zullen bedrijven waarschijnlijk
uiteenvallen, aangezien informatietechnologie het aantrekkelijker maakt
om taken via het prijmechanisme en de veilingsmarkt te laten uitvoeren
in plaats van deze intern te organiseren binnen een formele structuur.
Naarmate informatietechnologie het productieproces steeds verder
automatiseert, zal dit een deel wegnemen van de \emph{raison d'être} van
het bedrijf, namelijk de noodzaak om managers in te zetten en
individuele werknemers te controleren en te motiveren.

\subsection{``Waarom bestaan
bedrijven?''}\label{waarom-bestaan-bedrijven}

Bedenk dat de vraag `Waarom bestaan bedrijven?' veel ingewikkelder is
dan op het eerste gezicht lijkt. De micro-economie veronderstelt immers
dat het prijmechanisme het meest doeltreffende middel is om middelen zo
te coördineren dat ze optimaal worden benut. Volgens Putterman en
Kroszner houdt dit in dat organisaties, zoals bedrijven, geen
fundamentele economische bestaansreden bezitten.\footnote{Ibid., p.~9.}
Op deze manier ontstaan bedrijven in wezen als gevolg van informatie- en
transactiekosten, kosten die door informatietechnologie drastisch worden
teruggedrongen.

Hierdoor zal het informatietijdperk waarschijnlijk het tijdperk worden
van zelfstandige ondernemers zonder vaste banen, maar met duurzame
ondernemingen. Doordat technologie de transactiekosten verlaagt, kunnen
individuen zich losmaken van de zeggenschap van politici en voorkomen we
tegelijkertijd dat bedrijven de markt gaan domineren. Ondernemingen gaan
wedijveren met `virtuele ondernemingen' van over de hele wereld, wat
vrijwel alle gevestigde organisaties -- met enkele uitzonderingen --
flink zal doen schrikken. De meeste ondernemingen zullen als
organisaties gelukkig overleven in een omgeving met toenemende
concurrentie, naarmate de markten completer worden.

De verwachte uitkomst is zeker niet dat individuen hun lot in de handen
van ondernemingen leggen. Integendeel, ondernemingen hebben evenmin meer
invloed op de markt dan politici. Integendeel, mensen zullen eindelijk
de vrijheid hebben hun eigen koers te varen in een echte vrije markt,
die noch door grote overheden, noch door bedrijfsstructuren wordt
aangestuurd.

Deze daling van transactiekosten weerlegt bovendien de recent populaire
ideeën over `stakeholderkapitalisme'. Zulke opvattingen, dierbaar bij
Tony~Blair van de Britse Labour Party en enkele aanhangers van
Bill~Clinton, berusten op het idee dat de staat bedrijven zou kunnen
sturen. Nu het socialisme ingestort is, proberen interventionisten de
doelen van dat systeem te bereiken via efficiëntere marktmiddelen, door
bedrijven streng te reguleren. Volgens deze nieuwe, herverdelende
theorie behoren het management, de aandeelhouders, de werknemers en de
`gemeenschap' allen tot de stakeholders van een onderneming. Het betoog
luidt dat zij immers allemaal profiteren van langdurige
bedrijfsstructuren en zelfs afhankelijk zijn van die voordelen. Daarom
zou regelgeving de belangen moeten beschermen die managers, werknemers
en lokale belastingautoriteiten koesteren met betrekking tot hun
traditionele banden met ondernemingen.

`Stakeholderkapitalisme' is een doctrine die uiteindelijk niet alleen
uitgaat van de veronderstelling dat de staat de besluitvorming binnen
ondernemingen kan beïnvloeden, maar in de kern berust op de overtuiging
dat ondernemingen als duurzame organisaties zelfstandig kunnen
functioneren, ook zonder de prikkels van prijssignalen op de
veilingmarkt.

Wij vermoeden dat het verder ontwikkelen van markten niet alleen de
belastingcapaciteit van de natiestaat zal verlagen, maar ook de
mogelijkheid voor politici beperkt om via regelgeving hun wil arbitrair
op te leggen aan de eigenaren van middelen. In een wereld waarin
jurisdictievoordelen onderhevig raken aan markttesten en lokale markten
overal aan de internationale concurrentie blootstaan, is het nauwelijks
te verwachten dat lokale `gemeenschappen' doeltreffende methoden vinden
om bevoorrechte ondernemingen te beschermen tegen mondiale
concurrentiedruk. Zij zullen nauwelijks in staat zijn ervoor te zorgen
dat ondernemingen, die extra kosten maken -- bijvoorbeeld voor het in
dienst houden van overbodig personeel en management of voor het
openhouden van overtollige faciliteiten ter tegemoetkoming aan lokale
politieke eisen -- op een wijze worden belast die deze kosten dekt en
hen in bedrijf houdt.

Tijdens het industriële tijdperk konden politici de markten afsluiten en
de toetreding beperken tot enkele bevoorrechte ondernemingen om te
voldoen aan werkgelegenheids- en andere doelstellingen. In de toekomst,
wanneer informatie overal ter wereld vrij verhandelbaar is, zullen
overheden nauwelijks de macht hebben om lokale bedrijven af te schermen
van de mondiale concurrentiedruk.

Ook is het onwaarschijnlijk dat oproepen tot een `nieuw sociaal
contract', waarbij een zogenoemde onafhankelijke of vrijwillige sector
de periode opvangt waarin arbeiders anders werkloos of gemarginaliseerd
raken `in de gemeenschap', daadwerkelijk haalbaar blijken te
zijn.\footnote{Zie Jeremy Rifkin, \emph{The End of Work: The Decline of
  the Global Labor Force and the Dawn of the Post-Market Era} (New York:
  G.P. Putnam's Sons, 1995).} Jeremy Rifkin pleit voor een `nieuw
partnerschap tussen de overheid en de derde sector' om de sociale
economie opnieuw op te bouwen. \ldots{} Het doel is om de armen te
voeden, basisgezondheidszorg te bieden, de jeugd te onderwijzen,
betaalbare woningen te realiseren en het milieu te beschermen\ldots{}
\footnote{Ibid., p.~250.}

\subsection{De ondergang van openbare
goederen}\label{de-ondergang-van-openbare-goederen}

Voorstanders van dwang zullen ongetwijfeld stellen dat een afnemend
staatsgezag leidt tot een onvermogen om openbare goederen te leveren of
ervan te genieten. Maar dat lijkt om zowel competitieve als andere
redenen onwaarschijnlijk. Ten eerste verliezen rechtsgebieden -- nu
technologie plaatsgebonden voordelen grotendeels heeft weggenomen -- hun
klanten snel als ze essentiële openbare goederen, zoals het handhaven
van orde, niet kunnen aanbieden. In uiterste gevallen, zoals te zien was
in Somalië, Liberia, Rwanda en het voormalige Joegoslavië, zullen
waarschijnlijk hordes blutvluchtelingen de landsgrenzen overschrijden op
zoek naar een betrouwbaardere waarborging van wet en orde. Deze extreme
vormen van deserteeractie, of `stemmen met de voeten', wijken in
urgentie nauwelijks af van gewoon `rechtsgebiedswinkelen'. Hoe dan ook,
ondernemingen zullen lokale rechtsgebieden ertoe dwingen om aan de
behoeften van hun klanten te voldoen.

\subsection{Competitieve territoriale
clubs}\label{competitieve-territoriale-clubs}

Econoom Charles Tiebout verwoordde dit idee al in 1956 -- het gaat om
meer dan een louter theoretisch concept.\footnote{Zie Charles M.
  Tiebout, `A Pure Theory of Local Expenditure,' Journal of Political
  Economy 64 (1956), pp.~416--424.} Fred Foldvary documenteerde in
\emph{Public Goods and Private Communities: The Market Provision of
Social Services} dat er geen fundamentele reden bestaat om sociale
diensten en andere openbare goederen uitsluitend via politieke kanalen
te leveren. De voorbeelden die Foldvary aandraagt, bevestigen tevens de
controversiële stelling van de Nobelprijswinnende econoom Ronald Coase,
namelijk dat overheidsinterventie niet nodig is om externaliteiten,
zoals vervuilingsproblemen, op te lossen. Ondernemers leveren
collectieve goederen via marktmechanismen en velen hebben dat al in
praktijk gebracht binnen lokale gemeenschappen. Foldvary's casestudies
tonen aan dat het privatiseren van gemeenschappen kan leiden tot
innovatieve manieren om openbare goederen en diensten te leveren en te
financieren.

\subsection{De weg naar voorspoed}\label{de-weg-naar-voorspoed}

De opkomst van microtechnologie opent de deur naar nieuwe financierings-
en reguleringsmethoden voor goederen die tot voor kort als publieke
goederen werden beschouwd. Later blijkt dat sommige van deze goederen in
werkelijkheid private kenmerken hebben. Snelwegen vormen daar een
duidelijk voorbeeld van. Toen congestie nog geen groot probleem was,
behandelde men wegen en snelwegen als publieke goederen. Adam Smith
bekritiseerde dit namelijk omdat hij vond dat deze infrastructuur een
onevenredig groot voordeel biedt aan omwonenden, terwijl bewoners van
afgelegen gebieden gedwongen worden te betalen zonder er veel van te
profiteren.

Dankzij technologische ontwikkelingen kunnen we in het
informatietijdperk tolheffing -- inclusief congestieheffingen --
invoeren die de toegang tot snelwegen, start- en landingsbanen en andere
infrastructuur nauwkeurig prijzen, zonder de verkeersstroom te
verstoren. Op die manier privatiseren we de levering van
transportinfrastructuur discreet en laat de financiering ervan direct
over aan de gebruikers. Volgens econoom Paul Krugman voegt een
marktgerichte prijsstelling van de Amerikaanse transportinfrastructuur,
naar schatting, jaarlijks tussen de 60 en 100 miljard dollar toe aan de
CIDP in de Verenigde Staten, terwijl het de efficiëntie in het gebruik
van hulpbronnen verbetert en de vervuiling vermindert.\footnote{Paul R.
  Krugman, `The Tax-Reform Obsession,' \emph{New York Times Magazine}, 7
  april 1996, p.~37.}

Verder mogen we niet vergeten dat het duurste onderdeel van wat moderne
natiestaten doen -- namelijk inkomensherverdeling -- in feite niet
draait om het leveren van een openbaar goed, maar om het verstrekken van
private goederen, en dat ten koste van de belastingbetalers. Met
`publieke kost' bedoelen we immers `ten koste van de belastingbetalers'.

Hoe zit het dan met een écht openbaar goed, zoals het onderhouden van
een militaire macht die een aanval door een grootmacht kan afschrikken?
Zo'n macht is nu eenmaal traditioneel kostbaar. Zoals we eerder al
aangaven: een regering zonder onbeperkte bevoegdheid om de inkomens en
eigendommen van haar burgers in te nemen, kan nooit een volgende oorlog
tussen grootmachten -- vergelijkbaar met de Tweede Wereldoorlog --
financieren.

Toch vormt deze fiscale beperking minder een bedreiging dan de
reactionairen doen vermoeden, simpelweg omdat toekomstige conflicten
niet meer zullen plaatsvinden zoals in de Tweede Wereldoorlog. De
technologische doorbraak die individuen bevrijdt, zal daarvoor zorgen.

Los van de politiek

In plaats van de kwaliteit en aard van zulke diensten over te laten aan
de grillen van de politiek, kun je `regeringen' op een ondernemende
wijze besturen en omvormen tot wat Foldvary omschrijft als
`concurrerende territoriale clubs.' Wij vermoeden dat uiteindelijk de
manier waarop deze clubs hun besluitvorming inrichten minder belangrijk
zal blijken dan hun succes in de marktproeven op prestatiegebied.
Tegenwoordig maakt het voor de meeste consumenten nauwelijks uit of een
bedrijf dat een product of dienst levert een eenmanszaak is, een
besloten vennootschap of een onderneming die door externe bestuurders --
voorgedragen door pensioenfondsen -- wordt geleid. Wij twijfelen er
bovendien niet aan dat de rationele afnemer van soevereiniteitsdiensten
in dit informatietijdperk zich niet druk zal maken over de vraag of
Singapore een massademocratie vormt of het particuliere bezit is van Lee
Kwan Yew.

De inhoud van het te vertalen boek is:

\bookmarksetup{startatroot}

\chapter{Moraliteit en misdaad in de `natuurlijke economie' van het
informatietijdperk}\label{moraliteit-en-misdaad-in-de-natuurlijke-economie-van-het-informatietijdperk}

\begin{quote}
``Corruptie\ldots{} is veel verder verspreid en universeel dan voorheen
werd gedacht. Bewijs daarvan is overal te vinden, zowel in
ontwikkelingslanden als, steeds vaker, in geïndustrialiseerde landen. .
. . Prominente politieke figuren, waaronder staatshoofden en ministers,
zijn beschuldigd van corruptie. . . . Op een bepaalde manier betekent
dit een privatisering van de staat, waarbij haar macht niet naar de
markt verschuift, zoals privatisering normaliter impliceert, maar naar
overheidsfunctionarissen en bureaucraten.''\footnote{Vito Tanzi,
  `Corruption: Arm's-length Relationships and Markets,' in Gianluca
  Fiorentini en Sam Peltzman, eds., \emph{The Economics of Organized
  Crime} (Cambridge: Cambridge University Press, 1995), pp.167, 170.} --
VIRO TANZI
\end{quote}

Wij menen dat naarmate de moderne natiestaat uiteenvalt, hedendaagse
barbariërs steeds meer invloed achter de schermen gaan uitoefenen.
Groepen zoals de Russische maffia's, die de botten van de voormalige
Sovjet-Unie oprapen, andere etnische misdaadbendes, nomenklaturen,
drugskartels en afvallige geheime diensten hanteren steeds vaker hun
eigen regels. Dat doen ze nu al.

Veel meer dan men doorgaans denkt, hebben moderne barbariërs de
contouren van de natiestaat al geïnfiltreerd, zonder dat haar uiterlijke
verschijning wezenlijk verandert. Het gaat om microparasieten die zich
tegoed doen aan een stervend systeem. Net zo gewelddadig en meedogenloos
als een staat in oorlog zetten deze groepen staatsmethoden op een
kleinere schaal in. Hun groeiende invloed en macht illustreren de
verkleining van de politiek. Efficiënte, kleinschalige
organisatiemethoden verminderen namelijk de omvang die nodig is om
effectief geweld te gebruiken en te beheersen. Naarmate deze
technologische revolutie vordert, groeit de neiging om roofzuchtig
geweld buiten centrale sturing te organiseren. Pogingen om geweld in te
dammen decentraliseren bovendien zodanig dat zij meer leunen op
efficiëntie dan op pure machtsschatten.

De toename van clandestiene criminele activiteiten en corruptie binnen
natiestaten vormt een belangrijke rode draad te midden van de
wereldwijde veranderingen. Wat je zult aanschouwen, lijkt op een
duistere, geheime versie van een slechte film, \emph{Invasion of the
Body Snatchers}. Nog voordat de meeste natiestaten zichtbaar instorten,
zullen hedendaagse barbariërs de dienst uitmaken. Zoals in de beroemde
B-film uit de jaren vijftig treden zij vaak in vermomming op. De `Pod
People' van de toekomst blijken echter geen buitenaardse wezens, maar
criminelen met uiteenlopende achtergronden die officiële functies
bekleden, terwijl hun loyaliteit ten minste gedeeltelijk buiten de
constitutionele orde valt.

Het einde van een tijdperk valt vaak samen met een periode van intense
corruptie. Wanneer de fundamenten van het oude systeem afbrokkelen,
vervaagt ook de sociale ethos, waardoor een klimaat ontstaat waarin
invloedrijke personen publieke doelen met privé criminele activiteiten
vermengen.

Helaas mag je niet altijd rekenen op de reguliere informatiekanalen voor
een nauwkeurig en actueel beeld van de ondergang van de natiestaat. Dit
`aanhoudende verzinsel' -- van het type dat ooit de val van het Romeinse
Rijk verhulde -- blijkt een typisch kenmerk te zijn van de ontbinding
van grote politieke entiteiten. Tegenwoordig verhult en maskeert het de
teloorgang van de natiestaat. Om verschillende redenen kun je de
nieuwsmedia niet altijd vertrouwen om je de waarheid te vertellen. Velen
koesteren een conservatieve visie en vertegenwoordigen daarmee de
nostalgie van het verleden. Sommigen raken verblind door verouderde
ideologische toewijding aan socialisme en de natiestaat. Anderen
schuwen, om meer concrete redenen, om de groeiende corruptie in een
vervalachtig systeem bloot te leggen. Weer anderen missen simpelweg de
fysieke moed die zo'n taak vergt. Daarnaast vrezen sommigen om hun baan
of schromen om zich uit te spreken, omdat ze bang zijn voor
vergeldingsmaatregelen. Uiteraard is er geen reden te vermoeden dat
verslaggevers en redacteuren minder vatbaar zijn voor corruptieve
prikkels dan bouwinspecteurs of Italiaanse bestratingsaannemers. Veel
vaker dan je zou verwachten, blijken belangrijke informatiebronnen --
die zogenaamd alles willen rapporteren -- minder betrouwbaar te zijn dan
men algemeen gelooft. Velen laten hun eigen belangen, zoals het
versterken van steun voor een wankelend systeem, prevaleren boven het
eerlijk informeren van jou. Ze delen weinig mee en geven nog minder
uitleg.

\section{Voorbij de realiteit}\label{voorbij-de-realiteit}

Naarmate de technologieën voor kunstmatige realiteit en computerspellen
blijven verbeteren, zul je zelfs in staat zijn om een avondnieuwsbericht
te bestellen dat exact het nieuws nabootst dat je wilt horen. Wil je
bijvoorbeeld een verslag waarin jij als winnaar van de tienkamp op de
Olympische Spelen centraal staat? Geen probleem. Het kan moeiteloos het
hoofdnieuws van morgen vormen. Je zult elk verhaal dat je verlangt -- of
het nu op waarheid berust of volledig verzonnen is -- op je televisie of
computer zien ontvouwen met een overtuigende realiteitsgetrouwheid die
veel verder gaat dan wat \emph{NBC} of \emph{BBC} momenteel kunnen
leveren.

We bewegen ons in een razendsnel tempo naar een wereld waarin informatie
net zo ongebonden zal zijn van de beperkingen van de werkelijkheid als
dat menselijke vindingrijkheid het mogelijk maakt. Dit zal ongetwijfeld
enorme gevolgen hebben voor zowel de kwaliteit als de aard van de
informatie die je ontvangt. In een wereld waarin kunstmatige realiteit
en de onmiddellijke overdracht van alles en overal samenkomen, wordt het
essentieel om het vermogen te ontwikkelen om het ware van het valse te
onderscheiden en integer te blijven oordelen.

Maar het betekent minder een verandering in onze huidige situatie dan
velen verwachten. De grens tussen waar en onwaar vervaagt vaak door
invloeden die door de technologie nog worden versterkt. Binnen de
wetenschap stellen we dat veel gevolgen van de informatie-revolutie
juist bevrijdend zijn.

Technologie overschrijdt inmiddels grenzen van geografische nabijheid en
politieke overheersing. Overheden mogen wel barrières opwerpen om de
handel in goederen te verstoren, maar zij hebben weinig macht over de
overdracht van informatie. Bijna iedere gast in elk restaurant in Hong
Kong is via een mobiele telefoon verbonden met de hele wereld. De harde
staatsgreepcomplotters in Moskou in augustus 1991 slaagden er niet in
Jeltsins communicatielijnen tot stilstand te brengen, omdat hij
beschikte over mobiele telefoons.

\subsection{Meer informatie, minder
begrip}\label{meer-informatie-minder-begrip}

Nu de obstakels voor het verspreiden van informatie verdwenen zijn,
hebben we toegang tot meer informatie -- wat positief is. Tegelijkertijd
zorgt dat ook voor meer onduidelijkheid over de betekenis ervan. De
moderne technologie die informatie losmaakt van politieke controle en de
beperkingen van tijd en plaats, maakt traditionele oordeelsvorming
bovendien waardevoller. Het inzicht dat ons helpt onderscheiden wat
belangrijk en waar is te midden van de overvloed aan feiten en
fantasieën, wordt vrijwel dagelijks kostbaarder. Dit geldt om ten minste
drie redenen:

\begin{enumerate}
\def\labelenumi{\arabic{enumi}.}
\tightlist
\item
  De enorme hoeveelheid beschikbare informatie vraagt om bondigheid.
\end{enumerate}

Beknoptheid leidt tot afkortingen. Afkortingen slaan vaak cruciale
details over. Als je met talloze feiten te verwerken bent en daarnaast
nog ontelbare telefoontjes moet terugbellen, verlang je er van nature
naar om elk moment van informatieoverdracht zo beknopt mogelijk te
maken. Helaas levert verkorte informatie vaak slechts een zwakke basis
op voor diepgaand begrip. De diepere en rijkere lagen van de
geschiedenis vallen namelijk in de vijfentwintig seconden durende
soundbites vaak weg, terwijl \emph{CNN} ze regelmatig verkeerd
weergeeft. Het overbrengen van een boodschap als variatie op een al
bekend thema gaat veel eenvoudiger dan het verkennen van een geheel
nieuw paradigma van begrip. Je kunt een honkbal- of cricketscore veel
gemakkelijker melden dan uitleggen hoe deze sporten precies worden
gespeeld en wat ze betekenen.

\begin{enumerate}
\def\labelenumi{\arabic{enumi}.}
\setcounter{enumi}{1}
\item
  De razendsnel veranderende technologie ondermijnt de megapolitieke
  fundamenten van onze sociale en economische organisatie. Hierdoor
  verouderen brede paradigma-inzichten -- onuitgesproken theorieën over
  hoe de wereld functioneert -- sneller dan ooit tevoren. Dit vergroot
  de behoefte aan een overkoepelend overzicht en ondermijnt de waarde
  van individuele `feiten' die vrijwel iedereen direct kan opzoeken met
  een zoekmachine.
\item
  De groeiende tribalisering en marginalisering in onze samenleving
  remmen zowel het publieke debat als ons denkvermogen. Veel mensen
  raken er zo aan gewend dat ze conclusies mijden, zelfs als de feiten
  daar ondubbelzinnig voor spreken. Een recent psychologisch onderzoek,
  vermomd als een opiniepeiling, liet zien dat leden van afzonderlijke
  beroepsgroepen vrijwel unaniem geen enkele conclusie accepteerden die
  voor hen tot inkomensverlies zou leiden, hoe sluitend de onderliggende
  logica ook was. Door de toegenomen specialisatie richt de
  interpretatieve informatie over deze beroepsgroepen zich vooral op het
  behartigen van hun belangen. Zij tonen weinig interesse in opvattingen
  die als onbeschoft, onrendabel of politiek incorrect worden ervaren.
  Een treffend voorbeeld van deze neiging is de voortdurende herhaling
  van rooskleurige vooruitzichten voor aandeleninvesteringen. De meeste
  voorspellingen hierover komen van effectenmakelaars, die zelden
  toegeven dat aandelen overgewaardeerd zijn, omdat hun inkomen
  afhankelijk is van transacties waarbij de meerderheid van hun klanten
  koopt. Onafhankelijke, afwijkende stemmen hoor je zelden.
\end{enumerate}

Om deze en andere redenen is het informatietijdperk nog lang geen
tijdperk van begrip geworden. Integendeel, het publieke debat is
aanzienlijk minder streng. Hoewel de wereld tegenwoordig meer informatie
in huis heeft dan ooit tevoren, ontbreekt er vrijwel een publieke stem
die de betekenis van gebeurtenissen kan beoordelen en kan vaststellen
wat waar is. Daarom hebben wij de lauwe belangstelling -- met name in de
Amerikaanse media -- voor het melden van aanwijzingen van
sensatiebeluste corruptie op hoog niveau binnen de Amerikaanse overheid
altijd geboeid.

Een centraal thema waar we in dit boek mee hebben geworsteld, is hoe
veranderende technologie en andere `megapolitieke' factoren de
`natuurlijke economie' transformeren. Met de `natuurlijke economie'
bedoelen we de darwinistische toestand van de natuur, waarin uitkomsten
-- vaak oneerlijk -- bepaald worden door fysieke kracht. In die context
noemen biologen een belangrijke gedragsvorm `interferentieconcurrentie'.

\subsection{Interferentieconcurrentie}\label{interferentieconcurrentie}

«Interference competitors», zoals Jack Hirshleifer het verwoordde,
winnen en behouden de controle over middelen door hun rivalen direct af
te weren of te hinderen.\footnote{Hirshleifer, op. cit., p.176.} Hoezeer
we er ook naar verlangen dat menselijk gedrag altijd geleid wordt door
de rechtsstaat en andere maatschappelijk opgelegde spelregels (ook wel
`politieke economie' genoemd), het bewijs is overweldigend dat veel
mensen `naar de regels spelen' slechts wanneer het hen uitkomt.
Hirshleifer, een autoriteit op het gebied van conflicten, merkte op:
«{[}D{]}e volharding van criminaliteit, oorlog en politiek leert ons dat
de feitelijke menselijke aangelegenheden nog steeds grotendeels
onderhevig zijn aan de drukken vanuit de natuurlijke
economie.»\footnote{Ibid., p.~169.}

Met andere woorden: het vreedzame en wettige handelen van de Homo
economicus -- zoals in de leerboeken wordt beschreven, waarbij
eigendomsrechten worden gerespecteerd en men niet zomaar neemt wat niet
van hem is -- bepaalt slechts een deel van de economische uitkomsten.
Conflicten, waaronder openlijk geweld, spelen eveneens een belangrijke
rol. Zoals econoom Hirshleifer opmerkt: «Zelfs onder gezag van wet en
regering vindt het rationele, eigenbelanghebbende individu een afweging
tussen legale en illegale middelen om zijn doelen te bereiken -- tussen
productie en ruil aan de ene kant en diefstal, fraude en afpersing aan
de andere kant.»\footnote{Hirshlcifer, op. cit., p.~173.}

\section{Overval in het
informatietijdperk}\label{overval-in-het-informatietijdperk}

In een waardevol boek over geweld, misdaad en politiek, \emph{The
Political Economy of Conflict and Appropriation}, stellen Michelle R.
Garfinkel en Stergios Skaperdas dat individuen en groepen óf rijkdom
creëren door te produceren óf de door anderen opgebouwde welvaart in
beslag nemen.\footnote{Garfinkel en Skaperdas, op. cit., p.~1.} Zij
citeren een verhaal over moderne interferentieconcurrentie,
oorspronkelijk gerapporteerd door \emph{The Economist}: «Een Amerikaanse
zakenman, die onlangs in Moskou was aangekomen om een kantoor te openen,
werd bij zijn hotel opgewacht door vijf mannen met gouden horloges,
pistolen en een afdruk van de nettowaarde van zijn bedrijf. Zij eisten
7\% van toekomstige inkomsten. Hij nam de eerstvolgende vlucht naar New
York, waar overvallers minder geraffineerd zijn.»\footnote{Ibid.} Dit
verhaal laat zien dat overvallen in het informatietijdperk vooral te
danken zijn aan de mogelijkheden die nieuwe technologie biedt, en niet
louter het feit dat criminelen in Rusland via het internet toegang
hebben tot financiële profielen en kredietrapporten van hun
slachtoffers.

\subsection{Dalende beslissendheid van militaire
macht}\label{dalende-beslissendheid-van-militaire-macht}

In zowel positieve als negatieve zin verzwakt informatietechnologie de
mogelijkheid van de natiestaat om haar gezag in een onstabiele wereld op
te leggen, doordat grootschalige militaire macht minder doorslaggevend
is. Waar men ooit, zoals Voltaire vermaande, geloofde dat `God aan de
kant van grotere bataljons' stond, lijkt de goddelijke steun voor
grootschalig geweld met de dag af te nemen. Integendeel, alle signalen
wijzen op dalende opbrengsten van geweld, wat erop duidt dat grote
entiteiten zoals de natiestaat hun enorme vaste kosten niet langer
kunnen rechtvaardigen.

Het meest voor de hand liggende bewijs voor de afnemende effectiviteit
van gecentraliseerde macht is de opkomst van terrorisme. De
grootschalige bomaanslagen in de Verenigde Staten midden in de jaren
negentig tonen aan dat zelfs de militaire supermacht van de wereld niet
immuun is voor aanvallen.

Een andere duidelijke uiting van de dalende winstgevendheid van geweld
blijkt uit de wereldwijde toename van gangsterisme en georganiseerde
misdaad, vergezeld van politieke vriendjespolitiek en corruptie. Deze
ontwikkelingen karakteriseren een overwegend amorale sfeer waarin de
staat weliswaar kan dwingen, maar weinig in staat is bescherming te
bieden. Nu haar geweldsmonopolie afbrokkelt, dringen nieuwe spelers zich
op -- denk maar aan de pestkoppen die geprobeerd hebben particuliere
belastingen op te leggen aan de Amerikaanse zakenman in Moskou.

Kleine groeperingen, stammen, triaden, bendes, gangsters, mafia's,
milities en zelfs individuele strijders hebben hun militaire
effectiviteit versterkt. In de `natuurlijke economie' van het komende
millennium zullen zij waarschijnlijk meer reële macht uitoefenen dan in
de twintigste eeuw. Wapens die microchips inzetten, verschuiven de
machtsbalans naar de verdediging, waardoor beslissende agressie minder
winstgevend en daarmee minder waarschijnlijk wordt. Slimme wapens, zoals
Stinger-raketten, neutraliseren een groot deel van het voordeel dat
grote, welvarende staten vroeger haalden met dure luchtmacht om armere,
kleinere groepen aan te vallen.

\subsection{Informatieoorlog in het
verschiet}\label{informatieoorlog-in-het-verschiet}

Wat aan de horizon opdoemt, is de veelbesproken maar nauwelijks begrepen
mogelijkheid van een `informatieoorlog'. Dit duidt tevens op de
afnemende meeropbrengst van traditioneel geweld. De zogenaamde `logische
bommen' kunnen onder meer luchtverkeersleidingssystemen,
wisselmechanismen op spoorwegen, stroomgeneratoren en
distributienetwerken, water- en rioleringssystemen, telefonische relais
en zelfs de communicatiesystemen van het leger ontregelen of saboteren.
Nu samenlevingen steeds meer leunen op geautomatiseerde
besturingssystemen, veroorzaken `logische bommen' bijna evenveel schade
als fysieke explosies.

In tegenstelling tot conventionele bommen brengen vijandige overheden,
groepen freelance programmeurs en zelfs begaafde individuele hackers
`logische bommen' op afstand tot ontploffing. Neem bijvoorbeeld dat in
1996 de autoriteiten een Argentijnse tiener arresteerden omdat hij
herhaaldelijk Pentagoncomputers hackte. Hoewel hackers tot nu toe zelden
kozen voor het destructief saboteren van computergestuurde systemen, is
het gebrek aan effectieve tegenmaatregelen geen reden voor hun
terughoudendheid.

Als het tijdperk van de informatieoorlog eindelijk aanbreekt, zullen de
tegenstanders waarschijnlijk niet uitsluitend overheden betreffen. Een
bedrijf als \emph{Microsoft} beschikt immers over een grotere capaciteit
om een informatieoorlog te voeren dan 90 procent van de natiestaten
wereldwijd.

\subsection{De tijd van het soevereine
individu}\label{de-tijd-van-het-soevereine-individu}

Dat vormt voor een deel de reden waarom we ons boek \emph{het soevereine
individu} zo hebben genoemd. Naarmate de schaal van oorlogsvoering
afneemt, regelen we defensie en bescherming op een kleinere, meer lokale
schaal. Daardoor biedt men deze diensten steeds vaker als particuliere
in plaats van publieke goederen aan, waarbij particuliere aannemers op
winst nastreven. Dat zie je al duidelijk terug in de privatisering van
de politiezorg in Noord-Amerika. In de Verenigde Staten behoort het
beroep van `beveiliger' tot de snelst groeiende beroepen. Prognoses
voorspellen dat het aantal particuliere beveiligers in 2005 met zo'n 24
tot 40 procent zal toenemen ten opzichte van 1990.\footnote{Hamish
  McRae, \emph{The World in 2020} (London: \emph{Harper Collins}, 1995),
  p.~188.}

De privatisering van politietaken vormt inmiddels al een duidelijke
trend. Toch, zoals de Anglo-Ierse guru Hamish McRae opmerkt, komt dit
nauwelijks voort uit doelbewuste overheidsbeslissingen. In \emph{The
World in 2020} schrijft hij het volgende:

\begin{quote}
Geen enkele overheid heeft bewust besloten zich terug te trekken uit
bepaalde politietaken, noch hebben overheden dat daadwerkelijk gedaan;
in plaats daarvan nam de private sector hun plaats in. Zowel vanwege de
vermeende tekortkomingen bij de politie als door andere maatschappelijke
veranderingen hebben particuliere beveiligingsbedrijven in de loop der
jaren geleidelijk het merendeel van de taak overgenomen om burgers in
kantoren en winkelcentra te beschermen. Zoals de ommuurde woonwijken in
Los Angeles aantonen, keert men zelfs enigszins terug naar het
middeleeuwse stadsconcept, waarbij burgers achter stadsmuren leven die
door bewakers patrouilleren en waarvan de toegang uitsluitend via
gecontroleerde poorten verloopt.\footnote{Ibid., pp.~188--89.}
\end{quote}

Wij zijn ervan overtuigd dat dit nog maar het voorproefje is van een
veel bredere privatisering van bijna alle functies die de overheden in
de twintigste eeuw vervulden. Informatietechnologie heeft immers de
capaciteit van gecentraliseerde autoriteiten om macht uit te oefenen en
fysieke veiligheid te waarborgen in grootschalige systemen ernstig
ondermijnd, waardoor de optimale omvang van vrijwel elke onderneming in
de `natuurlijke economie' afneemt.

Om in te spelen op deze technologische verschuiving is een enorme
investering nodig (lees: een kans) om kwetsbare systemen opnieuw in te
richten met gedistribueerde in plaats van geconcentreerde capaciteiten.
Als men de grootschalige kwetsbaarheden niet verhelpt, raken die
systemen vatbaar voor catastrofale storingen.

Vroeg of laat -- bijna onvermijdelijk, al gebeurt het niet opzettelijk
-- ontwikkelen de diensten en producten die grote bureaucratische
instanties en bedrijven leveren zich tot uiterst concurrerende markten.
In plaats van vanuit één centraal hoofdkantoor te werken, organiseren
zij hun activiteiten via een gedistribueerd, decentraal netwerk.

Bedrijven met een hoofdkantoor dat makkelijk omsingeld kan worden door
betogers of gesaboteerd door terroristen lopen het risico kwetsbaar te
zijn totdat zij overstappen op een `virtuele onderneming' zonder vaste
locatie -- oftewel, een organisatie die `tegelijkertijd op tal van
plaatsen aanwezig is', zoals Kevin Kelly, uitvoerend redacteur van
\emph{Wired} magazine, in \emph{Out Of Control} beschrijft.

Kelly wijst erop dat technologische ontwikkelingen de behoefte
verminderen om productieprocessen centraal te organiseren. Tijdens het
grootste deel van de industriële revolutie vergaarde men aanzienlijke
rijkdom doordat alle productie onder één dak werd geconcentreerd -- men
dacht destijds dat `groter efficiënter is.' Tegenwoordig geldt dat
principe niet meer.

Kelly voorspelt dat een auto van de toekomst -- de Upstart Car --
ontworpen en geproduceerd kan worden door slechts een dozijn mensen die
via een virtueel netwerk samenwerken.

In de toekomst kan een te grote schaal niet alleen averechts werken,
maar ook gevaarlijk blijken. Grotere ondernemingen vormen immers
aantrekkelijkere doelwitten. Beoefenaars van de ondergrondse economie
tonen aan dat het vermijden van opsporing een van de geheimen is om
belastingdruk te ontduiken. Dit gelingt kleinschalige, `virtuele
ondernemingen' veel gemakkelijker dan bij gevestigde bedrijven die
vanuit een imposante wolkenkrabber opereren met hun naam in lichten.
Zulke ondernemingen trekken onvermijdelijk de aandacht van `mannen met
gouden horloges, pistolen en een print-out van de nettowaarde van het
bedrijf' -- gangsters die, bijvoorbeeld in Rusland, hun eigen
particuliere vorm van belasting heffen. Bedrijven van elke omvang lopen
daardoor risico op criminele afpersing en extra lasten opgelegd door
georganiseerde misdaadbendes.

\begin{quote}
`Als we iemand een racketeer noemen, bedoelen we iemand die eerst een
dreiging opzet en vervolgens kosten rekent om die weg te nemen. De
bescherming die overheden bieden, komt volgens deze definitie vaak neer
op afpersing.'\footnote{Tilly, \emph{War Making and State Making as
  Organized Crime}, in Evans, Rueschemeyer en Skoepol (op. cit.),
  p.~171.} - CHARLES TILLY
\end{quote}

\subsection{Natuur haat monopolies}\label{natuur-haat-monopolies}

Naarmate het geweldmonopolie waarover de `grotere bataljons' beschikken
verder afbrokkelt, mogen we als eerste gevolg verwachten dat de
georganiseerde misdaad floreert. Immers, georganiseerde misdaad
concurreert met natiestaten als het gaat om het inzetten van geweld voor
roofzuchtige doeleinden. Hoe onbeleefd dat ook klinkt, maar zoals
politicoloog Charles Tilly ons eraan herinnert, kwalificeren regeringen
-- het toonbeeld van beschermingsafpersing met het voordeel van
legitimiteit -- zich eigenlijk als onze grootste voorbeelden van
georganiseerde misdaad.

Als je verder niets wist van de wereld behalve dat een belangrijk
monopolie instortte, kun je er vrijwel zeker van zijn dat de directe
concurrenten het meeste zullen profiteren. Het is dan ook geen toeval
dat drugskartels, criminele bendes, mafia's en triaden wereldwijd in
opmars zijn.

\subsection{Sistema del potere}\label{sistema-del-potere}

Van Rusland tot Japan en de Verenigde Staten beïnvloedt georganiseerde
misdaad economieën veel meer dan economische leerboeken doen vermoeden.
Wat de Sicilianen `sistema del potere' noemen -- oftewel het
machtssysteem van de georganiseerde misdaad -- speelt een steeds
belangrijkere rol in het functioneren van economieën.

Europese politiefunctionarissen melden dat internationale
misdaadsyndicaten, waaronder Russische en Italiaanse mafia's, een
`dominante rol' hebben gespeeld bij het financieren van de genocidale
oorlogen die de Balkan de afgelopen jaren teisterden.

Ook drugssmokkelaars hebben een cruciale rol gespeeld in het financieren
van recente burgeroorlogen en opstanden in andere delen van de wereld.
Julio Fernandez, chef van de drugseenheid van de Spaanse nationale
politie in Catalonië, zegt: `Van 1986 tot 1988 vervoerden Tamil
Tiger-guerrilla's, in samenwerking met Pakistaanse inwoners in Barcelona
en Madrid, 80 procent van de heroïne in Spanje. Zodra we dat netwerk
door arrestaties hadden ontmanteld, namen Koerden uit Turkije het over
en domineerden zij volledig gedurende de daaropvolgende twee
jaar.'\footnote{Frank Viamo, `De nieuwe maffiaorde,' \emph{Mother
  Jones}, mei/juni 1995, p.~55.} Het is waarschijnlijk dat wanneer een
nieuwe burgeroorlog of opstand uitbreekt, wanhopig arme strijders hun
militaire inzet financieren door drugs te verhandelen en drugsgeld wit
te wassen.

\subsection{Door drugs gefinancierde
prijsverlaging}\label{door-drugs-gefinancierde-prijsverlaging}

Georganiseerde criminele syndicaten oefenen een neerwaartse druk uit op
de prijzen van goederen -- met uitzondering van drugs. Op microniveau
financieren deze groepen schijnbaar legitieme ondernemingen met de buit
uit hun illegale activiteiten. Ze witwassen drugswinst en ander onwettig
verkregen gelden door reguliere producten onder de kostprijs te
verkopen, waardoor ze de prijzen van hun eerlijke concurrenten
ondermijnen en talrijke bedrijven in de problemen brengen.

\subsection{Yakuza-deflatie}\label{yakuza-deflatie}

In Japan vervulden invloedrijke Yakuza-bendes een sleutelrol tijdens de
hyperactieve vastgoedzeepbel van de late jaren tachtig. Hoewel
negentigduizend Yakuza jaarlijks tussen de \$10,19 miljard (volgens
officiële schattingen) en \$71,35 miljard (volgens professor Takatsugu
Nato) genereren, gingen Japanse banken voor deals die door de Yakuza
werden gesteund vaak oninbare leningen aan, hetgeen hun solvabiliteit in
gevaar bracht.\footnote{Zie Velisarios Kattoulas, `Japans yakuza nemen
  hun plaats in de criminele elite,' \emph{Washington Times}, 25
  november 1994, p.~A22.} Daardoor ervaart de Japanse economie een
deflatoire druk -- `prijsvernietiging', zoals de Japanners het noemen --
die kenmerkend is voor die periode.

\subsection{Een blind oog}\label{een-blind-oog}

De Russische maffia's, zoals Yeltsin zelf al toegaf, zijn versmolten met
`commerciële organisaties, administratieve instanties, organen van het
ministerie van Binnenlandse Zaken en gemeentelijke overheden'
\footnote{Viamo, op. cit., p.~49.}. Dankzij de immuniteit die zij zich
hadden veiliggesteld door banden met de politie aan te gaan, kunnen zij
met flagrant geweld hun particuliere belastingen afdwingen. Betrouwbare
bronnen stellen dat tegenwoordig vier van de vijf Russische bedrijven
afpersingsgeld betalen. `Volgens sommige rapporten moeten lokale kleine
bedrijven in Rusland 30 tot 50 procent van hun winst afstaan aan
racketeers, en niet slechts de magere 7 procent die van de Amerikaanse
zakenman wordt geëist.'\footnote{Garfinkel en Skaperdas, op. cit., p.~2.}

In 1993 classificeerde men in Rusland officieel 355.500 misdrijven als
afpersing, waaronder bijna 30.000 voorbedachte moorden -- grotendeels
huurmoorden als gevolg van conflicten in de commerciële en financiële
sector. Volgens voormalig minister van Binnenlandse Zaken, generaal
Viktor Yerin, betrof het merendeel huurmoorden. Meestal sloegen de
autoriteiten de ogen dicht. Criminele organisaties, die dankzij hun
beheersing van dwang en corruptie opereren, spelen een sleutelrol in de
economie, zoals economen Gianluca Fiorentini en Sam Peltzman schrijven
in \emph{The Economics of Organized Crime}.\footnote{Fiorentini en
  Peltzman, op. cit., p.~15.} Theoretisch kan hun invloed soms zelfs
positief uitpakken, doordat zij de regelgeving temperen en overheden
aansporen beter publieke goederen te leveren. Een machtige maffia houdt
de monopolistische rol van overheidsinstanties in toom.\footnote{Ibid.}
In gebieden waar georganiseerde misdaadgroepen sterk vertegenwoordigd
zijn, slagen overheden er vaak niet in zelfstandig beleid te voeren,
doordat de maffia hen tegenspreekt.

\subsection{Collusie}\label{collusie}

Het valt op dat overheden de maffia's -- hun voornaamste concurrenten op
het gebied van georganiseerde dwang -- maar al zelden direct aanpakken.
Uit strikt economisch oogpunt is dat niet verrassend. Voor `de gekozen
leden van het openbaar bestuur' blijkt de meest winstgevende regeling
een `collusieakkoord' met de georganiseerde misdaad te zijn. Fiorentini
en Peltzman merken op dat `er aanwijzingen zijn voor grootschalige
overeenkomsten waarbij de georganiseerde misdaad politieke steun
waarborgt voor groepen kandidaten, terwijl laatstgenoemden de gunst
beantwoorden door een gunstig beheer van overheidsopdrachten en de
levering van overheidsdiensten of subsidies.'

In tegenstelling tot de indruk die Hollywood wekt, richten criminele
organisaties zoals de Siciliaanse maffia zich tegenwoordig op het
binnendringen in en oplichten van overheden. `De meeste geleerden menen
dat de kernactiviteit van de Siciliaanse maffia er tegenwoordig precies
in bestaat dat zij zich toe-eigenen van diverse bronnen van
overheidsuitgaven en fraude organiseren met betrekking tot lokale,
nationale en Europese subsidieregelingen.'\footnote{Ibid.}

\subsection{Narco republieken}\label{narco-republieken}

Zoals we in \emph{The Great Reckoning} hebben gewaarschuwd, zijn
talrijke regeringen wereldwijd volledig gecorrumpeerd door drugsherren.
Mexico vormt daar een onbetwistbaar voorbeeld van. Voormalig Mexicaans
federaal adjunct-aanklager Eduardo Valle Espinosa plaatste het
Mexicaanse systeem in perspectief in zijn ontslagverklaring: `Niemand
kan een politiek project opzetten waarin de kopstukken van de
drugshandel en hun geldschieters niet voorkomen. Want als je dat wel
doet, sterf je.' Valle verklaarde dat steekpenningen het dienen als
Mexicaanse politiechef zo lucratief maken dat kandidaten tot \$2 miljoen
betalen om aangenomen te worden. Als je het strikt vanuit een winst- en
verliesberekening bekijkt, kan het kopen van een functie binnen de
lokale politie al gauw een winstgevende investering blijken te zijn.
Drugskartels betalen fortuinen aan zelfs laaggeplaatste Mexicaanse
ambtenaren, omdat dat hen immuniteit biedt tegen vervolging voor hun
misdaden.

Colombia is een ander land waar de hoogste regionen van de overheid
gedomineerd worden door drugsherren. Onlangs trokken de Amerikaanse
autoriteiten het visum van de Colombiaanse president Ernesto Samper voor
de Verenigde Staten in, omdat hij opzettelijk politieke bijdragen van
drugsdealers aannam in ruil voor gunsten.

\subsection{De pot verwijt de ketel dat hij zwart
ziet}\label{de-pot-verwijt-de-ketel-dat-hij-zwart-ziet}

Wie de rapporten in onze nieuwsbrief \emph{Strategic Investment} uit de
jaren negentig heeft gevolgd, herkent direct de ironie in de houding van
de Clinton-administratie tegenover Samper. Geloofwaardig bewijs wijst
erop dat de Amerikaanse president Bill Clinton alles heeft gedaan
waarvan Samper wordt beschuldigd -- en zelfs meer. Zelfs als u ons woord
niet aanneemt, onthullen twee grondig onderzochte boeken, geschreven
door auteurs uit beide uiteinden van het politieke spectrum, Clintons
achtergrond in opvallend detail.

Roger Morris, die doorgaans een linkse invalshoek hanteert, werkte als
veiligheidsfunctionaris in het Nixon-tijdperk en was bovendien een
belangrijke medewerker van Dean Acheson, president Lyndon Johnson en
Walter Mondale. Morris behaalde zijn doctoraat aan \emph{Harvard
University}. In zijn boek \emph{Partners in Power} onthult hij een
smerig verleden van Clinton, waardoor Samper als een onschuldige
padvinder overkomt.

Morris beschrijft Clintons vaderloze jeugd in Hot Springs, Arkansas --
een broeinest van gokken, prostitutie en georganiseerde misdaad waarbij
bijna de gehele familie betrokken was. Bill Clintons stiefoom, Raymond
Clinton, naar wie hij verwees als een `vadersfiguur', zou naar verluidt
een invloedrijke `peetvader' zijn geweest binnen de Dixie-maffia. Morris
stelt dat Bill Clinton als rekruté bij de CIA instroomde en zijn
studietijd aan Oxford besteedde aan het observeren van
anti-Vietnamoorlogactivisten. Volgens Morris diende Clinton tijdens zijn
gouverneurschap als een verlengstuk van de CIA, doordat hij een operatie
faciliteerde voor het smokkelen van drugs en wapens, met als middelpunt
Mena, Arkansas. Hij beschuldigt de gehele CIA ervan betrokken te zijn
bij drugshandel, in plaats van de mogelijkheid te overwegen dat Clinton
zich bij een corrupte factie binnen het agentschap had aangesloten --
een kans die voor ons aannemelijker lijkt. Beide interpretaties
impliceren dat de voornaamste geheime inlichtingendienst van de
Amerikaanse overheid, al dan niet direct, op grote schaal betrokken is
bij georganiseerde drugssmokkel. Als de CIA geen verlengstuk blijkt te
vormen van de georganiseerde misdaad, nadert zij desondanks op
zorgwekkende wijze de grens van 50. \footnote{Voor aanvullend en
  expliciet bewijs van de betrokkenheid van de CIA bij de drugshandel,
  zie Michael Levine, \emph{The Big White Lie: The Deep Cover Operation
  That Exposed the CIA Sabotage of the Drug War} (New York: Thunder's
  Mouth Press, 1994).}

\subsection{Eén kans op 250.000.000}\label{euxe9n-kans-op-250.000.000}

Toch bevat \emph{Partners in Power} details die elke student van de
corruptie in de hedendaagse Amerikaanse politiek zeker zullen boeien.
Morris richt zijn aanwijten niet uitsluitend op Bill Clinton; ook zijn
vrouw krijgt volop kritische aandacht. Neem bijvoorbeeld dit fragment
uit Morris' verslag over Hillary Clintons opmerkelijke handel in
grondstoffen: `In 1995 lieten economen van \emph{Auburn University} en
\emph{North Florida University} een geavanceerd statistisch
computermodel draaien op basis van de handelsactiviteiten van de
eerstedame, ter voorbereiding op publicatie in het \emph{Journal of
Economics and Statistics}, waarbij zij alle beschikbare gegevens en
marktinformatie uit de \emph{Wall Street Journal} benutten. Zij
berekenden dat de kans dat Hillary Rodham haar transacties op legitieme
wijze uitvoerde, minder was dan één op 250.000.000.' \footnote{Roger
  Morris, \emph{Partners in Power} (New York: Henry Holt, 1996), p.~233.}

Morris verzamelt talloze belastende details over de drugssmokkel- en
witwasoperatie die in Arkansas onder Clinton floreerde. `Door de enorme
hoeveelheden drugs en geld die met zijn vluchten werden gegenereerd,
groeide het kleine Mena, Arkansas, in de jaren tachtig uit tot één van
's werelds centra van de narcoticahandel.' \footnote{Ibid., p.~393.}
Morris citeert een vertrouweling die over Clinton getuigde: `Hij wist
het.'

Clinton wist van de cocaïnesmokkel en vertelde staatspolitieagent L. D.
Brown -- een voormalig lijfwacht aan wie hij hielp een functie bij de
\emph{CIA} te bemachtigen -- dat het drugsvervoer geen
\emph{CIA}-operatie betrof. `Oh, nee,' zei Clinton, `dat is Lasater's
zaak.' \footnote{Ibid., p.~411.}

Dan Lasater, een veroordeelde cocaïnedistributeur, behoorde tot
Clinton's belangrijkste financiële supporters. Hij verdiende miljoenen
met het staatsbedrijfsleven in Arkansas en overhandigde naar verluidt
eens \$300.000 contant -- in een bruine papieren zak -- aan de
toenmalige gouverneur van Kentucky, John Y. Brown. Volgens Morris was
Lasater nooit louter een grote donor die speciale eerbied verdiende,
maar een buitengewoon vertrouweling die Clinton geregeld opzocht in zijn
makelaarskantoor en spontaan naar het landhuis kwam. \footnote{Ibid.,
  p.~418.} Morris meldt dat Lasater's chauffeur -- die hem vaak naar het
landhuis bracht -- een veroordeelde moordenaar was die altijd een wapen
bij zich droeg en erom bekendstond ook drugs te verhandelen. \footnote{Ibid.}
Volgens Morris had de president van de Verenigde Staten een warmere band
met een drugshandelaar dan de vermeende connectie tussen de Colombiaanse
president Ernesto Samper en het Cali-cartel.

\begin{quote}
`Poeh! Bob zegt dingen over Bill Clinton die zelfs Hillary niet zou
zeggen' -- P.J. O'ROURKE
\end{quote}

R. Emmett Tyrell, Jr., hoofdredacteur van \emph{The American Spectator},
is geen links-liberaal zoals Morris. Toch bevat zijn verslag \emph{Boy
Clinton} veel van dezelfde details die Morris aandroeg om Clinton af te
schilderen als een corrupte politicus, nauw verweven met de drugshandel
en andere misdaden. Sterker nog, de proloog van \emph{Boy Clinton}
citeert L. D. Brown, Clinton's voormalige lijfwacht, die sensationeel
beweert dat Clinton medeplichtig was aan de activiteiten van een
dodensdienst die was opgezet om getuigen met kennis van de drugshandel
in Mena uit te schakelen.

Brown vertelt dat men hem op 18 juni 1986 persoonlijk naar Puerto
Vallarta, Mexico stuurde en hem daarbij een in België vervaardigd F.A.L.
lichtautomatisch geweer overhandigde. Onder het pseudoniem Michael
Johnson kreeg Brown naar verluidt de opdracht om Terry Reed neer te
schieten en te vermoorden.

U herinnert zich ongetwijfeld dat Reed in 1994 bekend werd als co-auteur
van \emph{Compromised: Clinton, Bush en de CIA}. Volgens dat boek heeft
de CIA het presidentschap in haar greep gekregen en hebben haar `zwarte
operaties, als een kanker, de organen van de overheid gemetastasiseerd'.
Meer specifiek stellen Reed en zijn mede-auteur dat zowel Clinton als
Bush door hun betrokkenheid bij illegale activiteiten in Arkansas --
onder meer drugshandel -- diep gecompromitteerd raakten.

Brown voerde de opdracht tot de moord op Reed niet uit. Hij en Reed
overleefden, waardoor zij in ieder geval delen van hun verhalen konden
vertellen, en het er zo beter op hadden dan degenen die destijds en
later met Clinton in verband kwamen. Neem als voorbeeld de inmiddels
overleden Jerry Parks, die in 1992 de beveiliging verzorgde voor het
hoofdkwartier van Clinton-Gore en in september 1993 slachtoffer werd van
een ganglandachtige aanslag.

In een andere bizarre wending onthulde de Londense \emph{Sunday
Telegraph}, op basis van exclusieve informatie verstrekt door de weduwe
van Parks, dat de overleden Vincent Foster Parks had ingehuurd als spion
op Bill Clinton.

Waarom Foster een dossier met compromitterende informatie over Clinton
wilde samenstellen, blijft een raadsel. (Hij verklaarde dat hij het voor
Hillary deed.) Maar dit weerlegt in elk geval de officiële beschrijving
van Foster als een naïeve plattelandsjongen, die zo geschokt raakte door
de meedogenloze werkwijzen in Washington dat hij zichzelf uit wanhoop
van het leven beroofde. Dat altijd ongeloofwaardige verhaal verliest met
elke nieuwe onthulling steeds meer aan geloofwaardigheid.\footnote{Voor
  een grondige bespreking van het Foster-verhaal zie \emph{Christopher
  Ruddy, Vincent Foster: The Ruddy Investigation}, dat verkrijgbaar is
  voor \$19,95 via 1-800-711-1968.}

\subsection{De president van de
maffia}\label{de-president-van-de-maffia}

Hoewel de wereld als geheel weigert te accepteren dat nauwe banden met
de georganiseerde misdaad en criminelen de president van de Verenigde
Staten negatief beïnvloeden, wijst het bewijs eenduidig in die richting.
Morris verwijst naar een voormalig Amerikaanse aanklager die criminele
figuren en hun belangen nauwlettend in de gaten hield. Hij stelt dat de
verkiezing van Clinton als gouverneur in 1984 `de verkiezing was waarop
de maffia écht binnenkwam in de politiek van Arkansas, de jongens van de
hondenbaan en de renbaan, de profiteurs die een mooie kans zagen\ldots{}
het ging verder dan onze oude Dixie-maffia, die in vergelijking daarmee
een kleinigheid was. Dit was crimineel geld uit het oosten en de
Westkust dat, net als de legitieme ondernemingen, de kansen zag.'

Blijkbaar blijven andere gelijkgestemden de mogelijkheden rondom Clinton
opmerken. Volgens \emph{New York} magazine -- dat opvolgt op een eerder
artikel in \emph{Readers' Digest} -- meldt men dat `de belangrijkste
bondgenoten van de president binnen de vakbondsbeweging tevens mannen
zijn die gelieerd zijn aan wat naar alle aanzichten enkele van de
viesste, meest door de georganiseerde misdaad geïnfiltreerde vakbonden
in Amerika lijken te zijn.' Van bijzonder belang is de hechte band
tussen Clinton en Arthur Coia. Coia, een van de `prima fondsenwervers'
van Clinton, leidt de Laborers International Union of North America,
`een van de meest opzichtig corrupte vakbonden in de geschiedenis van de
vakbeweging.'

Blijkbaar sloot het ministerie van Justitie in de Clinton-periode een
deal met Coia -- wat \emph{New York} magazine omschrijft als een
`merkwaardig genereuze deal' -- om zijn positie te handhaven, ondanks de
overtuigende beschuldigingen dat hij al lange tijd bevriend is met
figuren uit de georganiseerde misdaad.

Of de these van Terry Reed, dat `de CIA de presidentsfunctie heeft
gekaapt', nu klopt of niet, bestaat er duidelijk een sterke verleiding
voor personen binnen geheime organisaties -- die belast zijn met het
uitvoeren van `zwarte operaties' -- om volgens professor Hirshleifer
bewust onwettige middelen in te zetten om hulpbronnen te verkrijgen.

Gezien technologische veranderingen die de doorslaggevende kracht van
geconcentreerde militaire macht wereldwijd ondermijnen, mag men een
toename van corruptie verwachten, zo niet zelfs een volledige overname
van regeringen door georganiseerde criminele ondernemingen.

Hirshleifer betoogt -- en wij zijn het daarmee eens -- dat de
instituties van de politieke economie nooit zo perfect kunnen
functioneren dat zij de grondslagen van de natuurlijke economie volledig
verdrijven.\footnote{Hirshleifer, op. cit., p.~173.} Binnen de
`natuurlijke economie' verschuift de macht, wat ingrijpende
veranderingen in de machtsverhoudingen binnen de samenleving met zich
meebrengt.

Politieke corruptie, zoals Vito Tanzi scherp opmerkte, vertegenwoordigt
een vorm van privatisering van de staat waarbij de macht niet de markt
betreedt zoals gebruikelijk is bij privatisering, maar in plaats daarvan
bij overheidsfunctionarissen en bureaucraten terechtkomt.\footnote{Tanzi,
  op. cit., pp.~167, 170.} In feite gebeurde dit al bij het \emph{FBI}
en andere politiediensten onder Clinton. Clinton en zijn vriendjes
vormen de `rechtsstaat' naar eigen smaak. Tot dusver blijkt er
nauwelijks bewijs dat deze corrupte connecties enige invloed uitoefenen
op de kiezers, ook al pakken de massamedia de verhalen op en bespreken
ze uitvoerig. Integendeel, er lijkt weinig bezorgdheid te bestaan over
signalen dat de president van de Verenigde Staten medeplichtig is aan
drugshandel, witwaspraktijken en nog ernstigere misdrijven.

Dit roept de vrees op die de inmiddels overleden Walter Lippmann ooit
uitte, namelijk dat kiezers niet in staat zijn om door te dringen tot
wat hij `fictieve persoonlijkheden' noemde. Hij stelde dat kiezers
slecht worden bediend door vleierij en oppervlakkige bewondering en dat
zij verraden worden door een slaafse hypocrisie, die hen voorhoudt dat
hun stem bepalend is voor wat waar en onwaar, juist en onjuist is.

Lippmann voorzag een `inbreuk op de constitutionele orde' als aanleiding
voor de plotselinge en catastrofale achteruitgang van de westerse
samenleving. We zijn in korte tijd ver afgedwaald\ldots{} Wat we
meemaken is niet alleen verval -- een groot deel van de oude structuur
valt uiteen -- maar het lijkt wel of we op de drempel van een
historische catastrofe staan.\footnote{Ibid., p.~15.}

Het probleem schuilt erin dat politieke oordelen minder een reactie zijn
op de werkelijkheid dan op een door het algemene publiek geconstrueerde
schijnrealiteit over fenomenen die buiten hun directe kennis
liggen.\footnote{Paul Roazen, \emph{Inleiding}, in Lippmann, op. cit.,
  p.~xv.} Toch is het een vergissing om je te laten beperken tot wat
anderen waarnemen. Zelfs als je er onverschillig tegenover staat -- of
als Vincent Foster vermoord zou zijn en zijn dood kunstmatig in scène is
gezet door de hoogste politiediensten en verantwoordelijke
functionarissen van de Amerikaanse regering -- inclusief de huidige
speciale aanklager, Kenneth Starr, overweeg je wellicht bewijs van een
breder patroon van banden tussen georganiseerde misdaad en het Witte
Huis.

Op de lange termijn ondermijnt politieke corruptie op het hoogste niveau
het traditionele geloof in de mogelijkheden van de democratie om
publieke problemen doelbewust aan te pakken. In het informatietijdperk
gaat eerlijkheid van de overheid veel meer wegen dan haar omvang of
macht. De meeste diensten die overheden vroeger verleenden, zullen in de
komende eeuw naar de private sector verschuiven. Toch wijzen bewijzen
van over de hele wereld erop dat je op lange termijn nauwelijks kunt
vertrouwen op een systeem met corrupte leiders voor de veiligheid van je
gezin en je investeringen.

Zoals Morris opmerkt, ``{[}D{]}e Clintons zijn niet louter
symptomatisch, maar staan symbool voor het bredere tweepartijenstelsel
dat aan het einde van de eeuw op een doodlopende weg stuit.''\footnote{Morris,
  op. cit., p.~469.} In zijn essay over corruptie betoogt Vito Tanzi dat
``de enige manier om corruptie te weren is door de omvang van
overheidsinterventie aanzienlijk te verminderen.''\footnote{Fiorentini
  en Peltzman, op. cit., p.~16.} De informatierevolutie zal die omvang
drastisch verkleinen en daarmee hoop bieden op een wedergeboorte van
moraliteit en eerlijkheid. Een andere voor de hand liggende consequentie
voor de morele orde is de toegenomen kwetsbaarheid, doordat cyberhandel
en virtuele ondernemingen voortaan met onbreekbare encryptie
communiceren. Dieven binnen een organisatie -- zelfs binnen een virtuele
omgeving -- zullen moeilijker te traceren zijn en het wordt vrijwel
onmogelijk om gestolen geld terug te krijgen, zeker niet wanneer dit in
het geheim wordt vergaard door de verkoop van bedrijfsgeheimen, patenten
of andere waardevolle economische activa.

Misdaad loont, en velen vinden het aantrekkelijk om legale, productieve
activiteiten aan te vullen met illegale, roofzuchtige ondernemingen. In
tegenstelling tot de situatie zoals die in de afgelopen twee eeuwen in
westerse samenlevingen gangbaar was, worden criminelen tegenwoordig niet
meer slechts gezien als buitenbeentjes zonder sociale status. Wanneer
misdaad winstgevend is, duikt er doorgaans een elite van criminelen op,
omdat er steeds minder sociale afkeuring heerst voor crimineel gedrag.
Denk aan de Siciliaanse maffia, die samen met talloze drugsdealers --
die lokaal personeel inhuren tegen torenhoge tarieven -- op eigen
grondgebied zowel respect als populaire steun geniet.

\section{De morele orde en haar
vijanden}\label{de-morele-orde-en-haar-vijanden}

Sterke samenlevingen rusten op een degelijke morele basis. Elk onderzoek
naar de geschiedenis van economische ontwikkeling illustreert de nauwe
samenhang tussen morele en economische factoren. Landen en groepen die
succesvol ontwikkelen, slagen er deels in omdat zij een ethiek hanteren
die zelfredzaamheid, hard werken, gezins- en maatschappelijke
verantwoordelijkheid, een hoge spaarzin en eerlijkheid bevordert. Dit
geldt ook voor kleinere sociale subgroepen. Het zakelijk succes van
Joden -- met name religieuze Joden --, van de Puriteinen in New England,
van de Quakers in de Britse zakenwereld in de achttiende en negentiende
eeuw, en van de Mormonen in het moderne Amerika, toont stuk voor stuk de
economische voordelen aan die voortvloeien uit een cultuur met een
stevig moreel fundament.

Neem als voorbeeld de Quakers. Deze groep werd zakelijk zeer succesvol
en stond om diverse redenen vooral bekend als bankiers. Zij hanteerden
de hoogste norm van betrouwbaarheid: in plaats van eden af te leggen,
beschouwden zij elke zakelijke toezegging als even bindend als een eed.
`Mijn woord is mijn borg' was voor hen een onwrikbaar principe. Ze kozen
voor een rustige, fatsoenlijke en zuinige levensstijl en beschouwden het
als hun religieuze plicht geld niet te verspillen aan de ijdelheden van
deze wereld. Ze vermeden ruzies en waren ervan overtuigd dat oorlog
altijd zondig is. Daarnaast vonden ze dat een zakenman als morele
verplichting eerlijke waarde moet leveren, wat hen als handelaren een
reputatie opleverde voor het bieden van hoge kwaliteit tegen gematigde
prijzen. De uitdrukking `caveat emptor' -- laat de koper maar oppassen
-- volstond voor hen niet. In een tijd waarin de meeste kooplieden hun
handelstheorie baseerden op hoge prijzen en grote marges, leidde de
Quaker-moraliteit onvermijdelijk tot een strategie met lage winstmarges
en een hoge omloopsnelheid. Zoals Henry Ford later aantoonde, kan zo'n
beleid potentieel veel winstgevender zijn. Ze kozen bewust voor deze
benadering omdat zij het als hun plicht zagen hun klanten niet te
bedriegen en ontdekten al snel dat dit tegelijkertijd de beste manier
bleek om hun zaken uit te breiden. Zo bleken de Quakers uitstekende
zakenpartners te zijn: hun klanten keerden keer op keer terug, waardoor
beide partijen voordeel hadden. Bovendien gaven ze als een zuinige
gemeenschap die haar verplichtingen nakwam de Quakers een voorsprong als
bankiers, en het lidmaatschap van de Quakers op zich vormde al een
betrouwbaar bedrijfsinstrument.

Helaas kan het succes waarmee zulke zakelijke voordelen worden behaald,
er uiteindelijk voor zorgen dat ze zichzelf ondermijnen. Landen
doorlopen immers een cyclus -- zoals beschreven in de sociologische
theorie van Adam Ferguson uit de achttiende eeuw -- die begint bij
armoede en hard werken, overgaat in rijkdom, luxe en decadentie, en
uiteindelijk leidt tot achteruitgang. De oude Romeinen keken met heimwee
terug op de deugden van de republikeinse periode waarin het rijk werd
opgebouwd en betreurden de luxe en luiheid die zij als oorzaak van hun
ondergang beschouwden. Deze ondermijning van de ijverige deugden door
welvaart kan verrassend snel plaatsvinden. Hoewel de Duitsers nog steeds
een capabel en efficiënt volk zijn, werken zij onmogelijk meer zo hard
als toen zij hun land herbouwden na de verwoestende nederlaag in 1945.
In slechts twee generaties stapten zij over van lange werkdagen -- bijna
met blote handen in tijden van acute armoede -- naar korte werkdagen in
ruil voor het hoogste loon en de beste sociale voorzieningen op aarde.
In oktober 1995 tekenden zestien Duitse werkgeversverenigingen de
Petersburg-verklaring, een overzicht van goed onderbouwde klachten die
de achteruitgang van de industriële moraal in Duitsland weerspiegelen.

\begin{quote}
In 1995 bereikte de belastingdruk in Duitsland recordhoogten, mede door
de solidariteitstoeslag en de bijdragen aan de langdurige
zorgverzekering. Met een totale ondernemingsbelasting van meer dan 60
procent ligt Duitsland ver boven het internationale gemiddelde van 35
tot 40 procent. Praktijken in de publieke sector, zoals gereguleerde
promoties, een baan voor het leven en hogere pensioenuitkeringen, moeten
wijken voor vrijemarkteisen op het gebied van meritocratische promotie
en beloning. Omdat Duitsland de hoogste arbeidskosten ter wereld kent,
moet het loonbeleid bijdragen aan het terugdringen van de werkloosheid
door de kosten voor ondernemingen te verlagen. Loonstijgingen zouden
gebaseerd moeten worden op concurrentievermogen en productiviteit, en
het gedrag van de vakbonden dient te veranderen. Het jaarlijkse ritueel
van campagnes, eisen, mobilisatie van werknemers, dreigementen en
waarschuwingsstakingen is schadelijk.
\end{quote}

Die bezorgdheid dat vooral de Duitsers -- in het bijzonder de jongeren
en de erfgenamen van de welvaart -- de gewoonte van hard werken dreigen
te verliezen, deelt bondskanselier Kohl.

Het arbeidscontract bij Volkswagen kent autowerkers ter wereld het
hoogste loon toe, waarboven sociale zekerheidsbijdragen komen, in ruil
voor een werkweek van 28 uur -- vier dagen van elk zeven uur. Het
naoorlogse Duitsland exporteert nu massaal banen. Halverwege de
negentiende eeuw stonden de Britten bekend als de meest efficiënte
industriële natie, een reputatie die zij een eeuw later zeker hadden
ingeleverd. De cyclus van welvaart ondermijnt onvermijdelijk de deugden
van hard werken en bescheiden verwachtingen die kenmerkend zijn voor de
vroege fasen van een succesvolle industriële opbouw. Naties slagen er
vaak niet in hun oorspronkelijke deugden te behouden, net zoals
individuen bij een te gemakkelijke vorm van succes hebberig en lui
kunnen worden.

Wereldwijde investeringen prijzen terecht deze ijverige deugden en
straffen wie hebberig en lui worden, zoals het hoort. Men kan stellen
dat een verstandige investering zowel moreel als financieel verantwoord
moet zijn. Die Engelsman uit de achttiende eeuw, die investeerde in het
kapitaal van een Quakerbank, boekte naar alle waarschijnlijkheid groot
succes. In de negentiende eeuw stopten de Quakers hun geld in
chocoladebedrijven, ervan overtuigd dat cacao gezonder was dan alcohol.
Dat was waarschijnlijk zo, maar investeren in Fry's of Cadbury's bleek
eveneens een slimme zet. Investeerders dienen oppassen dat ze niet
verstrikt raken in perioden van decadentie. Hoewel Duitsland op de
Europese markt sterk staat en beschikt over geavanceerde industriële
kennis, beperken hoge arbeidskosten en korte werktijden het toekomstige
potentieel van het land.

Sociale moraliteit en economisch succes gaan hand in hand. Maar welke
factoren versterken juist de sociale moraal en welke ondermijnen die?
Arnold Toynbee, de invloedrijke filosofisch-historicus uit de eerste
helft van de twintigste eeuw, introduceerde de theorie van uitdaging en
respons. Uitdagingen stuwen samenlevingen vooruit en brengen de deugden
naar boven waarvan men niet eerder wist dat ze daar in schuilden.

Mensen beseffen al lang dat moeilijke tijden vaak gezondere reacties
oproepen dan periodes van overvloed. We streven er allemaal naar het
comfortabel te hebben; we dromen van een fijne woning, een baan waar we
voldoening uit halen en een stevige spaarpot. De strijd om deze doelen
te bereiken levert zijn vruchten af. We studeren hard, trainen onszelf
en werken keihard. Tegelijkertijd moet het loonbeleid de werkloosheid
bestrijden door bedrijfs-kosten te verlagen. Loonstijgingen moeten
gebaseerd zijn op werkelijke concurrentiekracht en productiviteit, en
het gedrag van de vakbonden verdient een herziening. Het jaarlijkse
ritueel van campagnes, eisen, werknemersmobilisaties, dreigementen en
waarschuwingstakingen werkt immers schadelijk.

Voor velen blijkt de strijd om die doelen uiteindelijk aantrekkelijker
dan het resultaat zelf. Begin deze eeuw behandelde de gerenommeerde
Zwitserse psycholoog Carl Jung een Amerikaanse zakenman die als jongeman
dezelfde ambities koesterde. Hij werkte onvermoeibaar om een eigen
bedrijf op te zetten en genoeg geld te verdienen zodat hij op
veertigjarige leeftijd met pensioen kon. Hij trouwde met een jonge,
aantrekkelijke vrouw, kocht een prachtig huis, stichtte een gezin en
verkocht uiteindelijk met groot succes zijn bedrijf, zodat hij als
rijke, onafhankelijke man zonder zorgen met pensioen ging. In het begin
omarmde hij zijn vrijheid en deed hij alles wat hij zichzelf had
beloofd. Hij nam zijn gezin mee op reis door Europa, bezocht
kunstgalerijen en dergelijke. Maar na verloop van tijd vervaagden die
interesses en het gevoel van vrijheid zelf. Hij keek terug op de periode
dat hij `niet vrij' was -- toen hij dag en nacht aan zijn bedrijf werkte
en alle gebruikelijke zakelijke zorgen had -- en beschouwde die tijd als
de gelukkigste van zijn leven. Uiteindelijk raakte hij in een depressie,
waarna zijn vrouw hem naar Jung bracht. Jung stelde vast dat hij geen
uitlaatklep had voor zijn creatieve energie, die naar binnen gekeerd hem
verwoestte. Hoewel de diagnose waarschijnlijk klopte, leidde dat niet
tot genezing; de zakenman herstelde nooit van zijn zenuwinzinking.

Voor veel mensen is de strijd belangrijker dan het uiteindelijke
resultaat; wij zijn gemaakt voor actie, en succes kan uiteindelijk
teleurstellend blijken te zijn. Ambitie -- wat voor vorm die ook
aanneemt -- stuwt ons in de strijd, maar die strijd blijkt vaak
bevredigender dan het beoogde eindresultaat, zelfs wanneer we ons doel
bereiken. Natuurlijk slagen de meesten van ons er slechts gedeeltelijk
in onze dromen waar te maken. We hebben niet zoveel geld als we zouden
willen en wonen niet in het perfecte droomhuis. We moeten genoegen nemen
met minder.

Men kwam tot het besef dat deugdzaamheid een dynamisch begrip is -- dat
ze vooral wordt gevormd door de inzet en niet louter door het behaalde
resultaat -- en dit inzicht ontwikkelde zich op uiteenlopende manieren
in de negentiende eeuw. Er circuleert een bekend gedicht van Arthur Hugh
Clough dat tijdens de Tweede Wereldoorlog veel troost bood te midden van
het leven en de dood. Het valt op dat het zelfmoordcijfer in de
strijdende landen in die perioden daalde; zelfs het voeren van oorlog
bleek soms beter dan het wegzinken in de verlamming van inactiviteit.

\begin{quote}
\emph{Zeg niet: de strijd is zinloos,\\
Het zwoegen en de littekens tellen niet voor niets,\\
De vijand geeft niet toe en verliest geen kracht,\\
En, zoals altijd, blijft hij voortbestaan.}
\end{quote}

\begin{quote}
\emph{Als hoop je op het verkeerde been zet, kunnen angsten je eveneens
misleiden;\\
Misschien schuilt er, verborgen in die rook, dat je kameraden de
vluchtenden achterna zitten,\\
En -- ware het niet voor jou -- al het slagveld bezetten.}
\end{quote}

\begin{quote}
\emph{Want terwijl de vermoeide golven tevergeefs beuken,\\
Lijken hier geen duidelijke signalen naar voren te komen;\\
Ver weg, gevormd door kreken en inhammen,\\
stromen ze rustig binnen en overspoelen de zee.}
\end{quote}

\begin{quote}
\emph{En niet uitsluitend via oosterse ramen,\\
Wanneer het daglicht doorbreekt, baant het zich een weg naar binnen,\\
Aan de horizon klimt de zon traag, oh zo traag,\\
Maar kijk naar het westen: het land straalt helderheid uit.}
\end{quote}

Deze energieke competitie spreekt nog steeds tot de moderne
gevoeligheid. Immers, velen van ons leiden hun leven als een
voortdurende strijd om kansen in een vaak vijandige omgeving te
benutten. We bevinden ons allemaal in een concurrerende wereld en de
meesten kiezen er niet voor om er buiten te blijven. Natuurlijk bestaat
er ook een contemplatieve, spirituele inslag, maar die komt zelden voor.

Een vergelijkbare 19e-eeuwse visie op deze dynamische moraal verwoordde
William James, de vooraanstaande Amerikaanse filosoof, in een toespraak
voor de Yale Philosophical Club in 1891:

\begin{quote}
Het fundamentele verschil in het morele leven van de mens ligt in het
onderscheid tussen een ontspannen en een inspannende stemming.\\
Als we in een ontspannen stemming verkeren, staat het afwijken van het
actuele kwaad als doorslaggevende overweging voorop.\\
De inspannende stemming daarentegen zorgt ervoor dat we het actuele
kwaad links laten liggen, mits het hogere ideaal wordt bereikt.\\
Waarschijnlijk bezit iedere mens het vermogen voor een inspannende
stemming, maar bij sommigen komt dat moeilijk tot uiting.\\
Deze stemming verlangt ernaar gewekt te worden door intensere passies,
door grote angsten, hartstocht en verontwaardiging, of door de diep
doordringende roep van hogere trouwswaarden als rechtvaardigheid,
waarheid en vrijheid.\\
Voor de visie ervan is een krachtige verlichting noodzakelijk; en een
wereld waarin bergen instorten en valleien oprijzen, biedt geen
gastvrije omgeving voor die stemming.\\
Daarom kan deze stemming bij een eenzame denker eeuwig sluimeren zonder
ooit ontwaakt te worden.\\
Zijn uiteenlopende idealen, waarvan hij weet dat het slechts
persoonlijke voorkeuren zijn, dragen vrijwel dezelfde symbolische
waarde, waardoor hij er naar hartenlust mee kan jongleren.\\
Dit verklaart bovendien dat in een puur menselijke wereld zonder God de
oproep tot onze morele energie niet zijn maximale stimulerende kracht
bezit.\\
Het leven vormt, zelfs in zo'n wereld, een oprechte ethische symfonie;
maar die speelt zich af binnen het beperkte bereik van een paar arme
octaven, waarbij de oneindige rijkdom aan waarden niet de ruimte krijgt
zich te openbaren.
\end{quote}

William James was ervan overtuigd dat de dynamische moraal -- waarin
doen boven louter zijn en handelen boven afzien van actie centraal staan
-- ook toepasbaar is binnen de religieuze sfeer. Tevens komt in het werk
van Adam Smith (1776) een krachtige ontwikkeling naar voren in de moraal
van competitie en overleving -- de morele doctrine die de basis vormt
van de moderne, wereldwijde economische orde -- een centraal thema dat
zorgvuldig heroverwogen dient te worden.

Het darwinisme gaat ervan uit dat soorten overleven doordat ze zich
aanpassen aan hun omgeving en dat het proces van natuurlijke selectie
hun kenmerken vormgeeft. Bij dieren komt dat proces door willekeurige
mutaties tot stand, waarvan inmiddels bekend is dat ze onderdeel zijn
van een genetisch proces waar Darwin zelf slechts op kon raden. Het
voortbestaan van menselijke samenlevingen berust echter op culturele
keuzes, gebaseerd op menselijke intelligentie. Cultuur transformeert de
samenleving op dezelfde wijze als genen andere soorten veranderen.
Hierdoor kunnen in onze samenlevingen veranderingen veel sneller
plaatsvinden; het draait immers niet om vele generaties, zoals bij
willekeurige genetische mutaties het geval is.

In tegenstelling tot dieren, die via natuurlijke selectie evolueren,
ontwikkelden mensen een vorm van culturele selectie. Sommige
samenlevingen voerden in een bepaald tijdvak baanbrekende technologieën
in die hen een doorslaggevend voordeel gaven bij het vergaren van
welvaart of het verkrijgen van macht.

Het culturele voordeel van nieuwe technologieën blijkt vaak bepalend --
denk maar aan de mens in het ijzertijdperk tegenover die in het bronzen
tijdperk, of de `elektronische' mens tegenover de `mechanische' mens.

Adam Smith was misschien niet de eerste econoom die het welzijn van
naties reduceerde tot individuele handelingen, maar hij verwoordde dit
op de meest bondige en gezaghebbende wijze:

\begin{quote}
Ieder individu streeft er voortdurend naar om het beste uit het kapitaal
te halen dat hij bezit. Hij werkt immers voor zijn eigenbelang -- niet
voor dat van de samenleving. Toch leidt zijn zoektocht naar persoonlijk
gewin er onvermijdelijk toe dat hij kiest voor hetgeen uiteindelijk het
meeste voordeel voor de gemeenschap oplevert.
\end{quote}

Thomas Malthus, de pionier op het gebied van bevolkingsstudies, stelde
vast dat Adam Smiths redenering niet alleen op de economische
ontwikkeling van naties van toepassing is, maar ook op het voortbestaan
van menselijke populaties. Hij is vooral beroemd om zijn stelling dat
`de bevolking, wanneer zij niet wordt ingeperkt, toeneemt volgens een
geometrisch tempo, terwijl het levensonderhoud slechts rekenkundig
groeit. Al een korte kennismaking met getallen toont de gigantische
verschillen tussen deze groeipatronen aan.'

Lang voor Darwin wees Malthus erop dat hetzelfde principe in de gehele
natuur geldt:

\begin{quote}
De natuur verspreidt de zaden van het leven in dieren- en plantenrijken
met een ongekende overvloed en vrijgevigheid. Tegelijkertijd gaat ze
zuinig om met de beschikbare ruimte en de voeding die nodig is om die te
laten groeien. Die levenskiemen, gehuisvest in een klein stukje aarde en
omgeven door overvloedig voedsel en ruime groeimogelijkheden, zouden in
enkele duizenden jaren miljoenen werelden kunnen vormen. Maar de
allesomvattende wet van de natuur houdt hen binnen vaste grenzen.
\end{quote}

Aan het einde van de achttiende eeuw bleek al dat de wereld zich op een
dynamische manier ontwikkelt -- een inzicht dat al in de tijd van
Adam~Smith en Malthus werd opgemerkt. De mens, als slechts één van de
vele levensvormen, moet concurreren vanwege de spanning tussen een
onbeperkte voortplantingscapaciteit en een beperkt vermogen om voedsel
te produceren. Het voortbestaan van menselijke samenlevingen, net als
dat van diersoorten, vraagt om een succesvolle aanpassing aan de
omgeving. Een dynamische moraliteit pakt daarom de uitdagingen van deze
aanpassing actief aan, doordat mensen hun handelen afstemmen op de
kansen die hun omgeving biedt en zo de beschikbare middelen optimaal
benutten.

Malthus merkte op dat de ideeën van Adam~Smith de wereld wezenlijk
hadden veranderd en stelde dat zijn eigen betoog over de bevolking niet
geheel origineel was: `De principes waarop het berust, zijn deels
uitgelegd door Hume en deels door Dr.~Adam~Smith.' Tevens benadrukte hij
dat deze voortdurende strijd om te overleven niet louter een praktische,
maar in essentie een morele zaak is. De afsluitende alinea van het
`Essay' uit 1798 luidt als volgt:

\begin{quote}
Het kwaad bestaat in de wereld niet om wanhoop te zaaien, maar om ons
tot actie te bewegen. Wij mogen het niet passief accepteren, maar moeten
ons inspannen het te vermijden. Het is zowel in ons eigen belang als
onze plicht om met alle macht het kwaad uit onszelf en uit de kring
waarop wij invloed uitoefenen weg te nemen; hoe vastberadener wij deze
taak volbrengen, des te wijzer en succesvoller onze inspanningen worden
en hoe meer wij onze eigen geest weten te verheffen, zodat wij
vollediger lijken te voldoen aan de wil van onze Schepper.
\end{quote}

Men kan het belang van dit betoog verder illustreren met Darwins
samenvatting van hoofdstuk~3 uit zijn baanbrekende boek \emph{On the
Origin of Species}, voor het eerst gepubliceerd in 1859. Hij noemde dit
cruciale hoofdstuk `\emph{Struggle for Existence}.' De onderwerpstitels
luiden als volgt: `Betekenis van natuurlijke selectie -- de term in
brede zin gebruikt -- geometrische vermenigvuldigingskracht -- snelle
toename van ingeburgerde dieren en planten -- de aard van de
groeibelemmeringen -- universele concurrentie -- effecten van het
klimaat -- bescherming door het aantal individuen -- complexe relaties
tussen alle dieren en planten in de natuur -- de strijd om het leven,
die het hevigst woedt tussen individuen en variëteiten van dezelfde
soort -- vaak nog feller tussen soorten van hetzelfde geslacht -- de
relaties tussen organismen, die de belangrijkste van alle relaties
vormen.'

Sinds 1776 blijkt al duidelijk dat naties hun welvaart het best
optimaliseren wanneer zij mensen de vrijheid bieden hun eigen
kapitaalrendement te maximaliseren in een systeem van vrije
concurrentie. Sinds 1798 weten we dat het voortbestaan van populaties
afhangt van samenlevingen die economisch en politiek voldoende succes
boeken om in hun eigen onderhoud te voorzien, zich te beschermen tegen
infectieziekten en hun bevolking in oorlogstijd te verdedigen. Sinds
1859 is gebleken dat het hele drama van het leven -- of het nu om
mensen, dieren of planten gaat -- een voortdurende strijd om te
overleven inhoudt, waarbij nauw verwante soorten of culturen vaak
elkaars grootste rivalen vormen. Deze strijd vraagt om een dynamische
moraliteit die het kwaad actief afweert in plaats van er enkel reactief
mee om te gaan.

Deze ideeën bleken zo krachtig dat, sinds hun ontstaan, niemand nadacht
over de aard van de mensheid of morele vraagstukken zonder er meteen op
in te springen. Karl Marx hechtte evenveel waarde aan de strijd om te
overleven als Charles Darwin, maar hij interpreteerde die als een oorlog
tussen sociale klassen, gevormd door economische krachten. Ook
Adolf~Hitler geloofde in de strijd om te overleven en bekijktte zijn
politieke carrière bijna uitsluitend vanuit dat perspectief. Hij vond
echter dat de strijd tussen verschillende rassen werd uitgevochten.
Marx, Lenin, Stalin, Mao en Hitler vertoonden allemaal kenmerken van
sociaal-Darwinisme, omdat zij de strijd om te overleven -- `Mein Kampf',
zoals Hitler dat noemde -- als de kern van de politieke realiteit
beschouwden. Marxisten zagen sociale klassen als afzonderlijke groepen,
terwijl nazi's rassen op een vergelijkbare wijze benoemden.

Dit leidt echter niet tot een dynamische moraliteit, zoals Malthus had
beoogd, maar juist tot een dynamische immoraliteit. Zowel het marxisme
als het nazisme wilden de strijd om te overleven aanpakken, maar deden
dat door de concurrentie te vernietigen. Ze drongen op onbekend terrein
door conflicten te zaaien tussen klassen die om sociale macht streden of
tussen rassen die zij beschouwden als economische uitbuiters (zoals vaak
door anti-semiten jegens joden wordt beweerd) of als een gevaarlijke
onderklasse (zoals de angst die blanke tegenstanders voor zwarten
koesterden). De Tweede Wereldoorlog vormde voor Adolf~Hitler -- die
mislukte -- een poging om voor het Duitse volk een overlevingsvoordeel
te creëren door potentiële rivalen, met name Slavische volkeren en
joden, uit te schakelen. Paradoxaal genoeg kwam Duitsland met een
nederlaag in de oorlog uiteindelijk beter af dan wanneer een
nazi-overwinning ooit was bewerkstelligd.

Het alternatief voor destructieve interferentie in de concurrentie is
samenwerkende concurrentie -- het kernidee van Adam~Smith, maar ook van
Malthus en William~James. Het model van destructieve concurrentie is de
veroveraar, die zijn rivalen verplettert om hun bezittingen binnen te
halen -- wat soms inhoudt dat hij hun landen overneemt en hun volk tot
slaaf maakt. Daartegenover staat de handelaar, het voorbeeld van
samenwerkende concurrentie. Een handelaar zorgt ervoor dat de klant
tevreden is, want alleen tevreden klanten komen terug voor meer
transacties. Daarnaast is het gunstig als de klant welvarend is,
aangezien een welvarende klant meer kan kopen. Verovering betekent de
vernietiging van de ander, terwijl handel draait om het vervullen van de
behoeften van de ander. Nu moderne technologie verovering tot een
buitengewoon risicovol beleid heeft gemaakt, vormt handel de enige
rationele benadering van de overlevingsvraagstukken.

Deze wederzijdse afhankelijkheid krijgt extra kracht door een ander
belangrijk principe van Adam~Smith, namelijk de specialisatie van
functies. `The Wealth of Nations' opent met een beroemde passage waarin
Smith opmerkt dat de grootste vooruitgang in de productieve krachten van
arbeid en het merendeel van de vaardigheid, behendigheid en het
oordeelsvermogen die ergens wordt toegepast, schijnbaar voortkomt uit
arbeidsdeling. Hij merkt bijvoorbeeld op dat het ambacht van het
vervaardigen van een speld is opgedeeld in zo'n achttien verschillende
handelingen, die in sommige fabrieken door diverse arbeiders worden
uitgevoerd. Hoe verder de specialisatie, hoe efficiënter de productie,
maar zo'n economie is vanzelfsprekend sterk afhankelijk van
samenwerking: succes vraagt om gezamenlijk optreden.

Een succesvolle sociale moraliteit moet aan een aantal voorwaarden
voldoen. Ze moet krachtig zijn -- een zwakke moraliteit is immers
kwetsbaar en ineffectief -- en de strijd om te overleven bevorderen op
een samenwerkende in plaats van een vernietigende manier. Hitler
hanteerde een krachtige overlevingsmoraliteit, maar zijn destructieve
drang heeft bijna zijn eigen samenleving verwoest. Verder moet een
degelijke sociale moraliteit dynamisch zijn, zodat hij meebeweegt met de
voortdurende veranderingen in moderne technologie en alle hedendaagse
sociale systemen, en bovendien economisch efficiënt opereren. De mix van
egalitaire en autoritaire ideeën binnen het Leninistische systeem bleek
simpelweg niet te werken. Dit omvat echter niet alle aspecten van
sociale moraliteit: ze moet ook de samenleving prettig en samenhangend
houden. Bovendien moeten moraliteiten zich flexibel kunnen aanpassen en
overleven -- een fragiele moraliteit mag in onze generatie lijken te
werken, maar in de volgende niet standhouden, terwijl een starre,
traditionele moraliteit te inflexibel blijkt om mee te bewegen met de
veranderende sociale structuur. Tegelijkertijd levert een puur
relativistische benadering geen moraliteit op, omdat zij geen duidelijke
richtlijnen verschaft voor gewenst gedrag.

Allereerst plaatsen we de sociale moraal in een bredere context. Een
hechte gemeenschap, zelfs een virtuele, is afhankelijk van een moraal
die breed wordt gedeeld. De meest succesvolle perioden in de
geschiedenis ontstaan wanneer de collectieve moraal algemeen gedragen
wordt. Zo'n moraal vervult niet alleen specifieke functies -- zoals het
verminderen van criminaliteit en het versterken van familie- en sociale
structuren -- maar geeft burgers ook een gevoel van doelgerichtheid en
richting. Historische eensgezindheid over moraal hangt vaak samen met
het bestaan van een dominante religie, of het nu gaat om een
staatsgodsdienst die in vroege tijden het overleven van een verspreid
volk waarborgde, de islam met haar sociale regels, het middeleeuwse
katholicisme of het protestantisme in het vroege New England. De
begrippen volk, moraal en religie zijn onlosmakelijk verbonden en
versterken elkaar.

In zo'n morele samenleving kan iedere burger binnen een kader van
sociale steun aan persoonlijke doelen werken. Natuurlijk kunnen de
morele voorschriften soms arbitrair overkomen -- of op zijn minst zo
worden ervaren door buitenstaanders. Een orthodoxe jood verliest de
vrijheid om varkensvlees of schelpdieren te eten en op de sabbat te
werken. Een trouwe katholiek wordt beroven van de mogelijkheid
kunstmatige anticonceptie te gebruiken, om nog maar te zwijgen van
abortus. Een moslim mag geen alcohol nuttigen. En een vrome confuciaan
moet mogelijk een langdurige, oncomfortabele rouwperiode doorstaan na
het verlies van zijn eerwaardige vader -- waarbij zelfs Confucius
waarschuwde dat rouwrituelen te ver zouden kunnen gaan. Toch zien de
volgelingen van deze geloofssystemen hun voorschriften als een geringe
opoffering voor de voordelen van een gedeeld en samenhangend
wereldbeeld, waarin ieder individu een vaste plaats heeft. Zo kan een
orthodoxe jood stellen dat het in acht nemen van de sabbat een kleine
opoffering is in ruil voor de kracht van de wet en de saamhorigheid
binnen het joodse gezin.

John Locke en de vroege voorvechters van vrijheid streefden naar een
gedeelde moraal in een tolerante samenleving. Zij waren ervan overtuigd
dat elke maatschappij -- ongeacht haar aard -- regels nodig heeft, maar
dat men enkel verplicht zou moeten worden de essentiële normen te
volgen, mits deze zijn gebaseerd op de hoogste rede. Ze erkenden dat
enige dwang in de sociale moraal onvermijdelijk is -- zeker ter
bescherming van leven en eigendom -- omdat geen enkele samenleving kan
voortbestaan zonder veiligheid. Tegelijkertijd pleitten zij voor vrijwel
absolute tolerantie betreffende persoonlijke keuzes die het welzijn van
anderen niet raken. Een confuciaan die veertig dagen om zijn vader
rouwde, kon naast een jood leven -- die de sabbat eerde -- zonder elkaar
in de weg te lopen of de ander te dwingen de eigen religieuze gebruiken
over te nemen.

Uit de combinatie van een sociale moraal voor de essentiële zaken en de
tolerantie voor persoonlijke beslissingen ontstaat een kernethiek
waaraan alle burgers gebonden zouden moeten zijn, naast een vrijwillige
ethiek die zij als individuen of als leden van subgroepen binnen de
samenleving omarmen. Wanneer een benedictijner monnik geloften van
armoede, kuisheid en gehoorzaamheid aflegt, doet hij dat als lid van
zijn kloostergemeenschap. Hij verwacht niet dat alle katholieken -- laat
staan al zijn medeburgers -- dezelfde geloften afleggen of dezelfde
regels naleven. Hij volgt de bevelen van zijn abt, maar gaat er niet van
uit dat iemand buiten het klooster hier aandacht aan besteedt. Het
naleven van minder essentiële aspecten van de moraal kan vrijwillig
blijven, maar de kernwaarden moet iedereen delen; wie die fundamentele
moraal weigert, schaadt zowel zichzelf als de samenleving. In het
uiterste geval biedt een maatschappij die overspoeld wordt door rovers
die zonder aarzelen vermoorden -- zoals in grote delen van Europa ná de
val van het Romeinse Rijk -- niemand een bevredigend bestaan, zelfs de
rovers zelf niet, omdat zij voortdurend onder druk staan van andere
moordenaars. Dit scenario doet zich ook in bepaalde stedelijke gebieden
in de huidige Verenigde Staten voor. Anarchie vormt immers géén ideaal
model, want zonder handhaving van de wet ontbreekt de menselijke
veiligheid.

Als men kijkt naar de krachten die de moraal van de samenleving
ondermijnen, dient men de kernethiek als uitgangspunt te nemen -- een
moraal die in de meeste moderne religies globaal min of meer hetzelfde
is. Ten minste twee van de Tien Geboden -- `Gij zult niet doodslaan' en
`Gij zult niet stelen' -- gelden als universele normen, zowel voor
christenen als voor joden. Zelfs bijna alle serieuze agnosten zien zowel
moord als diefstal, de ultieme bedreigingen voor leven en eigendom, als
ontoelaatbaar en erkennen dat de samenleving het recht heeft om hen die
beroven of vermoorden te straffen. Over de juiste strafmaat voor
individuele misdrijven kunnen zij wellicht van mening verschillen, maar
zij zijn het er unaniem over eens dat de samenleving een fundamenteel
recht heeft om besluiten te nemen omtrent bestraffing.

John Locke verwoorde het treffend: ieder mens heeft recht op `leven,
vrijheid en bezit.' In 1776 voegde Thomas Jefferson hieraan toe: `het
nastreven van geluk.' Hoewel dit een mooie en nobele uitdrukking is,
spreken `leven, vrijheid en bezit' meer over de praktische basis dan
`het nastreven van geluk.' De maatschappij is immers in hoge mate
afhankelijk van de rechten op leven en eigendom. De geschiedenis leert
ons dat deze fundamentele rechten uitsluitend in een vrije samenleving
beschermd kunnen worden. Wanneer de staat alomtegenwoordig en al te
machtig is, wordt zij de grootste vijand van het leven -- zoals te
merken is in agressieworlogen -- en van individueel eigendom, doordat
zij een onevenredig deel van de nationale rijkdom opeist voor eigen,
vaak ongewenste en verspillerige doeleinden.

De kernmoraal krijgt in de meest vooruitstrevende landen flinke kritiek,
mede dankzij de krachten van de moderniteit die hen hun technologische
voorsprong geven. De Verenigde Staten vormen 's werelds voornaamste
technologische macht. Tot ver in de vroege jaren zestig beschouwden
velen -- waaronder de meeste Amerikanen -- het land als een moreel
voorbeeld voor de rest van de wereld. Vandaag de dag komt zo'n oordeel
zelden meer naar voren, zelfs niet onder Amerikanen die trots op hun
land zijn. Men kon, net als de rest van de wereld, niet luisteren naar
het proces tegen O.J. Simpson en tegelijk de VS zien als de eenvoudige,
deugdzame republiek waarmee het ooit begon.

Als men terugkijkt op de benamingen uit het oude Amerika, bleek dat deze
de noden van een grensmaatschappij weerspiegelden en zelfs in de grote
steden de houding van de burgers bepaalden. De grensgebieden ademen een
intrinsiek democratische sfeer uit: mensen voelen zich gelijkwaardig en
de vroege Amerikanen wierpen de Europese klassehiërarchieën overboord.
Zelfs dienstmannen, die vanuit Engeland als gevangenen naar Amerika
werden gestuurd, vestigden zich als zelfstandige ambachtslieden, boeren
of vrije arbeiders zodra hun contract afliep. De lonen waren hoger dan
in Europa en de kosten van levensonderhoud laag, al waren geïmporteerde
vervaardigde goederen prijzig. Men was in de grensgebieden sterk op
elkaar aangewezen; ondanks de hardheid van het bestaan kende het leven,
in Europese termen, voorspoed. Immigranten begonnen vaak als
laagbetaalde arbeiders in de sloppenwijken van Boston of New York, maar
wisten daar doorgaans al snel aan te ontsnappen en bereikten generatie
na generatie voorspoed. Na de burgeroorlog voelden zwarte Amerikanen
zich als behorende tot een andere groep immigranten, en velen omarmden
de Amerikaanse waarden en idealen. Hieruit groeide uiteindelijk de
zwarte middenklasse.

Die ambitie, gesterkt door de ervaringen aan de grens en de invloed van
zowel protestantse als katholieke kerken, vormde het patriottisme van de
Amerikanen. Ze waren ervan overtuigd dat ze in Gods eigen land woonden
-- een opvatting die volledig werd gevoed door democratische idealen en
het christelijk geloof en die leidde tot 's werelds eerste en meest
succesvolle democratie. Dat beeld kennen we allemaal; het komt tot
uiting in de bijna iconische voorstelling van Abraham Lincoln, hoewel
sommigen in het zuiden hem nog steeds zien als de man die de gruwelen
van de eerste moderne oorlog ontketende om te voorkomen dat vrije staten
een unie verlieten die ze niet langer vertrouwden.

Toch blijft het beeld van Lincoln -- ruig, eenvoudig, eerlijk en
welbespraakt -- het hoogste Amerikaanse ideaal en dient hij in wezen als
moreel voorbeeld. Veel Amerikanen ervaren nog steeds de krachtige
oorspronkelijke tegenstelling tussen de democratische energie van het
nieuwe land en de versleten hiërarchieën uit Europa. Buitenlanders zien
dit ideaal van een intrinsiek dynamische meritocratie nauwelijks terug
in hedendaagse steden als Los Angeles, New York, Houston of Washington,
hoewel er in de grote voorstedelijke gebieden en op het platteland nog
meer dan slechts sporen van te vinden zijn. De Amerikaanse puriteinse
ethiek, met al haar historische betekenis, houdt het best stand ten
noorden van de sneeuwgrens, terwijl het ondernemende dynamisme over het
hele land verspreid is.

Veel Amerikanen wijzen op het verval van de grote steden, die
broedplaatsen zijn geworden voor criminaliteit -- met name in verband
met de narco-industrie -- als het meest ernstige symptoom van een
afnemend gemeenschappelijk moreel besef. Ook erkennen zij dat er een
botsing plaatsvindt tussen uiteenlopende morele culturen die wedijveren
om hun gezag en aanspraak. De `politiek correcte' cultuur wijst vele,
maar niet alle, morele beginselen van de oude cultuur af. Zij legt
sterke nadruk op de rol en rechten van groepen die historisch gezien
uitgebuit werden door een dominante witte mannencultuur en verwerpt die
cultuur, hoewel deze juist de oprichtingscultuur van de Verenigde Staten
vormt.

De dominante mannencultuur in de eerste helft van de twintigste eeuw
draaide om het behoud van het kerngezin. Hierdoor behield de echtgenoot
en vader -- zij het vaak slechts symbolisch -- de leiding in huis,
terwijl de echtgenote of moeder in de praktijk het dagelijkse beheer
voerde en haar man als nominale meester onderdanig accepteerde. Ook op
de werkvloer oefende de man een flinke macht uit, een positie die de
feministische beweging tot op heden weliswaar aan de kaak heeft gesteld,
maar niet heeft weten om te draaien. Het grote belang dat aan het gezin
werd gehecht én de historische christelijke leer schreven abortus ten
stelligste verboden. Vroeger beschouwde de oude moraal abortus als een
onrechtmatige doding, volstrekt ontoelaatbaar -- een visie die haar
aanhangers nog steeds delen, terwijl voorstanders van de nieuwe moraal
juist het tegendeel inzien. In \emph{Roe v. Wade} stelde het
Hooggerechtshof het grondwettelijke recht op abortus -- dat voorheen als
een aangelegenheid voor de individuele staten gold -- vast op basis van
het recht op privacy, een doctrine die ver verwijderd is van enige
expliciete formulering in de Grondwet of haar amendementen.

Men interpreteerde de privacy van een vrouw zodanig dat zij zelf kon
bepalen of zij kinderen wilde krijgen, zonder rekening te houden met de
gevolgen voor het embryo. Het Hooggerechtshof achtte het embryo namelijk
niet als drager van grondwettelijke rechten -- in de late twintigste
eeuw beschouwde men embryo's als extraconstitutionele entiteiten,
vergelijkbaar met de positie van slaven in de eerste helft van de
negentiende eeuw. `Leven, vrijheid en de zoektocht naar geluk' gold
immers niet voor slaven, en de rechters in \emph{Roe v. Wade} pasten
deze bewoordingen dan ook niet toe op embryo's.

Het abortusdebat illustreert op treffende wijze het conflict tussen de
oude en de nieuwe moraal, al daagt een vernieuwde visie in tal van
andere domeinen de traditionele sociale organisatie uit. De traditionele
christelijke moraal, in zowel protestantse als katholieke kerken, legde
strikte normen vast voor seksuele rollen: relaties buiten of vóór het
huwelijk schonk men geen plaats, en ook genitale homoseksuele omgang
werd afgewezen. Lesbianisme kwam minder nadrukkelijk aan bod, daar men
het bestaan ervan nauwelijks erkende. Toen koningin Victoria er voor het
eerst van hoorde, weigerde zij koppig te geloven dat vrouwen zulke
relaties zouden hebben. Politieke correctheid vertegenwoordigt de moraal
van zogenaamd onderdrukte groepen. Homoseksuelen eisten gelijke
erkenning voor hun levensstijl en tartten de traditionele afkeuring van
hun seksuele gedrag. Men bestempelde `homofobie' als een verwerpelijke
vorm van vooroordeel, vergelijkbaar met rassendiscriminatie. Critici op
dit vlak worden door de nieuwe moraal net zo onaanvaardbaar geacht als
kritiek op zwarten, joden of vrouwen.

Tegelijkertijd doorbraken en schafte men ook andere seksuele taboes af.
In de jaren zestig ontwaakte een nieuwe golf van `vrije liefde', deels
dankzij de vermeende betrouwbaarheid van de anticonceptiepil voor
vrouwen, maar ook gestimuleerd door stemmingsveranderende drugs en
popmuziek. Hierdoor werd samenwonen buiten het huwelijk steeds
gangbaarder. Tegen de jaren negentig beschouwde men in Groot-Brittannië
-- een samenleving die vergeleken met de meeste delen van de Verenigde
Staten wat ouderwets was -- als volstrekt normaal dat prins Edward in
Buckingham Palace met zijn vriendin sliep in een stabiele, ongehuwde
relatie, vergelijkbaar met de manier waarop studenten in de jaren zestig
in hun studentenkamers samenleefden. Weinig mensen vonden het raar dat
koningin Elizabeth~II, hoofd van \emph{Church of England}, het gedrag
van haar jongste zoon tolereerde, terwijl de huwelijken van haar drie
oudste kinderen al waren verbroken. Wie hierover klaagde, noemde men
ouderwets en betweterig, terwijl velen nog steeds de traditionele moraal
prefereerden, ook al leefden zij er niet naar en verwachtten zij niet
dat hun kinderen dat vanaf jonge leeftijd zouden doen.

Binnen de politiek correcte beweging doemde ook een puriteinse inslag
op. Doordat deze beweging voortkwam uit de vermeende belangen van
vrouwen -- beschouwd als de meest onderdrukte groep -- ontwikkelde zij
een uitgesproken vijandigheid tegenover de mannelijke seksualiteit,
zowel in open agressieve uitingen als in gedragingen die men voorheen
als onschadelijk beschouwde. Sommige vrouwen waren ervan overtuigd dat
alle mannen van nature verkrachters zijn, waardoor de natuurlijke
afschuw voor verkrachting uitmondde in een algemene veroordeling van het
mannelijk geslacht. Anderen richtten hun aandacht op seksuele
intimidatie, een terechte klacht omdat vele mannen zich extreem grof
seksueel benaderen; in sommige triviale gevallen maakte men er zelfs
belachelijke situaties van. Sommigen beweerden zelfs dat een enkele blik
al als seksuele intimidatie kon worden aangemerkt, zonder dat iemand ook
maar één woord zei of fysiek contact maakte. Hierdoor kwam de nieuwe
moraal soms over als bijzonder censurerend. Witte mensen konden
beschuldigd worden van raciale vooroordelen, niet omdat zij per se
bevooroordeeld waren, maar simpelweg vanwege hun huidskleur. Mannen
kregen de vinger over de neus gelegd omdat zij door hun uitingen lieten
blijken een vrouw aantrekkelijk te vinden -- een houding die men in
voorgaande generaties als een compliment beschouwde in plaats van als
belediging.

De politiek correcte groepen en de fundamentalistische christenen
leveren elkaar bittere kritiek toe, maar in de moderne wereld vertonen
zij opvallend veel overeenkomsten. Beide kampen vinden de autoriteit van
hun eigen morele doctrine universeel, ook al verschillen hun morele
leerstellingen wezenlijk. Men bekritiseert hen beiden vanwege een
overdreven, zelfverzekerde moraliteit die vaak ontbreekt aan diepgang,
historisch perspectief en tolerantie. Sommigen vergelijken hen met het
zeventiende-eeuwse puritanisme, ervaren moralisten zoals Oliver Cromwell
in Engeland -- die bijna naar Nieuw-Engeland vertrok -- of met de
heksenjagers van Salem. Noch de meer rigide vrouwenbeweging, noch de
conservatieve predikanten van de \emph{Bible Belt} kunnen hen verweten
moreel tekort te schieten; integendeel, hun moraliteit blijkt juist
overontwikkeld en inflexibel. Soms lijkt het hart van hun morele
overtuigingen tot steen verworden te zijn. Zo'n verharding schaadt de
gemeenschappelijke moraliteit in de samenleving net zozeer als de `alles
mag'-anarchie waartegen zij zich verzetten.

Het leidt tot een vertekende weergave van morele krachten en culmineert
in zelfgerechtigdheid. Het farizeïsme -- het onwrikbare geloof in de
eigen deugd -- is zo oud als de mensheid en werd door Jezus Christus
krachtig veroordeeld. Daartegenover staat de recente opvatting dat
ethische keuzes louter een kwestie van persoonlijke voorkeur zijn, net
zo individueel als de keuze van kleding. Dit standpunt weerspiegelt het
ontbreken van een gedeelde moraliteit, herdefinieert de klassieke
vrijheidstheorie en verandert `het nastreven van geluk' -- zoals
John~Locke dat oorspronkelijk bedoelde en Jefferson in 1776 verwoorde --
in een hedonistisch streven dat roekeloos omgaat met de gevolgen.

De uitdrukking «het nastreven van geluk» vindt zijn oorsprong in
John~Locke's \emph{Essay on Human Understanding} (1691). Daarin schrijft
hij: «De hoogste volmaaktheid van de intellectuele natuur ligt in een
zorgvuldige zoektocht naar waarachtig en solide geluk, zodat de zorg
voor onszelf -- opdat wij het denkbeeldige niet verwarren met het reële
-- de noodzakelijke basis van onze vrijheid vormt.» Vervolgens merkt hij
op: «Niet iedereen vindt zijn geluk in hetzelfde\ldots{} de geest heeft
een andere smaak dan het gehemelte\ldots{} Mensen mogen voor
verschillende dingen kiezen, maar allen maken de juiste keuze, indien
men hen beschouwt als een gezelschap arme insecten, waarvan sommigen
bijen zijn, die zich verheugen over bloemen en hun zoetheid, en anderen
kevers, die genieten van andere soorten kost.» Toch betoogt hij dat het
verkiezen van ondeugd boven deugd onmiskenbaar een foutieve afweging is.
Hij legt daarbij bijzondere nadruk op religieuze argumenten en stelt dat
de slechten het zwaarst te lijden hebben. Hij is ervan overtuigd dat
wanneer moraliteit op degelijke fundamenten rust, deze de keuzes van
iedereen die erover nadenkt, onvermijdelijk zal bepalen.

De Lockeniaanse doctrine van vrijheid biedt ongetwijfeld meer ruimte
voor individuele voorkeuren dan autoritaire morele systemen die trachten
iedereen gelijk te behandelen en uniform gedrag af te dwingen. Toch
erkent de klassieke vrijheidsleer al snel de noodzaak van collectieve
morele imperatieven, zoals respect voor medemensen -- in het bijzonder
voor hun leven en het vreedzaam bezit van hun eigendommen volgens de
wet. Een afname van de gedeelde moraliteit vormt een bedreiging voor de
vrijheid, enerzijds doordat het een element van anarchie introduceert en
anderzijds doordat het autoritaire krachten in de hand werkt. De
geschiedenis van de publieke moraliteit verloopt in een cyclus van
wanorde en autoritarisme; hedendaagse autoritaire morele stromingen,
zoals het feminisme en het fundamentalisme, zijn ontstaan als reactie op
het hedonisme van de jaren zestig.

We hebben reeds enkele kenmerken geschetst van de wereld van de komende
eeuw. Deze zal worden bepaald door twee hoofdontwikkelingen: de
technologische verschuiving die de Aziatische economieën opent en de
opkomst van wereldwijde elektronische communicatiemiddelen die burgers
steeds minder afhankelijk maken van de lokale overheid. Deze nieuwe
technologie vervangt -- of heeft al vervangen -- veel middelmatige
menselijke vaardigheden, zoals die van de productielijnwerker, de
kantoormedewerker en steeds vaker de middenmanager. Tegelijkertijd
beloont zij zeldzamere competenties en creëert zij een internationale
cognitieve elite van hoogopgeleiden, voor wie de nieuwe
communicatiemiddelen een zo breed mogelijke markt openen voor hun
talenten. Net als de meeste elites beschouwt deze cognitieve elite
zichzelf vaak als superieur, gaat zij arrogant te werk en meent zij haar
eigen normen te mogen stellen. Daardoor vervreemdt zij zich van de
samenleving.

Tijdens de eerste helft van de volgende eeuw vindt een enorme overdracht
van rijkdom plaats van het oude Westen naar het nieuwe Oosten. Politieke
tegenslagen -- en aangezien China nog steeds als politiek achterhaald
wordt gezien -- kunnen deze verschuiving wellicht vertragen, maar zullen
haar vrijwel zeker niet tegenhouden. Ze kunnen het proces niet
terugdraaien.

Deze rijkdomsoverdracht legt ongetwijfeld de grootste druk op de landen
op het noordelijk halfrond die door blanke machten worden gedomineerd,
zoals in Europa en Noord-Amerika. Momenteel telt dit gebied ongeveer 750
miljoen inwoners in de ontwikkelde landen; tot voor kort was Japan het
enige Aziatische, niet-blanke land dat de Euro-Amerikaanse
levensstandaard had bereikt, al waren er etnisch Europese
bevolkingsgroepen in Nieuw-Zeeland, Australië en in de blanke
gemeenschappen van Zuidelijk Afrika. In 1990 behoorden de geavanceerde
industriële landen slechts naar schatting tot 15 procent van een
wereldbevolking van 5 miljard. Damaar lag de verdeling van de
wereldwijde rijkdom op ongeveer 15 procent welvarenden en 85 procent
minderbegaarden -- een verhouding die sterk overeenkomt met de
inkomensverdeling in geavanceerde industriële samenlevingen van een eeuw
geleden. Tegen 2050 wordt verwacht dat de ontwikkelde economieën, binnen
een wereldbevolking die mogelijk tot 7 miljard groeit, ongeveer 3
miljard mensen omvatten -- een situatie die neerkomt op een
rijkdomsverdeling van 40 procent welvarenden en 60 procent armen. Tegen
het einde van de eeuw kunnen de aantallen zelfs omkeren, met 60 procent
welvarenden en 40 procent mensen met een lager inkomen, waarbij armoede
vooral geconcentreerd blijft in Afrika. Hoewel deze verschuiving tussen
naties mondiaal een grotere gelijkheid in rijkdom bevordert, zal de
ongelijkheid binnen landen waarschijnlijk juist toenemen. Wie zijn
talent en kapitaal efficiënt benut, krijgt een aanzienlijk voordeel ten
opzichte van mensen met gemiddelde vaardigheden of weinig middelen.
Bovendien beweegt deze rijkdom zich uiterst mobiel. De armen in de
ontwikkelde landen zullen de rijken niet langer kunnen belasten zoals in
de twintigste eeuw; landen die dat proberen, lopen het risico achterop
te raken in een felle, competitieve race.

Natuurlijk blijft de totale productiviteit van de wereldeconomie
groeien, waarschijnlijk met gemiddeld 3 procent wereldwijd, mits er geen
wereldoorlogen uitbreken. Als dat standhoudt, verdubbelt het
wereldproduct elke vijfentwintig jaar -- waardoor het in 2050 meer dan
vier keer zo hoog ligt als nu en tegen 2100 zestien tot twintig keer zo
groot zal zijn. Zelfs als de wereldbevolking tegen 2100 oploopt tot 8
miljard, resulteert dat per hoofd in een wereld-BBP dat aan het einde
van de eeuw tien keer het huidige niveau bedraagt. Een dergelijke
toename van rijkdom kan zowel de opkomst van nieuwe industriële
samenlevingen als de multimiljoeneninkomens van de cognitieve elite
opvangen, terwijl het tegelijkertijd voor de overige arbeidskrachten in
de ontwikkelde landen een stijgende en fatsoenlijke levensstandaard
waarborgt. De inkomensverschillen zullen daarbij echter radicaal anders
zijn dan in de twintigste eeuw. Op wereldschaal zullen de inkomens in
arme landen veel sneller groeien dan in rijke landen, terwijl nationaal
gezien juist de inkomens van de rijken -- zoals in het Amerika van de
jaren negentig -- veel sneller zullen stijgen dan die van de midden- en
lagere inkomenslagen. In de komende eeuw zullen we getuige zijn van de
opkomst van een wereldsuperklasse, mogelijk bestaande uit 500 miljoen
zeer rijke mensen, waarvan er 100 miljoen rijk genoeg zijn om zich als
onafhankelijke, soevereine individuen te onderscheiden.

Dit proces heeft onvermijdelijk consequenties. Samenlevingen worden veel
minder homogeen, de natiestaat verzwakt of valt zelfs volledig uiteen en
de cognitieve elite zal zichzelf als kosmopolitisch gaan beschouwen.
Mensen die wereldwijd in vergelijkbare functies werken, ontwikkelen
immers een cultuur die veel meer overeenkomt met die van hun collega's
in andere delen van de wereld dan met die van hun medeburgers in
traditionele natiestaten. Een investeringsbankier uit Londen voelt zich
waarschijnlijk meer thuis in Seoel dan in Glasgow, en een ambtenaar uit
Washington voelt zich wellicht prettiger in Bonn dan in de zwarte wijken
van Washington zelf. We zien nu al hoe dit proces de morele waarden
fragmentariseert. De moraal van een individu wordt immers deels bepaald
door zijn opvoeding -- door wat iemand als kind geleerd heeft -- en
deels door zijn levenservaringen. Zowel de scholing als de ervaringen
binnen de cognitieve elite hebben een duidelijke kosmopolitische inslag,
die mensen losmaakt van hun lokale gemeenschappen.

Naarmate we de volgende eeuw ingaan, valt op dat veel vertegenwoordigers
van de groeiende cognitieve elite nauwelijks enige religieuze of morele
vorming binnen het gezin hebben meegekregen. De overheersende
levensbeschouwing in deze kringen is agnostisch humanisme. Bovendien
wordt veelvuldig gezien dat deze families verscheurd worden door
echtscheidingen en hertrouw, met daaropvolgende derde huwelijken. Hoewel
het huwelijksmodel dat in Hollywood heerst niet representatief is voor
de gehele Verenigde Staten, kampt de cognitieve elite in Euro-Amerika
met een hoog percentage echtscheidingen -- gemiddeld wel een derde of
meer. Kinderen van gescheiden ouders krijgen zelden een fundamentele
religieuze vorming mee en raken al vroeg bewust van de uiteenlopende
morele opvattingen die hun ouders, stiefouders en stiefbroers en -zussen
hanteren. Vergelijk je deze initiële morele vorming met die in een
traditioneel Iers of Pools dorp, dan valt op dat de boerenopvoeding een
veel sterkere religieuze basis biedt. Een elite die goddeloos,
wortelloos en welvarend is, zal waarschijnlijk niet gelukkig worden of
geliefd zijn.

Latere levenservaringen versterken ongetwijfeld de tekortkomingen in de
vroege morele vorming van de toekomstige dominante economische groep.
Zij doorlopen een intensieve technische opleiding -- in welke vorm dan
ook -- om zich voor te bereiden op hun nieuwe rol als leiders in het
elektronisch tijdperk. Uit deze opleidingen halen zij echter slechts
enkele morele lessen mee, de lessen die historisch gezien de basis
vormden van ons sociale gedrag. Volgens de maatstaven van Confucius,
Boeddha of Plato (500 v.Chr.), van Sint-Paul (50 n.Chr.) of van Mahomet
(600 n.Chr.) voldoen zij immers nauwelijks; men zou hen als moreel
analfabeten bestempelen. Ze krijgen les in economische efficiëntie, in
het optimaal inzetten van hulpbronnen en in het najagen van geld, maar
niemand leert hen over de deugden van nederigheid, zelfopoffering of
kuisheid. In feite groeien de meesten als heidenen op, met een
waardenstelsel dat meer weg heeft van de late Romeinse Republiek dan van
het christendom. Bovendien zijn deze waarden sterk individualistisch in
plaats van gemeenschappelijk. Zoals we al betoogden, putten
samenlevingen alleen kracht uit morele waarden die breed worden gedeeld.
De ontwikkelde naties bewegen inmiddels richting een situatie waarin
velen zwakke of beperkte morele waarden hanteren, terwijl anderen dit
compenseren door zich fel vast te klampen aan irrationele waarden -- en
slechts enkele waarden vinden breed draagvlak in de samenleving. Zonder
twijfel zullen sommige van de eerder beschreven `concurrerende
territoriale clubs' strikte morele normen gaan hanteren voor wie er in
hun gemeenschap mag wonen.

Historisch gezien veroorzaakten verschillen in rijkdom op zich geen
wezenlijke variaties in religieuze waarden. In dichtbevolkte en stabiele
samenlevingen met sterke tradities kan een rigide hiërarchie -- `de
rijke man in zijn kasteel, de arme man bij zijn poort' -- waarden
bevatten die door de hele samenleving doorklinken. Dat vergt echter een
sterk gemeenschapsgevoel bij zowel de rijken als de armen en robuuste
sociale tradities. Tegenwoordig missen we deze voorwaarden; de
economische en technologische revolutie verzwakt immers zowel het
gemeenschapsgevoel als de tradities. Daardoor lopen het leven van de
meesten en dat van een selecte groep steeds verder uit elkaar.
Innovatoren doorbreken de oude werkwijzen, wat de technologische
revolutie mogelijk maakte; in elk domein komt de radicaal als winnaar
naar voren, terwijl conventionele denkers achterblijven en letterlijk
buiten de boot vallen. Hoewel politici als Bill Clinton, Helmut Kohl en
John Major vaak de politieke koers bepalen, leiden radicale ondernemers
met scherp inzicht in de technologische wereld onze meest succesvolle
bedrijven -- een toonbeeld hiervan is Bill Gates. Het traditionele
denken verliest zijn geloofwaardigheid doordat het niet in staat is de
razendsnel veranderende omstandigheden bij te benen.

Moraliteit houdt het echter anders. Als we Mozes' `wetenschap', ontstaan
rond 1000 v.Chr., nauwkeurig bestuderen, ontdekken we dat deze ons
weinig concreet leert. Het scheppingsverhaal in Genesis draagt
ongetwijfeld een theologische waarheid uit -- namelijk dat God het
universum en de mensheid schiep -- maar beschrijft niet hoe fysieke
structuren daadwerkelijk ontstonden. Daarentegen bevat de moraliteit van
Mozes, zoals verwoord in de Tien Geboden, een schat aan wijsheden.

Zo vormt respect voor ouders en trouw binnen het huwelijk de beste
waarborg voor het behoud van een stabiel gezinsleven, wat op zijn beurt
essentieel is voor de opvoeding van moreel gezonde kinderen. Diefstal
schaadt zowel de dader als de slachtoffers en ontmoedigt arbeid en
sparen. Bovendien steunt een goed functionerende samenleving op de
eerlijkheid van getuigen; moord is verwerpelijk, enzovoort.

Op het gebied van wetenschap heeft de mensheid in drie duizend jaar onze
kennis drastisch getransformeerd, maar moreel gezien lijken we mogelijk
achteruit te gaan. De gemiddelde psychotherapeut biedt zijn patiënt
waarschijnlijk minder gedegen moreel advies over de levenswijze dan een
Jood in Mozes' tijd van zijn leraar ontving. Natuurlijk leeft het
christendom nog voort, maar voor het merendeel van de wereld is het
slechts een schim van zijn vroegere glorie. Slechts weinigen koesteren
het geloof zoals dat in vroegere tijden of in minder verfijnde
gemeenschappen werd beleefd; men zoekt immers geen heiligen op Park
Avenue.

Het afbreken van traditie was onmisbaar voor wetenschappelijke
vooruitgang. Als we vandaag nog zouden geloven dat de zon om de aarde
draait, hadden we nooit satellietcommunicatie ontwikkeld. Wat wij als
wetenschap aanhangen, bestaat immers uit een reeks hypothesen --
onvolmaakte verklaringen die later door sterkere, maar nog steeds
gebrekkige uitspraken worden vervangen. Toch heeft het verwerpen van
traditie desastreuze gevolgen gehad voor de morele orde in de wereld.

Confucius onderwees dat we ons met mate moeten gedragen -- hij noemde de
gulden middenweg `chum yum', zoals zeventiende-eeuwse geleerden die term
hebben vertaald. Bovendien benadrukte hij dat we autoriteit dienen te
respecteren en anderen behandelen op de wijze waarop wij zelf behandeld
willen worden. Die leer is inmiddels 2500 jaar oud en heeft China
gedurende de gehele geschreven geschiedenis diepgaand beïnvloed. Toch
zien veel moderne Chinezen het confucianisme als een achterhaalde
traditie, waarin men geen waarde hecht aan gematigdheid, waarbij kracht
meer telt dan autoriteit en waarin anderen niet worden behandeld zoals
men dat zelf graag zou willen ervaren. Bij het vervallen van traditie
kan een samenleving haar volledige morele referentiekader kwijtraken. En
ondanks haar groeiende macht blijft China moreel achterhaald in
vergelijking met Tibet, waar mensen evenarm en onderdrukt leven als in
de rest van de Tibetaanse gemeenschap.

Een degelijke sociale moraliteit kent een aantal kenmerken. Zij moet
zowel het voortbestaan van de samenleving als dat van het individu
bevorderen, en dat op een dynamische in plaats van op een statische
wijze. Ze stimuleert tolerantie en verdringt zelfgenoegzaamheid.
Bovendien hoort ze een religieus karakter te hebben in plaats van louter
agnostisch te zijn, en doet ze niet alsof ze in staat is om
wetenschappelijke feiten vast te leggen. Ze mag noch anarchistisch noch
autoritair zijn en moet breed gedragen en diepgeworteld zijn. Zo'n
morele basis is van cruciaal belang voor het gezin en voor de opvoeding
van kinderen tot zelfstandige, verantwoordelijke volwassenen; zij vormt
immers de spil van een gezonde samenleving.

Onze ervaringen tonen dat deze morele basis steunt op de logica van
onderlinge afhankelijkheid, afkomstig uit handel en medemenselijkheid.
Tegelijkertijd bedreigen een oppervlakkig scientisme, de vervreemding
tussen een superklasse en een onderklasse en het verdwijnen van de oude,
regiogebonden economische grondslagen deze fundamenten. Wellicht komt er
ooit een tegenreactie op deze trends, want deze ontwikkelingen moeten we
erkennen als uiterst gevaarlijk voor de samenlevingen van de komende
eeuw.

Nu, terwijl volgens Isaiah Berlin `de meest verschrikkelijke eeuw in de
westerse geschiedenis' ten einde lijkt te komen, hoort ook het tijdperk
van gigantische sociale structuren tot het verleden. De laatste dagen
van de twintigste eeuw markeren een periode van inkrimping,
decentralisatie en reorganisatie. We bevinden ons in een tijdperk van
sociale dinosauriërs die vastzitten in de teerpit, en van aaseters die
boven de resten uitstijgen. Vogels zullen de beenderen van die
dinosauriërs rap oprapen. Overheden, bedrijven en vakbonden passen zich
-- zij het onwilkeurig -- aan aan nieuwe, metaconstitutionele
voorwaarden, bepaald door de opkomst van microtechnologie. Deze
technologische doorbraak heeft de grenzen waarbinnen geweld wordt
uitgeoefend drastisch verlegd. Onze wereld is al veel meer veranderd dan
we ons normaal voorstellen, veel meer dan \emph{CNN} en de kranten ons
doen geloven. En de veranderingen sluiten precies aan bij wat studies
over megapolitieke condities voorspellen. Zoals we in \emph{Blood in the
Streets} en later in \emph{The Great Reckoning} betoogden: elke
wijziging in technologie of in andere factoren die de grenzen van geweld
bepalen, leidt onvermijdelijk tot een andere aard van de samenleving.
Alles wat te maken heeft met hoe mensen met elkaar omgaan -- van
moraliteit tot het alledaagse gezond verstand -- zal mee veranderen. Na
een periode van verslapte normen, die het einde van een tijdperk
markeert, komt een heropleving van een strengere moraliteit, met
veeleisender standaarden die passen bij de intensere voorwaarden van een
wereld waar soevereiniteit met elkaar concurreert.

We mogen verschillende kenmerken van de nieuwe moraliteit verwachten.
Enerzijds komt het belang van productiviteit centraal te staan, samen
met het principe dat mensen recht hebben op het behouden van de
beloningen die zij zelf verdienen. Daarnaast speelt efficiëntie bij
investeringen een cruciale rol. De ethiek van het informatietijdperk
stelt efficiëntie voorop en erkent dat middelen pas volledig tot hun
recht komen wanneer ze worden ingezet voor hun meest waardevolle
toepassingen. Kortom, deze nieuwe moraliteit belichaamt in essentie de
ethiek van de markt.

Zoals James Bennett betoogt, berust de moraliteit van het
informatietijdperk eveneens op vertrouwen. De cybereconomie vormt immers
een gemeenschap waarin vertrouwen hoog in het vaandel staat.

Wanneer onbreekbare encryptie een verduisteraar of dief in staat stelt
de opbrengsten van zijn misdrijven veilig buiten bereik te houden,
ontstaat er een krachtige prikkel om verliezen te voorkomen door
aanvankelijk geen transacties met zulke criminelen aan te gaan.

Zoals het voorbeeld van de Quakers al liet zien, telt een reputatie van
integriteit zwaar in de cybereconomie. In de anonimiteit van cyberspace
hoeft deze reputatie niet per se aan een bekende identiteit gekoppeld te
zijn; men kan deze immers op betrouwbare wijze verifiëren via
cryptografische sleutels.

Het vooruitzicht op ernstige problemen als gevolg van een corruptie in
de encryptie of de certificering van versleutelde identiteiten door
gangsters of anderen, biedt al voldoende reden om geen zaken te doen met
personen die onbetrouwbaar overkomen.

Bennett voorziet een `gentleman's club in cyberspace': beschermde zones
waarin deelname strengere veiligheidsmaatregelen vereist, eventueel
ondersteund door biometrische validatie zoals stemprintherkenning. De
beheerders nemen daarbij de verantwoordelijkheid om de identiteit -- en
tot op zekere hoogte de betrouwbaarheid -- van de deelnemers te
waarborgen. Zo ontstaat er een `gentleman's club in cyberspace' (hoewel
vrouwen uiteraard altijd welkom zijn). Binnen deze zones verrichten
mensen transacties met meer zekerheid en vertrouwen dan in het bredere
domein van cyberspace. Hierdoor zou de eenentwintigste eeuw een
terugkeer kunnen zien naar een op Victoriaanse waarden gebaseerde nadruk
op vertrouwen en karakter, in een context die geen enkele Victoriaan had
kunnen voorspellen.

Ook de beveiligde zones in cyberspace kunnen garanties bieden om
risico's te beperken. Ze doen immers denken aan de extraterritoriale
waarborgen die de graven van Champagne verleenden om kooplieden te
beschermen tijdens hun reizen naar en van de Champagne-markten. Andere
rechtsgebieden vergoedden zelfs reizende kooplieden voor eventuele
verliezen die zij opliepen bij het passeren van het grondgebied onder
het gezag van een edelman.

De `Guards of the Fair', ambtenaren die oorspronkelijk door de graven
werden aangesteld, waarborgden de beveiliging en fungeerden als een
`tribunaal van gerechtigheid' voor de marktkooplieden. In de loop der
tijd groeiden zij uit tot zelfstandige entiteiten met een eigen zegel,
die contracten legaliseerden en de nakoming daarvan afdwongen. Zij
kregen zelfs de bevoegdheid om een handelaar die er niet in slaagde zijn
schulden te voldoen of zijn contractuele verplichtingen na te komen, de
toegang tot toekomstige markten te ontzeggen. Deze straf bleek zo streng
dat maar weinigen bereid waren het risico te lopen op het mislopen van
toekomstige winstkansen. Als minder ingrijpende maatregel konden de
bewakers daarnaast de goederen van een wanbetaler in beslag nemen en ten
behoeve van zijn schuldeisers verkopen. \footnote{James Bennett,
  \emph{Cyberspace and the Return of Trust}, \emph{Strategic
  Investment}, oktober 1996.}

Naarmate het aantal alternatieve markten toenam, verloor het uitsluiten
van personen als handhavingsmiddel voor contracten aan belang. Dankzij
de opkomst van nieuwe informatietechnologie kan het uitsluiten van
oplichters en wanpresteerders in hun contractuele verplichtingen echter
opnieuw dienen als een krachtig mechanisme binnen de gefragmenteerde
soevereiniteiten van de volgende fase van de samenleving.
Computernetwerken stellen ons in staat cyberspace te monitoren met
onvervalsbare informatie over kredietwaardigheid en fraude. Doordat de
wereld op dit vlak een uiterst hechte gemeenschap vormt, worden
oplichters en fraudeurs effectief ontmoedigd.

Naast de nadruk op verdiensten en efficiëntie, en een hernieuwd accent
op karakter en vertrouwen, zal de nieuwe moraliteit hoogstwaarschijnlijk
ook een sterke afkeer van geweld benadrukken. Met name ontvoering en
afpersing -- middelen die steeds vaker worden ingezet om mensen af te
persen die anders minder snel het doelwit zouden worden van
criminaliteit -- zullen hierbij centraal staan.

Nog een mogelijke stimulans voor strengere moraliteit zal het einde
inluiden van uitkeringen en inkomensherverdeling. Als de hoop op steun
voor achterblijvers voornamelijk steunt op een beroep op particulieren
en liefdadigheidsinstellingen, wordt het belangrijker dan in de
twintigste eeuw dat de ontvangers van deze hulp moreel welverdiend
overkomen in de ogen van degenen die deze ondersteuning vrijwillig
verlenen.

\begin{quote}
``Subsidies, windfalls en de vooruitzichten op economische kansen nemen
de urgentie weg om te moeten besparen. De mantra's van democratie,
herverdeling en economische ontwikkeling verhogen zowel de verwachtingen
als de vruchtbaarheid, wat de bevolkingsgroei stimuleert en daarmee een
neerwaartse milieu- en economische spiraal versnelt.'' \footnote{Virginia
  Abernethy, \emph{Optimisme en overbevolking}, \emph{Atlantic Monthly},
  december 1994, p.~88.} - VIRGINIA ABERNATHY
\end{quote}

Op bepaalde terreinen staat de nieuwe informatiewereld beter in staat om
morele vraagstukken serieus aan te pakken. De beloftes van
inkomensherverdeling, die de hoop deden oplaaien bij de minder
fortuinlijken in de Verenigde Staten, Canada en West-Europa, hebben
internationaal vaak een averechts effect gehad. Er bestaat overtuigend
bewijs dat buitenlandse hulp en interventiebeloftes -- bedoeld om
hongersnood te voorkomen en de levensstandaard te verhogen --
belangrijke factoren zijn die de bevolkingsgroei stimuleren, zelfs
wanneer deze groei de draagkracht van achterblijvende economieën
overschrijdt. De opzienbarende groei van de wereldbevolking sinds de
Tweede Wereldoorlog, met de vaak verwoestende gevolgen voor bossen,
bodems en watervoorraden, vloeit voort uit wereldwijde interventies die
de negatieve terugkoppelingsmechanismen, waarbij lokale bevolkingen
lange tijd in evenwicht werden gehouden met de beschikbare hulpbronnen,
teniet hebben gedaan.

Degenen die in kleine gemeenschappen met weinig middelen en nauwelijks
groeikansen leefden, waren aanvankelijk zeer tevreden dat de beperkende
praktijken van hun dorpsleven zouden verdwijnen. Zij namen met
enthousiasme de positieve boodschap over, verkondigd door internationale
hulpverleners, vrijwilligers van het \emph{Peace Corps}, lokale
revolutionairen en de strijdende ideologen tijdens de Koude Oorlog, die
iedereen verzekerden dat er betere tijden in het verschiet lagen. Dit
bleek echter precies de verkeerde boodschap te zijn.

Een belangrijk gevolg van culturele herverdeling is dat mensen die in
niet-industriële samenlevingen leefden en trouw bleven aan traditionele
waarden, kunstmatig in concurrentie met elkaar kwamen te staan.
Internationale hulp, reddingsoperaties om hongersnood en epidemieën te
bestrijden en technologische interventies hebben velen voor de gek
gehouden, waardoor ze gingen geloven dat hun vooruitzichten drastisch
waren verbeterd, zonder dat ze hun waarden hoefden aan te passen of hun
gedrag wezenlijk te veranderen.

Internationale inkomensherverdeling heeft niet alleen geleid tot een
onhoudbare bevolkingsgroei wereldwijd, maar heeft ook in belangrijke
mate bijgedragen aan cultureel relativisme en tot wijdverspreide
verwarring over de cruciale rol van cultuur bij het mogelijk maken dat
mensen in hun eigen omgeving floreren. Tegenwoordig zijn de meesten
ervan overtuigd dat culturen meer een kwestie van smaak zijn dan bronnen
van gedragsrichtlijnen, die zowel kunnen misleiden als informeren. We
nemen te snel aan dat alle culturen gelijkwaardig zijn en wachten te
lang met het herkennen van de nadelen van contraproductieve
cultuurvormen. Dit geldt vooral voor de hybride culturen die zich in
deze eeuw in veel delen van de wereld ontwikkelen door subsidies en
overheidsinterventies. Vergelijkbaar met de criminele subculturen in de
binnensteden van Amerika dragen zij onsamenhangende fragmenten uit
vroegere culturele fasen met zich mee en mengen die met waarden die het
gedrag in het informatietijdperk bepalen.

De informatie-revolutie zal daarom niet alleen de geest van genialiteit
bevrijden, maar ook de drang tot vergelding ontketenen. Beide krachten
gaan in het komende millennium strijden als nooit tevoren.

De overgang van een industriële naar een informatiemaatschappij belooft
werkelijk adembenemend te worden. Elke verschuiving van het ene stadium
van economisch leven naar het volgende ging altijd gepaard met een
revolutie. Wij zijn ervan overtuigd dat de informatie-revolutie
waarschijnlijk de meest ingrijpende omwenteling wordt. Zij zal het leven
grondiger herstructureren dan zowel de agrarische als de industriële
revolutie, en de impact zal binnen een fractie van de tijd voelbaar
zijn. Zet uw gordels vast.


\backmatter


\end{document}
